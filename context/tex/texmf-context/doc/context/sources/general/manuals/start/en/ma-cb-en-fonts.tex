\startcomponent ma-cb-en-fonts

\enablemode[**en-us]

\project ma-cb

\startchapter[title=Fonts and font switches]

\startsection[title=Introduction]

\index{Computer Modern Roman}
\index{Lucida Bright}
\index{AMS}

The default font in \CONTEXT\ is the {\em Computer Modern Roman} (\type{cmr}).
In \CONTEXT\ the following fonts are available.

\startplacetable[reference=tab:fonts in context,title={Fonts in \CONTEXT.}]
\starttable[|l|l|l|]
\HL
\NC \bf Name              \NC \bf Logical name \NC \bf Also known as     \NC\MR
\HL
\NC Computer Modern Roman \NC cmr              \NC Computer Modern Roman \NC\FR
\NC Termes                \NC termes           \NC Times New Roman       \NC\MR
\NC Adventor              \NC adventor         \NC Avant Garde           \NC\MR
\NC Bonum                 \NC bonum            \NC Bookman               \NC\MR
\NC Chorus                \NC chorus           \NC Zapf Chancery         \NC\MR
\NC Cursor                \NC cursor           \NC Courier               \NC\MR
\NC Heros                 \NC heros            \NC Helvetica             \NC\MR
\NC Pagella               \NC pagella          \NC Palatino              \NC\MR
\NC Schola                \NC schola           \NC Century Schoolbook    \NC\MR
\NC Dejavu                \NC dejavu           \NC                       \NC\MR
\NC Iwona                 \NC iwona            \NC                       \NC\MR
\NC Gentium               \NC gentium          \NC                       \NC\MR
\NC Cambria               \NC cambria          \NC                       \NC\MR
\NC Antykwa               \NC antykwa          \NC                       \NC\MR
\NC Utopia                \NC utopia           \NC                       \NC\MR
\NC LucidaBright          \NC lucidanova       \NC                       \NC\LR
\HL
\stoptable
\stopplacetable

% in map: tex-context\tex\texmf\fonts

For further reading we refer to the \goto {\em Fonts in \CONTEXT}
[ url (manual:fonts) ] manual where you can find information on how to install 
your own font.

\stopsection

\startsection[title=Fontstyle and size]

\index{font+style}
\index{font+size}

\Command{\tex{setupbodyfont}}
\Command{\tex{switchtobodyfont}}

You can select the font family, style and size for a document with:

\shortsetup{setupbodyfont}

If you typed \type{\setupbodyfont[chorus,9pt]} {\switchtobodyfont[chorus,9pt] in
the setup area of the input file your text would look something like this.}

For changes in mid-document and on section level you should use:

\shortsetup{switchtobodyfont}

\startbuffer
On November 10th (one day before Saint Martinsday) the youth of
Hasselt go from door to door to sing a special song and they
accompany themselves on a {\em foekepot}. They won't leave
before you give them some money or sweets. The song goes like this:

\startnarrower
\switchtobodyfont[heros,small]
\startlines
Foekepotterij, foekepotterij,
Geef mij een centje dan ga'k voorbij.
Geef mij een alfje dan blijf ik staan,
'k Zal nog liever naar m'n arrenmoeder gaan.
Hier woont zo'n rieke man, die zo vulle gèven kan.
Gèf wat, old wat, gèf die arme stumpers wat,
'k Eb zo lange met de foekepot elopen.
'k Eb gien geld om brood te kopen.
Foekepotterij, foekepotterij,
Geef mij een centje dan ga'k voorbij.
\stoplines
\stopnarrower
\stopbuffer

\typebuffer

Notice that \type{\start...\stopnarrower} is also used as a begin and end of the
fontswitch. The function of \type{\start...\stoplines} in this example is
obvious.

\start
\getbuffer
\stop

If you want an overview of the available font family you can type:

\startbuffer
\showbodyfont[pagella]
\stopbuffer

\typebuffer

\getbuffer

\stopsection

\startsection[title=Style and size switch in commands]

In a number of commands one of the parameters is \type{style} to indicate the
desired typestyle. For example:

\startbuffer
\setuphead[chapter][style=\tfd]
\stopbuffer

\typebuffer

In this case the character size for chapters is indicated with a command
\type{\tfd}. But instead of a command you could use the predefined options that
are related to the actual typeface:

\startbuffer
normal  bold  slanted  boldslanted  type  mediaeval
small  smallbold  smallslanted  smallboldslanted smalltype
capital cap
\stopbuffer

\typebuffer

\stopsection

\startsection[title=Local font style and size]

\Command{\tex{rm}}
\Command{\tex{ss}}
\Command{\tex{tt}}
\Command{\tex{sl}}
\Command{\tex{bf}}
\Command{\tex{tfa}}
\Command{\tex{tfb}}
\Command{\tex{tfc}}
\Command{\tex{tfd}}

In the running text (local) you can change the {\em typestyle} into roman, sans
serif and teletype with \type{\rm}, \type{\ss} and \type{\tt}.

You can change the {\em typeface} like italic and boldface with \type{\sl} and
\type{\bf}.

The {\em typesize} is changed with \type{\switchtobodyfont}.

The actual style is indicated with \type{\tf}. If you want to change into a
somewhat greater size you can type \type{\tfa}, \type{\tfb}, \type{\tfc} and
\type{\tfd}. An addition of \type{a}, \type{b}, \type{c} and \type{d} to
\type{\sl}, \type{\it} and \type{\bf} is also allowed.

\startbuffer
{\tfc Mintage}

In the period from {\tt 1404} till {\tt 1585} Hasselt had its own
{\sl right of coinage}. This right was challenged by other cities,
but the {\switchtobodyfont[7pt] bishops of Utrecht} did not honour
these {\slb protests}.
\stopbuffer

\typebuffer

The curly braces indicate begin and end of style or size switches.

\getbuffer

\stopsection

\startsection[title=Redefining fontsize]

\index{fontsize}

\Command{\tex{definebodyfont}}

For special purposes you can define your own size of the bodyfont.

\shortsetup{definebodyfont}

A definition could look like this:

\startbuffer
\definebodyfont[10pt][rm][tfe=Regular at 36pt]

{\tfe Hasselt!}
\stopbuffer

\typebuffer

Now \type{\tfe} will produce 36pt characters saying:
{\hbox{\getbuffer}}

\stopsection

\startsection[title=Small caps]

\index{small caps}

\Command{\tex{cap}}

Abbreviations like \PDF\ (\infull{PDF}) are printed in pseudo small caps. A small
capital is somewhat smaller than the capital of the actual typeface. Pseudo small
caps are produced with:

\shortsetup{cap}

If you compare \type{\cap{hasselt}} and \type{\sc hasselt}: \cap{hasselt} and
{\sc hasselt} you can see the difference. The command \type{\sc} shows the real
small caps. The reason for using pseudo small caps instead of real small caps is
just a matter of taste.

\stopsection

\startsection[title=Emphasized]

\index{emphasized}

\Command{\tex{em}}

To emphasize words consistently throughout your document
you use:

\starttyping
\em
\stoptyping

Empasized words appear in a slanted style.

\startbuffer
If you walk through Hasselt you should {\bf \em watch out} for
{\em Amsterdammers}. An {\em Amsterdammer} is {\bf \em not} a
person from Amsterdam but a little stone pillar used to separate
sidewalk and road. A pedestrian should be protected by these
{\em Amsterdammers} against cars but more often people get hurt
from tripping over them.
\stopbuffer

\typebuffer

This becomes:

\getbuffer

{\em An emphasize within an emphasize is {\em normal} again
and a boldface emphasize looks like {\bf this or \em this}}.

\stopsection

\startsection[title=Teletype / verbatim]

\index{type}
\index{verbatim}

\Command{\tex{starttyping}}
\Command{\tex{type}}
\Command{\tex{setuptyping}}
\Command{\tex{setuptype}}

If you want to display typed text and want to keep your line breaking exactly as
it is you use:

\shortsetup{starttyping}

In the text you can use:

\shortsetup{type}

The curly braces enclose the text you want in teletype. You have to be careful
with \type{\type} because the line breaking mechanism does not work anymore.

You can set up the 'typing' with:

\shortsetup{setuptyping}
\shortsetup{setuptype}

\stopsection

\startsection[title=Encodings]

In \CONTEXT\ \MKIV\ font ecoding is no issue (anymore).

\stopsection

\stopchapter

\stopcomponent


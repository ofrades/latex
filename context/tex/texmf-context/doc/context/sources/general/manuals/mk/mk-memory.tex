% language=uk

\environment mk-environment

\usemodule[timing]

\startcomponent mk-memory

\chapter{Collecting garbage}

We use the \type {mk.tex} document for testing and because it keeps track of
how \LUATEX\ evolves. As a result it has some uncommon characteristics. For
instance, you can see increments in memory usage at points where we
load fonts: the chapters on Zapfino, Arabic and CJK (unfinished). This memory
is not freed because the font memory is used permanently. In the following
graphic, the red line is the memory consumption of \LUATEX\ for the current
version of \type {mk.tex}. The blue line is the runtime per page.

\ShowNamedUsage{mk-luatex-progress}{luastate_bytes}{elapsed_time}

At the moment of writing this Taco has optimized the \LUATEX\ code base and
I have added dynamic feature support to the \MKIV\ and optimized much of the
critical \LUA\ code. At the time of writing this (December 23, 2007),
\type {mk.tex} counted 142 pages. Our rather aggressive optimizations brought
down runtime from about 29 seconds to under 16 seconds. By sharing as much font
data as possible at the \LUA\ end (at the cost of a more complex implementation)
the memory consumption of huge fonts was brought down to a level where
a somewhat \quote {older} computer with 512 MB memory could also cope with
\MKIV. Keep in mind that some fonts are just real big. Eventually we may decide
to use a more compact table model for passing \OPENTYPE\ fonts to \LUA, but this
will not happen in 2007.

The following tests show when \LUA's garbage collector becomes active. The blue
spike shows that some extra time is spent on this initially. After that garbage
more garbage is collected, which makes the time spent per page slightly higher.

\starttyping
\usemodule[timing] \starttext \dorecurse{2000}{
    \input tufte \par \input tufte \par \input tufte \page
} \stoptext
\stoptyping

\ShowNamedUsage{mk-timing-1-luatex-progress}{luastate_bytes}{elapsed_time}

The maximum memory footprint is somewhat misleading because \LUA\ reserves more
than needed. As discussed in an earlier chapter, it is possible to tweak to control
memory management somewhat, but eventually we decided that it does not make much sense
to divert from the default settings.

\starttyping
\usemodule[timing] \starttext \dorecurse{2000}{
    \input tufte \par \input tufte \par \input tufte \par
} \stoptext
\stoptyping

\ShowNamedUsage{mk-timing-2-luatex-progress}{luastate_bytes}{elapsed_time}

The last example of this set does not load files, but stores the text in
a macro. This is faster, although not that mich because the operating system
caches the file and there is not \UTF\ collapsing needed for this file.

\starttyping
\usemodule[timing] \starttext \dorecurse{2000}{
    \tufte \par \tufte \par \tufte \par
} \stoptext
\stoptyping

\ShowNamedUsage{mk-timing-3-luatex-progress}{luastate_bytes}{elapsed_time}

There are subtle differences in memory usage between the examples and eventually
test like these will permit us to optimize the code even further. For the record:
the first test runs in 39.5 seconds, the second on in 36.5 seconds and the last one
only takes 31.5 seconds (all in batch mode).

Keep in mind that these quotes in \type {tufte.tex} are just test samples, and
not that realistic in everyday documents. On the other hand, these tests involve
the usual font loading, node processing, attribute handling etc. They provide a
decent baseline.

Another document that we use for testing functionality and performance is the
reference manual. The preliminary beta~2 version gives the following statistics.

\ShowNamedUsage{luatexref-t-luatex-progress-runtime}{luastate_bytes}{elapsed_time}

The previous graphic shows the statistics of a run with runtime \METAPOST\ graphics
enabled. This means that, because each pagenumber comes with a graphic, for each
page \METAPOST\ is called. The speed of this call is heavily influenced by the
\METAPOST\ startup time, which in turn (in a windows platform) is influences by the
initialization time of the \KPSE\ library. Technically the call time can near zero
but this demands sharing libraries and databases. Anyhow, we're moving towards an
embedded \METAPOST\ library anyway, and the next graphic shows what will happen then.
Here we run \CONTEXT\ in delayed \METAPOST\ mode: graphics are collected and processed
between runs. Where the runtime variant takes some 45 seconds processing time, the
intermediate versions takes 15.

\ShowNamedUsage{luatexref-t-luatex-progress-intermediate}{luastate_bytes}{elapsed_time}

In the \type {mk.tex} document we use \TYPEONE\ fonts for the main body of the text and
load some (huge) \OPENTYPE\ fonts later on. Here we use \OPENTYPE\ fonts exclusively and
since \CONTEXT\ loads fonts only when needed, you see several spikes in the time per page
bars and memory consumption quickly becomes stable. Interesting is that contrary to the
\type {tufte.tex} samples, memory usage is quite stable. Here we don't have a memory
sawtooth and no garbage collection spikes.

The previous graphics combine \LUA\ memory consumption with time spent per page. The
following graphics show variants of this. The graphics concern this document
(\type{mk.tex}). Again, the blue lines represent the runtime per page.

\ShowMemoryUsage{mk-luatex-progress}

In \LUATEX\ node memory management is rewritten. Contrary to what you may expect, node memory
consumption is not that large. Pages seldom contain more than 5000 nodes, although
extensive use of attributes can easily duplicate this. Node usage in this documents is
as follows.

\ShowNodeUsage{mk-luatex-progress}

If node memory usage stays high, i.e.\ is not reclaimed, this can be an indication of
a memory leak. In the December 2007 beta version there is such a leak in math subformulas,
something that will be resolved when math node processing is opened up. The current \MKIV\
code cleans up most of its temporary data. We do so, because it permits us to keep an eye
on unwanted memory leaks. When writing this chapter, some of the peaks in the graphics
coincided with peaks in the runtime per page, which is no surprise.

If you want to run such tests yourself, you need to load a module at startup:

\starttyping
\usemodule[timing]
\stoptyping

The graphics can be generated with:

\starttyping
\def\ShowUsage      {optional filename}
\def\ShowNamedUsage {optional filename}{red graphic}{blue graphic}
\def\ShowMemoryUsage{optional filename}
\def\ShowNodeUsage  {optional filename}
\stoptyping

(This interface may change.)

\stopcomponent

% language=uk

\startcomponent colors-introduction

\environment colors-environment

\startchapter[title=Introduction][color=darkgray]

This manual fits in the series where we discus fundamental subsystems like fonts
and languages. Here we will collect the more technical backgrounds. This document
is not meant as a manual for users who start with \CONTEXT, for that we have
other manuals.

Color has a rather long history in \CONTEXT\ because we supported it right from
the start. In the times that \DVI\ backend drivers were used, specials were the
way to force color in the result. However, each driver had different demands:
some expected specific color directives, others a sequence of for instance
\POSTSCRIPT\ commands. When \PDF\ showed up, resource management entered the
game. Because ot always used a backend driver model in \CONTEXT, it could easily
be adapted. All management, for instance of nested colors, was done in \TEX\
code. If advanced color support hadn't been available right from the start, we'd
probably not be using \TEX\ now.

In \MKIV\ color support was implemented from scratch but in a for the user
downward compatible way. In that respect this manual is not going to reveal
anything revolutionary. Much of the work is now delegated to \LUA\ and because of
that directives are no longer part of the (expanded) input stream. As a result
color is now more robust and less intrusive.

Because \METAPOST\ support is well integrated, we also communicate colors to
\METAPOST. In \MKIV\ the communication between the two engines was upgraded and
hopefully evolved into an (even) more convenient interface.

External graphics are in fact islands in the document flow: they manage their
resources like colors themselves. However, there are some ways to deal with the
demands of publishers and printers with respect to colors. These will be
discussed too.

\getbuffer[underconstruction]

\startlines
Hans Hagen
PRAGMA ADE, Hasselt NL
2016
\stoplines

\stopchapter

\stopcomponent

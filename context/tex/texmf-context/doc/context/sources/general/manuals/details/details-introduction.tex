% language=uk

\startcomponent details-introduction

\environment details-environment

\starttitle[title={Introduction}]

On the \CONTEXT\ mailing list, occasionally a user asks if we can post a complete
document with the associated style. One reason for not honouring this request is
that we want users to cook up their own styles. Besides that, there are a couple
of styles in the regular \CONTEXT\ distribution.

When browsing through this document, a \CONTEXT\ user may wonder what style was
used to achieve its look and feel. We hope that while reading the text and
playing with the examples, the reader will accomplish the skills to define more
than just simple layouts.

This document is not easy reading. Occasionally we spend some time explaining
features not described in other manuals. The design of this document is to a
large extent determined by its purpose, and as a result not always functional.
For instance, we typeset on a grid which doesn't look too good. Also the order of
presenting features, tips and tricks is kind of random and unstructured. The idea
is that the visual effects will draw you to the right trick. Also, if you really
want to benefit from these features, there is no way but to read the whole story.

In spite of all its shortcomings, I hope that you enjoy reading this (yet
unfinished) manual. Keep in mind that this manual is far from finished.

\blank

\startlines
Hans Hagen
Hasselt NL
\blank
2002\high{+} MkII
2015\high{+} MkIV
\stoplines

\stoptitle

\stopcomponent

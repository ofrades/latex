\startcomponent ma-cb-en-document

\enablemode[**en-us]

\project ma-cb

\startchapter[title=How to create a textbook]

\index{input file}

Let's assume you want to create a simple textbook. It has some structure and
contains a title page, a few chapters, sections and sub sections. Of course there
is a table of contents.

\CONTEXT\ can create such a document automatically if you offer the right input
by means of a file. So first you have to create an input file. An input file
consists of a name and an extension. You can choose any name you want but the
extension has to be \type{.tex}. If you create a file with the name
\type{mybook.tex} you will find no difficulties in running \CONTEXT.

An \pagereference[inputfile] input file could look like this:

\startbuffer
\starttext

\startstandardmakeup
  \midaligned{From Hasselt to America}
  \midaligned{by}
  \midaligned{J. Jonker and C. van Marle}
\stopstandardmakeup

\placecombinedlist[content]

\chapter{Introduction}

... ties between Hasselt and America ...

\chapter[rensselaer]{The Rensselaer family}

\section{The first born}

... was born in the year ...

\section{The early years}

... in those days Hasselt was ...

\section{Living and workin in America}

... life in America was ...

\chapter[lansing]{The Lansing family}

... the Lansing family was also ...

\chapter[cuyler]{The Cuyler family}

... much later Tydeman Cuyler ...

\stoptext
\stopbuffer

{\switchtobodyfont[9pt]\typebuffer}

\CONTEXT\ expects a plain \ASCII\ input file. Of course you can use any
text|-|editor, as long as you save the file as standard \ASCII\ (also called txt
file) with the extension \type{.tex}. Note that spaces in the filename are
not allowed.

The input file contains the text you want to typeset and the \CONTEXT\ commands.
A \CONTEXT\ command begins with a backslash~\tex{}. With the command
\type{\starttext} you indicate the beginning of your text.

A command is sometimes followed by an argument which is enclosed by curly braces
\type{{}}. The command \type{\chapter[cuyler]{The Cuyler family}} that you see in
the example will have its effect on {\em The Cuyler family}. Its actions will
have effect on the design, typography and structure. The actions may be:

\startitemize[n,packed]
\item start a new page
\item increase chapter number by one
\item place chapter number in front of chapter title
\item reserve some vertical space
\item use a big font
\item put chapter title (and page number) in the table of contents
\stopitemize

Other actions concerning running heads, number resetting and interactivity are
disregarded at this moment.

Sometimes you will see two brackets (\type{[]}) directly after the command. These
brackets are used to feed specific options to the command. Further on in this
manual you will get more information on these brackets.

The commands in your input file can have the following appearance:

\starttabulate[|||]
\HL
\NC \bf Appearance of command  \NC \bf Example \NC\NR
\HL
\NC \type{\startcommand ... \stopcommand}   \NC \type{\starttext ... \stoptext} \NC\NR
\NC \type{\startcommand[] ... \stopcommand} \NC \type{\startitemize[packed] ... \stopitemize} \NC\NR
\NC \type{\command}                         \NC \type{\item} \NC\NR
\NC \type{\command[]}                       \NC \type{\in[cuyler]} \NC\NR
\NC \type{\command{}[]}                     \NC \type{\at{page}[cuyler]} \NC\NR
\NC \type{\command{}}                       \NC \type{\index{America}} \NC\NR
\NC \type{\command[]{}}                     \NC \type{\chapter[cuyler]{The Cuyler family}} \NC\NR
\HL
\stoptabulate

If you have \CONTEXT\ process the above example file, you would obtain a very
simple document with a title page, a few numbered chapters and section headers
and a table of content (because of \type{\placecombinedlist[content]}).

\stopchapter

\stopcomponent

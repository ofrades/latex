\startcomponent ma-cb-en-composedcharacters

\enablemode[**en-us]

\project ma-cb

\startchapter[title=Composite characters]

\index{accents}
\index{foreign characters}

In \in{chapter}[special chars] you have already seen that you have to type more
than one token to obtain special characters like \# \$ \% \& \_ $\{$ and $\}$.

Characters with accents for example can be composed or coded with specific
\CONTEXT\ commands in order to display them on paper. In case you have a text
editor that can display utf8 you can type the composed characters directly.

It is not within the scope of this manual to go into accented characters in math
mode. See the {\TEX Book} by Donald E. Knuth on that subject.

\in{Table}[tab:composed-characters] shows a few examples and the way you can code
composed characters.

\placetable
  [here,force]
  [tab:composed-characters]
  {Composed characters.}
  {\starttable[|c|l|l|c|]
   \HL
   \NC \bf Character \NC \bf Composed \NC \bf \CONTEXT\ command \NC \bf UTF8 \NC\SR
   \HL
   \NC ü            \NC \type{\"u}	     \NC \type{\uacute}        \NC \type{ü} \NC\FR
   \NC é            \NC \type{\'e}	     \NC \type{\egrave}        \NC \type{é} \NC\MR
   \NC â            \NC \type{\^a}       \NC \type{\acircumflex}   \NC \type{â} \NC\MR
   \NC ä            \NC \type{\"a}       \NC \type{\aacute}        \NC \type{ä} \NC\MR
   \NC à            \NC \type{\`a}	     \NC \type{\agrave}        \NC \type{à} \NC\MR
   \NC å            \NC \type{\aa}       \NC \type{\aring}         \NC \type{å} \NC\MR
   \NC ç            \NC \type{\c{c}}     \NC \type{\ccedilla}      \NC \type{ç} \NC\MR
   \NC ï            \NC \type{\"{\i}}    \NC \type{\idiaeresis}    \NC \type{ï} \NC\MR
   \NC î            \NC \type{\^{\i}}    \NC \type{\icircumflex}   \NC \type{î} \NC\MR
   \NC Ä            \NC \type{\"A}       \NC \type{\Adiaeresis}    \NC \type{Ä} \NC\MR
   \NC Å            \NC \type{\AA}       \NC \type{\Aring}         \NC \type{Å} \NC\MR
   \NC É            \NC \type{\'E}       \NC \type{\Egrave}        \NC \type{É} \NC\MR
   \NC æ            \NC \type{\ae}       \NC \type{\aeligature}    \NC \type{æ} \NC\MR
   \NC Æ            \NC \type{\AE}       \NC \type{\AEligature}    \NC \type{Æ} \NC\MR
   \NC ÿ            \NC \type{\"y}	     \NC \type{\ydiaeresis}    \NC \type{ÿ} \NC\LR
   \HL
   \stoptable}

The character you want to display should be in the font.

\stopchapter

\stopcomponent

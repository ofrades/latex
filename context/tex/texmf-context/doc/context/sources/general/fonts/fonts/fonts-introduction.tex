% language=uk

\startcomponent fonts-introduction

\environment fonts-environment

\startchapter[title=Introduction][color=darkgray]

You sit in a cave and wonder how to keep track of your winter stock. While
playing with some burned wood you end up with vertical strokes on the wall
representing how much you have in store.

You walk through the woods and wonder how to find your way back. Suddenly it
strikes you that you can put markers on trees. Years from that moment the whole
forest is marked with routes. Different symbols carry different meanings.

The next thing you want to do is to carry around information and pass it onto
following generations. So, you turn those symbols into shapes that make up the
scripts that can be used to express your languages in.

For ages scripts have evolved and the rendering of them on stone or wood and
later paper has resulted in a multitude of coherent collections of so called
glyphs. Manual labour turned into (semi) automated mass production and once that
took off, developments went fast. But the quality was still somewhat dubious,
especially when for instance specialized scripts like math had to be dealt with.

Some 30 years ago Don Knuth wrote a book, and in the process invented the \TEX\
typesetting system, the graphical language \METAFONT\ and a bunch of fonts. He
made it open and free of charge. He was well aware that the new ideas were built
on older ones that had evolved from common sense: how to keep track of things on
paper.

It is no surprise that an active community formed around these goodies. First of
all the system has no strings attached: the licence is generous and there are no
patents involved. There is also a network of user groups that takes care of
coordinated updates to the whole machinery. Of course it helps that it all
relates to Don Knuth.

Since \TEX\ showed up several open and closed source typesetting systems have
surfaced and only some of them survived. Also regular word processing has become
more clever and still become better. The \TEX\ typesetting system also moved on.
Some of its ideas have been used in other programs and some of the ideas of other
programs made their way into \TEX. However, its main property is still there: you
can tweak and tune it to your needs and are not hampered by too many limitations.

Don Knuth had this chicken or egg problem: once you can typeset a source you need
fonts but you can only make fonts if you can use them in a typesetting program.
As a result \TEX\ came with its own fonts and it has special ways to deal with
them. Given the limitations of that time \TEX\ puts some limitations on fonts and
also expects them to have certain properties, something that is most noticeable
in math fonts.

Rather soon from the start it has been possible to use third party fonts in \TEX,
for instance \TYPEONE. As \TEX\ only needs some information about the shapes, it
was the backend that integrated the font resources in the final document. One of
its descendants, \PDFTEX, had this backend built in and could do some more clever
things with fonts in the typesetting process, like protrusion and expansion. The
integration of front- and backend made live much easier. Another descendant,
\XETEX\ made it possible to move on to the often large \OPENTYPE\ fonts. On the
one hand this made live even more easy but at the other end it introduced users
to the characteristics of such fonts and making the right choices, i.e.\ not fall
in the trap of too fancy font usage.

In this manual we will look at fonts from the perspective of yet another
descendant, \LUATEX. It inherits the font technology from traditional \TEX, but
also extends it so that we can deal with modern font technologies. Of course it
offers much more, but in practice much relates to fonts one way or the other.

Of course this exploration will be from the perspective of the \CONTEXT\ macro
package but this is not a manual about how to use fonts in \CONTEXT\ as we have
another manual for that. Much of what we say here applies to the generic font
code as well, although some more advanced control is \CONTEXT\ specific. There is
nothing real new here, and it all evolved from common sense and dealing with
\TEX\ for many years. The perspective is mostly that of being a user myself so
don't complain too loudly if things look complicated and unclear.

There is some overlap between the chapters. This is because each chapter is
written from another perspective and this document quite certainly will not be
read as a whole but more by looking at examples.

\startnotabene
    This document will probably have an \quote {still under construction} state
    for a long time. The functionality discussed here will stay and more might
    show up. Of course there are errors, and they're all mine.
\stopnotabene

\startlines
Hans Hagen
PRAGMA ADE, Hasselt NL
Summer 2011 \endash\ Spring 2016
\stoplines

\stopchapter

\stopcomponent

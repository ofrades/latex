\subsectioninfo{Santa Águeda, V. e Márt.}{5 de Fevereiro}

\textit{Como na Missa Loquébar, página \pageref{virgensmartires1}, excepto:}

\paragraph{Intróito}
\begin{paracol}{2}\latim{
\rlettrine{G}{audeámus} omnes in Dómino, diem festum celebrántes sub honóre beátæ Agathæ Vírginis et Martyris: de cujus passióne gaudent Angeli et colláudant Fílium Dei. \emph{Ps. 44, 2} Eructávit cor meum verbum bonum: dico ego ópera mea Regi.
℣. Gloria Patri \emph{\&c.}
}\switchcolumn\portugues{
\rlettrine{A}{legremo-nos} todos no Senhor, celebrando neste dia a festividade em honra da B. Águeda, Virgem e Mártir, de cujo martírio se regozijam os Anjos, que louvam o Filho de Deus. \emph{Sl. 44, 2} Meu coração exprimiu uma excelente palavra: «Consagro ao Rei as minhas obras!».
℣. Glória ao Pai \emph{\&c.}
}\end{paracol}

\paragraphinfo{Epístola}{1. Cor. 1, 26-31}
\begin{paracol}{2}\latim{
Léctio Epístolæ beáti Pauli Apóstoli ad Corinthios.
}\switchcolumn\portugues{
Lição da Ep.ª do B. Ap.º Paulo aos Coríntios.
}\switchcolumn*\latim{
\rlettrine{F}{ratres:} Vidéte vocatiónem vestram: quia non multi sapiéntes secúndum carnem, non multi poténtes, non multi nóbiles: sed quæ stulta sunt mundi elégit Deus, ut confúndat sapiéntes: et infírma mundi elégit Deus, ut confúndat fórtia: et ignobília mundi et contemptibília elégit Deus, et ea quæ non sunt, ut ea quæ sunt destrúeret: ut non gloriétur omnis caro in conspéctu ejus. Ex ipso autem vos estis in Christo Jesu, qui factus est nobis sapiéntia a Deo, et justítia, et sanctificátio, et redémptio: ut, quemádmodum scriptum est: Qui gloriátur, in Dómino gloriétur.
}\switchcolumn\portugues{
\rlettrine{E}{xaminai} a vossa vocação, porque não há, segundo a carne, nem muitos sábios, nem muitos poderosos, nem muitos nobres; mas Deus escolheu aquele que o mundo julga insensato, para confundir os sábios; e aquele que o mundo considera fraco, para confundir os fortes; e aquele que o mundo considera vil e desprezível e o que não vale nada, para destruir o que tem valor, a fim de que nenhuma criatura humana se glorie diante d’Ele. É por Ele que estais em Jesus Cristo, que nos foi dado por Deus para ser nossa sabedoria, justiça, santificação e redenção, para que se cumpra o que está escrito: «O que se gloria, glorie-se no Senhor».
}\end{paracol}

\paragraphinfo{Gradual}{Sl. 45, 6 \& 5}
\begin{paracol}{2}\latim{
\rlettrine{A}{djuvábit} eam Deus vultu suo: Deus in médio ejus, non commovébitur. ℣. Flúminis impetus lætíficat civitátem Dei: sanctificávit tabernáculum suum Altíssimus.
}\switchcolumn\portugues{
\rlettrine{O}{} Senhor a auxiliará com seu olhar: Deus está no meio dela e a não deixará vacilar. ℣. Um rio com suas águas alegra a cidade de Deus. O Altíssimo santificou o seu tabernáculo.
}\switchcolumn*\latim{
Allelúja, allelúja. ℣. \emph{Ps. 118, 46} Loquébar de testimóniis tuis in conspéctu regum, et non confundébar. Allelúja.
}\switchcolumn\portugues{
Aleluia, aleluia. ℣. \emph{Sl. 118, 46} Perante os reis, publicarei os vossos testemunhos e não me envergonharei. Aleluia.
}\end{paracol}

\textit{Após a Septuagésima, omite-se o Aleluia e o Verso, e diz-se o:}

\paragraphinfo{Trato}{Sl. 125, 5-6}
\begin{paracol}{2}\latim{
\qlettrine{Q}{ui} séminant in lácrimis, in gáudio metent. ℣. Eúntes ibant et fiébant, mitténtes semina sua. ℣. Veniéntes autem vénient cum exsultatióne, portántes manípulos suos.
}\switchcolumn\portugues{
\rlettrine{A}{queles} que semearam com lágrimas colherão com alegria. ℣. Iam, caminhavam e lançavam a semente à terra, chorando. ℣. Mas regressaram com alegria, transportando os seus molhos de trigo.
}\end{paracol}

\paragraphinfo{Evangelho}{Mt. 19, 3-12}
\begin{paracol}{2}\latim{
\cruz Sequéntia sancti Evangélii secúndum Matthǽum.
}\switchcolumn\portugues{
\cruz Continuação do santo Evangelho segundo S. Mateus.
}\switchcolumn*\latim{
\blettrine{I}{n} illo témpore: Accessérunt ad Jesum pharisǽi, tentántes eum et dicéntes: Si licet hómini dimíttere uxórem suam quacúmque ex causa? Qui respóndens, ait eis: Non legístis, quia, qui fecit hóminem ab inítio, másculum et féminam fecit eos? et dixit: Propter hoc dimíttet homo patrem, et matrem, et adhærébit uxóri suæ, et erunt duo in carne una. Itaque jam non sunt duo, sed una caro. Quod ergo Deus conjúnxit, homo non séparet. Dicunt illi: Quid ergo Móyses mandávit dare libéllum repúdii, et dimíttere? Ait illis: Quóniam Móyses ad durítiam cordis vestri permísit vobis dimíttere uxóres vestras: ab inítio autem non fuit sic. Dico autem vobis, quia, quicúmque dimíserit uxórem suam, nisi ob fornicatiónem, et áliam dúxerit, mœchátur: et qui dimíssam duxerit, mœchátur. Dicunt ei discípuli ejus: Si ita est causa hóminis cum uxore, non expedit nubere. Qui dixit illis: Non omnes cápiunt verbum istud, sed quibus datum est. Sunt enim eunúchi, qui de matris útero sic nati sunt; et sunt eunúchi, qui facti sunt ab homínibus; et sunt eunúchi, qui seípsos castravérunt propter regnum cœlórum. Qui potest cápere, cápiat.
}\switchcolumn\portugues{
\blettrine{N}{aquele} tempo, aproximaram-se os fariseus de Jesus para O tentar e disseram-Lhe: «É lícito ao homem repudiar sua mulher por qualquer causa?». Respondendo Jesus, disse-lhes: «Não lestes: «Aquele que criou o homem no Princípio do mundo criou um homem e uma mulher, e disse que por causa disto o homem deixará seu pai e sua mãe, e se unirá com sua mulher; e serão dois em uma só carne?». Assim, não serão mais dois, mas uma só carne. Que o homem, pois, não separe o que Deus uniu». Eles disseram-Lhe: «Porque mandou, então, Moisés dar carta de repúdio e deixá-la?». Ele respondeu: «Foi por causa da dureza do vosso coração que Moisés permitiu que repudiásseis vossas mulheres; mas no princípio não foi assim. E Eu vos digo: todo aquele que deixar sua mulher, a não ser por adultério, e casar com outra, comete adultério; e aquele que casar com uma mulher repudiada, também comete adultério». Disseram-Lhe, então, os discípulos: «Se tal é a situação do homem diante da mulher, melhor é não se casar». E Ele disse-lhes: «Nem todos são capazes de compreender estas palavras, mas só aqueles a quem isso é dado; pois há eunucos que já assim vieram do seio de sua mãe; há outros que foram feitos pelos homens; e há ainda outros que se fizeram a si mesmo, por causa do reino dos céus. Quem pode compreender isto, compreenda».
}\end{paracol}

\paragraph{Comúnio}
\begin{paracol}{2}\latim{
\qlettrine{Q}{ui} me dignátus est ab omni plaga curáre et mamíllam meam meo péctori restitúere, ipsum ínvoco Deum vivum.
}\switchcolumn\portugues{
\rlettrine{I}{nvoco} como Deus vivo Aquele que se dignou curar as minhas chagas e restituir o meu seio ao meu peito!
}\end{paracol}

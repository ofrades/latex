\subsectioninfo{S. S. Ábdon e Senen, Mártires}{30 de Julho}

\textit{Como Missa Intret in, página \pageref{muitosmartires1}, excepto:}

\paragraph{Oração}
\begin{paracol}{2}\latim{
\rlettrine{D}{eus,} qui sanctis tuis Abdon et Sennen ad hanc glóriam veniéndi copiósum munus grátiæ contulísti: da fámulis tuis suorum véniam peccatórum; ut, Sanctórum tuórum intercedéntibus méritis, ab ómnibus mereántur adversitátibus liberáti. Per Dóminum \emph{\&c.}
}\switchcolumn\portugues{
\slettrine{Ó}{} Deus, que conferistes aos vossos Santos Ábdon e Senen os tesouros abundantes da vossa graça para poderem alcançar a glória, concedei aos vossos servos o perdão dos seus pecados, a fim de que pelo valor dos méritos dos vossos Santos mereçamos ser livres de todas as adversidades. Por nosso Senhor \emph{\&c.}
}\end{paracol}

\paragraphinfo{Epístola}{2. Cor. 6, 4-10}
\begin{paracol}{2}\latim{
Lectio Epístolæ beati Pauli Apostoli ad Corinthios.
}\switchcolumn\portugues{
Lição da Ep.ª do B. Ap.º Paulo aos Coríntios.
}\switchcolumn*\latim{
\rlettrine{F}{ratres:} Exhibeámus nosmetípsos sicut Dei minístros, in multa patiéntia, in tribulatiónibus, in necessitátibus, in angústiis, in plagis, in carcéribus, in seditiónibus, in labóribus, in vigíliis, in jejúniis, in castitáte, in sciéntia, in longanimitáte, in suavitáte, in Spíritu Sancto, in caritáte non ficta, in verbo veritátis, in virtúte Dei, per arma justítiæ a dextris et a sinístris: per glóriam et ignobilitátem: per infámiam et bonam famam: ut seductóres et veráces: sicut qui ignóti et cógniti: quasi moriéntes et ecce, vívimus: ut castigáti et non mortificáti: quasi tristes, semper autem gaudéntes: sicut egéntes, multos autem locupletántes: tamquam nihil habéntes et ómnia possidéntes.
}\switchcolumn\portugues{
\rlettrine{M}{eus} irmãos: Mostremo-nos dignos ministros de Deus em todas as coisas, principalmente com muita paciência, tanto nas tribulações, nas necessidades, nas angústias, nos açoites, nas prisões, nas revoltas, nas fadigas, nas vigílias e nos jejuns, como pela pureza, pela ciência, pela longanimidade, pela bondade, pelo Espírito Santo, pela verdadeira caridade, pela palavra da verdade, pelo poder de Deus, pelas armas da justiça com que combatemos à direita e à esquerda; na honra e na ignomínia; na boa e na má fama; sendo julgados sedutores, ainda que sejamos sinceros e verdadeiros; sendo julgados desconhecidos, e, contudo, sendo bastante conhecidos; sendo considerados moribundos, e, contudo, estando bem vivos; sendo considerados condenados, e, contudo, escapando à morte; sendo julgados tristes, mas estando alegres; sendo julgados pobres, mas enriquecendo muitos; sendo considerados como não tendo nada, mas possuindo tudo.
}\end{paracol}

\paragraphinfo{Gradual}{Ex. 15, 11}
\begin{paracol}{2}\latim{
\rlettrine{G}{loriosus} Deus in Sanctis suis: mirábilis in majestáte, fáciens prodígia. ℣. \emph{ibid., 6} Déxtera tua, Dómine, glorificáta est in virtúte: déxtera manus tua confrégit inimícos.
}\switchcolumn\portugues{
\rlettrine{D}{eus} é glorioso em seus Santos: e admirável na sua majestade, praticando prodígios. ℣. \emph{ibid., 6} Senhor, a vossa dextra engrandeceu-se pela sua força: a vossa dextra esmagou os inimigos.
}\switchcolumn*\latim{
Allelúja, allelúja. ℣. \emph{Sap. 3, 1} Justórum ánimæ in manu Dei sunt, et non tanget illos torméntum malítiæ. Allelúja.
}\switchcolumn\portugues{
Aleluia, aleluia. ℣. \emph{Sb. 3, 1} As almas dos justos estão nas mãos de Deus e o tormento da malícia os não ferirá. Aleluia.
}\end{paracol}

\paragraphinfo{Evangelho}{Mt. 5, 1-12}
\begin{paracol}{2}\latim{
\cruz Sequéntia sancti Evangélii secúndum Matthǽum.
}\switchcolumn\portugues{
\cruz Continuação do santo Evangelho segundo S. Mateus.
}\switchcolumn*\latim{
\blettrine{I}{n} illo témpore: Videns Jesus turbas, ascéndit in montem, et cum sedísset, accessérunt ad eum discípuli ejus, et apériens os suum, docébat eos, dicens: Beáti páuperes spíritu: quóniam ipsórum est regnum cœlórum. Beáti mites: quóniam ipsi possidébunt terram. Beáti, qui lugent: quóniam ipsi consolabúntur. Beáti, qui esúriunt et sítiunt justítiam: quóniam ipsi saturabúntur. Beáti misericórdes: quóniam ipsi misericórdiam consequántur. Beáti mundo corde: quóniam ipsi Deum vidébunt. Beáti pacífici: quóniam fílii Dei vocabúntur. Beáti, qui persecutiónem patiúntur propter justítiam: quóniam ipsórum est regnum cœlórum. Beáti estis, cum maledíxerint vobis, et persecúti vos fúerint, et díxerint omne malum advérsum vos, mentiéntes, propter me: gaudete et exsultáte, quóniam merces vestra copiósa est in cœlis.
}\switchcolumn\portugues{
\blettrine{N}{aquele} tempo, vendo Jesus as turbas do povo, que O seguiam, subiu para uma montanha. Então assentou-se, aproximando-se d’Ele os discípulos. Depois, tomando a palavra, ensinou assim aos discípulos: «Bem-aventurados os pobres de espírito, porque deles é o reino dos céus. Bem-aventurados os mansos, porque possuirão a terra. Bem-aventurados os que choram, porque serão consolados. Bem-aventurados os que têm fome e sede de justiça, porque serão saciados. Bem-aventurados os misericordiosos, porque serão tratados com misericórdia. Bem-aventurados os que possuem o coração puro, porque verão Deus. Bem-aventurados os pacíficos, porque serão chamados filhos de Deus. Bem-aventurados os que padecem perseguição por amor da justiça, porque lhes pertencerá o reino dos céus. Bem-aventurados vós, quando os homens vos amaldiçoarem, perseguirem e caluniarem por minha causa. Regozijai-vos, então, e exultai de alegria, pois uma copiosa recompensa vos está preparada no céu».
}\end{paracol}

\paragraph{Secreta}
\begin{paracol}{2}\latim{
\rlettrine{H}{æc} hóstia, quǽsumus, Dómine, quam sanctórum Mártyrum tuórum natalítia recenséntes offérimus: et víncula nostræ pravitátis absolvat, et tuæ nobis misericórdiæ dona concíliet. Per Dóminum \emph{\&c.}
}\switchcolumn\portugues{
\rlettrine{S}{enhor,} Vos suplicamos, fazei que esta hóstia, que Vos oferecemos em honra do nascimento no céu dos vossos Santos Mártires, nos livre das cadeias dos nossos pecados e nos alcance os dons da vossa misericórdia. Por nosso Senhor \emph{\&c.}
}\end{paracol}

\paragraphinfo{Comúnio}{Sl. 78, 2 \& 11}
\begin{paracol}{2}\latim{
\rlettrine{P}{osuérunt} mortália servórum tuórum, Dómine, escas volatílibus cœli, carnes Sanctórum tuórum béstiis terræ: secúndum magnitúdinem bráchii tui pósside fílios morte punitórum.
}\switchcolumn\portugues{
\rlettrine{S}{enhor,} deram como alimento às aves do céu os corpos dos vossos servos, que haviam sido mortos, e deram as carnes dos vossos Santos às feras da terra. Pelo poder do vosso braço Conservai os filhos daqueles que foram mortos.
}\end{paracol}

\paragraph{Postcomúnio}
\begin{paracol}{2}\latim{
\rlettrine{P}{er} hujus, Dómine, operationem mystérii, et vitia nostra purgéntur: et, intercedéntibus sanctis Martyribus tuis Abdon et Sennen, justa desidéria compleántur. Per Dóminum \emph{\&c.}
}\switchcolumn\portugues{
\rlettrine{F}{azei,} Senhor, pela virtude deste mistério, que os nossos vícios sejam apagados e que, pela intercessão dos vossos Santos Mártires Ábdon e Senen, sejam realizados os nossos justos desejos. Por nosso Senhor \emph{\&c.}
}\end{paracol}

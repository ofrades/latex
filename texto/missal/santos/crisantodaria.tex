\subsectioninfo{S. S. Crisanto e Daria, Mártires}{25 de Outubro}

\textit{Como Missa Intret in, página \pageref{muitosmartires1}, excepto:}

\paragraph{Oração}
\begin{paracol}{2}\latim{
\rlettrine{B}{eatórum} Mártyrum tuórum, Dómine, Chrysánthi et Dáriæ, quǽsumus, adsit nobis orátio: ut, quos venerámur obséquio, eórum pium júgiter experiámur auxílium. Per Dóminum \emph{\&c.}
}\switchcolumn\portugues{
\qlettrine{Q}{ue} a oração dos vossos B. B. Mártires Crisanto e Daria nos assista sempre, Senhor, Vos rogamos, a fim de que, venerando-os com as nossas homenagens, experimentemos incessantemente o seu piedoso auxílio. Por nosso Senhor \emph{\&c.}
}\end{paracol}

\paragraphinfo{Evangelho}{Lc. 11, 47-51}
\begin{paracol}{2}\latim{
\cruz Sequéntia sancti Evangélii secúndum Lucam.
}\switchcolumn\portugues{
\cruz Continuação do santo Evangelho segundo S. Lucas.
}\switchcolumn*\latim{
\blettrine{I}{n} illo témpore: Dicébat Jesus scribis et pharisǽis: Væ vobis, qui ædificátis monuménta prophetárum: patres autem vestri occidérunt illos. Profécto testificámini, quod consentítis opéribus patrum vestrórum: quóniam ipsi quidem eos occidérunt, vos autem ædificátis eórum sepúlcra. Proptérea et sapiéntia Dei dixit: Mittam ad illos prophétas et apóstolos, et ex illis occídent et persequántur: ut inquirátur sanguis ómnium prophetárum, qui effúsus est a constitutióne mundi a generatióne ista, a sánguine Abel usque ad sánguinem Zacharíæ, qui périit inter altáre et ædem. Ita dico vobis, requirétur ab hac generatióne.
}\switchcolumn\portugues{
\blettrine{N}{aquele} tempo, disse Jesus aos escribas e fariseus: «Ai de vós, que edificais túmulos aos Profetas, que vossos pais mataram. Deste modo servis de testemunho e aplaudis as obras dos vossos pais, pois mataram-nos; e edificais túmulos em sua honra. Eis porque a sabedoria de Deus disse: enviar-lhes-ei Profetas e Apóstolos; mas matarão uns e expulsarão outros, a fim de que a esta geração seja tomada conta do sangue dos profetas, que derramou desde a criação do mundo e do sangue de Abel, até ao sangue de Zacarias, morto entre o altar e o santuário. Sim, eu vo-lo digo: disto será pedida conta a esta geração».
}\end{paracol}

\paragraph{Secreta}
\begin{paracol}{2}\latim{
\rlettrine{P}{óuli} tui, quǽsumus, Dómine, tibi grata sit hóstia, quæ in natalítiis sanctórum Mártyrum tuórum Chrysánthi et Dáriæ sollémniter immolátur. Per Dóminum nostrum \emph{\&c.}
}\switchcolumn\portugues{
\rlettrine{S}{enhor,} Vos rogamos, fazei que Vos seja agradável esta hóstia, que solenemente é imolada em honra dos vossos Santos Mártires Crisanto e Daria. Por nosso Senhor \emph{\&c.}
}\end{paracol}

\paragraph{Postcomúnio}
\begin{paracol}{2}\latim{
\rlettrine{M}{ýsticis,} Dómine, repléti sumus votis et gáudiis: præsta, quǽsumus; ut, intercessiónibus sanctórum Mártyrum tuórum Chrysánthi et Dáriæ, quæ temporáliter ágimus, spirituáliter consequámur. Per Dóminum \emph{\&c.}
}\switchcolumn\portugues{
\rlettrine{H}{avendo} sido repletos com os gozos místicos, qual era o objecto dos nossos votos, Senhor, Vos suplicamos, concedei-nos que por intercessão dos vossos Santos Mártires Crisanto e Daria alcancemos espiritualmente o que agora celebrámos. Por nosso Senhor \emph{\&c.}
}\end{paracol}
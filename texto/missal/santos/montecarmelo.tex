\subsectioninfo{B. V. Maria do Monte Carmelo}{16 de Julho}\label{montecarmelo}

\paragraphinfo{Intróito}{Sedulius}
\begin{paracol}{2}\latim{
\rlettrine{G}{audeámus} omnes in Dómino, diem festum celebrántes sub honóre beátæ Maríæ Vírginis: de cujus sollemnitáte gaudent Angeli et colláudant Fílium Dei. \emph{Ps. 44, 2} Eructávit cor meum verbum bo
num: dico ego ópera mea Regi.
℣. Gloria Patri \emph{\&c.}
}\switchcolumn\portugues{
\rlettrine{A}{legremo-nos} todos no Senhor, no dia em que celebramos a festa em honra da B. V. Maria: os Anjos regozijam-se com esta festa e louvam unissonamente o Filho de Deus. \emph{Sl. 44, 2} Meu coração exprimiu uma excelente palavra: Consagro ao Rei as minhas obras.
℣. Glória ao Pai \emph{\&c.}
}\end{paracol}

\paragraph{Oração}
\begin{paracol}{2}\latim{
\rlettrine{D}{eus,} qui beatíssimæ semper Vírginis et Genetrícis tuæ Maríæ singulári título Carméli órdinem decorásti: concéde propítius; ut, cujus hódie Commemoratiónem sollémni celebrámus offício, ejus muníti præsídiis, ad gáudia sempitérna perveníre mereámur: Qui vivis \emph{\&c.}
}\switchcolumn\portugues{
\slettrine{Ó}{} Deus, que ornastes a Ordem do Carmelo com a insigne honra de usar o nome da B. Maria, sempre Virgem e vossa Mãe, concedei-nos propício que, sendo nós auxiliados pela protecção daquela cuja comemoração celebramos hoje, solenemente, sejamos dignos de alcançar as eternas alegrias. Ó Vós, que \emph{\&c.}
}\end{paracol}

\paragraphinfo{Epístola}{Ecl. 24, 23-31}
\begin{paracol}{2}\latim{
Léctio libri Sapiéntiæ.
}\switchcolumn\portugues{
Lição do Livro da Sabedoria.
}\switchcolumn*\latim{
\rlettrine{E}{go} quasi vitis fructificávi suavitátem odóris: et flores mei fructus honóris et honestátis. Ego mater pulchræ dilectiónis et timóris et agnitiónis et sanctæ spei. In me grátia omnis viæ et veritátis: in me omnis spes vitæ et virtútis. Transíte ad me, omnes qui concupíscitis me, et a generatiónibus meis implémini. Spíritus enim meus super mel dulcis, et heréditas mea super mel et favum. Memória mea in generatiónes sæculórum. Qui edunt me, adhuc esúrient: et qui bibunt me, adhuc sítient. Qui audit me, non confundétur: et qui operántur in me, non peccábunt. Qui elúcidant me, vitam ætérnam habébunt.
}\switchcolumn\portugues{
\rlettrine{E}{u} produzi, como a vinha, flores de suave odor, e as minhas flores são frutos de honra e de honestidade. Eu sou a mãe do amor puro, do temor, da ciência e da esperança santa. Em mim existe toda a graça do caminho e da verdade; em mim existe toda a esperança da vida e da virtude. Vinde a mim, ó vós, que me desejais com ardor, e saciai-vos com meus frutos; pois o meu espírito é mais doce do que o mel e a minha herança excede em doçura o próprio favo de mel! Minha memória permanecerá nas gerações de todos os séculos. Aqueles que me comerem terão ainda fome; e aqueles que me beberem terão ainda sede. Aqueles que me escutam não serão confundidos; aqueles que se orientarem em mim não pecarão; e aqueles que me tornarem conhecida alcançarão a vida eterna.
}\end{paracol}

\paragraph{Gradual}
\begin{paracol}{2}\latim{
\rlettrine{B}{enedícta} et venerábilis es, Virgo María: quæ sine tactu pudóris invénta es Mater Salvatóris. ℣. Virgo, Dei Génetrix, quem totus non capit orbis, in tua se clausit víscera factus homo.
}\switchcolumn\portugues{
\rlettrine{B}{endita} e venerável sois, ó Virgem Maria, que, sem a mais leve mancha de impureza, fostes a Mãe do Salvador. ℣. Ó Virgem, Mãe de Deus, Aquele que nem todo o universo é capaz de conter, esteve encerrado no vosso seio, fazendo-se homem.
}\switchcolumn*\latim{
Allelúja, allelúja. ℣. Per te, Dei Génetrix, nobis est vita pérdita data: quæ de cœlo suscepísti prolem, et mundo genuísti Salvatórem. Allelúja.
}\switchcolumn\portugues{
Aleluia, aleluia. ℣. Por vós, ó Mãe de Deus, nos foi restituída a vida que havíamos perdido! Vós recebestes do céu a graça de serdes Mãe, gerando o Salvador do mundo. Aleluia.
}\end{paracol}

\paragraphinfo{Evangelho}{Página \pageref{comumfestasmaria1}}

\paragraphinfo{Ofertório}{Jr. 18, 20}
\begin{paracol}{2}\latim{
\rlettrine{R}{ecordáre,} Virgo Mater, in conspéctu Dei, ut loquáris pro nobis bona, et ut avértat indignatiónem suam a nobis.
}\switchcolumn\portugues{
\rlettrine{R}{ecordai-vos,} ó Virgem Maria, de interceder por nós junto de Deus e de conseguirdes afastar de nós a sua indignação.
}\end{paracol}

\paragraph{Secreta}
\begin{paracol}{2}\latim{
\rlettrine{S}{anctífica,} Dómine, quǽsumus, obláta libámina: et, beátæ Dei Genetrícis Maríæ salubérrima intercessióne, nobis salutária fore concéde. Per eúndem Dóminum \emph{\&c.}
}\switchcolumn\portugues{
\rlettrine{S}{antificai,} Senhor, Vos pedimos, as oblatas que Vos são apresentadas, e, pela eficacíssima intercessão da B. V. Maria, fazei que nos sejam salutares. Pelo mesmo nosso Senhor \emph{\&c.}
}\end{paracol}

\paragraph{Comúnio}
\begin{paracol}{2}\latim{
\rlettrine{R}{egina} mundi digníssima, María, Virgo perpétua, intercéde pro nostra pace et salúte, quæ genuísti Christum Dóminum, Salvatórem ómnium.
}\switchcolumn\portugues{
\slettrine{Ó}{} Maria, digníssima Rainha do mundo e sempre Virgem, que gerastes Cristo, Senhor e Salvador de todos, alcançai-nos pela vossa intercessão a paz e a salvação.
}\end{paracol}

\paragraph{Postcomúnio}
\begin{paracol}{2}\latim{
\rlettrine{A}{djuvet} nos, quǽsumus, Dómine, gloriósæ tuæ Genetrícis sempérque Vírginis Maríæ intercéssio veneránda: ut, quos perpétuis cumulávit benefíciis, a cunctis perículis absolútos, sua fáciat pietáte concórdes: Qui vivis \emph{\&c.}
}\switchcolumn\portugues{
\qlettrine{Q}{ue} a augusta intercessão de Maria, vossa gloriosa Mãe e sempre Virgem, nos sirva de auxílio, Senhor, Vos rogamos; e que, depois de havermos sido perpetuamente cumulados dos seus benefícios e livres de todos os perigos, a sua bondade nos faça viver em concórdia. Ó Vós, que, sendo Deus \emph{\&c.}
}\end{paracol}

\subsectioninfo{S. Nazário e Outros, Mártires}{28 de Julho}

\textit{Como Missa Intret in, página \pageref{muitosmartires1}, excepto:}

\paragraph{Oração}
\begin{paracol}{2}\latim{
\rlettrine{S}{anctórum} tuórum nos, Dómine, Nazarii, Celsi, Vittóris et Innocéntii conféssio beáta commúniat: et fragilitáti nostræ subsídium dignánter exóret. Per Dóminum \emph{\&c.}
}\switchcolumn\portugues{
\qlettrine{Q}{ue} a gloriosa profissão de fé dos vossos Santos Nazário, Celso, Vítor e Inocêncio nos fortaleça, Senhor, e que da vossa bondade alcancemos socorro para a nossa fraqueza. Por nosso Senhor \emph{\&c.}
}\end{paracol}

\paragraphinfo{Epístola}{Sb. 10, 17-20}
\begin{paracol}{2}\latim{
Léctio libri Sapiéntiæ.
}\switchcolumn\portugues{
Lição do Livro da Sabedoria.
}\switchcolumn*\latim{
\rlettrine{R}{éddidit} Deus justis mercédem labórum suorum, et deduxit illos in via mirábili: et fuit illis in velaménto diei et in luce stellárum per noctem: tránstulit illos per Mare Rubrum, et transvéxit illos per aquam nímiam. Inimícos autem illórum demérsit in mare, et ab altitúdine inferórum edúxit illos. Ideo justi tulérunt spolia impiórum, et decantavérunt, Dómine, nomen sanctum tuum, et victrícem manum tuam laudavérunt páriter, Dómine, Deus noster.
}\switchcolumn\portugues{
\rlettrine{D}{eus} concedeu aos justos a recompensa dos seus trabalhos, conduzindo-os por um caminho admirável. Foi para eles sombra durante o dia e luz das estrelas durante a noite. Fê-los atravessar o mar Vermelho e guiou-os através das águas caudalosas. E submergiu no mar os seus inimigos, arrojando à praia os cadáveres. Eis porque os justos arrebataram os despojos dos ímpios, cantaram hinos em honra do vosso santo nome, ó Senhor, e louvaram em harmonia a vossa mão vitoriosa, ó Senhor, nosso Deus.
}\end{paracol}

\paragraph{Secreta}
\begin{paracol}{2}\latim{
\rlettrine{C}{oncéde} nobis, omnípotens Deus: ut his munéribus, quæ in sanctórum tuórum Nazarii, Celsi, Victóris et Innocéntii honóre deférimus, et te placémus exhíbitis, et nos vivificémur accéptis. Per Dóminum \emph{\&c.}
}\switchcolumn\portugues{
\slettrine{Ó}{} Deus omnipotente, permiti que, oferecendo-Vos estas oblatas em honra dos vossos Santos Nazário, Celso, Vítor e Inocêncio, possamos aplacar-Vos, e, aceitando-as Vós, por elas alcancemos a vida. Por nosso Senhor \emph{\&c.}
}\end{paracol}

\paragraph{Postcomúnio}
\begin{paracol}{2}\latim{
\rlettrine{S}{anctórum} Nazárii, Celsi, Victóris et Innocéntii, Dómine, intercessióne placátus: præsta, quǽsumus; ut, quod temporáli celebrámus actióne, perpétua salvatióne capiámus. Per Dóminum \emph{\&c.}
}\switchcolumn\portugues{
\rlettrine{D}{eixai-Vos} aplacar, Senhor, pela intercessão dos vossos Santos Nazário, Celso, Vítor e Inocêncio e, Vos suplicamos, permiti que estes mystérios, celebrados temporalmente, nos façam alcançar a salvação perpétua. Por nosso Senhor \emph{\&c.}
}\end{paracol}
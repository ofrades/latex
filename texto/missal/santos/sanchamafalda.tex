\subsectioninfo{B. B. Sancha e Mafalda, Virgens}{13 de Março}

\paragraphinfo{Intróito}{Sl. 44, 9 \& 16}
\begin{paracol}{2}\latim{
\rlettrine{}D{electavérunt} te, Deus, fíliæ regum in honóre tuo: afferéntur vírgines in lætítia et exultatióne: adducéntur in templum Regis Dómini. \emph{Ps. ibid., 2} Eructávit cor meum verbum bonum: dico ego ópera mea Regi.
℣. Gloria Patri \emph{\&c.}
}\switchcolumn\portugues{
\rlettrine{A}{mam-Vos,} ó Deus, as filhas dos reis e formam a vossa corte de honra: As virgens serão apresentadas com alegria e em transportes de júbilo e serão conduzidas ao templo do Pai: do Senhor. \emph{Sl. ibid., 2} Meu coração exprimiu uma palavra sublime: «Consagro ao Rei as minhas obras».
℣. Glória ao Pai \emph{\&c.}
}\end{paracol}

\paragraph{Oração}
\begin{paracol}{2}\latim{
\rlettrine{D}{eus,} qui beátas Virgines Sanciam et Mafáldam, mundáno principátu et sæculi pompa despéctis, a terrénis ad cæléstes Agni núptias evocásti: da nobis fámulis tuis; ut terréna despiciéntes, cæléstium bonórum fácias esse consórtes. Per eúmdem Dóminum \emph{\&c.}
}\switchcolumn\portugues{
\slettrine{Ó}{} Deus, que chamastes das bodas terrenas para, as núpcias celestiais do Cordeiro Imaculado as B. B. Sancha e Mafalda, que logo desprezaram as honras da realeza humana e outras pompas terrestres, permiti aos vossos servos que, desprezando também os interesses terrenos, possam compartilhar do gozo dos bens celestiais. Pelo mesmo nosso Senhor \emph{\&c.}
}\end{paracol}

\paragraphinfo{Epístola}{Página \pageref{19muitasvirgensmartires}}

\paragraphinfo{Gradual}{Sl. 148, 12-13}
\begin{paracol}{2}\latim{
\rlettrine{V}{írgines} laudent nomen Dómini: quia exaltátum est nomen ejus solíus. ℣. \emph{Ps. 23, 6} Hæc est generátio quæréntium Dóminum, quærémtium fáciem Dei Jacbob.
}\switchcolumn\portugues{
\qlettrine{Q}{ue} as virgens louvem o nome do Senhor: pois só o seu nome foi exaltado. ℣. Esta pertence à gereção das que procuram a face de Deus de Jacob.
}\end{paracol}

\paragraphinfo{Trato}{Sl. 44, 7}
\begin{paracol}{2}\latim{
\rlettrine{S}{edes} tua, Deus, in sæculum sæculi: virga directiónis virga regni tui. \emph{Ps. ibid., 13 \& 10} Vultum tuum deprecabúntur omnes divites plebis: filiæ regum in honóre tuo. ℣. \emph{Ps. ibid., 16} Afferéntur in lætítia et exsultatióne: adducéntur in templum Regi Dómino.
}\switchcolumn\portugues{
\slettrine{Ó}{} Deus, vosso trono é eterno: o ceptro da rectidão é o ceptro do vosso reino. \emph{Ps. ibid., 13 \& 10} Todos os ricos da terra implorarão o vosso olhar: as filhas dos reis formarão a vossa corte. ℣. \emph{Ps. ibid., 16} Virão em transportes de alegria e de júbilo: e serão conduzidas ao templo do Rei e Senhor.
}\end{paracol}

\paragraphinfo{Evangelho}{Página \pageref{virgensmartires1}}

\paragraph{Ofertório}
\begin{paracol}{2}\latim{
\rlettrine{P}{rudéntes} virgines, apostáte vestras lámpades: Ecce Sponsus venit, exite óbviam Christo Dómino.
}\switchcolumn\portugues{
\slettrine{Ó}{} virgens prudentes, preparai as vossas lâmpadas: eis aí vem o Esposo; ide ao encontro de Cristo Senhor.
}\end{paracol}

\paragraph{Secreta}
\begin{paracol}{2}\latim{
\rlettrine{I}{mmaculátam} hóstiam tibi Dómine offérimus deprecántes: ut beatárum Virginum Sánciæ et Mafáldæ interveniénte suffrágio, semper in nobis dilécti Fílii tui passiónis memória persevéret, et fructus. Per eúmdem Dóminum \emph{\&c.}
}\switchcolumn\portugues{
\rlettrine{S}{enhor,} Vos oferecemos a Hóstia Imaculada, suplicando pela intercessão das Virgens Sancha e Mafalda a graça de gozarmos sempre a memória e o fruto da Paixão do vosso amado Filho. Pelo mesmo nosso Senhor \emph{\&c.}
}\end{paracol}

\paragraphinfo{Comúnio}{Pr. 4, 1}
\begin{paracol}{2}\latim{
\rlettrine{O}{} quam pulchra est casta generátio cum claritáte: immortális est enim memória illíus, quóniam et apud Deum nota est, et apud hómines.
}\switchcolumn\portugues{
\rlettrine{O}{h!} como é formosa a geração casta, quando é revestida com o brilho da virtude! Sua memória é imortal; pois ela é louvada diante de Deus e dos homens.
}\end{paracol}

\paragraph{Postcomúnio}
\begin{paracol}{2}\latim{
\rlettrine{S}{píritum} nobis, Dómine, humilitátis et caritátis tribuat hæc mensa cæléstis: qua reféctæ beátæ Virgines Sáncia et Mafálda, sæculi vanitátibus exútis, ad ætérna regna felíciter pervenérunt. Per Dóminum \emph{\&c.}
}\switchcolumn\portugues{
\rlettrine{S}{enhor,} permiti que recebamos o espírito da humildade e caridade nesta mesa celestial, onde as B. B. Sancha e Mafalda encontraram força para se libertarem das vaidades do mundo e alcançarem a glória eterna. Por nosso Senhor Jesus Cristo \emph{\&c.}
}\end{paracol}

\subsectioninfo{S. Roberto Belarmino, B. C. e Doutor}{13 de Maio}

\textit{Como na Missa In médio Ecclésiae, página \pageref{doutores}, excepto:}

\paragraph{Oração}
\begin{paracol}{2}\latim{
\rlettrine{D}{eus,} qui ad errórem insídias repelléndas et apostólicæ Sedis jura propugnánda, beátum Robértum Pontíficem tuuam atque Doctórem mira eruditióne et virtúte decorásti: ejus méritis et intercessióne concéde; ut nos in veritátis amóre crescámus et errántium corda ad Ecclésiæ tuæ rédeant unitátem. Per Dóminum \emph{\&c.}
}\switchcolumn\portugues{
\slettrine{Ó}{} Deus, que para refutar as insídias dos erros e defender os direitos da Santa Sé Apostólica enriquecestes o B. Roberto, vosso Pontífice e Doutor, com admirável erudição e constância, concedei-nos pelos seus méritos e intercessão que em nós aumente o amor à verdade e regressem à unidade da vossa Igreja os corações dos que permanecem no erro. Por nosso Senhor \emph{\&c.}
}\end{paracol}

\paragraphinfo{Epístola}{Página \pageref{tomasaquino}}

\paragraphinfo{Ofertório}{Sl. 72, 28}
\begin{paracol}{2}\latim{
\rlettrine{M}{ihi} autem adhærére Deo bonum est, pónere in Dómino Deo spem meam: ut annúntiem pmnes prædicatiónes tuas in portis fíliæ Sion, allelúja.
}\switchcolumn\portugues{
\rlettrine{P}{or} isso bom é para mim unir-me a Deus e pôr no Senhor Deus a minha esperança, a fim de publicar todos seus louvores às portas da filha de Sião, aleluia.
}\end{paracol}

\paragraph{Secreta}
\begin{paracol}{2}\latim{
\rlettrine{H}{óstias} tibi, Dómine, in odórem suavitátis offérimus: et præsta; ut, beáti Robérti mónitis et exémplis edócti, per sémitam mandatórum tuórum dilatáto corde currámus. Per Dóminum \emph{\&c.}
}\switchcolumn\portugues{
\rlettrine{A}{} Vós, Senhor, oferecemos estas hóstias em odor de suavidade; e concedei-nos que, edificados com os ensinos e exemplos do B. Roberto, caminhemos generosamente pela via dos vossos preceitos. Por nosso Senhor \emph{\&c.}
}\end{paracol}

\paragraphinfo{Comúnio}{Mt. 5, 14, 16}
\begin{paracol}{2}\latim{
\rlettrine{V}{os} estis lux mundi: sic lúceat lux vestra coram homínibus, ut vídeant ópera vestra bona, et gloríficent Patrem vestrum qui in cœlis est, allelúja.
}\switchcolumn\portugues{
\rlettrine{S}{ois} a luz do mundo. Assim a vossa luz brilhe diante dos homens, para que vejam as vossas boas obras e glorifiquem o vosso Pai, que está nos céus, aleluia.
}\end{paracol}

\paragraph{Postcomúnio}
\begin{paracol}{2}\latim{
\rlettrine{S}{acraménta,} quæ súmpsimus, Dómine Deus noster, in nobis fóveant caritátis ardórem: quo beátus Robértus veheménter accénsus, pro Ecclésia tua se júgiter impendébat. Per Dóminum \emph{\&c.}
}\switchcolumn\portugues{
\qlettrine{Q}{ue} os sacramentos, que recebemos, Senhor, nosso Deus, em nós infundam o ardor da caridade com o qual o B. Roberto, intensamente abrasado, se esforçava incessantemente em defender a vossa Igreja. Por nosso Senhor \emph{\&c.}
}\end{paracol}

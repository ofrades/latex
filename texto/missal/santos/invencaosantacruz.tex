\subsectioninfo{Invenção da Santa Cruz}{3 de Maio}\label{invencaosantacruz}

\paragraphinfo{Intróito}{Gl. 6, 14}
\begin{paracol}{2}\latim{
\rlettrine{N}{os} autem gloriári opórtet in Cruce Dómini nostri Jesu Christi: in quo est salus, vita et resurréctio nostra: per quem salváti et liberáti sumus, allelúja, allelúja. \emph{Ps. 66, 2} Deus misereátur nostri, et benedícat nobis: illúminet vultum suum super nos, et misereátur nostri.
℣. Gloria Patri \emph{\&c.}
}\switchcolumn\portugues{
\rlettrine{C}{onvém} que nos gloriemos na Cruz de nosso Senhor Jesus Cristo, pois Ele é a nossa salvação, vida e ressurreição: e por Ele fomos salvos e livres, aleluia, aleluia. \emph{Sl. 66, 2} Que o Senhor tenha misericórdia de nós e nos abençoe; que o Senhor nos ilumine com o brilho que resplandece da sua face e se compadeça de nós.
℣. Glória ao Pai \emph{\&c.}
}\end{paracol}

\paragraph{Oração}
\begin{paracol}{2}\latim{
\rlettrine{D}{eus,} qui in præclára salutíferæ Crucis Inventióne passiónis tuæ mirácula suscitásti: concéde; ut, vitális ligni prétio, ætérnæ vitæ suffrágia consequámur: Qui vivis et regnas \emph{\&c.}
}\switchcolumn\portugues{
\slettrine{Ó}{} Deus, que na memorável ocasião da Invenção da Cruz, donde nos veio a salvação, repetistes as maravilhas da vossa Paixão, concedei-nos, pela virtude deste Lenho de vida, que alcancemos a graça da vida eterna. Ó Vós, que sendo Deus, viveis e reinais \emph{\&c.}
}\end{paracol}

\paragraphinfo{Epístola}{Fl. 2, 5-11}
\begin{paracol}{2}\latim{
Léctio Epístolæ beáti Pauli Apóstoli ad Philippenses.
}\switchcolumn\portugues{
Lição da Ep.ª do B. Ap.º Paulo aos Filipenses.
}\switchcolumn*\latim{
\rlettrine{F}{ratres:} Hoc enim sentíte in vobis, quod et in Christo Jesu: qui, cum in forma Dei esset, non rápinam arbitrátus est esse se æquálem Deo: sed semetípsum exinanívit formam servi accípiens, in similitudinem hóminum factus, et habitu inventus ut homo. Humiliávit semetípsum, factus obǿdiens usque ad mortem, mortem autem crucis. Propter quod et Deus exaltávit illum: et dojiávit illi nomen, quod est super omne nomen: (hic genuflectitur) ut in nomine Jesu omne genu flectátur cœléstium, terréstrium et infernórum: et omnis lingua confiteátur, quia Dóminus Jesus Christus in glória est Dei Patris.
}\switchcolumn\portugues{
\rlettrine{M}{eus} irmãos: Tende os mesmos sentimentos que animaram Jesus Cristo, o qual, embora fosse Deus por natureza (e não era usurpação o julgar-se igual a Deus), contudo humilhou-se a si próprio, reduzindo-se à condição de servo, tornando-se semelhante aos homens e reconhecido como homem pelas aparências. Humilhou-se a si próprio, obedecendo até à morte, e morte na cruz. Por isso Deus O exaltou e deu-Lhe um nome que é superior a todo o nome (devemos genuflectir) para que ao ser proferido o nome de Jesus todos os joelhos se dobrem nos céus, na terra e até nos infernos; e todas as línguas confessem que nosso Senhor Jesus Cristo está na glória de Deus Pai!
}\end{paracol}

\begin{paracol}{2}\latim{
Allelúja, allelúja. ℣. \emph{Ps. 95, 10} Dícite in géntibus, quia Dóminus regnávit a ligno. Allelúja. ℣. Dulce lignum, dulces clavos, dúlcia ferens póndera: quæ sola fuísti digna sustinére Regem cœlórum et Dóminum. Allelúja.
}\switchcolumn\portugues{
Aleluia, aleluia. ℣. \emph{Sl. 95, 10} Anunciai aos povos que o Senhor reinou pela Cruz. Aleluia. ℣. Ó doce Lenho, ó doces cravos que suportais o peso dum fardo tão doce! Só vós, ó Cruz, fostes digna de ter suspenso o Senhor, Rei dos céus. Aleluia.
}\end{paracol}

\paragraphinfo{Evangelho}{Jo. 3, 1-15}
\begin{paracol}{2}\latim{
\cruz Sequéntia sancti Evangélii secúndum Joánnem.
}\switchcolumn\portugues{
\cruz Continuação do santo Evangelho segundo S. João.
}\switchcolumn*\latim{
\blettrine{I}{n} illo témpore: Erat homo ex pharisǽis, Nicodémus nómine, princeps Judæórum. Hic venit ad Jesum nocte et dixit ei: Rabbi, scimus, quia a Deo venísti magíster; nemo enim potest hæc signa fácere, quæ tu facis, nisi fúerit Deus cum eo. Respóndit Jesus et dixit ei: Amen, amen, dico tibi, nisi quis renátus fúerit denuo, non potest vidére regnum Dei. Dicit ad eum Nicodémus: Quómodo potest homo nasci, cum sit senex? numquid potest in ventrem matris suæ iteráto introíre et renásci? Respóndit Jesus: Amen, amen, dico tibi, nisi quis renátus fúerit ex aqua et Spíritu Sancto, non potest introíre in regnum Dei. Quod natum est ex carne, caro est: et quod natum est ex spíritu, spíritus est. Non miréris, quia dixi tibi: opórtet vos nasci dénuo. Spíritus, ubi vult, spirat, et vocem ejus audis, sed nescis, unde véniat aut quo vadat: sic est omnis, qui natus est ex spíritu. Respondit Nicodémus et dixit ei: Quómodo possunt hæc fíeri? Respóndit Jesus et dixit ei: Tu es magíster in Israël, et hæc ignóras? Amen, amen, dico tibi, quia, quod scimus, lóquimur, et quod vídimus, testámur, et testimónium nostrum non accípitis. Si terréna dixi vobis et non creditis: quómodo, si díxero vobis cœléstia, credétis? Et nemo ascéndit in cœlum, nisi qui descéndit de cœlo, Fílius hóminis, qui est in cœlo. Et sicut Moyses exaltávit serpéntem in desérto: ita exaltári opórtet Fílium hóminis, ut omnis, qui credit in ipsum, non péreat, sed hábeat vitam ætérnam.
}\switchcolumn\portugues{
\blettrine{N}{aquele} tempo, havia entre os judeus um homem, chamado Nicodemos, que era um dos príncipes dos judeus. Em uma noite veio ele encontrar Jesus, dizendo-Lhe: «Mestre, sabemos que viestes da parte de Deus, como um Doutor, pois ninguém pode fazer os milagres, que fazeis, se Deus não estiver com ele». Jesus disse-lhe: «Em verdade, em verdade vos digo: Ninguém poderá ver o reino dos céus, se não nascer novamente». Nicodemos perguntou a Jesus: «Como pode um homem nascer, sendo já velho? Pode ele, porventura, entrar segunda vez no seio de sua mãe e renascer?». Jesus respondeu: «Em verdade, em verdade vos digo: Se alguém não renascer na água e no Espírito Santo, não pode entrar no reino de Deus. O que nasceu da carne, é carne; e o que nasceu do espírito, é espírito. Não vos admireis, pois, de que vos houvesse dito: é necessário que nasçais outra vez. O vento sopra onde lhe parece, e ouvis o seu som; mas não sabeis donde vem, nem para onde vai. O mesmo acontece com qualquer que nasce do Espírito». E Nicodemos respondeu: «Como poderá isso acontecer?». Jesus retorquiu: «Sois doutor em Israel e ignorais estas cousas? Em verdade, em verdade vos digo: Dizemos aquilo que sabemos e damos testemunho daquilo que vimos; mas não aceitais o nosso testemunho. Se me não acreditais, quando falo das cousas terrenas, como me acreditareis, falando-vos das cousas do céu? Ninguém subiu ao céu, senão O que desceu do céu: o Filho do homem, que está no céu. Assim como Moisés levantou a serpente no deserto, assim também é necessário que o Filho do homem seja levantado, a fim de que quem acreditar n’Ele não pereça, mas alcance a vida eterna».
}\end{paracol}

\paragraphinfo{Ofertório}{Sl. 117, 16 \& 17}
\begin{paracol}{2}\latim{
\rlettrine{D}{éxtera} Dómini fecit virtútem, déxtera Dómini exaltávit me: non móriar, sed vivam et narrábo ópera Dómini, allelúja.
}\switchcolumn\portugues{
\rlettrine{A}{} dextra do Senhor mostrou o seu poder: a dextra do Senhor exaltou-me. Não morrerei, mas viverei e publicarei as obras do Senhor. Aleluia.
}\end{paracol}

\paragraph{Secreta}
\begin{paracol}{2}\latim{
\rlettrine{S}{acrifícium,} Dómine, quod tibi immolámus, placátus inténde: ut ab omni nos éruat bellórum nequítia, et per vexíllum sanctæ Crucis Fílii tui, ad conteréndas potestátis advérsa? insídias, nos in tuæ protectiónis securitáte constítuat. Per eúndem Dóminum nostrum \emph{\&c.}
}\switchcolumn\portugues{
\rlettrine{A}{placai-Vos,} Senhor, e dignai-Vos olhar para esta vítima, que imolamos em vossa honra, a fim de que por ela sejamos preservados de todos os males das guerras e pelo estandarte da Santa Cruz do vosso Filho vençamos as insídias dos nossos poderosos inimigos, ficando em segurança sob a vossa protecção. Pelo mesmo nosso Senhor \emph{\&c.}
}\end{paracol}

\paragraph{Comúnio}
\begin{paracol}{2}\latim{
\rlettrine{P}{er} signum Crucis de inimícis nostris líbera nos, Deus noster, allelúja.
}\switchcolumn\portugues{
\rlettrine{P}{elo} sinal da Santa Cruz, livrai-nos, ó nosso Deus, de nossos inimigos. Aleluia.
}\end{paracol}

\paragraph{Postcomúnio}
\begin{paracol}{2}\latim{
\rlettrine{R}{epléti} alimónia cœlésti et spiritáli poculo recreáti, quǽsumus, omnípotens Deus: ut ab hoste malígno deféndas, quos per lignum sanctæ Crucis Fílii tui, arma justítiæ pro salúte mundi, triumpháre jussísti. Per eúndem Dóminum \emph{\&c.}
}\switchcolumn\portugues{
\rlettrine{S}{aciados} com o alimento celestial e confortados com a bebida espiritual, Vos suplicamos, ó Deus omnipotente, defendei da malícia do inimigo aqueles que quisestes que triunfassem pelo sagrado Lenho da Cruz de vosso Filho, que é a arma de justiça que salvou o mundo. Pelo mesmo nosso Senhor \emph{\&c.}
}\end{paracol}

\subsectioninfo{S. Policarpo, B. e Márt.}{26 de Janeiro}

\textit{Como Missa Sacerdótes tui, página \pageref{confessorespontifices2}, excepto:}

\paragraphinfo{Epístola}{1. Jo. 3, 10-16}
\begin{paracol}{2}\latim{
Léctio Epístolæ beáti Joánnis Apóstoli.
}\switchcolumn\portugues{
Lição da Ep.ª do B. Ap.º S. João.
}\switchcolumn*\latim{
\rlettrine{C}{aríssimi:} Omnis qui non est justus, non est ex Deo, et qui non díligit fratrem suum: quóniam hæc est annuntiátio, quam audístis ab inítio, ut diligátis altérutrum. Non sicut Cain, qui ex malígno erat, et occídit fratrem suum. Et propter quid occídit eum? Quóniam ópera ejus malígna erant: fratris autem ejus justa. Nolíte mirári fratres, si odit vos mundus. Nos scimus quóniam transláti sumus de morte ad vitam, quóniam dilígimus fratres. Qui non díligit, manet in morte: omnis qui odit fratrem suum, homicída est. Et scitis, quóniam omnis homicída non habet vitam ætérnam in semetípso manéntem. In hoc cognóvimus caritátem Dei, quóniam ille ánimam suam pro nobis pósuit: et nos debémus pro frátribus ánimas pónere.
}\switchcolumn\portugues{
\rlettrine{C}{aríssimos:} Aquele que não é justo e não ama seu irmão não é de Deus. Porque esta é a recomendação que ouvistes desde o princípio: «Que vos ameis uns aos outros». Não seja, porém, como Caim, que era maligno o qual matou o seu irmão. E porque o matou ele? Porque as suas obras eram más, enquanto que as do seu irmão eram justas. Não vos admireis, irmãos, se o mundo vos odeia; porquanto sabemos que passamos da morte à vida, porque amamos os nossos irmãos. Aquele que não ama permanece na morte. Aquele que odeia o seu irmão é um homicida. E vós sabeis que o homicida não tem a vida eterna, permanecendo em si. Nisto conhecemos o amor de Deus: porquanto Ele deu sua vida por nós; e nós devemos também dar a nossa vida por nossos irmãos.
}\end{paracol}

\paragraphinfo{Evangelho}{Página \pageref{martirnaopontifice2}}

\paragraph{Postcomúnio}
\begin{paracol}{2}\latim{
\rlettrine{R}{efécti} participatióne múneris sacri, quǽsumus, Dómine, Deus noster: ut, cujus exséquimur cultum, intercedénte beáto Polycárpo Mártyre tuo atque Pontífice, sentiámus efféctum. Per Dóminum nostrum \emph{\&c.}
}\switchcolumn\portugues{
\rlettrine{F}{ortalecidos} com a participação deste dom sacratíssimo, Vos suplicamos, Senhor, nosso Deus, que, por intercessão do B. Policarpo, vosso Mártir e Pontífice, sintamos o efeito do mystério que celebrámos. Por nosso Senhor \emph{\&c.}
}\end{paracol}

\subsectioninfo{S. António Maria Zacarias, Conf.}{5 de Julho}

\paragraphinfo{Intróito}{1. Cor. 2, 4}
\begin{paracol}{2}\latim{
\rlettrine{S}{ermo} meus et prædicátio mea non in persuasibílibus humánæ sapiéntiæ verbis, sed in ostensióne spíritus et virtútis. \emph{Ps. 110, 1} Confitébor tibi, Dómine, in toto corde meo, in consílio justórum et congregatióne.
℣. Gloria Patri \emph{\&c.}
}\switchcolumn\portugues{
\rlettrine{A}{} minha conversação e a minha pregação não são apoiadas nos argumentos persuasivos da sabedoria humana, mas na manifestação do espírito e do poder de Deus. \emph{Sl. 110, 1} Senhor, eu vos louvarei de todo meu coração no conselho dos justos e na assembleia do povo!
℣. Glória ao Pai \emph{\&c.}
}\end{paracol}

\paragraph{Oração}
\begin{paracol}{2}\latim{
\rlettrine{F}{ac} nos, Dómine Deus, supereminéntem Jesu Christi sciéntiam, spíritu Pauli Apóstoli, edíscere: qua beátus Antónius María mirabíliter erudítus, novas in Ecclésia tua clericórum et vírginum famílias congregávit. Per eúndem Dóminum \emph{\&c.}
}\switchcolumn\portugues{
\slettrine{Ó}{} Deus e Senhor, fazei-nos adquirir a eminente ciência de Jesus Cristo, segundo o espírito do Apóstolo Paulo, em cuja ciência o B. António Maria foi admiravelmente instruído, e lhe fez estabelecer na vossa Igreja novas congregações de clérigos e de virgens. Pelo mesmo nosso Senhor \emph{\&c.}
}\end{paracol}

\paragraphinfo{Epístola}{1. Tm. 4. 8-16}
\begin{paracol}{2}\latim{
Léctio Epístolæ beáti Pauli Apóstoli ad Timótheum.
}\switchcolumn\portugues{
Lição da Ep.ª do B. Ap.º Paulo a Timóteo.
}\switchcolumn*\latim{
\rlettrine{C}{aríssime:} Píetas ad ómnia utilis est: promissiónem habens vitæ, quæ nunc est, et futúræ. Fidélis sermo et omni acceptióne dignus. In hoc enim laborámus et maledícimur, quia sperámus in Deum vivum, qui est Salvátor ómnium hóminum, maxime fidélium. Prǽcipe hæc et doce. Nemo adolescentiam tuam contémnat: sed exémplum esto fidélium in verbo, in conversatióne, in caritáte, in fide, in castitáte. Dum vénio, atténde lectióni, exhortatióni et doctrínæ. Noli neglégere grátiam, quæ in te est, quæ data est tibi per prophétiam, cum impositióne mánuum presbytérii. Hæc meditáre, in his esto: ut proféctus tuus maniféstus sit ómnibus. Attende tibi et doctrínæ: insta in illis. Hoc enim fáciens, et teípsum salvum fácies, et eos qui te áudiunt.
}\switchcolumn\portugues{
\rlettrine{C}{aríssimo:} A piedade é útil para tudo, tendo a promessa certa e digna de toda a aceitação. Nós, pois, suportamos tantos trabalhos e ultrajes porque temos esperança em Deus vivo, que é o Salvador de todos os homens, principalmente dos fiéis. Recomenda estas cousas e ensina-as. Que ninguém menospreze a tua juventude; e que sirvas de exemplo aos fiéis nas palavras, na conduta, na caridade, na fé e na castidade. Enquanto não vou, aplica-te na leitura, na exortação e no ensino. Não desprezes a graça que há em ti, a qual te foi dada, segundo uma revelação profética, pela imposição das mãos, na assembleia dos presbíteros. Medita nestas cousas e ocupa-te nelas inteiramente, a fim de que os teus progressos sejam evidentes a todos. Vigia-te a ti mesmo e à tua doutrina; aplica-te constantemente nisso, porque, procedendo assim, serás salvo, assim como aqueles que te ouvem.
}\end{paracol}

\paragraphinfo{Gradual}{Fl. 1. 8-9}
\begin{paracol}{2}\latim{
\rlettrine{T}{estis} mihi est Deus, quo modo cúpiam omnes vos in viscéribus Jesu Christi. Et hoc oro, ut cáritas vestra magis ac magis abúndet in sciéntia et in omni sensu. ℣. \emph{ibid., 10} Ut probétis potióra, ut sitis sincéri et sine offénsa in diem Christi.
}\switchcolumn\portugues{
\rlettrine{D}{eus} é testemunha do modo como vos amo a todos nas entranhas de Jesus Cristo. E o que vos peço é que a vossa caridade aumente cada vez mais na vossa inteligência e em todos vossos sentidos. ℣. \emph{ibid., 10} A fim de que possais distinguir o que é melhor e estejais puros e irrepreensíveis no dia de Cristo.
}\switchcolumn*\latim{
Allelúja, allelúja. ℣. \emph{ibid., 11} Repléti fructu justítiæ per Jesum Christum, in glóriam et laudem Dei. Allelúja.
}\switchcolumn\portugues{
Aleluia, aleluia. ℣. \emph{ibid., 11} Que sejais cheios dos frutos da justiça por Jesus Cristo, para glória e louvor de Deus. Aleluia
}\end{paracol}

\paragraphinfo{Evangelho}{Mc. 10, 15-21}
\begin{paracol}{2}\latim{
\cruz Sequéntia sancti Evangélii secúndum Marcum. 
}\switchcolumn\portugues{
\cruz Continuação do santo Evangelho segundo S. Marcos.
}\switchcolumn*\latim{
\blettrine{I}{n} illo témpore: Dixit Jesus discípulis suis: Quisquis non recéperit regnum Dei velut párvulus, non intrábit in illud. Et compléxans párvulos et impónens manus super illos, benedicébat eos. Et cum egréssus esset in viam, procúrrens quidam, genu flexo ante eum, rogábat eum: Magíster bone, quid fáciam, ut vitam ætérnam percípiam? Jesus autem dixit ei: Quid me dicis bonum? Nemo bonus, nisi unus Deus. Præcépta nosti: Ne adúlteres, ne occídas, ne furéris, ne falsum testimónium díxeris, ne fraudem féceris, honora patrem tuum et matrem. At ille respóndens, ait illi: Magíster, hæc ómnia observávi a juventúte mea. Jesus autem intúitus eum, diléxit eum et dixit ei: Unum tibi deest: vade, quæcúmque habes, vende et da paupéribus, et habébis thesáurum in cælo: et veni, séquere me.
}\switchcolumn\portugues{
\blettrine{N}{aquele} tempo, disse Jesus aos seus discípulos: «Todo aquele que não receber o reino de Deus, como um menino, não entrará nele». E, abraçando os meninos e pondo as mãos sobre eles, abençoava-os. Então, havendo saído, para começar a sua jornada, correu logo um certo jovem ao seu encontro e, ajoelhando diante d’Ele, perguntou-lhe: «Bom Mestre, que deverei fazer para alcançar a vida eterna?». Jesus disse-lhe: «Porque me chamais bom? Ninguém é bom senão Deus. Tu conheces os mandamentos: não cometas adultério; não mates; não furtes; não digas falso testemunho; não cometas fraudes; honra teu pai e tua mãe?». Então, respondendo ele, disse-lhe: «Tudo isso tenho observado desde a minha juventude». E Jesus fitou-o, mostrou-lhe amizade e disse-lhe: «Uma cousa te falta: vai, vende tudo quanto tens, dá-o aos pobres, e terás um tesouro no céu; depois vem e segue-me».
}\end{paracol}

\paragraphinfo{Ofertório}{Sl. 137, 1-2}
\begin{paracol}{2}\latim{
\rlettrine{I}{n} conspéctu Angelórum psallam tibi: adorábo ad templum sanctum tuum, et confitébor nómini tuo. 
}\switchcolumn\portugues{
\rlettrine{C}{antarei} os vossos louvores na presença dos Anjos; ajoelharei no vosso sagrado templo e louvarei o vosso nome.
}\end{paracol}

\paragraph{Secreta}
\begin{paracol}{2}\latim{
\rlettrine{A}{d} mensam cœléstis convívii fac nos, Dómine, eam mentis et córporis puritátem afférre, qua beátus Antónius María, hanc sacratíssimam hóstiam ófferens, mirífice ornátus enítuit. Per Dóminum \emph{\&c.}
}\switchcolumn\portugues{
\rlettrine{P}{ermiti,} Senhor, que me acompanhe à mesa do celestial banquete aquela pureza de alma e de corpo que ornava de um modo tão brilhante e maravilhoso o B. António Maria, quando oferecia esta hóstia. Por nosso Senhor \emph{\&c.}
}\end{paracol}

\paragraphinfo{Comúnio}{Fl. 8, 17}
\begin{paracol}{2}\latim{
\rlettrine{I}{mitatóres} mei estóte, fratres, et observáte eos, qui ita ámbulant, sicut habétis formam nostram.
}\switchcolumn\portugues{
\rlettrine{M}{eus} irmãos, sede meus imitadores e olhai para aqueles que procedem segundo o exemplo que tendes em nós.
}\end{paracol}

\paragraph{Postcomúnio}
\begin{paracol}{2}\latim{
\rlettrine{C}{œlésti} dape, qua pasti sumus, Dómine Jesu Christe, eo corda nostra caritátis igne flamméscant: quo beátus Antónius María salutáris hóstiæ vexíllum, contra Ecclésiæ tuæ hostes, éxtulit ad victóriam: Qui vivis \emph{\&c.}
}\switchcolumn\portugues{
\rlettrine{S}{enhor} Jesus Cristo, fazei que, pela virtude do celestial festim, de que nos saciastes, os nossos corações se inflamem naquele fogo da Caridade que deu ao B. António Maria a coragem de levar contra os inimigos da vossa Igreja o estandarte da hóstia da salvação, que o conduziu à vitória. Ó Vós, que viveis e reinais \emph{\&c.}
}\end{paracol}
\subsectioninfo{S. Gregório}{12 de Março}

\textit{Como na Missa Si díligis me, página \pageref{sumospontifices}, excepto:}

\paragraph{Oração}
\begin{paracol}{2}\latim{
\rlettrine{D}{eus,} qui ánimæ fámuli tui Gregórii ætérnæ beatitúdinis prǽmia contulísti: concéde propítius; ut, qui peccatórum nostrórum póndere prémimur, ejus apud te précibus sublevémur. Per Dóminum \emph{\&c.}
}\switchcolumn\portugues{
\slettrine{Ó}{} Deus, que concedestes à alma do vosso servo Gregório a recompensa da bem-aventurança eterna, permiti benigno que pelos seus rogos junto de Vós sejamos aliviados do peso dos nossos pecados, que tanto nos oprimem. Por nosso Senhor \emph{\&c.}
}\end{paracol}

\paragraph{Secreta}
\begin{paracol}{2}\latim{
\rlettrine{A}{nnue} nobis, quǽsumus, Dómine: ut intercessióne beáti Gregórii hæc nobis prosit oblátio, quam immolándo totíus mundi tribuísti relaxári delícta. Per Dóminum \emph{\&c.}
}\switchcolumn\portugues{
\rlettrine{S}{enhor,} concedei-nos, Vos imploramos, que por intercessão do B. Gregório nos seja Proveitosa esta oblação, em virtude de cuja imolação nos alcançastes o perdão dos pecados do mundo inteiro. Por nosso Senhor \emph{\&c.}
}\end{paracol}

\paragraph{Postcomúnio}
\begin{paracol}{2}\latim{
\rlettrine{D}{eus,} qui beátum Gregórium Pontíficem Sanctórum tuórum méritis coæquásti: concéde propítius; ut, qui commemoratiónis ejus festa percólimus, vitæ quoque imitémur exémpla. Per Dóminum \emph{\&c.}
}\switchcolumn\portugues{
\slettrine{Ó}{} Deus, que igualastes o B. Gregório aos merecimentos dos vossos Santos, concedei-nos benigno que, celebrando solenemente a sua festa, possamos também imitar os exemplos da sua vida. Por nosso Senhor \emph{\&c.}
}\end{paracol}
\subsectioninfo{Aparição de Arcanjo S. Miguel}{8 de Maio}\label{aparicaoarcanjomiguel}

\paragraphinfo{Intróito}{Sl. 102, 20}
\begin{paracol}{2}\latim{
\rlettrine{B}{enedícite} Dóminum, omnes Angeli ejus: poténtes virtúte, qui fácitis verbum ejus, ad audiéndam vocem sermónum ejus. \emph{Ps. ibid., 1} Benedic, ánima mea. Dómino: et ómnia, quæ intra me sunt, nómini sancto ejus.
℣. Gloria Patri \emph{\&c.}
}\switchcolumn\portugues{
\rlettrine{B}{endizei} o Senhor, ó vós, todos os Anjos do Senhor: Sois cheios de poder e virtude: e fazeis o que Ele vos manda, obedecendo às suas palavras e ordens. \emph{Sl. ibid., 1} Bendiz o Senhor, ó minha alma: Que tudo quanto houver em mim bendiga o santo Nome do Senhor.
℣. Glória ao Pai \emph{\&c.}
}\end{paracol}

\paragraph{Oração}
\begin{paracol}{2}\latim{
\rlettrine{D}{eus,} qui, miro órdine, Angelórum ministéria hominúmque dispénsas: concéde propítius; ut, a quibus tibi ministrántibus in cœlo semper assístitur, ab his in terra vita nostra muniátur. Per Dóminum \emph{\&c.}
}\switchcolumn\portugues{
\slettrine{Ó}{} Deus, que com sublime harmonia dispusestes os ministérios dos Anjos e dos homens, concedei-nos propício que sejamos protegidos durante a vida na terra por aqueles que, servindo-Vos no céu, gozam sempre a vossa companhia. Por nosso Senhor \emph{\&c.}
}\end{paracol}

\paragraphinfo{Epístola}{Apoc. 1, 1-5}
\begin{paracol}{2}\latim{
Léctio libri Apocalýpsis beáti Joánnis Apóstoli.
}\switchcolumn\portugues{
Lição do Livro do Apocalipse do B. Ap.º João.
}\switchcolumn*\latim{
\rlettrine{I}{n} diébus illis: Significávit Deus, quæ opórtet fíeri cito, mittens per Angelum suum servo suo Joánni, qui testimónium perhíbuit verbo Dei, et testimónium Jesu Christi, quæcúmque vidit. Beátus, qui legit et audit verba prophetíæ hujus: et servat ea, quæ in ea scripta sunt: tempus enim prope est. Joánnes septem ecclésiis, quæ sunt in Asia. Grátia vobis et pax ab eo, qui est et qui erat et qui ventúrus est: et a septem spirítibus, qui in conspéctu throni ejus sunt: et a Jesu Christo, qui est testis fidélis, primogénitus mortuórum et princeps regum terræ, qui diléxit nos et lavit nos a peccátis nostris in sánguine suo.
}\switchcolumn\portugues{
\rlettrine{N}{aqueles} dias, manifestou Deus o que logo deveria acontecer, enviando o seu Anjo ao seu servo João, o qual deu testemunho pela palavra de Deus e deu testemunho de tudo o que viu em Jesus Cristo. Bem-aventurado o que lê e ouve as palavras desta Profecia e observa o que nela se contém, pois o tempo está próximo. João diz às sete igrejas, que estão na Ásia: «Que a graça e a paz vos sejam dadas por Aquele que é, que era e que há-de vir, pelos sete espíritos, que estão diante do seu trono, e por Jesus Cristo, que é a testemunha fiel, o primogénito dos mortos e o príncipe dos reis da terra, O qual nos amou e nos lavou das manchas dos nossos pecados no seu sangue».
}\end{paracol}

\begin{paracol}{2}\latim{
Allelúja, allelúja. ℣. Sancte Míchael Archángele, defénde nos in prǿlio: ut non pereámus in treméndo judício. Allelúja. ℣. Concússum est mare et contrémuit terra, ubi Archángelus Míchaël descéndit de cœlo. Allelúja.
}\switchcolumn\portugues{
Aleluia, aleluia. ℣. S. Miguel Arcanjo, defendei-nos neste combate, a fim de que não pereçamos no dia do juízo tremendo. Aleluia. ℣. O mar ficou agitado e a terra tremeu, quando o Arcanjo S. Miguel desceu do céu. Aleluia.
}\end{paracol}

\paragraphinfo{Evangelho}{Mt. 18, 1-10}
\begin{paracol}{2}\latim{
\cruz Sequéntia sancti Evangélii secúndum Matthǽum.
}\switchcolumn\portugues{
\cruz Continuação do santo Evangelho segundo S. Mateus.
}\switchcolumn*\latim{
\blettrine{I}{n} illo témpore: Accessérunt discípuli ad Jesum, dicéntes: Quis, putas, major est in regno cœlórum? Et ádvocans Jesus parvulum, statuit eum in médio eórum et dixit: Amen, dico vobis, nisi convérsi fuéritis et efficiámini sicut párvuli, non intrábitis in regnum cælorum. Quicúmque ergo humiliáverit se sicut párvulus iste, hic est major in regno cœlórum. Et qui suscéperit unum párvulum talem in nómine meo, me súscipit. Qui autem scandalizáverit unum de pusíllis istis, qui in me credunt, expédit ei, ut suspendátur mola asinária in collo ejus, et demergátur in profúndum maris. Væ mundo a scándalis! Necésse est enim, ut véniant scándala: verúmtamen væ hómini illi, per quem scándalum venit! Si autem manus tua vel pes tuus scandalízat te, abscíde eum et prójice abs te: bonum tibi est ad vitam íngredi débilem vel cláudum, quam duas manus vel duos pedes habéntem mitti in ignem ætérnum. Et si óculus tuus scandalízat te, érue eum et prójice abs te: bonum tibi est cum uno óculo in vitam intráre, quam duos óculos habéntem mitti in gehénnam ignis. Vidéte, ne contemnátis unum ex his pusíllis: dico enim vobis, quia Angeli eórum in cœlis semper vident fáciem Patris mei, qui in cœlis est.
}\switchcolumn\portugues{
\blettrine{N}{aquele} tempo, os discípulos aproximaram-se de Jesus, dizendo-Lhe: «Quem, pois, pensais Vós que é o maior no reino dos céus?». Jesus, havendo chamado um pequeno, colocou-o no meio deles e disse: «Em verdade vos digo: Se vos não converteis e não vos tornais como os pequenos, não entrareis no reino dos céus; todo aquele que se fizer pequeno, como este menino, esse, é o maior no reino dos céus; e quem receber um destes pequenos em meu nome, recebe-me a mim mesmo; e quem escandalizar um destes pequenos, que crêem em mim, melhor será que se lhe pendure ao pescoço uma mó de moinho e seja lançado ao fundo do mar! Ai do mundo por causa dos escândalos! Contudo é necessário que aconteçam os escândalos; mas ai do homem por quem vier o escândalo. Se a vossa mão ou o vosso pé são ocasião de escândalo, cortai-os e lançai-os para longe; pois melhor é entrar na vida manco ou coxo do que ser lançado no fogo eterno, tendo as duas mãos ou os dous pés. Se o vosso olho é ocasião de escândalo, arrancai-o e lançai-o para longe; pois é melhor entrar na vida só com um olho do que ser lançado no fogo eterno, tendo os dous olhos. Vede que não seja desprezado nenhum destes pequeninos; pois, digo-vos: os seus Anjos nos céus estão sempre na presença de meu Pai, que está nos céus».
}\end{paracol}

\paragraphinfo{Ofertório}{Ap. 8, 3 \& 4}
\begin{paracol}{2}\latim{
\rlettrine{S}{tetit} Angelus juxta aram templi, habens thuríbulum áureum in manu sua, et data sunt ei incénsa multa: et ascéndit fumus aromátum in conspéctu Dei, allelúja. Secreta
}\switchcolumn\portugues{
\qlettrine{J}{unto} ao altar, no templo, estava de pé um Anjo, tendo na mão um turíbulo de ouro: e deitaram-lhe muito incenso, subindo o fumo dos perfumes à presença de Deus. Aleluia.
}\end{paracol}

\paragraph{Secreta}
\begin{paracol}{2}\latim{
\rlettrine{H}{óstias} tibi, Dómine, laudis offérimus, supplíciter deprecántes: ut easdem, angélico pro nobis interveniénte suffrágio, et placátus accípias, et ad salútem nostram proveníre concédas. Per Dóminum \emph{\&c.}
}\switchcolumn\portugues{
\rlettrine{V}{os} oferecemos estas hóstias de louvor, Senhor, implorando-Vos humildemente que as aceiteis com indulgência pela intercessão do vosso Santo Anjo, e que elas sejam úteis à nossa salvação. Por nosso Senhor \emph{\&c.}
}\end{paracol}

\paragraphinfo{Comúnio}{Dn. 3, 58}
\begin{paracol}{2}\latim{
\rlettrine{B}{enedícite,} omnes Angeli Dómini, Dóminum: hymnum dícite et superexaltáte eum in sǽcula.
}\switchcolumn\portugues{
\rlettrine{A}{njos} do Senhor, bendizei todos o Senhor: cantai hinos em seu louvor e exaltai-O em todos os séculos.
}\end{paracol}

\paragraph{Postcomúnio}
\begin{paracol}{2}\latim{
\rlettrine{B}{eáti} Archángeli tui Michaelis intercessióne suffúlti: súpplices te, Dómine, deprecámur; ut, quod ore prosequimur, contingamus et mente. Per Dóminum \emph{\&c.}
}\switchcolumn\portugues{
\rlettrine{C}{onfiado} na intercessão do vosso B. Arcanjo Miguel, Senhor, Vos oferecemos as nossas humildes súplicas, para que a nossa alma alcance o que a nossa boca pede. Por nosso Senhor \emph{\&c.}
}\end{paracol}

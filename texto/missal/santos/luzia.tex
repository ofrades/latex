\subsectioninfo{Santa Luzia, Virgem e Mártir}{13 de Dezembro}

\textit{Como na Missa Dilexísti justitiam, página \pageref{virgemnaomartir1}, excepto:}

\paragraphinfo{Gradual}{Sl. 44, 8}
\begin{paracol}{2}\latim{
\rlettrine{D}{ilexísti} justítiam, et odísti iniquitátem. ℣. Proptérea unxit te Deus, Deus tuus, óleo lætítiæ.
}\switchcolumn\portugues{
\rlettrine{A}{mastes} a justiça e odiastes a iniquidade. ℣. Por isso o Senhor, vosso Deus, ungiu-vos com o óleo da alegria, de preferência às vossas companheiras.
}\switchcolumn*\latim{
Allelúja, allelúja. ℣. \emph{ibid., 3} Diffúsa est grátia in lábiis tuis: proptérea benedíxit te Deus in ætérnum. Allelúja.
}\switchcolumn\portugues{
Aleluia, aleluia. ℣. \emph{ibid., 3} A graça espalhou-se nos vossos lábios; por isso Deus vos abençoou por todos os séculos. Aleluia.
}\end{paracol}

\paragraphinfo{Evangelho}{Página \pageref{virgensmartires2}}

\paragraphinfo{Ofertório}{Sl. 44, 15-16}
\begin{paracol}{2}\latim{
\rlettrine{A}{fferéntur} Regi Vírgines post eam: próximæ ejus afferéntur tibi in lætítia et exsultatióne: adducéntur in templum Regi Dómino.
}\switchcolumn\portugues{
\rlettrine{A}{pós} ela serão apresentadas virgens ao Rei; as suas companheiras serão introduzidas no meio da alegria e júbilo: serão conduzidas ao Senhor no templo do Rei.
}\end{paracol}

\paragraphinfo{Comúnio}{Sl. 118, 161-162}
\begin{paracol}{2}\latim{
\rlettrine{P}{ríncipes} persecúti sunt me gratis, et a verbis tuis formidávit cor meum: lætábor ego super elóquia tua, quasi qui invénit spólia multa.
}\switchcolumn\portugues{
\rlettrine{O}{s} príncipes perseguiram-me injustamente, mas o meu coração não temeu senão as vossas palavras. Regozijar-me-ei com vossas palavras, como se um homem houvera achado ricos despojos.
}\end{paracol}

\subsectioninfo{S. Agostinho de Cantorbéry, B. e C.}{28 de Maio}

\textit{Como Missa Sacerdótes tui, página \pageref{confessorespontifices2}, excepto:}

\paragraph{Oração}
\begin{paracol}{2}\latim{
\rlettrine{D}{eus,} qui Anglórum gentes, prædicatióne et miráculis beáti Augustíni Confessóris tui atque Pontíficis, veræ fídei luce illustráre dignátus es: concéde; ut, ipso interveniénte, errántium corda ad veritátis tuæ rédeant unitátem, et nos in tua simus voluntáte concórdes. Per Dóminum nostrum \emph{\&c.}
}\switchcolumn\portugues{
\slettrine{Ó}{} Deus, que pela pregação e milagres do B. Agostinho, vosso Confessor e Pontífice, Vos dignastes ilustrar com a luz da verdadeira fé a nação inglesa, concedei-nos por sua intercessão que os corações dos que andam transviados regressem à unidade da vossa fé, e sejamos concordes com vossa vontade. Por nosso Senhor \emph{\&c.}
}\end{paracol}

\paragraphinfo{Epístola}{1. Ts. 2, 2-9}
\begin{paracol}{2}\latim{
Léctio Epístolæ beáti Pauli Apóstoli ad Thessalonicénses.
}\switchcolumn\portugues{
Lição da Ep.ª do B. Ap.º Paulo aos Tessalonicenses.
}\switchcolumn*\latim{
\rlettrine{F}{ratres:} Fidúciam habúimus in Deo nostro loqui ad vos Evangélium Dei in multa sollicitúdine. Exhortátio enim nostra non de erróre neque de immundítia neque in dolo, sed sicut probáti sumus a Deo, ut crederétur nobis Evangélium: ita lóquimur, non quasi homínibus placéntes, sed Deo, qui probat corda nostra. Neque enim aliquándo fuimus in sermóne adulatiónis, sicut scitis: neque in occasióne avarítiæ: Deus testis est: nec quæréntes ab homínibus glóriam, neque a vobis neque ab áliis; cum possémus vobis óneri esse ut Christi Apóstoli: sed facti sumus párvuli in médio vestrum, tamquam si nutrix fóveat fílios suos. Ita desiderántes vos, cúpide volebámus trádere vobis non solum Evangélium Dei, sed étiam ánimas nostras: quóniam caríssimi nobis facti estis. Mémores enim estis, fratres, labóris nostri et fatigatiónis: nocte ac die operántes, ne quem vestrum gravarémus, prædicávimus in vobis Evangélium Dei.
}\switchcolumn\portugues{
\rlettrine{M}{eus} irmãos: Tivemos confiança em o nosso Deus de vos pregar o Evangelho com muita solicitude, pois a nossa pregação não foi baseada nem no erro, nem em nenhuma intenção viciosa, nem na impostura, mas porque Deus nos julgou dignos de nos confiar o Evangelho. E, assim, falámos, não para agradar aos homens, mas a Deus, que perscruta os nossos corações. Com efeito, nunca a nossa linguagem (vós o sabeis e Deus é testemunha) serviu para adular, foi inspirada pela avareza, procurou a glória humana, ou a vossa ou a de outrem. Podíamos, como Apóstolos de Cristo, impor-nos a vós, mas tornamo-nos pequenos, como uma mãe que acaricia seus filhos. Assim, no nosso afecto por vós, desejamos ardentemente dar-vos, não somente o Evangelho de Deus, mas até a nossa própria vida, pois tanto vos tornastes caríssimos ao nosso coração! Decerto vos recordais, meus irmãos, do nosso trabalho e fadiga, porquanto trabalhámos dia e noite, a fim de não sermos onerosos a nenhum de vós, a quem pregamos o Evangelho de Deus.
}\end{paracol}

\paragraphinfo{Evangelho}{Página \pageref{tito}}

\paragraph{Secreta}
\begin{paracol}{2}\latim{
\rlettrine{S}{acrifícium} tibi offérimus. Dómine, in sollemnitáte beáti Augustíni Pontíficis et Confessóris tui, humíliter deprecántes: ut oves, quæ periérunt, ad unum ovile revérsæ, hoc salutári pábulo nutriántur. Per Dóminum \emph{\&c.}
}\switchcolumn\portugues{
\rlettrine{V}{os} oferecemos, Senhor, este sacrifício na solenidade do B. Agostinho, vosso Pontífice e Confessor, suplicando-Vos humildemente que as ovelhas perdidas, tendo regressado ao único aprisco, se sustentem com este salutar alimento. Por nosso Senhor \emph{\&c.}
}\end{paracol}

\paragraph{Postcomúnio}
\begin{paracol}{2}\latim{
\rlettrine{H}{óstia} salutári refécti: te. Dómine, súpplices exorámus; ut eadem, beáti Augustíni interveniénte suffrágio, in omni loco nómini tuo júgiter immolétur. Per nominum nostrum \emph{\&c.}
}\switchcolumn\portugues{
\rlettrine{S}{aciados} com a hóstia salutar, Senhor, Vos pedimos humildemente que pela intercessão dos sufrágios do B. Agostinho ela seja imolada em honra do vosso nome em todos os lugares e constantemente. Por nosso Senhor \emph{\&c.}
}\end{paracol}

\subsectioninfo{Vigília de S. Lourenço, Mártir}{9 de Agosto}

\paragraphinfo{Intróito}{Sl. 111, 9}
\begin{paracol}{2}\latim{
\rlettrine{D}{ispérsit,} dedit paupéribus: justítia ejus manet in sǽculum sǽculi: cornu ejus exaltábitur in glória. \emph{Ps. ibid., 1} Beátus vir, qui timet
Dóminum: in mandátis ejus cupit nimis.
℣. Gloria Patri \emph{\&c.}
}\switchcolumn\portugues{
\rlettrine{D}{istribuiu} liberalmente os seus bens pelos pobres: a sua justiça subsistirá em todos os séculos dos séculos: e o seu poder será exaltado com glória. \emph{Sl. ibid., 1} Bem-aventurado o varão que teme o Senhor e que põe todo seu zelo em cumprir os seus Mandamentos.
℣. Glória ao Pai \emph{\&c.}
}\end{paracol}

\paragraph{Oração}
\begin{paracol}{2}\latim{
\rlettrine{A}{désto,} Dómine, supplicatiónibus nostris: et intercessióne beáti Lauréntii Mártyris tui, cujus prǽvénimus festivitátem; perpétuam nobis misericórdiam benígnus impénde. Per Dóminum \emph{\&c.}
}\switchcolumn\portugues{
\rlettrine{O}{uvi} as nossas súplicas, Senhor, e, pela intercessão do B. Lourenço, vosso Mártir, cuja festa antecipamos, concedei-nos benignamente a vossa perpétua misericórdia. Por nosso Senhor \emph{\&c.}
}\end{paracol}

\paragraphinfo{Epístola}{Página \pageref{virgensmartires1}}

\paragraphinfo{Gradual}{Sl. 111, 9 \& 2}
\begin{paracol}{2}\latim{
\rlettrine{D}{ispersit,} dedit paupéribus: justítia ejus manet in sǽculum sǽculi. ℣. Potens in terra erit semen ejus: generátio rectórum benedicétur
}\switchcolumn\portugues{
\rlettrine{D}{istribuiu} liberalmente os seus bens pelos pobres: a sua justiça subsistirá em todos os séculos dos séculos. ℣. Sua descendência será poderosa na terra, pois a geração dos justos será abençoada.
}\end{paracol}

\paragraphinfo{Evangelho}{Página \pageref{martirpontifice}}

\paragraphinfo{Ofertório}{Jb. 16, 20}
\begin{paracol}{2}\latim{
\rlettrine{O}{rátio} mea munda est: et ídeo peto, ut detur locus voci meæ in cœlo: quia ibi est judex meus, et cónscius meus in excélsis: ascéndat ad Dóminum deprecátio mea.
}\switchcolumn\portugues{
\rlettrine{A}{} minha oração é pura: eis porque peço que minha voz seja escutada no céu, pois lá está o meu Juiz; é nas alturas dos céus que está Aquele que conhece o íntimo do meu coração. Que minha deprecação suba até ao Senhor.
}\end{paracol}

\paragraph{Secreta}
\begin{paracol}{2}\latim{
\rlettrine{H}{óstias,} Dómine, quas tibi offérimus, propítius súscipe: et, intercedénte beáto Lauréntio Mártyre tuo, víncula peccatórum nostrorum absólve. Per Dóminum \emph{\&c.}
}\switchcolumn\portugues{
\rlettrine{R}{ecebei} propício, Senhor, estas hóstias, que Vos oferecemos; e, pela intercessão do B. Lourenço, vosso Mártir, livrai-nos das cadeias dos nossos pecados. Por nosso Senhor \emph{\&c.}
}\end{paracol}

\paragraphinfo{Comúnio}{Mt. 16, 24}
\begin{paracol}{2}\latim{
\qlettrine{Q}{ui} vult veníre post me, ábneget semetípsum, et tollat crucem suam, et sequátur me.
}\switchcolumn\portugues{
\rlettrine{S}{e} alguém quer vir após mim, negue-se a si mesmo, tome a sua cruz e siga-me!
}\end{paracol}

\paragraph{Postcomúnio}
\begin{paracol}{2}\latim{
\rlettrine{D}{a,} quǽsumus, Dómine, Deus noster: ut, sicut beáti Lauréntii Mártyris tui commemoratióne, temporáli gratulámur offício; ita perpétuo lætémur aspéctu. Per Dóminum nostrum \emph{\&c.}
}\switchcolumn\portugues{
\slettrine{Ó}{} Senhor, nosso Deus, Vos suplicamos, assim como tivemos a alegria de honrar temporalmente com este ofício a memória do B. Lourenço, vosso Mártir, assim também gozemos perpetuamente a felicidade da sua presença. Por nosso Senhor \emph{\&c.}
}\end{paracol}

\subsectioninfo{S. João Câncio, Conf.}{20 de Outubro}

\paragraphinfo{Intróito}{Sl. 91, 13-14}
\begin{paracol}{2}\latim{
\rlettrine{M}{iserátio} hóminis circa próximum: misericórdia autem Dei super omnem carnem. Qui misericórdiam habet, docet et érudit quasi pastor gregem suum. \emph{Ps. 1, 1} Beátus vir, qui non ábiit in consílio impiórum, et in via peccatórum non stetit, et in cáthedra pestiléntiæ non sedit.
℣. Gloria Patri \emph{\&c.}
}\switchcolumn\portugues{
\rlettrine{A}{} misericórdia do homem exerce-se para com seu próximo; enquanto que a misericórdia de Deus espalha-se sobre toda a carne. Aquele que possui misericórdia ensina e guia os homens, como um pastor guia o seu rebanho. \emph{Sl. 1, 1} Bem-aventurado o homem que não entrou na assembleia dos ímpios, nem seguiu os caminhos dos pecadores, nem se assentou na cadeira da maldade.
℣. Glória ao Pai \emph{\&c.}
}\end{paracol}

\paragraph{Oração}
\begin{paracol}{2}\latim{
\rlettrine{D}{a,} quǽsumus, omnípotens Deus: ut, sancti Joánnis Confessóris exémplo in scientia Sanctórum proficiéntes atque áliis misericórdiam exhibéntes; ejus méritis, indulgéntiam apud te consequámur. Per Dóminum \emph{\&c.}
}\switchcolumn\portugues{
\rlettrine{C}{oncedei-nos,} ó Deus omnipotente. Vos suplicamos, que, progredindo nós na ciência dos Santos e praticando a misericórdia para com o próximo, a exemplo do Santo Confessor João, obtenhamos pelos seus méritos a vossa indulgência.
Por nosso Senhor \emph{\&c.}
}\end{paracol}

\paragraphinfo{Epístola}{Tg. 2, 12-17}
\begin{paracol}{2}\latim{
Léctio Epístolæ beáti Jacóbi Apóstoli.
}\switchcolumn\portugues{
Lição da Ep.ª do B. Ap.º Tiago.
}\switchcolumn*\latim{
\rlettrine{S}{ic} loquímini, et sic fácite sicut per legem libertátis incipiéntes judicári. Judícium enim sine misericórdia illi, qui non fecit misericórdiam: superexáltat autem misericórdia judícium. Quid próderit, fratres mei, si fidem quis dicat se habére, ópera autem non hábeat? Numquid poterit fides salváre eum? Si autem frater et soror nudi sint, et indígeant victu quotidiáno, dicat autem áliquis ex vobis illis: Ite in pace, calefacímini et saturámini: non dedéritis autem eis, quæ necessária sunt córpori, quid próderit? Sic et fides, si non hábeat ópera, mórtua est in semetípsa.
}\switchcolumn\portugues{
\rlettrine{F}{alai} e procedei como devendo ser julgados pela lei da liberdade; pois o juízo de Deus será sem misericórdia para com aquele que não houver tido misericórdia; enquanto que a misericórdia triunfará no juízo. Meus irmãos: de que serve alguém dizer que tem fé, se não possui obras? Porventura poderá esta fé salvá-lo? Ora, se a um irmão ou irmã, estando nu e carecendo cada dia de sustento, algum de vós disser: «Ide em paz, aquecei-vos e saciai-vos», mas lhe não der aquilo que lhe é necessário ao corpo, de que lhe servirão as palavras? Assim, se a fé não é acompanhada de obras, está morta por si própria.
}\end{paracol}

\paragraphinfo{Gradual}{Sl. 106, 8-9}
\begin{paracol}{2}\latim{
\rlettrine{C}{onfiteántur} Dómino misericórdiæ ejus: et mirabília ejus fíliis hóminum. ℣. Quia satiávit ánimam inánem: et ánimam esuriéntem satiavit bonis.
}\switchcolumn\portugues{
\rlettrine{L}{ouvai} o Senhor pelas suas misericórdias e pelas suas maravilhas em favor dos filhos dos homens. ℣. Pois Ele saciou a alma vazia e encheu de benefícios a alma faminta.
}\switchcolumn*\latim{
Allelúja, allelúja. ℣. \emph{Prov. 31, 20} Manum suam apéruit ínopi: et palmas suas exténdit ad páuperem. Allelúja.
}\switchcolumn\portugues{
Aleluia, aleluia. ℣. \emph{Pr. 31, 20} Abriu a sua mão para o indigente: e estendeu os seus braços para o pobre. Aleluia.
}\end{paracol}

\paragraphinfo{Evangelho}{Página \pageref{confessoresnaopontifices1}}

\paragraphinfo{Ofertório}{Jb 29, 14-16}
\begin{paracol}{2}\latim{
\qlettrine{J}{ustítia} indútus sum, et vestívi me, sicut vestiménto et diadémate, judício meo. Oculus fui cæco et pes claudo: pater eram páuperum.
}\switchcolumn\portugues{
\rlettrine{R}{evesti-me} de justiça, e a equidade dos meus juízos serviu-me como que de vestido e de diadema. Tenho sido olhos para o cego e pés para o coxo. Tenho sido o pai dos pobres.
}\end{paracol}

\paragraph{Secreta}
\begin{paracol}{2}\latim{
\rlettrine{H}{as,} quǽsumus, Dómine, hóstias sancti Joánnis Confessóris tui méritis benígnus assúme: et præsta; ut, te super ómnia et omnes propter te diligéntes, corde tibi et ópere placeámus. Per Dóminum \emph{\&c.}
}\switchcolumn\portugues{
\rlettrine{A}{ceitai} benignamente estas hóstias pelos méritos do vosso Santo Confessor João, Vos suplicamos, Senhor, e fazei que, amando-Vos sobre todas as cousas e ao próximo por amor de Vós, nos tornemos agradáveis a Deus pelos nossos sentimentos e obras. Por nosso Senhor \emph{\&c.}
}\end{paracol}

\paragraphinfo{Comúnio}{Lc. 6, 38}
\begin{paracol}{2}\latim{
\rlettrine{D}{ate,} et dábitur vobis: mensúram bonam et confértam et coagitátam et supereffluéntem dabunt in sinum vestrum.
}\switchcolumn\portugues{
\rlettrine{D}{ai} e ser-vos-á dado: derramar-se-á no vosso seio uma boa medida, cheia, calcada, acogulada e a trasbordar.
}\end{paracol}

\paragraph{Postcomúnio}
\begin{paracol}{2}\latim{
\rlettrine{P}{retiósi} Córporis et Sánguinis tui, Dómine, pasti delíciis, tuam súpplices deprecámur clementiam: ut, sancti Joánnis Confessóris tui méritis et exémplis, ejúsdem caritátis imitatóres effécti, consórtes simus et glóriæ: Qui vivis et regnas \emph{\&c.}
}\switchcolumn\portugues{
\rlettrine{H}{avendo} sido alimentados com as delícias do vosso preciosíssimo Corpo e Sangue, Senhor, imploramos humildemente da vossa clemência, que, pelos méritos e exemplos do vosso Santo Confessor João, nos tornemos imitadores da sua caridade e compartilhemos também da sua glória. Ó Vós, que viveis e reinais \emph{\&c.}
}\end{paracol}

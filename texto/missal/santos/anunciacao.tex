\subsectioninfo{Anunciação da B. V. Maria }{25 de Março}

\paragraphinfo{Intróito}{Sl. 44, 13, 15 \& 16}
\begin{paracol}{2}\latim{
\rlettrine{V}{ultum} tuum deprecabúntur omnes dívites plebis: adducántur Regi Vírgines post eam: próximæ ejus adducántur tibi in lætítia et exsultatióne. (T. P. Allelúja, allelúja.) \emph{Ps. ib., 2} Eructávit cor meum verbum bonum: dico ego ópera mea Regi.
℣. Gloria Patri \emph{\&c.}
}\switchcolumn\portugues{
\rlettrine{T}{odos} os poderosos da terra implorarão os vossos olhares: após ela serão apresentadas virgens ao Rei: as suas companheiras serão apresentadas ao Rei com grande alegria e jubilo. (T. P. Aleluia, aleluia). \emph{Sl. ib., 2} Meu coração exprimiu uma palavra excelente: «Consagro ao Rei as minhas obras!».
℣. Glória ao Pai \emph{\&c.}
}\end{paracol}

\paragraph{Oração}
\begin{paracol}{2}\latim{
\rlettrine{D}{eus,} qui de beátæ Maríæ Vírginis útero Verbum tuum, Angelo nuntiánte, carnem suscípere voluísti: præsta supplícibus tuis; ut, qui vere eam Genetrícem Dei crédimus, ejus apud te intercessiónibus adjuvémur. Per eúndem Dóminum \emph{\&c.}
}\switchcolumn\portugues{
\slettrine{Ó}{} Deus, que, segundo a anunciação do Anjo, quisestes que o vosso Verbo assumisse a carne humana no seio da bem-aventurada Virgem Maria, concedei aos vossos suplicantes que os que crêem que Ela é verdadeira Mãe de Deus sejam amparados na vossa presença com o auxílio das suas preces. Pelo mesmo nosso Senhor \emph{\&c.}
}\end{paracol}

\paragraphinfo{Epístola}{Is. 7, 10-15}
\begin{paracol}{2}\latim{
Léctio Isaíæ Prophétæ.
}\switchcolumn\portugues{
Lição do Profeta Isaías.
}\switchcolumn*\latim{
\rlettrine{I}{n} diébus illis: Locútus est Dóminus ad Achaz, dicens: Pete tibi signum a Dómino, Deo tuo, in profúndum inférni, sive in excélsum supra. Et dixit Achaz: Non petam ei non tentábo Dóminum. Et dixit: Audíte ergo, domus David: Numquid parum vobis est, moléstos esse homínibus, quia molésti estis et Deo meo? Propter hoc dabit Dóminus ipse vobis signum. Ecce, Virgo concípiet et páriet fílium, et vocábitur nomen ejus Emmánuel. Butýrum ei mel cómedet, ut sciat reprobáre malum et elígere bonum.
}\switchcolumn\portugues{
\rlettrine{N}{aqueles} dias, falou o Senhor a Acaz e disse-lhe: «Pedi ao Senhor, vosso Deus, um prodígio nas profundezas do inferno ou nas alturas do céu». Acaz respondeu: «Não pedirei tal coisa e não tentarei o Senhor». E Isaías disse: «Escutai, então, casa de David: Porventura vos não basta que fatigueis a paciência dos homens, senão que queirais fatigar a do meu Deus? Eis porque o Senhor vos dará um sinal: Uma virgem conceberá e dará à luz um filho, e o seu nome será Emanuel: Ele comerá manteiga e mel, para que saiba condenar o mal e escolher o bem».
}\end{paracol}

\paragraphinfo{Gradual}{Sl. 44, 3 et 5}
\begin{paracol}{2}\latim{
\rlettrine{D}{iffúsa} est grátia in lábiis tuis: proptérea benedíxit te Deus in ætérnum. ℣. Propter veritátem et mansuetúdinem et justítiam: et dedúcet te mirabíliter déxtera tua.
}\switchcolumn\portugues{
\rlettrine{A}{} graça espalhou-se nos vossos lábios; eis porque Deus vos abençoou para a eternidade. ℣. Reinareis pela verdade, mansidão e justiça; e a vossa dextra conduzir-vos-á admiravelmente.
}\switchcolumn*\latim{

}\switchcolumn\portugues{

}\end{paracol}

\paragraphinfo{Trato}{ibid., 11 \& 12}
\begin{paracol}{2}\latim{
\rlettrine{A}{udi,} fília, et vide, et inclína aurem tuam: quia concupívit Rex speciem tuam. ℣. \emph{ibid., 13 \& 10} Vultum tuum deprecabúntur omnes dívites plebis: fíliæ regum in honóre tuo. ℣. \emph{ibid., 15-16} Adducántur Regi Vírgines post eam: próximæ ejus afferéntur tibi. ℣. Adducántur in lætítia et exsultatióne: adducántur in templum Regis.

}\switchcolumn\portugues{
\rlettrine{O}{uvi,} minha filha, vede e abri os vossos ouvidos; pois o Rei está extasiado com vossa formosura. ℣. \emph{ibid., 13 \& 10} Todos os poderosos da terra implorarão os vossos olhares; as filhas dos reis formarão a vossa corte de honra. ℣. \emph{ibid., 15-16} As virgens serão apresentados ao Rei após ela: as suas companheiras ser-vos-ão apresentadas. Serão conduzidas por entre a alegria e o júbilo e apresentadas no templo do rei.
}\end{paracol}

\textit{No T. Pascal omite-se o Gradual e o Trato e diz-se:}

\begin{paracol}{2}\latim{
Allelúja, allelúja. ℣. \emph{Luc. 1, 28} Ave, María, grátia plena; Dóminus tecum: benedicta tu in muliéribus. Allelúja. ℣. \emph{Num. 17, 8} Virga Jesse flóruit: Virgo Deum et hóminem génuit: pacem Deus réddidit, in se reconcílians ima summis. Allelúja.
}\switchcolumn\portugues{
Aleluia, aleluia. ℣. \emph{Lc. 1, 28} Ave, Maria: o Senhor é convosco. Bendita sois Vós entre as mulheres. Aleluia. ℣. \emph{Nm. 17, 8} A vara de Jessé floresceu e a Virgem deu à luz o Homem-Deus: restabeleceu Deus a paz, reconciliando na sua pessoa a nossa baixeza com a suprema grandeza. Aleluia.
}\end{paracol}

\paragraphinfo{Evangelho}{Lc. 1, 26-38}
\begin{paracol}{2}\latim{
\cruz Sequéntia sancti Evangélii secúndum Lucam.
}\switchcolumn\portugues{
\cruz Continuação do santo Evangelho segundo S. Lucas.
}\switchcolumn*\latim{
\blettrine{I}{n} illo témpore: Missus est Angelus Gábriel a Deo in civitátem Galilǽæ, cui nomen Názareth, ad Vírginem desponsátam viro, cui nomen erat Joseph, de domo David, et nomen Vírginis María. Ei ingréssus Angelus ad eam, dixit: Ave, grátia plena; Dóminus tecum: benedícta tu in muliéribus. Quæ cum audísset, turbáta est in sermóne ejus: et cogitábat, qualis esset ista salutátio. Et ait Angelus ei: Ne tímeas, María, invenísti enim grátiam apud Deum: ecce, concípies in útero et páries fílium, et vocábis nomen ejus Jesum. Hic erit magnus, et Fílius Altíssimi vocábitur, et dabit illi Dóminus Deus sedem David, patris ejus: et regnábit in domo Jacob in ǽtérnum, et regni ejus non erit finis. Dixit autem María ad Angelum: Quómodo fiet istud, quóniam virum non cognósco? Et respóndens Angelus, dixit ei: Spíritus Sanctus supervéniet in te, et virtus Altíssimi obumbrábit tibi. Ideóque et quod nascétur ex te Sanctum, vocábitur Fílius Dei. Et ecce, Elísabeth, cognáta tua, et ipsa concépit fílium in senectúte sua: et hic mensis sextus est illi, quæ vocátur stérilis: quia non erit impossíbile apud Deum omne verbum. Dixit autem María: Ecce ancílla Dómini, fiat mihi secúndum verbum tuum.
}\switchcolumn\portugues{
\blettrine{N}{aquele} tempo, foi mandado por Deus o Anjo Gabriel a uma cidade da Galileia, chamada Nazaré, a uma Virgem, desposada com um varão, cujo nome era José, da casa de David; e o nome da Virgem era Maria. Entrando o Anjo onde ela estava, disse: «Eu te saúdo, cheia de graça: o Senhor é contigo: bendita és tu entre todas as mulheres». Ouvindo ela isto, perturbou-se, e pensava na significação desta saudação. Então, disse-lhe o Anjo: «Não temas, Maria, porquanto alcançaste graça diante do Senhor: eis que conceberás no teu seio, e darás à luz um Filho, e o seu nome será Jesus. Ele será grande e será chamado Filho do Altíssimo; o Senhor Deus Lhe dará o trono de David, seu pai; reinará eternamente na casa de Jacob; e o seu reino não terá fim». Porém Maria disse ao Anjo: «Como acontecerá isso, se não conheço varão?». O Anjo, respondendo, disse-lhe: «O Espírito Santo descerá sobre ti, e a virtude do Altíssimo te tocará com sua sombra. Por isso o Santo que nascer de ti será chamado Filho de Deus. E eis que Isabel, tua parente, concebeu um filho na sua velhice: este é o sexto mês daquela que é chamada estéril; porque nada é impossível a Deus». Então disse Maria: «Eis aqui a escrava do Senhor, faça-se em mim segundo a tua palavra».
}\end{paracol}

\paragraphinfo{Ofertório}{Lc. l, 28 \& 42}
\begin{paracol}{2}\latim{
\rlettrine{A}{ve,} Maria, grátia plena; Dóminus tecum: benedícta tu in muliéribus, et benedíctus fructus ventris tui. (T. P. Allelúja.)
}\switchcolumn\portugues{
\rlettrine{A}{ve,} Maria, cheia de graça: o Senhor é convosco: bendita sois vós entre as mulheres, e bendito é o fruto do vosso ventre. (T. P. Aleluia.)
}\end{paracol}

\paragraph{Secreta}
\begin{paracol}{2}\latim{
\rlettrine{I}{n} méntibus nostris, quǽsumus, Dómine, veræ fídei sacraménta confírma: ut, qui concéptum de Vírgine Deum verum et hóminem confitémur; per ejus salutíferæ resurrectiónis poténtiam, ad ætérnam mereámur perveníre lætítiam. Per eúndem Dóminum nostrum \emph{\&c.}
}\switchcolumn\portugues{
\rlettrine{D}{ignai-Vos} confirmar nas nossas almas, Senhor, os mistérios da verdadeira fé, a fim de que nós, que confessamos que Aquele que foi concebido pela Virgem Maria é verdadeiro Deus e Homem, mereçamos alcançar pela virtude da sua salutar ressurreição a felicidade eterna. Por nosso Senhor \emph{\&c.}
}\end{paracol}

\paragraphinfo{Comúnio}{Is. 7, 14}
\begin{paracol}{2}\latim{
\rlettrine{E}{cce,} Virgo concípiet et páriet fílium: et vocábitur nomen ejus Emmánuel. (T. P. Allelúja.)
}\switchcolumn\portugues{
\rlettrine{E}{is} que a Virgem conceberá, dará à luz um filho e o seu nome será Emanuel. (T. P. Aleluia.)
}\end{paracol}

\paragraph{Postcomúnio}
\begin{paracol}{2}\latim{
\rlettrine{G}{rátiam} tuam, quǽsumus, Dómine, méntibus nostris infúnde: ut, qui. Angelo nuntiánte, Christi Fílii tui incarnatiónem cognóvimus; per passiónem ejus et crucem, ad resurrectiónis glóriam perducámur. Per eúndem Dóminum nostrum \emph{\&c.}
}\switchcolumn\portugues{
\rlettrine{I}{nfundi,} Senhor, Vos suplicamos, a vossa graça em nossas almas, para que nós, que pela anunciação do Anjo conhecemos a Incarnação do vosso Filho, sejamos conduzidos à glória da ressurreição pela sua Paixão e Cruz. Pelo mesmo nosso Senhor \emph{\&c.}
}\end{paracol}

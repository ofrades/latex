\subsectioninfo{S. José, Esposo da B. V. Maria}{19 de Março}

\paragraphinfo{Intróito}{}
\begin{paracol}{2}\latim{
\qlettrine{J}{ustus} ut palma florébit: sicut cedrus Líbani multiplicábitur: plantátus in domo Dómini: in átriis domus Dei nostri. (T. P. Allelúja, allelúja.) \emph{Ps. ibid., 2} Bonum est confiteri Dómino: et psállere nómini tuo, Altíssime.
℣. Gloria Patri \emph{\&c.}
}\switchcolumn\portugues{
\rlettrine{O}{} justo florescerá, como a palmeira, e multiplicar-se-á, como o cedro do Líbano: Ele está plantado na casa do Senhor, nos átrios da casa do nosso Deus. \emph{Sl. ibid., 2} É bom louvar o Senhor: e cantar hinos em honra do vosso nome, ó Altíssimo.
℣. Glória ao Pai \emph{\&c.}
}\end{paracol}

\paragraph{Oração}
\begin{paracol}{2}\latim{
\rlettrine{S}{anctíssimæ} Genetrícis tuæ Sponsi, quǽsumus. Dómine, méritis adjuvémur: ut, quod possibílitas nostra non óbtinet, ejus nobis intercessióne donétur: Qui vivis \emph{\&c.}
}\switchcolumn\portugues{
\rlettrine{S}{enhor,} Vos suplicamos, permiti que sejamos auxiliados pelos méritos do Esposo da vossa Santíssima Mãe, a fim de que aquelas graças que não temos possibilidade de obter por causa da nossa fraqueza as alcancemos pela sua intercessão. Ó Vós, que viveis e reinais \emph{\&c.}
}\end{paracol}

\paragraphinfo{Epístola}{Ecl. 45, 1-6}
\begin{paracol}{2}\latim{
Léctio libri Sapiéntiæ.
}\switchcolumn\portugues{
Lição do Livro da Sabedoria.
}\switchcolumn*\latim{
\rlettrine{D}{iléctus} Deo et homínibus, cujus memória in benedictióne est. Símilem illum fecit in glória sanctórum, et magnificávit eum in timóre inimicórum, et in verbis suis monstra placávit. Glorificávit illum in conspéctu regum, et jussit illi coram pópulo suo, et osténdit illi glóriam suam. In fide et lenitáte ipsíus sanctum fecit illum, et elégit eum ex omni carne. Audívit enim eum et vocem ipsíus, et indúxit illum in nubem. Et dedit illi coram præcépta, et legem vitæ et disciplínæ.
}\switchcolumn\portugues{
\rlettrine{F}{oi} amado de Deus e dos homens; a sua memória é uma bênção. O Senhor deu-lhe uma glória, semelhante à dos Santos; tornou-o temeroso e invencível perante seus inimigos; e com suas palavras aplacou os monstros. O Senhor honrou-o diante dos reis; deu-lhe as suas ordens diante do seu povo; e mostrou-lhe a sua glória. O Senhor santificou-o pela sua fé e mansidão; e escolheu-o entre todos os homens. Deus escutou-o; ouviu a sua voz; e fê-lo entrar na nuvem. Então, deu-lhe face a face os seus preceitos e a lei da vida e da doutrina.
}\end{paracol}

\paragraphinfo{Gradual}{Sl. 20, 4-5}
\begin{paracol}{2}\latim{
\rlettrine{D}{ómine,} prævenísti eum in benedictiónibus dulcédinis: posuísti in cápite ejus corónam de lápide pretióso. ℣. Vitam pétiit a te, et tribuísti ei longitúdinem diérum in sǽculum sǽculi.
}\switchcolumn\portugues{
\rlettrine{S}{enhor,} concedestes-lhe bênçãos escolhidas as mais suaves; e impusestes na sua cabeça uma coroa de pedras preciosas. ℣. Concedestes-lhe a vida, que ele Vos suplicara, e prolongastes-lhe a duração dos seus dias pelos séculos dos séculos.
}\end{paracol}

\paragraphinfo{Trato}{Sl. 111, 1-3}
\begin{paracol}{2}\latim{
\rlettrine{B}{eátus} vir, qui timet Dóminum: in mandátis ejus cupit nimis. ℣. Potens in terra erit semen ejus: generátio rectórum benedicétur. ℣. Glória et divítiæ in domo ejus: et justítia ejus manet in sǽculum sǽculi.
}\switchcolumn\portugues{
\rlettrine{B}{em-aventurado} o varão que teme o Senhor e que emprega todo o zelo em obedecer-Lhe. Sua descendência será poderosa na terra; pois a geração dos justos será abençoada. Na sua casa haverá glória e riqueza: e a sua justiça subsistirá em todos os séculos dos séculos.
}\end{paracol}

\emph{No Tempo da Quaresma, omite-se o Gradual e o Trato e diz-se:}

\begin{paracol}{2}\latim{
Allelúja, allelúja. ℣. \emph{Eccli. 45, 9} Amávit eum Dóminus, et ornávit eum: stolam glóriæ índuit eum. Allelúja. ℣. \emph{Osee 14. 6} Justus germinábit sicut lílium: et florébit in ætérnum ante Dóminum. Allelúja.
}\switchcolumn\portugues{
Aleluia, aleluia. ℣. \emph{Ecl. 45, 9} O Senhor o amou e adornou: e revestiu-o com uma túnica de glória. Aleluia. ℣. \emph{Os. 14. 6} O justo florescerá, como a palmeira, e crescerá, como o cedro do Líbano.
}\end{paracol}

\paragraphinfo{Evangelho}{Mt. 1, 18-21}
\begin{paracol}{2}\latim{
\cruz Sequéntia sancti Evangélii secúndum Matthǽum.
}\switchcolumn\portugues{
\cruz Continuação do santo Evangelho segundo S. Mateus.
}\switchcolumn*\latim{
\blettrine{C}{um} esset desponsáta Mater Jesu María Joseph, ántequam convenírent, invénta est in útero habens de Spíritu Sancto. Joseph autem, vir ejus, cum esset justus et nollet eam tradúcere, vóluit occúlte dimíttere eam. Hæc autem eo cogitánte, ecce, Angelus Dómini appáruit in somnis ei, dicens: Joseph, fili David, noli timére accípere Maríam cónjugem tuam: quod enim in ea natum est, de Spíritu Sancto est. Páriet autem fílium, et vocábis nomen ejus Jesum: ipse enim salvum fáciet pópulum suum a peccátis eórum.
}\switchcolumn\portugues{
\blettrine{N}{aquele} tempo, estando Maria, Mãe de Jesus, desposada com José, achou este que, sem que os dous houvessem coabitado, havia ela concebido por obra do Espírito Santo. Então José, seu marido, que era justo e não queria difamá-la, resolveu abandoná-la secretamente. E, pensando nisto, apareceu-lhe um Anjo do Senhor em sonhos, dizendo-lhe: «José, filho de David, não temas receber Maria como tua esposa porque Aquele que ela gerou vem do Espírito Santo. Ela dará à luz um filho, e tu Lhe darás o nome de Jesus, O qual salvará o seu povo, livrando-o dos seus pecados».
}\end{paracol}

\paragraphinfo{Ofertório}{Sl. 88, 25}
\begin{paracol}{2}\latim{
\rlettrine{V}{éritas} mea et misericórdia mea cum ipso: et in nómine meo exaltábitur cornu ejus. (T. P. Allelúja.)
}\switchcolumn\portugues{
\rlettrine{A}{} minha verdade e a minha misericórdia estarão com ele; e o seu poder será exaltado em meu nome. (T. P. Aleluia.)
}\end{paracol}

\paragraph{Secreta}
\begin{paracol}{2}\latim{
\rlettrine{D}{ébitum} tibi, Dómine, nostræ réddimus servitútis, supplíciter exorántes: ut, suffrágiis beáti Joseph, Sponsi Genetrícis Fílii tui Jesu Christi, Dómini nostri, in nobis tua múnera tueáris, ob cujus venerándam festivitátem laudis tibi hóstias immolámus. Per eúndem Dóminum nostrum \emph{\&c.}
}\switchcolumn\portugues{
\rlettrine{V}{os} oferecemos, Senhor, o justo tributo da nossa servidão, e instantemente Vos suplicamos queirais conservar em nós os vossos dons pela intercessão do B. José, Esposo da Mãe de Jesus Cristo, vosso Filho e nosso Senhor, em cuja festa, que celebramos, imolamos nossas hóstias em vossa honra. Pelo mesmo nosso Senhor \emph{\&c.}
}\end{paracol}

\paragraphinfo{Comúnio}{Mt. 1, 20}
\begin{paracol}{2}\latim{
\qlettrine{J}{oseph,} fili David, noli timére accípere Maríam cónjugem tuam: quod enim in ea natum est, de Spíritu Sancto est. (T. P. Allelúja.)
}\switchcolumn\portugues{
\qlettrine{J}{osé,} filho de David, não temas receber Maria como tua esposa, porque Aquele que ela gerou vem do Espírito Santo. (T. P. Aleluia.)
}\end{paracol}

\paragraph{Postcomúnio}
\begin{paracol}{2}\latim{
\rlettrine{A}{désto} nobis, quǽsumus, miséricors Deus: et, intercedénte pro nobis beáto Joseph Confessóre, tua circa nos propitiátus dona custódi. Per Dóminum \emph{\&c.}
}\switchcolumn\portugues{
\rlettrine{A}{ssisti-nos,} ó Deus de misericórdia, Vos suplicamos; e pela intercessão do B. José, Confessor, dignai-Vos propício conservar em nós os vossos dons. Por nosso Senhor \emph{\&c.}
}\end{paracol}

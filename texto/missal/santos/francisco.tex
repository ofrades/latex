\subsectioninfo{S. Francisco}{4 de Outubro}

\textit{Como na Missa Imp. dos Estigmas em S. Francisco, a 17 de Setembro, página \pageref{estigmasfrancisco}, excepto:}

\paragraph{Oração}
\begin{paracol}{2}\latim{
\rlettrine{D}{eus,} qui Ecclésiam tuam, beáti Francisci méritis fœtu novæ prolis amplíficas: tríbue nobis; ex ejus imitatióne, terréna despícere et cœléstium donórum semper participatióne gaudére. Per Dóminum \emph{\&c.}
}\switchcolumn\portugues{
\slettrine{Ó}{} Deus, que pelos méritos do B. Francisco enriquecestes a vossa Igreja, dando-lhe uma nova família, concedei-nos a graça de imitá-lo, desprezando os bens terrenos, e de sempre nos alegrarmos com a participação dos dons celestiais. Por nosso Senhor \emph{\&c.}
}\end{paracol}

\paragraphinfo{Evangelho}{Página \pageref{pauloeremita}}

\paragraph{Postcomúnio}
\begin{paracol}{2}\latim{
\rlettrine{E}{cclésiam} tuam, quǽsumus, Dómine, grátia cœléstis amplíficet: quam beáti Francísci Confessóris tui illumináre voluísti gloriósis méritis et exémplis. Per Dóminum nostrum \emph{\&c.}
}\switchcolumn\portugues{
\rlettrine{D}{ignai-Vos,} Senhor, Vos suplicamos, com a graça celestial dilatar a vossa Igreja, a qual quisestes ilustrar com os gloriosos méritos e exemplos do B. Francisco, vosso Confessor. Por nosso Senhor \emph{\&c.}
}\end{paracol}

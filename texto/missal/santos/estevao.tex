\subsectioninfo{S. Estêvão, Primeiro Mártir}{26 de Dezembro}

\paragraphinfo{Intróito}{Sl. 118, 23, 86 \& 23}
\begin{paracol}{2}\latim{
\rlettrine{S}{edérunt} príncipes, et advérsum me loquebántur: et iníqui persecúti sunt me: ádjuva me, Dómine, Deus meus, quia servus tuus exercebátur in tuis justificatiónibus. \emph{Ps. ibid., 1} Beati immaculáti in via, qui ámbulant in lege Dómini.
℣. Gloria Patri \emph{\&c.}
}\switchcolumn\portugues{
\rlettrine{R}{euniram-se} os príncipes em assembleia e pronunciaram-se contra mim: e os maus perseguiram-me injustamente: Auxiliai-me, Senhor, meu Deus, pois o vosso servo tem observado os vossos ensinos. \emph{Ps. ibid., 1} Bem-aventurados os que são rectos nos seus caminhos e cumprem a Lei do Senhor.
℣. Glória ao Pai \emph{\&c.}
}\end{paracol}

\paragraph{Oração}
\begin{paracol}{2}\latim{
\rlettrine{D}{a} nobis, quǽsumus, Dómine, imitári quod cólimus: ut discámus et inimícos dilígere; quia ejus natalícia celebrámus, qui novit étiam pro persecutóribus exoráre Dóminum nostrum Jesum Christum, Fílium tuum: Qui tecum vivit \emph{\&c.}
}\switchcolumn\portugues{
\rlettrine{C}{oncedei-nos,} Senhor, Vos imploramos, a graça de imitar aquele que honramos, a fim de que neste dia aprendamos a amar os nossos inimigos, pois que celebramos a festa daquele que soube rogar pelos seus perseguidores a nosso Senhor Jesus Cristo: Que, sendo Deus \emph{\&c.}
}\end{paracol}

\paragraphinfo{Epístola}{Act. 6, 8-10; 7, 54-59}
\begin{paracol}{2}\latim{
Lectio Actuum Apostolorum.
}\switchcolumn\portugues{
Lição dos Actos dos Apóstolos.
}\switchcolumn*\latim{
\rlettrine{I}{n} diebus illis: Stéphanus plenus grátia et fortitúdine, faciébat prodígia et signa magna in pópulo. Surrexérunt autem quidam de synagóga, quæ appellátur Libertinórum, et Cyrenénsium, et Alexandrinórum, et eórum, qui erant a Cilícia et Asia, disputántes cum Stéphano: et non póterant resístere sapiéntiæ et Spirítui, qui loquebátur. Audiéntes autem hæc, dissecabántur córdibus suis, et stridébant déntibus in eum. Cum autem esset Stéphanus plenus Spíritu Sancto, inténdens in cœlum, vidit glóriam Dei, et Jesum stantem a dextris Dei. Et ait: Ecce, vídeo cœlos apértos, et Fílium hóminis stantem a dextris Dei. Exclamántes autem voce magna continuérunt aures suas, et ímpetum fecerunt unanímiter in eum. Et ejiciéntes eum extra civitatem, lapidábant: et testes deposuérunt vestiménta sua secus pedes adolescéntis, qui vocabátur Saulus. Et lapidábant Stéphanum invocántem et dicéntem: Dómine Jesu, súscipe spíritum meum. Pósitis autem génibus, clamávit voce magna, dicens: Dómine, ne státuas illis hoc peccátum. Et cum hoc dixísset, obdormívit in Dómino.
}\switchcolumn\portugues{
\rlettrine{N}{aqueles} dias, Estêvão, cheio de graça e de fortaleza, praticava grandes prodígios e sinais perante o povo. Levantaram-se, então, alguns membros da sinagoga chamada dos Libertos, dos Cirenenses, dos Alexandrinos, e outros da Cilícia e da Ásia, e disputaram com Estêvão, não podendo resistir à sabedoria e ao Espírito com que falava. Ouvindo, pois, suas palavras, a raiva minava-lhes os corações, e os dentes rangiam-lhes de furor contra ele. Porém Estêvão, estando cheio do Espírito Santo e havendo fixado os olhos no céu, viu a glória de Deus, e Jesus, que estava de pé, à dextra de Deus. E disse-lhes: «Eis que eu vejo os céus abertos e o Filho do homem, de pé, à dextra de Deus». Porém eles, clamando com grande ruído, taparam os ouvidos e arremessaram-se todos contra ele; e, arrastando-o para fora da cidade, apedrejaram-no. As testemunhas puseram os vestidos dele ao pé de um jovem, chamado Saulo. E apedrejaram Estêvão. Mas este, invocando o Senhor, dizia: «Senhor Jesus, recebei o meu espírito». E, pondo-se de joelhos, clamou com voz forte: «Senhor, lhes não imputeis este pecado». Tendo acabado de dizer isto, adormeceu no Senhor.
}\end{paracol}

\paragraphinfo{Gradual}{Sl. 118, 23 \& 86}
\begin{paracol}{2}\latim{
\rlettrine{S}{edérunt} príncipes, et advérsum me loquebántur: et iníqui persecúti sunt me. ℣. \emph{Ps. 6, 5} Adjuva me, Dómine, Deus meus: salvum me fac propter misericórdiam tuam.
}\switchcolumn\portugues{
\rlettrine{R}{euniram-se} os príncipes em assembleia, e pronunciaram-se contra mim: e os maus perseguiram-me injustamente. ℣. \emph{Sl. 6, 5} Auxiliai-me, ó Senhor, meu Deus, salvai-me pela vossa misericórdia.
}\switchcolumn*\latim{
Allelúja, allelúja. ℣. \emph{Act. 7, 55} Vídeo cœlos apértos, et Jesum stantem a dextris virtútis Dei. Allelúja.
}\switchcolumn\portugues{
Aleluia, aleluia. ℣. \emph{Act. 7, 55} Vejo os céus abertos, e Jesus, de pé, à dextra de Deus omnipotente. Aleluia.
}\end{paracol}

\paragraphinfo{Evangelho}{Mt. 23, 34-39}
\begin{paracol}{2}\latim{
\cruz Sequéntia sancti Evangélii secúndum Lucam.
}\switchcolumn\portugues{
\cruz Continuação do santo Evangelho segundo S. Mateus.
}\switchcolumn*\latim{
\blettrine{I}{n} illo témpore: Dicébat Jesus scribis et pharisǽis: Ecce, ego mitto ad vos prophétas, et sapiéntes, et scribas, et ex illis occidétis et crucifigétis, et ex eis flagellábitis in synagógis vestris, et persequémini de civitáte in civitátem: ut véniat super vos omnis sanguis justus, qui effúsus est super terram, a sánguine Abel justi usque ad sánguinem Zacharíæ, filii Barachíæ, quem occidístis inter templum et altáre. Amen, dico vobis, vénient hæc ómnia super generatiónem istam. Jerúsalem, Jerúsalem, quæ occídis prophétas, et lápidas eos, qui ad te missi sunt, quóies vólui congregáre fílios tuos, quemádmodum gallína cóngregat pullos suos sub alas, et noluísti? Ecce, relinquétur vobis domus vestra desérta. Dico enim vobis, non me vidébitis ámodo, donec dicátis: Benedíctus, qui venit in nómine Dómini.
}\switchcolumn\portugues{
\blettrine{N}{aquele} tempo, Jesus dizia aos escribas e fariseus: «Eis que vos envio profetas, sábios e escribas; mas vós matareis e crucificareis uns, e açoitareis outros nas vossas sinagogas; e persegui-los-eis de terra em terra. Porém, cairá sobre vós todo o sangue inocente que tem sido derramado na terra desde sangue do justo Abel até ao sangue de Zacarias, filho de Baraquias, a quem matastes entre o santuário e o altar. Em verdade vos digo: tudo isto cairá sobre esta geração. Jerusalém, Jerusalém, que matas os profetas e apedrejas os que te são enviados, quantas vezes quis eu reunir os teus filhos, como a galinha agasalha os seus pintainhos debaixo das asas, e tu não quiseste?! Eis que a tua casa vai tornar-se deserta e abandonada. Pois eu te digo: «desde agora me não verás até que digas: Bendito seja Aquele que vem em nome do Senhor».
}\end{paracol}

\paragraphinfo{Ofertório}{Act. 6, 5 \& 7, 59}
\begin{paracol}{2}\latim{
\rlettrine{E}{legérunt} Apóstoli Stéphanum Levítam, plenum fide et Spíritu Sancto: quem lapidavérunt Judǽi orántem, et dicéntem: Dómine Jesu, áccipe spíritum meum, allelúja.
}\switchcolumn\portugues{
\rlettrine{O}{s} Apóstolos escolheram Estêvão para Diácono, pois ele era cheio de fé e do Espírito Santo; o qual os judeus apedrejaram, enquanto rezava e dizia: «Senhor Jesus, recebei o meu espírito». Aleluia.
}\end{paracol}

\paragraph{Secreta}
\begin{paracol}{2}\latim{
\rlettrine{S}{úscipe,} Dómine, múnera pro tuórum commemoratióne Sanctórum: ut, sicut illos pássio gloriósos effécit; ita nos devótio reddat innócuos. Per Dóminum \emph{\&c.}
}\switchcolumn\portugues{
\rlettrine{R}{ecebei,} Senhor, as nossas ofertas que Vos apresentamos em memória dos vossos Santos, para que, assim como o seu martírio os tornou gloriosos, assim também a nossa piedade nos torne inocentes. Por nosso Senhor \emph{\&c.}
}\end{paracol}

\paragraphinfo{Comúnio}{Act. 7, 55, 58 \& 59}
\begin{paracol}{2}\latim{
\rlettrine{V}{ídeo} cœlos apértos, et Jesum stantem a dextris virtútis Dei: Dómine Jesu, accipe spíritum meum, et ne státuas illis hoc peccátum.
}\switchcolumn\portugues{
\rlettrine{V}{ejo} os céus abertos e Jesus, de pé, à dextra de Deus omnipotente. Ó Senhor Jesus, recebei o meu espírito e lhes não imputeis este pecado.
}\end{paracol}

\paragraph{Postcomúnio}
\begin{paracol}{2}\latim{
\rlettrine{A}{uxiliéntur} nobis, Dómine, sumpta mystéria: et, intercedénte beáto Stéphano Mártyre tuo, sempitérna protectióne confírment. Per Dóminum nostrum \emph{\&c.}
}\switchcolumn\portugues{
\rlettrine{P}{ermiti,} Senhor, que os mistérios, que acabámos de receber, nos sirvam de perpétuo auxílio, e que por intercessão do B. Estêvão, vosso Mártir, nos confirmem na vossa contínua protecção. Por nosso Senhor \emph{\&c.}
}\end{paracol}

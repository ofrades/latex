\subsectioninfo{Santa Mónica, Viúva}{4 de Maio}\label{monica}

\textit{Como na Missa Cognóvi, Dómine, página \pageref{nemvirgensnemmartires}, excepto:}

\paragraph{Oração}
\begin{paracol}{2}\latim{
\rlettrine{D}{eus,} mæréntium consolátor et in te sperántium salus, qui beátæ Mónicæ pias lácrimas in conversióne fílii sui Augustíni misericórditer suscepísti: da nobis utriúsque intervéntu; peccáta nostra deploráre, et grátiæ tuæ indulgéntiam inveníre. Per Dóminum nostrum \emph{\&c.}
}\switchcolumn\portugues{
\slettrine{Ó}{} Deus, que consolais os tristes e salvais os que em Vós esperam, e que Vos dignastes atender misericordiosamente às piedosas lágrimas da B. Mónica para a conversão de seu filho Agostinho, concedei-nos pela intercessão de ambos que choremos os nossos pecados e alcancemos a indulgência da vossa graça. Por nosso Senhor \emph{\&c.}
}\end{paracol}

\paragraphinfo{Epístola}{1. Tm. 5, 3-10}
\begin{paracol}{2}\latim{
Léctio Epístolæ beáti Pauli Apóstoli ad Timótheum.
}\switchcolumn\portugues{
Lição da Ep.ª do B. Ap.º Paulo a Timóteo.
}\switchcolumn*\latim{
\rlettrine{C}{aríssime:} Víduas honóra, quæ vere víduæ sunt. Si qua autem vidua fílios aut nepótes habet, discat primum domum suam régere, et mútuam vicem réddere paréntibus: hoc enim accéptum est coram Deo. Quæ autem vere vidua est et desoláta, speret in Deum, et instet obsecratiónibus et oratiónibus nocte ac die. Nam quæ in delíciis est, vivens mórtua est. Et hoc prǽcipe, ut irreprehensíbiles sint. Si quis autem suórum, et máxime domesticórum curam non habet, fidem negávit, et est infidéli detérior. Vidua eligátur non minus sexagínta annórum, quæ fúerit unius viri uxor, in opéribus bonis testimónium habens, si fílios educávit, si hospítio recépit, si sanctórum pedes lavit, si tribulatiónem patiéntibus
subministrávit, si omne opus bonum subsecúta est.
}\switchcolumn\portugues{
\rlettrine{C}{aríssimo:} Honrai as viúvas que são verdadeiramente viúvas. Se alguma viúva tem filhos ou netos, ensine-os, primeiramente, a governar a sua casa e a retribuir a seus pais, conforme o que haviam recebido deles; porque tal é a vontade de Deus. Aquela viúva, que é verdadeiramente viúva e vive só, espere em Deus e persevere noite e dia em suas súplicas e preces; porém aquela viúva, que vive nas delícias, não está viva, mas sim morta. Fazei-lhes, pois, saber isto, a fim de que sejam irrepreensíveis. Se alguém se não interessa pelos seus, e principalmente pelos de sua casa, nega a fé e é pior do que um infiel. Que a viúva que for escolhida não tenha menos de sessenta anos, nem haja tido mais do que um marido; e que tenha reputação de ter praticado boas obras; educado os seus filhos; praticado a hospitalidade; lavado os pés aos santos; socorrido os aflitos; e, enfim, praticado toda a espécie de boas obras.
}\end{paracol}

\paragraphinfo{Evangelho}{Lc. 7, 11-16}
\begin{paracol}{2}\latim{
\cruz Sequéntia sancti Evangélii secúndum Lucam.
}\switchcolumn\portugues{
\cruz Continuação do santo Evangelho segundo S. Lucas.
}\switchcolumn*\latim{
\blettrine{I}{n} illo témpore: Ibat Jesus in civitátem, quæ vocátur Naïm: et ibant cum eo discípuli ejus et turba copiósa. Cum autem appropinquáret portæ civitátis, ecce, defúnctus efferebátur fílius únicus matris suæ: et hæc vídua erat, et turba civitátis multa cum illa. Quam cum vidísset Dóminus, misericórdia motus super eam, dixit illi: Noli flere. Et accéssit et tétigit lóculum. (Hi autem, qui portábant, stetérunt.) Et ait: Adoléscens, tibi dico, surge. Et resédit, qui erat mórtuus, et cœpit loqui. Et dedit illum matri suæ. Accépit autem omnes timor: et magnificábant Deum, dicéntes: Quia Prophéta magnus surréxit in nobis: et quia Deus visitávit plebem suam.
}\switchcolumn\portugues{
\blettrine{N}{aquele} tempo, dirigiu-se Jesus para uma cidade chamada Naim, sendo acompanhado pelos discípulos e muito povo. Tendo chegado próximo da porta da cidade, viu que levavam um morto daquela terra, filho único de sua mãe, que era viúva e ia acompanhada por muitas pessoas da cidade. Vendo, então, o Senhor tudo isto, encheu-se de compaixão da mãe e disse-lhe: «Não chores». Depois, aproximando-se do defunto, tocou no esquife (pois aqueles que o levavam haviam parado) e disse: «Jovem, ordeno-te Eu, levanta-te!». E no mesmo instante se ergueu e assentou o que estava morto, começando a falar! Então Jesus entregou-o a sua mãe. E toda a multidão ficou aterrada; e glorificavam Deus, dizendo: «Apareceu entre nós um grande Profeta: Deus visitou o seu povo».
}\end{paracol}

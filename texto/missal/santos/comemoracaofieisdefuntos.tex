\subsectioninfo{Comemoração dos Fiéis Defuntos}{2 de Novembro}\label{fieisdefuntos}

\paragraphinfo{Intróito}{4. Esd. 2, 34 \& 35}
\begin{paracol}{2}\latim{
\rlettrine{R}{équiem} ætérnam dona eis, Dómine: et lux perpétua lúceat eis. \emph{Ps. 64, 2-3} Te decet hymnus, Deus, in Sion, et tibi reddétur votum in Jerúsalem: exáudi oratiónem meam, ad te omnis caro véniet.
}\switchcolumn\portugues{
\rlettrine{D}{ai-lhes,} Senhor, o eterno repouso, e que lhes resplandeça a luz perpétua. \emph{Sl. 64, 2-3} A Vós, Senhor, dirigimos estes hinos em Sião: e oferecemos estes votos em Jerusalém: ouvi a minha oração: todas as criaturas devem comparecer ante Vós.
}\end{paracol}

\paragraph{Oração}
\begin{paracol}{2}\latim{
\rlettrine{F}{idélium,} Deus, ómnium Cónditor et Redémptor: animábus famulórum famularúmque tuárum remissiónem cunctórum tríbue peccatórum; ut indulgéntiam, quam semper optavérunt, piis supplicatiónibus consequántur: Qui vivis \emph{\&c.}
}\switchcolumn\portugues{
\slettrine{Ó}{} Deus, Criador e Redentor de todos os fiéis, concedei às almas dos vossos servos a remissão de todos seus pecados, a fim de que pelas nossas humilíssimas orações alcancem da vossa misericórdia o perdão que sempre desejaram. Ó Vós, que, sendo Deus \emph{\&c.}
}\end{paracol}

\paragraphinfo{Epístola}{1. Cor. 15, 51-57}
\begin{paracol}{2}\latim{
Léctio Epístolæ beáti Pauli Apóstoli ad Corínthios.
}\switchcolumn\portugues{
Lição da Ep.ª do B. Ap.º Paulo aos Coríntios.
}\switchcolumn*\latim{
\rlettrine{F}{ratres:} Ecce, mystérium vobis dico: Omnes quidem resurgámus, sed non omnes immutábimur. In moménto, in ictu óculi, in novíssima tuba: canet enim tuba, et mórtui resúrgent incorrúpti: et nos immutábimur. Opórtet enim corruptíbile hoc induere incorruptiónem: et mortále hoc indúere immortalitátem. Cum autem mortále hoc indúerit immortalitátem, tunc fiet sermo, qui scriptus est: Absórpta est mors in victória. Ubi est, mors, victória tua? Ubi est, mors, stímulus tuus? Stímulus autem mortis peccátum est: virtus vero peccáti lex. Deo autem grátias, qui dedit nobis victóriam per Dóminum nostrum Jesum Christum.
}\switchcolumn\portugues{
\rlettrine{E}{is} um mystério que vos revelo: Ressuscitaremos todos certamente; mas não seremos todos mudados. Num instante, num abrir e fechar de olhos, ao som da última trombeta, porque, então a trombeta soará e os mortos ressuscitarão incorruptíveis, e nós seremos transmudados. Pois é preciso que este corpo corruptível se revista da incorruptibilidade, e que este corpo mortal se revista da imortalidade. Quando este corpo corruptível se revestir da incorruptibilidade e este corpo mortal se revestir da imortalidade, então se cumprirá a palavra que está escrita: «A morte foi absorvida pela vitória. Onde está, ó morte, a tua vitória? Onde está, ó morte, o teu aguilhão?». Ora, o aguilhão da morte é o pecado; e a força do pecado é a lei. Graças, pois, sejam rendidas a Deus, que nos concedeu a vitória por Jesus Cristo, Senhor nosso.
}\end{paracol}

\paragraphinfo{Gradual}{4. Esd. 2, 34 et 35}
\begin{paracol}{2}\latim{
\rlettrine{R}{équiem} ætérnam dona eis, Dómine: et lux perpétua lúceat eis. ℣. \emph{Ps. 111, 7} In memória ætérna erit justus: ab auditióne mala non timébit.
}\switchcolumn\portugues{
\rlettrine{D}{ai-lhes,} Senhor, o repouso eterno, e que lhes resplandeça a luz perpétua. ℣. \emph{Sl. 111, 7} A recordação do homem justo permanecerá eternamente; este não temerá ouvir as sentenças más dos homens.
}\end{paracol}

\paragraph{Trato}
\begin{paracol}{2}\latim{
\rlettrine{A}{bsólve,} Dómine, ánimas ómnium fidélium defunctórum ab omni vínculo delictórum. ℣. Et grátia tua illis succurrénte, mereántur evádere judícium ultiónis. ℣. Et lucis ætérnæ beatitúdine pérfrui.
}\switchcolumn\portugues{
\rlettrine{L}{ivrai,} Senhor, as almas dos fiéis defuntos das cadeias dos seus pecados: ℣. E que com o socorro da vossa graça consigam evitar o juízo da vingança: ℣. E gozem a bem-aventurança da luz eterna.
}\end{paracol}

\subsubsection{Sequência}\label{diesirae}
\begin{paracol}{2}\latim{
\rlettrine{D}{ies} iræ, dies illa Solvet sæclum in favílla: Teste David cum Sibýlla.
}\switchcolumn\portugues{
\rlettrine{D}{ia} da ira aquele dia em que o universo será reduzido a cinzas, segundo as profecias de David e Sibila.
}\switchcolumn*\latim{
Quantus tremor est futúrus, Quando judex est ventúrus, Cuncta stricte discussúrus!
}\switchcolumn\portugues{
Qual não será o terror dos homens quando o soberano Juiz vier examinar todas as acções com rigor!
}\switchcolumn*\latim{
Tuba, mirum spargens sonum, Per sepúlcra regiónum, Coget omnes ante thronum.
}\switchcolumn\portugues{
O som estridente da trombeta acordará os mortos, nas profundezas das sepulturas, reunindo-os todos diante do trono do Senhor.
}\switchcolumn*\latim{
Mors stupébit et natúra, Cum resúrget creatúra, Judicánti responsúra.
}\switchcolumn\portugues{
A morte e a natureza ficarão estupefactas quando a criatura comparecer para ser julgada pelo Juiz.
}\switchcolumn*\latim{
Liber scriptus proferétur, In quo totum continétur, Unde mundus judicétur.
}\switchcolumn\portugues{
Um livro aparecerá, onde está escrito tudo sobre o que há-de consistir o julgamento do mundo.
}\switchcolumn*\latim{
Judex ergo cum sedébit, Quidquid latet, apparébit: Nil multum remanébit.
}\switchcolumn\portugues{
Quando o Juiz se assentar no tribunal, tudo o que estiver oculto ficará descoberto, e nenhum crime ficará impune.
}\switchcolumn*\latim{
Quid sum miser tunc dictúrus? Quem patrónum rogatúrus, Cum vix justus sit secúrus?
}\switchcolumn\portugues{
Infeliz de mim! Que poderei dizer então? Que protector procurarei, quando somente o justo está tranquilo?!
}\switchcolumn*\latim{
Rex treméndæ majestátis, Qui salvándos salvas gratis, Salva me, fons pietátis.
}\switchcolumn\portugues{
Ó Rei, cuja majestade é tremenda, mas que salvais, gratuitamente, os escolhidos, salvai-me, ó fonte de piedade!
}\switchcolumn*\latim{
Recordáre, Jesu pie, Quod sum causa tuæ viæ: Ne me perdas illa die.
}\switchcolumn\portugues{
Recordai-Vos, ó piíssimo Jesus, de que vieste ao mundo por minha causa: não me condeneis nesse dia.
}\switchcolumn*\latim{
Quærens me, sedísti lassus: Redemísti Crucem passus: Tantus labor non sit cassus.
}\switchcolumn\portugues{
Ó Vós, que Vos fatigastes em minha procura e que para me resgatardes morrestes na Cruz: não queirais que fiquem infrutíferos tantos esforços.
}\switchcolumn*\latim{
Juste judex ultiónis, Donum fac remissiónis Ante diem ratiónis.
}\switchcolumn\portugues{
Ó justo Juiz, que castigais com justiça, concedei-me o perdão das minhas faltas antes do dia do julgamento.
}\switchcolumn*\latim{
Ingemísco, tamquam reus: Culpa rubet vultus meus: Supplicánti parce, Deus.
}\switchcolumn\portugues{
Eu choro, como réu; as minhas culpas envergonham-me. Ó Deus, que minhas súplicas me alcancem perdão.
}\switchcolumn*\latim{
Qui Maríam absolvísti, Et latrónem exaudísti, Mihi quoque spem dedísti.
}\switchcolumn\portugues{
Ó Vós, que absolvestes Maria e ouvistes o ladrão, e me concedestes também a esperança!
}\switchcolumn*\latim{
Preces meæ non sunt dignæ: Sed tu bonus fac benígne, Ne perénni cremer igne.
}\switchcolumn\portugues{
Bem sei que minhas preces não são dignas; mas Vós, que sois bom, não consintais que eu arda no fogo eterno.
}\switchcolumn*\latim{
Inter oves locum præsta, Et ab hœdis me sequéstra, Státuens in parte dextra.
}\switchcolumn\portugues{
Colocai-me entre os cordeiros, à vossa direita, e separai-me dos pecadores.
}\switchcolumn*\latim{
Confutátis maledíctis, Flammis áctibus addíctis: Voca me cum benedíctis.
}\switchcolumn\portugues{
Livrai-me da confusão e do suplicio dos malditos condenados e introduzi-me junto dos benditos de vosso Pai.
}\switchcolumn*\latim{
Oro supplex et acclínis, Cor contrítum quasi cinis: Gere curam mei finis.
}\switchcolumn\portugues{
Prostrado ante Vós, suplicante, com o coração esmagado, como reduzido a cinzas, Vos imploro, ó da morte.
}\switchcolumn*\latim{
Lacrimósa dies illa, Qua resúrget ex favílla Judicándus homo reus.
}\switchcolumn\portugues{
Dia de lágrimas aquele em que o homem renasça da sua cinza para ser julgado!
}\switchcolumn*\latim{
Huic ergo parce, Deus: Pie Jesu Dómine, Dona eis réquiem.
}\switchcolumn\portugues{
Tende, pois, piedade dele, ó meu Deus! Ó piíssimo Jesus, ó Senhor, concedei-lhe o repouso eterno.
}\switchcolumn*\latim{
℟. Amen.
}\switchcolumn\portugues{
℟. Amen.
}\end{paracol}

\paragraphinfo{Evangelho}{Jo. 5, 25-29}
\begin{paracol}{2}\latim{
\cruz Sequéntia sancti Evangélii secúndum Joánnem.
}\switchcolumn\portugues{
\cruz Continuação do santo Evangelho segundo S. João.
}\switchcolumn*\latim{
\blettrine{I}{n} illo témpore: Dixit Jesus turbis Judæórum: Amen, amen, dico vobis, quia venit hora, et nunc est, quando mórtui áudient vocem Fílii Dei: et qui audíerint, vivent. Sicut enim Pater habet vitam in semetípso, sic dedit et Fílio habére vitam in semetípso: et potestátem dedit ei judícium fácere, quia Fílius hóminis est. Nolíte mirári hoc, quia venit hora, in qua omnes, qui in monuméntis sunt, áudient vocem Fílii Dei: et procédent, qui bona fecérunt, in resurrectiónem vitæ: qui vero mala egérunt, in resurrectiónem judícii.
}\switchcolumn\portugues{
\blettrine{N}{aquele} tempo, disse Jesus às turbas dos judeus: «Em verdade, em verdade vos digo: A hora chega (ela chegou já) em que os mortos ouvirão a voz do Filho de Deus; e os que a ouvirem, viverão. Porque, assim como o Pai tem a vida em si, assim Ele deu ao Filho o poder de ter a vida em si; e deu-lhe, também, o poder de julgar, pois Ele é o Filho do homem. Não vos admireis disto, pois vem a hora em que todos aqueles que estão nos túmulos ouvirão a voz do Filho de Deus. Então, aqueles que tiverem praticado obras boas sairão para a ressurreição da vida eterna; e aqueles que tiverem praticado obras más ressuscitarão para a condenação».
}\end{paracol}

\paragraph{Ofertório}
\begin{paracol}{2}\latim{
\rlettrine{D}{ómine} Jesu Christe, Rex glóriæ, líbera ánimas ómnium fidélium defunctórum de pœnis inférni et de profúndo lacu: líbera eas de ore leónis, ne absórbeat eas tártarus, ne cadant in obscúrum: sed sígnifer sanctus Míchaël repræséntet eas in lucem sanctam: Quam olim Abrahæ promisísti et sémini ejus. ℣. Hóstias et preces tibi, Dómine, laudis offérimus: tu súscipe pro animábus illis, quarum hódie memóriam fácimus: fac eas, Dómine, de morte transíre ad vitam. * Quam olim Abrahæ promisísti et sémini ejus.
}\switchcolumn\portugues{
\rlettrine{S}{enhor} Jesus Cristo, Rei da glória, livrai as almas de todos os fiéis defuntos das penas do inferno e do lago profundo; livrai-as da boca do leão; que o inferno as não sepulte, nem elas se abismem nas trevas desse lugar tremendo; mas que S. Miguel, que é o porta-estandarte divino, as conduza até à luz santa. Como outrora prometestes a Abraão e à sua posteridade. ℣. Vos oferecemos, Senhor, estas hóstias e estas orações de louvor: aceitai-as pelas almas daqueles que hoje comemoramos, e fazei-as passar da morte à vida: Como outrora prometestes a Abraão e à sua posteridade.
}\end{paracol}

\paragraph{Secreta}
\begin{paracol}{2}\latim{
\rlettrine{H}{óstias,} quǽsumus, Dómine, quas tibi pro animábus famulórum famularúmque tuárum offérimus, propitiátus inténde: ut, quibus fídei christiánæ méritum contulísti, dones et prǽmium. Per Dóminum \emph{\&c.}
}\switchcolumn\portugues{
\rlettrine{O}{lhai} benigno, Senhor, Vos suplicamos, para as hóstias que Vos oferecemos pelas almas dos vossos servos, a fim de que, depois de lhes haverdes concedido o dom da fé cristã, lhes proporcioneis a recompensa. Por nosso Senhor \emph{\&c.}
}\end{paracol}

\paragraphinfo{Comúnio}{4. Esd. 2, 35 \& 34}
\begin{paracol}{2}\latim{
\rlettrine{L}{ux} ætérna lúceat eis, Dómine: Cum Sanctis tuis in ætérnum: quia pius es. ℣. Requiem ætérnam dona eis, Dómine: et lux perpétua lúceat eis. Cum Sanctis tuis in ætérnum: quia pius es.
}\switchcolumn\portugues{
\qlettrine{Q}{ue} a luz eterna lhes resplandeça: Com os vossos Santos em todos os séculos, ó Senhor, pois sois bom. ℣. Dai-lhes, Senhor, o eterno repouso, e que a luz perpétua lhes resplandeça: Com os vossos santos em todos os séculos, ó Senhor, pois sois misericordioso.
}\end{paracol}

\paragraph{Postcomúnio}
\begin{paracol}{2}\latim{
\rlettrine{A}{nimábus,} quǽsumus, Dómine, famulórum famularúmque tuárum orátio profíciat supplicántium: ut eas et a peccátis ómnibus éxuas, et tuæ redemptiónis fácias esse partícipes: Qui vivis \emph{\&c.}
}\switchcolumn\portugues{
\qlettrine{Q}{ue} as nossas humildes preces sejam proveitosas às almas dos vossos servos e servas, a fim de que, soltas das cadeias dos seus pecados, participem dos frutos da vossa redenção. Por nosso Senhor \emph{\&c.}
}\end{paracol}

\textit{2.ª Missa, como a Missa do Dia do Aniversário, página \pageref{diadoaniversario}.}

\textit{3.ª Missa, como a Missa Quotidiana, página \pageref{quotidiana}, omitindo-se a 1.ª e 3.ª Oração, Secreta e Postcomúnio.}
\subsectioninfo{Santa Ana}{26 de Julho}

\paragraph{Intróito}
\begin{paracol}{2}\latim{
\rlettrine{G}{audeámus} omnes in Dómino, diem festum celebrántes sub honóre beátæ Annæ: de cujus sollemnitáte gaudent Angeli et colláudant Fílium Dei. \emph{Ps. 44, 2} Eructávit cor meum verbum bonum: dico ego ópera mea Regi.
℣. Gloria Patri \emph{\&c.}
}\switchcolumn\portugues{
\rlettrine{A}{legremo-nos} todos no Senhor, celebrando neste dia a festa em honra da B. Ana; pois os Anjos rejubilam com esta festividade e em harmonia louvam o Filho de Deus. \emph{Sl. 44, 2} Meu coração exalou uma palavra excelente: Consagro ao Rei as minhas obras.
℣. Glória ao Pai \emph{\&c.}
}\end{paracol}

\paragraph{Oração}
\begin{paracol}{2}\latim{
\rlettrine{D}{eus,} qui beátæ Annæ grátiam conférre dignatus es, ut Genetrícis unigéniti Fílii tui mater effici mererétur: concéde propítius; ut, cujus sollémnia celebrámus, ejus apud te patrocíniis adjuvémur. Per eúndem Dóminum \emph{\&c.}
}\switchcolumn\portugues{
\slettrine{Ó}{} Deus, que Vos dignastes conferir à B. Ana a graça de ser escolhida para dar ao mundo a Mãe do vosso Filho Unigénito, concedei-nos propício que sejamos auxiliados junto de Vós pelo patrocínio daquela cuja festa celebramos. Pelo mesmo nosso Senhor \emph{\&c.}
}\end{paracol}

\paragraphinfo{Epístola}{Página \pageref{nemvirgensnemmartires}}

\paragraphinfo{Gradual}{Página \pageref{mariamadalena}}

\paragraphinfo{Evangelho}{Página \pageref{martiresnaovirgens}}

\paragraphinfo{Ofertório}{Sl. 44, 10}
\begin{paracol}{2}\latim{
\rlettrine{F}{íliæ} regum in honóre tuo, ástitit regína a dextris tuis in vestítu deauráto, circúmdata varietáte.
}\switchcolumn\portugues{
\rlettrine{A}{s} filhas dos reis formam a vossa corte de glória: a própria rainha está colocada à vossa direita, envergando um vestido de ouro, recamado da mais rica variedade.
}\end{paracol}

\paragraph{Secreta}
\begin{paracol}{2}\latim{
\rlettrine{S}{acrifíciis} præséntibus, quǽsumus, Dómine, placatus inténde: ut per intercessiónem beátæ Annæ, quæ Genetrícis Fílii tui, Dómini nostri Jesu Christi, mater éxstitit, et devotióni nostræ profíciant et salúti. Per eúndem Dóminum \emph{\&c.}
}\switchcolumn\portugues{
\rlettrine{O}{lhai} propício, Senhor, Vos suplicamos, para estes sacrifícios, a fim de que, pela intercessão da B. Ana, que foi Mãe daquela que deu ao mundo vosso Filho, nosso Senhor Jesus Cristo, sejam proveitosos à nossa piedade e salvação. Pelo mesmo nosso Senhor \emph{\&c.}
}\end{paracol}

\paragraphinfo{Comúnio}{Sl. 44, 3}
\begin{paracol}{2}\latim{
\rlettrine{D}{iffúsa} est grátia in lábiis tuis: proptérea benedíxit te Deus in ætérnum, et in sǽculum sǽculi.
}\switchcolumn\portugues{
\rlettrine{A}{} graça espalhou-se nos vossos lábios: eis porque Deus vos abençoou para a eternidade e para todos os séculos.
}\end{paracol}

\paragraph{Postcomúnio}
\begin{paracol}{2}\latim{
\rlettrine{C}{œléstibus} sacraméntis vegetáti, quǽsumus, Dómine, Deus noster: ut, intercessióne beátæ Annæ, quam Genetrícis Fílii tui matrem esse voluísti, ad ætérnam salútem perveníre mereámur. Per eúndem Dóminum \emph{\&c.}
}\switchcolumn\portugues{
\rlettrine{A}{lentados} com os celestiais sacramentos, Vos suplicamos, ó Senhor, nosso Deus, permiti pela intercessão da B. Ana, que foi Mãe daquela que deu ao mundo vosso Filho, que possamos alcançar a salvação eterna. Por nosso Senhor \emph{\&c.}
}\end{paracol}

\subsectioninfo{S. João de Brito, Mártir}{4 de Fevereiro}

\textit{Como na Missa Lætábitur justus, página \pageref{martirnaopontifice2}, excepto:}

\paragraph{Oração}
\begin{paracol}{2}\latim{
\rlettrine{D}{eus} qui ad fidem cathólicam apud Indos propagándam beátam Joánnem Mártyrem tuum invícta constántia roborásti: ipsíus méritis et intercessióne concéde; ut, qui triúmphi ejus memóriam recólimus, étiam fídei exémpla imitémur. Per Dóminum \emph{\&c.}
}\switchcolumn\portugues{
\slettrine{Ó}{} Deus, que para a propagação da fé católica entre os Índios fortalecestes o B. João, vosso Mártir, com uma constância invencível, fazei, pelos seus merecimentos e intercessão; que, celebrando nós a memória dos seus triunfos, imitemos também os exemplos da sua fé. Por nosso Senhor \emph{\&c.}
}\end{paracol}

\paragraphinfo{Oração}{S. André Corsino}
\begin{paracol}{2}\latim{
\rlettrine{D}{eus,} qui in Ecclésia tua nova semper instáuras exémpla virtútum: da pópulo tuo beáti Andréæ Confessóris tui atque Pontíficis ita sequi vestígia; ut assequátur et prǽmia. Per Dóminum \emph{\&c.}
}\switchcolumn\portugues{
\slettrine{Ó}{} Deus, que na vossa Igreja apresentais constantemente novos exemplos de virtudes, concedei ao vosso povo que de tal modo siga os vestígios do B. André, vosso Confessor e Pontífice, que possa alcançar o mesmo prémio. Por nosso Senhor \emph{\&c.}
}\end{paracol}

\paragraphinfo{Epístola}{2 Cor. 11, 19-33; 12, 1-9}
\begin{paracol}{2}\latim{
Léctio Epístolæ beáti Pauli Apóstoli ad Corínthios.
}\switchcolumn\portugues{
Lição da Ep.ª do B. Ap.º Paulo aos Coríntios.
}\switchcolumn*\latim{
\rlettrine{P}{atres:} Libénter suffértis insipiéntens: cum sitis ipsi sapiéntes. Sustinétis enim, si quis vos in servitútem rédigit, si quis dévorat, si quis áccipit, si quis extóllitur, si quis in fáciem vos cædit. Secúndum ignobilitátem dico, quasi nos infírmi fuérimus in hac parte. In quo quis audet, (in insipiéntia dico) áudeo et ego: Hebrǽi sunt, et ego: Israelítæ sunt, et ego: Semen Abrahæ sunt, et ego: Minístri Christi sunt, (ut minus sápiens dico) plus ego: in labóribus plúrimis, in carcéribus abundántius, in plagis supra modum, in mórtibus frequénter. A Judǽis quínquies quadragénas, una minus, accépi. Ter virgis cæsus sum, semel lapidátus sum, ter naufrágium feci, nocte et die in profúndo maris fui: in itinéribus sæpe, perículis flúminum, perículis latrónum, perículis ex génere, perículis ex géntibus, perículis in civitáte, perículis in solitúdine, perículis in mari, perículis in falsis frátribus: in labóre et ærúmna, in vigíliis multis, in fame et siti, in jejúniis multis, in frigóre et nuditáte: præter illa, quæ extrínsecus sunt, instántia mea cotidiána, sollicitúdo ómnium Ecclesiárum.
}\switchcolumn\portugues{
\rlettrine{M}{eus} irmãos: Como homens sensatos que sois, generosamente suportais os insensatos. E suportais, também, se vos sujeitam à escravidão, se vos devoram, se vos roubam, se vos tratam com arrogância, ou se vos esbofeteiam. Digo-o com vergonha, como se neste ponto houvéssemos sido fracos! Contudo, quem quer que ouse vangloriar-se (falo como se fora insensato), também eu me vanglorio. Eles são hebreus? Também eu. São israelitas? Também eu. São descendentes de Abraão? Também eu. São ministros de Cristo? Muito mais (falo insensatamente) sou eu do que eles: pelos meus muitos trabalhos, mais do que os deles; pelas minhas frequentes prisões, mais do que as deles; pelas pancadas sem conta que sofri, mais do que as deles; e até, frequentemente, tendo quase visto a morte. Dos judeus recebi chicotadas, em cinco quarenta vezes menos uma; três vezes fui açoitado com varas; uma vez fui apedrejado; três vezes naufraguei: passei um dia e uma noite no fundo do mar! Em minhas contínuas viagens encontrei sempre perigos: perigos nas águas, perigos nos ladrões, perigos nos meus compatriotas, perigos nos pagãos, perigos nas cidades, perigos nos desertos, perigos no mar, perigos nos irmãos falsos, nos trabalhos, nas fadigas, nas numerosas vigílias, na fome, na sede, nos muitos jejuns, no frio e na nudez! E, além destes males, que são exteriores, preocupa-me também quotidianamente a solicitude de todas as cristandades.
}\end{paracol}

\paragraphinfo{Gradual}{Sl. 111, 1-2}
\begin{paracol}{2}\latim{
\rlettrine{B}{eátus} vir, qui timet Dóminum: in mandátis ejus cupit nimis. ℣. Potens in terra erit semen ejus: generátio rectórum benedicétur.
}\switchcolumn\portugues{
\rlettrine{B}{em-aventurado} o varão que teme o Senhor e que põe todo seu zelo em obedecer-Lhe. ℣. Sua descendência será poderosa na terra; pois a geração dos justos será abençoada.
}\switchcolumn*\latim{
Allelúja, allelúja. ℣. \emph{Ps. 20, 4} Posuísti, Dómine, super caput ejus corónam de lápide pretióso. Allelúja.
}\switchcolumn\portugues{
Aleluia, aleluia. ℣. \emph{Sl. 20, 4} Senhor, impusestes na sua cabeça uma coroa de pedras preciosas. Aleluia.
}\end{paracol}

\paragraph{Secreta}
\begin{paracol}{2}\latim{
\rlettrine{S}{úscipe,} Dómine, múnera, dignánter obláta; et beáti Mártyris tui Joánnis suffragántibus méritis concéde: ut passióni et morti Unigéniti Fílii tui configuráti, resurrectiónis quoque et glóriæ consórtes éffici merámur. Qui tecum vivit \emph{\&c.}
}\switchcolumn\portugues{
\rlettrine{R}{ecebei,} Senhor, os dons que humildemente Vos oferecemos; e, tendo em atenção os méritos do B. Mártir João e havendo-nos conformado com a paixão e morte do vosso Filho Unigénito, permiti que mereçamos também comparticipar da sua ressurreição e glória. Ele, que sendo Deus \emph{\&c.}
}\end{paracol}

\paragraphinfo{Secreta}{S. André Corsino}
\begin{paracol}{2}\latim{
\rlettrine{S}{ancti} tui, quǽsumus, Dómine, nos úbique lætíficent: ut, dum eórum mérita recólimus, patrocínia sentiámus. Per Dóminum \emph{\&c.}
}\switchcolumn\portugues{
\qlettrine{Q}{ue} os vossos Santos, Senhor, Vos suplicamos, nos alegrem em toda a parte, a fim de que, honrando os seus méritos, sintamos o efeito do seu patrocínio. Por nosso Senhor \emph{\&c.}
}\end{paracol}

\paragraph{Postcomúnio}
\begin{paracol}{2}\latim{
\rlettrine{R}{edemptiónis} humánæ pígnore sacro per hæc mystéria refécti te, Dómine, súpplices exorámus: ut, qui peccatórum nostrórum póndere prémimur, beáto Joánne Mártyre tuo intercedénte, véniam consequámur et pacem. Per Dóminum \emph{\&c.}
}\switchcolumn\portugues{
\rlettrine{H}{avendo} sido alimentados nestes sagrados mystérios com o sacrossanto penhor da redenção humana, humildemente Vos imploramos, por intercessão do B. João, vosso Mártir, que sejamos livres do peso dos nossos pecados e alcancemos o perdão e a paz. Por nosso Senhor \emph{\&c.}
}\end{paracol}

\paragraphinfo{Postcomúnio}{S. André Corsino}
\begin{paracol}{2}\latim{
\rlettrine{P}{ræsta,} quǽsumus, omnípotens Deus: ut, de percéptis munéribus grátias exhibéntes, intercedénte beáto Andréa Confessóre tuo atque Pontífice, benefícia potióra sumámus. Per Dóminum \emph{\&c.}
}\switchcolumn\portugues{
\rlettrine{D}{ignai-Vos} permitir, ó Deus omnipotente, que, dando-Vos nós graças pelo benefícios recebidos, alcancemos por intercessão do B. André, vosso Confessor e Pontífice, ainda outros maiores. Por nosso Senhor \emph{\&c.}
}\end{paracol}

\subsectioninfo{S. Francisco Xavier, Confessor}{3 de Dezembro}\label{franciscoxavier}

\paragraphinfo{Intróito}{Sl. 118, 46-47}
\begin{paracol}{2}\latim{
\rlettrine{L}{oquébar} de testimóniis tuis in conspéctu regum, et non confundébar: et meditábar in mandátis tuis, quæ diléxi nimis. \emph{Ps. 110, 1-2} Laudáte Dóminum, omnes gentes, laudáte eum, omnes pópuli: quóniam confirmáta est super nos misericórdia ejus, et véritas Dómini manet in ætérnum.
℣. Gloria Patri \emph{\&c.}
}\switchcolumn\portugues{
\rlettrine{F}{alava} dos vossos testemunhos sem vergonha na presença dos reis: e meditava nos vossos mandamentos, que amava profundamente. \emph{Sl. 110, 1-2} Louvai o Senhor, ó vós, todas as nações; louvai-O, ó vós, todos os povos; pois a sua misericórdia para connosco manifestou-se claramente e a verdade do Senhor permanece eternamente.
℣. Glória ao Pai \emph{\&c.}
}\end{paracol}

\paragraph{Oração}
\begin{paracol}{2}\latim{
\rlettrine{D}{eus,} qui Indiárum gentes beáti Francísci prædicatióne et miráculis Ecclésiæ tuæ aggregáre voluísti: concéde propítius; ut, cujus gloriósa mérita venerámur, virtútum quoque imitémur exémpla. Per Dóminum \emph{\&c.}
}\switchcolumn\portugues{
\slettrine{Ó}{} Deus, que pela pregação e milagres do B. Francisco quisestes chamar à vossa Igreja os povos das Índias, concedei-nos propício que, venerando os seus gloriosos méritos, imitemos, também, os exemplos das suas virtudes. Por nosso Senhor \emph{\&c.}
}\end{paracol}

\paragraphinfo{Epístola}{Página \pageref{andreapostolo}}

\paragraphinfo{Gradual}{Sl. 91, 13 \& 14}
\begin{paracol}{2}\latim{
\qlettrine{J}{ustus} ut palma florébit: sicut cedrus Líbani multiplicábitur in domo Dómini. ℣. \emph{ibid., 3} Ad annuntiándum mane misericórdiam tuam, et veritátem tuam per noctem.
}\switchcolumn\portugues{
\rlettrine{O}{} justo florescerá, como a palmeira, e crescerá, como o cedro do Líbano, na casa do Senhor. ℣. \emph{ibid., 3} Para publicar de manhã a vossa misericórdia; e de noite a vossa verdade.
}\switchcolumn*\latim{
Allelúja, allelúja. ℣. \emph{Jac. 1, 12} Beátus vir, qui suffert tentatiónem: quóniam, cum probátus fúerit, accípiet corónam vitæ. Allelúja.
}\switchcolumn\portugues{
Aleluia, aleluia. ℣. \emph{Tg. 1, 12} Bem-aventurado o varão que sabe sofrer a tentação, porque, quando acabar a tentação, receberá a coroa da vida. Aleluia.
}\end{paracol}

\paragraphinfo{Evangelho}{Mc. 16, 15-18}
\begin{paracol}{2}\latim{
\cruz Sequéntia sancti Evangélii secúndum Marcum.
}\switchcolumn\portugues{
\cruz Continuação do santo Evangelho segundo S. Marcos.
}\switchcolumn*\latim{
\blettrine{I}{n} illo témpore: Dixit Jesus discípulis suis: Eúntes in mundum univérsum, prædicáte Evangélium omni creatúra. Qui credíderit, et baptizátus fúerit, salvus erit: qui vero non credíderit, condemnábitur. Signa autem eos, qui credíderint, hæc sequántur: In nómine meo dæmónia ejícient: linguis loquántur novis: serpéntes tollent: et si mortíferum quid bíberint, non eis nocébit: super ægros manus impónent, et bene habébunt.
}\switchcolumn\portugues{
\blettrine{N}{aquele} tempo, disse Jesus aos seus discípulos: «Ide pelo mundo inteiro e pregai o Evangelho a todas as criaturas. Quem acreditar e for baptizado será salvo; quem não acreditar será condenado. Eis os milagres que acompanharão aqueles que acreditarem: Em meu nome expulsarão os demónios; falarão novas línguas; tirarão com suas serpentes; se beberem alguma cousa mortífera, lhes não fará dano; e imporão as mãos sobre os enfermos, que serão curados».
}\end{paracol}

\paragraphinfo{Ofertório}{Sl. 88, 25}
\begin{paracol}{2}\latim{
\rlettrine{V}{éritas} mea et misericórdia mea cum ipso: et in nómine meo exaltábitur cornu ejus.
}\switchcolumn\portugues{
\rlettrine{A}{} minha verdade e a minha misericórdia estarão com ele, e, por virtude do meu nome, será exaltado o seu poder.
}\end{paracol}

\paragraph{Secreta}
\begin{paracol}{2}\latim{
\rlettrine{P}{ræsta} nobis, quǽsumus, omnípotens Deus: ut nostræ humilitátis oblátio, et pro tuórum tibi grata sit honóre Sanctórum, et nos córpore páriter et mente puríficet. Per Dóminum \emph{\&c.}
}\switchcolumn\portugues{
\rlettrine{C}{oncedei-nos,} ó Deus omnipotente, que esta oferta da nossa humildade, servindo para honrar os vossos Santos, Vos seja agradável; e que ao mesmo tempo nos purifique o corpo e a alma. Por nosso Senhor \emph{\&c.}
}\end{paracol}

\paragraphinfo{Comúnio}{Mt. 24,46-47}
\begin{paracol}{2}\latim{
\rlettrine{B}{eátus} servus, quem, cum vénerit dóminus, invénerit vigilántem: amen, dico vobis, super ómnia bona sua constítuet eum.
}\switchcolumn\portugues{
\rlettrine{B}{em-aventurado} o servo que o Senhor, quando vier, achar vigilante. Em verdade vos digo que lhe dará a administração de todos seus bens.
}\end{paracol}

\paragraph{Postcomúnio}
\begin{paracol}{2}\latim{
\qlettrine{Q}{uǽsumus,} omnípotens Deus: ut, qui cœléstia aliménta percépimus, intercedénte beáto Francísco Confessóre tuo, per hæc contra ómnia advérsa muniámur. Per Dóminum \emph{\&c.}
}\switchcolumn\portugues{
\rlettrine{H}{avendo} nós, ó Deus omnipotente, recebido o alimento celestial, permiti, Vos suplicamos, que, pela intercessão do B. Francisco, vosso Confessor, sejamos fortalecidos contra todas as adversidades. Por nosso Senhor \emph{\&c.}
}\end{paracol}

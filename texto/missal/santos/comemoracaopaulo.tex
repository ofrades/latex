\subsectioninfo{Comemoração de S. Paulo }{30 de Junho}\label{comemoracaopaulo}

\paragraphinfo{Intróito}{Página \pageref{conversaopaulo}}

\paragraph{Oração}
\begin{paracol}{2}\latim{
\rlettrine{D}{eus,} qui multitúdinem géntium beáti Pauli Apóstoli prædicatióne docuísti: da nobis, quǽsumus; ut, cujus natalítia (commemoratiónem) cólimus, ejus apud te patrocínia sentiámus. Per Dóminum \emph{\&c.}
}\switchcolumn\portugues{
\slettrine{Ó}{} Deus, que instruístes a multidão dos povos com a pregação do B. Apóstolo Paulo, concedei-nos, Vos suplicamos, que, celebrando o seu natalício junto de Vós, sintamos também junto de Vós o seu patrocínio. Por nosso Senhor \emph{\&c.}
}\end{paracol}

\subsectioninfo{Comemoração de S. Pedro}{Página \pageref{cadeirapedro}}

\paragraphinfo{Epístola}{Gl. 1, 11-20}
\begin{paracol}{2}\latim{
Léctio Epístolæ beáti Pauli Apóstoli ad Gálatas.
}\switchcolumn\portugues{
Lição da Ep.ª do B. Ap.º Paulo aos Gálatas.
}\switchcolumn*\latim{
\rlettrine{F}{ratres:} Notum vobis facio Evangélium, quod evangelizátum est a me, quia non est secúndum hóminem: neque enim ego ab hómine accépi illud neque dídici, sed per revelatiónem Jesu Christi. Audístis enim conversatiónem meam aliquándo in Judaísmo: quóniam supra modum persequébar Ecclésiam Dei, et expugnábam illam, et proficiébam in Judaísmo supra multos coætáneos meos in génere meo, abundántius æmulátor exsístens paternárum mearum traditiónum. Cum autem plácuit ei, qui me segregávit ex útero matris meæ, et vocávit per grátiam suam, ut reveláret Fílium suum in me, ut evangelizárem illum in géntibus: contínuo non acquiévi carni et sánguini, neque veni Jerosólymam ad antecessóres meos Apóstolos: sed ábii in Arábiam: et íterum revérsus sum Damáscum: déinde post annos tres veni Jerosólymam vidére Petrum, et mansi apud eum diébus quíndecim: álium autem Apostolórum vidi néminem, nisi Jacóbum fratrem Dómini. Quæ autem scribo vobis, ecce coram Deo, quia non méntior.
}\switchcolumn\portugues{
\rlettrine{M}{eus} irmãos: «Eu vos declaro que o Evangelho que vos tenho pregado não tem nada de humano, pois não foi de algum homem que o recebi ou aprendi, mas por revelação de Jesus Cristo. Sabeis, com efeito, como eu vivia outrora no judaísmo; como persegui com furor a Igreja de Deus; e como a devastei, distinguindo-me no judaísmo mais do que muitos meus contemporâneos, no seio da minha nação, mostrando um zelo excessivo pelas tradições dos meus pais. Mas, quando aprouve Àquele que me escolheu no seio de minha mãe, e me chamou pela sua graça para me revelar seu Filho, a fim de que O anunciasse entre as nações logo, sem antes tomar conselho com a carne ou com o sangue e sem voltar a Jerusalém junto daqueles que antes de mim eram Apóstolos, parti para a Arábia e voltei ainda a Damasco. Em seguida, decorridos três anos, fui a Jerusalém para ver Pedro, ali permanecendo com ele durante quinze dias. Eu não vi nenhum Outro Apóstolo, excepto Tiago, o irmão do Senhor. Tomo a Deus como testemunha de que não minto em tudo quanto vos escrevo».
}\end{paracol}

\paragraphinfo{Gradual}{Gl. 2, 8-9}
\begin{paracol}{2}\latim{
\qlettrine{Q}{ui} operátus est Petro in apostolátum, operátus est et mihi inter gentes: et cognovérunt grátiam Dei, quæ data est mihi. ℣. \emph{1. Cor. 15, 10} Grátia Dei in me vácua non fuit: sed grátia ejus semper in me manet.
}\switchcolumn\portugues{
\rlettrine{A}{quele} que instituiu Pedro Apóstolo para o apostolado dos circuncisos, instituiu-me também a mim Apóstolo dos gentios: ℣. \emph{1. Cor. 15, 10} E eles conheceram a graça de Deus que me foi dada. A graça de Deus não ficou estéril em mim.
}\switchcolumn*\latim{
Allelúja, allelúja. ℣. Sancte Paule Apóstole, prædicátor veritátis et doctor géntium, intercéde pro nobis. Allelúja.
}\switchcolumn\portugues{
Aleluia, aleluia. ℣. S. Paulo, pregador da verdade e doutor dos povos, intercedei por nós. Aleluia.
}\end{paracol}

\paragraphinfo{Evangelho}{Mt. 10, 16-22}
\begin{paracol}{2}\latim{
\cruz Sequéntia sancti Evangélii secúndum Matthǽum.
}\switchcolumn\portugues{
\cruz Continuação do santo Evangelho segundo S. Mateus.
}\switchcolumn*\latim{
\blettrine{I}{n} illo témpore: Dixit Jesus discípulis suis: Ecce, ego mitto vos sicut oves in médio lupórum. Estóte ergo prudéntes sicut serpentes, et símplices sicut colúmbæ. Cavéte autem ab homínibus. Tradent enim vos in concíliis, et in synagógis suis flagellábunt vos: et ad prǽsides et ad reges ducémini propter me in testimónium illis et géntibus. Cum autem tradent vos, nolíte cogitáre, quómodo aut quid loquámini: dábitur enim vobis in illa hora, quid loquámini. Non enim vos estis, qui loquímini, sed Spíritus Patris vestri, qui lóquitur in vobis. Tradet autem frater fratrem in mortem, et pater fílium: et insúrgent fílii in paréntes, et morte eos affícient: et éritis odio ómnibus propter nomen meum: qui autem perseveráverit usque in finem, hic salvus erit.
}\switchcolumn\portugues{
\blettrine{N}{aquele} tempo, disse Jesus aos seus discípulos: «Eis que vos envio, como ovelhas no meio de lobos. Sede, pois, prudentes, como as serpentes, e simples, como as pombas. Acautelai-vos, portanto, com os homens; porque vos entregarão aos tribunais e vos flagelarão nas suas sinagogas. Sereis conduzidos por amor de mim à presença dos governadores e dos reis para dardes testemunho de mim, diante deles e dos povos. Quando vos entregarem, não penseis de que maneira lhes haveis de falar e que palavras lhes devereis dizer, porque naquela mesma hora vos será transmitido o que houverdes de responder; pois não sois vós que falareis: o Espírito do vosso Pai é que falará em vós. E o irmão entregará à morte o irmão; o pai entregará o filho; e os filhos se levantarão contra os pais e lhes darão a morte. Sereis odiados por todos por causa do meu nome, mas quem perseverar até ao fim será salvo».
}\end{paracol}

\paragraphinfo{Ofertório}{Sl. 138, 17}
\begin{paracol}{2}\latim{
\rlettrine{M}{ihi} autem nimis honoráti sunt amíci tui, Deus: nimis confortátus est principátus eórum.
}\switchcolumn\portugues{
\rlettrine{V}{ejo,} ó meu Deus, que honrais de um modo singular os vossos amigos: o seu poder firmou-se extraordinariamente.
}\end{paracol}

\paragraph{Secreta}
\begin{paracol}{2}\latim{
\rlettrine{A}{póstoli} tui Pauli précibus, Dómine, plebis tuæ dona sanctífica: ut, quæ tibi tuo grata sunt institúto, gratióra fiant patrocínio supplicántis. Per Dóminum \emph{\&c.}
}\switchcolumn\portugues{
\rlettrine{S}{antificai,} Senhor, pelas preces do vosso Apóstolo Paulo as ofertas do vosso povo, a fim de que, sendo-Vos elas já agradáveis, porque as instituístes, mais agradáveis Vos sejam ainda pelo patrocínio do suplicante. Por nosso Senhor \emph{\&c.}
}\end{paracol}

\paragraphinfo{Comúnio}{Mt. 19, 28 \& 29}
\begin{paracol}{2}\latim{
\rlettrine{A}{men,} dico vobis: quod vos, qui reliquístis ómnia et secúti estis me, céntuplum accipiétis et vitam ætérnam possidébitis.
}\switchcolumn\portugues{
\rlettrine{E}{m} verdade vos digo: Vós, que abandonastes tudo e me seguistes, recebereis o cêntuplo e possuireis a vida eterna.
}\end{paracol}

\paragraph{Postcomúnio}
\begin{paracol}{2}\latim{
\rlettrine{P}{ercéptis,} Dómine, sacraméntis: beáto Paulo Apóstolo tuo interveniénte, deprecámur; ut, quæ pro illíus celebráta sunt glória, nobis profíciant ad medélam. Per Dóminum \emph{\&c.}
}\switchcolumn\portugues{
\rlettrine{H}{avendo} recebido estes sacramentos, Senhor, Vos pedimos, permiti pela intercessão do B. Paulo, vosso Apóstolo, que este sacrifício, que foi oferecido em vossa honra, nos sirva de remédio. Por nosso Senhor \emph{\&c.}
}\end{paracol}

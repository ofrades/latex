\subsectioninfo{S. Tomé, Apóstolo}{21 de Dezembro}

\paragraphinfo{Intróito}{Sl. 138, 17}
\begin{paracol}{2}\latim{
\rlettrine{M}{ihi} autem nimis honoráti sunt amíci tui, Deus: nimis confórtátus est principatus eórum. \emph{Ps. ibid., 1-2} Dómine, probásti me et cognovísti me: tu cognovísti sessiónem meam et resurrectiónem meam.
℣. Gloria Patri \emph{\&c.}
}\switchcolumn\portugues{
\rlettrine{E}{u} vejo, ó Deus, que honrastes largamente os vossos amigos: e que seu poder se fortaleceu extraordinariamente. \emph{Sl. ibid., 1-2} Senhor, perscrutastes o meu íntimo e ficastes conhecendo-me: ficastes conhecendo quando me deito e quando me levanto.
℣. Glória ao Pai \emph{\&c.}
}\end{paracol}

\paragraph{Oração}
\begin{paracol}{2}\latim{
\rlettrine{D}{a} nobis, quǽsumus, Dómine, beáti Apóstoli tui Thomæ sollemnitátibus gloriári: ut ejus semper et patrocíniis sublevémur; et fidem cóngrua devotióne sectémur. Per Dóminum \emph{\&c.}
}\switchcolumn\portugues{
\rlettrine{C}{oncedei-nos,} Senhor, Vos suplicamos, que nos alegremos nas festividades do vosso B. Ap.º Tomé, a fim de que sejamos sempre amparados com seu patrocínio e imitemos a sua fé com a devida devoção. Por nosso Senhor \emph{\&c.}
}\end{paracol}

\paragraphinfo{Epístola}{Ef. 2, 19-22}
\begin{paracol}{2}\latim{
Léctio Epístolæ beáti Pauli Apóstoli ad Ephésios.
}\switchcolumn\portugues{
Lição da Ep.ª do B. Ap.º Paulo aos Efésios.
}\switchcolumn*\latim{
\rlettrine{F}{ratres:} Jam non estis hóspites et ádvenæ: sed estis cives sanctórum et doméstici Dei: superædificáti super fundaméntum Apostolórum et Prophetárum, ipso summo angulári lápide Christo Jesu: in quo omnis ædificátio constrúcta crescit in templum sanctum in Dómino, in quo et vos coædificámini in habitáculum Dei in Spíritu.
}\switchcolumn\portugues{
\rlettrine{M}{eus} irmãos: Já não sois estrangeiros, nem hóspedes, mas concidadãos dos santos e da família de Deus, instituídos sobre o fundamento dos Apóstolos e dos Profetas, de que o próprio Jesus Cristo é a pedra angular e em quem todo o edifício bem construído se deve elevar para formar um templo santo no Senhor. É n’Ele que também sois edificados para vos tornardes pelo Espírito Santo em morada de Deus.
}\end{paracol}

\paragraphinfo{Gradual}{Sl. 138, 17-18}
\begin{paracol}{2}\latim{
\rlettrine{N}{imis} honorati sunt amíci tui, Deus: nimis confortátus est principátus eórum. ℣. Dinumerábo eos, et super arénam multiplicabúntur.
}\switchcolumn\portugues{
\rlettrine{H}{onrais} largamente os vossos amigos, ó Deus; e o seu poder fortaleceu-se extraordinariamente. Hei-de contá-los e verei que ultrapassam os grãos de areia da praia.
}\switchcolumn*\latim{
Allelúja, allelúja. ℣. \emph{Ps. 32, 1} Gaudéte, justi, in Dómino: rectos decet collaudátio. Allelúja.
}\switchcolumn\portugues{
Aleluia, aleluia. ℣. \emph{Sl. 32, 1} Ó justos, alegrai-vos no Senhor; pois aos corações rectos convém celebrar os louvores do Senhor. Aleluia.
}\end{paracol}

\paragraphinfo{Evangelho}{Jo. 20, 24-29}
\begin{paracol}{2}\latim{
\cruz Sequéntia sancti Evangélii secúndum Joánnem.
}\switchcolumn\portugues{
\cruz Continuação do santo Evangelho segundo S. João.
}\switchcolumn*\latim{
\blettrine{I}{n} illo témpore: Thomas, unus ex duódecim, qui dícitur Dídymus, non erat cum eis, quando venit Jesus. Dixérunt ergo ei alii discípuli: Vídimus Dóminum. Ille autem dixit eis: Nisi videre in mánibus ejus fixúram clavórum, et mittam dígitum meum in locum clavórum, et mittam manum meam in latus ejus, non credam. Et post dies octo, íterum erant discípuli ejus intus, et Thomas cum eis. Venit Jesus jánuis clausis, et stetit in médio, et dixit: Pax vobis. Deinde dicit Thomæ: Inter dígitum tuum huc, et vide manus meas, et affer manum tuam, et mitte in latus meum: et noli esse incrédulus, sed fidélis. Respóndit Thomas et dixit ei: Dóminus meus et Deus meus. Dixit ei Jesus: Quia vidisti me, Thoma, credidísti: beáti, qui non vidérunt, et crediderunt.
}\switchcolumn\portugues{
\blettrine{N}{aquele} tempo, Tomé, um dos Doze, que era chamado Dídimo, não estava com eles. Disseram-lhe, então, os outros discípulos: «Vimos o Senhor!». Ele disse-lhes: «Se não vir nas suas mãos o sinal dos cravos, se não meter o meu dedo no lugar dos cravos e se não meter a minha mão no seu lado, não acreditarei». Passados oito dias, encontravam-se outra vez no mesmo lugar, estando Tomé com eles. Veio então Jesus, estando as portas fechadas; e, pondo-se no meio deles, disse: «A paz seja convosco». Em seguida disse a Tomé: «Mete aqui o teu dedo e vê as minhas mãos; aproxima, também, a tua mão e mete-a no meu lado; não sejas incrédulo, mas fiel». Respondeu Tomé: «Meu Senhor e meu Deus!». Disse-lhe Jesus: «Porque me viste, ó Tomé, acreditaste: bem-aventurados aqueles que não viram e acreditaram».
}\end{paracol}

\paragraphinfo{Ofertório}{Sl. 18, 5}
\begin{paracol}{2}\latim{
\rlettrine{I}{n} omnem terram exívit sonus eórum: et in fines orbis terræ verba eórum.
}\switchcolumn\portugues{
\rlettrine{O}{} som da sua voz ecoou por toda a terra; e as suas palavras prolongaram-se até às extremidades da terra.
}\end{paracol}

\paragraph{Secreta}
\begin{paracol}{2}\latim{
\rlettrine{D}{ébitum} tibi, Dómine, nostræ réddimus servitútis, supplíciter exorántes: ut, suffrágiis beáti Thomæ Apóstoli, in nobis tua múnera tueáris, cujus honoránda confessióne laudis tibi hóstias immolámus. Per Dóminum \emph{\&c.}
}\switchcolumn\portugues{
\rlettrine{S}{enhor,} a vossos pés depomos o tributo da nossa sujeição, suplicando-Vos instantemente que em nós conserveis os vossos dons por intercessão do B. Tomé, Apóstolo, em quem honramos a gloriosa confissão, imolando hóstias em vosso louvor. Por nosso Senhor \emph{\&c.}
}\end{paracol}

\paragraphinfo{Comúnio}{Jo. 20, 27}
\begin{paracol}{2}\latim{
\rlettrine{M}{itte} manum tuam, et cognósce loca clavórum: et noli esse incrédulus, sed fidélis.
}\switchcolumn\portugues{
\rlettrine{M}{ete} aqui a tua mão e reconhece o lugar dos cravos. Não sejas incrédulo, mas fiel.
}\end{paracol}

\paragraph{Postcomúnio}
\begin{paracol}{2}\latim{
\rlettrine{A}{désto} nobis, miséricors Deus: et, intercedénte pro nobis beáto Thoma Apóstolo, tua circa nos propitiátus dona custódi. Per Dóminum nostrum \emph{\&c.}
}\switchcolumn\portugues{
\rlettrine{A}{ssisti-nos,} ó Deus de misericórdia; e, por intercessão do B. Apóstolo Tomé dignai-Vos conservar a nossa alma na posse dos dons que benignamente nos concedestes. Por nosso Senhor \emph{\&c.}
}\end{paracol}

\subsectioninfo{Santo António de Lisboa}{13 de Junho}\label{antoniolisboa}

\textit{Como na Missa In médio Ecclésiae, página \pageref{doutores}, excepto:}

\paragraph{Oração}
\begin{paracol}{2}\latim{
\rlettrine{E}{cclésiam} tuam, Deus, beáti Antónii Confessóris tui atque Doctóris solémnitas votiva lætíficet: ut spirituálibus semper muniátur auxíliis et gáudiis pérfrui mereátur ætérnis. Per Dóminum \emph{\&c.}
}\switchcolumn\portugues{
\qlettrine{Q}{ue} a festa anual do B. António, vosso Confessor e Doutor, alegre a vossa Igreja, Senhor, a fim de que, fortalecida sempre com os auxílios espirituais, mereça desfrutar os gozos eternos. Por nosso Senhor \emph{\&c.}
}\end{paracol}

\paragraph{Secreta}
\begin{paracol}{2}\latim{
\rlettrine{P}{ræsens} oblátio fiat, Dómine, pópulo tuo salutáris: pro quo dignátus es Patri tuo te vivéntem hóstiam immoláre: Qui cum eódem Deo Patre et Spíritu Sancto vivis et regnas \emph{\&c.}
}\switchcolumn\portugues{
\rlettrine{F}{azei,} Senhor, que a presente oblação seja salutar ao vosso povo, pelo qual Vos dignastes imolar-Vos ao vosso Pai, como hóstia viva. Ó Vós, que, sendo Deus \emph{\&c.}
}\end{paracol}

\paragraph{Postcomúnio}
\begin{paracol}{2}\latim{
\rlettrine{D}{ivínis,} Dómine, munéribus satiáti: quǽsumus; ut, beáti Antónii Confessóris tui atque Doctóris méritis et intercessióne, salutáris sacrifícii sentiámus efféctum. Per Dóminum \emph{\&c.}
}\switchcolumn\portugues{
\rlettrine{S}{aciados} com os divinos dons, Vos suplicamos, Senhor, que pela intercessão e méritos do B. António, vosso Confessor e Doutor, sintamos o efeito deste salutar sacrifício. Por nosso Senhor \emph{\&c.}
}\end{paracol}
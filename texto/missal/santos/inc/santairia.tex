\subsectioninfo{Santa Iria, Virgem e Mártir\footnote{Nalgumas Dioceses.}}{20 de Outubro}

\textit{Como na Missa Me exspectavérunt, página \pageref{virgensmartires2}, excepto:}

\paragraph{Oração}
\begin{paracol}{2}\latim{
\rlettrine{}{} \emph{\&c.}
}\switchcolumn\portugues{
\slettrine{Ó}{} Deus, que por um sinal celestial livrastes da infâmia a B. Iria, vossa Virgem e Mártir, concedei-nos propício pelos seus méritos e preces que sejamos purificados das manchas dos nossos pecados. Por nosso Senhor \emph{\&c.}
}\end{paracol}

\paragraphinfo{Evangelho}{}
\begin{paracol}{2}\latim{
\cruz Sequéntia sancti Evangélii secúndum Lucam.
}\switchcolumn\portugues{
\cruz Continuação do santo Evangelho segundo S. Mateus.
}\switchcolumn*\latim{
\blettrine{I}{n}
}\switchcolumn\portugues{
\blettrine{N}{aquele} tempo, disse Jesus aos seus discípulos esta parábola: O reino dos céus é semelhante a dez virgens que, empunhando suas lâmpadas, saíram ao encontro do esposo e da esposa. Porém, cinco destas virgens eram loucas e as outras cinco eram prudentes. Ora, as cinco loucas, empunhando suas lâmpadas, não levaram azeite. Ao contrário, as prudentes tomaram azeite em seus vasos para suas lâmpadas. Como o esposo se demorasse em chegar, tiveram sono e dormiram. Quando era meia noite, ouviu-se um clamor dizer: «Eis que chega o esposo; ide ao seu encontro». Então todas estas virgens se ergueram e prepararam as suas lâmpadas. As loucas disseram às prudentes: «Dai-nos do vosso azeite, porque as nossas lâmpadas apagam-se». As prudentes responderam-lhes: «Não, porque pode suceder que, como a vós, nos falte o azeite: ide antes aos que o vendem, e comprai-o». Ora, enquanto elas foram comprar o azeite, veio o esposo. Então, as que estavam preparadas entraram com ele para as bodas; e fechou-se a porta. Por fim vieram as outras virgens e disseram: «Senhor, senhor, abri-nos a porta». Ele respondeu: «Na verdade vos digo: não vos conheço». Vigiai, pois, visto que não sabeis nem o dia, nem a hora.
}\end{paracol}
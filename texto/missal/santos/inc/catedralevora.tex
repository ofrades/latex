\subsectioninfo{Dedicação da Catedral de Évora}{22 de Maio}

\textit{Como na Missa Terríbilis est, página \pageref{dedicacaoigreja}, excepto:}

\paragraphinfo{Intróito}{}
\begin{paracol}{2}\latim{
\rlettrine{}{}
\emph{}
℣. Gloria Patri \emph{\&c.}
}\switchcolumn\portugues{
\rlettrine{E}{xultemos} todos no Senhor, celebrando a festa da Dedicação da Igreja Eborense, de cuja santificação se alegram os Anjos, que louvam o Filho de Deus. Aleluia, aleluia. Amei, Senhor, o esplendor da vossa Casa e o lugar em que habita a vossa glória. 
\emph{}
℣. Glória ao Pai \emph{\&c.}
}\end{paracol}

\paragraph{Oração}
\begin{paracol}{2}\latim{
\rlettrine{}{} \emph{\&c.}
}\switchcolumn\portugues{
\slettrine{Ó}{} Deus, que quisestes reformar a Igreja Eborense na festividade do B. Mâncio, vosso discípulo e Mártir, Vos suplicamos que auxilieis com os dons celestes o vosso povo, a fim de que, cumprindo sempre a disciplina eclesiástica, alcance a vida eterna. Ó Vós, que viveis e reinais \emph{\&c.}
}\end{paracol}

\paragraphinfo{Evangelho}{}
\begin{paracol}{2}\latim{
\cruz Sequéntia sancti Evangélii secúndum Lucam.
}\switchcolumn\portugues{
\cruz Continuação do santo Evangelho segundo S. João.
}\switchcolumn*\latim{
\blettrine{I}{n}
}\switchcolumn\portugues{
\blettrine{N}{aquele} tempo, celebrava-se em Jerusalém a festa da Dedicação. Era no Inverno. E Jesus passeava no templo, no pórtico de Salomão. Rodearam-n’O, então, os judeus e disseram-Lhe: «Até quando nos trareis perplexos? Se sois o Cristo, dizei-nos claramente». Jesus respondeu-lhes: «Eu já vo-lo disse, mas não me acreditais. As obras que faço em nome de meu Pai dão testemunho de mim; porém, vós não acreditais, porque não sois das minhas ovelhas. Minhas ovelhas ouvem a minha voz. Eu conheço-as e elas seguem-me. Dou-lhes a vida eterna e jamais perecerão, porque ninguém as arrebata da minha mão».
}\end{paracol}
\subsectioninfo{S. Lourenço de Brindes, Conf.}{22 de Julho}

\paragraphinfo{Intróito}{}
\begin{paracol}{2}\latim{
\rlettrine{}{}
\emph{}
℣. Gloria Patri \emph{\&c.}
}\switchcolumn\portugues{
\rlettrine{D}{escobrirei} as obras de Deus pelas suas palavras. O sol, iluminando o mundo, vê todas as cousas: e a glória do Senhor brilha nas suas obras. Apareça o Senhor e desapareçam os seus inimigos! Que aqueles que O odeiam fujam da sua presença.
\emph{}
℣. Glória ao Pai \emph{\&c.}
}\end{paracol}

\paragraph{Oração}
\begin{paracol}{2}\latim{
\rlettrine{}{} \emph{\&c.}
}\switchcolumn\portugues{
\slettrine{Ó}{} Deus, que para glória do vosso nome e salvação das almas ornastes o B. Lourenço, vosso Confessor, com o espírito de conselho e de fortaleza nas obras ainda as mais árduas, concedei-nos pela sua intercessão o mesmo espírito, a fim de conhecermos o que devemos praticar e de praticarmos o que houvermos conhecido. Por nosso Senhor \emph{\&c.}
}\end{paracol}

\paragraphinfo{Epístola}{}
\begin{paracol}{2}\latim{
Lectio Epístolæ beati Pauli Apostoli ad Corinthios.
}\switchcolumn\portugues{
Lição da Ep.ª do B. Ap.º Paulo aos Coríntios.
}\switchcolumn*\latim{
\rlettrine{F}{ratres:}
}\switchcolumn\portugues{
\rlettrine{A}{} caridade de Cristo obriga-nos. Se considerarmos que um só morreu por todos, então todos morreremos. Ora Cristo morreu por todos, para que os que vivem, já não vivam para si, mas para Aquele que morreu e ressuscitou por eles. Eis porque não conhecemos ninguém, segundo a carne; e, se conhecemos Jesus Cristo segundo a carne, agora já o não conhecemos assim. Se alguém, pois. é de Cristo, é uma criatura nova; o passado já desapareceu e tudo se tornou novo. Tudo vem de Deus, que nos reconciliou consigo por Cristo e que nos confiou o ministério da reconciliação; porquanto Deus estava verdadeiramente em Cristo quando se reconciliou com o mundo, lhe não imputando mais os seus pecados e revestindo-nos com o poder da reconciliação. Nós cumprimos, pois, o cargo de embaixadores de Cristo, e é Deus quem vos exorta, servindo-se de nós. Nós vos conjuramos, invocando o nome de Cristo, a que vos reconcilieis com Deus, o qual por amor de vós tratou Aquele que não tinha pecado como se o tivesse, a fim de que por Ele nos tornássemos justos na justiça que vem de Deus.
}\end{paracol}

\paragraphinfo{Gradual}{}
\begin{paracol}{2}\latim{
\rlettrine{}{}
}\switchcolumn\portugues{
\rlettrine{O}{} Senhor é a minha fortaleza e a minha glória, pois foi o meu Salvador. Ele é o meu Deus: eu O glorificarei. O Senhor apareceu, como um guerreiro: e chama-se omnipotente.
}\switchcolumn*\latim{

}\switchcolumn\portugues{
Aleluia, aleluia. Invocou o Altíssimo e o Omnipotente quando os inimigos o atacaram de todos os lados; e Deus, que é excelso e santo, ouviu-o. Aleluia.
}\end{paracol}

\paragraphinfo{Evangelho}{Página \pageref{sextafeirapentecostes}}

\paragraphinfo{Ofertório}{}
\begin{paracol}{2}\latim{
\rlettrine{}{}
}\switchcolumn\portugues{
\rlettrine{E}{le} quis que minha boca fosse como uma espada aguda. Protegeu-me com a sombra da sua mão de reserva, como uma flecha escolhida.
}\end{paracol}

\paragraph{Secreta}
\begin{paracol}{2}\latim{
\rlettrine{}{} \emph{\&c.}
}\switchcolumn\portugues{
\rlettrine{F}{azei,} ó Deus, que as lágrimas da salutar penitência nos tornem dignos de nos aproximarmos deste celestial banquete, o qual era tão suave à candura da alma do B. Lourenço. Por nosso Senhor \emph{\&c.}
}\end{paracol}

\paragraphinfo{Comúnio}{}
\begin{paracol}{2}\latim{
\rlettrine{}{}
}\switchcolumn\portugues{
\rlettrine{O}{s} poderosos ficarão admirados quando me virem; e a face dos príncipes manifestará a sua admiração.
}\end{paracol}

\paragraph{Postcomúnio}
\begin{paracol}{2}\latim{
\rlettrine{}{} \emph{\&c.}
}\switchcolumn\portugues{
\rlettrine{P}{ossamos} nós, Senhor, ficar eternamente saciados com esta alegria da vossa divindade, cujo gozo o B. Lourenço experimentou no sacrossanto mystério do altar. Por nosso Senhor \emph{\&c.}
}\end{paracol}

\textit{Nalguns lugares diz-se a Missa Os justi, página \pageref{confessoresnaopontifices1}, com a Oração, Secreta e Postcomúnio precedentes.}

\subsectioninfo{S. Martinho de Dume, B. e Conf.\footnote{Na Arquidiocese de Braga}}{20 de Março}

\textit{Como na Missa In médio Ecclésiae, página \pageref{doutores}, excepto:}

\paragraph{Oração}
\begin{paracol}{2}\latim{
\rlettrine{}{} \emph{\&c.}
}\switchcolumn\portugues{
\slettrine{Ó}{} Deus, que ao vosso povo concedestes o B. Martinho como ministro da salvação eterna, permiti, Vos rogamos, que mereçamos sempre ter como intercessor nos céus aquele que na terra possuímos como Patrono e Mestre de vida. Por nosso Senhor \emph{\&c.}
}\end{paracol}

\paragraphinfo{Epístola}{}
\begin{paracol}{2}\latim{
Lectio Epístolæ beati Pauli Apostoli ad Corinthios.
}\switchcolumn\portugues{
Lição do Livro da Sabedoria.
}\switchcolumn*\latim{
\rlettrine{}{}
}\switchcolumn\portugues{
\rlettrine{D}{esejei} a inteligência, e foi-me dada; invoquei o espírito da sabedoria, e veio a mim. Preferi-a aos reinos e aos tronos; e creio que as riquezas nada são comparadas com ela. Nem mesmo a compararei com as pedras preciosas; pois todo o oiro, comparando-o com ela, é como um grão de areia; e toda a prata, ao pé dela, é desprezível lodo. Amo-a mais do que a saúde e a beleza; e, assim, resolvi tomá-la para minha luz, pois o seu brilho não tem ocaso. Todos os bens me vieram dela e recebi das suas mãos inumeráveis riquezas. Regozijei-me em todas as coisas, pois a sabedoria guiava-me, e eu ignorava que ela era a mãe de todos os bens. Conheci a sabedoria sem fingimento e comunico-a sem inveja, não ocultando as suas riquezas. Ela é um tesouro infinito para os homens. Aqueles que a aproveitam tornam-se amigos de Deus e recomendam-se pelos dons da ciência.
}\end{paracol}

\paragraphinfo{Gradual}{}
\begin{paracol}{2}\latim{
\rlettrine{}{}
}\switchcolumn\portugues{
\rlettrine{E}{ncontrei} o meu, servo David e ungi-o com meu óleo sagrado; a minha mão, pois, o auxiliará e o meu braço o fortalecerá. O inimigo nunca alcançará vitória contra ele: e o filho da iniquidade o não prejudicará.
}\switchcolumn*\latim{

}\switchcolumn\portugues{

}\end{paracol}

\paragraphinfo{Ofertório}{}
\begin{paracol}{2}\latim{
\rlettrine{}{}
}\switchcolumn\portugues{
\rlettrine{B}{em-aventurado} o varão que o Senhor, quando vier, encontrar vigilante. Em verdade vos digo que o colocará à testa de todos seus bens.
}\end{paracol}

\paragraph{Secreta}
\begin{paracol}{2}\latim{
\rlettrine{}{} \emph{\&c.}
}\switchcolumn\portugues{
\rlettrine{D}{eus} omnipotente e sempiterno, fazei que estes dons, que oferecemos à vossa majestade por intercessão do B. Martinho, vosso Confessor e Pontífice, nos sirvam de proveito para a salvação eterna. Por nosso Senhor \emph{\&c.}
}\end{paracol}

\paragraphinfo{Comúnio}{}
\begin{paracol}{2}\latim{
\rlettrine{}{}
}\switchcolumn\portugues{
\rlettrine{S}{enhor,} entregastes-me cinco talentos; eis outros cinco que lucrei. Está bem, servo bom e fiel; visto que foste fiel em pouca coisa, Eu te estabelecerei sobre muitas: entra no gozo do teu senhor.
}\end{paracol}

\paragraph{Postcomúnio}
\begin{paracol}{2}\latim{
\rlettrine{}{} \emph{\&c.}
}\switchcolumn\portugues{
\rlettrine{H}{avendo} recebido os sacramentos da nossa salvação, concedei-nos, ó misericordioso Deus, Vos suplicamos, que nos sirvam sempre de auxílio as preces do B. Martinho, em cuja veneração os oferecemos à vossa majestade. Por nosso \emph{\&c.}
}\end{paracol}

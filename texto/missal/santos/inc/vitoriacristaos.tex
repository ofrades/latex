\subsectioninfo{Vitória dos Cristãos\footnote{Missa de Acção de Graças pela vitória que os cristãos alcançaram na luta contra os sarracenos.}}{30 de Outubro}

\textit{Missa de Acção de Graças pela vitória que os cristãos alcançaram na luta contra os sarracenos.}

\paragraphinfo{Evangelho, Ofertório e Secreta}{Página \pageref{invencaosantacruz}}

\paragraphinfo{Intróito}{}
\begin{paracol}{2}\latim{
\rlettrine{}{}
\emph{}
℣. Gloria Patri \emph{\&c.}
}\switchcolumn\portugues{
\rlettrine{A}{legremo-nos} em Deus em todo este dia: e louvemos eternamente o vosso nome, Senhor; pois salvastes-nos dos nossos inimigos e confundistes os que nos odiavam. Ó Deus, ouvimos com os nossos ouvidos e nossos país contaram-nos os feitos que praticastes nestes dias e nos dias passados.
℣. Glória ao Pai \emph{\&c.}
}\end{paracol}

\paragraph{Oração}
\begin{paracol}{2}\latim{
\rlettrine{}{} \emph{\&c.}
}\switchcolumn\portugues{
\slettrine{Ó}{} Deus, que pela vossa Cruz quisestes conceder ao povo, que em Vós crê, a vitória contra os inimigos, permiti pela vossa piedade, Vos pedimos, que aqueles que adoram a Cruz, alcancem sempre a vitória na terra e o gozo eterno nos céus. Ó Vós, que viveis e reinais \emph{\&c.}
}\end{paracol}

\paragraphinfo{Epístola}{}
\begin{paracol}{2}\latim{
Lectio Epístolæ beati Pauli Apostoli ad Corinthios.
}\switchcolumn\portugues{
Lição do Livro dos Macabeus.
}\switchcolumn*\latim{
\rlettrine{F}{ratres:}
}\switchcolumn\portugues{
\rlettrine{N}{aqueles} dias, Macabeu esperava sempre com toda a confiança que um socorro lhe viria de Deus. E exortava os seus a que se não amedrontassem com a chegada das nações, mas que se lembrassem dos socorros, que lhes haviam sido dados pelo céu, e esperassem, agora, que a vitória lhes viesse do Omnipotente. Ora, quando todos esperavam já a futura decisão do combate, estando à vista o inimigo e o seu exército formado em batalha, os elefantes e a cavalaria dispostos no lugar competente, considerando Macabeu aquela multidão de gentes, aquele aparato de armas tão diversas, que vinha contra eles, e a ferocidade dos animais, ergueu as mãos ao céu e invocou o Senhor, autor de todos os prodígios, que dá a vitória aos que a merecem, não pelo poder das armas, mas a quem Lhe apraz. Disse, então, Macabeu: «Ó Senhor, que no tempo de Ezequias, rei de Judá, mandastes o vosso Anjo e matastes cento e oitenta e cinco mil homens dos exércitos de Senaqueribe, mandai agora também, diante de nós, o vosso bom Anjo, para que inspire o temor e o tremor da grandeza do vosso braço, e aqueles que, blasfemando vosso nome, vêm atacar o vosso povo se amedrontem». Logo, Judas e os companheiros, invocando Deus, ergueram-se e, pelejando e encomendando-se ao Senhor, mataram não menos de trinta e cinco mil homens, sentindo-se alegres pela presença de Deus. Acabada a peleja, quando regressavam jubilosos, souberam que Nicanor tinha caído morto, coberto com suas armas. Então, com forte alarido e estrondosas ovações, aclamaram o omnipotente Senhor e decretaram que não mais passasse aquele dia sem que se realizasse festiva comemoração.
}\end{paracol}

\paragraphinfo{Gradual}{}
\begin{paracol}{2}\latim{
\rlettrine{}{}
}\switchcolumn\portugues{
\rlettrine{E}{is} o dia que o Senhor criou. Exultemos e alegremo-nos n’Ele. Assim devem cantar aqueles que o Senhor resgatou e tirou das mãos dos inimigos.
}\switchcolumn*\latim{

}\switchcolumn\portugues{
Aleluia, aleluia. Cantemos em honra do Senhor, pois assinalou gloriosamente a sua grandeza, arrojando ao mar o cavalo e o cavaleiro. Aleluia.
}\end{paracol}

\paragraphinfo{Comúnio}{}
\begin{paracol}{2}\latim{
\rlettrine{}{}
}\switchcolumn\portugues{
\rlettrine{C}{om} o vosso poder, Senhor, assinalou-se a vossa dextra que esmagou o inimigo. Pela vossa misericórdia, fostes o guia do povo, que resgatastes e conduzistes pelo vosso poder ao vosso santo tabernáculo.
}\end{paracol}

\paragraph{Postcomúnio}
\begin{paracol}{2}\latim{
\rlettrine{}{} \emph{\&c.}
}\switchcolumn\portugues{
\rlettrine{O}{uvi-nos,} ó Deus, nosso Salvador; e pela vitória da Santa Cruz livrai-nos de todos os perigos. Por nosso Senhor \emph{\&c.}
}\end{paracol}

\subsectioninfo{B. Francisco Pacheco e Outros, Márts.\footnote{Nalgumas Dioceses}}{20 de Junho}

\paragraphinfo{Intróito}{}
\begin{paracol}{2}\latim{
\rlettrine{}{}
\emph{}
℣. Gloria Patri \emph{\&c.}
}\switchcolumn\portugues{
\rlettrine{O}{} Senhor dirigirá ao seu povo palavras de paz: e igualmente assim falará aos seus fiéis e àqueles que se converterem a Ele. Abençoastes, Senhor, a vossa terra e fizestes cessar o cativeiro de Jacob.
\emph{}
℣. Glória ao Pai \emph{\&c.}
}\end{paracol}

\paragraph{Oração}
\begin{paracol}{2}\latim{
\rlettrine{}{} \emph{\&c.}
}\switchcolumn\portugues{
\slettrine{Ó}{} Deus, que nos alegrais com o aniversário solene do martírio do B. Francisco e seus Companheiros, concedei-nos propício que aqueles cujos méritos nos enchem de alegria, nos inflamem também com seus exemplos. Por nosso Senhor \emph{\&c.}
}\end{paracol}

\paragraphinfo{Epístola}{}
\begin{paracol}{2}\latim{
Lectio Epístolæ beati Pauli Apostoli ad Corinthios.
}\switchcolumn\portugues{
Lição da Ep.ª do B. Ap.º Pedro.
}\switchcolumn*\latim{
\rlettrine{F}{ratres:}
}\switchcolumn\portugues{
\rlettrine{C}{aríssimos:} Alegrai-vos, se tomais parte nos sofrimentos de Cristo; pois exultareis de alegria na manifestação da sua glória. Sereis felizes, se fordes ultrajados por causa do nome de Cristo; pois o espírito de honra, de glória e de virtude de Deus, que é o seu Espírito, repousará sobre vós. Porém, nenhum de vós sofra como homicida, como ladrão, como malfeitor ou como cobiçador dos bens alheios. Todavia, se é como cristão que padece, não se envergonhe, e antes glorifique Deus neste nome. É chegado o tempo de principiar o juízo pela casa de Deus. Ora, se começa por vós, qual será o fim daqueles que não crêem no Evangelho de Deus? Se só o justo será salvo, que acontecerá ao ímpio e ao pecador? Assim, pois, aqueles que sofrem segundo a vontade de Deus encomendam as suas almas ao seu fiel Criador, praticando obras boas.
}\end{paracol}

\paragraphinfo{Gradual}{}
\begin{paracol}{2}\latim{
\rlettrine{}{}
}\switchcolumn\portugues{
\rlettrine{D}{eus} é glorificado nos seus Santos: Deus é admirável na sua majestade: Deus pratica muitos prodígios. Vossa dextra esmagou os vossos inimigos.
}\switchcolumn*\latim{

}\switchcolumn\portugues{
Aleluia, aleluia. Esta é a verdadeira fraternidade que venceu os crimes do mundo: ela segue Cristo e possuirá gloriosamente o reino celestial. Aleluia.
}\end{paracol}

\paragraphinfo{Evangelho}{Página \pageref{muitosmartires2}}

\paragraphinfo{Ofertório}{}
\begin{paracol}{2}\latim{
\rlettrine{}{}
}\switchcolumn\portugues{
\slettrine{Ó}{} justos, alegrai-vos no Senhor e exultai de júbilo: todos aqueles que possuem o coração recto serão glorificados.
}\end{paracol}

\paragraph{Secreta}
\begin{paracol}{2}\latim{
\rlettrine{}{} \emph{\&c.}
}\switchcolumn\portugues{
\rlettrine{D}{eixai-Vos} aplacar, Senhor, com a oferta que Vos apresentamos, e pela intercessão dos vossos B. B. Mártires defendei-nos de todos os perigos. Por nosso Senhor \emph{\&c.}
}\end{paracol}

\paragraphinfo{Comúnio}{}
\begin{paracol}{2}\latim{
\rlettrine{}{}
}\switchcolumn\portugues{
\rlettrine{S}{enhor,} deram como alimento às aves do céu os corpos dos vossos servos, que haviam sido mortos, e deram as carnes dos vossos Santos às feras da terra. Pelo poder do vosso braço conservai os filhos daqueles que foram mortos.
}\end{paracol}

\paragraph{Postcomúnio}
\begin{paracol}{2}\latim{
\rlettrine{}{} \emph{\&c.}
}\switchcolumn\portugues{
\qlettrine{Q}{ue} esta comunhão, Senhor, nos purifique das nossas faltas; e pela intercessão dos B. B. Mártires Francisco e seus Companheiros fazei que nos torne participantes do remédio celestial. Por nosso Senhor \emph{\&c.}
}\end{paracol}

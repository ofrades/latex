\subsectioninfo{Santa Colomba, Virgem e Mártir\footnote{Na Diocese de Coimbra.}}{20 de Julho}

\paragraphinfo{Intróito}{}
\begin{paracol}{2}\latim{
\rlettrine{}{}
\emph{}
℣. Gloria Patri \emph{\&c.}
}\switchcolumn\portugues{
\qlettrine{Q}{uem} me dará asas, como as da pomba! Então voarei e descansarei. Eis que me afastei, fugindo, e permaneci na solidão. Esperava Aquele que me salvou. Ouvi, ó Deus, a minha oração; não desprezeis a minha súplica; atendei-me e ouvi-me.
\emph{}
℣. Glória ao Pai \emph{\&c.}
}\end{paracol}

\paragraph{Oração}
\begin{paracol}{2}\latim{
\rlettrine{}{} \emph{\&c.}
}\switchcolumn\portugues{
\slettrine{Ó}{} omnipotente e eterno Deus, olhai propício para a nossa fraqueza; e, assim como concedestes a fortaleza à B. Colomba, vossa Virgem, para suportar o martírio da Cruz, assim também, protegendo-nos com sua intercessão, acolhei-nos à dextra da vossa majestade. Por nosso Senhor \emph{\&c.}
}\end{paracol}

\paragraphinfo{Epístola}{Página \pageref{virgemnaomartir2}}

\paragraphinfo{Gradual}{}
\begin{paracol}{2}\latim{
\rlettrine{}{}
}\switchcolumn\portugues{
\rlettrine{O}{} meu inimigo arruinou as minhas veredas e afligiu-me até à desolação. Armou o seu arco, apontou-o para mim, como alvo da sua seta, e cravou nos meus rins as setas da sua aljava.
}\switchcolumn*\latim{

}\switchcolumn\portugues{
Aleluia, aleluia. Suportei a dor das setas até à morte de cruz, para ser fiel ao meu Senhor Jesus. Aleluia.
}\end{paracol}

\paragraphinfo{Evangelho}{Página \pageref{virgensmartires1}}

\paragraphinfo{Ofertório}{}
\begin{paracol}{2}\latim{
\rlettrine{}{}
}\switchcolumn\portugues{
\rlettrine{}N{unca} Deus permita que me glorie senão na Cruz de nosso Senhor Jesus Cristo, por quem o mundo está crucificado para mim e eu para o mundo.
}\end{paracol}

\paragraph{Secreta}
\begin{paracol}{2}\latim{
\rlettrine{}{} \emph{\&c.}
}\switchcolumn\portugues{
\rlettrine{S}{enhor,} que este sacrifício, que Vos é oferecido pelos auxílios dos méritos da B. Colomba, vossa Virgem e Mártir, nos conserve a vida e nos proteja. Por nosso Senhor \emph{\&c.}
}\end{paracol}

\paragraphinfo{Comúnio}{}
\begin{paracol}{2}\latim{
\rlettrine{}{}
}\switchcolumn\portugues{
\rlettrine{E}{rgue-te,} minha amiga, minha única beleza, e vem. Ó minha pomba, escondida nas fendas das rochas e nas cavernas dos muros em ruínas, mostra-me o teu rosto e faz-me ouvir a tua voz. A tua voz é doce, o teu rosto é belo.
}\end{paracol}

\paragraph{Postcomúnio}
\begin{paracol}{2}\latim{
\rlettrine{}{} \emph{\&c.}
}\switchcolumn\portugues{
\rlettrine{P}{ela} participação que tivemos neste mistério, Senhor, confirmai os vossos servos na confissão da verdadeira fé, pela qual a B. Colomba não duvidou sofrer o martírio da Cruz e derramar o sangue. Por nosso Senhor \emph{\&c.}
}\end{paracol}

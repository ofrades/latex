\subsectioninfo{S. Pedro de Rates}{26 de Abril}

\textit{Como na Missa Protexísti me, página \pageref{martir}, excepto:}

\paragraph{Oração}
\begin{paracol}{2}\latim{
\rlettrine{}{} \emph{\&c.}
}\switchcolumn\portugues{
\slettrine{Ó}{} Deus, que consagrastes este dia com o martírio do B. Pedro, vosso Mártir e Pontífice, concedei à vossa Igreja Bracarense a graça de seguir em todas as cousas os gloriosos vestígios daquele por cujo ministério ela recebeu as primícias da fé. Por nosso Senhor \emph{\&c.}
}\end{paracol}

\paragraphinfo{Evangelho}{}
\begin{paracol}{2}\latim{
\cruz Sequéntia sancti Evangélii secúndum Lucam.
}\switchcolumn\portugues{
\cruz Continuação do santo Evangelho segundo S. João.
}\switchcolumn*\latim{
\blettrine{I}{n}
}\switchcolumn\portugues{
\blettrine{N}{aquele} tempo, disse Jesus aos fariseus: «Eu sou o bom Pastor. O bom Pastor dá a vida pelas ovelhas. Porém, o mercenário, que não é pastor e a quem as ovelhas não pertencem, vê vir o lobo, abandona as ovelhas e foge. E, então, o lobo arrebata as ovelhas e dispersa-as. O mercenário procede assim porque é mercenário e porque não tem cuidado com as ovelhas. Eu sou o bom Pastor: eu conheço as minhas ovelhas, e as minhas ovelhas conhecem-me; assim como meu Pai me conhece e eu conheço meu Pai. Eu dou a minha vida pelas minhas ovelhas. Tenho ainda outras ovelhas que não pertencem a este aprisco; mas é preciso que eu as atraia e ouçam a minha voz, para que não haja senão um só aprisco e um só Pastor».
}\end{paracol}

\paragraph{Secreta}
\begin{paracol}{2}\latim{
\rlettrine{}{} \emph{\&c.}
}\switchcolumn\portugues{
\rlettrine{S}{antificai,} Senhor, os dons que Vos oferecemos, e pela intercessão do B. Pedro, vosso Mártir e Pontífice, e em virtude desses dons, purificai-nos de todas as manchas dos nossos pecados. Por nosso Senhor \emph{\&c.}
}\end{paracol}

\paragraph{Postcomúnio}
\begin{paracol}{2}\latim{
\rlettrine{}{} \emph{\&c.}
}\switchcolumn\portugues{
\qlettrine{Q}{ue} esta solenidade, que celebramos com estes celestiais mystérios em honra do B. Pedro, vosso Mártir e Pontífice, ó Deus omnipotente, nos alcance o perdão da vossa misericórdia. Por nosso Senhor \emph{\&c.}
}\end{paracol}
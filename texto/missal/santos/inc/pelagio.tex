\subsectioninfo{S. Pelágio\footnote{Nalgumas Dioceses}}{26 de Junho}

\paragraphinfo{Intróito}{}
\begin{paracol}{2}\latim{
\rlettrine{}{}
\emph{}
℣. Gloria Patri \emph{\&c.}
}\switchcolumn\portugues{
\rlettrine{F}{izestes-Vos,} Senhor, o meu auxiliar e o meu protector: e livrastes-me da perdição: livrastes-me das mãos daqueles que procuravam tirar-me a vida, assim como do rei iníquo e das línguas injustas. Entoai em honra do Senhor um cântico novo, pois Ele operou maravilhas.
\emph{}
℣. Glória ao Pai \emph{\&c.}
}\end{paracol}

\paragraph{Oração}
\begin{paracol}{2}\latim{
\rlettrine{}{} \emph{\&c.}
}\switchcolumn\portugues{
\slettrine{Ó}{} Deus, que pela magnitude da vossa inefável piedade permitistes que o B. Pelágio, sendo de pouca idade, se mostrasse grande na fé e virtude, Concedei-nos, Vos suplicamos, que, assim como Veneramos a sua glória, assim também imitemos a sua inocência. Por nosso Senhor \emph{\&c.}
}\end{paracol}

\paragraphinfo{Epístola}{Página \pageref{martirnaopontifice1}}

\paragraphinfo{Gradual}{}
\begin{paracol}{2}\latim{
\rlettrine{}{}
}\switchcolumn\portugues{
\rlettrine{S}{ois} a minha esperança, Senhor, desde a minha juventude: por Vós fui fortalecido ainda antes de nascer: desde o seio de minha mãe que sois o meu protector. Aceitastes-me por causa da minha inocência e fortalecestes-me para sempre na vossa presença.
}\switchcolumn*\latim{

}\switchcolumn\portugues{
Aleluia, aleluia. Sou, fui e serei cristão (diz Pelágio ao rei) eis porque não temo a morte. Aleluia.
}\end{paracol}

\paragraphinfo{Evangelho}{Página \pageref{martirnaopontifice2}}

\paragraphinfo{Ofertório}{}
\begin{paracol}{2}\latim{
\rlettrine{}{}
}\switchcolumn\portugues{
\rlettrine{S}{ua} vida consumou-se em breve; contudo encheu seus anos com muitas coisas, pois sua alma era agradável a Deus. Eis porque Ele se apressou a tirá-lo do meio das iniquidades.
}\end{paracol}

\paragraph{Secreta}
\begin{paracol}{2}\latim{
\rlettrine{}{} \emph{\&c.}
}\switchcolumn\portugues{
\rlettrine{V}{os} oferecemos, Senhor, a hóstia imaculada, rogando-Vos insistentemente que em virtude das preces e da intercessão do B. Pelágio, vosso Mártir, alcancemos o que humildemente Vos pedimos. Por nosso Senhor \emph{\&c.}
}\end{paracol}

\paragraphinfo{Comúnio}{}
\begin{paracol}{2}\latim{
\rlettrine{}{}
}\switchcolumn\portugues{
\rlettrine{A}{quele} que vencer envergará os vestidos brancos: e Eu confessarei o seu nome na presença de meu Pai e dos seus Anjos, aleluia.
}\end{paracol}

\paragraph{Postcomúnio}
\begin{paracol}{2}\latim{
\rlettrine{}{} \emph{\&c.}
}\switchcolumn\portugues{
\rlettrine{S}{aciados} com o banquete do alimento espiritual e animados com a sacratíssima bebida, Vos suplicamos, Senhor, nosso Deus, que, assim como nos alegramos com a coroa triunfal do B. Pelágio, vosso Mártir, assim também continuamente gozemos o seu patrocínio. Por nosso Senhor \emph{\&c.}
}\end{paracol}

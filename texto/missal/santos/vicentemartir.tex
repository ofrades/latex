\subsectioninfo{S. S. Vicente e Anastácio, Márts.}{22 de Janeiro}

\textit{Como Missa Intret in, página \pageref{muitosmartires1}, excepto:}

\paragraph{Oração}
\begin{paracol}{2}\latim{
\rlettrine{A}{désto,} Dómine, supplicatiónibus nostris: ut, qui ex iniquitáte nostra reos nos esse cognóscimus, beatórum Mártyrum tuórum Vincéntii et Anastásii intercessióne liberémur. Per Dóminum \emph{\&c.}
}\switchcolumn\portugues{
\rlettrine{S}{enhor,} dignai-Vos ouvir as nossas súplicas, a fim de que, reconhecendo-nos réus diante de Vós pelas nossas iniquidades, sejamos livres delas por intercessão dos vossos B. B. Mártires Vicente e Anastácio. Por nosso Senhor \emph{\&c.}
}\end{paracol}

\paragraph{Secreta}
\begin{paracol}{2}\latim{
\rlettrine{M}{únera} tibi, Dómine, nostræ devotiónis offérimus: quæ et pro tuórum tibi grata sint honóre Justórum, et nobis salutária, te miseránte, reddántur. Per Dóminum \emph{\&c.}
}\switchcolumn\portugues{
\rlettrine{S}{enhor,} Vos oferecemos estes dons da nossa devoção; e que em consideração dos vossos Santos eles Vos sejam agradáveis, e pela vossa misericórdia nos sejam salutares. Por nosso Senhor \emph{\&c.}
}\end{paracol}

\paragraph{Postcomúnio}
\begin{paracol}{2}\latim{
\qlettrine{Q}{uǽsumus,} omnípotens Deus: ut, qui cœléstia aliménta percépimus, intercedéntibus beátis Martýribus tuis Vincéntio et Anastásio, per hæc contra ómnia advérsa muniámur. Per Dóminum \emph{\&c.}
}\switchcolumn\portugues{
\slettrine{Ó}{} Deus omnipotente, Vos suplicamos, fazei que, havendo nós recebido os alimentos celestiais, sejamos fortalecidos contra todas as adversidades por intercessão dos vossos B. B. Mártires Vicente e Anastácio. Por nosso Senhor \emph{\&c.}
}\end{paracol}

\subsectioninfo{S. Vicente, Mártir\footnote{No Patriarcado de Lisboa e na Diocese do Algarve.}}{22 de Janeiro}\label{vicente}

\textit{Como Missa In virtúte tua, página \pageref{martirnaopontifice1}, excepto:}

\paragraph{Oração}
\begin{paracol}{2}\latim{
\rlettrine{A}{désto,} Dómine, supplicatiónibus nostris: ut, qui ex iniquitáte nostra reos nos esse cognóscimus, beáti Vincéntii Martyris tui intercessióne liberémur. Per Dóminum \emph{\&c.}
}\switchcolumn\portugues{
\rlettrine{S}{enhor,} dignai-Vos ouvir as nossas súplicas, a fim de que, reconhecendo-nos réus diante de Vós pelas nossas iniquidades, sejamos livres delas por intercessão do vosso B. Mártir Vicente. Por nosso Senhor \emph{\&c.}
}\end{paracol}

\paragraphinfo{Evangelho}{Jo. 12, 24-26}
\begin{paracol}{2}\latim{
\cruz Sequéntia sancti Evangélii secúndum Joánnem.
}\switchcolumn\portugues{
\cruz Continuação do santo Evangelho segundo S. João.
}\switchcolumn*\latim{
\blettrine{I}{n} illo témpore: Dixit Jesus discípulis suis: Amen, amen, dico vobis, nisi granum fruménti cadens in terram, mórtuum fúerit, ipsum solum manet: si autem mórtuum fúerit, multum fructum affert. Qui amat ánimam suam, perdet eam: et qui odit ánimam suam in hoc mundo, in vitam ætérnam custódit eam. Si quis mihi mínistrat, me sequátur: et ubi sum ego, illic et miníster meus erit. Si quis mihi ministráverit, honorificábit eum Pater meus.
}\switchcolumn\portugues{
\blettrine{N}{aquele} tempo, disse Jesus aos seus discípulos: «Se o grão de trigo, caindo na terra, não morrer, permanece estéril; mas, se morrer, dará muito fruto. Aquele que ama a sua vida perdê-la-á; mas aquele que aborrece a sua vida neste mundo conservá-la-á para a vida eterna. Se alguém me serve, siga-me; e onde eu estiver lá estará também o meu servo. Se alguém me servir, meu Pai o honrará».
}\end{paracol}

\paragraphinfo{Ofertório}{Sl. 95, 6}
\begin{paracol}{2}\latim{
\rlettrine{C}{onféssio} et pulchritúdo in conspéctu ejus: sánctitas, et magnificéntia in sanctificatióne ejus.
}\switchcolumn\portugues{
\rlettrine{R}{odeiam-no} a glória e a majestade: e no seu santuário reluzem a santidade e a magnificência.
}\end{paracol}
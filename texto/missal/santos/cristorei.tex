\subsectioninfo{Nosso Senhor Jesus Cristo-Rei}{Último Domingo de Outubro}

\paragraphinfo{Intróito}{Ap. 5, 12; 1, 6}
\begin{paracol}{2}\latim{
\rlettrine{D}{ignus} est Agnus, qui occísus est, accípere virtútem, et divinitátem, et sapiéntiam, et fortitúdinem, et honórem. Ipsi glória et impérium in sǽcula sæculórum. \emph{Ps. 71, 1} Deus, judícium tuum Regi da: et justítiam tuam Fílio Regis.
℣. Gloria Patri \emph{\&c.}
}\switchcolumn\portugues{
\rlettrine{O}{} Cordeiro, que foi imolado, é digno de receber o poder, a divindade, a sabedoria, a fortaleza, a honra: A Ele a glória e o império em todos os séculos dos séculos. \emph{Sl. 71, 1} Ó Deus, dai ao Rei o poder de julgar; e a vossa justiça ao Filho do Rei.
℣. Glória ao Pai \emph{\&c.}
}\end{paracol}

\paragraph{Oração}
\begin{paracol}{2}\latim{
\rlettrine{O}{mnípotens} sempitérne Deus, qui in dilécto Fílio tuo, universórum Rege, ómnia instauráre voluísti: concéde propítius; ut cunctæ famíliæ géntium, peccáti vúlnere disgregátæ, ejus suavissímo subdántur império: Qui tecum vivit et regnat \emph{\&c.}
}\switchcolumn\portugues{
\rlettrine{O}{mnipotente} e eterno Deus, que tudo quisestes restaurar no vosso dilecto Filho, Rei de todas as coisas, concedei-nos propício que todas as famílias do mundo, livres da chaga do pecado, se submetam ao Seu suavíssimo império. Ele, que, sendo Deus, vive e reina \emph{\&c.}
}\end{paracol}

\paragraphinfo{Epístola}{Cl. 1, 12-20}
\begin{paracol}{2}\latim{
Léctio Epístolæ beáti Pauli Apóstoli ad Colossénses.
}\switchcolumn\portugues{
Lição da Ep.ª do B. Ap.º Paulo aos Colossenses.
}\switchcolumn*\latim{
\rlettrine{F}{ratres:} Grátias ágimus Deo Patri, qui dignos nos fecit in partem sortis sanctórum in lúmine: qui eripuit nos de potestáte tenebrárum, et tránstulit in regnum Fílii dilectiónis suæ, in quo habémus redemptiónem per sánguinem ejus, remissiónem peccatórum: qui est imágo Dei invisíbilis, primogénitus omnis creatúra: quóniam in ipso cóndita sunt univérsa in cœlis et in terra, visibília et invisibília, sive Throni, sive Dominatiónes, sive Principátus, sive Potestátes: ómnia per ipsum, et in ipso creáta sunt: et ipse est ante omnes, et ómnia in ipso constant. Et ipse est caput córporis Ecclésiæ, qui est princípium, primogénitus ex mórtuis: ut sit in ómnibus ipse primátum tenens; quia in ipso complácuit omnem plenitúdinem inhabitáre; et per eum reconciliáre ómnia in ipsum, pacíficans per sánguinem crucis ejus, sive quæ in terris, sive quæ in cœlis sunt, in Christo, Jesu, Dómino nostro.
}\switchcolumn\portugues{
\rlettrine{M}{eus} irmãos: Damos graças a Deus Pai, porque nos fez dignos de participar da herança dos Santos na luz, nos livrou do poder das trevas e nos conduziu para o reino do seu muito amado Filho, no qual possuímos a redenção pelo seu sangue e a remissão dos pecados. Ele é a imagem de Deus invisível e o primogénito de todas as criaturas; porquanto n’Ele foram criadas todas as coisas nos céus e na terra, visíveis e invisíveis, quer sejam os Tronos, quer as Dominações, quer os Principados, quer as Potestades. Tudo foi criado por Ele e n’Ele próprio. Ele existe antes de todas as coisas e todas as coisas subsistem por Ele. Ele é a cabeça do corpo da Igreja e o princípio e o primogénito de todos os mortais, para que assim conserve a primazia de todas as coisas, pois foi do agrado do Pai que n’Ele residisse toda a plenitude e por Ele se reconciliem em si próprio todas as coisas, pacificando pelo seu sangue na Cruz tanto o que está na terra, como o que está no céu, em nosso Senhor Jesus Cristo.
}\end{paracol}

\paragraphinfo{Gradual}{Sl. 71, 8 \& 11}
\begin{paracol}{2}\latim{
\rlettrine{D}{ominábitur} a mari usque ad mare, et a flúmine usque ad términos orbis terrárum. ℣. Et adorábunt eum omnes reges terræ: omnes gentes sérvient ei.
}\switchcolumn\portugues{
\rlettrine{D}{ominará} de mar a mar e desde o rio aos confins da redondeza da terra! ℣. Adorá-l’O-ão todos os reis da terra: e todos os povos O servirão!
}\switchcolumn*\latim{
Allelúja, allelúja. ℣. \emph{Dan. 7, 14} Potéstas ejus, potéstas ætérna, quæ non auferétur: et regnum ejus, quod non corrumpétur. Allelúja.
}\switchcolumn\portugues{
Aleluia, aleluia. ℣. \emph{Dn. 7, 14} Seu poder é eterno, nunca Lhe será tirado; seu reino é tal que nunca será corrompido. Aleluia.
}\end{paracol}

\paragraphinfo{Evangelho}{Jo. 18, 33-37}
\begin{paracol}{2}\latim{
\cruz Sequéntia sancti Evangélii secúndum Joánnem.
}\switchcolumn\portugues{
\cruz Continuação do santo Evangelho segundo S. João.
}\switchcolumn*\latim{
\blettrine{I}{n} illo témpore: Dixit Pilátus ad Jesum: Tu es Rex Judæórum? Respóndit Jesus: A temetípso hoc dicis, an alii dixérunt tibi de me? Respóndit Pilátus: Numquid ego Judǽus sum? Gens tua et pontífices tradidérunt te mihi: quid fecísti? Respóndit Jesus: Regnum meum non est de hoc mundo. Si ex hoc mundo esset regnum meum, minístri mei útique decertárent, ut non tráderer Judǽis: nunc autem regnum meum non est hinc. Dixit ítaque ei Pilátus: Ergo Rex es tu? Respóndit Jesus: Tu dicis, quia Rex sum ego. Ego in hoc natus sum et ad hoc veni in mundum, ut testimónium perhíbeam veritáti: omnis, qui est ex veritáte, audit vocem meam.
}\switchcolumn\portugues{
\blettrine{N}{aquele} tempo, disse Pilatos a Jesus: «Tu és o rei dos Judeus?». Respondeu Jesus: «Tu dizes isso de ti mesmo, ou foram outros que to disseram de mim?». Respondeu Pilatos: «Sou, porventura, judeu? Foram os teus compatriotas e os pontífices que te entregaram nas minhas mãos. O que fizeste?». Jesus respondeu: «Meu reino não é deste mundo. Se o meu reino fosse deste mundo, certamente os meus ministros haviam de pelejar para que Eu não fosse entregue aos judeus; porém, Eu o declaro agora, o meu reino não é daqui». Disse-Lhe, então, Pilatos: «És, portanto, rei?». Jesus respondeu: «Tu dizes que Eu sou rei. Eu para isso nasci e para isso vim a este mundo, a fim de dar testemunho da verdade. Todo aquele que é da verdade, escuta a minha voz».
}\end{paracol}

\paragraphinfo{Ofertório}{Sl. 2, 8}
\begin{paracol}{2}\latim{
\rlettrine{P}{óstula} a me, et dabo tibi gentes hereditátem tuam, et possessiónem tuam términos terræ.
}\switchcolumn\portugues{
\rlettrine{P}{ede-me:} e dar-te-ei as nações como herança; e como domínio os confins da terra!
}\end{paracol}

\paragraph{Secreta}
\begin{paracol}{2}\latim{
\rlettrine{H}{óstiam} tibi, Dómine, humánæ reconciliatiónis offérimus: præsta, quǽsumus; ut, quem sacrifíciis præséntibus immolámus, ipse cunctis géntibus unitátis et pacis dona concédat, Jesus Christus Fílius tuus, Dóminus noster: Qui tecum \emph{\&c.}
}\switchcolumn\portugues{
\rlettrine{V}{os} oferecemos, Senhor, a hóstia da reconciliação humana, implorando-Vos que este mesmo Jesus Cristo, vosso Filho e Senhor nosso, o qual imolamos no presente sacrifício, conceda a todos os povos os benefícios da união e da paz: Ele, que convosco vive e reina \emph{\&c.}
}\end{paracol}

\paragraphinfo{Comúnio}{Sl. 28, 10 \& 11}
\begin{paracol}{2}\latim{
\rlettrine{S}{edébit} Dóminus Rex in ætérnum: Dóminus benedícet pópulo suo in pace.
}\switchcolumn\portugues{
\rlettrine{A}{ssentar-se-á,} eternamente, o Senhor como Rei e abençoará o seu povo em paz.
}\end{paracol}

\paragraph{Postcomúnio}
\begin{paracol}{2}\latim{
\rlettrine{I}{mmortalitátis} alimóniam consecúti, quǽsumus, Dómine: ut, qui sub Christi Regis vexíllis militáre gloriámur, cum ipso, in cœlésti sede, júgiter regnáre póssimus: Qui tecum \emph{\&c.}
}\switchcolumn\portugues{
\rlettrine{T}{endo} nós alcançado o alimento da imortalidade, Vos suplicamos, Senhor, permiti que possamos reinar perpetuamente na celestial mansão, juntamente com Jesus Cristo, sob cujas bandeiras nos gloriamos de militar: Ele, que, sendo Deus, convosco vive e reina \emph{\&c.}
}\end{paracol}

\subsectioninfo{S. Paulo da Cruz, Conf.}{28 de Abril}

\paragraphinfo{Intróito}{Gl. 2, 19-20}
\begin{paracol}{2}\latim{
\rlettrine{C}{hristo} confíxus sum Cruci: vivo autem, jam non ego; vivit vero in me Christus: in fide vivo Filii Dei, qui diléxit me, et trádidit semetípsum pro me, allelúja. allelúja. \emph{Ps. 40, 2} Beátus, qui intéllegit super egénum et páuperem: in die mala liberábit eum Dóminus.
℣. Gloria Patri \emph{\&c.}
}\switchcolumn\portugues{
\rlettrine{E}{stou} pregado na Cruz com Cristo; e vivo, mas já não sou eu que vivo: é Cristo que vive em mim. Vivo na fé do Filho de Deus, que me amou e se entregou voluntariamente por mim, aleluia, aleluia. \emph{Sl. 40, 2} Bem-aventurado aquele que é cuidadoso para com o pobre e o indigente, pois o Senhor o livrará no dia infeliz.
℣. Glória ao Pai \emph{\&c.}
}\end{paracol}

\paragraph{Oração}
\begin{paracol}{2}\latim{
\rlettrine{D}{ómine} Jesu Christe, qui, ad mystérium Crucis prædicándum, sanctum Paulum singulári caritáte donásti, et per eum novam in Ecclésia famíliam floréscere voluísti: ipsíus nobis intercessióne concéde; ut, passiónem tuam júgiter recoléntes in terris, ejúsdem fructum cónsequi mereámur in cœlis: Qui vivis et regnas \emph{\&c.}
}\switchcolumn\portugues{
\rlettrine{S}{enhor} Jesus Cristo, que destes a S. Paulo uma caridade singular para pregar o mistério da Cruz e que por seu intermédio quisestes que florescesse na Igreja uma nova família, concedei-nos por sua intercessão que, recordando-nos sempre, na terra, da vossa Paixão, mereçamos saborear os seus frutos no céu. Ó Vós, que viveis e \emph{\&c.}
}\end{paracol}

\paragraphinfo{Epístola}{1. Cor. 1, 17-25}
\begin{paracol}{2}\latim{
Lectio Epístolæ beati Pauli Apostoli ad Corinthios.
}\switchcolumn\portugues{
Lição da Ep.ª do B. Ap.º Paulo aos Coríntios.
}\switchcolumn*\latim{
\rlettrine{F}{ratres:} Non misit me Christus baptizáre, sed evangelizáre: non in sapiéntia verbi, ut non evacuétur Crux Christi. Verbum enim Crucis pereúntibus quidem stultítia est: iis autem, qui salvi fiunt, id est nobis, Dei virtus est. Scriptum est enim: Perdam sapiéntiam sapiéntium et prudéntiam prudéntium reprobábo. Ubi sápiens? ubi scriba? ubi conquisítor hujus sǽculi? Nonne stultam fecit Deus sapiéntiam hujus mundi? Nam quia in Dei sapiéntia non cognóvit mundus per sapiéntiam Deum: plácuit Deo per stultítiam prædicatiónis salvos fácere credéntes. Quóniam et Judǽi signa petunt et Græci sapiéntiam quærunt: nos autem prædicámus Christum crucifíxum: Judǽis quidem scándalum, géntibus autem stultítiam, ipsis autem vocátis Judǽis atque Græcis Christum Dei virtútem et Dei sapiéntiam: quia, quod stultum est Dei, sapiéntius est homínibus: et, quod infírmum est Dei, fórtius est homínibus.
}\switchcolumn\portugues{
\rlettrine{M}{eus} irmãos: Cristo me não mandou baptizar, mas pregar o Evangelho; não, porém, simplesmente segundo a sabedoria das palavras, para não tornar vã a Cruz de Cristo. Com efeito, a doutrina da Cruz é loucura para aqueles que se perdem; mas para aqueles que se salvam, isto é, para nós, é virtude de Deus; pois está escrito: «Eu destruirei a sabedoria dos sábios e reprovarei a prudência dos prudentes». Onde está o sábio? Onde está o doutor da lei? Onde está o investigador deste mundo? Porventura não mostrou Deus que a sabedoria deste mundo é loucura? Na verdade, ainda que na sabedoria de Deus o mundo não conheceu Deus pela sua sabedoria, contudo aprouve a Deus pela loucura da pregação salvar aqueles que acreditassem n’Ele. Enquanto que os judeus pedem milagres e os gregos procuram a sabedoria, nós pregamos Cristo crucificado, que é, na verdade, escândalo para os judeus e loucura para os pagãos, mas para os que são chamados, quer judeus, quer gregos, é poder de Deus e sabedoria de Deus; pois o que parece estulto em Deus é mais sábio que a sabedoria dos homens; e o que parece fraco em Deus é mais forte que os homens.
}\end{paracol}

\begin{paracol}{2}\latim{
Allelúja, allelúja. ℣. \emph{2 Cor. 5, 15} Pro ómnibus mórtuus est Christus: ut, et qui vivunt, jam non sibi vivant, sed ei, qui pro ipsis mórtuus est, et resurréxit. Allelúja. ℣. \emph{Rom. 8, 17} Si fílii, et herédes: heródes quidem Dei, coherédes autem Christi: si tamen compátimur, ut et conglorificémur. Allelúja.
}\switchcolumn\portugues{
Aleluia, aleluia. ℣. \emph{2 Cor. 5, 15} Cristo morreu por todos, a fim de que aqueles que vivem não vivam já para si, mas para Aquele que morreu e ressuscitou por eles. Aleluia. ℣. \emph{Rm. 8, 17} Se somos filhos, somos também herdeiros: herdeiros de Deus e co-herdeiros de Cristo. Se, pois, sofremos com Ele, com Ele seremos glorificados. Aleluia.
}\end{paracol}

\paragraphinfo{Evangelho}{Página \pageref{tito}}

\paragraphinfo{Ofertório}{Ef. 5, 2}
\begin{paracol}{2}\latim{
\rlettrine{A}{mbuláte} in dilectióne, sicut et Christus diléxit nos, et trádidit semetípsum pro nobis oblatiónem et hóstiam Deo in odórem suavitátis, allelúja.
}\switchcolumn\portugues{
\rlettrine{V}{ivei} no amor, assim como Cristo nos amou e se entregou voluntariemente por nós, como oblação e vítima oferecida a Deus em suave odor. Aleluia.
}\end{paracol}

\paragraph{Secreta}
\begin{paracol}{2}\latim{
\rlettrine{C}{œléstem} nobis, Dómine, prǽbeant mystéria hæc passiónis et mortis tuæ fervórem: quo sanctus Paulus, ea offeréndo, corpus suum hóstiam vivéntem, sanctam tibíque placéntem exhíbuit: Qui vivis et regnas \emph{\&c.}
}\switchcolumn\portugues{
\rlettrine{P}{ermiti,} Senhor, que estes mistérios da vossa Paixão e morte nos inspirem o fervor celestial com que S. Paulo, celebrando-os, ofereceu o seu corpo como uma hóstia viva, santa e agradável a vossos olhos. Ó Vós, que, sendo Deus \emph{\&c.}
}\end{paracol}

\paragraphinfo{Comúnio}{1. Pe. 4, 13}
\begin{paracol}{2}\latim{
\rlettrine{C}{ommunicántes} Christi passiónibus gaudéte, ut in revelatióne glóriæ ejus gaudeátis exsultántes, allelúja.
}\switchcolumn\portugues{
\rlettrine{A}{legrai-vos,} participando nos sofrimentos de Cristo, a fim de que, quando for a manifestação da sua glória, vos regozijeis em transportes de alegria, aleluia.
}\end{paracol}

\paragraph{Postcomúnio}
\begin{paracol}{2}\latim{
\rlettrine{S}{úmpsimus,} Dómine, divínum sacraméntum, imménsæ caritátis tuæ memoriále perpétuum: tríbue, quǽsumus; ut, sancti Pauli méritis et imitatióne, aquam de fóntibus tuis hauriámus in vitam ætérnam saliéntem, et tuam sacratíssimam passiónem córdibus nostris impréssam móribus et vita teneámus: Qui vivis \emph{\&c.}
}\switchcolumn\portugues{
\rlettrine{R}ecebemos,{} Senhor, este divino sacramento memorial perpétuo da vossa infinita caridade; e concedei-nos, Vos imploramos, que pelos méritos de S. Paulo e imitando o seu exemplo, bebamos nas fontes da vossa graça a água que brota até à vida eterna; e pela nossa vida e costumes conservemos sempre a vossa santíssima Paixão impressa nos nossos corações. Ó Vós, que viveis e \emph{\&c.}
}\end{paracol}

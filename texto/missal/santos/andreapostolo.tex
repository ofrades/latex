\subsectioninfo{S. André, Apóstolo}{30 de Novembro}\label{andreapostolo}

\paragraphinfo{Intróito}{Sl. 138, 17}
\begin{paracol}{2}\latim{
\rlettrine{M}{ihi} autem nimis honoráti sunt amíci tui, Deus: nimis confortátus est principatus eórum. \emph{Ps. ibid., 1-2} Dómine, probásti me et cognovísti me: tu cognovísti sessiónem meam et resurrectiónem meam.
℣. Gloria Patri \emph{\&c.}
}\switchcolumn\portugues{
\rlettrine{V}{ejo,} ó Deus, que honrastes largamente os vossos amigos: e que seu poder se fortaleceu extraordinariamente. \emph{Sl. ibid., 1-2} Senhor, perscrutastes o meu íntimo e ficastes conhecendo-me: ficastes conhecendo quando me deito e quando me levanto.
℣. Glória ao Pai \emph{\&c.}
}\end{paracol}

\paragraph{Oração}
\begin{paracol}{2}\latim{
\rlettrine{M}{ajestátem} tuam, Dómine, supplíciter exorámus: ut, sicut Ecclésiæ tuæ beátus Andréas Apóstolus éxstitit prædicátor et rector; ita apud te sit pro nobis perpétuus intercéssor. Per Dóminum \emph{\&c.}
}\switchcolumn\portugues{
\rlettrine{H}{umildemente} suplicamos à vossa divina majestade que, assim como o Apóstolo André foi pregador e guia da vossa Igreja, assim também interceda por nós perpetuamente junto de Vós. Por nosso Senhor \emph{\&c.}
}\end{paracol}

\paragraphinfo{Epístola}{Rm. 10, 10-18}
\begin{paracol}{2}\latim{
Léctio Epístolæ beáti Pauli Apóstoli ad Romános.
}\switchcolumn\portugues{
Lição da Ep.ª do B. Ap.º Paulo aos Romanos.
}\switchcolumn*\latim{
\rlettrine{F}{ratres:} Corde enim créditur ad justítiam: ore autem conféssio fit ad salútem. Dicit enim Scriptúra: Omnis, qui credit in illum, non confundétur. Non enim est distínctio Judǽi et Græci: nam idem Dóminus ómnium, dives in omnes, qui ínvocant illum. Omnis enim, quicúmque invocáverit nomen Dómini, salvus erit. Quómodo ergo invocábunt, in quem non credidérunt? Aut quómodo credent ei, quem non audiérunt? Quómodo autem áudient sine prædicánte? Quómodo vero prædicábunt, nisi mittántur? sicut scriptum est: Quam speciósi pedes evangelizántium pacem, evangelizándum bona! Sed non omnes obǿdiunt Evangelio. Isaías enim dicit: Dómine, quis crédidit audítui nostro? Ergo fides ex audítu, audítus autem per verbum Christi. Sed dico: Numquid non audiérunt? Et quidem in omnem terram exívit sonus eórum, et in fines orbis terræ verba eórum.
}\switchcolumn\portugues{
\rlettrine{M}{eus} irmãos: Com o coração se crê para chegar à justiça; e com a boca se faz profissão de fé para alcançar a salvação. Com efeito diz a Escritura: «Todo aquele que crê n’Ele não será confundido», porque não há diferença entre judeu e grego, visto que não há senão um só e o mesmo Senhor de todos, que é rico para todos os que O invocam; pois todo o que invocar o nome do Senhor será salvo. Porém como invocarão Aquele em quem não acreditam? E como hão-de crer sem haverem ouvido falar d’Ele? E como se ouvirá falar, se não houver pregadores? E como haverá pregadores se não tiverem sido enviados? Pois está escrito: «Como são preciosos os pés daqueles que evangelizam a paz, que evangelizam a felicidade!». Mas nem todos obedecem ao Evangelho. Isaías disse: «Senhor, quem acreditou na nossa pregação?». Assim, a fé nasce da pregação ouvida, e a pregação vem da palavra de Cristo. E eu vos pergunto: «Porventura eles a não ouviram?». Sim, por certo: pois por toda a terra se espalhou a sua voz, e suas palavras soaram até aos extremos do mundo!
}\end{paracol}

\paragraphinfo{Gradual}{Sl. 44, 17-18}
\begin{paracol}{2}\latim{
\rlettrine{C}{onstítues} eos príncipes super omnem terram: mémores erunt nóminis tui, Dómine. ℣. Pro pátribus tuis nati sunt tibi fílii: proptérea pópuli confitebúntur tibi.
}\switchcolumn\portugues{
\rlettrine{S}{erão} constituídos príncipes em toda a terra e perpetuarão a glória do vosso nome, Senhor! ℣. Para substituir os vossos pais nasceram-vos filhos; por isso os povos Vos louvarão eternamente.
}\switchcolumn*\latim{
Allelúja, allelúja. ℣. Diléxit Andréam Dóminus in odórem suavitátis. Allelúja.
}\switchcolumn\portugues{
Aleluia, aleluia. ℣. O Senhor amou André, como um perfume suave. Aleluia.
}\end{paracol}

\paragraphinfo{Evangelho}{Mt. 4, 18-22}
\begin{paracol}{2}\latim{
\cruz Sequéntia sancti Evangélii secúndum Matthǽum.
}\switchcolumn\portugues{
\cruz Continuação do santo Evangelho segundo S. Mateus.
}\switchcolumn*\latim{
\blettrine{I}{n} illo témpore: Ambulans Jesus juxta mare Galilǽæ, vidit duos fratres, Simónem, qui vocátur Petrus, et Andréam fratrem ejus, mitténtes rete in mare (erant enim piscatóres), et ait illis: Veníte post me, et fáciam vos fíeri piscatóres hóminum. At illi contínuo, relíctis rétibus, secúti sunt eum. Et procédens inde, vidit álios duos fratres, Jacóbum Zebedǽi et Joánnem, fratrem ejus, in navi cum Zebedǽo patre eórum reficiéntes rétia sua: et vocávit eos. Illi autem statim, relíctis rétibus et patre, secúti sunt eum.
}\switchcolumn\portugues{
\blettrine{N}{aquele} tempo, caminhando Jesus junto ao mar da Galileia, viu dois irmãos: Simão, que se chama Pedro, e André, irmão deste, os quais lançavam as redes no mar, pois eram pescadores. Disse-lhes Jesus: «Vinde comigo e far-vos-ei pescadores de homens». Imediatamente eles, deixando as redes, O seguiram. Continuando Jesus a andar, encontrou mais adiante em uma barca, consertando as redes, outros dois irmãos: Tiago, filho de Zebedeu, e João, seu irmão, com seu pai Zebedeu. Então chamou-os. Logo, no mesmo instante, eles, deixando as redes e o pai, seguiram-n’O.
}\end{paracol}

\paragraphinfo{Ofertório}{Sl. 138, 17}
\begin{paracol}{2}\latim{
\rlettrine{M}{ihi} autem nimis honoráti sunt amíci tui, Deus: nimis confortátus est principátus eórum.
}\switchcolumn\portugues{
\rlettrine{V}{ejo,} ó Deus, que honrastes largamente os vossos amigos; por isso o seu poder se fortaleceu extraordinariamente.
}\end{paracol}

\paragraph{Secreta}
\begin{paracol}{2}\latim{
\rlettrine{S}{acrificium} nostrum tibi, Dómine, quǽsumus, beáti Andréæ Apóstoli precátio sancta concíliet: ut, in cujus honóre sollémniter exhibétur, ejus méritis efficiátur accéptum. Per Dóminum \emph{\&c.}
}\switchcolumn\portugues{
\rlettrine{S}{enhor,} permiti que a oração do B. Apóstolo André Vos torne agradável este nosso sacrifício, a fim de que seja aceite, em virtude dos méritos daquele em cuja honra Vos é solenemente oferecido. Por nosso Senhor \emph{\&c.}
}\end{paracol}

\paragraphinfo{Comúnio}{Mt. 4, 19-20}
\begin{paracol}{2}\latim{
\rlettrine{V}{eníte} post me: fáciam vos fíeri piscatóres hóminum; at illi contínuo, relíctis rétibus, secúti sunt Dóminum.
}\switchcolumn\portugues{
\rlettrine{V}{inde} comigo: e far-vos-ei pescadores de homens. Imediatamente eles, deixando as redes, seguiram o Senhor.
}\end{paracol}

\paragraph{Postcomúnio}
\begin{paracol}{2}\latim{
\rlettrine{S}{úmpsimus,} Dómine, divína mystéria, beáti Andréæ Apóstoli festivitáte lætántes: quæ, sicut tuis Sanctis ad glóriam, ita nobis, quǽsumus, ad véniam prodésse perfícias. Per Dóminum \emph{\&c.}
}\switchcolumn\portugues{
\rlettrine{S}{enhor,} que os divinos mistérios, que com alegria recebemos nesta festa do B. André, servindo para a glória dos vossos Santos, nos alcancem, também, o perdão das nossas culpas. Por nosso Senhor \emph{\&c.}
}\end{paracol}

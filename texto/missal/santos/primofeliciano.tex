\subsectioninfo{S. S. Primo e Feliciano, Mártires}{9 de Junho}

\textit{Como na Missa Sapiéntiam sanctórum, página \pageref{muitosmartires2}, excepto:}

\paragraph{Oração}
\begin{paracol}{2}\latim{
\rlettrine{F}{ac} nos, quǽsumus. Dómine, sanctórum Martyrum tuórum Primi et Feliciáni semper festa sectári: quorum suffrágiis protectiónis tuæ dona sentiámus. Per Dóminum \emph{\&c.}
}\switchcolumn\portugues{
\rlettrine{P}{ermiti,} Senhor, Vos rogamos, que celebremos sempre fielmente a festa dos vossos Santos Mártires Primo e Feliciano, a fim de que pela sua intercessão sintamos os benefícios da vossa protecção. Por nosso Senhor \emph{\&c.}
}\end{paracol}

\paragraphinfo{Gradual}{Sl. 88, 6 \& 2}
\begin{paracol}{2}\latim{
\rlettrine{C}{onfitebúntur} cœli mirabília tua, Dómine: etenim veritátem tuam in ecclésia sanctórum. ℣. Misericórdias tuas, Dómine, in ætérnum cantábo: in generatióne et progénie.
}\switchcolumn\portugues{
\qlettrine{Q}{ue} os céus publiquem as vossas maravilhas, Senhor! Que se publique também na assembleia dos santos a vossa fidelidade. ℣. Cantarei eternamente as vossas misericórdias, Senhor! Sim, de geração em geração eu as cantarei!
}\switchcolumn*\latim{
Allelúja, allelúja. ℣. Hæc est vera fratérnitas, quæ vicit mundi crímina: Christum secúta est, ínclita tenens regna cœléstia. Allelúja.
}\switchcolumn\portugues{
Aleluia, aleluia. ℣. Esta é a verdadeira fraternidade que venceu os crimes do mundo. Ela seguiu Cristo, possuindo gloriosamente o reino celestial. Aleluia.
}\end{paracol}

\paragraphinfo{Evangelho}{Página \pageref{pauloeremita}}

\paragraphinfo{Ofertório}{Sl. 67, 36}
\begin{paracol}{2}\latim{
\rlettrine{M}{irábilis} Deus in Sanctis suis: Deus Israël, ipse dabit virtútem et fortitúdinem plebi suæ: benedíctus Deus, allelúja.
}\switchcolumn\portugues{
\rlettrine{D}{eus} é admirável em seus santos. É o Deus de Israel quem dá força e coragem ao seu povo. Bendito seja Deus, aleluia.
}\end{paracol}

\paragraph{Secreta}
\begin{paracol}{2}\latim{
\rlettrine{F}{iat} tibi, quǽsumus, Dómine, hóstia sacránda placábilis, pretiósi celebritáte martýrii: quæ et peccáta nostra puríficet, et tuórum tibi vota concíliet famulórum. Per Dóminum \emph{\&c.}
}\switchcolumn\portugues{
\rlettrine{S}{enhor,} que esta hóstia, que vai ser consagrada na celebração deste precioso mystério, Vos aplaque; e, Vos suplicamos, que ela apague os nossos pecados e Vos torne agradáveis os votos dos vossos servos. Por nosso Senhor \emph{\&c.}
}\end{paracol}

\paragraphinfo{Comúnio}{Jo. 15, 16}
\begin{paracol}{2}\latim{
\rlettrine{E}{go} vos elegi de mundo, ut eátis et fructum afferátis: et fructus vester máneat.
}\switchcolumn\portugues{
\rlettrine{E}{u} vos escolhi no mundo, para que possais ir e alcanceis fruto; e que esse fruto permaneça.
}\end{paracol}

\paragraph{Postcomúnio}
\begin{paracol}{2}\latim{
\qlettrine{Q}{uǽsumus,} omnípotens Deus: ut sanctórum Mártyrum tuórum Primi et Feliciáni cœléstibus mystériis celebráta sollémnitas, indulgéntiam nobis tuæ propitiatiónis acquírat. Per Dóminum \emph{\&c.}
}\switchcolumn\portugues{
\slettrine{Ó}{} Deus omnipotente, Vos suplicamos, permiti que estes celestiais mystérios, com os quais celebramos a solenidade dos vossos Santos Mártires Primo e Feliciano, nos alcancem o perdão da vossa misericórdia. Por nosso Senhor \emph{\&c.}
}\end{paracol}

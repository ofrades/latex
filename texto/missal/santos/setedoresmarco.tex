\subsectioninfo{Sete Dores da B. V. Maria}{15 de Setembro}\label{setedoresmarco}

\paragraphinfo{Intróito}{Jo. 19, 25}
\begin{paracol}{2}\latim{
\rlettrine{S}{tabant} juxta Crucem Jesu Mater ejus, et soror Matris ejus, María Cléophæ, et Salóme et María Magdaléne. \emph{ibid., 26-27} Múlier, ecce fílius tuus: dixit Jesus; ad discípulum autem: Ecce Mater tua.
℣. Gloria Patri \emph{\&c.}
}\switchcolumn\portugues{
\rlettrine{E}{stavam} junto à Cruz de Jesus, sua Mãe, a irmã de sua Mãe, Maria, mulher de Cléofas, Salomé e Maria Madalena. \emph{ibid., 26-27} «Mulher disse Jesus eis o vosso Filho»; e, dirigindo-se ao discípulo, disse: «Eis a tua Mãe».
℣. Glória ao Pai \emph{\&c.}
}\end{paracol}

\paragraph{Oração}
\begin{paracol}{2}\latim{
\rlettrine{D}{eus,} in cujus passióne, secúndum Simeónis prophétiam, dulcíssimam ánimam gloriósæ Vírginis et Matris Maríæ dolóris gladius pertransívit: concéde propítius; ut, qui transfixiónem ejus et passiónem venerándo recólimus, gloriósis méritis et précibus ómnium Sanctórum Cruci fidéliter astántium intercedéntibus, passiónis tuæ efféctum felícem consequámur: Qui vivis \emph{\&c.}
}\switchcolumn\portugues{
\slettrine{Ó}{} Deus, em cuja Paixão, segundo a profecia de Simeão, uma espada de dor traspassou a terníssima alma da Virgem Maria, vossa Mãe, concedei-nos propício que, celebrando devotamente a memória da sua Transfixão e das suas Dores, alcancemos o feliz efeito da vossa Paixão, pelos gloriosos méritos e preces de todos os Santos, que fielmente permaneceram junto à Cruz. Ó Vós, que \emph{\&c.}
}\end{paracol}

\paragraphinfo{Epístola}{Jdt. 13, 22 \& 23-25}
\begin{paracol}{2}\latim{
Léctio libri Judith.
}\switchcolumn\portugues{
Lição do Livro de Judite.
}\switchcolumn*\latim{
\rlettrine{B}{enedíxit} te Dóminus in virtúte sua, quia per te ad níhilum redégit inimícos nostros. Benedícta es tu, fília, a Dómino, Deo excélso, præ ómnibus muliéribus super terram. Benedíctus Dóminus, qui creávit cœlum et terram: quia hódie nomen tuum ita magnificávit, ut non recédat laus tua de ore hóminum, qui mémores fúerint virtútis Dómini in ætérnum, pro quibus non pepercísti ánimæ tuæ propter angústias et tribulatiónem géneris tui, sed subvenísti ruínæ ante conspéctum Dei nostri.
}\switchcolumn\portugues{
\rlettrine{O}{} Senhor vos abençoou com seu poder, aniquilando por vosso intermédio todos nossos inimigos. Bendita sois vós entre as mulheres da terra, ó filha do Senhor, Deus Altíssimo. Bendito seja o Senhor, que criou o céu e a terra; pois de tal modo glorificou o vosso nome que todos os homens vos louvarão, lembrando-se para sempre do nome do Senhor. Não poupastes a vossa vida, sentindo a aflição e as angústias do povo, mas remediastes a sua ruína, perante o nosso Deus.
}\end{paracol}

\paragraph{Gradual}
\begin{paracol}{2}\latim{
\rlettrine{D}{olorósa} et lacrimábilis es, Virgo María, stans juxta Crucem Dómini Jesu, Fílii tui, Redemptóris. ℣. Virgo Dei Génetrix, quem totus non capit orbis, hoc crucis fert supplícium, auctor vitæ factus homo.
}\switchcolumn\portugues{
\rlettrine{C}{heia} de dores e de lágrimas, ó Virgem Maria, estavas junto à Cruz do Senhor Jesus, vosso Filho e Redentor. ℣. Ó Virgem, Mãe de Deus, Aquele a quem o mundo não pode conter o autor da vida feito homem sofre este suplício da cruz!
}\switchcolumn*\latim{
Allelúja, allelúja. ℣. Stabat sancta María, cœli Regína et mundi Dómina, juxta Crucem Dómini nostri Jesu Christi dolorósa.
}\switchcolumn\portugues{
Aleluia, aleluia. ℣. Estava doloroso, junto à Cruz de nosso Senhor Jesus Cristo, a Rainha do céu e Soberana do mundo, Santa Maria.
}\end{paracol}

\paragraph{Trato}
\begin{paracol}{2}\latim{
\rlettrine{S}{tabat} sancta María, cœli Regína et mundi Dómina, juxta Crucem Dómini nostri Jesu Christi dolorósa. ℣. \emph{Thren. 1, 12} O vos omnes, qui tránsitis per viam, atténdite et vidéte, si est dolor sicut dolor meus.
}\switchcolumn\portugues{
\rlettrine{E}{stava,} dolorosa, Santa Maria, Rainha do céu e Senhora do mundo, junto à Cruz de nosso Senhor Jesus Cristo. ℣. \emph{Lm. 1, 12} Ó vós todos, que passais pelo mundo, atendei e vede se há dor semelhante à minha dor!
}\end{paracol}

\subsubsection{Sequência}\label{stabatmaterdolorosa}
\begin{paracol}{2}\latim{
\rlettrine{S}{tabat} Mater dolorósa iuxta Crucem lacrimósa, dum pendébat Fílius.
}\switchcolumn\portugues{
\rlettrine{E}{stava} dolorosa e lacrimosa a Mãe junto à Cruz, donde pendia o Filho.
}\switchcolumn*\latim{
Cuius ánimam geméntem, con­tris­tátam et doléntem per­tran­sívit gládius.
}\switchcolumn\portugues{
Sua alma, gemendo triste e aflita, foi traspassada por uma espada.
}\switchcolumn*\latim{
O quam tristis et afflícta fuit illa benedícta, mater Unigéniti!
}\switchcolumn\portugues{
Oh! Quão triste e aflita ficou aquela Mãe bendita do Filho Unigénito!
}\switchcolumn*\latim{
Quæ mærébat et dolébat, pia Mater, dum vidébat nati pœnas íncliti.
}\switchcolumn\portugues{
Ela, piedosa Mãe, gemia e chorava, sentido as penas do divino Filho!
}\switchcolumn*\latim{
Quis est homo qui non fleret, matrem Christi si vidéret in tanto supplício?
}\switchcolumn\portugues{
Qual o homem que não chorará, vendo a Mãe de Cristo em tal suplício?
}\switchcolumn*\latim{
Quis non posset contristári Christi Matrem contemplári doléntem cum Fílio?
}\switchcolumn\portugues{
Quem poderá contemplar sem tristeza a Mãe de Cristo, afligindo-se com seu Filho?
}\switchcolumn*\latim{
Pro peccátis suæ gentis vidit Jesum in torméntis, et flagéllis súbditum.
}\switchcolumn\portugues{
Por causa dos pecados do povo, Ela via Jesus entregue aos sofrimentos e flagelado pelos açoites.
}\switchcolumn*\latim{
Vidit suum dulcem Natum moriéndo desolátum, dum emísit spíritum.
}\switchcolumn\portugues{
Via este terno Filho morrer desolado, entregar o espírito!
}\switchcolumn*\latim{
Éia, Mater, fons amóris me sentíre vim dolóris fac, ut tecum lúgeam.
}\switchcolumn\portugues{
Ó Mãe, fonte de amor, fazei-me sentir a violência das vossas dores, a fim de que chore convosco.
}\switchcolumn*\latim{
Fac, ut árdeat cor meum in amándo Christum Deum ut sibi compláceam.
}\switchcolumn\portugues{
Fazei que meu coração seja ardente em amar Cristo Deus, para que Lhe agrade.
}\switchcolumn*\latim{
Sancta Mater, istud agas, crucifíxi fige plagas cordi meo válide.
}\switchcolumn\portugues{
Ó Santa Mãe, gravai profundamente as chagas do Crucificado no meu coração.
}\switchcolumn*\latim{
Tui Nati vulneráti, tam dignáti pro me pati, pœnas mecum dívide.
}\switchcolumn\portugues{
Reparti comigo as dores do vosso Filho, que tanto se dignou sofrer por mim.
}\switchcolumn*\latim{
Fac me tecum píe flere, crucifíxo condolére, donec ego víxero.
}\switchcolumn\portugues{
Fazei-me chorar convosco e compartilhar perpétua e piedosamente os sofrimentos do Crucificado.
}\switchcolumn*\latim{
Iuxta Crucem tecum stare, et me tibi sociáre in planctu desídero.
}\switchcolumn\portugues{
Quero convosco ficar ao pé da Cruz e unir-me a vós nos gemidos.
}\switchcolumn*\latim{
Virgo vírginum præclara, mihi iam non sis amára, fac me tecum plángere.
}\switchcolumn\portugues{
Ó Virgem, ilustre entre as virgens, não rejeiteis a minha prece; deixai-me chorar convosco.
}\switchcolumn*\latim{
Fac, ut portem Christi mortem, passiónis fac consórtem, et plagas recólere.
}\switchcolumn\portugues{
Fazei que se imprima em mim a morte de Cristo; que compartilhe as suas dores e venere as suas chagas.
}\switchcolumn*\latim{
Fac me plagis vulnerári, fac me Cruce inebriári, et cruóre Fílii.
}\switchcolumn\portugues{
Fazei que, ferido com suas feridas, fique inebriado de amor à Cruz e ao sangue do vosso Filho.
}\switchcolumn*\latim{
Flammis ne urar succénsus, per te, Virgo, sim defénsus in die iudícii.
}\switchcolumn\portugues{
Que eu não seja consumido pelas chamas devoradoras e seja defendido por vós, ó Virgem, no dia do juízo.
}\switchcolumn*\latim{
Christe, cum sit hinc exíre, da per Matrem me veníre ad palmam victóriæ.
}\switchcolumn\portugues{
Ó Cristo, quando eu tiver que deixar o mundo, concedei-me por vossa Mãe que alcance a palma da vitória.
}\switchcolumn*\latim{
Quando corpus moriétur, fac, ut ánimæ donétur paradísi glória.
}\switchcolumn\portugues{
Quando o meu corpo morrer, dai à minha alma a glória do Paraíso.
}\switchcolumn*\latim{
℟. Amen.
}\switchcolumn\portugues{
℟. Amen.
}\end{paracol}

\paragraphinfo{Evangelho}{Página \pageref{missamaria4}}

\paragraphinfo{Ofertório}{Jr. 18, 20}
\begin{paracol}{2}\latim{
\rlettrine{R}{ecordáre,} Virgo, Mater Dei, dum stéteris in conspéctu Dómini, ut loquáris pro nobis bona, et ut avértat indignatiónem suam a nobis.
}\switchcolumn\portugues{
\slettrine{Ó}{} Virgem, Mãe de Deus, quando estiverdes na presença do Senhor, lembrai-Vos de nós; intercedei em nosso favor junto d’Ele e afastai de nós a sua indignação.
}\end{paracol}

\paragraph{Secreta}
\begin{paracol}{2}\latim{
\rlettrine{O}{fférimus} tibi preces et hóstias, Dómine Jesu Christe, humiliter supplicántes: ut, qui Transfixiónem dulcíssimi spíritus beátæ Maríæ, Matris tuæ, précibus recensémus; suo suorúmque sub Cruce Sanctórum consórtium multiplicáto piíssimo intervéntu, méritis mortis tuæ, méritum cum beátis habeámus: Qui vivis \emph{\&c.}
}\switchcolumn\portugues{
\rlettrine{S}{enhor} Jesus Cristo, Vos oferecemos as nossas preces e oblações e Vos rogamos humildemente que, recordando nas nossas orações a Transfixão da terníssima alma da B. Maria, vossa Mãe, e pelos seus rogos e piíssima intercessão dos Santos, que permaneceram ao pé da Cruz, tenhamos parte com os bem-aventurados nos méritos da vossa morte. Ó Vós, que, sendo Deus, viveis e \emph{\&c.}
}\end{paracol}

\paragraph{Comúnio}
\begin{paracol}{2}\latim{
\rlettrine{F}{elices} sensus beátæ Maríæ Vírginis, qui sine morte meruérunt martýrii palmam sub Cruce Dómini.
}\switchcolumn\portugues{
\rlettrine{D}{itosos} os sentidos da B. V. Maria, que sem serem feridos pela morte alcançaram a palma do martírio, ao pé da Cruz do Senhor.
}\end{paracol}

\paragraph{Postcomúnio}
\begin{paracol}{2}\latim{
\rlettrine{S}{acrifícia,} quæ súmpsimus, Dómine Jesu Christe, Transfixiónem Matris tuæ et Vírginis devóte celebrántes: nobis ímpetrent apud cleméntiam tuam omnis boni salutáris efféctum: Qui vivis \emph{\&c.}
}\switchcolumn\portugues{
\qlettrine{Q}{ue} os sacratíssimos dons com que nos alimentámos, Senhor Jesus Cristo, celebrando piedosamente a Transfixão da vossa Mãe, sempre Virgem, nos obtenham da vossa clemência o efeito de todo o bem salutar. Ó Vós, que, sendo Deus, viveis e \emph{\&c.}
}\end{paracol}

\subsectioninfo{S. Aleixo, Conf.}{17 de Julho}

\textit{Como na Missa Os justi, página \pageref{confessoresnaopontifices1}, excepto:}

\paragraphinfo{Epístola}{1 Tm. 6, 6-12}
\begin{paracol}{2}\latim{
Léctio Epístolæ beáti Pauli Apóstoli ad Timótheum.
}\switchcolumn\portugues{
Lição da Ep.ª do B. Ap.º Paulo aos Coríntios.
}\switchcolumn*\latim{
\rlettrine{C}{aríssime:} Est quæstus magnus píetas cum sufficiéntia. Nihil enim intúlimus in hunc mundum: haud dúbium, quod nec auférre quid póssumus. Habéntes autem aliménta, et quibus tegámur, his conténti simus. Nam qui volunt dívites fíeri, incídunt in tentatiónem et in láqueum diáboli, et desidéria multa inutília et nocíva: quæ mergunt hómines in intéritum et perditiónem. Radix enim ómnium malórum est cupíditas: quam quidam appeténtes, erravérunt a fide, et inseruérunt se dolóribus multis. Tu autem, o homo Dei, hæc fuge: sectáre vero justítiam, pietátem, fidem, caritátem, patiéntiam, mansuetúdinem. Certa bonum certámen fídei, apprehénde vitam ætérnam.
}\switchcolumn\portugues{
\rlettrine{C}{aríssimo:} É uma grande riqueza possuir a piedade e contentar-se com o suficiente. Na verdade, não trouxemos nada a este mundo e sem dúvida nada poderemos levar dele. Se, portanto, temos de comer e de vestir, estejamos contentes, pois aqueles que querem ser ricos caem nas tentações e nas ciladas do demónio e em muitos outros desejos inúteis e perniciosos, que arrastam o homem para a ruína e perdição. A raiz de todos os males é, na verdade, a cobiça; e, assim, alguns, guiando-se por ela, afastaram-se da fé e enredaram-se em muitas aflições. Porém tu, ó homem de Deus, foge destas coisas e procura a justiça, a piedade, a fé, a caridade, a paciência e a mansidão. Pugna com valor no bom combate da fé e alcança a vida eterna.
}\end{paracol}

\paragraphinfo{Evangelho}{Mt. 19, 27-29}
\begin{paracol}{2}\latim{
\cruz Sequéntia sancti Evangélii secúndum Matthǽum. 
}\switchcolumn\portugues{
\cruz Continuação do santo Evangelho segundo S. Mateus.
}\switchcolumn*\latim{
\blettrine{I}{n} illo témpore: Dixit Petrus ad Jesum: Ecce, nos relíquimus ómnia, et secúti sumus te: quid ergo erit nobis? Jesus autem dixit illis: Amen, dico vobis, quod vos, qui secúti estis me, in regeneratióne, cum séderit Fílius hóminis in sede majestátis suæ, sedébitis et vos super sedes duódecim, judicántes duódecim tribus Israël. Et omnis, qui relíquerit domum, vel fratres, aut soróres, aut patrem, aut matrem, aut uxórem, aut fílios, aut agros, propter nomen meum, céntuplum accípiet, et vitam ætérnam possidebit.
}\switchcolumn\portugues{
\blettrine{N}{aquele} Naquele tempo, disse Pedro a Jesus: «Eis que deixámos tudo e Vos seguimos. Que recompensa teremos por isso?». Jesus respondeu-lhe: «Em verdade vos digo: vós, que me seguistes, quando, no tempo da regeneração, o Filho do homem se assentar no trono da sua glória, também vos assentareis sobre doze tronos, para julgar as doze tribos de Israel. Todo aquele que deixar a sua casa, ou os seus irmãos, ou os seus campos, ou o seu pai, ou a sua mãe, ou a sua mulher por causa do meu nome receberá o cêntuplo e possuirá a vida eterna».
}\end{paracol}
\subsectioninfo{S. S. João e Paulo, Mártires}{26 de Junho}

\paragraphinfo{Intróito}{Página \pageref{vitomodestocrescencia}}

\paragraph{Oração}
\begin{paracol}{2}\latim{
\qlettrine{Q}{uǽsumus,} omnípotens Deus: ut nos gemináta lætítia hodiérnæ festivitátis excípiat, quæ de beatórum Joánnis et Pauli glorificatióne procédit; quos eadem fides et pássio vere fecit esse germános. Per Dóminum \emph{\&c.}
}\switchcolumn\portugues{
\slettrine{Ó}{} Senhor omnipotente, Vos rogamos, permiti que participemos da alegria desta dupla festa, neste dia em que são glorificados os B. B. João e Paulo, que a mesma fé e o mesmo martírio tornaram verdadeiramente irmãos. Por nosso Senhor \emph{\&c.}
}\end{paracol}

\paragraphinfo{Epístola}{Ecl. 44, 10-15}
\begin{paracol}{2}\latim{
Léctio libri Sapiéntiæ.
}\switchcolumn\portugues{
Lição do Livro da Sabedoria.
}\switchcolumn*\latim{
\rlettrine{H}{i} viri misericórdiæ sunt, quorum pietátes non defuérunt: cum semine eórum pérmanent bona, heréditas sancta nepótes eórum, et in testaméntis stetit semen eórum: et fílii eórum propter illos usque in ætérnum manent: semen eórum et glória eórum non derelinquétur. Córpora ipsórum in pace sepúlta sunt, et nomen eórum vivit in generatiónem et generatiónem. Sapiéntiam ipsórum narrent pópuli, et laudem eórum núntiet Ecclésia.
}\switchcolumn\portugues{
\rlettrine{S}{ão} homens de misericórdia e as obras da sua piedade não faltaram. Os bens, que deixaram à posteridade, permanecerão sempre. Seus descendentes constituirão uma herança sagrada; a sua raça manterá a aliança com Deus; e em virtude dela os seus filhos subsistirão eternamente, nunca mas acabando a sua geração, assim como a sua glória. Seus corpos foram sepultados em paz e o seu nome vive em todas as gerações. Que os povos, pois, publiquem a sua sabedoria e que a Igreja cante os seus louvores.
}\end{paracol}

\paragraphinfo{Gradual}{Sl. 132, 1-2}
\begin{paracol}{2}\latim{
\rlettrine{E}{cce,} quam bonum et quam jucúndum, habitáre fratres in unum! ℣. Sicut unguéntum in cápite, quod descéndit in barbam, barbam Aaron.
}\switchcolumn\portugues{
\rlettrine{A}{h!} como é bom e consolador que os irmãos habitem juntamente. ℣. É como o perfume espalhado na cabeça, que desceu sobre a barba de Aarão.
}\switchcolumn*\latim{
Allelúja, allelúja. ℣. Hæc est vera fratérnitas, quæ vicit mundi crímina: Christum secúta est, ínclita tenens regna cœléstia. Allelúja.
}\switchcolumn\portugues{
Aleluia, aleluia. ℣. Esta é a verdadeira fraternidade que venceu os crimes do mundo: ela seguiu Cristo e por isso possuirá gloriosamente o reino do céu. Aleluia.
}\end{paracol}

\paragraphinfo{Evangelho}{Página \pageref{muitosmartires3}}

\paragraphinfo{Ofertório}{Sl. 5, 12-13}
\begin{paracol}{2}\latim{
\rlettrine{G}{loriabúntur} in te omnes, qui díligunt nomen tuum, quóniam tu, Dómine, benedíces justo: Dómine, ut scuto bonæ voluntátis tuæ coronásti nos.
}\switchcolumn\portugues{
\rlettrine{T}{odos} aqueles que amam o vosso nome, Senhor, serão glorificados convosco, pois abençoais o justo. Vós o protegereis, Senhor, com vossa boa vontade, como se fora um escudo.
}\end{paracol}

\paragraph{Secreta}
\begin{paracol}{2}\latim{
\rlettrine{H}{óstias} tibi, Dómine, sanctórum Martyrum tuórum Joánnis et Pauli dicátas méritis, benígnus assúme: et ad perpétuum nobis tríbue proveníre subsídium. Per Dóminum nostrum \emph{\&c.}
}\switchcolumn\portugues{
\rlettrine{A}{ceitai} benigno, Senhor, as hóstias que Vos oferecemos pelos méritos dos vossos Santos Mártires João e Paulo e dignai-Vos, em virtude delas, fazer descer sobre nós o vosso perpétuo socorro. Por nosso Senhor \emph{\&c.}
}\end{paracol}

\paragraphinfo{Comúnio}{Sb. 3, 4, 5 \& 6}
\begin{paracol}{2}\latim{
\rlettrine{E}{t} si coram homínibus torménta passi sunt, Deus tentavit eos: tamquam aurum in fornáce probávit eos, et quasi holocáusta accépit eos.
}\switchcolumn\portugues{
\rlettrine{S}{e} sofreram tormentos diante dos homens, foi porque Deus os provou. Deus provou-os, como ao ouro, na fornalha, e recebeu-os, como hóstia de holocausto.
}\end{paracol}

\paragraph{Postcomúnio}
\begin{paracol}{2}\latim{
\rlettrine{S}{úmpsimus,} Dómine, sanctórum Martyrum tuórum Joánnis et Pauli sollémnia celebrántes, sacraménta cœléstia: præsta, quǽsumus; ut, quod temporáliter gérimus, ætérnis gáudiis consequámur. Per Dóminum \emph{\&c.}
}\switchcolumn\portugues{
\rlettrine{R}{ecebemos,} Senhor, os vossos celestiais sacramentos, celebrando a festa dos vossos Santos Mártires João e Paulo; e concedei-nos, Vos suplicamos, que alcancemos nos gozos eternos o que agora celebrámos. Por nosso Senhor \emph{\&c.}
}\end{paracol}

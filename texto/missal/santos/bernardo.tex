\subsectioninfo{S. Bernardo, Confo e Doutor}{20 de Agosto}

\textit{Como na Missa In médio Ecclésiae, página \pageref{doutores}, excepto:}

\paragraphinfo{Epístola}{Ecl. 39, 6-14}
\begin{paracol}{2}\latim{
Léctio libri Sapiéntiæ.
}\switchcolumn\portugues{
Lição da Ep.ª do B. Ap.º Paulo aos Coríntios.
}\switchcolumn*\latim{
\qlettrine{J}{ustus} cor suum tradet ad vigilándum dilúculo ad Dóminum, qui fecit illum, et in conspéctu Altíssimi deprecábitur. Apériet os suum in oratióne, et pro delíctis suis deprecábitur. Si enim Dóminus magnus volúerit, spíritu intellegéntiæ replébit illum: et ipse tamquam imbres mittet elóquia sapiéntiæ suæ, et in oratióne confitébitur Dómino: et ipse díriget consílium ejus et disciplínam, et in abscónditis suis consiliábitur. Ipse palam fáciet disciplínam doctrínæ suæ, et in lege testaménti Dómini gloriábitur. Collaudábunt multi sapiéntiam ejus, et usque in sǽculum non delébitur. Non recédet memória ejus, et nomen ejus requirétur a generatióne in generatiónem. Sapiéntiam ejus enarrábunt gentes, et laudem ejus enuntiábit ecclésia.
}\switchcolumn\portugues{
\rlettrine{O}{} justo aplicará o seu coração e vigiará desde o romper do dia para se unir ao Senhor, que o criou, e oferecer as suas preces ao Altíssimo. Abrirá a sua boca para orar e implorar o perdão dos seus pecados; pois, se o soberano Senhor quiser, enchê-lo-á com o espírito da inteligência. Então ele espalhará, como chuva, as palavras da sua sabedoria e abençoará o Senhor na sua oração. O Senhor inspirará os seus conselhos e instruções; e ele compreenderá os mistérios divinos. Publicará a doutrina, que tiver aprendido, e a sua glória será manter-se na lei da aliança com o Senhor. Sua sabedoria receberá louvor de muitos e não cairá no esquecimento. Sua memória se não apagará. Seu nome será honrado de geração em geração. As nações publicarão a sua sabedoria e a Igreja anunciará os seus louvores.
}\end{paracol}

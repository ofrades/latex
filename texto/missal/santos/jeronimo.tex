\subsectioninfo{S. Jerónimo, Presb. C. e Doutor}{30 de Setembro}

\textit{Como na Missa In médio Ecclésiae, página \pageref{doutores}, excepto:}

\paragraph{Oração}
\begin{paracol}{2}\latim{
\rlettrine{D}{eus,} qui Ecclésiæ tuæ in exponéndis sacris Scriptúris beátum Hierónymum, Confessórem tuum, Doctórem máximum providére dignátus es: præsta, quǽsumus; ut, ejus suffragántibus méritis, quod ore simul et ópere dócuit, te adjuvánte, exercére valeámus. Per Dóminum \emph{\&c.}
}\switchcolumn\portugues{
\slettrine{Ó}{} Deus, que para explicar as Sagradas Escrituras Vos dignastes prover a vossa Igreja com um eminente Doutor na pessoa do vosso B. confessor Jerónimo, concedei-nos, Vos suplicamos. que pelos sufrágios dos seus méritos possamos com o auxílio da vossa graça praticar aquilo que ele ensinou, tanto pelas palavras, como pelas acções. Por nosso Senhor \emph{\&c.}
}\end{paracol}

\paragraph{Secreta}
\begin{paracol}{2}\latim{
\rlettrine{D}{onis} cœléstibus da nobis, quǽsumus, Dómine, líbera tibi mente servíre: ut múnera, quæ deférimus, interveniénte beáto Hierónymo Confessóre tuo, et medélam nobis operéntur et glóriam. Per Dóminum \emph{\&c.}
}\switchcolumn\portugues{
\rlettrine{P}{ela} virtude destes dons, Senhor, concedei-nos a graça de Vos servirmos com inteira liberdade de espírito, a fim de que os dons, que Vos apresentamos, nos alcancem, por intercessão do vosso B. Confessor Jerónimo, a cura dos nossos males e a glória eterna. Por nosso Senhor \emph{\&c.}
}\end{paracol}

\paragraph{Postcomúnio}
\begin{paracol}{2}\latim{
\rlettrine{R}{epleti} alimónia cœlésti, quǽsumus, Dómine: ut, interveniénte beáto Hierónymo Confessóre tuo, misericórdiæ tuæ grátiam cónsequi mereámur. Per Dóminum nostrum \emph{\&c.}
}\switchcolumn\portugues{
\rlettrine{S}{aciados} com o alimento celestial, permiti, Senhor, Vos rogamos, que pela intercessão do vosso B. Confessor Jerónimo mereçamos conseguir a graça da vossa misericórdia. Por nosso Senhor \emph{\&c.}
}\end{paracol}
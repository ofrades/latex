\subsectioninfo{Santíssimo Nome de Maria}{12 de Setembro}

\textit{Como na Missa Salus autem, página \pageref{muitosmartires3}, excepto:}

\paragraph{Oração}
\begin{paracol}{2}\latim{
\rlettrine{C}{oncéde,} quǽsumus, omnípotens Deus: ut fidéles tui, qui sub sanctíssimæ Vírginis Maríæ Nómine et protectióne lætántur; ejus pia intercessióne a cunctis malis liberéntur in terris, et ad gáudia ætérna perveníre mereántur in cœlis. Per Dóminum \emph{\&c.}
}\switchcolumn\portugues{
\slettrine{Ó}{} Deus omnipotente, Vos rogamos, concedei, por sua intercessão, aos vossos fiéis, que se alegram com o nome e a protecção da SS. Virgem Maria, a graça de serem livres de todos os males terrenos e de merecerem a posse das celestiais alegrias eternas. Por nosso Senhor \emph{\&c.}
}\end{paracol}

\paragraph{Secreta}
\begin{paracol}{2}\latim{
\rlettrine{T}{ua,} Dómine, propitiatióne, et beátæ Maríæ semper Vírginis intercessióne, ad perpétuam atque præséntem hæc oblátio nobis profíciat prosperitátem et pacem. Per Dóminum \emph{\&c.}
}\switchcolumn\portugues{
\rlettrine{P}{ela} vossa misericórdia, Senhor, e pela intercessão da B. Maria, sempre Virgem, fazei que esta oblação nos assegure a prosperidade e a paz, agora e sempre. Por nosso Senhor \emph{\&c.}
}\end{paracol}

\paragraph{Postcomúnio}
\begin{paracol}{2}\latim{
\rlettrine{S}{umptis,} Dómine, salútis nostræ subsídiis: da, quǽsumus, beátæ Maríæ semper Vírginis patrocíniis nos úbique protegi; in cujus veneratióne hæc tuæ obtúlimus majestáti. Per Dóminum \emph{\&c.}
}\switchcolumn\portugues{
\rlettrine{H}{avendo} nós alcançado o poderoso auxílio da vossa salvação, Senhor, fazei, Vos imploramos, que sejamos protegidos com o patrocínio da B. Maria, sempre Virgem, em cuja honra oferecemos este sacrifício à vossa majestade. Por nosso Senhor \emph{\&c.}
}\end{paracol}

\subsectioninfo{S. José Calasans, Conf.}{27 de Agosto}

\paragraphinfo{Intróito}{Sl. 33, 12}
\begin{paracol}{2}\latim{
\rlettrine{V}{eníte,} fílii, audíte me: timorem Dómini docébo vos. \emph{Ps. ibid., 2} Benedícam Dóminum in omni témpore: semper laus ejus in ore meo.
℣. Gloria Patri \emph{\&c.}
}\switchcolumn\portugues{
\rlettrine{V}{inde,} meus filhos; escutai-me e vos ensinarei a temer o Senhor. \emph{Sl. ibid., 2} Bendirei o Senhor em todas as ocasiões: o seu louvor estará sempre na minha boca.
℣. Glória ao Pai \emph{\&c.}
}\end{paracol}

\paragraph{Oração}
\begin{paracol}{2}\latim{
\rlettrine{D}{eus,} qui per sanctum Joséphum Confessórem tuum, ad erudiéndam spíritu intellegéntiæ ac pietátis juventútem, novum Ecclésiæ tuæ subsídium providére dignátus es: præsta, quǽsumus; nos, ejus exémplo et intercessióne, ita fácere et docére, ut prǽmia consequámur ætérna. Per Dóminum \emph{\&c.}
}\switchcolumn\portugues{
\slettrine{Ó}{} Deus, que por S. José, vosso Confessor, Vos dignastes proporcionar à vossa Igreja novo auxílio para a formação da juventude na ciência e na piedade, concedei-nos, vos rogamos, que pelo seu exemplo e intercessão possamos proceder e ensinar de tal modo que consigamos alcançar os prémios eternos. Por nosso Senhor \emph{\&c.}
}\end{paracol}

\paragraphinfo{Epístola}{Página \pageref{martirnaopontifice1}}

\paragraphinfo{Gradual}{Sl. 36, 30-31}
\begin{paracol}{2}\latim{
\rlettrine{O}{s} justi meditábitur sapiéntiam, et lingua ejus loquétur judícium. ℣. Lex Dei ejus in corde ipsíus: et non supplantabúntur gressus ejus.
}\switchcolumn\portugues{
\rlettrine{A}{} boca do justo falará com sabedoria e a sua língua proclamará a justiça! ℣. A lei do seu Deus está no seu coração e os seus pés não tropeçarão.
}\switchcolumn*\latim{
Allelúja, allelúja. ℣. \emph{Jac. 1, 12} Beátus vir, qui suffert tentatiónem: quóniam, cum probátus fúerit, accípiet corónam vitæ. Allelúja.
}\switchcolumn\portugues{
Aleluia, aleluia. ℣. \emph{Jac. 1, 12} Bem-aventurado o varão que sofre a tentação, porque, depois de ser provado, receberá a coroa da vida. Aleluia.
}\end{paracol}

\paragraphinfo{Evangelho}{Mt. 18, 1-5}
\begin{paracol}{2}\latim{
\cruz Sequéntia sancti Evangélii secúndum Matthǽum.
}\switchcolumn\portugues{
\cruz Continuação do santo Evangelho segundo S. Mateus.
}\switchcolumn*\latim{
\blettrine{I}{n} illo témpore: Accessérunt discípuli ad Jesum, dicéntes: Quis, putas, major est in regno cœlorum? Et ádvocans Jesus parvulum, státuit eum in médio eórum et dixit: Amen, dico vobis, nisi convérsi fuéritis et efficiámini sicut párvuli, non intrabitis in regnum cœlórum. Quicúmque ergo humiliáverit se sicut párvulus iste, hic est major in regno cœlorum. Et qui suscéperit unum párvulum talem in nómine meo, me súscipit.
}\switchcolumn\portugues{
\blettrine{N}{aquele} tempo, aproximaram-se de Jesus os discípulos, dizendo-Lhe: «Qual pensais Vós que é o maior no reino dos céus?». E Jesus, havendo chamado um pequeno, colocou-o no meio deles e disse: «Em verdade vos digo: se vos não converteis e não vos tornais como os pequenos, não entrareis no reino dos céus. Todo aquele, pois, que se fizer pequeno, como este menino, esse é o maior no reino dos céus; e quem receber um pequeno, como este, em meu nome, recebe-me a mim mesmo».
}\end{paracol}

\paragraphinfo{Ofertório}{Sl. 9, 17}
\begin{paracol}{2}\latim{
\rlettrine{D}{esidérium} páuperum exaudívit Dóminus: præparatiónem cordis eórum audívit auris tua.
}\switchcolumn\portugues{
\rlettrine{O}{} Senhor ouviu o desejo dos pobres; os seus ouvidos escutaram a pureza do seu coração.
}\end{paracol}

\paragraph{Secreta}
\begin{paracol}{2}\latim{
\rlettrine{A}{ltáre} tuum, Dómine, munéribus cumulamus oblatis: ut ejus nobis fiant supplicatione propitia, cujus nos donasti patrocínio adjuvári. Per Dóminum \emph{\&c.}
}\switchcolumn\portugues{
\rlettrine{C}{umulamos,} Senhor, os vossos altares com dons e oblatas, a fim de que nos alcancem misericórdia pelas orações daquele que nos proporcionastes para nos auxiliar com seu patrocínio. Por nosso Senhor \emph{\&c.}
}\end{paracol}

\paragraphinfo{Comúnio}{Mc. 10, 14}
\begin{paracol}{2}\latim{
\rlettrine{S}{ínite} párvulos veníre ad me, et ne prohibuéritis eos: tálium est enim regnum Dei.
}\switchcolumn\portugues{
\rlettrine{D}{eixai} aproximarem-se de mim as criancinhas e as não afasteis, pois delas é o reino de Deus.
}\end{paracol}

\paragraph{Postcomúnio}
\begin{paracol}{2}\latim{
\rlettrine{S}{anctificáti,} Dómine, salutári mystério: quǽsumus; ut, intercedénte sancto Josépho Confessóre tuo, ad majus semper proficiámus pietátis increméntum. Per Dóminum nostrum \emph{\&c.}
}\switchcolumn\portugues{
\rlettrine{S}{antificados,} Senhor, com estes salutares mistérios, Vos rogamos que pela intercessão de S. José, vosso Confessor, permitais que aumentemos sempre cada vez mais a nossa piedade. Por nosso Senhor \emph{\&c.}
}\end{paracol}

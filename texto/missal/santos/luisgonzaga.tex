\subsectioninfo{S. Luís Gonzaga, Conf.}{21 de Junho}

\paragraphinfo{Intróito}{Sl. 8, 6}
\begin{paracol}{2}\latim{
\rlettrine{M}{inuísti} eum paulo minus ab Angelis: glória et honóre coronásti eum. \emph{Ps. 148, 2 } Laudáte Dóminum, omnes Angeli ejus: laudáte eum, omnes virtútes ejus.
℣. Gloria Patri \emph{\&c.}
}\switchcolumn\portugues{
\rlettrine{V}{ós} o criastes um pouco abaixo dos Anjos: Vós o coroastes com honra e glória. \emph{Sl. 148, 2 } Que todos os Anjos do Senhor louvem o Senhor; que todos os exércitos do Senhor louvem o Senhor.
℣. Glória ao Pai \emph{\&c.}
}\end{paracol}

\paragraph{Oração}
\begin{paracol}{2}\latim{
\rlettrine{C}{œléstium} donórum distribútor, Deus, qui in angélico júvene Aloísio miram vitæ innocéntiam pari cum pœniténtia sociásti: ejus méritis et précibus concéde; ut, innocéntem non secúti, pœniténtem imitémur. Per Dóminum nostrum \emph{\&c.}
}\switchcolumn\portugues{
\slettrine{Ó}{} Deus, distribuidor dos dons celestiais, que reunistes no angélico jovem Luís uma brilhante inocência de vida com uma não menos admirável penitência, concedei-nos pelos seus méritos e preces que o imitemos na penitência, já que o não acompanhamos na inocência. Por nosso Senhor \emph{\&c.}
}\end{paracol}

\paragraphinfo{Epístola}{Página \pageref{confessoresnaopontifices1}}

\paragraphinfo{Gradual}{Sl. 70, 5-6}
\begin{paracol}{2}\latim{
\rlettrine{D}{ómine,} spes mea a juventúte mea: in te confirmátus sum ex útero: de ventre matris meæ tu es protéctor meus. ℣. \emph{Ps. 40, 13} Me autem propter innocéntiam suscepísti: et confirmásti me in conspéctu tuo in ætérnum.
}\switchcolumn\portugues{
\rlettrine{D}{esde} a minha juventude, Senhor, sois a minha esperança; desde o meu nascimento fiquei ligado a Vós; e desde o seio de minha mãe sois o meu protector. ℣. \emph{Sl. 40, 13} Acolhestes-me por causa da minha inocência: e quisestes que ficasse sempre na vossa presença.
}\switchcolumn*\latim{
Allelúja, allelúja. ℣. \emph{Ps. 64, 5} Beátus, quem elegísti et assumpsísti: inhabitábit in átriis tuis. Allelúja.
}\switchcolumn\portugues{
Aleluia, aleluia. ℣. \emph{Sl. 64, 5} Bem-aventurado aquele que escolhestes e chamastes para junto de Vós, pois habitará nos átrios do Senhor. Aleluia.
}\end{paracol}

\paragraphinfo{Evangelho}{Mt. 22, 28-40}
\begin{paracol}{2}\latim{
\cruz Sequéntia sancti Evangélii secúndum Matthǽum.
}\switchcolumn\portugues{
\cruz Continuação do santo Evangelho segundo S. Mateus.
}\switchcolumn*\latim{
\blettrine{I}{n} illo témpore: Respóndens Jesus, ait sadducǽis: Errátis, nesciéntes Scriptúras neque virtútem Dei. In resurrectióne enim neque nubent neque nubéntur: sed erunt sicut Angeli Dei in cœlo. De resurrectióne autem mortuórum non legístis, quod dictum est a Deo dicénte vobis: Ego sum Deus Abraham et Deus Isaac et Deus Jacob? Non est Deus mortuórum, sed vivéntium. Et audiéntes turbæ, mirabántur in doctrína ejus. Pharisǽi autem audiéntes, quod siléntium imposuísset sadducǽis, convenérunt in unum: et interrogávit eum unus ex eis legis doctor, tentans eum: Magíster, quod est mandátum magnum in lege? Ait illi Jesus: Díliges Dóminum, Deum tuum, ex toto corde tuo, et in tota ánima tua, et in tota mente tua. Hoc est máximum et primum mandátum. Secúndum autem símile est huic: Díliges próximum tuum, sicut teípsum. In his duóbus mandátis univérsa lex pendet et Prophétæ.
}\switchcolumn\portugues{
\blettrine{N}{aquele} tempo, respondendo Jesus aos saduceus, disse-lhes: «Estais no erro e não compreendeis nem as Escrituras, nem o poder de Deus; pois após a ressurreição os homens se não ligarão a mulheres, nem as mulheres tomarão maridos; mas serão como os Anjos de Deus no céu. Acerca da ressurreição dos mortos, não lestes o que está escrito quando Deus vos disse: «Eu son o Deus de Abraão, o Deus de Isaque e de Jacob? Deus não é dos mortos, mas dos vivos». E as turbas, que o escutavam, estavam admiradas da sua doutrina. Porém os fariseus, ouvindo dizer que Ele fizera calar os saduceus, reuniram-se em conselho; e um deles, que era doutor da Lei, fez-lhe esta interrogação para O tentar: «Mestre, qual é o maior mandamento da Lei?». Jesus disse-lhe: «Amarás o Senhor, teu Deus, com todo teu coração, com toda tua alma e com todo teu entendimento. Este é o maior e o primeiro mandamento. Porém, há um segundo, igual a este, qual é: amarás o teu próximo como a ti mesmo. Nestes dous mandamentos se encerram toda a Lei e os Profetas».
}\end{paracol}

\paragraphinfo{Ofertório}{Sl. 23, 3-4}
\begin{paracol}{2}\latim{
\qlettrine{Q}{uis} ascéndet in montem Dómini, aut quis stabit in loco sancto ejus? Innocens mánibus, et mundo corde.
}\switchcolumn\portugues{
\qlettrine{Q}{uem} ascenderá à montanha do Senhor? Quem permanecerá no seu lugar sagrado? Aqueles cujas mãos são inocentes e cujo coração é puro.
}\end{paracol}

\paragraph{Secreta}
\begin{paracol}{2}\latim{
\rlettrine{C}{œlésti} convívio fac nos, Dómine, nuptiáli veste indútos accúmbere: quam beáti Aloísii pia præparátio et juges lácrimæ inæstimabílibus ornábant margarítis. Per Dóminum \emph{\&c.}
}\switchcolumn\portugues{
\rlettrine{P}{ara} que sejamos admitidos ao celestial banquete, concedei-nos, Senhor, que sejamos revestidos com a veste nupcial que o B. Luís, com suas fervorosas disposições e lágrimas contínuas, ornava de pérolas preciosas. Por nosso Senhor \emph{\&c.}
}\end{paracol}

\paragraphinfo{Comúnio}{Sl. 77, 24-25}
\begin{paracol}{2}\latim{
\rlettrine{P}{anem} cœli dedit eis: panem Angelórum manducávit homo.
}\switchcolumn\portugues{
\rlettrine{D}{eu-lhe} o pão do céu: o homem comeu o pão dos Anjos.
}\end{paracol}

\paragraph{Postcomúnio}
\begin{paracol}{2}\latim{
\rlettrine{A}{ngelórum} esca nutrítos, angélicis étiam, Dómine, da móribus vívere: et ejus, quem hódie cólimus, exémplo in gratiárum semper actióne manére. Per Dóminum nostrum \emph{\&c.}
}\switchcolumn\portugues{
\rlettrine{H}{avendo} nós sido alimentados com o Pão dos Anjos, concedei-nos, Senhor, que vivamos também como os Anjos, e, imitando o exemplo daquele que festejamos hoje, vivamos apresentando-Vos sempre a nossa acção de graças. Por nosso Senhor \emph{\&c.}
}\end{paracol}

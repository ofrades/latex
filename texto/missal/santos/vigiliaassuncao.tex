\subsectioninfo{Vigília da Assunção da B. V. Maria}{14 de Agosto}

\paragraphinfo{Intróito}{Página \pageref{virgemnaomartir2}}

\paragraph{Oração}
\begin{paracol}{2}\latim{
\rlettrine{D}{eus,} qui virginálem aulam beátæ Maríæ, in qua habitáres, elígere dignátus es: da, quǽsumus; ut, sua nos defensióne munitos, jucúndos fácias suæ interésse festivitáti: Qui vivis \emph{\&c.}
}\switchcolumn\portugues{
\slettrine{Ó}{} Deus, que escolhestes para vossa morada o seio virginal da B. Virgem Maria, concedei-nos, Vos rogamos, que, munidos com sua protecção, possamos com alegria associar-nos à sua festa. Ó Vós, que viveis e reinais \emph{\&c.}
}\end{paracol}

\paragraphinfo{Oração, Secreta e Postcomúnio}{Página \pageref{confessoresnaopontifices1}}

\paragraphinfo{Epístola}{Página \pageref{montecarmelo}}

\paragraphinfo{Gradual}{Página \pageref{visitacao}}

\paragraphinfo{Evangelho}{Página \pageref{comumfestasmaria1}}

\paragraph{Ofertório}
\begin{paracol}{2}\latim{
\rlettrine{B}{eáta} es, Virgo María, quæ ómnium portásti Creatórem: genuísti qui te fecit, et in ætérnum pérmanes Virgo.
}\switchcolumn\portugues{
\rlettrine{B}{em-aventurada} sois, ó Virgem Maria, pois trouxestes no vosso seio o Criador de todas as coisas. Gerastes Aquele que vos criou; e permanecereis eternamente Virgem.
}\end{paracol}

\paragraph{Secreta}
\begin{paracol}{2}\latim{
\rlettrine{M}{únera} nostra, Dómine, apud cleméntiam tuam Dei Genetrícis comméndet orátio: quam idcírco de praesénti sǽculo transtulísti; ut pro peccátis nostris apud te fiduciáliter intercédat. Per eúndem Dóminum \emph{\&c.}
}\switchcolumn\portugues{
\qlettrine{Q}{ue} as nossas ofertas, Senhor, tenham como recomendação junto da vossa clemência as súplicas da Mãe de Deus, a qual arrebatastes deste mundo para interceder com confiança pelos nossos pecados junto de Vós. Pelo mesmo nosso S \emph{\&c.}
}\end{paracol}

\paragraph{Comúnio}
\begin{paracol}{2}\latim{
\rlettrine{B}{eáta} víscera Maríæ Vírginis, quæ portavérunt ætérni Patris Fílium.
}\switchcolumn\portugues{
\qlettrine{B}{em-aventuradas} as entranhas da B. V. Maria, que trouxeram encerrado o Filho do Pai Eterno.
}\end{paracol}

\paragraph{Postcomúnio}
\begin{paracol}{2}\latim{
\rlettrine{C}{oncéde,} miséricors Deus, fragilitáti nostræ præsídium: ut, qui sanctæ Dei Genetrícis festivitátem prævénimus; intercessiónis ejus auxílio a nostris iniquitátibus resurgámus. Per eúndem Dóminum nostrum \emph{\&c.}
}\switchcolumn\portugues{
\slettrine{Ó}{} Deus de misericórdia, dignai-Vos vir em auxílio da nossa fragilidade, a fim de que, antecipando a festividade da Santa Mãe de Deus, possamos, com o auxílio da sua intercessão, ressuscitar das nossas iniquidades. Pelo mesmo nosso \emph{\&c.}
}\end{paracol}

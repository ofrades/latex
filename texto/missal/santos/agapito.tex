\subsectioninfo{S. Agapito, Mártir}{18 de Agosto}

\paragraph{Oração}
\begin{paracol}{2}\latim{
\rlettrine{L}{ætétur} Ecclésia tua, Deus, beáti Agápiti Mártyris tui confísa suffrágiis: atque, ejus précibus gloriósis, et devóta permáneat et secúra consístat. Per Dóminum \emph{\&c.}
}\switchcolumn\portugues{
\qlettrine{Q}{ue} a vossa Igreja, ó Deus, rejubile com a confiança que lhe dão os sufrágios do B. Agapito, vosso Mártir, e que pelas suas gloriosas preces persevere na piedade e se conserve na paz. Por nosso Senhor \emph{\&c.}
}\end{paracol}

\paragraphinfo{Evangelho}{Jo. 12, 24-26}
\begin{paracol}{2}\latim{
\cruz Sequéntia sancti Evangélii secúndum Lucam.
}\switchcolumn\portugues{
\cruz Continuação do santo Evangelho segundo S. João.
}\switchcolumn*\latim{
\blettrine{I}{n} illo témpore: Dixit Jesus discípulis suis: Amen, amen dico vobis, nisi granum fruménti, cadens in terram, mórtuum fúerit, ipsum solum manet: si autem mórtuum fúerit, multum fructum affert. Qui amat ánimam suam, perdet earn: et qui odit ánimam suam in hoc mundo, in vitam ætérnam custódit earn. Si quis mihi minístrat, me sequátur: et ubi sum ego, illic et minister meus erit. Si quis mihi ministráverit, honorificábit eum Pater meus.
}\switchcolumn\portugues{
\blettrine{N}{aquele} tempo, disse Jesus aos seus discípulos: «Se o grão de trigo, caindo na terra, não morrer, permanece estéril; mas, se morrer, dará muito fruto. Aquele que ama a sua vida perdê-la-á; mas aquele que aborrece a sua vida neste mundo conservá-la-á para a vida eterna. Se alguém me serve, siga-me; e, onde Eu estiver, lá estará também o meu servo. Se alguém me servir, meu Pai o honrará.
}\end{paracol}

\paragraph{Secreta}
\begin{paracol}{2}\latim{
\rlettrine{S}{úscipe,} Dómine, múnera, quæ in ejus tibi sollemnitáte deférimus: cujus nos confídimus patrocínio liberári. Per Dóminum \emph{\&c.}
}\switchcolumn\portugues{
\rlettrine{R}{ecebei,} Senhor, as ofertas que Vos apresentamos na festa daquele por cuja protecção esperamos ser livres. Por nosso Senhor \emph{\&c.}
}\end{paracol}

\paragraph{Postcomúnio}
\begin{paracol}{2}\latim{
\rlettrine{S}{atiásti,} Dómine, famíliam tuam munéribus sacris: ejus, quǽsumus, semper interventióne nos réfove, cujus sollémnia celebrámus. Per Dóminum \emph{\&c.}
}\switchcolumn\portugues{
\rlettrine{S}{aciastes,} Senhor, a vossa família com os sacrossantos dons; e, Vos suplicamos, fortalecei-a sempre pela intercessão daquele cuja festa celebramos. Por nosso Senhor \emph{\&c.}
}\end{paracol}
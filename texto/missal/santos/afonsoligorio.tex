\subsectioninfo{S. Afonso Ligório, B. C. e Doutor}{2 de Agosto}

\paragraphinfo{Intróito}{Lc. 4, 18}
\begin{paracol}{2}\latim{
\rlettrine{S}{píritus} Dómini super me: propter quod unxit me: evangelizáre paupéribus misit me, sanáre contrítos corde. \emph{Ps. 77, 1} Atténdite, pópule meus, legem meam: inclináte aurem vestram in verba oris mei.
℣. Gloria Patri \emph{\&c.}
}\switchcolumn\portugues{
\rlettrine{O}{} Espírito do Senhor está sobre mim; eis porque me ungiu; e me mandou evangelizar os pobres e sarar aqueles cujos corações estão feridos. \emph{Sl. 77, 1} Atendei à minha lei, ó meu povo; escutai as palavras que saem da minha boca.
℣. Glória ao Pai \emph{\&c.}
}\end{paracol}

\paragraph{Oração}
\begin{paracol}{2}\latim{
\rlettrine{D}{eus,} qui per beátum Alfónsum Maríam Confessórem tuum atque Pontíficem, animárum zelo succénsum, Ecclésiam tuam nova prole fœcundásti: quǽsumus; ut, ejus salutáribus mónitis edócti et exémplis roboráti, ad te perveníre felíciter valeámus. Per Dóminum \emph{\&c.}
}\switchcolumn\portugues{
\slettrine{Ó}{} Deus, que pelo ministério do B. Afonso Maria, vosso Confessor e Pontífice, que ardia em zelo pela salvação das almas, destes à vossa Igreja uma nova família, permiti, Vos suplicamos, que, instruídos com suas salutares lições e fortalecidos com seus exemplos, possamos chegar até junto de Vós com felicidade. Por nosso Senhor \emph{\&c.}
}\end{paracol}

\paragraphinfo{Epístola}{2. Tm. 2, 1-7}
\begin{paracol}{2}\latim{
Léctio Epístolæ beáti Pauli Apóstoli ad Timótheum.
}\switchcolumn\portugues{
Lição da Ep.ª do B. Ap.º Paulo a Timóteo.
}\switchcolumn*\latim{
\rlettrine{C}{aríssime:} Confortáre in grátia, quæ est in Christo Jesu: et quæ audísti a me per multos testes, hæc comménda fidálibus homínibus, qui idónei erunt et alios docére. Labóra sicut bonus miles Christi Jesu. Nemo mílitans Deo ímplicat se negótiis sæculáribus: ut ei pláceat, cui se probávit. Nam et qui certat in agóne, non coronátur, nisi legítime certáverit. Laborántem agrícolam opórtet primum de frúctibus percípere. Intéllege quæ dico: dabit enim tibi Dóminus in ómnibus intelléctum.
}\switchcolumn\portugues{
\rlettrine{C}{aríssimo:} Fortificai-vos na graça que está em Jesus Cristo; e, guardando o ensino, que aprendestes de mim diante de várias testemunhas, transmiti-o a homens fiéis, que sejam idóneos para instruir outros. Trabalhai como bom soldado de Jesus Cristo. Aquele que se alista no serviço de Deus nunca se embaraça com os negócios do mundo, mas deve procurar agradar Àquele a quem se entregou. Aquele que combate nos jogos públicos não será coroado se não tiver combatido segundo as regras. O trabalhador agrícola deve ser o primeiro a saborear os frutos. Compreendei bem o que vos digo; pois o Senhor vos dará inteligência em todas as cousas.
}\end{paracol}

\paragraphinfo{Gradual}{Sl. 118, 52-53}
\begin{paracol}{2}\latim{
\rlettrine{M}{emor} fui judiciórum tuórum a sǽculo, Dómine, et consolátus sum: deféctio ténuit me pro peccatóribus derelinquéntibus legem tuam. ℣. \emph{Ps. 39, 11} Justítiam tuam non abscóndi in corde meo: veritátem tuam et salutáre tuum dixi.
}\switchcolumn\portugues{
\rlettrine{R}{ecordei-me,} Senhor, das vossas sentenças, que existiam antes dos séculos: e fiquei consolado. O desânimo apoderou-se de mim à vista dos pecadores, que se afastaram da vossa lei. ℣. \emph{Sl. 39, 11} Não ocultei a vossa justiça no meu coração, publiquei a vossa verdade e a vossa salvação.
}\switchcolumn*\latim{
Allelúja, allelúja. ℣. \emph{Eccli. 49, 3-4} Ipse est diréctus divínitus in pœniténtiam gentis, et tulit abominatiónes impietátis: et gubernávit ad Dóminum cor ipsíus: et in diébus peccatórum corroborávit pietátem. Allelúja.
}\switchcolumn\portugues{
Aleluia, aleluia. ℣. \emph{Ecl. 49, 3-4} Foi predestinado pelo alto para levar o povo à penitência; e fez desaparecer as abominações da impiedade. Volveu o seu coração para o Senhor: e nos dias dos pecadores desenvolveu a piedade. Aleluia.
}\end{paracol}

\paragraphinfo{Evangelho}{Página \pageref{tito}}

\paragraphinfo{Ofertório}{Pr. 3, 9 \& 27}
\begin{paracol}{2}\latim{
\rlettrine{H}{ónora} Dóminum de tua substántia, et de primítiis ómnium frugum tuárum da ei. Noli prohibére benefácere eum, qui potest: si vales, et ipse bénefac.
}\switchcolumn\portugues{
\rlettrine{H}{onra} o Senhor, oferecendo-Lhe alguma cousa que te pertença: dá-lhe primícias de todos teus frutos. Ninguém proíba de praticar o bem a quem pode fazê-lo: e, se és capaz de fazer algum bem, fá-lo.
}\end{paracol}

\paragraph{Secreta}
\begin{paracol}{2}\latim{
\rlettrine{C}{œlésti,} Dómine Jesu Christe, sacrifícii igne corda nostra in odórem suavitátis exúre: qui beáto Alfónso Maríæ tribuísti et hæc mystéria celebráre, et per éadem hóstiam tibi sanctam seípsum exhibére: Qui vivis \emph{\&c.}
}\switchcolumn\portugues{
\rlettrine{S}{enhor} Jesus Cristo acendei nos nossos corações o fogo celestial do sacrifício para os consumir em odor de santidade, pois concedestes ao B. Afonso Maria a graça de celebrar estes mystérios e de se oferecer a Vós, pelo mesmo mystério, como vítima sagrada. Ó Vós, que viveis, e reinais \emph{\&c.}
}\end{paracol}

\paragraphinfo{Comúnio}{Ecl. 50, 1 \& 9}
\begin{paracol}{2}\latim{
\rlettrine{S}{acérdos} magnus, qui in vita sua suffúlsit domum, et in diébus suis corroborávit templum, quasi ignis effúlgens et thus ardens in igne.
}\switchcolumn\portugues{
\rlettrine{E}{ste} grande Pontífice, que durante a sua vida sustentou a casa do Senhor e empregou os seus dias em fortificar o templo, apareceu, como uma chama, a arder e, como o incenso, abrasado no fogo!
}\end{paracol}

\paragraph{Postcomúnio}
\begin{paracol}{2}\latim{
\rlettrine{D}{eus,} qui beátum Alfónsum Maríam Confessórem tuum atque Pontíficem fidelem divíni mystérii dispensatórem et præcónem effecísti: ejus méritis precibúsque concéde; ut fidéles tui et frequénter percípiant, et percipiéndo sine fine colláudent. Per Dóminum \emph{\&c.}
}\switchcolumn\portugues{
\slettrine{Ó}{} Deus, que tornastes o B. Afonso Maria, vosso Confessor e Pontífice, fiel dispensador e pregador dos divinos mystérios, permiti que pelas suas preces e méritos os vossos fiéis os recebam frequentemente e, recebendo-os, Vos louvem incessantemente. Por nosso Senhor \emph{\&c.}
}\end{paracol}

\subsectioninfo{S. João Leonardo, Conf.}{9 de Outubro}\label{joaoleonardo}

\paragraphinfo{Intróito}{Ecl. 42, 15-16}
\begin{paracol}{2}\latim{
\rlettrine{I}{n} sermónibus Dómini ópera ejus: sol illúminans per ómnia respéxit, et glória Dómini plenum est opus ejus. \emph{Ps. 95, 1} Cantáte Dómino cánticum novum: cantáte Dómino, omnis terra.
℣. Gloria Patri \emph{\&c.}
}\switchcolumn\portugues{
\rlettrine{P}{elas} palavras do Senhor existem as suas obras. Assim como o sol resplandecente ilumina todas as coisas, assim as suas obras estão cheias da glória do Senhor. \emph{Sl. 95, 1} Cantai ao Senhor um cântico novo: cantai ao Senhor, ó habitantes de toda a terra.
℣. Glória ao Pai \emph{\&c.}
}\end{paracol}

\paragraph{Oração}
\begin{paracol}{2}\latim{
\rlettrine{D}{eus,} qui beátum Joánnem Confessórem tuum ad fidem in géntibus propagándam mirabíliter excitáre dignátus es, ac per eum in erudiéndis fidélibus novam in Ecclésia tua famíliam congregásti: da nobis fámulis tuis; ita ejus institútis profícere, ut prǽmia consequámur ætérna. Per Dominum \emph{\&c.}
}\switchcolumn\portugues{
\slettrine{Ó}{} Deus, que pelo B. João, vosso Confessor, Vos dignastes animar de um modo admirável a propagação da fé entre os gentios, e por ele instituístes na vossa Igreja uma nova família para a instrução dos fiéis, concedei-nos a nós, vossos servos, que de tal maneira nos aproveitem os seus conselhos que alcancemos os prémios eternos. Por nosso Senhor \emph{\&c.}
}\end{paracol}

\paragraphinfo{Epístola}{2 Cor. 4, 1-6 \& 15-18}
\begin{paracol}{2}\latim{
Lectio Epístolæ beati Pauli Apóstoli ad Corinthios.
}\switchcolumn\portugues{
Lição da Ep.ª do B. Ap.º Paulo aos Coríntios.
}\switchcolumn*\latim{
\rlettrine{F}{ratres:} Habéntes administratiónem juxta quod misericórdiam consecúti sumus, non defícimus, sed abdicámus occúlta dedécóris, non ambulántes in astútia, neque adulterántes verbum Dei, sed in manifestatióne veritátis commendántes nosmetípsos ad omnem consciéntiam hóminum coram Deo. Quod si étiam opértum est Evangélium nostrum : in iis, qui péreunt, est opértum: in quibus Deus hujus sǽculi excæcávit mentes infidélium, ut non fúlgeat illis illuminátio Evangélii glóriæ Christi, qui est imágo Dei. Non enim nosmetípsos prædicámus, sed Jesum Christum Dóminum nostrum: nos autem servos vestros per Jesum: quóniam Deus, qui dixit de ténebris lucem splendéscere, ipse illúxit in córdibus nostris ad illuminatiónem sciéntiæ claritátis Dei, in fácie Christi Jesu. Omnia enim propter vos: ut grátia abúndans, per multos in gratiárum actione, abúndet in glóriam Dei. Propter quod non deficimus: sed licet is, qui foris est, noster homo corrumpátur: tamen is, qui intus est, renovatur de die in diem. Id enim, quod in praesenti est momentáneum et leve tribulatiónis nostræ, supra modum in sublimitáte ætérnum glóriæ pondus operátur in nobis, non contemplántibus nobis quæ vidéntur, sed quæ non vidéntur. Quæ enim vidéntur, temporália sunt: quæ autem non vidéntur, ætérna sunt.
}\switchcolumn\portugues{
\rlettrine{M}{eus} irmãos: Pelo que, tendo nós tal ministério em virtude da misericórdia que alcançámos, não perdemos a coragem, antes renunciámos a coisas que a vergonha manda ocultar, não nos conduzindo com artifício, nem adulterando a palavra de Deus, mas recomendando-nos à consciência de todos os homens diante de Deus, por meio da manifestação da verdade. E, se o nosso Evangelho ainda está oculto, é só para aqueles que se perdem: para aqueles infiéis a quem o deus deste mundo cegou os entendimentos, para que não resplandeça para eles a luz do Evangelho da glória de Cristo, o qual é a imagem de Deus. Nós não nos pregamos a nós mesmos, mas a Jesus Cristo, nosso Senhor; nós, pois, vossos servos, por amor de Jesus; porque Deus, que quis que das trevas resplandecesse a luz, Ele mesmo resplandeceu nos nossos corações, para que fizéssemos brilhar o conhecimento da glória de Deus, que brilha na face de Jesus Cristo. Pois tudo isto aconteceu por amor de vós, a fim de que a graça, espalhando-se com abundância, faça também abundar a acção de graças, à glória de Deus, em muitos outros de entre vós. É por tudo isto que não desfalecemos; antes, pelo contrário, embora se destrua em nós o homem exterior, todavia o interior vai-se renovando de dia para dia; porque aquelas coisas que agora são para nós uma tribulação momentânea e ligeira produzirão um estado eterno duma sublime e incomparável glória, não atendendo nós às coisas que se vêem, mas sim às que se não vêem, pois as coisas que se vêem são passageiras, e as que se não vêem são eternas.
}\end{paracol}

\paragraphinfo{Gradual}{Sl. 72, 21; 68, 10}
\begin{paracol}{2}\latim{
\rlettrine{I}{nflammátum} est cor meum, et renes mei commutáti sunt: zelus domus tuæ comédit me. ℣. \emph{Isai. 49, 2} Pósuit os meum quasi gládium acutum: in umbra manus suæ protéxit me, et pósuit me sicut sagíttam eléctam.
}\switchcolumn\portugues{
\rlettrine{I}{nflamou-se} o meu coração e comoveram-se as minhas entranhas: o zelo da tua casa devorou-me. ℣. \emph{Is. 49, 2} O Senhor tornou a minha boca como uma espada aguda: protegeu-me com a sombra da sua mão: e fez de mim como que urna seta escolhida.
}\switchcolumn*\latim{
Allelúja, allelúja. ℣. \emph{Ps. 70, 7} Tamquam prodígium factus sum multis: et tu adjútor fortis, Allelúja.
}\switchcolumn\portugues{
Aleluia, aleluia. ℣. \emph{Sl. 70, 7} Fui considerado por muitos corno um prodígio: tu és um poderoso auxiliar. Aleluia.
}\end{paracol}

\paragraphinfo{Evangelho}{Página \pageref{tito}}

\paragraphinfo{Ofertório}{Col. 1, 25}
\begin{paracol}{2}\latim{
\rlettrine{C}{hristi} factus sum ego miníster secúndum dispensatiónem Dei, quæ data est mihi, ut ímpleam verbum Dómini.
}\switchcolumn\portugues{
\rlettrine{E}{u} fui constituído ministro de Cristo, segundo o cargo que Deus me deu junto de vós, para que se cumpra a palavra do Senhor.
}\end{paracol}

\paragraph{Secreta}
\begin{paracol}{2}\latim{
\rlettrine{S}{úscipe,} Dómine, oblatiónem mundam salutáris hóstiæ: et præsta; ut, intercedénte beáto Joánne Confessóre tuo, úbique géntium júgiter offerátur. Per Dóminum \emph{\&c.}
}\switchcolumn\portugues{
\rlettrine{R}{ecebei,} Senhor, a oblação pura da hóstia salutar; e concedei-nos por intercessão do B. João, vosso Confessor, que ela seja oferecida constantemente em todos os povos da terra. Por nosso Senhor \emph{\&c.}
}\end{paracol}

\paragraphinfo{Comúnio}{Fl. 3, 7}
\begin{paracol}{2}\latim{
\qlettrine{Q}{uæ} mihi fúerunt lucra, hæc arbitrátus sum propter Christum detriménta.
}\switchcolumn\portugues{
\rlettrine{A}{quelas} coisas que reputava como lucro, considerei-as depois, por amor de Cristo, como prejudiciais.
}\end{paracol}

\paragraph{Postcomúnio}
\begin{paracol}{2}\latim{
\rlettrine{P}{retiósi} córporis, et sánguinis tui sacris refécti mystériis, Dómine, adprecámur: ut beáti Joánnis Confessóris tui exémplo, studeámus confitéri quod crédidit, et ópere exercére quod dócuit: Qui vivis \emph{\&c.}
}\switchcolumn\portugues{
\rlettrine{A}{limentados,} Senhor, com os sacrossantos mistérios do vosso precioso Corpo e Sangue, Vos rogamos instantemente que com o exemplo do B. João, vosso Confessor, procuremos confessar o que ele acreditou e praticar com nossas obras o que ele ensinou. Ó Vós, que \emph{\&c.}
}\end{paracol}

\subsectioninfo{S. Clemente, Papa e Mártir}{23 de Novembro}

\textit{Como na Missa Si díligis me, página \pageref{sumospontifices}, excepto:}

\paragraphinfo{Intróito}{Is. 59, 21; 56, 7}
\begin{paracol}{2}\latim{
\rlettrine{D}{icit} Dóminus: Sermónes mei, quos dedi in os tuum, non defícient de ore tuo: et múnera tua accépta erunt super altáre meum. \emph{Ps. 111, 1} Beátus vir, qui timet Dóminum: in mandátis ejus cupit nimis.
℣. Gloria Patri \emph{\&c.}
}\switchcolumn\portugues{
\rlettrine{D}{isse} o Senhor: Minhas palavras, que pus na vossa boca, não cessarão de estar nos vossos lábios; e então os dons, que apresentardes nos meus altares, ser-me-ão agradáveis. \emph{Sl. 111, 1} Bem-aventurado o varão que teme o Senhor: e cujo zelo é ardente no cumprimento dos seus mandamentos.
℣. Glória ao Pai \emph{\&c.}
}\end{paracol}

\paragraphinfo{Epístola}{Fl. 3, 17-21; 4, 1-3}
\begin{paracol}{2}\latim{
Léctio Epístolæ beáti Pauli Apóstoli ad Philippenses.
}\switchcolumn\portugues{
Lição da Ep.ª do B. Ap.º Paulo aos Filipenses.
}\switchcolumn*\latim{
\rlettrine{F}{ratres:} Imitatóres mei estóte, et observáte eos, qui ita ámbulant, sicut habétis formam nostram. Multi enim ámbulant, quos sæpe dicébam vobis (nunc autem et flens dico) inimícos Crucis Christi: quorum finis intéritus: quorum Deus venter est: et glória in confusióne ipsórum, qui terréna sápiunt. Nostra autem conversátio in cœlis est: unde étiam Salvatórem exspectámus, Dóminum nostrum Jesum Christum, qui reformábit corpus humilitátis nostræ, configurátum córpori claritátis suæ, secúndum operatiónem, qua étiam possit subjícere sibi ómnia. Itaque, fratres mei caríssimi et desideratíssimi, gáudium meum et coróna mea: sic state in Dómino, caríssimi. Evódiam rogo et Sýntychen déprecor idípsum sápere in Dómino. Etiam rogo et te, germáne compar, ádjuva illas, quæ mecum laboravérunt in Evangélio cum Cleménte et céteris adjutóribus meis, quorum nómina sunt in libro vitæ.
}\switchcolumn\portugues{
\rlettrine{M}{eus} irmãos: Sede meus imitadores e segui aqueles que se conduzem segundo o modelo que tendes em nós, porque há muitos, de quem vos tenho falado (e ainda falo deles com lágrimas), que se portam como inimigos da Cruz de Cristo. Seu fim será a condenação; pois fazem do estômago o seu Deus, põem a sua glória naquilo que deveria ser motivo de vergonha e não têm prazer senão nas coisas terrenas. Mas, quanto a nós, o nosso pensamento está nos céus, donde também esperamos o Salvador, nosso Senhor Jesus Cristo, que renovará o nosso corpo vil, tornando-o semelhante ao seu corpo glorioso, por meio daquela virtude que Ele possui de sujeitar a si todas as coisas. Portanto, queridos e amados irmãos, que sois a minha alegria e a minha coroa, permanecei firmes no Senhor, meus caríssimos irmãos. Peço a Evódia e suplico a Sintiquene que tenham os mesmos sentimentos no Senhor. Também vos rogo, ó fiel companheiro, que auxilieis aqueles que trabalharam comigo pelo Evangelho com Clemente e com os outros meus coadjutores, cujos nomes estão escritos no livro da vida.
}\end{paracol}

\subsubsection{Comemoração de Santa Felicidade}

\paragraph{Oração}
\begin{paracol}{2}\latim{
\rlettrine{P}{ræsta,} quǽsumus, omnípotens Deus: ut, beátæ Felicitátis Martyris tuæ sollémnia recenséntes, méritis ipsíus protegámur et précibus. Per Dóminum. \emph{\&c.}
}\switchcolumn\portugues{
\rlettrine{C}{oncedei-nos,} ó Deus omnipotente, Vos suplicamos, que, celebrando nós a solenidade da B. Felicidade, vossa Mártir, sejamos protegidos pelos seus méritos e preces. Por nosso Senhor \emph{\&c.}
}\end{paracol}

\paragraph{Secreta}
\begin{paracol}{2}\latim{
\rlettrine{V}{ota} pópuli tui, Dómine, propitiátus inténde: et, cujus nos tríbuis sollémnia celebráre, fac gaudére suffrágiis. Per Dóminum. \emph{\&c.}
}\switchcolumn\portugues{
\rlettrine{A}{ceitai} propício, Senhor, os votos do vosso povo e fazei-nos gozar o efeito dos sufrágios daquela cuja festa nos permitis celebrar. Por nosso Senhor \emph{\&c.}
}\end{paracol}

\paragraph{Postcomúnio}
\begin{paracol}{2}\latim{
\rlettrine{S}{úpplices} te rogámus, omnípotens Deus: ut, intercedéntibus Sanctis tuis, et tua in nobis dona multíplices, et témpora nostra dispónas. Per Dóminum \emph{\&c.}
}\switchcolumn\portugues{
\slettrine{Ó}{} Deus omnipotente, humildemente Vos suplicamos, pela intercessão dos vossos Santos, que multipliqueis sobre nós os vossos benefícios e governeis os dias da nossa vida. Por nosso Senhor \emph{\&c.}
}\end{paracol}

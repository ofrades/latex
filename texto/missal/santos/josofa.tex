\subsectioninfo{S. Josafá, B. e Mártir}{14 de Novembro}

\paragraph{Intróito}
\begin{paracol}{2}\latim{
\rlettrine{G}{audeámus} omnes in Dómino, diem festum celebrántes sub honóre beáti Jósaphat Mártyris: de cujus passióne gaudent Angeli et colláudant Fílium Dei. \emph{Ps. 32, 1} Exsultáte, justi, in Dómino: rectos decet collaudátio.
℣. Gloria Patri \emph{\&c.}
}\switchcolumn\portugues{
\rlettrine{A}{legremo-nos} todos no Senhor, neste dia em que celebramos a festa em honra do B. Mártir Josafá, de cujo martírio se regozijam os Anjos, louvando, unissonamente, o Filho de Deus. \emph{Sl. 32, 1} Aclamai o Senhor, ó justos, pois é aos que possuem coração recto que pertence louvar o Senhor.
℣. Glória ao Pai \emph{\&c.}
}\end{paracol}

\paragraph{Oração}
\begin{paracol}{2}\latim{
\rlettrine{E}{xcita,} quǽsumus, Dómine, in Ecclésia tua Spíritum, quo replétus beátus Jósaphat Martyr et Póntifex tuus ánimam suam pro óvibus pósuit: ut, eo intercedénte, nos quoque eódem Spíritu moti ac roboráti, ánimam nostram pro frátribus pónere non vereámur. Per Dóminum \emph{\&c.}
}\switchcolumn\portugues{
\rlettrine{S}{enhor,} Vos imploramos, despertai na vossa Igreja, o Espírito de que o B. Josafá, vosso Mártir, estava cheio e que o levou a dar a vida pelas suas ovelhas, a fim de que pela sua intercessão, animados e fortalecidos pelo mesmo Espírito, nunca temamos sacrificar a nossa vida pelos nossos irmãos. Por nosso Senhor \emph{\&c.}
}\end{paracol}

\paragraphinfo{Epístola}{Página \pageref{tomascantorbery}}

\paragraphinfo{Gradual}{Sl. 88, 21-23}
\begin{paracol}{2}\latim{
\rlettrine{I}{nvéni} David servum meum, óleo sancto meo unxi eum: manus enim mea auxiliábitur ei, et bráchium meum confortábit eum. ℣. Nihil profíciet inimícus in eo, et fílius iniquitátis non nocébit ei.
}\switchcolumn\portugues{
\rlettrine{E}{ncontrei} o meu servo David e ungi-o com meu óleo sagrado: a minha mão o auxiliará e o meu braço o fortalecerá. ℣. Meu inimigo nada poderá contra ele e o filho da iniquidade nenhum mal lhe fará.
}\switchcolumn*\latim{
Allelúja, allelúja. ℣. Hic est sacérdos, quem coronávit Dóminus. Allelúja.
}\switchcolumn\portugues{
Aleluia, aleluia. Eis o sacerdote que o Senhor coroou. Aleluia.
}\end{paracol}

\paragraphinfo{Evangelho}{Página \pageref{tomascantorbery}}

\paragraphinfo{Ofertório}{Jo. 15, 13}
\begin{paracol}{2}\latim{
\rlettrine{M}{ajórem} caritátem nemo habet, ut ánimam suam ponat quis pro amícis suis.
}\switchcolumn\portugues{
\rlettrine{N}{inguém} pode dar maior prova de amor do que sacrificar a vida pelos seus amigos.
}\end{paracol}

\paragraph{Secreta}
\begin{paracol}{2}\latim{
\rlettrine{C}{lementíssime} Deus, múnera hæc tua benedictióne perfunde, et nos in fide confírma: quam sanctus Jósaphat Martyr et Póntifex tuus, effúso sánguine, asséruit. Per Dóminum \emph{\&c.}
}\switchcolumn\portugues{
\slettrine{Ó}{} clementíssimo Deus, infundi abundantemente a vossa bênção sobre estas ofertas; e dignai-Vos fortalecer-nos na fé, pela qual o vosso Santo Mártir e Pontífice Josafá derramou o sangue. Por nosso Senhor \emph{\&c.}
}\end{paracol}

\paragraphinfo{Comúnio}{Jo. 10, 14}
\begin{paracol}{2}\latim{
\rlettrine{E}{go} sum pastor bonus: et cognósco oves meas et cognóscunt me meæ.
}\switchcolumn\portugues{
\rlettrine{S}{ou} o bom pastor: conheço as minhas ovelhas; e as minhas ovelhas conhecem-me.
}\end{paracol}

\paragraph{Postcomúnio}
\begin{paracol}{2}\latim{
\rlettrine{S}{píritum,} Dómine, fortitúdinis hæc nobis tríbuat mensa cœléstis: quæ sancti Jósaphat Mártyris tui atque Pontíficis vitam pro Ecclésiæ honóre júgiter áluit ad victóriam. Per Dóminum \emph{\&c.}
}\switchcolumn\portugues{
\qlettrine{Q}{ue} o Espírito da fortaleza, Senhor, nos seja dado nesta celestial mesa, em que o vosso Santo Mártir e Pontífice Josafá encontrou sempre alimento de vida até alcançar a vitória em honra da Igreja. Por nosso Senhor \emph{\&c.}
}\end{paracol}

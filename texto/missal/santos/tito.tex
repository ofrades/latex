\subsectioninfo{S. Tito, B. e Conf.}{6 de Fevereiro}\label{tito}

\textit{Como na Missa Státuit ei, página \pageref{confessorespontifices1}, excepto:}

\paragraph{Oração}
\begin{paracol}{2}\latim{
\rlettrine{D}{eus,} qui beátum Titum Confessórem tuum atque Pontíficem apostólicis virtútibus decorásti: ejus méritis et intercessióne concéde; ut, juste et pie vivéntes in hoc sǽculo, ad cœléstem pátriam perveníre mereámur. Per Dóminum \emph{\&c.}
}\switchcolumn\portugues{
\slettrine{Ó}{} Deus, que ornastes o B. Tito, vosso Confessor e Pontífice, com as virtudes apostólicas, concedei-nos, pelos seus méritos e intercessão, que vivendo neste mundo justa e piamente, mereçamos alcançar a pátria celestial. Por nosso Senhor \emph{\&c.}
}\end{paracol}

\paragraphinfo{Evangelho}{Lc. 10, 1-9}
\begin{paracol}{2}\latim{
\cruz Sequéntia sancti Evangélii secúndum Lucam.
}\switchcolumn\portugues{
\cruz Continuação do santo Evangelho segundo S. Lucas.
}\switchcolumn*\latim{
\blettrine{I}{n} illo témpore: Designávit Dóminus et álios septuagínta duos: et misit illos binos ante fáciem suam in omnem civitátem et locum, quo erat ipse ventúrus. Et dicebat illis: Messis quidem multa, operárii autem pauci. Rogáte ergo Dóminum messis, ut mittat operários in messem suam. Ite: ecce, ego mitto vos sicut agnos inter lupos. Nolíte portáre sǽculum neque peram neque calceaménta; et néminem per viam salutavéritis. In quamcúmque domum intravéritis, primum dícite: Pax huic dómui: et si ibi fúerit fílius pacis, requiéscet super illum pax vestra: sin autem, ad vos revertátur. In eádem autem domo manéte, edéntes et bibéntes quæ apud illos sunt: dignus est enim operárius mercéde sua. Nolíte transíre de domo in domum. Et in quamcúmque civitátem intravéritis, et suscéperint vos, manducáte quæ apponúntur vobis: et curáte infírmos, qui in illa sunt, et dícite illis: Appropinquávit in vos regnum Dei.
}\switchcolumn\portugues{
\blettrine{N}{aquele} tempo, escolheu o Senhor ainda setenta e dois discípulos e mandou-os, dois a dois, adiante d’Ele, a todas as cidades e lugares onde devia ir, dizendo-lhes: «A messe é abundante, mas os operários são poucos; pedi, pois ao Senhor da messe que mande mais operários para a messe. Ide; eis que vos envio como cordeiros para o meio dos lobos; não leveis bolsa, nem saco, nem saudeis ninguém pelo caminho. Em qualquer casa em que entrardes, dizei em primeiro lugar: «A paz esteja nesta casa»! Se aí houver um filho da paz, a vossa paz repousará nele; e, se não um houver, voltará para vós. Permanecei na mesma casa e comei e bebei do que houver nessa casa, pois o operário merece salário. Não transiteis de casa para casa. Em qualquer casa em que entrardes e vos receberem comei do que vos derem, curai os enfermos que aí houver, e dizei-lhes: «Eis que se aproxima o reino de Deus».
}\end{paracol}
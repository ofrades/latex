\subsectioninfo{S. Gorgónio, Mártir}{9 de Setembro}

\textit{Como na Missa Lætábitur justus, página \pageref{martirnaopontifice2}, excepto:}

\paragraph{Oração}
\begin{paracol}{2}\latim{
\rlettrine{S}{anctus} tuus, Dómine, Gorgónius sua nos intercessióne lætíficet: et pia fáciat sollemnitáte gaudére. Per Dóminum \emph{\&c.}
}\switchcolumn\portugues{
\rlettrine{A}{legre-nos,} Senhor, o vosso Santo Gorgónio com sua intercessão; e nos faça sentir o gozo desta pia solenidade. Por nosso Senhor \emph{\&c.}
}\end{paracol}

\paragraph{Secreta}
\begin{paracol}{2}\latim{
\rlettrine{G}{rata} tibi sit, Dómine, nostræ servitútis oblátio: pro qua sanctus Gorgónius Martyr intervéntor exsístat. Per Dóminum \emph{\&c.}
}\switchcolumn\portugues{
\qlettrine{Q}{ue} Vos seja agradável, Senhor, a oferta da nossa servidão, a qual Vo-la apresentamos pela intervenção do Santo Mártir Gorgónio. Por nosso Senhor \emph{\&c.}
}\end{paracol}

\paragraph{Postcomúnio}
\begin{paracol}{2}\latim{
\rlettrine{F}{amíliam} tuam, Deus, suávitas ætérna contíngat et végetet: quæ in Mártyre tuo Gorgónio Christi, Fílii tui, bono júgiter odóre pascátur: Qui tecum \emph{\&c.}
}\switchcolumn\portugues{
\qlettrine{Q}{ue} a vossa família, ó Deus, seja alimentada e fortalecida com as delícias eternas; e que pelo vosso S. Mártir Gorgónio ela se alimente incessantemente com o bom odor de vosso Filho Jesus Cristo: Que convosco \emph{\&c.}
}\end{paracol}

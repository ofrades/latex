\subsectioninfo{S. Joaquim}{16 de Agosto}\label{joaquim}

\paragraphinfo{Intróito}{Sl. 111, 9}
\begin{paracol}{2}\latim{
\rlettrine{D}{ispérsit,} dedit paupéribus: justítia ejus manet in sǽculum sǽculi: cornu ejus exaltábitur in glória. \emph{Ps. ibid., 1} Beátus vir, qui timet Dóminum: in mandátis ejus cupit nimis.
℣. Gloria Patri \emph{\&c.}
}\switchcolumn\portugues{
\rlettrine{D}{istribuiu} liberalmente os seus bens pelos pobres: a sua justiça subsistirá em todos os séculos dos séculos: e o seu poder será exaltado com glória. \emph{Sl. ibid., 1} Bem-aventurado o varão que teme o Senhor e que põe todo o zelo em cumprir os mandamentos.
℣. Glória ao Pai \emph{\&c.}
}\end{paracol}

\paragraph{Oração}
\begin{paracol}{2}\latim{
\rlettrine{D}{eus,} qui præ ómnibus Sanctis tuis beátum Jóachim Genetrícis Fílii tui patrem esse voluísti: concéde, quǽsumus; ut, cujus festa venerámur, ejus quoque perpétuo patrocínia sentiámus. Per eúndem Dóminum \emph{\&c.}
}\switchcolumn\portugues{
\slettrine{Ó}{} Deus, que, de preferência a todos vossos Santos, quisestes que o B. Joaquim fosse o Pai da Mãe de vosso Filho, concedei-nos, Vos suplicamos, que experimentemos o perpétuo patrocínio daquele cuja festa celebramos. Pelo mesmo nosso Senhor \emph{\&c.}
}\end{paracol}

\paragraphinfo{Epístola}{Página \pageref{confessoresnaopontifices1}}

\paragraphinfo{Gradual}{Sl. 111, 9 \& 2}
\begin{paracol}{2}\latim{
\rlettrine{D}{ispérsit,} dedit paupéribus: justítia ejus manet in sǽculum sǽculi. ℣. Potens in terra erit semen ejus: generátio rectórum benedicétur.
}\switchcolumn\portugues{
\rlettrine{D}{istribui} liberalmente os seus bens pelos pobres: a sua justiça permanecerá em todos os séculos dos séculos. ℣. Sua descendência será poderosa na terra, pois a geração dos justos será abençoada.
}\switchcolumn*\latim{
Allelúja, allelúja. ℣. O Jóachim, sanctæ conjux Annæ, pater almæ Vírginis, hic fámulis ferto salútis opem. Allelúja.
}\switchcolumn\portugues{
Aleluia, aleluia. ℣. Ó S. Joaquim, Esposo de Santa Ana, Pai da Virgem-Mãe, concedei na terra aos vossos os socorros necessários para a salvação. Aleluia.
}\end{paracol}

\paragraphinfo{Evangelho}{Mt, 1, 1-16}
\begin{paracol}{2}\latim{
\cruz Initium sancti Evangélii secúndum Matthǽum.
}\switchcolumn\portugues{
\cruz Início do santo Evangelho segundo S. Mateus.
}\switchcolumn*\latim{
\blettrine{L}{iber} generatiónis Jesu Christi, fílii David, fílii Abralam. Abraham génuit Isaac, Isaac autem génuit Jacob. Jacob autem génuit Judam et fratres ejus. Judas autem génuit Phares et Zaram de Thamar. Phares autem génuit Esron. Esron autem génuit Aram. Aram autem génuit Amínadab. Amínadab autem génuit Naásson. Naásson autem génuit Salmon. Salmon autem génuit Booz de Rahab. Booz autem génuit Obed ex Ruth. Obed autem génuit Jesse. Jesse autem génuit David regem. David autem rex génuit Salomónem ex ea, quæ fuit Uriæ. Sálomon autem génuit Róboam. Róboam autem génuit Abíam. Abías autem génuit Asa. Asa autem génuit Jósaphat. Jósaphat autem génuit Joram. Joram autem génuit Ozíam. Ozías autem génuit Jóatham. Jóatham autem génuit Achaz. Achaz autem génuit Ezechíam. Ezechias autem génuit Manássen. Manásses autem génuit Amen. Amon autem génuit Josíatn. Josías autem génuit Jechoníam et fratres ejus in transmigratióne Babylónis. Et post transmigratiónem Babylónis: Jeehonías génuit Saláthiel. Saláthiel autem génuit Zoróbabel. Zoróbabel autem génuit Abiud. Abiud autem génuit Elíacim. Elíacim autem génuit Azor. Azor autem génuit Sadoc. Sadoc autem génuit Achim. Achim autem génuit Eliud. Eliud autem génuit Eleázar. Eleázar autem génuit Mathan. Mathan autem génuit Jacob. Jacob autem génuit Joseph, virum Mariæ, de qua natus est Jesus, qui vocátur Christus.
}\switchcolumn\portugues{
\blettrine{L}{ivro} da geração de Jesus Cristo, filho de David, filho de Abraão. Abraão gerou Isaque. Isaque gerou Jacob. Jacob gerou Judas e seus irmãos. Judas gerou Fares e Zarão de Tamar. Fares gerou Esron. Esron gerou Aarão. Aarão gerou Aminadabe. Aminadabe gerou Naássão. Naássão gerou Salmão. Salmão gerou Booz de Raabe. Booz gerou Obede de Rute. Obede gerou Jesse. Jesse gerou o Rei David. David gerou Salomão daquela que fora mulher de Urias. Salomão gerou Roboão. Roboão gerou Abias. Abias gerou Asa. Asa gerou Josafá. Josafá gerou Jorão. Jorão gerou Ozias. Ozias gerou Joatão. Joatão gerou Acás. Acás gerou Ezequias. Ezequias gerou Manasses. Manasses gerou Amão. Amão gerou Josias. Josias gerou Jeconias e os seus irmãos, na deportação da Babilónia. E depois da deportação da Babilónia Jeconias gerou Salátiel. Salátiel gerou Zoróbabel. Zoróbabel gerou Abiude. Abiude gerou Eliacim. Eliacim gerou Azor. Azor gerou Sadoc. Sadoc gerou Aquim. Aquim gerou Éliude. Éliude gerou Eleazar. Eleazar gerou Matam. Matam gerou Jacob. E Jacob gerou José, esposo de Maria, da qual nasceu Jesus, que é chamado Cristo.
}\end{paracol}

\paragraphinfo{Ofertório}{Sl. 8, 6-7}
\begin{paracol}{2}\latim{
\rlettrine{G}{lória} et honóre coronásti eum: et constituísti eum super ópera mánuum tuárum, Dómine.
}\switchcolumn\portugues{
\rlettrine{V}{ós} o coroastes, Senhor, com glória e honras; Vós o estabelecestes acima das obras das vossas mãos.
}\end{paracol}

\paragraph{Secreta}
\begin{paracol}{2}\latim{
\rlettrine{S}{úscipe,} clementíssime Deus, sacrifícium in honórem sancti Patriarchæ Jóachim, patris Maríæ Vírginis, majestáti tuæ oblátum: ut, ipso cum cónjuge sua et beatíssima prole intercedénte, perféctam cónsequi mereámur remissiónem peccatórum et glóriam sempitérnam. Per Dóminum \emph{\&c.}
}\switchcolumn\portugues{
\rlettrine{R}{ecebei,} ó clementíssimo Deus, o sacrifício que oferecemos à vossa majestade em honra do santo Patriarca Joaquim, Pai da Virgem Maria, a fim de que, pela sua intercessão, unida à de sua esposa e à de sua B. Filha, mereçamos alcançar a plena remissão dos nossos pecados e a glória eterna. Por nosso Senhor \emph{\&c.}
}\end{paracol}

\paragraphinfo{Comúnio}{Lc. 12, 42}
\begin{paracol}{2}\latim{
\rlettrine{F}{idélis} servus et prudens, quem constítuit dóminus super famíliam suam: ut det illis in témpore trítici mensuram.
}\switchcolumn\portugues{
\rlettrine{E}{is} o servo fiel e prudente que o Senhor estabeleceu acima da sua família para distribuir oportunamente a cada um a sua medida de trigo.
}\end{paracol}

\paragraph{Postcomúnio}
\begin{paracol}{2}\latim{
\qlettrine{Q}{uæsumus,} omnípotens Deus: ut per hæc sacraménta, quæ súmpsimus, intercedéntibus méritis et précibus beáti Jóachim patris Genetrícis dilécti Fílii tui, Dómini nostri Jesu Christi, tuæ grátiæ in præsénti et ætérnæ glóriæ in futúro partícipes esse mereámur. Per eúndem Dóminum \emph{\&c.}
}\switchcolumn\portugues{
\rlettrine{F}{azei,} ó Deus omnipotente, Vos suplicamos, que pelas preces e méritos do B. Joaquim, Pai da Mãe do vosso amado Filho, nosso Senhor Jesus Cristo, estes sacramentos, que recebemos, nos tornem comparticipantes da vossa graça na vida presente e da vossa eterna glória na vida futura. Pelo mesmo nosso Senhor \emph{\&c.}
}\end{paracol}

\subsectioninfo{S. Bartolomeu, Apóstolo}{24 de Agosto}

\paragraphinfo{Intróito}{Sl. 138, 17}
\begin{paracol}{2}\latim{
\rlettrine{M}{ihi} autem nimis honoráti sunt amíci tui, Deus: nimis confortatus est principatus eórum. \emph{Ps. ibid., 1-2} Dómine, probásti me et cognovísti me: tu cognovisti sessiónem meam et resurrectionem meam.
℣. Gloria Patri \emph{\&c.}
}\switchcolumn\portugues{
\rlettrine{E}{u} vejo, ó Deus, que honrais largamente os vossos amigos: e que seu poder se tem fortalecido extraordinariamente. \emph{Sl. ibid., 1-2} Senhor, perscrutastes o meu íntimo e ficastes-me conhecendo; ficastes conhecendo quando me deito e quando me levanto.
℣. Glória ao Pai \emph{\&c.}
}\end{paracol}

\paragraph{Oração}
\begin{paracol}{2}\latim{
\rlettrine{O}{mnípotens} sempitérne Deus, qui hujus diei venerándam sanctámque lætítiam in beáti Apóstoli tui Bartholomǽi festivitáte tribuísti: da Ecclésiæ tuæ, quǽsumus; et amáre quod crédidit, et prædicáre quod dócuit. Per Dóminum \emph{\&c.}
}\switchcolumn\portugues{
\slettrine{Ó}{} Deus omnipotente e sempiterno, que neste dia, em que celebramos a festa do B. Bartolomeu, Apóstolo, nos proporcionais uma santa e veneranda alegria, concedei à vossa Igreja, Vos suplicamos, o dom de amar o que ele acreditou e de pregar o que ensinou. Por nosso Senhor \emph{\&c.}
}\end{paracol}

\paragraphinfo{Epístola}{1. Cor. 12, 27-31}
\begin{paracol}{2}\latim{
Léctio Epístolæ beáti Pauli Apóstoli ad Corínthios.
}\switchcolumn\portugues{
Lição da Ep.ª do B. Ap.º Paulo aos Coríntios.
}\switchcolumn*\latim{
\rlettrine{F}{ratres:} Vos estis corpus Christi et membra de membro. Et quosdam quidem pósuit Deus in Ecclésia primum apóstolos, secúndo prophetas, tertio doctores, deinde virtútes, exinde grátias curationum, opitulatiónes, gubernatiónes, genera linguarum, interpretatiónes sermonum. Numquid omnes apóstoli? numquid omnes prophétæ? numquid omnes doctóres? numquid omnes virtútes? numquid omnes grátiam habent curationum? numquid omnes linguis loquúntur? numquid omnes interpretántur? Æmulámini autem charísmata melióra.
}\switchcolumn\portugues{
\rlettrine{M}{eus} irmãos: Sois os corpos de Cristo e os membros uns dos outros. Assim, Deus estabeleceu na sua Igreja primeiramente Apóstolos, em segundo lugar Profetas, em terceiro lugar Doutores, depois os que fazem milagres, em seguida os que têm o dom de curar, de assistir, de governar, de falar várias línguas e de interpretá-las. Porém são todos Apóstolos? São todos Profetas? São todos Doutores? Fazem todos milagres? Têm todos o dom de curar? Falam todos as línguas? Têm todos o dom de interpretá-las? Aspirai aos dons mais perfeitos.
}\end{paracol}

\paragraphinfo{Gradual}{Sl. 44, 17 \& 18}
\begin{paracol}{2}\latim{
\rlettrine{C}{onstítues} eos príncipes super omnem terram: mémores erunt nóminis tui, Dómine. ℣. Pro patribus tuis nati sunt tibi fílii: proptérea pópuli confitebúntur tibi.
}\switchcolumn\portugues{
\rlettrine{S}{erão} constituídos príncipes em toda a terra e perpetuarão a glória do vosso nome, Senhor! ℣. Para substituir os vossos pais, nascer-vos-ão filhos; por isso os povos Vos honrarão eternamente.
}\switchcolumn*\latim{
Allelúja, allelúja. ℣. Te gloriósus Apostolórum chorus laudat, Dómine. Allelúja.
}\switchcolumn\portugues{
Aleluia, aleluia. ℣. O coro glorioso dos Apóstolos canta os vossos louvores. Aleluia.
}\end{paracol}

\paragraphinfo{Evangelho}{Lc. 6, 12-19}
\begin{paracol}{2}\latim{
\cruz Sequéntia sancti Evangélii secúndum Lucam.
}\switchcolumn\portugues{
\cruz Continuação do santo Evangelho segundo S. Lucas.
}\switchcolumn*\latim{
\blettrine{I}{n} illo témpore: Exiit Jesus in montem oráre, et erat pernóctans in oratióne Dei. Et cum dies factus esset, vocavit discípulos suos, et elégit duódecim ex ipsis (quos et Apóstolos nominávit): Simónem, quem cognominávit Petrum, et Andream fratrem ejus, Jacóbum et Joánnem, Philíppum et Bartholomǽum, Matthǽum et Thomam, Jacóbum Alphǽi et Simónem, qui vocátur Zelótes, et Judam Jacóbi, et Judam Iscariótem, qui fuit próditor. Et descéndens cum illis, stetit in loco campéstri, et turba discipulórum ejus, et multitúdo copiósa plebis ab omni Judǽa, et Jerúsalem, et marítima, et Tyri, et Sidónis, qui vénerant, ut audírent eum et sanaréntur a languóribus suis. Et, qui vexabántur a spirítibus immúndis, curabántur. Et omnis turba quærébat eum tángere: quia virtus de illo exíbat, et sanábat omnes.
}\switchcolumn\portugues{
\blettrine{N}{aquele} tempo, retirara-se Jesus para um monte, e aí passara toda a noite em oração a Deus. Quando o dia rompeu, chamou os discípulos e escolheu doze de entre eles, aos quais chamou Apóstolos: Simão, a quem deu o cognome de Pedro, e André, seu irmão, Tiago e João, Filipe e Bartolomeu, Mateus e Tomé, Tiago, filho de Alfeu, e Simão, chamado o Zelote, Judas, irmão de Tiago, e Judas Iscariotes, que foi o traidor. E, descendo com eles, parou em um campo, onde encontrou a turba dos discípulos e uma grande multidão de povo de toda a Judeia, de Jerusalém e das paragens de Tiro e de Sidónia, que tinham vindo para O ouvirem e para que seus enfermos fossem curados. Entre eles havia alguns que estavam possessos de espíritos imundos, os quais foram curados. Toda aquela multidão procurava tocar em Jesus, pois d’Ele saía uma tal virtude que a todos sarava.
}\end{paracol}

\paragraphinfo{Ofertório}{Sl. 138, 17}
\begin{paracol}{2}\latim{
\rlettrine{M}{ihi} autem nimis honoráti sunt amíci tui, Deus: nimis confortátus est principátus eórum.
}\switchcolumn\portugues{
\rlettrine{E}{u} vejo, ó Deus, que honrais largamente os vossos amigos; e por isso o seu poder se tem fortalecido extraordinariamente.
}\end{paracol}

\paragraph{Secreta}
\begin{paracol}{2}\latim{
\rlettrine{B}{eáti} Apóstoli tui Bartholomǽi sollémnia recenséntes, quǽsumus, Dómine: ut ejus auxílio tua benefícia capiámus, pro quo tibi laudis hóstias immolámus. Per Dóminum \emph{\&c.}
}\switchcolumn\portugues{
\rlettrine{C}{elebrando} a festa do B. Bartolomeu, vosso Apóstolo, fazei, Senhor, Vos suplicamos, que alcancemos os vossos benefícios pelo socorro daquele em cuja honra imolamos esta hóstia de louvor. Por nosso Senhor \emph{\&c.}
}\end{paracol}

\paragraphinfo{Comúnio}{Mt. 19, 28}
\begin{paracol}{2}\latim{
\rlettrine{V}{os,} qui secúti estis me, sedébitis super sedes, judicántes duódecim tribus Israël, dicit Dóminus.
}\switchcolumn\portugues{
\rlettrine{V}{ós,} que me seguistes, vos assentareis sobre tronos e julgareis as doze tribos de Israel.
}\end{paracol}

\paragraph{Postcomúnio}
\begin{paracol}{2}\latim{
\rlettrine{S}{umptum,} Dómine, pignus redemptiónis ætérnæ: sit nobis, quǽsumus; interveniénte beáto Bartholomǽo Apóstolo tuo, vitæ præséntis auxílium páriter et futúræ. Per Dóminum \emph{\&c.}
}\switchcolumn\portugues{
\rlettrine{P}{ermiti,} Senhor, que o penhor da redenção eterna, que recebemos, seja para nós, por intercessão do B. Bartolomeu, Apóstolo, auxílio na vida presente e na futura. Por nosso Senhor \emph{\&c.}
}\end{paracol}

\subsectioninfo{Vigília de S. S. Pedro e Paulo, Aps.}{28 de Junho}

\paragraphinfo{Intróito}{Jo. 21, 18-19}
\begin{paracol}{2}\latim{
\rlettrine{D}{icit} Dóminus Petro: Cum esses júnior, cingébas te et ambulábas, ubi volébas: cum autem senúeris, exténdes manus tuas, et álius te cinget et ducet, quo tu non vis: hoc autem dixit, signíficans, qua morte clarificatúrus esset Deum. \emph{Ps. 18, 1} Cœli enárrant glóriam Dei: et ópera mánuum ejus annúntiat firmaméntum.
℣. Gloria Patri \emph{\&c.}
}\switchcolumn\portugues{
\rlettrine{O}{} Senhor disse a Pedro: «Quando tu eras mais novo, cingias-te a ti mesmo e ias onde querias; mas, quando fores velho, estenderás as tuas mãos, um outro te cingirá e te conduzirá onde não queiras». Isto foi-lhe dito para lhe indicar com que morte deveria glorificar Deus. \emph{Sl. 18, 1} Os céus proclamam a glória de Deus e o firmamento anuncia as obras das suas mãos.
℣. Glória ao Pai \emph{\&c.}
}\end{paracol}

\paragraph{Oração}
\begin{paracol}{2}\latim{
\rlettrine{P}{ræsta,} quǽsumus, omnípotens Deus: ut nullis nos permíttas perturbatiónibus cóncuti; quos in apostólicæ confessiónis petra solidásti. Per Dóminum \emph{\&c.}
}\switchcolumn\portugues{
\slettrine{Ó}{} Deus omnipotente, havendo nós sido estabelecidos sobre a pedra sólida da fé dos Apóstolos, dignai-Vos permitir que nenhuma perturbação abale a nossa fé. Por nosso Senhor \emph{\&c.}
}\end{paracol}

\paragraphinfo{Epístola}{Act. 3, 1-10}
\begin{paracol}{2}\latim{
Léctio Actuum Apostolórum. 
}\switchcolumn\portugues{
Lição dos Actos dos Apóstolos.
}\switchcolumn*\latim{
\rlettrine{I}{n} diébus illis: Petrus et Joánnes ascendébant in templum ad horam oratiónis nonam. Et quidam vir, qui erat claudus ex útero matris suæ, bajulabátur: quem ponébant cotídie ad portam templi, quæ dícitur Speciósa, ut péteret eleemósynam ab introeúntibus in templum. Is cum vidísset Petrum et Joánnem incipiéntes introíre in templum, rogábat, ut eleemósynam acciperet. Intuens autem in eum Petrus cum Joánne, dixit: Réspice in nos. At ille intendébat in eos, sperans se áliquid acceptúrum ab eis. Petrus autem dixit: Argéntum et aurum non est mihi; quod autem habeo, hoc tibi do: In nómine Jesu Christi Nazaréni surge, et ámbula. Et apprehénsa manu ejus déxtera, allevávit eum, et protínus consolidátæ sunt bases ejus et plantæ. Et exsíliens stetit, et ambulábat: et intrávit cum illis in templum, ámbulans et exsíliens et laudans Deum. Et vidit omnis populus eum ambulántem et laudántem Deum. Cognoscébant autem illum, quod ipse erat, qui ad eleemósynam sedébat ad Speciósam portam templi: et impléti sunt stupore et écstasi in eo, quod contígerat illi.
}\switchcolumn\portugues{
\rlettrine{C}{omo} Pedro e João subissem ao templo para a hora nona da oração, encontraram um homem, que era coxo desde o seio de sua mãe, a quem colocavam todos os dias à porta do templo, chamada «Especiosa», para pedir esmola àqueles que entravam. Ora, havendo este homem visto Pedro e João, que iam a entrar no templo, pediu-lhes esmola. Então Pedro fixou os olhos nele, ao mesmo tempo que João, e disse-lhe: «Olha para nós». Ele fitou-os atentamente, esperando que lhe dessem alguma cousa. Mas Pedro disse-lhe: «Eu não tenho ouro, nem prata; mas dou-te aquilo que tenho. Em nome de Jesus Cristo Nazareno, levanta-te e caminha!». E, segurando-o pela mão direita, ergueu-o. Logo as suas pernas e os seus pés se firmaram, e o coxo, saltando pelos seus pés, se pôs de pé e começou a andar, entrando com eles no templo e caminhando, saltando e louvando Deus. E todo o povo o via caminhar e louvar Deus; e, reconhecendo que aquele era o mesmo que estava assentado à porta «Especiosa» do templo a mendigar, ficou cheio de admiração e de espanto pelo que havia acontecido.
}\end{paracol}

\paragraphinfo{Gradual}{Sl. 18, 5 \& 2}
\begin{paracol}{2}\latim{
\rlettrine{I}{n} omnem terram exívit sonus eórum: et in fines orbis terræ verba eórum. ℣. Cœli enárrant glóriam Dei: et ópera mánuum ejus annúntiat firmaméntum.
}\switchcolumn\portugues{
\rlettrine{O}{} eco da sua voz espalhou-se por toda a terra: e as suas palavras soaram até aos confins da terra. ℣. Os céus proclamam a glória de Deus e o firmamento anuncia as obras das suas mãos.
}\end{paracol}

\paragraphinfo{Evangelho}{Jo. 21, 15-10}
\begin{paracol}{2}\latim{
\cruz Sequéntia sancti Evangélii secúndum Joánnem.
}\switchcolumn\portugues{
\cruz Continuação do santo Evangelho segundo S. João.
}\switchcolumn*\latim{
\blettrine{I}{n} illo témpore: Dixit Jesus Simóni Petro: Simon Joánnis, díligis me plus his? Dicit ei: Etiam, Dómine, tu scis, quia amo te. Dicit ei: Pasce agnos meos. Dicit ei íterum: Simon Joánnis, díligis me? Ait illi: Etiam, Dómine, tu scis, quia amo te. Dicit ei: Pasce agnos meos. Dicit ei tértio: Simon Joánnis, amas me? Contristátus est Petrus, quia dixit ei tértio, Amas me? et dixit ei: Dómine, tu ómnia nosti: tu scis, quia amo te. Dixit ei: Pasce oves meas. Amen, amen, dico tibi: cum esses júnior, cingébas te et ambulábas, ubi volébas: cum autem senúeris, exténdes manus tuas, et álius te cinget et ducet, quo tu non vis. Hoc autem dixit, signíficans, qua morte clarificatúrus esset Deum.
}\switchcolumn\portugues{
\blettrine{N}{aquele} tempo, Jesus disse a Simão-Pedro: «Simão, filho de João, tu amas-me mais do que estes?». Ele respondeu: «Sim, Senhor, sabeis que Vos amo». E Jesus disse-lhe: «Apascenta os meus cordeiros». Novamente Jesus lhe perguntou: «Simão, filho de João, tu amas-me?». Pedro respondeu-Lhe: «Sim, Senhor, sabeis que Vos amo». Jesus disse-lhe: «Apascenta os meus cordeiros». Ainda Jesus lhe perguntou terceira vez: «Simão, filho de João, tu amas-me?». Contristou-se Pedro de que Ele lhe tivesse perguntado terceira vez se o amava, e disse-Lhe: «Senhor, conheceis tudo, e portanto sabeis que Vos amo». E Jesus disse-lhe: «Apascenta as minhas ovelhas. Em verdade, em verdade vos digo: quando eras mais novo, cingias-te a ti mesmo e ias onde querias; mas, quando fores velho, estenderás as mãos, um outro te cingirá e te conduzirá onde tu não queiras». Ora isto dizia Ele para indicar com que morte glorificaria Deus.
}\end{paracol}

\paragraphinfo{Ofertório}{Sl. 138, 17}
\begin{paracol}{2}\latim{
\rlettrine{M}{ihi} autem nimis honoráti sunt amíci tui, Deus: nimis confortátus est principátus eórum. 
}\switchcolumn\portugues{
\rlettrine{V}{ejo,} ó meu Deus, que honrais de um modo singular os vossos amigos: o seu poder firmou-se extraordinariamente.
}\end{paracol}

\paragraph{Secreta}
\begin{paracol}{2}\latim{
\rlettrine{M}{unus} pópuli tui, quǽsumus, Dómine, apostólica intercessióne sanctífica: nosque a peccatórum nostrórum máculis emúnda. Per Dóminum \emph{\&c.}
}\switchcolumn\portugues{
\rlettrine{S}{enhor,} Vos suplicamos, santificai pela intercessão dos vossos Apóstolos a oblata do vosso povo e purificai-nos das manchas dos nossos pecados. Por nosso Senhor \emph{\&c.}
}\end{paracol}

\paragraphinfo{Comúnio}{Jo. 21, 15 \& 17}
\begin{paracol}{2}\latim{
\rlettrine{S}{imon} Joánnis, díligis me plus his? Dómine, tu ómnia nosti: tu scis, Dómine, quia amo te.
}\switchcolumn\portugues{
\rlettrine{S}{imão,} filho de João, tu amas-me mais do que estes? Senhor, conheceis tudo, e, portanto, sabeis que Vos amo.
}\end{paracol}

\paragraph{Postcomúnio}
\begin{paracol}{2}\latim{
\qlettrine{Q}{uos} cœlésti, Dómine, álii ménto satiásti: apostólicis intercessiónibus ab omni adversitáte custódi. Per Dóminum nostrum \emph{\&c.}
}\switchcolumn\portugues{
\rlettrine{D}{ignai-Vos,} Senhor, pela intercessão dos vossos Apóstolos preservar de todas as adversidades aqueles que saciastes com o alimento celestial. Por nosso Senhor \emph{\&c.}
}\end{paracol}
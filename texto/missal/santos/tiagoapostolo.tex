\subsectioninfo{S. Tiago}{25 de Julho}

\paragraphinfo{Intróito}{Sl. 138, 17}
\begin{paracol}{2}\latim{
\rlettrine{M}{ihi} autem nimis bonoráti sunt amíci tui, Deus: nimis confortátus est principátus eórum. \emph{Ps. ibid., 1-2} Dómine, probásti me et cognovísti me: tu cognovísti sessiónem meam ei resurrectiónem meam.
℣. Gloria Patri \emph{\&c.}
}\switchcolumn\portugues{
\rlettrine{E}{u} vejo, ó Deus, que honrais largamente os vossos amigos: e que seu poder se tem fortalecido extraordinariamente. \emph{Sl. ibid., 1-2} Senhor, perscrutastes o meu íntimo e ficastes-me conhecendo: ficastes conhecendo quando me deito e quando me levanto.
℣. Glória ao Pai \emph{\&c.}
}\end{paracol}

\subsubsectioninfo{Comemoração S. Cristovão}{Página \pageref{martirnaopontifice1}}

\paragraph{Oração}
\begin{paracol}{2}\latim{
\rlettrine{E}{sto,} Dómine, plebi tuæ sanctificátor et custos: ut, Apóstoli tui Jacóbi muníta præsídiis, et conversatióne tibi pláceat, et secúra mente desérviat. Per Dóminum nostrum \emph{\&c.}
}\switchcolumn\portugues{
\rlettrine{S}{ede,} Senhor, o santificador e o protector do povo, a fim de que, munido com o auxílio do vosso Apóstolo Tiago, ele Vos seja agradável pela sua sã conduta e Vos sirva com tranquilidade de espírito. Por nosso Senhor \emph{\&c.}
}\end{paracol}

\paragraphinfo{Epístola}{1. Cor. 4. 9-15}
\begin{paracol}{2}\latim{
Léctio Epístolæ beáti Pauli Apóstoli ad Corinthios
}\switchcolumn\portugues{
Lição da Ep.ª do B. Ap.º Paulo aos Coríntios.
}\switchcolumn*\latim{
\rlettrine{F}{ratres:} Puto, quod Deus nos Apóstolos novíssimos osténdit, tamquam morti destinátos: quia spectáculum facti sumus mundo et Angelis et homínibus. Nos stulti propter Christum, vos autem prudéntes in Christo: nos infírmi, vos autem fortes: vos nóbiles, nos autem ignóbiles. Usque in hanc horam et esurímus, et sitímus, et nudi sumus, et cólaphis cǽdimur, et instábiles sumus, et laborámus operántes mánibus nostris: maledícimur, et benedícimus: persecutiónem pátimur, et sustinémus: blasphemámur, et obsecrámus: tamquam purgaménta hujus mundi facti sumus, ómnium peripséma usque adhuc. Non ut confúndam vos, hæc scribo, sed ut fílios meos caríssimos móneo. Nam si decem mília pædagogórum habeátis in Christo: sed non multos patres. Nam in Christo Jesu per Evangélium ego vos génui.
}\switchcolumn\portugues{
\rlettrine{M}{eus} irmãos: Penso que Deus nos trata a nós, seus Apóstolos, como se fôssemos os últimos homens destinados à morte, pois nos tornámos espectáculo do mundo, dos Anjos e dos homens. Tornámo-nos insensatos por amor de Cristo; mas vós sois prudentes em Cristo. Nós somos fracos, mas vós sois fortes; sois respeitados, e nós desprezados. Até a esta hora, sofremos a fome, a sede, a nudez, os maus tratos; não temos morada fixa; trabalhamos com muita dificuldade com as nossas mãos; amaldiçoam-nos e abençoamos; perseguem-nos e suportamo-los; injuriam-nos e rezamos por eles; enfim, até agora têm-nos tratado como se fôssemos a escumalha do mundo, por todos desprezada. Escrevo estas coisas não para vos envergonhar, mas para vo-las dar a conhecer, como meus filhos caríssimos que sois, pois, ainda que tivésseis dez mil mestres em Cristo, nem por isso teríeis muitos pais, visto que fui eu que pelo Evangelho vos gerei em Jesus Cristo.
}\end{paracol}

\paragraphinfo{Gradual}{Sl. 44, 17 \& 18}
\begin{paracol}{2}\latim{
\rlettrine{C}{onstítues} eos príncipes super omnem terram: mémores erunt nóminis tui, Dómine. ℣. Pro pátribus tuis nati sunt tibi fílii: proptérea pópuli confitebúntur tibi.
}\switchcolumn\portugues{
\rlettrine{V}{ós} os instituístes príncipes em todo o universo: e eles, Senhor, perpetuarão a glória do vosso nome em toda a terra. ℣. Para substituir os vossos pais, nascer-vos-ão filhos: pelo que os povos vos louvarão.
}\switchcolumn*\latim{
Allelúja, allelúja. ℣. \emph{Joann. 15, 16} Ego vos elegi de mundo, ut eátis, et fructum afferátis, et fructus vester máneat. Allelúja.
}\switchcolumn\portugues{
Aleluia, aleluia. ℣. \emph{Jo. 15, 16} Eu vos escolhi para irdes pelo mundo e alcançardes fruto; e que o vosso fruto permaneça. Aleluia.
}\end{paracol}

\paragraphinfo{Evangelho}{Mt. 20, 20-23}
\begin{paracol}{2}\latim{
\cruz Sequéntia sancti Evangélii secúndum Matthǽum.
}\switchcolumn\portugues{
\cruz Continuação do santo Evangelho segundo S. Mateus.
}\switchcolumn*\latim{
\blettrine{I}{n} llo témpore: Accessit ad Jesum mater filiórum Zebedǽi cum fíliis suis, adórans et petens áliquid ab eo. Qui dixit ei: Quid vis? Ait illi: Dic, ut sédeant hi duo fílii mei, unus ad déxteram tuam et unus ad sinístram in regno tuo. Respóndens autem Jesus, dixit: Néscitis, quid petátis. Potéstis bíbere cálicem, quem ego bibitúrus sum? Dicunt ei: Póssumus. Ait illis: Cálicem quidem meum bibétis: sédere autem ad déxteram meam vel sinístram, non est meum dare vobis, sed quibus parátum est a Patre meo.
}\switchcolumn\portugues{
\blettrine{N}{aquele} tempo, a mãe dos filhos de Zebedeu aproximou-se de Jesus com seus dois filhos, adorando-O e querendo pedir-Lhe alguma coisa. Jesus disse-lhe: «Que quereis?». Ela respondeu: «Ordenai que estes meus dois filhos se assentem, um à vossa direita e o outro à vossa esquerda, no vosso reino». Jesus respondeu-lhe: «Não sabeis o que pedis. Podeis beber o cálice que Eu devo beber?». Eles responderam: «Podemos». E Jesus disse-lhes: «Bebereis, com efeito, o meu cálice; porém não depende de mim conceder-vos um lugar à minha direita ou à minha esquerda, pois isso é para aqueles para quem meu Pai o preparou».
}\end{paracol}

\paragraphinfo{Ofertório}{Sl. 18, 5}
\begin{paracol}{2}\latim{
\rlettrine{I}{n} omnem terram exívit sonus eórum: et in fines orbis terræ verba eórum.
}\switchcolumn\portugues{
\rlettrine{O}{} som da sua voz ecoou por toda a terra; e as suas palavras prolongaram-se até às extremidades da terra.
}\end{paracol}

\paragraph{Secreta}
\begin{paracol}{2}\latim{
\rlettrine{O}{blatiónes} pópuli tui, quǽsumus, Dómine, beáti Jacóbi Apóstoli pássio beáta concíliet: et, quæ nostris non aptæ sunt méritis, fiant tibi plácitæ ejus deprecatióne. Per Dóminum \emph{\&c.}
}\switchcolumn\portugues{
\rlettrine{P}{ermiti,} Senhor, Vos suplicamos, que o glorioso martírio do B. Apóstolo Tiago nos alcance a graça de receberdes as ofertas do vosso povo, e, apesar dos nossos méritos serem insuficientes, fazei que sua deprecação Vo-las torne agradáveis. Por nosso Senhor \emph{\&c.}
}\end{paracol}

\paragraphinfo{Comúnio}{Mt. 19, 28}
\begin{paracol}{2}\latim{
\rlettrine{V}{os,} qui secúti estis me, sedébitis super sedes, judicántes duódecim tribus Israël.
}\switchcolumn\portugues{
\rlettrine{V}{ós,} que me seguistes, assentar-vos-eis sobre tronos e julgareis as doze tribos de Israel.
}\end{paracol}

\paragraph{Postcomúnio}
\begin{paracol}{2}\latim{
\rlettrine{B}{eáti} Apóstoli tui Jacóbi, quǽsumus, Dómine, intercessióne nos ádjuva: pro cujus festivitáte percépimus tua sancta lætántes. Per Dóminum nostrum \emph{\&c.}
}\switchcolumn\portugues{
\rlettrine{A}{uxiliai-nos,} Senhor, Vos suplicamos, pela intercessão do vosso B. Apóstolo Tiago, em cuja festa recebemos com júbilo os sacrossantos méritos. Por nosso Senhor \emph{\&c.}
}\end{paracol}

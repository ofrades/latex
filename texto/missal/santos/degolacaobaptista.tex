\subsectioninfo{Degolação de S. João Baptista}{29 de Agosto}

\paragraphinfo{Intróito}{Sl. 118, 46-47}
\begin{paracol}{2}\latim{
\rlettrine{L}{oquébar} de testimóniis tuis in conspéctu regum, et non confundébar: et meditábar in mandátis tuis, quæ diléxi nimis. \emph{Ps. 91, 2} Bonum est confitéri Dómino: et psállere nómini tuo, Altíssime.
℣. Gloria Patri \emph{\&c.}
}\switchcolumn\portugues{
\rlettrine{F}{alava} na presença dos reis sobre a vossa lei e não me envergonhava: meditava nos vossos mandamentos, que muito amei. \emph{Sl. 91, 2} É bom louvar o Senhor e cantar hinos em honra do vosso nome, ó Altíssimo.
℣. Glória ao Pai \emph{\&c.}
}\end{paracol}

\paragraph{Oração}
\begin{paracol}{2}\latim{
\rlettrine{S}{ancti} Joánnis Baptístæ Præcursóris et Martyris tui, quǽsumus, Dómine, veneránda festívitas: salutáris auxílii nobis præstet efféctum: Qui vivis et regnas \emph{\&c.}
}\switchcolumn\portugues{
\rlettrine{P}{ermiti,} Senhor, Vos rogamos, que a veneranda festividade de S. João Baptista, vosso Precursor e Mártir, nos alcance o efeito do vosso salutar auxílio. Ó Vós, que \emph{\&c.}
}\end{paracol}

\paragraphinfo{Oração}{Santa Sabina}
\begin{paracol}{2}\latim{
\rlettrine{D}{eus,} qui inter cétera poténtiæ tuæ mirácula étiam in sexu frágili victóriam martýrii contulísti: concéde propítius; ut, qui beátæ Sabínæ Mártyris tuæ natalítia cólimus, per ejus ad te exémpla gradiámur. Per Dóminum nostrum \emph{\&c.}
}\switchcolumn\portugues{
\slettrine{Ó}{} Deus, que entre outros milagres do vosso poder permitistes que o sexo frágil alcançasse a vitória do martírio, concedei-nos propício que, venerando o nascimento da vossa B, Virgem e Mártir Sabina, caminhemos para Vós, imitando os seus exemplos. Por nosso Senhor \emph{\&c.}
}\end{paracol}

\paragraphinfo{Epístola}{Jr. 1, 17-19}
\begin{paracol}{2}\latim{
Léctio Jeremíæ Prophétæ.
}\switchcolumn\portugues{
Lição do Profeta Jeremias.
}\switchcolumn*\latim{
\rlettrine{I}{n} diébus illis: Factum est verbum Dómini ad me, dicens: Accínge lumbos tuos, et surge, et lóquere ad Juda ómnia, quæ ego præcípio tibi. Ne formides a fácie eórum: nec enim timére te fáciam vultum eórum. Ego quippe dedi te hódie in civitátem munítam, et in colúmnam férream, et in murum ǽreum, super omnem terram, régibus Juda, princípibus ejus, et sacerdótibus, et pópulo terræ. Et bellábunt advérsum te, et non prævalebunt: quia ego tecum sum, ait Dóminus, ut líberem te.
}\switchcolumn\portugues{
\rlettrine{N}{aqueles} dias, o Senhor falou-me e disse-me: «Cinge os teus rins, levanta-te e diz à Judeia tudo quanto te ordenar. Não temas aparecer diante deles, porque farei que tu não tenhas medo. Formei-te hoje como uma cidade forte, como uma coluna de ferro e como um muro de bronze, em toda a terra, diante dos reis de Judá, dos seus príncipes, dos seus sacerdotes e do seu povo. Eles combaterão contra ti, mas não triunfarão, pois estou contigo para te livrar», diz o Senhor.
}\end{paracol}

\paragraphinfo{Gradual}{Sl. 91, 13 \& 14}
\begin{paracol}{2}\latim{
\qlettrine{J}{ustus} ut palma florébit: sicut cedrus Líbani multiplicábitur in domo Dómini. ℣. \emph{ibid., 3} Ad annuntiándum mane misericórdiam tuam, et veritátem tuam per noctem.
}\switchcolumn\portugues{
\rlettrine{O}{} justo florescerá, como a palmeira, e crescerá, como o cedro do Líbano, na casa do Senhor. ℣. \emph{ibid., 3} Para publicar de manhã a vossa misericórdia; e de noite a vossa verdade.
}\switchcolumn*\latim{
Allelúja, allelúja. ℣. \emph{Osee 14, 6} Justus germinábit sicut lílium: et florébit in ætérnum ante Dóminum. Allelúja.
}\switchcolumn\portugues{
Aleluia, aleluia. ℣. \emph{Os. 14, 6} O justo germinará, como o lírio, e florescerá eternamente perante o Senhor. Aleluia.
}\end{paracol}

\paragraphinfo{Evangelho}{Mc. 6, 17-29}
\begin{paracol}{2}\latim{
\cruz Sequéntia sancti Evangélii secúndum Marcum.
}\switchcolumn\portugues{
\cruz Continuação do santo Evangelho segundo S. Marcos.
}\switchcolumn*\latim{
\blettrine{I}{n} illo témpore: Misit Heródes, ac ténuit Joánnem, et vinxit eum in cárcere propter Herodíadem, uxorem Philíppi fratris sui, quia dúxerat eam. Dicebat enim Joánnes Heródi: Non licet tibi habére uxórem fratris tui. Heródias autem insidiabátur illi, et volébat occídere eum, nec póterat. Heródes enim metuébat Joánnem, sciens eum virum justum et sanctum: et custodiébat eum, et audíto eo multa faciébat, et libénter eum audiébat. Et cum dies opportúnus accidísset, Heródes natális sui cœnam fecit princípibus et tribúnis et primis Galilǽæ. Cumque introísset fília ipsíus Herodíadis, et saltásset, et placuísset Heródi simúlque recumbéntibus; rex ait puéllæ: Pete a me, quod vis, et dabo tibi. Et jurávit illi: Quia quidquid petiéris dabo tibi, licet dimídium regni mei. Quæ cum exiísset, dixit matri suæ: Quid petam? At illa dixit: Caput Joánnis Baptístæ. Cumque introísset statim cum festinatióne ad regem, petívit dicens: Volo, ut protínus des mihi in disco caput Joánnis Baptístæ. Et contristátus est rex: propter jusjurándum et propter simul discumbéntes nóluit eam contristáre: sed misso spiculatóre, præcépit afférri caput ejus in disco. Et decollávit eum in cárcere. Et áttulit caput ejus in disco: et dedit illud puéllæ, et puella dedit matri suæ. Quo audíto, discípuli ejus venérunt et tulérunt corpus ejus: et posueérunt illud in monuménto.
}\switchcolumn\portugues{
\blettrine{N}{aquele} tempo, mandou Herodes prender João, ligá-lo e metê-lo no cárcere, para agradar a Herodíade, mulher de Filipe, seu irmão, a qual ele havia desposado. Ora, João dizia a Herodes: «Não te é lícito tomar a mulher de teu irmão». Por isso Herodíade armava-lhe ciladas e queria matá-lo, mas não podia; pois Herodes temia João, sabendo que era homem justo e santo. Contudo Herodes conservava-o em custódia, ainda que fizesse muitas cousas inspirado pelos seus conselhos e até, espontaneamente, o ouvisse. Chegou, porém, um dia favorável o dia do aniversário do nascimento de Herodes em que este oferecia um banquete aos grandes, aos tribunos e aos príncipes da Galileia. Então, entrou na sala a filha de Herodíade, dançando e agradando a Herodes e aos presentes. Disse-lhe o rei: «Pede o que quiseres que te darei». E fez-lhe um juramento: «Tudo o que me pedires te darei, ainda que seja metade do meu reino». Ouvindo ela isto, saiu e perguntou à mãe: «Que pedirei eu?». A mãe respondeu-lhe: «A cabeça de João Baptista». Logo, entrou na sala, apressadamente, e, indo ao pé do rei, fez-lhe o pedido nestes termos: «Quero que me dês, já, em um prato, a cabeça de João Baptista». Contristou-se muito o rei; mas, por causa do seu juramento e dos convivas, não quis contrariá-la enviando um dos seus guardas com a ordem de lhe trazer a cabeça de João Baptista, em um prato. O guarda cortou-lhe a cabeça na prisão, colocou-a em um prato, deu-a à filha de Herodíade e esta deu-a a sua mãe. Ouvindo isto, os discípulos vieram, levaram o seu corpo e colocaram-no em um túmulo.
}\end{paracol}

\paragraphinfo{Ofertório}{Sl. 20, 2-3}
\begin{paracol}{2}\latim{
\rlettrine{I}{n} virtúte tua, Dómine, lætábitur justus, et super salutáre tuum exsultábit veheménter: desidérium ánimæ ejus tribuísti ei.
}\switchcolumn\portugues{
\rlettrine{O}{} justo, Senhor, alegrar-se-á com vosso poder e rejubilará, vendo-se salvo por Vós: concedestes-lhe o desejo do seu coração.
}\end{paracol}

\paragraph{Secreta}
\begin{paracol}{2}\latim{
\rlettrine{M}{únera,} quæ tibi, Dómine, pro sancti Martyris tui Joánnis Baptístæ passióne deférimus: quǽsumus; ut ejus obténtu nobis profíciant ad salútem. Per Dóminum \emph{\&c.}
}\switchcolumn\portugues{
\rlettrine{V}{os} apresentamos, Senhor, estas oblatas em honra dos sofrimentos do vosso santo Mártir João Baptista, e, Vos suplicamos, fazei pelos seus méritos que sirvam de proveito à nossa salvação. Por nosso Senhor \emph{\&c.}
}\end{paracol}

\paragraphinfo{Secreta}{Santa Sabina}
\begin{paracol}{2}\latim{
\rlettrine{H}{óstias} tibi, Dómine, beátæ Sabínæ Mártyris tuæ dicátas méritis, benígnus assúme: et ad perpétuum nobis tríbue proveníre subsídium. Per Dóminum \emph{\&c.}
}\switchcolumn\portugues{
\rlettrine{A}{ceitai} benignamente, Senhor, as hóstias que Vos oferecemos pelos méritos da B. Virgem e Mártir Sabina, e dignai-Vos permitir que ela nos sirva de perpétuo auxílio. Por nosso Senhor \emph{\&c.}
}\end{paracol}

\paragraphinfo{Comúnio}{Sl. 20, 4}
\begin{paracol}{2}\latim{
\rlettrine{P}{osuísti,} Dómine, in cápite ejus corónam de lápide pretióso.
}\switchcolumn\portugues{
\rlettrine{S}{enhor,} colocastes na sua cabeça uma coroa de pedras preciosas.
}\end{paracol}

\paragraph{Postcomúnio}
\begin{paracol}{2}\latim{
\rlettrine{C}{ónferat} nobis, Dómine, sancti Joánnis Baptístæ sollémnitas: ut et magnífica sacraménta, quæ súmpsimus, significáta venerémur, et in nobis pótius édita gaudeámus. Per Dóminum \emph{\&c.}
}\switchcolumn\portugues{
\qlettrine{Q}{ue} a solenidade de S. João Baptista, Senhor, nos alcance a graça de venerarmos nos augustos sacramentos, que recebemos, o que as aparências dos mesmos significam; e, mais ainda, de gozarmos a alegria de os haver recebido. Por nosso Senhor \emph{\&c.}
}\end{paracol}

\paragraphinfo{Postcomúnio}{Santa Sabina}
\begin{paracol}{2}\latim{
\rlettrine{D}{ivíni} muneris largitáte satiáti, quǽsumus, Dómine, Deus noster: ut, intercedénte beáta Sabína Mártyre tua, in ejus semper participatióne vivámus. Per Dóminum \emph{\&c.}
}\switchcolumn\portugues{
\rlettrine{S}{aciados} com a liberalidade do dom divino, ó Senhor, nosso Deus, Vos suplicamos pela intercessão da B. Virgem Sabina, vossa Mártir, que durante a nossa vida comparticipemos sempre deste dom divino. Por nosso Senhor \emph{\&c.}
}\end{paracol}

\subsubsection{Comemoração de Santa Sabina}

\paragraphinfo{Oração}{Página \pageref{virgensmartires1}}

\paragraphinfo{Secreta e Postcomúnio}{Página \pageref{virgensmartires2}}

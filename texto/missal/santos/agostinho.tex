\subsectioninfo{S. Agostinho}{28 de Agosto}

\textit{Como na Missa In médio Ecclésiae, página \pageref{doutores}, excepto:}

\paragraph{Oração}
\begin{paracol}{2}\latim{
\rlettrine{A}{désto} supplicatiónibus nostris, omnípotens Deus: et, quibus fidúciam sperándæ pietátis indúlges, intercedénte beáto Augustíno Confessóre tuo atque Pontífice, consuétae misericórdiæ tríbue benígnus efféctum. Per Dóminum nostrum \emph{\&c.}
}\switchcolumn\portugues{
\rlettrine{O}{uvi} benigno, ó Deus omnipotente, as nossas súplicas, e, visto que nos permitis confiarmos na vossa bondade, concedei-nos, pela intercessão do B. Agostinho, vosso Confessor e Pontífice, a graça de alcançarmos o efeito benigno da vossa habitual misericórdia. Por nosso Senhor \emph{\&c.}
}\end{paracol}

\paragraphinfo{Gradual}{Sl. 36, 30-31}
\begin{paracol}{2}\latim{
\rlettrine{O}{s} justi meditábitur sapiéntiam, et lingua ejus loquétur judícium. ℣. Lex Dei ejus in corde ipsíus: et non supplantabúntur gressus ejus.
}\switchcolumn\portugues{
\rlettrine{A}{} boca do justo falará com sabedoria e a sua língua proclamará a justiça. ℣. A lei do seu Deus está no seu coração e os seus pés não tropeçarão.
}\switchcolumn*\latim{
Allelúja, allelúja. ℣. \emph{Ps. 88, 21} Invéni David servum meum, óleo sancto meo unxi eum. Allelúja. 
}\switchcolumn\portugues{
Aleluia, aleluia. ℣. \emph{Sl. 88, 21} Encontrei o meu servo David e ungi-o com meu óleo sagrado. Aleluia.
}\end{paracol}
\subsectioninfo{S. Marcos, Evangelista}{25 de Abril}

\textit{Como na Missa Protexísti me, página \pageref{martir}, excepto:}

\paragraph{Oração}
\begin{paracol}{2}\latim{
\rlettrine{D}{eus,} qui beátum Marcum Evangelístam tuum evangélicæ prædicatiónis grátia sublimásti: tríbue, quǽsumus; ejus nos semper et eruditióne profícere et oratióne deféndi. Per Dóminum \emph{\&c.}
}\switchcolumn\portugues{
\slettrine{Ó}{} Deus, que glorificastes o Beato Marcos, vosso Evangelista, elevando-o à dignidade de pregador do vosso Evangelho, concedei-nos a graça, Vos suplicamos, de aproveitarmos sempre os seus ensinos e de sermos defendidos pela sua oração. Por nosso Senhor \emph{\&c.}
}\end{paracol}

\paragraphinfo{Epístola}{Ez. 1, 10-14}
\begin{paracol}{2}\latim{
Léctio Ezechiélis Prophétæ.
}\switchcolumn\portugues{
Lição do Profeta Ezequiel.
}\switchcolumn*\latim{
\rlettrine{S}{imilitúdo} vultus quátuor animálium: fácies hóminis, et fácies leónis a dextris ipsórum quatuor: fácies autem bovis a sinístris ipsórum quátuor, et fácies áquilæ désuper ipsórum quátuor. Fácies eórum et pennæ eórum exténtæ désuper: duæ pennæ singulórum jungebántur et duæ tegébant córpora eórum: et unumquódque eórum coram fácie sua ambulábat: ubi erat ímpetus spíritus, illuc gradiebántur, nec revertebántur cum ambulárent. Et similitúdo animálium, aspéctus eórum quasi carbónum ignis ardéntium et quasi aspéctus lampadárum. Hæc erat visio discúrrens in médio animálium, splendor ignis, et de igne fulgur egrédiens. Et animália ibant et revertebántur in similitúdinem fúlguris coruscántis.
}\switchcolumn\portugues{
\rlettrine{E}{is} a semelhança do rosto dos quatro seres animados: Tinham todos os quatro uma face de homem; todos os quatro à direita uma face de leão; todos os quatro à esquerda uma face de touro; e todos os quatro por cima uma face de águia! As faces e as asas mostravam-se estendidas por cima; e estavam unidos uns aos outros por duas asas, cobrindo os corpos com as outras duas. Cada um deles caminhava para a frente do seu rosto, e iam até onde os impelia o espírito, não se voltando enquanto andavam. Estes seres tinham o aspecto de carvões de fogo a arder e de lâmpadas acesas. Viam-se crepitar no meio deles chamas de fogo, saindo do fogo relâmpagos. E eles iam e vinham, semelhante ao fuzilar dos relâmpagos.
}\end{paracol}

\paragraphinfo{Evangelho}{Página \pageref{tito}}

\paragraph{Secreta}
\begin{paracol}{2}\latim{
\rlettrine{B}{eáti} Marci Evangelístæ tui sollemnitáte tibi múnera deferéntes, quǽsumus, Dómine: ut, sicut illum prædicátio evangélica fecit gloriósum: ita nos ejus intercéssio et verbo et ópere tibi reddat accéptos. Per Dóminum \emph{\&c.}
}\switchcolumn\portugues{
\rlettrine{O}{ferecendo-Vos} estes dons na solenidade do B. Marcos, vosso Evangelista, Vos rogamos, Senhor, que, assim como a pregação do Evangelho tornou o seu nome glorioso, assim também a sua intercessão nos torne agradáveis à vossa majestade, tanto pelas nossas palavras, como pelas nossas obras. Por nosso Senhor \emph{\&c.}
}\end{paracol}

\paragraph{Postcomúnio}
\begin{paracol}{2}\latim{
\rlettrine{T}{ríbuant} nobis, quǽsumus, Dómine, contínuum tua sancta præsídium: quo, beáti Marci evangelístæ tui précibus, nos ab ómnibus semper tueántur advérsis. Per Dóminum \emph{\&c.}
}\switchcolumn\portugues{
\rlettrine{V}{os} imploramos, Senhor, que os vossos sacrossantos mystérios nos sirvam de contínua protecção, e que pelas preces do B. Marcos, vosso Evangelista, nos defendam de todas as adversidades. Por nosso Senhor \emph{\&c.}
}\end{paracol}

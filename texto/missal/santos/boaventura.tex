\subsectioninfo{S. Boaventura, B. Confessor e Doutor}{14 de Julho}

\textit{Como na Missa In médio Ecclésiae, página \pageref{doutores}, excepto:}

\paragraphinfo{Gradual}{Sl. 36, 30-31}
\begin{paracol}{2}\latim{
\rlettrine{O}{s} justi meditábitur sapiéntiam, et lingua ejus loquétur judícium. ℣. Lex Dei ejus in corde ipsíus: et non supplantabúntur gressus ejus.
}\switchcolumn\portugues{
\rlettrine{A}{} boca do justo falará com sabedoria e a sua língua proclamará a justiça. ℣. A lei do seu Deus está no seu coração e os seus pés não tropeçarão.
}\switchcolumn*\latim{
Allelúja, allelúja. ℣. \emph{Ps. 109, 4} Jurávit Dóminus, et non pœnitébit eum: Tu es sacérdos in ætérnum, secúndum órdinem Melchísedech. Allelúja.
}\switchcolumn\portugues{
Aleluia, aleluia. ℣. \emph{Ps. 109, 4} O Senhor jurou, e não se arrependerá: tu és sacerdote para sempre segundo a ordem de Melquisedeque. Aleluia.
}\end{paracol}

\paragraphinfo{Ofertório}{Sl. 88, 25}
\begin{paracol}{2}\latim{
\rlettrine{V}{eritas} mea et misericórdia mea cum ipso: et in nómine meo exaltábitur cornu ejus.
}\switchcolumn\portugues{
\rlettrine{A}{} minha fidelidade e a minha misericórdia estarão com ele: e o seu poder exaltar-se-á pelo meu nome.
}\end{paracol}

\paragraphinfo{Secreta e Postcomúnio}{Página \pageref{confessorespontifices2}}

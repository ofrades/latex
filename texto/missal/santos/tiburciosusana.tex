\subsectioninfo{S. S. Tibúrcio e Susana, Mártires}{11 de Agosto}

\textit{Como na Missa Salus autem, página \pageref{muitosmartires3}, excepto:}

\paragraph{Oração}
\begin{paracol}{2}\latim{
\rlettrine{S}{anctórum} Martyrum tuórum Tiburtii et Susánnæ nos, Dómine, fóveant continuáta præsídia: quia non désinis propítius intuéri; quos tálibus auxíliis concésseris adjuvári. Per Dóminum \emph{\&c.}
}\switchcolumn\portugues{
\rlettrine{P}{ermiti,} Senhor, que nos favoreça o contínuo auxílio dos vossos Santos Mártires Tibúrcio e Susana, pois não podeis deixar de acolher propiciamente aqueles a quem concedeis o socorro de tal protecção. Por nosso Senhor \emph{\&c.}
}\end{paracol}

\paragraphinfo{Epístola}{Página \pageref{fabiaosebastiao}}

\paragraph{Secreta}
\begin{paracol}{2}\latim{
\rlettrine{A}{désto,} Dómine, précibus pópuli tui, adésto munéribus: ut, quæ sacris sunt obláta mystériis, tuórum tibi pláceant intercessióne Sanctórum. Per Dóminum \emph{\&c.}
}\switchcolumn\portugues{
\rlettrine{A}{tendei,} Senhor, às preces do vosso povo e recebei as suas ofertas, a fim de que, pela intercessão dos vossos Santos Mártires, a oblação destes mistérios Vos seja agradável. Por nosso Senhor \emph{\&c.}
}\end{paracol}

\paragraph{Postcomúnio}
\begin{paracol}{2}\latim{
\rlettrine{S}{úmpsimus,} Dómine, pignus redemptiónis ætérnæ: quod sit nobis, quǽsumus, interveniéntibus sanctis Martýribus tuis, vitæ præséntis auxílium páriter et futúræ. Per Dóminum \emph{\&c.}
}\switchcolumn\portugues{
\rlettrine{R}{ecebemos,} Senhor, o penhor da eterna redenção; e permiti, pela intercessão dos vossos Santos Mártires, que nos sirva de auxílio, tanto na vida presente, como na futura. Por nosso Senhor \emph{\&c.}
}\end{paracol}

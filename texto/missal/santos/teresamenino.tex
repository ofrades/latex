\subsectioninfo{Santa Teresa do Menino Jesus}{3 de Outubro}

\paragraphinfo{Intróito}{Ct. 4, 8-9}
\begin{paracol}{2}\latim{
\rlettrine{V}{eni} de Líbano, sponsa mea, veni de Líbano, veni: vulnerásti cor meum, soror mea sponsa, vulnerásti cor meum. \emph{Ps. 112, 1} Laudáte, púeri, Dóminum: laudáte nomen Dómini.
℣. Gloria Patri \emph{\&c.}
}\switchcolumn\portugues{
\rlettrine{V}{em} comigo do Líbano, ó minha esposa; Vem comigo do Líbano; vem. Tu feriste o meu coração, minha irmã, minha esposa; tu feriste o meu coração. \emph{Sl. 112, 1} Louvai o Senhor, ó meninos; louvai o nome do Senhor. 
℣. Glória ao Pai \emph{\&c.}
}\end{paracol}

\paragraph{Oração}
\begin{paracol}{2}\latim{
\rlettrine{D}{ómine,} qui dixisti: Nisi efficiámini sicut párvuli, non intrábitis in regnum cœlórum: da nobis, quǽsumus: ita sanctæ Teresiæ Vírginis in humilitáte et simplicitáte cordis vestígia sectári, ut prǽmia consequámur ætérna. Qui vivis \emph{\&c.}
}\switchcolumn\portugues{
\rlettrine{S}{enhor,} que Vos dignastes ensinar-nos «Se vos não tornardes semelhantes a estes pequenos não entrareis no reino dos céus», concedei-nos, Vos suplicamos, que de tal modo imitemos os exemplos da humildade e simplicidade da Virgem Santa Teresa que consigamos alcançar os prémios eternos. Ó Vós, que viveis e reinais \emph{\&c.}
}\end{paracol}

\paragraphinfo{Epístola}{Is. 66, 12-14}
\begin{paracol}{2}\latim{
Léctio Isaíæ Prophétæ.
}\switchcolumn\portugues{
Lição do Profeta Isaías.
}\switchcolumn*\latim{
\rlettrine{H}{æc} dicit Dóminus: Ecce, ego declinábo super eam quasi flúvium pacis, et quasi torréntem inundántem glóriam géntium, quam sugétis: ad úbera portabímini, et super génua blandiéntur vobis. Quómodo si cui mater blandiátur, ita ego consolábor vos, et in Jerúsalem consolabímini. Vidébitis, et gaudébit cor vestrum, et ossa vestra quasi herba germinábunt, et cognoscétur manus Dómini servis ejus.
}\switchcolumn\portugues{
\rlettrine{A}{ssim} fala o Senhor: «Eis que farei correr sobre ela como que um rio de paz; e como que uma torrente caudalosa inundará com riquezas os povos. Sereis alimentados com leite, acalentados no seu seio e acariciados sobre os seus joelhos. Eu vos consolarei, como aquele a quem a mãe consola; e serei a vossa alegria em Jerusalém. Vê-lo-eis, e regozijar-se-á o vosso coração; os vossos ossos reconquistarão o vigor, como a erva dos campos; e a mão do Senhor se manifestará em favor dos seus servos».
}\end{paracol}

\paragraphinfo{Gradual}{Mt. 11, 25}
\begin{paracol}{2}\latim{
\rlettrine{C}{onfíteor} tibi, Pater, Dómine cœli et terræ, quia abscondísti hæc a sapiéntibus, et prudéntibus, et revelásti ea párvulis. ℣. \emph{Ps. 70, 5} Dómine, spes mea a juventúte mea.
}\switchcolumn\portugues{
\rlettrine{D}{ou-Vos} graças, ó Pai, Senhor do céu e da terra, porque ocultastes estas coisas aos sábios e aos prudentes deste mundo, e as revelastes aos «pequenos». ℣. \emph{Sl. 70, 5} Sois a minha esperança, Senhor, desde a minha juventude. 
}\switchcolumn*\latim{
Allelúja, allelúja. ℣. \emph{Eccli. 39, 17-19} Quasi rosa plantáta super rivos aquárum fructificate: quasi Libanus odórem suavitátis habete: florete, flores, quasi lílium, et date odórem, et frondete in grátiam, et collaudate cánticum, et benedicite Dóminum in opéribus suis. Allelúja.
}\switchcolumn\portugues{
Aleluia, aleluia. ℣. \emph{Ecl. 39, 17-19} Frutificai, como a rosa plantada à beira das águas; espalhai vosso suave perfume, como o monte Líbano; como os lírios, desabrochai vossas flores e exalai vossos perfumes; enchei-vos de beleza; entoai hinos e cânticos em honra do Senhor, louvando-O Pela magnificência das suas obras. Aleluia.
}\end{paracol}

\paragraphinfo{Evangelho}{Mt. 18, 1-4}
\begin{paracol}{2}\latim{
\cruz Sequéntia sancti Evangélii secúndum Matthǽum.
}\switchcolumn\portugues{
\cruz Continuação do santo Evangelho segundo S. Mateus.
}\switchcolumn*\latim{
\blettrine{I}{n} illo témpore: Accessérunt discípuli ad Jesum, dicéntes: Quis, putas, major est in regno cœlórum? Et advocans Jesus párvulum, státuit eum in médio eórum, et dixit; Amen, dico vobis, nisi convérsi fuéritis, et efficiámini sicut párvuli, non intrábitis in regnum cœlorum. Quicúmque ergo humiliáverit se sicut párvulus iste, hic est major in regno cœlórum.
}\switchcolumn\portugues{
\blettrine{N}{aquele} tempo, aproximaram-se de Jesus os seus discípulos, dizendo-Lhe: «Qual pensais Vós que é o maior no reino dos céus?». E Jesus, havendo chamado um pequeno, colocou-o no meio deles e disse: «Em verdade vos digo: se vos não converteis e não vos tornais como os pequenos, não entrareis no reino dos céus».
}\end{paracol}

\paragraphinfo{Ofertório}{Lc. 1, 46-48 \& 49}
\begin{paracol}{2}\latim{
\rlettrine{M}{agníficat} ánima mea Dóminum: et exsultávit spíritus meus in Deo salutári meo: quia respéxit humilitátem ancíllæ suæ: fecit mihi magna qui potens est.
}\switchcolumn\portugues{
\rlettrine{A}{} minha alma engrandece o Senhor e o meu espírito alegra-se em Deus, meu Salvador, pois Ele dignou-se olhar benignamente para a humildade da sua escrava; e praticou em mim grandes coisas.
}\end{paracol}

\paragraph{Secreta}
\begin{paracol}{2}\latim{
\rlettrine{S}{acrifícium} nostrum tibi, Dómine, quǽsumus, sanctæ Terésiæ Vírginis tuæ precátio sancta concíliet: ut, in cujus honóre sollémniter exhibétur, ejus méritis efficiátur accéptum. Per Dóminum \emph{\&c.}
}\switchcolumn\portugues{
\rlettrine{S}{enhor,} Vos suplicamos, permiti que a oração da vossa Virgem Santa Teresa Vos torne agradável o nosso sacrifício, a fim de que por Vós seja aceite, pelos méritos daquela em cuja honra Vo-lo oferecemos. Por nosso Senhor \emph{\&c.}
}\end{paracol}

\paragraphinfo{Comúnio}{Dt. 32, 10-12}
\begin{paracol}{2}\latim{
\rlettrine{C}{ircumdúxit} eam, et dócuit: et custodívit quasi pupíllam óculi sui. Sicut aquila expándit alas suas, et assúmpsit eam, atque portávit in húmeris suis. Dóminus solus dux ejus fuit. 
}\switchcolumn\portugues{
\rlettrine{R}{odeou-a,} protegeu-a e guardou-a, como à pupila dos seus olhos. Como a águia, abriu suas asas, arrebatou-a e colocou-a sobre os seus joelhos. O Senhor foi o seu único guia.
}\end{paracol}

\paragraph{Postcomúnio}
\begin{paracol}{2}\latim{
\rlettrine{I}{llo} nos, Dómine amóris igne cœléste mystérium inflámmet: quo sancta Teresia Virgo tua se tibi pro homínibus caritátis víctimam devóvit. Per Dóminum \emph{\&c.}
}\switchcolumn\portugues{
\qlettrine{Q}{ue} este mistério, Senhor, nos abrase no fogo celestial, ao qual a vossa Virgem Santa Teresa se ofereceu como vítima de amor pelos homens. Por nosso Senhor \emph{\&c.}
}\end{paracol}
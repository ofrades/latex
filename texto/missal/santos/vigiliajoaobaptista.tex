\subsectioninfo{Vigília de S. João Baptista}{23 de Junho}

\paragraphinfo{Intróito}{Lc. 1,13, 15 \& 14}
\begin{paracol}{2}\latim{
\rlettrine{N}{e} tímeas, Zacharía, exaudíta est orátio tua: et Elísabeth uxor tua páriet tibi fílium, et vocábis nomen ejus Joánnem: et erit magnus coram Dómino: et Spíritu Sancto replébitur adhuc ex útero matris suæ: et multi in nativitáte ejus gaudébunt. \emph{Ps. 20, 2} Dómine, in virtúte tua lætábitur rex: et super salutáre tuum exsultábit veheménter.
℣. Gloria Patri \emph{\&c.}
}\switchcolumn\portugues{
\rlettrine{N}{ão} temas, Zacarias, pois a tua oração foi ouvida: Isabel, tua esposa, dará à luz um filho, a quem darás o nome de João. Ele será grande diante de Deus e cheio do Espírito Santo desde o seio de sua mãe. Muitos se regozijarão com seu nascimento. \emph{Sl. 20, 2} Senhor, alegrar-se-á o Rei com seu poder: e exultará com viva alegria, esperando a vossa salvação.
℣. Glória ao Pai \emph{\&c.}
}\end{paracol}

\paragraph{Oração}
\begin{paracol}{2}\latim{
\rlettrine{P}{ræsta,} quǽsumus, omnipotens Deus: ut familia tua per viam salútis incedat; et, beáti Joánnis Præcursóris hortaménta sectándo ad eum, quem prædíxit, secura perveniat, Dóminum nostrum Jesum Christum, Fílium tuum: Qui tecum vivit et regnat \emph{\&c.}
}\switchcolumn\portugues{
\slettrine{Ó}{} Deus omnipotente, Vos suplicamos, permiti que a vossa família caminhe sempre pela via da salvação, a fim de que, sendo fiel aos ensinos do B. João, o Precursor, consiga encontrar Aquele que ele anunciou: nosso Senhor Jesus Cristo, que convosco vive e reina \emph{\&c.}
}\end{paracol}

\paragraphinfo{Epístola}{Jr. 1, 4-10}
\begin{paracol}{2}\latim{
Léctio Jeremíæ Prophétæ.
}\switchcolumn\portugues{
Lição do Profeta Jeremias.
}\switchcolumn*\latim{
\rlettrine{I}{n} diébus illis: Factum est verbum Dómini ad me, dicens: Priúsquam te formárem in útero, novi te: et ántequam exíres de vulva, sanctificári te, et prophétam in géntibus dedi te. Et dixi: A a a, Dómine Deus: ecce, néscio loqui, quia puer ego sum. Et dixit Dóminus ad me: Noli dícere: Puer sum; quóniam ad ómnia, quæ mittam te, ibis: et univérsa, quæcúmque mandávero tibi, lóqueris. Ne 
tímeas a fácie eórum: quia tecum ego sum, ut éruam te, dicit Dóminus. Et misit Dóminus manum suam, et tétigit os meum: et dixit Dóminus ad me: Ecce, dedi verba mea in ore tuo; ecce, constítui te hódie super gentes et super regna, ut evéllas, et déstruas et dispérdas et díssipes et ædífices et plantes: dicit Dóminus omnípotens.
}\switchcolumn\portugues{
\rlettrine{N}{aquele} dias, foi-me dirigida a palavra do Senhor, dizendo-me: «Antes de te formar no seio de tua mãe, já te conhecia; e antes de saíres dele, já te santificara e consagrara Profeta entre os povos». E eu disse: «Ah! ah! ah! Senhor Deus, não sei falar, pois sou menino!». Então o Senhor disse: «Não digas «sou menino»; pois tu irás junto de todos aqueles a quem te mandar, e dirás tudo quanto te ordenar. Não temas, pois, quando estiveres na presença deles, porque estarei contigo para te livrar», diz o Senhor. Depois o Senhor estendeu a sua mão e tocou com ela na minha boca. E o Senhor disse-me: «Eis que pus as minhas palavras na tua boca; e te estabeleci hoje sobre os povos e os reinos, para arrancares e destruíres, para plantares e edificares», diz o Senhor omnipotente.
}\end{paracol}

\paragraphinfo{Gradual}{Jo. 1, 6-7}
\begin{paracol}{2}\latim{
\rlettrine{F}{uit} homo missus a Deo, cui nomen erat Joánnes. ℣. Hic venit, ut testimónium perhibéret de lúmine, paráre Dómino plebem perféctam.
}\switchcolumn\portugues{
\rlettrine{F}{oi} enviado por Deus um homem que se chamava João. ℣. Ele veio para dar testemunho da luz e preparar um povo perfeito para o Senhor.
}\end{paracol}

\paragraphinfo{Evangelho}{Lc. 1, 5-17}
\begin{paracol}{2}\latim{
\cruz Initium sancti Evangélii secúndum Lucam.
}\switchcolumn\portugues{
\cruz Início do santo Evangelho segundo S. Lucas.
}\switchcolumn*\latim{
\blettrine{F}{uit} in diébus Heródis, regis Judǽæ, sacérdos quidam nómine Zacharías, de vice Abía, et uxor illíus de filiábus Aaron, et nomen ejus Elísabeth. Erant autem justi ambo ante Deum, incedéntes in ómnibus mandátis et justificatiónibus Dómini sine queréla, et non erat illis fílius, eo quod esset Elísabeth stérilis, et ambo processíssent in diébus suis. Factum est autem, cum sacerdótio fungerátur in órdine vicis suæ ante Deum, secúndum consuetúdinem sacerdótii, sorte éxiit, ut incénsum póneret, ingréssus in templum Dómini: et omnis multitúdo pópuli erat orans foris hora incénsi. Appáruit autem illi Angelus Dómini, stans a dextris altáris incénsi. Et Zacharias turbátus est, videns, et timor írruit super eum. Ait autem ad illum Angelus: Ne tímeas, Zacharía, quóniam exaudíta est deprecátio tua: et uxor tua Elísabeth páriet tibi fílium, et vocábis nomen ejus Joánnem: et erit gáudium tibi et exsultátio, et multi in nativitáte ejus gaudébunt: erit enim magnus coram Dómino: et vinum, et síceram non bibet, et Spíritu Sancto replébitur adhuc ex útero matris suæ: et multos filiórum Israël convértet ad Dóminum, Deum ipsórum: et ipse præcédet ante illum in spíritu et virtúte Elíæ: ut convértat corda patrum in fílios, et incrédulos ad prudéntiam justórum, paráre Dómino plebem perféctam.
}\switchcolumn\portugues{
\blettrine{N}{o} tempo em que Herodes reinava na Judeia, havia um sacerdote chamado Zacarias, da classe de Abias. Sua mulher era filha de Aarão e chamava-se Isabel. Ambos eram justos diante de Deus, cumprindo irrepreensivelmente todos os mandamentos e preceitos do Senhor. Não tinham filhos (pois Isabel era estéril); e ambos já estavam em idade avançada. Ora aconteceu que, enquanto Zacarias desempenhava as funções do seu ministério diante de Deus, na altura da sua classe, segundo o costume entre os sacerdotes, foi escolhido pela sorte para entrar no santuário do Senhor e oferecer aí o incenso. E quando era a hora própria do incenso, toda a multidão do povo estava fora em oração. Então, um Anjo do Senhor apareceu-lhe, estando de pé e à direita do altar do incenso. Logo que Zacarias o viu, ficou perturbado e cheio de temor. Mas o Anjo disse-lhe: «Não temas, Zacarias; pois a tua oração foi ouvida: tua mulher dará à luz um filho, ao qual chamarás João, e será para ti motivo de júbilo e de alegria, regozijando-se muitos com seu nascimento, pois será grande diante do Senhor; não beberá vinho, nem cidra e será cheio do Espírito Santo desde o seio de sua mãe. Ele converterá ao Senhor, seu Deus, muitos filhos de Israel e caminhará adiante d’Ele com o espírito e o poder de Elias para chamar os corações dos pais aos filhos e os rebeldes à prudência dos justos, e assim preparar para o Senhor um povo perfeito».
}\end{paracol}

\paragraphinfo{Ofertório}{Página \pageref{1vigiliaapostolos}}

\paragraph{Secreta}
\begin{paracol}{2}\latim{
\rlettrine{M}{únera,} Dómine, obláta sanctífica: et, intercedénte beáto Joánne Baptista, nos per hæc a peccatórum nostrórum máculis emúnda. Per Dóminum \emph{\&c.}
}\switchcolumn\portugues{
\rlettrine{S}{antificai,} Senhor os dons que Vos são oferecidos e pela intercessão do B. João Baptista fazei que nos purifiquem das manchas dos nossos pecados. Por nosso Senhor \emph{\&c.}
}\end{paracol}

\paragraphinfo{Comúnio}{Sl. 20, 6}
\begin{paracol}{2}\latim{
\rlettrine{M}{agna} est glória ejus in salutári tuo: glóriam et magnum decórem ímpones super eum, Dómine.
}\switchcolumn\portugues{
\rlettrine{G}{rande} é, Senhor, a sua glória, graças à vossa protecção. Vós o rodeastes de glória e de magnificência.
}\end{paracol}

\paragraph{Postcomúnio}
\begin{paracol}{2}\latim{
\rlettrine{B}{eáti} Joánnis Baptístæ nos, Dómine, præclára comitétur orátio: et, quem ventúrum esse prædíxit, poscat nobis fore placátum, Dóminum nostrum Jesum Christum, Fílium tuum: Qui tecum vivit et regnat \emph{\&c.}
}\switchcolumn\portugues{
\rlettrine{F}{azei,} Senhor, que a insigne Oração do B. João Baptista acompanhe a nossa e que implore em nosso favor a misericórdia d’Aquele que ele profetizou: nosso Senhor Jesus Cristo, vosso Filho, que convosco vive e reina \emph{\&c.}
}\end{paracol}
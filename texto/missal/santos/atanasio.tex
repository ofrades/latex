\subsectioninfo{S. Atanásio}{2 de Maio}

\paragraphinfo{Intróito}{Ecl. 15, 5}
\begin{paracol}{2}\latim{
\rlettrine{I}{n} médio Ecclésiæ apéruit os ejus: et implévit eum Dóminus spíritu sapiéntiæ et intelléctus: stolam glóriæ índuit eum, allelúja, allelúja. \emph{Ps. 91, 2} Bonum est confitéri Dómino: et psállere nómini tuo, Altíssime.
℣. Gloria Patri \emph{\&c.}
}\switchcolumn\portugues{
\rlettrine{A}{briu-lhe} o Senhor a boca no meio da Igreja e encheu-o com o espírito da sabedoria e da inteligência: cobriu-o com o manto da glória, aleluia, aleluia. \emph{Sl. 91, 2} É bom louvar o Senhor e cantar o vosso nome, ó Altíssimo.
℣. Glória ao Pai \emph{\&c.}
}\end{paracol}

\paragraphinfo{Oração, Secreta e Postcomúnio}{Página \pageref{confessorespontifices2}}

\paragraphinfo{Epístola}{2. Cor. 4, 5-14}
\begin{paracol}{2}\latim{
Léctio Epístolæ beáti Pauli Apóstoli ad Corínthios.
}\switchcolumn\portugues{
Lição da Ep.ª do B. Ap.º Paulo aos Coríntios.
}\switchcolumn*\latim{
\rlettrine{F}{ratres:} Non nosmetípsos prædicámus, sed Jesum Christum, Dóminum nostrum: nos autem servos vestros per Jesum: quóniam Deus, qui dixit de ténebris lucem splendéscere, ipse illúxit in córdibus nostris ad illuminatiónem sciéntiæ claritátis Dei, in fácie Christi Jesu. Habémus autem thesáurum istum in vasis fictílibus: ut sublímitas sit virtútis Dei, et non ex nobis. In ómnibus tribulatiónem pátimur, sed non angustiámur: aperiántur, sed non destitúimur: persecutiónem pátimur, sed non derelínquimur: dejícimur, sed non perímus: semper mortificatiónem Jesu in córpore nostro circumferéntes, ut et vita Jesu manifestétur in corpóribus nostris. Semper enim nos, qui vívimus, in mortem trádimur propter Jesum: ut et vita Jesu manifestétur in carne nostra mortáli. Ergo mors in nobis operátur, vita autem in vobis. Habéntes autem eúndem spíritum fidei, sicut scriptum est: Crédidi, propter quod locútus sum: et nos crédimus, propter quod et lóquimur: sciéntes, quóniam, qui suscitávit Jesum, et nos cum Jesu suscitábit et constítuet vobíscum.
}\switchcolumn\portugues{
\rlettrine{M}{eus} irmãos: Não nos pregamos a nós próprios, mas a Jesus Cristo, nosso Senhor. Consideramo-nos como vossos servos por Jesus, pois o mesmo Deus, que fez sair a luz das trevas, fez também brilhar a sua luz nos nossos corações, a fim de que pudéssemos iluminar os outros com o conhecimento da glória de Deus, que resplandece na face de Jesus Cristo. Porém, possuímos este tesouro em vasos de barro, para que a sublimidade deste trabalho seja atribuída a Deus e não a nós. Sofremos tribulações em todas as cousas, mas não desanimamos; encontramo-nos em grandes apuros, mas não sucumbimos; somos perseguidos, mas não desesperamos; somos esmagados, mas não perecemos. Trazemos sempre no nosso corpo a morte de Jesus, a fim de que a vida de Jesus se manifeste também na nossa carne mortal; porque nós, que vivemos, somos incessantemente entregues à morte por causa de Jesus, a fim de que a vida de Jesus seja manifestada em nós. Desta maneira a morte opera em nós, e a vida em vós. Mas, visto que temos um mesmo espírito de fé, conforme o que está escrito: «acreditei e por isso falei», nós também acreditamos, por isso mesmo é que falamos, convencidos de que Aquele que ressuscitou Jesus nos ressuscitará também com Jesus e nos colocará convosco.
}\end{paracol}

\begin{paracol}{2}\latim{
Allelúja, allelúja. ℣. \emph{Ps. 109, 4} Tu es sacérdos in ætérnum, secúndum órdinem Melchísedech. Allelúja. ℣. \emph{Jac. 1, 12} Beátus vir, qui suffert tentatiónem: quóniam, cum probátus fúerit, accípiet corónam vitæ. Allelúja.
}\switchcolumn\portugues{
Aleluia, aleluia. ℣. \emph{Sl. 109, 4} Tu és sacerdote para sempre, segundo a ordem de Melquisedeque! Aleluia. ℣. \emph{Tg. 1, 12} Bem-aventurado o varão que sofre tentação; pois, quando acabar a tentação, receberá a coroa da vida. Aleluia.
}\end{paracol}

\paragraphinfo{Evangelho}{Página \pageref{cirilojerusalem}}

\paragraphinfo{Ofertório}{Sl. 88, 21-22}
\begin{paracol}{2}\latim{
\rlettrine{I}{nvéni} David servum meum, óleo sancto meo unxi eum: manus enim mea auxiliábitur ei, et bráchium meum confortábit eum, allelúja.
}\switchcolumn\portugues{
\rlettrine{E}{ncontrei} o meu servo David e ungi-o com meu óleo sagrado: e a minha mão o auxiliará e o meu braço o fortalecerá, aleluia.
}\end{paracol}

\paragraphinfo{Comúnio}{Mt. 10, 27}
\begin{paracol}{2}\latim{
\qlettrine{Q}{uod} dico vobis in ténebris, dícite in lúmine, dicit Dóminus: et quod in aure audítis, prædicáte super tecta, allelúja.
}\switchcolumn\portugues{
\rlettrine{A}{quilo} que vos digo ao ouvido pregai-o sobre os tectos, aleluia.
}\end{paracol}

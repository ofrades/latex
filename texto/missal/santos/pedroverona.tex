\subsectioninfo{S. Pedro de Verona, Mártir}{29 de Abril}

\textit{Como na Missa Protexísti me, página \pageref{martir}, excepto:}

\paragraph{Oração}
\begin{paracol}{2}\latim{
\rlettrine{P}{ræsta,} quǽsumus, omnípotens Deus: ut beáti Petri Martyris tui fidem cóngrua devotióne sectémur; qui, pro ejúsdem fídei dilatatióne, martýrii palmam méruit obtinére. Per Dóminum \emph{\&c.}
}\switchcolumn\portugues{
\rlettrine{V}{os} suplicamos, ó Deus, omnipotente, permiti que imitemos com conveniente devoção a fé do B. Pedro, vosso Mártir, que pela manifestação desta mesma fé mereceu alcançar a palma do martírio. Por nosso Senhor \emph{\&c.}
}\end{paracol}

\paragraphinfo{Epístola}{Página \pageref{martirnaopontifice2}}

\paragraph{Secreta}
\begin{paracol}{2}\latim{
\rlettrine{P}{reces,} quas tibi, Dómine, offérimus, intercedénte beáto Petro Mártyre tuo, cleménter inténde: et propugnatóres fídei sub tua protectióne custódi. Per Dóminum \emph{\&c.}
}\switchcolumn\portugues{
\rlettrine{D}{ignai-Vos,} Senhor, pela intercessão do B. Pedro, vosso Mártir, ouvir clementemente as preces que Vos dirigimos, e acolhei sob a vossa protecção os defensores da fé. Por nosso Senhor \emph{\&c.}
}\end{paracol}

\paragraph{Postcomúnio}
\begin{paracol}{2}\latim{
\rlettrine{F}{idéles} tuos, Dómine, custódiant sacraménta, quæ súmpsimus: et, intercedénte beáto Petro Mártyre tuo, contra omnes advérsos tueántur incúrsus. Per Dóminum \emph{\&c.}
}\switchcolumn\portugues{
\qlettrine{Q}{ue} os vossos fiéis, Senhor, sejam amparados com os sacramentos que recebemos, e que pela intercessão do B. Pedro, vosso Mártir, sejam protegidos contra todos os ataques do inimigo. Por nosso Senhor \emph{\&c.}
}\end{paracol}

\subsectioninfo{S. Francisco Caracíolo, Conf.}{4 de Junho}

\paragraphinfo{Intróito}{Sl. 21, 15; 68, 10}
\begin{paracol}{2}\latim{
\rlettrine{F}{actum} est cor meum tamquam cera liquéscens in médio ventris mei: quóniam zelus domus tuæ comédit me. (T.P. Allelúja, allelúja.) \emph{Ps. 72, 1} Quam bonus Israël Deus: his, qui recto sunt corde!
℣. Gloria Patri \emph{\&c.}
}\switchcolumn\portugues{
\rlettrine{O}{} meu coração é como a cera: funde-se no meio do meu peito, pois o zelo da vossa Casa devora-me. (T. P. Aleluia, aleluia.) \emph{Sl. 72, 1} Como Deus é bom para com Israel! e para com aqueles que possuem coração recto!
℣. Glória ao Pai \emph{\&c.}
}\end{paracol}

\paragraph{Oração}
\begin{paracol}{2}\latim{
\rlettrine{D}{eus,} qui beátum Francíscum, novi órdinis institutórem, orándi stúdio et pœniténtiæ amóre decorásti: da fámulis tuis in ejus imitatióne ita profícere; ut, semper orántes et corpus in servitútem redigéntes, ad cœléstem glóriam perveníre mereántur. Per Dóminum \emph{\&c.}
}\switchcolumn\portugues{
\slettrine{Ó}{} Deus, que ilustrastes o B. Francisco, destinando-o para fundador duma nova ordem e dotando-o com o zelo da oração e amor à penitência, concedei aos vossos servos a graça de aproveitarem de tal modo com seus exemplos que, rezando sempre e reduzindo os corpos à servidão, mereçam alcançar a glória celestial. Por nosso Senhor \emph{\&c.}
}\end{paracol}

\paragraphinfo{Epístola}{Sb. 4, 7-14}
\begin{paracol}{2}\latim{
Léctio libri Sapiéntiæ.
}\switchcolumn\portugues{
Lição do Livro da Sabedoria.
}\switchcolumn*\latim{
\qlettrine{J}{ustus,} si morte præoccupátus fúerit, in refrigério erit. Senéctus enim venerábilis est non diutúrna, neque annórum número computáta: cani autem sunt sensus hóminis, et ætas senectútis vita immaculáta. Placens Deo Jactus est diléctus, et vivens inter peccatóres translátus est. Raptus est, ne malítia mutáret intelléctum ejus, aut ne fíctio decíperet ánimam illíus. Fascinátio enim nugacitátis obscúrat bona, et inconstántia concupiscéntia? transvértit sensum sine malítia. Consummátus in brevi explévit témpora multa, plácita enim erat Deo ánima illíus: propter hoc properávit edúcere illum de médio iniquitátum.
}\switchcolumn\portugues{
\rlettrine{A}{inda} que o justo morra prematuramente, alcançará repouso. O que torna a velhice venerável não é a vida longa nem o número dos anos, mas a prudência do homem; pois as cãs do homem não são os seus sentimentos. Sua vida imaculada, sim, é uma verdadeira velhice. Tendo-se (o justo) tornado agradável a Deus, foi por Ele amado; e Deus o tirou do meio dos pecadores com quem vivia. Deus elevou-o, receando que a malícia corrompesse o seu espírito ou a ilusão seduzisse a sua alma; pois a fascinação da frivolidade obscurece o bem, e a inconstância transtorna o espírito ainda que não possua malícia. Ainda que tenha vivido pouco, preencheu a carreira com larga vida, pois a sua alma era agradável a Deus; pelo que Deus se apressou em o tirar do meio da iniquidade.
}\end{paracol}

\paragraphinfo{Gradual}{Sl. 41, 2}
\begin{paracol}{2}\latim{
\qlettrine{Q}{uemádmodum} desíderat cervus ad fontes aquárum: ita desíderat ánima mea ad te, Deus. ℣. \emph{Ps. ibid., 3} Sitívit ánima mea ad Deum fortem vivum.
}\switchcolumn\portugues{
\rlettrine{A}{ssim} como o veado suspira pelas fontes das águas, assim a minha alma suspira por Vós, ó Deus. ℣. \emph{Sl. ibid., 3} Minha alma tem sede de Deus forte e vivo.
}\switchcolumn*\latim{
Allelúja, allelúja. ℣. \emph{Ps. 72, 26 } Defécit caro mea et cor meum: Deus cordis mei, et pars mea Deus in ætérnum. Allelúja.
}\switchcolumn\portugues{
Aleluia, aleluia. ℣. \emph{Sl. 72, 26 } Minha carne e o meu coração desfalecem! Ó Deus, sois o Deus do meu coração e a minha herança na eternidade. Aleluia.
}\end{paracol}

\textit{No T. Pascal omite-se o Gradual, e diz-se:}

\begin{paracol}{2}\latim{
Allelúja, allelúja. ℣. \emph{Ps. 64, 5} Beátus, quem elegísti et assumpsísti: inhabitábit in átriis tuis. Allelúja. ℣. \emph{Ps. 111, 9} Dispérsit, dedit paupéribus: justítia ejus manet in sǽculum sǽculi. Allelúja.
}\switchcolumn\portugues{
Aleluia, aleluia. ℣. \emph{Sl. 64, 5} Bem-aventurado aquele que escolhestes e elevastes, para que habitasse nos vossos átrios. Aleluia. ℣. \emph{Sl. 111, 9} Distribuiu e deu esmola aos pobres e a sua justiça permanecerá em todos os séculos. Aleluia.
}\end{paracol}

\paragraphinfo{Evangelho}{Página \pageref{confessoresnaopontifices1}}

\paragraphinfo{Ofertório}{Sl. 91, 13}
\begin{paracol}{2}\latim{
\qlettrine{J}{ustus} ut palma florébit: sicut cedrus Líbani multiplicábitur. (T.P. Allelúja.)
}\switchcolumn\portugues{
\rlettrine{O}{} justo florescerá, como a palmeira, e crescerá, como o cedro do Líbano. (T. P. Aleluia.)
}\end{paracol}

\paragraph{Secreta}
\begin{paracol}{2}\latim{
\rlettrine{D}{a} nobis, clementíssime Jesu: ut præclára beáti Francísci mérita recoléntes, eódem nos, ac ille, caritátis igne succénsi, digne in circúitu sacræ hujus mensæ tuæ esse valeámus: Qui vivis \emph{\&c.}
}\switchcolumn\portugues{
\rlettrine{C}{lementíssimo} Jesus, honrando os preclaros méritos do B. Francisco e sendo abrasados, como ele, no fogo da caridade, concedei-nos que possamos tomar lugar dignamente junto da vossa sacrossanta mesa. Ó Vós, que viveis e \emph{\&c.}
}\end{paracol}

\paragraphinfo{Comúnio}{Sl. 30, 20}
\begin{paracol}{2}\latim{
\qlettrine{Q}{uam} magna multitúdo dulcédinis tuæ, Dómine, quam abscondísti timéntibus te! (T.P. Allelúja.)
}\switchcolumn\portugues{
\rlettrine{C}{omo} é grande, Senhor, a felicidade que reservais para aqueles que Vos temem! (T. P. Aleluia.)
}\end{paracol}

\paragraph{Postcomúnio}
\begin{paracol}{2}\latim{
\rlettrine{S}{acrosáncta} sacrifícii, quǽsumus, Dómine, quod hódie in sollemnitáte beáti Francísci tuæ obtúlimus majestáti, grata semper in méntibus nostris memória persevéret et fructus. Per Dóminum nostrum \emph{\&c.}
}\switchcolumn\portugues{
\rlettrine{P}{ermiti,} Senhor, Vos suplicamos, que o nosso espírito conserve sempre, cheio de reconhecimento, a recordação e os frutos do sacrossanto sacrifício que hoje oferecemos à vossa majestade em honra do B. Francisco. Por nosso Senhor \emph{\&c.}
}\end{paracol}

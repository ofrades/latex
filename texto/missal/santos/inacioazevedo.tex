\subsectioninfo{B. Inácio de Azevedo e Companheiros}{15 de Julho}

\textit{Como Missa Intret in, página \pageref{muitosmartires1}, excepto:}

\paragraph{Oração}
\begin{paracol}{2}\latim{
\rlettrine{O}{mnipotens} ætérne Deus, qui nos in beáto Ignátio et Sóciis ejus, ad prædicándam Brasilic géntibus Cathólicam fidem missis, quadraginta Mártyrum palmas sub una tribuis solemnitáte venerári: concéde propitius: ut quorum in cælis glóriam læti suspicimus, eórum invictam in fide constántiam æmulémur. Per Dóminum \emph{\&c.}
}\switchcolumn\portugues{
\slettrine{Ó}{} Deus omnipotente e eterno, que permitistes venerássemos em uma só solenidade as quarenta palmas conquistadas pelo B. Inácio e seus Companheiros de martírio, concedei-nos propício que a alegre contemplação da sua glória no céu nos estimule a imitarmos a invencível constância da sua fé. Por nosso Senhor \emph{\&c.}
}\end{paracol}

\paragraphinfo{Oração}{S. Henrique}
\begin{paracol}{2}\latim{
\rlettrine{D}{eus,} qui hodiérna die beátum Henrícum Confessórem tuum e terréni cúlmine impérii ad regnum ætérnum transtulísti: te súpplices exorámus; ut, sicut illum, grátiæ tuæ ubertáte prævéntum, illécebras sǽculi superáre fecísti, ita nos fácias, ejus imitatióne, mundi hujus blandiménta vitáre, et ad te puris méntibus perveníre. Per Dóminum nostrum \emph{\&c.}
}\switchcolumn\portugues{
\slettrine{Ó}{} Deus, que neste dia fizestes passar o B. Henrique, vosso Confessor, de um trono da terra para o reino eterno, Vos suplicamos humildemente que, assim como, enchendo-o com a abundância da vossa graça, o fizestes triunfar dos atractivos deste mundo, assim também nos façais evitar as seduções deste mundo e chegar até Vós com pureza de espírito. Por nosso Senhor \emph{\&c.}
}\end{paracol}

\paragraph{Secreta}
\begin{paracol}{2}\latim{
\rlettrine{S}{úscipe,} Dómine, acceptíssimum Unigéniti tui sacrifícium: ac, beáto Ignátio et Sóciis ejus intercedéntibus, præsta: ut, quod illos roborávit ad pugnam, nos effíciat in tuo servítio atque amóre fervéntes. Per eúndem \emph{\&c.}
}\switchcolumn\portugues{
\rlettrine{R}{ecebei,} Senhor, o agradabilíssimo sacrifício do vosso Filho Unigénito; e, pela intercessão do B. Inácio e seus Companheiros, fazei que este mistério, que os fortaleceu para o combate, nos torne fervorosos no vosso serviço e amor. Pelo mesmo \emph{\&c.}
}\end{paracol}

\paragraphinfo{Secreta}{S. Henrique}
\begin{paracol}{2}\latim{
\rlettrine{L}{audis} tibi, Dómine, hóstias immolámus in tuórum commemoratióne Sanctórum: quibus nos et præséntibus éxui malis confídimus et futúris. Per Dóminum \emph{\&c.}
}\switchcolumn\portugues{
\rlettrine{V}{os} oferecemos este sacrifício de louvor em memória dos vossos Santos, para que por meio dele nos livremos dos males presentes e futuros. Por nosso Senhor \emph{\&c.}
}\end{paracol}

\paragraph{Postcomúnio}
\begin{paracol}{2}\latim{
\rlettrine{F}{idem} in nobis, Dómine, quam de miseréntis grátiæ tuæ múnere suscépimus, áugeant hæc sancta mystéria: nosque ad eam contra spirituália nequitiæ colluctántes profiténdam, beatórum Ignátii et Sociórum ejus exémpla confirment. Per Dóminum nostrum Jesum Christum \emph{\&c.}
}\switchcolumn\portugues{
\rlettrine{F}{azei,} Senhor, que estes sacrossantos mistérios aumentem em nós a fé, que possuímos como um dom da vossa misericordiosa graça; e que na luta contra os espíritos do mal sejamos fiéis à profissão desta fé, segundo os exemplos do B. Inácio e seus Companheiros. Por nosso Senhor \emph{\&c.}
}\end{paracol}

\paragraphinfo{Postcomúnio}{S. Henrique}
\begin{paracol}{2}\latim{
\rlettrine{R}{efécti} cibo potúque cœlésti, Deus noster, te súpplices exorámus: ut, in cujus hæc commemoratióne percépimus, ejus muniámur et précibus. Per Dóminum \emph{\&c.}
}\switchcolumn\portugues{
\rlettrine{F}{ortalecidos} com o alimento e com a bebida celestiais, ó nosso Deus, Vos suplicamos humildemente que nos protejam as preces daquele em cuja memória os recebemos. Por \emph{\&c.}
}\end{paracol}

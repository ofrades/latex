\subsectioninfo{S. José Cupertino, Conf.}{18 de Setembro}

\paragraphinfo{Intróito}{Ecl. 1, 14-15}
\begin{paracol}{2}\latim{
\rlettrine{D}{iléctio} Dei honorábilis sapiéntia: quibus autem apparúerit in visu, díligunt eam in visióne et in agnitióne magnálium suórum. \emph{Ps. 83, 2} Quam dilécta tabernácula tua, Dómine virtútum! concupíscit, et déficit ánima mea in átria Dómini.
℣. Gloria Patri \emph{\&c.}
}\switchcolumn\portugues{
\rlettrine{O}{} amor de Deus é sabedoria verdadeiramente digna de ser honrada. Aqueles a quem ela se manifesta amam-na, contemplando e admirando as suas maravilhas. \emph{Sl. 83, 2} Quão dilectos são os vossos tabernáculos, ó Deus dos exércitos!
℣. Glória ao Pai \emph{\&c.}
}\end{paracol}

\paragraph{Oração}
\begin{paracol}{2}\latim{
\rlettrine{D}{eus,} qui ad unigénitum Fílium tuum exaltátum a terra ómnia tráhere disposuísti: pérfice propítius; ut, méritis et exémplo seráphici Confessóris tui Joséphi, supra terrénas omnes cupiditátes eleváti, ad eum perveníre mereámur: Qui tecum vivit \emph{\&c.}
}\switchcolumn\portugues{
\slettrine{Ó}{} Deus, que, depois que o vosso Filho Unigénito foi elevado da terra, quisestes atrair tudo a Ele, concedei-nos propício que, pelos méritos e exemplo do vosso Seráfico Confessor José, elevando-nos acima de todos os desejos terrenos, consigamos chegar até Àquele: Que convosco vive \emph{\&c.}
}\end{paracol}

\paragraphinfo{Epístola}{1. Cor. 13, 1-8}
\begin{paracol}{2}\latim{
Léctio Epístolæ beáti Pauli Apóstoli ad Corínthios.
}\switchcolumn\portugues{
Lição da Ep.ª do B. Ap.º Paulo aos Coríntios.
}\switchcolumn*\latim{
\rlettrine{F}{ratres:} Si linguis hóminum loquar et Angelorum, caritátem autem non hábeam, factus sum velut æs sonans aut cýmbalum tínniens. Et si habúero prophetiam, et nóverim mystéria ómnia et omnem sciéntiam: et si habúero omnem fidem, ita ut montes tránsferam, caritátem autem non habúero, nihil sum. Et si distribúere in cibos páuperum omnes facultátes meas, et si tradídere corpus meum, ita ut árdeam, caritátem autem non habúero, nihil mihi prodest. Cáritas pátiens est, benígna est: cáritas non æmulátur, non agit pérperam, non inflátur, non est ambitiósa, non quærit quæ sua sunt, non irritátur, non cógitat malum, non gaudet super iniquitáte, congáudet autem veritáti: ómnia suffert, ómnia credit, ómnia sperat, ómnia sústinet. Cáritas numquam éxcidit: sive prophétiæ evacuabúntur, sive linguæ cessábunt, sive sciéntia destruétur.
}\switchcolumn\portugues{
\rlettrine{M}{eus} irmãos: Se eu falar as línguas dos homens e dos Anjos, mas não tiver caridade, sou como o metal, que tine, ou como o sino, que soa. E se eu tiver o dom de profecia, conhecer todos os mystérios e possuir toda a ciência; e se tiver toda a fé, até ser capaz de transportar montanhas, mas não tiver caridade, nada sou. Se eu distribuir todos meus bens para sustento dos pobres, e se entregar o meu corpo para ser queimado, mas não tiver caridade, de nada me aproveitará, A caridade é paciente, é benigna: não é invejosa, não é leviana, não é soberba, não é ambiciosa, não procura o próprio interesse, não se irrita, não julga mal, não se alegra com a injustiça, antes se regozija com a verdade, sofre tudo, acredita em tudo, tudo espera, tudo suporta. A caridade nunca perecerá, ainda que já não houvesse mais profecias, ainda que as línguas acabassem, ainda que a ciência desaparecesse.
}\end{paracol}

\paragraphinfo{Gradual}{Sl. 20, 4-5}
\begin{paracol}{2}\latim{
\rlettrine{D}{ómine,} prævenísti eum in benedictiónibus dulcédinis: posuísti in cápite ejus corónam de lápide pretióso. ℣. Vitam pétiit a te, et tribuísti ei longitudinem dierum in sǽculum, et in sǽculum sǽculi.
}\switchcolumn\portugues{
\rlettrine{S}{enhor,} concedestes-lhe bênçãos escolhidas, as mais suaves; e impusestes na sua cabeça uma coroa de pedras preciosas. ℣. Concedestes-lhe a vida, que ele Vos suplicara, e prolongastes-lhe a duração dos seus dias pelos séculos dos séculos.
}\switchcolumn*\latim{
Allelúja, allelúja. ℣. \emph{Eccli. 11, 13} Oculus Dei respéxit illum in bono, et eréxit eum ab humilitáte ipsíus, et exaltávit caput ejus. Allelúja.
}\switchcolumn\portugues{
Aleluia, aleluia. ℣. \emph{Ecl. 11, 13} Deus olhou benignamente para ele e ergueu-o da sua humilhação; e ele elevou a cabeça. Aleluia.
}\end{paracol}

\paragraphinfo{Evangelho}{Página \pageref{19domingopentecostes}}

\paragraphinfo{Ofertório}{Sl. 34, 13}
\begin{paracol}{2}\latim{
\rlettrine{E}{go} autem, cum mihi molésti essent, induébar cilício. Humiliábam in jejúnio ánimam meam: et orátio mea in sinu meo convertétur.
}\switchcolumn\portugues{
\rlettrine{E}{u,} porém, enquanto eles me atormentavam, revestia-me com um cilício; humilhava a minha alma com o jejum; e a minha oração repousava no meu peito.
}\end{paracol}

\paragraph{Secreta}
\begin{paracol}{2}\latim{
\rlettrine{L}{audis} tibi, Dómine, hóstias immolámus in tuórum commemoratióne Sanctórum: quibus nos et præséntibus éxui malis confídimus et futúris. Per Dóminum \emph{\&c.}
}\switchcolumn\portugues{
\rlettrine{V}{os} oferecemos, Senhor, este sacrifício de louvor em memória dos vossos Santos, para que por meio dele nos livremos dos males presentes e futuros. Por nosso Senhor \emph{\&c.}
}\end{paracol}

\paragraphinfo{Comúnio}{Sl. 68. 30-31}
\begin{paracol}{2}\latim{
\rlettrine{E}{go} sum pauper et dolens: salus tua, Deus, suscépit me. Laudábo nomen Dei cum cantico: et magnificábo eum in laude.
}\switchcolumn\portugues{
\rlettrine{E}{stou} pobre e aflito: vossa salvação, ó meu Deus, acolheu-me. Louvarei o nome de Deus com cânticos e glorificá-l’O-ei com louvores.
}\end{paracol}

\paragraph{Postcomúnio}
\begin{paracol}{2}\latim{
\rlettrine{R}{efécti} cibo potúque cœlésti, Deus noster, te súpplices exorámus: ut, in cujus hæc commemoratióne percépimus, ejus muniámur et précibus. Per Dóminum \emph{\&c.}
}\switchcolumn\portugues{
\rlettrine{F}{ortalecidos} com o alimento e a bebida celestiais, Vos suplicamos, humildemente, ó nosso Deus, que sejamos amparados com a protecção e as preces daquele em cuja memória os recebemos. Por nosso Senhor \emph{\&c.}
}\end{paracol}

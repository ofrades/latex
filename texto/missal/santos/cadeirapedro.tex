\subsectioninfo{Cadeira de S. Pedro em Roma}{22 de Fevereiro}\label{cadeirapedro}

\paragraphinfo{Intróito}{Ecl. 45, 30}
\begin{paracol}{2}\latim{
\rlettrine{S}{tátuit} ei Dóminus testaméntum pacis, et príncipem fecit eum: ut sit illi sacerdótii dígnitas in ætérnum. \emph{Ps. 131, 1} Meménto, Dómine, David: et omnis mansuetúdinis ejus.
℣. Gloria Patri \emph{\&c.}
}\switchcolumn\portugues{
\rlettrine{O}{} Senhor fez com ele uma aliança de paz e proclamou-o príncipe, para que a dignidade sacerdotal lhe pertencesse eternamente. \emph{Sl. 131, 1} Lembrai-Vos de David, ó Senhor, e da sua grande solicitude.
℣. Glória ao Pai \emph{\&c.}
}\end{paracol}

\paragraph{Oração}
\begin{paracol}{2}\latim{
\rlettrine{D}{eus,} qui beáto Petro Apóstolo tuo, collátis clávibus regni cœléstis, ligándi atque solvéndi pontifícium tradidísti: concéde; ut, intercessiónis ejus auxílio, a peccatórum nostrórum néxibus liberémur: Qui vivis \emph{\&c.}
}\switchcolumn\portugues{
\slettrine{Ó}{} Deus, que, entregando ao B. Pedro, vosso Apóstolo, as chaves do reino celestial, lhe destes a autoridade pontifícia de «ligar e de desligar», concedei-nos, por intercessão do seu auxílio, que sejamos livres das cadeias dos nossos pecados. Ó Vós, que \emph{\&c.}
}\end{paracol}

\paragraphinfo{Oração}{S. Paulo}
\begin{paracol}{2}\latim{
\rlettrine{D}{eus,} qui multitúdinem géntium beáti Pauli Apóstoli prædicatióne docuísti: da nobis, quǽsumus; ut, cujus commemoratiónem cólimus, ejus apud te patrocínia sentiámus. (Per Dóminum \emph{\&c.})
}\switchcolumn\portugues{
\slettrine{Ó}{} Deus, que ensinastes a multidão dos povos por meio da pregação do B. Ap.º Paulo, concedei-nos, Vos pedimos, que, honrando a sua memória, gozemos o seu patrocínio junto de Vós. (Por nosso Senhor \emph{\&c.})
}\end{paracol}

\paragraphinfo{Oração}{Santa Prisca}
\begin{paracol}{2}\latim{
\rlettrine{D}{a,} quǽsumus, omnípotens Deus: ut, qui beátæ Priscæ Vírginis et Mártyris tuæ natalítia cólimus; et ánnua sollemnitáte lætémur, et tantae fídei proficiámus exémplo. Per Dóminum \emph{\&c.}
}\switchcolumn\portugues{
\rlettrine{C}{oncedei-nos,} ó Deus omnipotente, Vos rogamos, que, celebrando o nascimento da B. Prisca, vossa Virgem e Mártir, nos alegremos nesta solenidade anual e aproveitemos com os exemplos da sua tão grande fé. Por nosso Senhor \emph{\&c.}
}\end{paracol}

\paragraphinfo{Epístola}{1. Pe. 1, 1-7}
\begin{paracol}{2}\latim{
Léctio Epístolæ beáti Petri Apóstoli.
}\switchcolumn\portugues{
Lição da Ep.ª do B. Ap.º Pedro.
}\switchcolumn*\latim{
\rlettrine{P}{etrus,} Apóstolus Jesu Christi, eléctis ádvenis dispersiónis Ponti, Galátiæ, Cappadóciæ, Asiæ et Bithýniæ secúndum præsciéntiam Dei Patris, in sanctificatiónem Spíritus, in obœdiéntiam, et aspersiónem sánguinis Jesu Christi: grátia vobis et pax multiplicátus Benedíctus Deus et Pater Dómini nostri Jesu Christi, qui secúndum misericórdiam suam magnam regenerávit nos in spem vivam, per resurrectiónem Jesu Christi ex mórtuis, in hereditátem incorruptíbilem et incontaminátam et immarcescíbilem, conservátam in cœlis in vobis, qui in virtúte Dei custodímini per fidem in salútem, parátam revelári in témpore novíssimo. In quo exsultábitis, módicum nunc si opórtet contristári in váriis tentatiónibus: ut probátio vestræ fídei multo pretiósior auro (quod per ignem probatur) inveniátur in laudem et glóriam et honórem, in revelatióne Jesu Christi, Dómini nostri.
}\switchcolumn\portugues{
\rlettrine{P}{edro,} Apóstolo de Jesus Cristo, aos fiéis estrangeiros, que estão dispersos no Ponto, na Galácia, na Capadócia, na Ásia e na Bitínia, eleitos, segundo a presciência de Deus Pai, para receberem a santificação do Espírito Santo, obedecerem e serem aspergidos com o sangue de Jesus Cristo: Que a graça e a paz de Deus cresçam cada vez mais em vós! Bendito seja Deus e Pai de nosso Senhor Jesus Cristo, que, segundo a sua infinita misericórdia, nos regenerou para nos dar uma viva esperança, pela Ressurreição de Jesus Cristo dos mortos, a fim de nos conduzir à herança incorruptível, incontaminada e imarcescível, reservada nos céus para vós, que com o poder de Deus sois guardados na fé, para a salvação, que será revelada no fim dos tempos. Então exultareis em transportes de alegria, ainda que, por agora, deveis estar um pouco contristados com algumas tentações, a fim de que a vossa fé, assim provada, se torne muito mais preciosa do que o ouro (que também é provado pelo fogo) e seja encontrada digna de louvor, de glória e de honra, na vida gloriosa de nosso Senhor Jesus Cristo.
}\end{paracol}

\paragraphinfo{Gradual}{Sl. 106, 32, 31}
\begin{paracol}{2}\latim{
\rlettrine{E}{xáltent} eum in Ecclésia plebis: et in cáthedra seniórum laudent eum. ℣. Confiteántur Dómino misericórdiæ ejus; et mirabília ejus fíliis hóminum.
}\switchcolumn\portugues{
\rlettrine{E}{xaltai-O} na assembleia do povo e louvai-O no conselho dos anciãos. ℣. Glorificai o Senhor pela sua misericórdia e pelas suas maravilhas, operadas em favor dos filhos dos homens.
}\switchcolumn*\latim{
Allelúja, allelúja. ℣. \emph{Matth. 16, 18} Tu es Petrus, et super hanc petram ædificábo Ecclésiam meam. Allelúja.
}\switchcolumn\portugues{
Aleluia, aleluia. ℣. \emph{Mt. 16, 18} Tu és Pedro, e sobre esta pedra edificarei a minha Igreja. Aleluia.
}\end{paracol}

\textit{Após a Septuagésima omite-se o Aleluia e o Verso e diz-se:}

\paragraphinfo{Trato}{Mt. 16, 18-19}
\begin{paracol}{2}\latim{
\rlettrine{P}{etrus,} et super hanc petram ædificábo Ecclésiam meam. ℣. Et portæ ínferi non prævalébunt advérsus eam: et tibi dabo claves regni cœlórum. ℣. Quodcúmque ligáveris super terram, erit ligátum et in cœlis. ℣. Et quodcúmque sólveris super terram, erit solútum et in cœlis.
}\switchcolumn\portugues{
\rlettrine{T}{u} és Pedro, e sobre esta pedra edificarei a minha Igreja. ℣. E as portas do inferno não prevalecerão contra ela: E dar-te-ei as chaves do reino dos céus. ℣. Tudo o que ligares sobre a terra será ligado nos céus. ℣. E tudo o que desligares sobre a terra será desligado nos céus.
}\end{paracol}

\paragraphinfo{Evangelho}{Mt. 16, 13-19}
\begin{paracol}{2}\latim{
\cruz Sequéntia sancti Evangélii secúndum Lucam.
}\switchcolumn\portugues{
\cruz Continuação do santo Evangelho segundo S. Mateus.
}\switchcolumn*\latim{
\blettrine{I}{n} illo témpore: Venit Jesus in partes Cæsaréæ Philíppi, et interrogábat discípulos suos, dicens: Quem dicunt hómines esse Fílium hóminis? At illi dixérunt: Alii Joánnem Baptístam, alii autem Elíam, alii vero Jeremíam aut unum ex prophétis. Dicit illis Jesus: Vos autem quem me esse dícitis? Respóndens Simon Petrus, dixit: Tu es Christus, Fílius Dei vivi. Respóndens autem Jesus, dixit ei: Beátus es, Simon Bar Jona: quia caro et sanguis non revelávit tibi, sed Pater meus, qui in cœlis est. Et ego dico tibi, quia tu es Petrus, et super hanc petram ædificábo Ecclésiam meam, et portæ ínferi non prævalébunt advérsus eam. Et tibi dabo claves regni cœlórum. Et quodcúmque ligáveris super terram, erit ligátum et in cœlis: et quodcúmque sólveris super terram, erit solútum et in cœlis.
}\switchcolumn\portugues{
\blettrine{N}{aquele} tempo, veio Jesus para os lados de Cesareia, de Filipe, e interrogou os discípulos, dizendo: «Quem dizem os homens que é o Filho do homem?». Eles responderam: «Uns dizem que é João Baptista; outros que é Elias; outros que é Jeremias ou algum dos Profetas». Disse-lhes Jesus: «Vós, porém, quem dizeis que Eu sou?». Então, Simão-Pedro, tomando a palavra, disse: «Vós sois o Cristo, Filho de Deus vivo!». Respondendo Jesus, disse-lhe: «Bem-aventurado és tu, Simão, filho de João, pois não foi a carne nem o sangue que te revelaram isso, mas meu Pai, que está nos céus. Pois Eu te digo que tu és Pedro e que sobre esta pedra edificarei a minha Igreja; e as portas do inferno não prevalecerão contra ela. Dar-te-ei as chaves do reino dos céus: tudo o que ligares na terra será ligado nos céus; e tudo o que desligares na terra será desligado nos céus».
}\end{paracol}

\paragraphinfo{Ofertório}{Mt. 16, 18-19}
\begin{paracol}{2}\latim{
\rlettrine{T}{u} es Petrus, et super hanc petram ædificábo Ecclésiam meam: et portæ inferi non prævalébunt advérsus eam: et tibi dabo claves regni cœlórum.
}\switchcolumn\portugues{
\rlettrine{T}{u} és Pedro, e sobre esta pedra edificarei a minha Igreja. E as portas do inferno não prevalecerão contra ela. Dar-te-ei as chaves do reino dos céus.
}\end{paracol}

\paragraph{Secreta}
\begin{paracol}{2}\latim{
\rlettrine{E}{cclésiæ} tuæ, quǽsumus, Dómine, preces et hóstias beáti Petri Apóstoli comméndet orátio: ut, quod pro illíus glória celebrámus, nobis prosit ad véniam. Per Dóminum \emph{\&c.}
}\switchcolumn\portugues{
\rlettrine{V}{os} pedimos, Senhor, que o sufrágio do B. Apóstolo Pedro Vos torne agradáveis as preces e as hóstias da vossa Igreja, para que aquilo que celebramos em sua glória nos alcance o vosso perdão. Por nosso Senhor \emph{\&c.}
}\end{paracol}

\paragraphinfo{Secreta}{S. Paulo}
\begin{paracol}{2}\latim{
\rlettrine{A}{póstoli} tui Pauli précibus, Dómine, plebis tuæ dona sanctífica: ut, quæ tibi tuo grata sunt institúto, gratióra fiant patrocínio supplicántis. (Per Dóminum \emph{\&c.})
}\switchcolumn\portugues{
\rlettrine{S}{antificai,} Senhor, as ofertas do vosso povo pelas preces do vosso Ap.º Paulo, a fim de que, sendo-vos já agradáveis (porque por Vós foram instituídas), mais agradáveis ainda se tornem pelo patrocínio do suplicante. (Por nosso Senhor \emph{\&c.})
}\end{paracol}

\paragraphinfo{Secreta}{Santa Prisca}
\begin{paracol}{2}\latim{
\rlettrine{H}{æc} hóstia, quǽsumus, Dómine, quam Sanctórum tuórum natalítia recenséntes offérimus, et víncula nostræ pravitátis absólvat, et tuæ nobis misericórdiæ dona concíliet. Per Dóminum \emph{\&c.}
}\switchcolumn\portugues{
\qlettrine{Q}{ue} esta hóstia, Senhor, que Vos oferecemos em honra do nascimento dos vossos Santos, nos livre dos vínculos dos nossos pecados e nos obtenha os dons da vossa misericórdia. Por nosso Senhor \emph{\&c.}
}\end{paracol}

\paragraphinfo{Comúnio}{Mt. 16, 18}
\begin{paracol}{2}\latim{
\rlettrine{T}{u} es Petrus, et super hanc petram ædificábo Ecclésiam meam.
}\switchcolumn\portugues{
\rlettrine{T}{u} és Pedro e sobre esta pedra edificarei a minha Igreja.
}\end{paracol}

\paragraph{Postcomúnio}
\begin{paracol}{2}\latim{
\rlettrine{L}{ætíficet} nos, Dómine, munus oblátum: ut, sicut in Apóstolo tuo Petro te mirábilem prædicámus; sic per illum tuæ sumámus indulgéntiæ largitátem. Per Dóminum nostrum \emph{\&c.}
}\switchcolumn\portugues{
\rlettrine{A}{legre-nos,} Senhor, este sacrifício que Vos oferecemos, a fim de que, assim como Vos proclamamos admirável no vosso Apóstolo Pedro, assim também por ele recebamos uma abundante efusão da vossa misericórdia. Por nosso Senhor \emph{\&c.}
}\end{paracol}

\paragraphinfo{Postcomúnio}{S. Paulo}
\begin{paracol}{2}\latim{
\rlettrine{S}{anctificáti,} Dómine, salutári mystério: quǽsumus; ut nobis ejus non desit orátio, cujus nos donásti patrocínio gubernari. (Per Dóminum nostrum \emph{\&c.})
}\switchcolumn\portugues{
\rlettrine{H}{avendo} nós sido santificados com este salutar mystério, Vos suplicamos, Senhor, que não cesse de interceder Por nós aquele a cujo amparo nos confiastes. (Por nosso Senhor \emph{\&c.})
}\end{paracol}

\paragraphinfo{Postcomúnio}{Santa Prisca}
\begin{paracol}{2}\latim{
\qlettrine{Q}{uǽsumus,} Dómine, salutáribus repléti mystériis: ut, cujus sollémnia celebrámus, ejus oratiónibus adjuvémur. Per Dóminum \emph{\&c.}
}\switchcolumn\portugues{
\rlettrine{S}{aciados} com os dons salutares Vos imploramos, Senhor, sejamos socorridos pelas preces daquela cuja festa celebrámos. Por nosso Senhor \emph{\&c.}
}\end{paracol}
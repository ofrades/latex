\subsectioninfo{Maternidade da B. V. Maria}{11 de Outubro}

\paragraphinfo{Intróito}{Is. 7, 14}
\begin{paracol}{2}\latim{
\rlettrine{E}{cce} Virgo concípiet, et páriet fílium, et vocábitur nomen ejus Emmánuel. \emph{Ps. 97, 1} Cantáte Dómino cánticum novum: quia mirabília fecit.
℣. Gloria Patri \emph{\&c.}
}\switchcolumn\portugues{
\rlettrine{E}{is} que a Virgem conceberá e dará à luz um Filho, cujo nome será Emanuel. \emph{Sl. 97, 1} Cantai ao Senhor um cântico novo, porque Ele praticou maravilhas.
℣. Glória ao Pai \emph{\&c.}
}\end{paracol}

\paragraph{Oração}
\begin{paracol}{2}\latim{
\rlettrine{D}{eus,} qui de beátæ Maríæ Vírginis útero Verbum tuum, Angelo nuntiánte, carnem suscípere voluísti: præsta supplícibus tuis; ut, qui vere eam Genitrícem Dei crédimus, ejus apud te intercessiónibus adjuvémur. Per eumdem Dóminum \emph{\&c.}
}\switchcolumn\portugues{
\slettrine{Ó}{} Deus, que segundo a anunciação do Anjo quisestes que o vosso Verbo assumisse a carne humana no seio da B. V. Maria, concedei-nos, Vos suplicamos, que, assim como acreditamos que ela é verdadeira Mãe de Deus, assim sejamos auxiliados na vossa presença com a intercessão das suas preces. Pelo mesmo nosso Senhor \emph{\&c.}
}\end{paracol}

\paragraphinfo{Epístola}{Página \pageref{montecarmelo}}

\paragraphinfo{Gradual}{Is. 11, 1-2}
\begin{paracol}{2}\latim{
\rlettrine{E}{grediétur} virga de rádice Jesse, et flos de rádice ejus ascéndet. ℣. Et requiéscet super eum Spíritus Dómini.
}\switchcolumn\portugues{
\rlettrine{S}{airá} uma vara do tronco de Jessé e uma flor brotará da sua raiz. ℣. E o espírito do Senhor repousará sobre ela.
}\switchcolumn*\latim{
Allelúja, allelúja. ℣. Virgo Dei Génitrix, quem totus non capit orbis, in tua se clausit víscera factus homo. Allelúja.
}\switchcolumn\portugues{
Aleluia, aleluia. ℣. Ó Virgem, Aquele que todo o mundo não é capaz de conter, quando se fez homem, esteve encerrado no vosso seio. Aleluia.
}\end{paracol}

\paragraphinfo{Evangelho}{Lc. 2, 43-51}
\begin{paracol}{2}\latim{
\cruz Sequéntia sancti Evangélii secúndum Lucam.
}\switchcolumn\portugues{
\cruz Continuação do santo Evangelho segundo S. Lucas.
}\switchcolumn*\latim{
\blettrine{I}{n} illo témpore: Cum redírent, remánsit puer Jesus in Jerúsalem, et non cognovérunt paréntes ejus. Existimántes autem illum esse in comitátu, venérunt iter diei, et requirébant eum inter cognátos, et notos. Et non inveniéntes, regréssi sunt in Jerúsalem, requiréntes eum. Et factum est, post tríduum invenérunt illum in templo sedéntem in médio doctórum, audiéntem illos, et interrogántem eos. Stupébant autem omnes, qui eum audiébant, super prudéntia et respónsis ejus. Et vidéntes admiráti sunt. Et dixit mater ejus ad illum: Fili, quid fecísti nobis sic? ecce pater tuus, et ego doléntes quærebámus te. Et ait ad illos: Quid est quod me quærebátis? nesciebátis quia in his, quæ Patris mei sunt, opórtet me esse. Et ipsi non intellexérunt verbum, quod locútus est ad eos. Et descéndit cum eis, et venit Názareth: et erat súbditus illis.
}\switchcolumn\portugues{
\blettrine{N}{aquele} tempo, quando voltaram para casa, ficou o Menino Jesus em Jerusalém, sem que de tal seus pais se apercebessem. Pensando que Ele viria com seus companheiros de jornada, fizeram um dia de viagem, procurando-O depois entre os parentes e conhecidos. Não O encontrando, voltaram logo a Jerusalém pelo mesmo caminho. Então aconteceu que, depois de três dias, foram achá-l’O no templo, assentado entre os doutores, ouvindo-os e interrogando-os. E aqueles que O ouviam estavam admirados da sua sabedoria e das suas respostas. Quando os pais O encontraram, ficaram admirados, dizendo-Lhe logo a Mãe: «Meu Filho, porque procedestes assim para connosco? Eis que vosso Pai e eu Vos buscávamos aflitos?!». Ele disse-lhes: «Porque me procuráveis? Não sabeis que é preciso que me ocupe das cousas de meu Pai?». Porém, eles não compreenderam o que Jesus lhes disse. Então desceu com eles, veio para Nazaré e era-lhes obediente.
}\end{paracol}

\paragraphinfo{Ofertório}{Mt. 1, 18}
\begin{paracol}{2}\latim{
\rlettrine{C}{um} esset desponsáta mater ejus María Joseph, invénta est in útero habens de Spíritu Sancto.
}\switchcolumn\portugues{
\rlettrine{E}{stando} Maria, sua Mãe, desposada com José, achou este que ela havia concebido do Espírito Santo.
}\end{paracol}

\paragraph{Secreta}
\begin{paracol}{2}\latim{
\rlettrine{T}{ua,} Dómine, propitiatióne, et beátæ Maríæ semper Vírginis, Unigéniti tui matris intercessióne, ad perpétuam atque præséntem hæc oblátio nobis profíciat prosperitátem, et pacem. Per eumdem Dóminum \emph{\&c.}
}\switchcolumn\portugues{
\rlettrine{P}{ela} vossa misericórdia, Senhor, e pela intercessão da B. Maria, sempre Virgem, fazei que esta oferta nos assegure agora e sempre a prosperidade e a paz. Por nosso Senhor \emph{\&c.}
}\end{paracol}

\paragraph{Comúnio}
\begin{paracol}{2}\latim{
\rlettrine{B}{eáta} víscera Maríæ Vírginis, quæ portavérunt ætérni Patris Fílium.
}\switchcolumn\portugues{
\rlettrine{B}{em-aventuradas} as entranhas da Virgem Maria, que trouxeram encerrado o Filho do Pai Eterno.
}\end{paracol}

\paragraph{Postcomúnio}
\begin{paracol}{2}\latim{
\rlettrine{H}{æc} nos commúnio, Dómine, purget a crímine: et, intercedénte beáta Vírgine Dei Genitríce María, cœléstis remédii fáciat esse consórtes. Per eumdem \emph{\&c.}
}\switchcolumn\portugues{
\qlettrine{Q}{ue} esta comunhão, Senhor, nos purifique de nossos crimes e que por intercessão da B. V. Maria, Mãe de Deus, nos torne participantes do remédio celestial. Pelo mesmo nosso Senhor \emph{\&c.}
}\end{paracol}

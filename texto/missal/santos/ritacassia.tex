\subsectioninfo{S. Rita de Cássia, Viúva}{22 de Maio}

\paragraphinfo{Intróito}{Sl. 117, 19 \& 22}
\begin{paracol}{2}\latim{
\rlettrine{A}{períte} mihi portas justítiæ, ingréssus in eas confitébor Dómino: lápidem, quem reprobavérunt ædificántes, hic factus est in caput ánguli. (T.P. Allelúja, Allelúja.) \emph{Ps. 117, 1} Confitémini Dómino, quóniam bonus, quóniam in sǽculum misericórdia ejus.
℣. Gloria Patri \emph{\&c.}
}\switchcolumn\portugues{
\rlettrine{A}{bri-me} as portas da justiça, e, ao entrar por elas, louvarei o Senhor. A pedra que os edificadores rejeitaram tornou-se na pedra angular. (T.P. Aleluia, aleluia.) \emph{Sl. 117, 1} Louvai o Senhor, porque Ele é bom; a sua misericórdia permanecerá em todos os séculos dos séculos.
℣. Glória ao Pai \emph{\&c.}
}\end{paracol}

\paragraph{Oração}
\begin{paracol}{2}\latim{
\rlettrine{D}{eus,} qui sanctæ Ritæ tantam grátiam conférre dignátus es, ut inimícos dilígeret et in corde ac fronte caritátis et passiónis tuæ signa portáret: da nobis, quǽsumus, ejus intercessióne et méritis; inimícis nostris sic párcere et passiónis tuæ dolóres contemplári, ut promíssa mítibus ac lugéntibus prǽmia consequámur: Qui vivis et regnas \emph{\&c.}
}\switchcolumn\portugues{
\slettrine{Ó}{} Deus, que Vos dignastes conceder a Santa Rita abundante graça para amar os inimigos e trazer no coração e na fronte os sinais da vossa caridade e Paixão, concedei-nos por sua intercessão e méritos que, perdoando nós aos nossos inimigos e contemplando os sofrimentos da vossa Paixão, consigamos alcançar os prémios prometidos aos que são mansos e misericordiosos de coração. Ó Vós, que viveis e reinais \emph{\&c.}
}\end{paracol}

\paragraphinfo{Epístola}{Pr. 2}
\begin{paracol}{2}\latim{
Lectio libri Sapiéntiæ.
}\switchcolumn\portugues{
Lição do Livro da Sabedoria.
}\switchcolumn*\latim{
\rlettrine{E}{go} flos campi, et lílium conválium. Sicut lílium inter spinas, sic amíca mea inter fílias. Sicut malus inter ligna silvárum, sic diléctus meus inter fílios. Sub umbra illíus, quem desideráveram, sedi: et fructus ejus dulcis gútturi meo. Introdúxit me in cellam vináriam, ordinávit in me caritátem. Fulcíte me flóribus, stipáte me malis: quia amóre lángueo. Læva ejus sub cápite meo, et déxtera illíus amplexábitur me. Adjúro vos, fíliæ Jerúsalem, per cápreas, cervósque campórum, ne suscitétis, neque evigiláre faciátis diléctam, quoadúsque ipsa velit. Vox dilécti mei, ecce iste venit sáliens in móntibus, tansíliens colles. Símilis est diléctus meus cápreæ, hinulóque cervórum. En ipse stat post paríetem nostrum, respíciens per fenéstras, prospíciens per cancéllos. Et diléctus meus lóquitur mihi: Surge, própera, amíca mea, colúmba mea, formóisa mea, et veni. Jam enim hiems tránsiit, imber ábiit, et recéssit. Flores apparuérunt in terra nostra: tempus putatiónis advénit: vox túrturis audíta est in terra nostra: ficus prótulit grossos suos: víneæ floréntes dedérunt odórem suum. Surge, amíca mea, speciósa mea, et veni.
}\switchcolumn\portugues{
\rlettrine{S}{ou} a flor do campo e a açucena dos vales. Como a açucena entre os espinhos, assim é a minha amiga entre as donzelas. Como a macieira entre as árvores dos bosques, assim o meu amado entre os bosques. Assentei-me à sombra daquele a quem tanto desejara, sendo o seu fruto doce ao meu paladar. Introduziu-me na dispensa do vinho e ordenou em mim a caridade. Confortai-me com flores e fortalecei-me com frutos, pois desfaleço de amor! Sua mão esquerda está debaixo de minha cabeça e a sua mão direita abraça-me. Conjuro-vos, ó filhas de Jerusalém, pelas gazelas e veados do campo, que não perturbeis nem acordeis a minha amada até que ela o queira. Ouço a voz do meu amado. Eis que ele vem, galgando os montes e transpondo os outeiros! Meu amado é semelhante ao gamo e ao filho das corças. Eis que ele vem por detrás da nossa Parede, olhando pelas janelas e espreitando pelas frestas. E o meu amado diz-me: «Ergue-te, apressa-te e vem, ó minha amiga, ó minha pomba, ó minha beleza! Já o inverno acabou; já as chuvas Cessaram: as flores brotaram nos nossos jardins; já chegou o tempo da poda; ouve-se a rola nos nossos campos; a figueira mostra os primeiros frutos e as vinhas em flor exalam seus aromas! Ergue-te e vem, minha amiga, minha beleza!».
}\end{paracol}

\begin{paracol}{2}\latim{
Allelúja, allelúja. ℣. \emph{Eccli. 24, 18} Quasi palma exaltáta sum in Cades, et quasi plantátio rosæ in Jéricho. ℣. \emph{ibid., 20} Sicut cinnamómum et bálsamum aromatízans odórem dedi: quasi myrrha elécta dedi suavitátem odóris. Allelúja.
}\switchcolumn\portugues{
Aleluia, aleluia. ℣. \emph{Ecl. 24, 18} Fui exaltada como a palmeira em Cades e como os roseirais das roseiras de Jericó. Aleluia. ℣. \emph{ibid., 20} Espalhei perfumes, como o cinamomo e o bálsamo aromático: e como mirra escolhida exalei suave odor. Aleluia.
}\end{paracol}

\paragraphinfo{Evangelho}{Página \pageref{virgensmartires2}}

\paragraphinfo{Ofertório}{Gn. 40, 9-10}
\begin{paracol}{2}\latim{
\rlettrine{V}{idébam} coram me vitem, in qua erant tres propágines, créscere paulátim in gemmas, et post flores uvas maturéscera. (T. P. Allelúja.)
}\switchcolumn\portugues{
\rlettrine{D}{iante} de mim via uma cepa, na qual havia três varas, pouco a pouco a crescer em gomos: e, depois de as flores amadurecerem, as uvas. (T.P. Aleluia.)
}\end{paracol}

\paragraph{Secreta}
\begin{paracol}{2}\latim{
\rlettrine{C}{orda} nostra, quǽsumus, Dómine, Sanctæ Ritæ méritis, supérni dolóris spina confíge: ut, a peccátis ómnibus tua grátia liberáti, sacrificáre tibi hóstiam laudis pura mente valeámus. Per Dóminum \emph{\&c.}
}\switchcolumn\portugues{
\rlettrine{C}{ompungi,} Senhor, Vos suplicamos, pelos méritos de Santa Rita, os nossos corações com os espinhos duma dor sobrenatural, a fim de que pela vossa graça, livres de todo o pecado, possamos sacrificar-Vos com o coração puro a hóstia de louvor. Por nosso Senhor \emph{\&c.}
}\end{paracol}

\paragraphinfo{Comúnio}{Sl. 20, 4}
\begin{paracol}{2}\latim{
\rlettrine{P}{rævenísti} eam, Dómine, in benedictiónibus dulcédinis: posuísti in cápite ejus corónam de lápide pretióso. (T. P. Allelúja.)
}\switchcolumn\portugues{
\rlettrine{P}{remuniste-la,} Senhor, com bênçãos de doçura: Impusestes na sua cabeça uma coroa de pedras preciosas. (T.P. Aleluia.)
}\end{paracol}

\paragraph{Postcomúnio}
\begin{paracol}{2}\latim{
\rlettrine{C}{æléstibus,} Dómine, pasti delíciis, súpplices te rogámus: ut, intercedénte sancta Rita, caritátis et passiónis tuæ in méntibus nostris signa ferámus, et perpétuæ pacis fructu júgiter perfruámur. Per Dóminum \emph{\&c.}
}\switchcolumn\portugues{
\rlettrine{A}{pascentados,} ó Senhor, com as delícias celestiais, suplicantes, Vos pedimos a graça de, por intercessão de Santa Rita, trazermos em nossas mentes os sinais da vossa caridade e Paixão e gozarmos constantemente o fruto da perpétua paz. Por nosso Senhor \emph{\&c.}
}\end{paracol}

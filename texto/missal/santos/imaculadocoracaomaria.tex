\subsectioninfo{Imaculado Coração da B. V. Maria}{22 de Agosto}\label{imaculadocoracaomaria}

\paragraphinfo{Intróito}{Heb. 4, 16}
\begin{paracol}{2}\latim{
\rlettrine{A}{deámus} cum fidúcia ad thronum grátiæ, ut misericórdiam consequámur, et grátiam inveniámus in auxílio opportúno. \emph{Ps. 44, 2} Eructávit cor meum verbum bonum: dico ego ópera mea regi.
℣. Gloria Patri \emph{\&c.}
}\switchcolumn\portugues{
\rlettrine{A}{cerquemo-nos} com confiança do trono da graça, a fim de alcançar misericórdia e encontrar graça e sermos auxiliados na ocasião necessária. \emph{Sl. 44, 2} Meu coração exprimiu uma excelente palavra: «Consagro ao Rei as minhas obras»!
℣. Glória ao Pai \emph{\&c.}
}\end{paracol}

\paragraph{Oração}
\begin{paracol}{2}\latim{
\rlettrine{O}{mnípotens} sempitérne Deus, qui in Corde beátæ Maríæ Vírginis dignum Spíritus Sancti habitáculum præparásti: concéde propítius; ut ejúsdem immaculáti Cordis festivitátem devóta mente recoléntes, secúndum cor tuum vívere valeámus. Per Dóminum \emph{\&c.}
}\switchcolumn\portugues{
\rlettrine{O}{mnipotente} e sempiterno Deus, que no Coração da B. V. Maria preparastes uma digna morada do Espírito Santo, concedi-nos propício que, havendo celebrado devotamente a festa deste Imaculado Coração, vivamos sempre segundo o vosso Coração. Por nosso Senhor \emph{\&c.}
}\end{paracol}

\paragraphinfo{Epístola}{Página \pageref{montecarmelo}}

\paragraphinfo{Gradual}{Sl. 12, 6}
\begin{paracol}{2}\latim{
\rlettrine{E}{xsultábit} cor meum in salutári tuo: cantábo Dómino, qui bona tríbuit mihi: et psallam nómini Dómini altíssimi. ℣. \emph{Ps. 44, 18} Mémores erunt nóminis tui in omni generatióne et generatiónem: proptérea pópuli confitebúntur tibi in ætérnum.
}\switchcolumn\portugues{
\rlettrine{E}{xultará} o meu coração com a salvação que de Vós virá; cantarei hinos ao Senhor, que me concedeu tantos benefícios; entoarei salmos em louvor do nome do altíssimo Senhor. ℣. \emph{Sl. 44, 18} De geração em geração glorificarão o vosso nome: e os povos vos louvarão eternamente.
}\switchcolumn*\latim{
Allelúja, allelúja. ℣. \emph{Luc. 1, 46, 47} Magníficat ánima mea Dóminum: et exsultávit spíritus meus in Deo salutári meo. Allelúja.
}\switchcolumn\portugues{
Aleluia, aleluia. ℣. \emph{Lc. 1, 46, 47} Minha alma glorifica o Senhor e o meu espírito se alegra em Deus, meu Salvador. Aleluia.
}\end{paracol}

\paragraphinfo{Evangelho}{Jo. 19, 25-27}
\begin{paracol}{2}\latim{
\cruz Sequéntia sancti Evangélii secúndum Joánnem.
}\switchcolumn\portugues{
\cruz Continuação do santo Evangelho segundo S. João.
}\switchcolumn*\latim{
\blettrine{I}{n} illo témpore: Stabant juxta crucem Jesu mater ejus, et soror matris ejus María Cléophæ, et María Magdaléne. Cum vidísset ergo Jesus matrem, et discípulum stantem, quem diligébat, dicit matri suæ: Múlier, ecce fílius tuus. Deinde dicit discípulo: Ecce mater tua. Et ex illa hora accépit eam discípulus in sua.
}\switchcolumn\portugues{
\blettrine{N}{aquele} tempo, estavam, junto à Cruz de Jesus, sua Mãe, a irmã de sua Mãe, Maria, mulher de Cléofas, e Maria Madalena. Então, vendo Jesus sua Mãe, e de pé, perto dela, o discípulo que Ele preferia, disse a sua Mãe: «Mulher, eis o vosso filho». Em seguida disse ao discípulo: «Eis a tua Mãe!». E desde aquela hora o discípulo a levou consigo.
}\end{paracol}

\paragraphinfo{Ofertório}{Lc. 1, 46, 49}
\begin{paracol}{2}\latim{
\rlettrine{E}{xsultávit} spíritus meus in Deo salutári meo; quia fecit mihi magna qui potens est, et sanctum nomen ejus.
}\switchcolumn\portugues{
\rlettrine{O}{} meu espírito alegra-se em Deus, meu Salvador, porque Aquele que é omnipotente, e o seu nome é Santo, operou em mim maravilhas.
}\end{paracol}

\paragraph{Secreta}
\begin{paracol}{2}\latim{
\rlettrine{M}{ajestáti} tuæ, Dómine, Agnum immaculátum offeréntes, quǽsumus: ut corda nostra ignis ille divínus accéndat, cui Cor beátæ Maríæ Vírginis ineffabíliter inflammávit. Per eundem Dóminum \emph{\&c.}
}\switchcolumn\portugues{
\rlettrine{E}{nquanto} oferecemos à vossa Majestade, Senhor, o Cordeiro Imaculado, dignai-Vos acender nos nossos corações aquele fogo divino que abrasou de uma maneira inefável o Coração da B. V. Maria. Pelo mesmo nosso Senhor \emph{\&c.}
}\end{paracol}

\paragraphinfo{Comúnio}{Jo. 19, 27}
\begin{paracol}{2}\latim{
\rlettrine{D}{ixit} Jesus matri suæ: Múlier, ecce fílius tuus: deinde dixit discípulo: Ecce mater tua. Et ex illa hora accépit eam discípulus in sua.
}\switchcolumn\portugues{
\rlettrine{D}{isse} Jesus a sua Mãe: «Mulher, eis o vosso filho»; depois disse ao discípulo: «Eis a tua Mãe». E desde aquela hora o discípulo a levou consigo.
}\end{paracol}

\paragraph{Postcomúnio}
\begin{paracol}{2}\latim{
\rlettrine{D}{ivínis} refécti munéribus te, Dómine, supplíciter exorámus: ut beátæ Maríæ Vírginis intercessióne, cujus immaculáti Cordis solémnia venerándo égimus, a præséntibus perículis liberáti, ætérnæ vitæ gáudia consequámur. Per Dóminum \emph{\&c.}
}\switchcolumn\portugues{
\rlettrine{S}{ustentados} com vossas divinas ofertas, Senhor, humildemente Vos imploramos pela intercessão da B. V. Maria, cujo puríssimo Coração acabamos solenemente de honrar, que, sendo livres dos perigos presentes, possamos gozar a alegria da vida eterna. Por nosso Senhor \emph{\&c.}
}\end{paracol}

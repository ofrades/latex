\subsectioninfo{Preciosíssimo Sangue de N. S. J. C.}{1 de Julho}

\paragraphinfo{Intróito}{Apoc. 5, 9-10}
\begin{paracol}{2}\latim{
\rlettrine{R}{edemísti} nos,Dómine, in sánguine tuo, ex omni tribu et lingua et pópulo et natióne: et fecísti nos Deo nostro regnum. \emph{Ps. 88, 2} Misericórdias Dómini in ætérnum cantábo: in generatiónem et generatiónem annuntiábo veritátem tuam in ore meo.
℣. Gloria Patri \emph{\&c.}
}\switchcolumn\portugues{
\rlettrine{C}{om} o vosso Sangue, Senhor, nos resgatastes de todas as tribos, de todas as línguas, de todos os povos e de todas as nações; e fizestes de nós um reino para o nosso Deus. \emph{Sl. 88, 2} Cantarei eternamente as misericórdias do Senhor; de geração em geração a minha boca publicará a vossa verdade.
℣. Glória ao Pai \emph{\&c.}
}\end{paracol}

\paragraph{Oração}
\begin{paracol}{2}\latim{
\rlettrine{O}{mnípotens} sempitérne Deus, qui unigénitum Fílium tuum mundi Redemptórem constituísti, ac ejus Sánguine placári voluísti: concéde, quǽsumus, salútis nostræ prétium sollémni cultu ita venerári, atque a præséntis vitæ malis ejus virtúte deféndi in terris; ut fructu perpétuo lætémur in cœlis. Per eúndem Dóminum \emph{\&c.}
}\switchcolumn\portugues{
\slettrine{Ó}{} Deus omnipotente e eterno, que instituístes o vosso Filho Unigénito Redentor do mundo e quisestes ser aplacado com seu Sangue, concedei-nos a graça, Vos suplicamos, de honrarmos com culto solene este preço da nossa salvação e de, pela sua virtude, sermos preservados dos males da vida presente de modo a gozarmos eternamente os seus frutos nos céus. Pelo mesmo nosso Senhor \emph{\&c.}
}\end{paracol}

\paragraphinfo{Epístola}{Página \pageref{domingopaicao}}

\paragraphinfo{Gradual}{1. Jo. 5, 6 \& 7-8}
\begin{paracol}{2}\latim{
\rlettrine{H}{ic} est, qui venit per aquam et sánguinem, Jesus Christus: non in aqua solum, sed in aqua et sánguine. ℣. Tres sunt, qui testimónium dant in cœlo: Pater, Verbum et Spíritus Sanctus; et hi tres unum sunt. Et tres sunt, qui testimónium dant in terra: Spíritus, aqua et sanguis: et hi tres unum sunt.
}\switchcolumn\portugues{
\rlettrine{E}{ste} é Jesus Cristo, que veio pela água e pelo sangue: não pela água, somente, mas pela água e pelo sangue. ℣. Três são os que dão testemunho no céu: o Pai, o Verbo e o Espírito Santo, e estes três são um só. E três são os que dão testemunho na terra: o espírito, a água e o sangue, e estes três são um só!
}\switchcolumn*\latim{
Allelúja, allelúja. ℣. \emph{ibid., 9} Si testimónium hóminum accípimus, testimónium Dei majus est. Allelúja.
}\switchcolumn\portugues{
Aleluia, aleluia. ℣. \emph{ibid., 9} Se recebemos o testemunho dos homens, maior é o testemunho de Deus. Aleluia.
}\end{paracol}

\paragraphinfo{Evangelho}{Jo. 19. 30-35}
\begin{paracol}{2}\latim{
\cruz Sequéntia sancti Evangélii secúndum Joánnem.
}\switchcolumn\portugues{
\cruz Continuação do santo Evangelho segundo S. João.
}\switchcolumn*\latim{
\blettrine{I}{n} illo témpore: Cum accepísset Jesus acétum, dixit: Consummátum est. Et inclináto cápite trádidit spíritum. Judǽi ergo (quóniam Parascéve erat), ut non remanérent in cruce córpora sábbato (erat enim magnus dies ille sábbati), rogavérunt Pilátum, ut frangeréntur eórum crura et tolleréntur. Venérunt ergo mílites: et primi quidem fregérunt crura et altérius, qui crucifíxus est cum eo. Ad Jesum autem cum venissent, ut vidérunt eum jam mórtuum, non fregérunt ejus crura, sed unus mílitum láncea latus ejus apéruit, et contínuo exívit sanguis et aqua. Et qui vidit, testimónium perhíbuit; et verum est testimónium ejus.
}\switchcolumn\portugues{
\blettrine{N}{aquele} tempo, havendo Jesus bebido o vinagre, disse: «Tudo está consumado!». E, inclinando a cabeça, entregou o Espírito. Porém os judeus (como aquele dia era o da Preparação), para que os corpos não ficassem na cruz no sábado (pois aquele Sábado era dia solene), rogaram a Pilatos licença para lhes quebrarem as pernas e os tirarem. Vieram, então, os soldados e quebraram as pernas ao primeiro e depois ao outro que haviam sido crucificados com Ele. Quando, porém, chegaram a Jesus, vendo que já estava morto, Lhe não quebraram as pernas, mas um dos soldados abriu-Lhe o lado com uma lança, saindo dali imediatamente sangue e água. E aquele que viu isto dá testemunho, sendo verdadeiro o seu testemunho.
}\end{paracol}

\paragraphinfo{Ofertório}{1. Cor. 10. 16}
\begin{paracol}{2}\latim{
\rlettrine{C}{alix} benedictiónis, cui benedícimus, nonne communicátio sánguinis Christi est? et panis, quem frángimus, nonne participátio córporis Dómini est?
}\switchcolumn\portugues{
\rlettrine{O}{} cálice de bênção, que nós benzemos, não é, porventura, a comunhão do Sangue de Cristo? E o pão, que nós partimos, não é a participação do Corpo do Senhor?
}\end{paracol}

\paragraph{Secreta}
\begin{paracol}{2}\latim{
\rlettrine{P}{er} hæc divína mystéria, ad novi, quǽsumus, Te
staménti mediatórem Jesum accedámus: et super altária tua, Dómine virtútum, aspersiónem sánguinis mélius loquéntem, quam Abel, innovémus. Per eúndem Dóminum \emph{\&c.}
}\switchcolumn\portugues{
\rlettrine{D}{ignai-Vos} permitir, Senhor, que por estes divinos mistérios nos aproximemos de Jesus o mediador do Novo Testamento e que renovemos a efusão do seu Sangue sobre os vossos altares, ó Senhor dos exércitos, o qual é mais eloquente do que o sangue de Abel. Pelo mesmo nosso Senhor \emph{\&c.}
}\end{paracol}

\paragraphinfo{Comúnio}{Heb. 9, 28}
\begin{paracol}{2}\latim{
\rlettrine{C}{hristus} semel oblítus est ad multórum exhauriénda peccáta: secúndo sine peccáto apparébit exspectántibus se in salútem.
}\switchcolumn\portugues{
\rlettrine{C}{risto} ofereceu-se uma vez para apagar os pecados de muitos; em uma segunda vez Ele aparecerá não já para expiar os pecados, mas para salvar aqueles que O esperam.
}\end{paracol}

\paragraph{Postcomúnio}
\begin{paracol}{2}\latim{
\rlettrine{A}{d} sacram, Dómine, mensam admíssi, háusimus aquas in gáudio de fóntibus Salvatóris: sanguis ejus fiat nobis, quǽsumus, fons aquæ in vitam ætérnam saliéntis: Qui tecum vivit \emph{\&c.}
}\switchcolumn\portugues{
\rlettrine{H}{avendo} sido admitidos à sagrada mesa, Senhor, bebemos com alegria as águas nas fontes do Salvador; e, Vos suplicamos, permiti que seu Sangue se torne para nós numa fonte de água viva, brotando até à vida eterna. Ele, que, sendo Deus, vive e reina \emph{\&c.}
}\end{paracol}

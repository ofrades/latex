\subsectioninfo{Santa Marta, Virgem}{29 de Julho}

\textit{Como na Missa Dilexísti justitiam, página \pageref{virgemnaomartir1}, excepto:}

\paragraphinfo{Evangelho}{Lc. 10, 38-42}
\begin{paracol}{2}\latim{
\cruz Sequéntia sancti Evangélii secúndum Lucam.
}\switchcolumn\portugues{
\cruz Continuação do santo Evangelho segundo S. Lucas.
}\switchcolumn*\latim{
\blettrine{I}{n} illo témpore: Intrávit Jesus in quoddam castéllum: et múlier quædam, Martha nómine, excépit illum in domum suam: et huic erat soror nómine María, quæ étiam sedens secus pedes Dómini, audiébat verbum illíus. Martha autem satagébat circa frequens ministérium: quæ stetit et ait: Dómine, non est tibi curæ, quod soror mea réliquit me solam ministráre? dic ergo illi, ut me ádjuvet. Et respóndens, dixit illi Dóminus: Martha, Martha, sollícita es et turbáris erga plúrima: porro unum est necessárium. María óptimam partem elégit, quæ non auferétur ab ea.
}\switchcolumn\portugues{
\blettrine{N}{aquele} tempo, entrou Jesus em um castelo, onde uma mulher, chamada Marta, O recebeu em sua casa. Tinha esta uma irmã, de nome Maria, que se assentou aos pés do Senhor, escutando suas palavras. Porém, Marta estava muito atarefada, a preparar quanto era necessário. Então esta veio ter com Jesus, dizendo-Lhe: «Senhor, não reparais que minha irmã me deixa só a servir? Dizei-lhe, pois, que venha ajudar-me». E o Senhor disse: «Marta, Marta, inquietai-vos e embaraçai-vos, cuidando solícitamente de muitas coisas, quando na verdade só uma é necessária. Maria escolheu a melhor parte, a qual lhe não será tirada».
}\end{paracol}

\subsubsection{Comemoração dos S. S. Mártires}

\paragraph{Oração}
\begin{paracol}{2}\latim{
\rlettrine{P}{ræsta,} quǽsumus, Dómine: ut, sicut pópulus christiánus Mártyrum tuórum Felícis, Simplícii, Faustíni et Beatrícis temporáli sollemnitáte congáudet, ita perfruátur ætérna; et, quod votis célebrat, comprehéndat efféctu. Per Dóminum \emph{\&c.}
}\switchcolumn\portugues{
\rlettrine{P}{ermiti,} Senhor, Vos suplicamos, que, assim como o povo cristão celebra com júbilo durante esta vida a festa dos vossos Santos Mártires Félix, Simplício, Faustino e Beatriz, assim também possa alegrar-se na eternidade, e alcance na realidade mais tarde o que agora honra com seus votos. Por nosso Senhor \emph{\&c.}
}\end{paracol}

\paragraph{Secreta}
\begin{paracol}{2}\latim{
\rlettrine{H}{óstias} tibi, Dómine, pro sanctórum Mártyrum tuórum Felícis, Simplícii, Faustíni et Beatrícis commemoratióne deférimus: supplíciter deprecántes; ut indulgéntiam nobis páriter cónferant et salútem. Per Dóminum nostrum \emph{\&c.}
}\switchcolumn\portugues{
\rlettrine{V}{os} oferecemos, Senhor, estas hóstias em memória dos vossos Santos Mártires Félix. Simplício, Faustino e Beatriz, suplicando-Vos humildemente que por eles nos concedais simultaneamente o perdão e a salvação. Por nosso Senhor \emph{\&c.}
}\end{paracol}

\paragraph{Postcomúnio}
\begin{paracol}{2}\latim{
\rlettrine{P}{ræsta,} quǽsumus, omnípotens Deus: ut sanctórum Martyrum tuórum Felícis, Simplícii, Faustíni et Beatrícis cœléstibus mýsteriis celebráta sollémnitas, indulgéntiam nobis tuæ propitiatiónis acquírat. Per Dóminum \emph{\&c.}
}\switchcolumn\portugues{
\slettrine{Ó}{} Deus omnipotente, Vos suplicamos, fazei que a solenidade dos vossos Santos Mártires Félix, Simplício, Faustino e Beatriz, que celebramos com estes celestiais mistérios, nos obtenha o perdão da vossa misericórdia. Por nosso Senhor \emph{\&c.}
}\end{paracol}

\subsectioninfo{B. Virgem Maria, Rainha}{31 de Maio}

\paragraph{Intróito}
\begin{paracol}{2}\latim{
\rlettrine{G}{audeámus} omnes in Dómino, diem festum celebrántes sub honóre beátæ Maríæ Vírginis Reginæ: de cujus solemnitáte gaudent Angeli, et colláudant Fílium Dei. (T. P. Allelúja, allelúja.) \emph{Ps. 44, 2} Effúndit cor meum verbum bonum: dico ego carmen meum Regi.
℣. Gloria Patri \emph{\&c.}
}\switchcolumn\portugues{
\rlettrine{A}{legremo-nos} todos no Senhor, no dia em que celebramos a festa em honra da B. Virgem Maria, rainha: os Anjos regozijam-se com esta festa e louvam unisonamente o Filho de Deus. (T. P. Aleluia, aleluia.) \emph{Sl. 44, 2} No meu coração reboa um belo discurso: Ao Rei dedico este meu cântico.
℣. Glória ao Pai \emph{\&c.}
}\end{paracol}

\paragraph{Oração}
\begin{paracol}{2}\latim{
\rlettrine{C}{oncéde} nobis, quǽsumus, Dómine: ut, qui solemnitátem beátæ Maríæ Vírginis Regínæ nostræ celebrámus; ejus múniti præsídio, pacem in præsénti et glóriam in futuro consequi mereámur. Per Dóminum \emph{\&c.}
}\switchcolumn\portugues{
\rlettrine{C}{oncedei-nos,} Senhor, Vos suplicamos, que, assim como celebramos a festa da B. Virgem Maria, nossa Rainha, assim também, munidos com seu auxílio, mereçamos alcançar a paz no presente e a glória no futuro. Por nosso Senhor Jesus Cristo \emph{\&c.}
}\end{paracol}

\paragraphinfo{Epístola}{Ecl. 24: 5; 7; 9-11, 30-31}
\begin{paracol}{2}\latim{
Léctio libri Sapiéntiæ.
}\switchcolumn\portugues{
Lição do Livro da Sabedoria.
}\switchcolumn*\latim{
\rlettrine{E}{go} ex ore Altíssimi prodívi, primogenitá ante ómnem creatúram; ego in altíssimis habitávi, et thronus meus in colúmna nubis. In omni terra steti et in omni pópulo, et in omni gente primátum hábui, et ómnium excelléntium et humílium corda virtúte calcávi. Qui audit me, non confúndetur; et qui operántur in me, non peccábunt; qui elucídant me, vitam ætérnam habébunt.
}\switchcolumn\portugues{
\rlettrine{S}{aí} dos lábios do Altíssimo, fui a primogénita antes de todas as criaturas. Habitei nos lugares mais elevados; o meu trono está erguido nas nuvens, sobre uma coluna. Percorri toda a terra e todos os povos, e em todas as nações tenho a primazia e sujeitei com meu poder os corações de todos, desde os mais nobres até aos mais humildes. Quem me ouvir, não será confundido; quem por Mim for orientado, não pecará; os que me tornarem conhecida, possuirão a vida eterna.
}\end{paracol}

\begin{paracol}{2}\latim{
Allelúja, allelúja. ℣. Beáta es, Virgo María, quæ sub cruce Dómini sustinuísti. Allelúja. ℣. Nunc cum eo regnas in ætérnum. Allelúja.
}\switchcolumn\portugues{
Aleluia, aleluia. ℣. Bem-aventurada sois, ó Virgem Maria, que ficastes de pé, sob a Cruz do Senhor, aleluia. ℣. Agora, com Ele reinais eternamente. Aleluia.
}\end{paracol}

\textit{Fora do Tempo Pascal, omite-se este Verso e diz-se o}

\paragraphinfo{Gradual}{Ap. 19, 16}
\begin{paracol}{2}\latim{
\rlettrine{I}{pse} habet in vestiménto et in femóre suo scriptum: Rex regum, et Dóminus dominántium. ℣. \emph{Ps. 44, 10} Regina adstat ad déxteram ejus, ornáta auro ex Ophir.
}\switchcolumn\portugues{
\rlettrine{E}{le} tem escrito no manto e no seu femur, Rei dos reis e Senhor de todos os senhores. ℣. \emph{Sl. 44, 10} A Rainha está à sua direita, recamada de ouro de Ofir.
}\switchcolumn*\latim{
Allelúja, allelúja. ℣. Salve, Regína misericórdiæ: tu nos ab hoste prótege, et mortis hora súscipe. Allelúja.
}\switchcolumn\portugues{
Aleluia, aleluia. ℣. Salve, Rainha de Misericórdia, protegei-nos contra o inimigo e recebei-nos na hora da morte. Aleluia.
}\end{paracol}

\paragraphinfo{Evangelho}{Lc. 1, 26-33}
\begin{paracol}{2}\latim{
\cruz Sequéntia sancti Evangélii secúndum Lucam.
}\switchcolumn\portugues{
\cruz Continuação do santo Evangelho segundo S. Lucas.
}\switchcolumn*\latim{
\blettrine{I}{n} illo témpore: Missus est Angelus Gábriel a Deo in civitátem Galilææ, cui nomen Nazareth, ad Vírginem desponsátam viro cui nomen Joseph, de domo David, et nomen Vírginis María. Et ingréssus Angelus ad eam, dixit: Ave, grátia plena: Dóminus tecum: benedícta tu in muliéribus. Quæ cum audísset, turbáta est in sermóne ejus; et cogitábat, qualis esset ista salutátio. Et ait Angelus ei: Ne timeas, María: invenísti enim grátiam apud Deum: ecce concípies in utero, et paries fílium, et vocábis nomen ejus Jesum. Hic erit magnus, et Fílius Altíssimi vocábitur, et dabit illi Dóminus Deus sedem David patris ejus: et regnábit in domo Jacob in ætérnum, et regni ejus non erit finis.
}\switchcolumn\portugues{
\blettrine{N}{aquele} tempo, foi mandado por Deus o Anjo Gabriel a uma cidade da Galileia, chamada Nazaré, a uma Virgem, desposada com um varão, cujo nome era José, da casa de David; e o nome da Virgem era Maria. Entrando o Anjo, onde ela estava, disse: «Eu te saúdo, cheia de graça: o Senhor é contigo: bendita és tu entre todas as mulheres». Ouvindo ela isto, perturbou-se; e pensava na significação desta saudação. Então, disse-lhe o Anjo: «Não temas, Maria, porquanto alcançaste graça diante do Senhor: eis que conceberás no teu seio, e darás à luz um Filho; e o seu Nome será Jesus. Ele será grande e será chamado Filho do Altíssimo; o Senhor Deus lhe dará o trono de David, seu pai; reinará eternamente na casa de Jacob; e o seu reino não terá fim.
}\end{paracol}

\paragraph{Ofertório}
\begin{paracol}{2}\latim{
\rlettrine{R}{egáli} ex progénie María exórta refúlget: cujus précibus nos adjuvári, mente et spíritu devotíssime póscimus. (T.P. Allelúja.)
}\switchcolumn\portugues{
\rlettrine{O}{riunda} de linhagem real, refulge Maria; auxiliem-nos suas preces, como piedosíssimamente suplicamos com inteligência e entusiasmo. (T. P. Aleluia).
}\end{paracol}

\paragraph{Secreta}
\begin{paracol}{2}\latim{
\rlettrine{A}{ccipe,} quǽsumus, Dómine, múnera lætántis Ecclésiæ, et beátæ Vírginis Maríæ Regínæ suffragántibus méritis, ad nostræ salútis auxílium proveníre concéde. Per Dóminum \emph{\&c.}
}\switchcolumn\portugues{
\rlettrine{A}{ceitai,} Senhor, Vos suplicamos, os dons da Igreja em júbilo, e, pelos preclaros votos da B. Virgem Maria, Rainha, concedei-nos que eles sejam proveitoso auxílio para a nossa salvação. Por nosso Senhor \emph{\&c.}
}\end{paracol}

\paragraph{Comúnio}
\begin{paracol}{2}\latim{
\rlettrine{R}{egína} mundi digníssima, María Virgo perpétua, intercéde pro nostra pace et salúte, quæ genuísti Christum Dóminum, Salvatórem ómnium. (T.P. Allelúja.)
}\switchcolumn\portugues{
\slettrine{Ó}{} digníssima Rainha do mundo, perpétua Virgem Maria, que gerastes Cristo, Senhor, Salvador de todos, intercedei pela nossa paz e salvação. (T. P. Aleluia.)
}\end{paracol}

\paragraph{Postcomúnio}
\begin{paracol}{2}\latim{
\rlettrine{C}{elebrátis} solémniis, Dómine, quæ pro sanctæ Maríæ, Regínæ nostræ, festivitáte perégimus: ejus, quǽsumus, nobis intercessióne fiant salutária; in cujus honóre sunt exsultánter impléta. Per Dóminum \emph{\&c.}
}\switchcolumn\portugues{
\rlettrine{C}{elebradas} estas solenidades, Senhor, em homenagem a Santa Maria, nossa Rainha, permiti, Vos suplicamos, que pela sua intercessão nos sejam salutares, visto que com exultação em sua honra foram realizadas. Por nosso Senhor \emph{\&c.}
}\end{paracol}

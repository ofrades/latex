\subsectioninfo{S. Pedro Celestino, Papa e Conf.}{19 de Maio}

\textit{Como na Missa Si díligis me, página \pageref{sumospontifices}, excepto:}

\paragraph{Oração}
\begin{paracol}{2}\latim{
\rlettrine{D}{eus,} qui beátum Petrum Cœlestínum ad summi pontificátus ápicem sublimásti, quique illum humilitáti postpónere docuísti: concéde propítius; ut ejus exémplo cuncta mundi despícere, et ad promíssa humílibus prǽmia perveníre felíciter mereámur. Per Dóminum \emph{\&c.}
}\switchcolumn\portugues{
\slettrine{Ó}{} Deus, que elevastes o B. Pedro Celestino à eminente dignidade de sumo pontífice, ensinando-o ao mesmo tempo a preferir a humildade, concedei-nos propício que, imitando o seu exemplo, aprendamos a desprezar todos os bens deste mundo, para que com felicidade mereçamos alcançar os prémios que prometestes aos humildes. Por nosso Senhor \emph{\&c.}
}\end{paracol}

\subsubsection{Comemoração de Santa Pudenciana}

\paragraphinfo{Oração, Secreta e Postcomúnio}{Página \pageref{virgemnaomartir1}}

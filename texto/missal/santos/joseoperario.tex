\subsectioninfo{S. José, Operário}{1 de Maio}

\paragraphinfo{Intróito}{Sb. 10, 17}
\begin{paracol}{2}\latim{
\rlettrine{S}{apiéntia} réddidit justis mercédem labórum suórum, et dedúxit illos in via mirábili, et fuit illis in velaménto diéi, et in luce stellárum per noctem, allelúja, allelúja. \emph{Ps. 126, 1} Nisi Dóminus ædificáverit domum, in vanum labórant qui ædíficant eam.
℣. Gloria Patri \emph{\&c.}
}\switchcolumn\portugues{
\rlettrine{A}{} Sabedoria deu aos justos a recompensa nos seus trabalhos, conduziu-os por um caminho de prodígios e foi para eles sombra durante o dia e luz das estrelas durante a noite, aleluia, aleluia. \emph{Sl. 126, 1} Se o Senhor não edificar a casa, em vão trabalharão os que a constroem.
℣. Glória ao Pai \emph{\&c.}
}\end{paracol}

\paragraph{Oração}
\begin{paracol}{2}\latim{
\rlettrine{R}{erum} cónditor Deus qui legem labóris humáno géneri statuísti concéde propítius; ut sancti Joseph exémplo et patrocínio, ópera perficiámus quæ praécipis, et praémia consequámur quæ promíttis. Per Dóminum \emph{\&c.}
}\switchcolumn\portugues{
\slettrine{Ó}{} Deus, autor do mundo, que ao género humano prescrevestes a lei do trabalho concedei-nos propício que com o exemplo e o patrocínio de São José nos desempenhemos dos trabalhos, que nos ordenastes, e alcancemos os prémios, que nos prometestes. Por nosso Senhor \emph{\&c.}
}\end{paracol}

\paragraphinfo{Epístola}{Cl. 3, 14-15, 17, 23-24}
\begin{paracol}{2}\latim{
Lectio Epístolæ beati Pauli Apostoli ad Colossénses.
}\switchcolumn\portugues{
Lição da Ep.ª do B. Ap.º Paulo aos Colossenses.
}\switchcolumn*\latim{
\rlettrine{F}{ratres:} Caritátem habéte, quod est vínculum perfectiónis, et pax Christi exúltet in córdibus vestris, in qua et vocáti estis in uno córpore, et grati estóte. Omne quodcúmque fácitis in verbo aut in ópere, ómnia in nómine Dómini Jesu Christi, grátias agéntes Deo et Patri per ipsum. Quodcúmque fácitis ex ánimo operámini sicut Dómino, et non homínibus, sciéntes quod a Dómino accipiétis retributiónem hereditátis. Dómino Christo servíte.
}\switchcolumn\portugues{
\rlettrine{M}{eus} irmãos: Revesti-vos da caridade, que é o vínculo da perfeição; e que a paz de Cristo, à qual fostes chamados de modo a formar um só corpo, reine nos vossos corações; sede reconhecidos. Tudo quanto fizerdes, seja em palavras, seja em obras, fazei-o em nome do Senhor Jesus Cristo, rendendo por Ele acções de graças a Deus Pai. Tudo o que fizerdes, fazei-o com generosidade de coração, como se o fizésseis para o Senhor e não para os homens, ficando cientes de que recebereis do Senhor a herança celestial. Servi o Senhor Jesus Cristo.
}\end{paracol}

\begin{paracol}{2}\latim{
Allelúja, allelúja. ℣. \emph{Ps. 36} De quacúmque tribulatióne clamáverint ad me, exáudiam eos, et ero protéctor eórum semper. Allelúja. ℣. Fac nos innócuam, Joseph, decúrrere vitam: sitque tuo semper tuta patrocínio. Allelúja.
}\switchcolumn\portugues{
Aleluia, aleluia. ℣. \emph{Sl. 36} Em qualquer tribulação em que se encontrem e a mim recorram, ouvi-los-ei, e serei sempre o seu protector. Aleluia. ℣. Alcançai-nos, ó José, que a nossa vida decorra sã, e que seja sempre livre de perigo pelo auxílio do vosso patrocínio. Aleluia.
}\end{paracol}

\textit{Fora do Tempo Pascal diz-se:}

\paragraphinfo{Gradual}{Sl. 127, 1-2}
\begin{paracol}{2}\latim{
\rlettrine{B}{eátus} quicúmque times Dóminum, qui ámbulas in viis eius. ℣. Labórem mánuum tuárum manducábis et bene tibi erit.
}\switchcolumn\portugues{
\rlettrine{B}{em-aventurado} tu, quem quer que sejas, que temes o Senhor e andas pelos seus caminhos. ℣. Comerás o trabalho das tuas mãos, o qual te será salutar.
}\switchcolumn*\latim{
Allelúja, allelúja. ℣. Fac nos innócuam, Joseph, decúrrere vitam: sitque tuo semper tuta patrocínio. allelúja.
}\switchcolumn\portugues{
Aleluia, aleluia. ℣. Alcançai-nos, ó José, que a nossa vida decorra sã, e que seja sempre livre de perigo pelo auxílio do vosso patrocínio. Aleluia.
}\end{paracol}

\textit{Depois de Septuagésima, omitem-se o Aleluia e o Gradual, e diz-se:}

\paragraphinfo{Trato}{Sl. 111, 1-3}
\begin{paracol}{2}\latim{
\rlettrine{B}{eátus} vir qui timet Dóminum, qui mandátis eius delectátur multum. ℣. Potens in terra erit semen eius; generatióni rectórum benedicétur. ℣. Opes et divitiæ erunt in domo eius, et munificéntia eius manébit semper.
}\switchcolumn\portugues{
\rlettrine{B}{em-aventurado} o varão que teme o Senhor e que se delicia com seus Mandamentos. Será poderosa na terra a sua descendência, pois a linhagem dos justos será abençoada. Em sua casa existirão haveres e riquezas e a sua liberalidade durará sempre.
}\end{paracol}

\paragraphinfo{Evangelho}{Mt. 13, 54-58}
\begin{paracol}{2}\latim{
\cruz Sequéntia sancti Evangélii secúndum Lucam.
}\switchcolumn\portugues{
\cruz Continuação do santo Evangelho segundo S. Mateus.
}\switchcolumn*\latim{
\blettrine{I}{n} illo témpore: Véniens Jesus in pátriam suam, docébat eos in synagógis eorum, ita ut miraréntur et dícerent: Unde huic sapiéntia hæc et virtútes? Nonne hic est fabri fílius? Nonne mater ejus dícitur María, et fratres ejus Jacóbus et Joseph et Simon et Judas? Et soróres ejus nonne omnes apud nos sunt? Unde ergo huic ómnia ista? Et scandalizabántur in eo. Jesus autem dixit eis: Non est prophéta sine honóre nisi in pátria sua et in domo sua. Et non fecit ibi virtútes multas propter incredulitátem illórum.
}\switchcolumn\portugues{
\blettrine{N}{aquele} tempo, vindo Jesus Para a sua pátria, ensinava nas suas sinagogas, de tal modo que diziam, cheios de admiração: «Donde Lhe vem tal sabedoria e tais prodígios? Porventura, não é Ele o Filho do carpinteiro? Sua Mãe se não chama Maria e seus irmãos não são Tiago e José, Simão e Judas? Suas irmãs não estão todas entre nós? Donde, pois, Lhe vem tudo quanto vemos?». escandalizavam-se por sua causa. Então, Jesus disse-lhes: «Somente na sua pátria e na sua casa é desprezado o Profeta». E não fez ali muitos milagres, por causa da sua incredulidade.
}\end{paracol}

\paragraphinfo{Ofertório}{Sl. 89, 17}
\begin{paracol}{2}\latim{
\rlettrine{B}{onítas} Dómini Dei nostri sit super nos, et opus mánuum nostrárum secúnda nobis, et opus mánuum nostrárum secúnda. Allelúja.
}\switchcolumn\portugues{
\qlettrine{Q}{ue} a bondade do Senhor, nosso Deus, seja sobre nós e abençoe o trabalho das nossas mãos; sim, Ele abençoe o trabalho das nossas mãos, aleluia.
}\end{paracol}

\paragraph{Secreta}
\begin{paracol}{2}\latim{
\qlettrine{Q}{uas} tibi, Dómine, de opéribus mánuum nostrárum offerímus hóstias, sancti Joseph interpósito suffrágio, pignus fácias nobis unitátis et pacis. Per Dóminum \emph{\&c.}
}\switchcolumn\portugues{
\rlettrine{E}{stas} hóstias, preparadas com o trabalho das nossas mãos, Vo-las oferecemos, Senhor, a fim de que, por intervenção dos sufrágios de São José, as convertais para nós em penhor de união e de paz. Por nosso Senhor \emph{\&c.}
}\end{paracol}

\paragraphinfo{Comúnio}{Mt. 13, 54 \& 55}
\begin{paracol}{2}\latim{
\rlettrine{U}{nde} huic sapiéntia hæc et virtútes? Nonne hic est fabri fílius? Nonne mater ejus dícitur María? Allelúja.
}\switchcolumn\portugues{
\rlettrine{D}{onde} Lhe vem tal sabedoria e tais prodígios? Porventura, não é Ele o Filho do carpinteiro? Sua Mãe se não chama Maria? Aleluia.
}\end{paracol}

\paragraph{Postcomúnio}
\begin{paracol}{2}\latim{
\rlettrine{H}{æc} sancta quæ súmpsimus Dómine: per intercessiónem beáti Joseph; et operatiónem nostram cómpleant, et praémia confírment. Per Dóminum. \emph{\&c.}
}\switchcolumn\portugues{
\qlettrine{Q}{ue} os Sacramentos agora recebidos, Senhor, completem por intercessão do B. José, o nosso labor e nos assegurem os prémios. Por nosso Senhor \emph{\&c.}
}\end{paracol}

\subsectioninfo{S. Cirilo de Alexandria, B. Conf. e Dr.}{9 de Fevereiro}

\textit{Como na Missa In médio Ecclésiae, página \pageref{doutores}, excepto:}

\paragraph{Oração}
\begin{paracol}{2}\latim{
\rlettrine{D}{eus,} qui beátum Cyríllum Confessórem tuum atque Pontíficem divínæ maternitátis beatíssimæ Vírginis Maríæ assertórem invíctum effecísti: concéde, ipso intercedénte; ut, qui vere eam Genetrícem Dei crédimus, matérna ejúsdem protectióne salvémur. Per eúndem Dóminum nostrum \emph{\&c.}
}\switchcolumn\portugues{
\slettrine{Ó}{} Deus, que tornastes o B. Cirilo, vosso Confessor e Pontífice, defensor invencível da divina maternidade da S. S. Virgem Maria, a nós, que acreditamos que ela é verdadeiramente Mãe de Deus, concedei por sua intercessão que sejamos salvos pela sua maternal protecção. Pelo \emph{\&c.}
}\end{paracol}

\paragraph{Secreta}
\begin{paracol}{2}\latim{
\rlettrine{M}{únera} nostra, omnípotens Deus, benígnus réspice: et, intercedénte beáto Cyríllo, præsta; ut unigénitum tuum Jesum Christum, Dóminum nostrum in tua tecum glória coætérnum, in córdibus nostris digne suscípere mereámur: Qui tecum \emph{\&c.}
}\switchcolumn\portugues{
\rlettrine{D}{eus} omnipotente, olhai benigno para os nossos dons; e por intercessão do B. Cirilo, concedei-nos que possamos receber dignamente nos nossos corações a N. S. Jesus Cristo, vosso Filho Unigénito, que é coeterno convosco na glória. Ele, que, sendo Deus \emph{\&c.}
}\end{paracol}

\paragraph{Postcomúnio}
\begin{paracol}{2}\latim{
\rlettrine{D}{ivínis,} Dómine, refécti mystériis, te súpplices deprecámur: ut, exémplis et méritis beáti Cyrílli Pontíficis adjúti, sanctíssimæ Genetríci Unigéniti tui digne famulári valeámus: Qui tecum vivit \emph{\&c.}
}\switchcolumn\portugues{
\rlettrine{F}{ortalecidos,} Senhor, com os divinos mystérios, Vos imploramos instantemente que, auxiliados com os exemplos e méritos do B. Pontífice Cirilo, possamos servir dignamente a S. S. Mãe do vosso Filho Unigénito. Ele, que \emph{\&c.}
}\end{paracol}

\subsectioninfo{S. Lourenço}{10 de Agosto}

\paragraphinfo{Intróito}{Sl. 95, 6}
\begin{paracol}{2}\latim{
\rlettrine{C}{onféssio} et pulchritúdo in conspéctu ejus: sánctitas et magnificéntia in sanctificatióne ejus. \emph{Ps. ibid., 1} Cantáte Dómino cánticum novum: cantáte Dómino, omnis terra.
℣. Gloria Patri \emph{\&c.}
}\switchcolumn\portugues{
\rlettrine{R}{odeiam-no} a glória e a majestade: e no seu santuário reluzem a santidade e a magnificência. \emph{Sl. ibid., 1} Cantai em honra do Senhor um hino novo: que toda a terra cante hinos ao Senhor.
℣. Glória ao Pai \emph{\&c.}
}\end{paracol}

\paragraph{Oração}
\begin{paracol}{2}\latim{
\rlettrine{D}{a} nobis, quǽsumus, omnípotens Deus: vitiórum nostrorum flammas exstínguere; qui beáto Lauréntio tribuísti tormentórum suórum incéndia superáre. Per Dóminum nostrum \emph{\&c.}
}\switchcolumn\portugues{
\slettrine{Ó}{} Deus omnipotente, que permitistes que o B. Lourenço triunfasse das chamas do seu suplício, concedei-nos a graça, Vos suplicamos, de extinguirdes as chamas dos nossos vícios. Por nosso Senhor \emph{\&c.}
}\end{paracol}

\paragraphinfo{Epístola}{2. Cor. 9, 6-10}
\begin{paracol}{2}\latim{
Léctio Epístolæ beáti Pauli Apóstoli ad Corinthios.
}\switchcolumn\portugues{
Lição da Ep.ª do B. Ap.º Paulo aos Coríntios.
}\switchcolumn*\latim{
\rlettrine{F}{ratres:} Qui parce séminat, parce et metet: et qui séminat in benedictiónibus, de benedictiónibus et metet. Unusquísque prout destinávit in corde suo, non ex tristítia aut ex necessitáte: hilárem enim datórem díligit Deus. Potens est autem Deus omnem grátiam abundáre fácere in vobis, ut, in ómnibus semper omnem sufficiéntiam habéntes, abundétis in omne opus bonum, sicut scriptum est: Dispérsit, dedit paupéribus: justítia ejus manet in sǽculum sǽculi. Qui autem admínistrat semen seminánti: et panem ad manducándum præstábit, et multiplicábit semen vestrum, et augébit increménta frugum justítiæ vestræ.
}\switchcolumn\portugues{
\rlettrine{M}{eus} irmãos: Aquele que semeia pouco, colherá pouco também; e aquele que semeia com abundância, colherá também com abundância. Que cada um dê segundo o que tiver resolvido no seu coração; mas não com tristeza, nem com constrangimento: pois Deus ama aquele que dá com alegria. Deus é assaz poderoso para vos cumular de todas as graças, a fim de que, possuindo, sempre, em todas as cousas, aquilo que vos é necessário, tenhais com abundância, para praticardes todas as espécies de boas obras, segundo o que está escrito: «Distribuiu liberalmente os seus bens pelos pobres: a sua justiça subsistirá em todos os séculos dos séculos». Com efeito, aquele que dá a semente ao semeador dar-vos-á também o pão para comida, multiplicará a vossa semente e dará incremento aos frutos da vossa justiça.
}\end{paracol}

\paragraphinfo{Gradual}{Sl. 16, 3}
\begin{paracol}{2}\latim{
\rlettrine{P}{robásti,} Dómine, cor meum, et visitásti nocte. ℣. Igne me examinásti, et non est invénta in me iníquitas.
}\switchcolumn\portugues{
\rlettrine{E}{xperimentastes} o meu coração e visitaste-lo durante a noite. ℣. Experimentastes-me com o fogo e não se encontrou em mim a iniquidade.
}\switchcolumn*\latim{
Allelúja, allelúja. ℣. Levíta Lauréntius bonum opus operátus est: qui per signum crucis cœcos illuminávit. Allelúja.
}\switchcolumn\portugues{
Aleluia, aleluia. ℣. O Levita Lourenço praticou uma boa acção: pois restituiu a vista aos cegos com o sinal da Santa Cruz. Aleluia.
}\end{paracol}

\paragraphinfo{Evangelho}{Página \pageref{vicente}}

\paragraphinfo{Ofertório}{Sl. 95, 6}
\begin{paracol}{2}\latim{
\rlettrine{C}{onféssio} et pulchritúdo in conspéctu ejus: sánctitas, et magnificéntia in sanctificatióne ejus.
}\switchcolumn\portugues{
\rlettrine{R}{odeiam-no} a glória e a majestade: e no seu santuário reluzem a santidade e a magnificência.
}\end{paracol}

\paragraph{Secreta}
\begin{paracol}{2}\latim{
\rlettrine{A}{ccipe,} quǽsumus, Dómine, múnera dignánter obláta: et, beáti Lauréntii suffragántibus méritis, ad nostræ salútis auxílium proveníre concéde. Per Dóminum \emph{\&c.}
}\switchcolumn\portugues{
\rlettrine{R}{ecebei,} Senhor, Vos suplicamos, os dons que reverentemente Vos oferecemos, e, pelo sufrágio dos méritos do B. Lourenço, permiti que nos sirvam de auxílio para a salvação. Por nosso Senhor Jesus Cristo \emph{\&c.}
}\end{paracol}

\paragraphinfo{Comúnio}{Jo. 12, 26}
\begin{paracol}{2}\latim{
\qlettrine{Q}{ui} mihi mínistrat, me sequátur: et ubi ego sum, illic et miníster meus erit.
}\switchcolumn\portugues{
\rlettrine{S}{e} alguém me serve, siga-me; e onde eu estiver lá estará também o meu servo.
}\end{paracol}

\paragraph{Postcomúnio}
\begin{paracol}{2}\latim{
\rlettrine{S}{acro} múnere satiáti, súpplices te, Dómine, deprecámur: ut, quod débitæ servitútis celebrámus offício, intercedénte beáto Lauréntio Mártyre tuo, salvatiónis tuæ sentiámus augméntum. Per Dóminum \emph{\&c.}
}\switchcolumn\portugues{
\rlettrine{S}{aciados} com este sacrossanto dom, humildemente Vos rogamos pela intercessão do B. Lourenço, vosso Mártir, que, celebrando este ofício em reconhecimento da nossa escravidão, alcancemos cada vez mais os efeitos da vossa Redenção. Por nosso Senhor \emph{\&c.}
}\end{paracol}

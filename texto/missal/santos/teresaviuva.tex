\subsectioninfo{B. Teresa, Viúva}{17 de Junho}

\paragraphinfo{Intróito}{Sl. 44, 15-16}
\begin{paracol}{2}\latim{
\rlettrine{A}{dducéntur} Regi Vírgines post eam, próximæ ejus afferéntur in lætitia et exsultatióne: adducéntur in templum Regis Dómini. \emph{Ps. ibid., 2} Eructávit cor meum verbum bonum: dico ego ópera mea Regi.
℣. Gloria Patri \emph{\&c.}
}\switchcolumn\portugues{
\rlettrine{A}{s} virgens serão apresentadas ao Rei após ela: as suas companheiras serão conduzidas em transportes de alegria e de júbilo: serão apresentadas ao Senhor no templo do Rei. \emph{Sl. ibid., 2} Meu coração proferiu uma excelente palavra: «Consagro ao Rei as minhas obras».
℣. Glória ao Pai \emph{\&c.}
}\end{paracol}

\paragraph{Oração}
\begin{paracol}{2}\latim{
\rlettrine{D}{eus,} qui ad Ecclésiam tuam novis semper exémplis imbuéndam, beátam Terésiam, mundáno principátu despécto, ad humilitátis tuæ sectánda vestigia toto corde transire fecísti: concéde; ut ipsíus méritis et exémplo discámus peritúras mundi calcáre delícias, et in ampléxu tuæ crucis ómnia nobis adversántia superáre: Qui vivis et regnas \emph{\&c.}
}\switchcolumn\portugues{
\slettrine{Ó}{} Deus, que, para instruir a vossa Igreja com exemplos sempre novos, fizestes que a B. Teresa, desprezando a realeza deste mundo, seguisse com todo seu coração os exemplos da vossa humildade, concedei-nos que pelos seus méritos e exemplos desprezemos as glórias e delícias deste mundo e, abraçados à vossa Paixão, vençamos todas as adversidades. Ó Vós, que viveis e reinais \emph{\&c.}
}\end{paracol}

\paragraphinfo{Epístola}{Ct. 2, 8-14}
\begin{paracol}{2}\latim{
Léctio libri Sapiéntiæ.
}\switchcolumn\portugues{
Lição do Livro da Sabedoria.
}\switchcolumn*\latim{
\rlettrine{V}{ox} dilécti mei: ecce, iste venit sáliens in móntibus, transíliens colles; símilis est diléctus meus cápreæ hinnulóque cervórum. En, ipse stat post paríetem nostrum, respíciens per fenéstras, prospíciens per cancéllos. En, diléctus meus lóquitur mihi: Surge, própera, amíca mea, colúmba mea, formósa mea, et veni. Jam enim hiems tránsiit, imber ábiit et recéssit. Flores apparuérunt in terra nostra, tempus putatiónis advénit: vox túrturis audíta est in terra nostra: ficus prótulit grossos suos: víneæ floréntes dedérunt odórem suum. Surge, amíca mea, speciósa mea, et veni: colúmba mea in foramínibus petra, in cavérna macériæ, osténde mihi fáciem tuam, sonet vox tua in áuribus meis: vox enim tua dulcis et fácies tua decóra.
}\switchcolumn\portugues{
\rlettrine{A}{quela} é a voz do meu amado: eis que ele vem, galgando montes e transpondo outeiros! Meu amado é semelhante ao gamo e ao filho das corças. Eis que ele vem por detrás da nossa parede, olhando pelas janelas e espreitando pelas frestas. E o meu amado fala-me e diz: «Ergue-te, apressa-te e vem, ó minha amiga, ó minha pomba, ó minha única beleza! Pois já o Inverno acabou; já as chuvas cessaram e se retiraram. As flores brotaram nos nossos jardins; chegou o tempo da poda; ouve-se a voz da rola nos nossos campos; a figueira começa a mostrar os primeiros frutos e as vinhas em flor exalam aromas! Ergue-te e vem, minha amiga, minha única beleza! Ó minha pomba escondida nas fendas das rochas e nas cavernas dos muros em ruínas, mostra-me a tua face e soe tua voz nos meus ouvidos. A tua voz é doce e a tua face graciosa!».
}\end{paracol}

\paragraphinfo{Gradual}{Sl. 115, 16-17}
\begin{paracol}{2}\latim{
\rlettrine{D}{irupísti} víncula mea: tibi sacrificábo hóstiam laudis, et nomen Dómini invocábo. ℣. \emph{Ps. ibid., 18-19} Vota mea Dómino reddam in conspéctu omnis pópuli ejus, in átriis domus Dómini.
}\switchcolumn\portugues{
\qlettrine{Q}{uebrastes} as minhas cadeias, ó meu Deus! Eu Vos oferecerei, pois, um sacrifício d e louvor e invocarei o nome do Senhor. ℣. \emph{Sl. ibid., 18-19} Oferecerei os meus votos ao Senhor na presença de todo seu povo e nos átrios da casa do Senhor.
}\switchcolumn*\latim{
Allelúja, allelúja. \emph{Cant. 2, 3} Sub umbra illíus, quem desideráveram, sedi: et fructus ejus dulcis gúturi mea. Allelúja.
}\switchcolumn\portugues{
Aleluia, aleluia. \emph{Ct. 2, 3} Sentei-me à sombra daquele que eu desejara e o seu fruto é muito saboroso ao meu paladar. Aleluia.
}\end{paracol}

\paragraphinfo{Evangelho}{Página \pageref{virgensmartires2}}

\paragraph{Ofertório}
\begin{paracol}{2}\latim{
\rlettrine{V}{ultum} tuum deprecabúntur omnes dívites plebis: filiæ regum in honóre tuo.
}\switchcolumn\portugues{
\rlettrine{T}{odos} os poderosos do povo implorarão o seu rosto: as filhas dos reis honrar-vos-ão.
}\end{paracol}

\paragraph{Secreta}
\begin{paracol}{2}\latim{
\rlettrine{S}{úscipe,} Dómine, sacrifícium, cujus te voluísti dignánter immolatióne placári: et præsta, ut beátæ Terésiæ intercessióne nobis profíciat ad salútem. Per Dominum \emph{\&c.}
}\switchcolumn\portugues{
\rlettrine{R}{ecebei,} Senhor, este sacrifício com a imolação do qual quisestes ser dignamente aplacado; e concedei-nos que pela intercessão da B. Teresa seja proveitoso à nossa salvação. Por nosso Senhor \emph{\&c.}
}\end{paracol}

\paragraphinfo{Comúnio}{Ct. 6, 8}
\begin{paracol}{2}\latim{
\rlettrine{V}{idérunt} eam fíliæ, et beatíssimam prædicavérunt; reginæ, et laudavérunt eam.
}\switchcolumn\portugues{
\rlettrine{A}{s} filhas (de Sião) viram-na e proclamaram-na bem-aventurada: e as rainhas louvaram-na.
}\end{paracol}

\paragraph{Postcomúnio}
\begin{paracol}{2}\latim{
\rlettrine{A}{scéndant} ad te, Dómine, preces nostræ, Beatæ Terésiæ suffragántibus méritis: ut cæléstibus dápibus temporáliter recreáti, ætérni convívii dulcédine perfruámur. Per Dóminum \emph{\&c.}
}\switchcolumn\portugues{
\rlettrine{S}{ubam} até Vós, Senhor, as nossas preces, e pelos sufrágios e méritos da B. Teresa permiti que, assim como neste mundo nos alegramos com a posse das riquezas celestiais, assim também gozemos a doçura do convívio eterno. Por nosso Senhor \emph{\&c.}
}\end{paracol}

\subsectioninfo{S. Timóteo, B. e Márt.}{24 de Janeiro}

\textit{Como na Missa Státuit ei Dóminus, página \pageref{martirpontificeforapascal}, excepto:}

\paragraphinfo{Epístola}{1. Tm. 6, 11-16}
\begin{paracol}{2}\latim{
Léctio Epístolæ beáti Pauli Apóstoli ad Timótheum. 
}\switchcolumn\portugues{
Lição da Ep.ª do B. Ap.º Paulo a Timóteo.
}\switchcolumn*\latim{
\rlettrine{C}{aríssime:} Sectáre justítiam, pietátem, fidem, caritátem, patiéntiam, mansuetúdinem. Certa bonum certámen fídei, apprehénde vitam ætérnam, in qua vocátus es, et conféssus bonam confessionem coram multis téstibus. Præcípio tibi coram Deo, qui vivíficat ómnia, et Christo Jesu, qui testimónium réddidit sub Póntio Piláto, bonam confessiónem: ut serves mandátum sine mácula, irreprehensíbile usque in advéntum Dómini nostri Jesu Christi, quem suis tempóribus osténdet beátus et solus potens, Rex regum et Dóminus dominántium: qui solus habet immortalitátem, et lucem inhábitat inaccessíbilem: quem nullus hóminum vidit, sed nec vidére potest: cui honor et impérium sempitérnum. Amen.
}\switchcolumn\portugues{
\rlettrine{C}{aríssimo:} Procurai a justiça, a piedade, a fé, a caridade, a paciência e a mansidão. Combatei o bom combate da fé; esforçai-vos em alcançar a vida eterna, para a qual fostes chamado e para a qual fizestes esta boa profissão de fé diante de muitos testemunhos. Eu vos ordeno diante de Deus, que dá a vida a todas as coisas, e perante Jesus Cristo, que deu testemunho por uma boa profissão de fé, sob Pôncio Pilatos, que oportunamente mostrará o Bem-aventurado e o único Soberano, o Rei dos reis e o Senhor dos senhores, que só possui a imortalidade e goza uma luz inacessível, que ninguém nunca viu, nem ainda pode ver e a quem seja dada a honra e poder eterno. Amen.
}\end{paracol}
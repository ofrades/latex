\subsectioninfo{S. Gonçalo de Amarante, Confesso}{19 de Janeiro}

\textit{Como Missa Justus ut palma, página \pageref{confessoresnaopontifices2}, excepto:}

\paragraph{Oração}
\begin{paracol}{2}\latim{
\rlettrine{D}{eus,} qui beáti Gundisálvi Confessóris tui mentem sancti nóminis tui amóre mirabíliter inflammásti: concéde, quæsumus; ut illius vestígiis inhæréntes, semper te cogitémus, quæque tibi grata sunt, inflammáto stúdio faciámus. Per Dóminum \emph{\&c.}
}\switchcolumn\portugues{
\slettrine{Ó}{} Deus, que maravilhosamente abrasastes no amor ao vosso nome a alma do B. Gonçalo, vosso Confessor, dignai-Vos conceder-nos que, seguindo os seus vestígios, pensemos sempre em Vós e com fervorosa solicitude façamos o que Vos seja agradável. Por nosso Senhor \emph{\&c.}
}\end{paracol}

\paragraphinfo{Oração}{S. S. Mário e Outros Mártires}
\begin{paracol}{2}\latim{
\rlettrine{E}{xáudi,} Dómine, pópulum tuum cum Sanctórum tuórum patrocínio supplicántem: ut et temporális vitæ nos tríbuas pace gaudére; et ætérnæ reperíre subsídium. Per Dóminum \emph{\&c.}
}\switchcolumn\portugues{
\rlettrine{O}{uvi,} Senhor, as súplicas que o vosso povo Vos dirige sob o patrocínio dos vossos Santos; e dignai-Vos permitir que gozemos a vossa paz na vida presente e alcancemos o vosso auxílio na vida eterna. \emph{\&c.}
}\end{paracol}

\paragraphinfo{Oração}{S. Canuto}
\begin{paracol}{2}\latim{
\rlettrine{D}{eus,} qui ad illustrándam Ecclésiam tuam beátum Canútum regem martýrii palma et gloriósis miráculis decoráre dignátus es: concéde propítius; ut, sicut ipse Domínicæ passiónis imitátor fuit, ita nos, per ejus vestígia gradiéntes, ad gáudia sempitérna perveníre mereámur. Per eúndem Dóminum \emph{\&c.}
}\switchcolumn\portugues{
\slettrine{Ó}{} Deus, que para glória da vossa Igreja Vos dignastes honrar o B. Rei Canuto com a palma do martírio e o dom de insignes milagres, concedei-nos propício que, assim como ele imitou a Paixão do Senhor, assim também nós, segundo os seus vestígios, mereçamos alcançar os sempiternos gozos. Por nosso Senhor \emph{\&c.}
}\end{paracol}

\paragraphinfo{Epístola}{Página \pageref{pauloeremita}}

\paragraphinfo{Gradual}{Sl. 20, 4}
\begin{paracol}{2}\latim{
\rlettrine{D}{ómine,} prævenísti eum in benedictiónibus dulcédinis: posuísti in cápite ejus corónam de lápide pretióso. ℣. \emph{ibid., 5} Vitam pétiit a te: et tribuísti ei longitúdinem diérum in sæculum sæculi.
}\switchcolumn\portugues{
\rlettrine{C}{oncedestes-lhe,} Senhor, bênçãos escolhidas, as mais suaves, e impusestes na sua cabeça uma coroa de pedras preciosas. ℣. \emph{ibid., 5} Concedestes-lhe a vida que Vos suplicou e prolongastes-lhe a duração dos seus dias pelos séculos dos séculos.
}\switchcolumn*\latim{
Allelúja, allelúja. Lætábitur justus in Dómino, et sperábit in eo: et laudabúntur omnes recti corde. Allelúja.
}\switchcolumn\portugues{
Aleluia, aleluia. O justo alegrar-se-á no Senhor e n’Ele porá suas esperanças; e todos aqueles cujo coração é recto serão glorificados. Aleluia.
}\end{paracol}

\textit{Após a Septuagésima omite-se o Aleluia e o Verso e diz-se:}

\paragraphinfo{Trato}{Sl. 111, 1-3}
\begin{paracol}{2}\latim{
\rlettrine{B}{eátus} vir, qui timet Dóminum: in mandátis ejus cupit nimis. ℣. Potens in terra erit semen ejus: generátio rectórum benedicétur. ℣. Glória et divitiæ in domo ejus: et justítia ejus manet in sǽculum sǽculi.
}\switchcolumn\portugues{
\rlettrine{B}{em-aventurado} o varão que teme o Senhor e que põe todo seu zelo em obedecer-Lhe. ℣. Sua descendência será poderosa na terra; pois a geração dos justos será abençoada. ℣. Na sua casa haverá glória e riqueza: e a justiça subsistirá em todos os séculos dos séculos.
}\end{paracol}

\paragraphinfo{Evangelho}{Página \pageref{pauloeremita}}

\paragraphinfo{Ofertório}{Sl. 91, 13}
\begin{paracol}{2}\latim{
\qlettrine{J}{ustus} ut palma florébit, sicut cedrus, quæ in Líbano est, multiplicábitur.
}\switchcolumn\portugues{
\rlettrine{O}{} justo florescerá, como a palmeira, e multiplicar-se-á, como o cedro do Líbano.
}\end{paracol}

\paragraphinfo{Secreta}{S. S. Mário e Outros Mártires}
\begin{paracol}{2}\latim{
\rlettrine{P}{reces,} Dómine, tuórum réspice oblationésque fidélium: ut et tibi gratæ sint pro tuórum festivitáte Sanctórum, et nobis cónferant tuæ propitiatiónis auxílium. Per Dóminum \emph{\&c.}
}\switchcolumn\portugues{
\rlettrine{A}{tendei,} Senhor, ás preces e ás oblatas dos vossos fiéis, a fim de que Vos sejam agradáveis nesta festa dos vossos Santos e nos alcancem o auxílio da vossa bondade. Por nosso Senhor \emph{\&c.}
}\end{paracol}

\paragraphinfo{Secreta}{S. Canuto}
\begin{paracol}{2}\latim{
\rlettrine{A}{ccépta} sit in conspéctu tuo, Dómine, nostra devótio: et ejus nobis fiat supplicatióne salutáris, pro cujus sollemnitáte defértur. Per Dóminum \emph{\&c.}
}\switchcolumn\portugues{
\rlettrine{R}{ecebei} benigno, Senhor, esta oferta da nossa piedade, e que ela nos alcance a salvação, por intercessão das preces daquele em cuja festa Vo-la apresentamos. Por nosso Senhor \emph{\&c.}
}\end{paracol}

\paragraphinfo{Comúnio}{Mt. 19, 91, 13}
\begin{paracol}{2}\latim{
\rlettrine{A}{men} dico vobis, quod vos, qui reliquístis ómnia, et secúti estis me, céntuplum accipiétis, et vitam ætérnam possidébitis.
}\switchcolumn\portugues{
\rlettrine{E}{m} verdade vos digo: vós, que abandonastes tudo e me seguistes, recebereis o cêntuplo e possuireis a vida eterna.
}\end{paracol}

\paragraphinfo{Postcomúnio}{S. S. Mário e Outros Mártires}
\begin{paracol}{2}\latim{
\rlettrine{S}{anctórum} tuórum, Dómine, intercessióne placátus: præsta, quǽsumus; ut, quæ temporáli celebrámus actióne, perpétua salvatióne capiámus. Per Dóminum \emph{\&c.}
}\switchcolumn\portugues{
\rlettrine{D}{eixai-Vos} aplacar, Senhor, pela intercessão dos vossos Santos; e permiti, Vos rogamos, que estes sacrifícios, que aogra celebrámos, nos sirvam de auxílio para a salvação eterna. Por nosso Senhor \emph{\&c.}
}\end{paracol}

\paragraphinfo{Postcomúnio}{S. Canuto}
\begin{paracol}{2}\latim{
\rlettrine{R}{efécti} participatióne múneris sacri, quǽsumus, Dómine, Deus noster: ut, cujus exséquimur cultum, intercedénte beáto Canúto Mártyre tuo, sentiámus efféctum. Per Dóminum \emph{\&c.}
}\switchcolumn\portugues{
\rlettrine{P}{ermiti,} ó Senhor, nosso Deus, que, assim como nos alegrámos, comemorando nesta vida pelo nosso ministério a memória dos vossos Santos, assim também tenhamos na eternidade a felicidade de os contemplar. Por nosso Senhor \emph{\&c.}
}\end{paracol}

\subsectioninfo{S. Jorge, Patrono de Portugal}{23 de Abril}

\textit{Como na Missa Protexísti me, página \pageref{martir}, excepto:}

\paragraph{Oração}
\begin{paracol}{2}\latim{
\rlettrine{D}{eus,} qui nos beáti Georgii Martyris tui méritis et intercessióne lætíficas: concéde propítius; ut, qui tua per eum benefícia póscimus, dono tuæ grátiæ consequámur. Per Dóminum \emph{\&c.}
}\switchcolumn\portugues{
\slettrine{Ó}{} Deus, que nos alegrais com os méritos e intercessão do B. Jorge, vosso Mártir, concedei-nos propício que, suplicando-Vos por sua intercessão os vossos benefícios, os obtenhamos por efeito da vossa graça. Por nosso Senhor \emph{\&c.}
}\end{paracol}

\paragraphinfo{Epístola}{Página \pageref{martirnaopontifice2}}

\paragraph{Secreta}
\begin{paracol}{2}\latim{
\rlettrine{M}{únera,} Dómine, obláta sanctífica: et, intercedénte beáto Geórgio Mártyre tuo, nos per hæc a peccatórum nostrórum máculis emúnda. Per Dóminum \emph{\&c.}
}\switchcolumn\portugues{
\rlettrine{S}{antificai,} Senhor, estas oblatas que Vos são oferecidas; e pela intercessão do B. Jorge, vosso Mártir, purificai-nos, pela sua virtude, das manchas dos nossos pecados. Por nosso Senhor \emph{\&c.}
}\end{paracol}

\paragraph{Postcomúnio}
\begin{paracol}{2}\latim{
\rlettrine{S}{úpplices} te rogámus, omnípotens Deus: ut, quos tuis réficis sacraméntis, intercedénte beáto Geórgio Mártyre tuo, tibi étiam plácitis móribus dignánter tríbuas deservíre. Per Dóminum \emph{\&c.}
}\switchcolumn\portugues{
\slettrine{Ó}{} Deus omnipotente, Vos imploramos, pela intercessão do B. Jorge, vosso Mártir, dignai-Vos conceder àqueles a quem sustentais com vossos sacramentos a graça de Vos servirem, como convém, com conduta de vida que Vos agrade. Por nosso Senhor \emph{\&c.}
}\end{paracol}

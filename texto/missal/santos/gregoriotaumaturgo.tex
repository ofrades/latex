\subsectioninfo{S. Gregório Taumaturgo, B. e C.}{17 de Novembro}

\textit{Como na Missa Státuit ei, página \pageref{confessorespontifices1}, excepto:}

\paragraphinfo{Evangelho}{Mc. 11, 22-24}
\begin{paracol}{2}\latim{
\cruz Sequéntia sancti Evangélii secúndum Marcum.
}\switchcolumn\portugues{
\cruz Continuação do santo Evangelho segundo S. Marcos.
}\switchcolumn*\latim{
\blettrine{I}{n} illo témpore: Respóndens Jesus discípulis suis, ait illis: Habéte fidem Dei. Amen, dico vobis, quia, quicúmque díxerit huic monti: Tóllere et míttere in mare, et non hæsitáverit in corde suo, sed credíderit, quia, quodcúmque díxerit, fiat, fiet ei. Proptérea dico vobis: Omnia quæcúmque orántes pétitis, crédite quia accipiétis, et evénient vobis. 
}\switchcolumn\portugues{
\blettrine{N}{aquele} tempo, Jesus disse aos seus discípulos: «Tende fé em Deus. Em verdade vos digo que todo aquele que disser a esta montanha «tira-te e lança-te no mar», e disser isto sem hesitar no seu coração e até acreditando que tudo o que disse acontecerá, fique certo de que o verá cumprir-se. Eis porque vos digo: Tudo quanto pedirdes na oração acreditai que o alcançareis e vereis».
}\end{paracol}
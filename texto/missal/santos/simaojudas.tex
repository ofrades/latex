\subsectioninfo{S. Simão e S. Judas}{28 de Outubro}\label{simaojudas}

\paragraphinfo{Intróito}{Sl, 138, 17}
\begin{paracol}{2}\latim{
\rlettrine{M}{ihi} autem nimis honoráti sunt amíci tui, Deus: nimis confortátus est principátus eórum. \emph{Ps. ibid., 1-2} Dómine, probásti me et cognovísti me: tu cognovísti sessiónem meam et resurrectiónem meam.
℣. Gloria Patri \emph{\&c.}
}\switchcolumn\portugues{
\rlettrine{E}{u} vejo, ó Deus, que honrais largamente os vossos amigos: e que seu poder se tem fortalecido extraordinariamente. \emph{Sl. ibid., 1-2} Senhor, perscrutastes o meu íntimo e ficastes conhecendo-me: ficastes conhecendo quando me deito e me levanto.
℣. Glória ao Pai \emph{\&c.}
}\end{paracol}

\paragraph{Oração}
\begin{paracol}{2}\latim{
\rlettrine{D}{eus,} qui nos per beatos Apóstolos tuos Simónem et Judam ad agnitiónem tui nóminis veníre tribuísti: da nobis eórum glóriam sempitérnam et proficiéndo celebráre, et celebrándo profícere. Per Dóminum \emph{\&c.}
}\switchcolumn\portugues{
\slettrine{Ó}{} Deus, que pelos vossos B. B. Apóstolos Simão e Judas nos proporcionastes o conhecimento do vosso Nome, concedei-nos a graça de celebrar a sua eterna glória, progredindo na virtude, e celebrando esta mesma glória, alcancemos novos progressos. Por nosso Senhor \emph{\&c.}
}\end{paracol}

\paragraphinfo{Epístola}{Ef. 4, 7-13}
\begin{paracol}{2}\latim{
Léctio Epístolæ beáti Pauli Apóstoli ad Ephésios.
}\switchcolumn\portugues{
Lição da Ep.ª do B. Ap.º Paulo aos Efésios.
}\switchcolumn*\latim{
\rlettrine{F}{ratres:} Unicuíque nostrum data est grátia secúndum mensúram donatiónis Christi. Propter quod dicit: Ascéndens in altum, captívam duxit captivitátem: dedit dona homínibus. Quod autem ascéndit, quid est, nisi quia et descéndit primum in inferióres partes terræ? Qui descéndit, ipse est et qui ascéndit super omnes cœlos, ut impléret ómnia. Et ipse dedit quosdam quidem apóstolos, quosdam autem prophétas, álios vero evangelístas, álios autem pastóres et doctóres, ad consummatiónem sanctórum in opus ministérii, in ædificatiónem córporis Christi: donec occurrámus omnes in unitátem fídei, et agnitiónis Fílii Dei, in virum perféctum, in mensúram ætátis plenitúdinis Christi.
}\switchcolumn\portugues{
\rlettrine{M}{eus} irmãos: A graça é dada a cada um de nós, segundo a medida do dom de Cristo. Por isso se diz na Escritura: «Subindo ao alto, levou consigo os cativos e repartiu os seus dons pelos homens». Que quer dizer «subiu» senão que, antes, tinha descido às regiões inferiores da terra? Aquele que desceu é o mesmo que subiu acima de todos os céus, a fim de completar todas as coisas. Foi Ele, também, quem deu à sua Igreja uns como Apóstolos, outros como Profetas, outros como Evangelistas, outros como Pastores e Doutores, para que uns e outros trabalhem para a perfeição dos santos, nas funções do ministério e na edificação do corpo de Cristo, até que todos cheguemos à unidade de uma só fé e do mesmo conhecimento do Filho de Deus, ao estado de um homem perfeito a um desenvolvimento orgânico proporcionado à plenitude de Cristo.
}\end{paracol}

\paragraphinfo{Gradual}{Sl. 44, 17-18}
\begin{paracol}{2}\latim{
\rlettrine{C}{onstítues} eos príncipes super omnem terram: mémores erunt nóminis tui, Dómine. ℣. Pro pátribus tuis nati sunt tibi fílii: proptérea pópuli confitebúntur tibi.
}\switchcolumn\portugues{
\rlettrine{V}{ós} os instituístes príncipes em todo o universo: e eles, Senhor, perpetuarão a glória do vosso nome em toda a terra. ℣. Para substituir os vossos pais, nascer-vos-ão filhos: pelo que os povos vos louvarão.
}\switchcolumn*\latim{
Allelúja, allelúja. ℣. \emph{Ps. 138, 17} Nimis honoráti sunt amíci tui, Deus: nimis confortátus est principátus eórum. Allelúja.
}\switchcolumn\portugues{
Aleluia, aleluia. ℣. \emph{Sl. 138, 17} Honrais largamente os vossos filhos, ó Deus; o seu poder tem-se fortalecido extraordinariamente. Aleluia.
}\end{paracol}

\paragraphinfo{Evangelho}{Jo. 15, 17-25}
\begin{paracol}{2}\latim{
\cruz Sequéntia sancti Evangélii secúndum Joánnem.
}\switchcolumn\portugues{
\cruz Continuação do santo Evangelho segundo S. João.
}\switchcolumn*\latim{
\blettrine{I}{n} illo témpore: Dixit Jesus discípulis suis: Hæc mando vobis, ut diligátis ínvicem. Si mundus vos odit: scitóte, quia me priórem vobis odio hábuit. Si de mundo fuissétis, mundus quod suum erat dilígeret; quia vero de mundo non estis, sed ego elegi vos de mundo, proptérea odit vos mundus. Mementóte sermónis mei, quem ego dixi vobis: Non est servus major dómino suo. Si me persecúti sunt, et vos persequántur: si sermónem meum servavérunt, et vestrum servábunt. Sed hæc ómnia fácient vobis propter nomen meum: quia nésciunt eum, qui misit me. Si non veníssent et locútus fuíssem eis, peccátum non háberent: nunc autem excusatiónem non habent de peccáto suo. Qui me odit: et Patrem meum odit. Si ópera non fecíssem in eis, quæ nemo álius fecit, peccátum non háberent: nunc autem et vidérunt et odérunt et me et Patrem meum. Sed ut adimpleátur sermo, qui in lege eórum scriptus est: Quia ódio habuérunt me gratis.
}\switchcolumn\portugues{
\blettrine{N}{aquele} tempo, disse Jesus aos seus discípulos: «Recomendo-vos que vos ameis uns aos outros. Se o mundo vos odeia, lembrai-vos de que me odiou a mim antes. Se fôsseis do mundo, o mundo amaria o que lhe pertenceria; mas, como não sois do mundo e Eu vos escolhi do mundo, por isso ele vos odiará. Lembrai-vos de que vos disse: «O servo mais não é do que o senhor». Se, pois, me perseguiram, perseguir-vos-ão, também; se guardaram as minhas palavras, guardarão também as vossas. Farão todas estas coisas por causa do meu nome, porque não conhecem Aquele que me mandou. Se não tivesse vindo e lhes não houvesse falado, não seriam culpados; mas agora não têm desculpa do seu pecado. Quem me odeia, odeia também meu Pai. Se Eu no meio de vós não houvesse praticado obras, que ninguém pratica, não teriam pecado; porém, as viram e me odeiam, bem como a meu Pai. Assim se cumpre neles aquela sentença que está escrita na sua lei: «Odiaram-me sem motivo algum».
}\end{paracol}

\paragraphinfo{Ofertório}{Sl. 18, 5}
\begin{paracol}{2}\latim{
\rlettrine{I}{n} omnem terram exívit sonus eórum: et in fines orbis terræ verba eórum.
}\switchcolumn\portugues{
\rlettrine{O}{} som da sua voz ecoou por toda a parte, e as suas palavras estenderam-se até às extremidades da terra.
}\end{paracol}

\paragraph{Secreta}
\begin{paracol}{2}\latim{
\rlettrine{G}{lóriam,} Dómine, sanctórum Apostolórum tuórum Simónis et Judæ perpétuam venerántes: quǽsumus; ut eam, sacris mystériis expiati, dígnius celebrémus. Per Dóminum \emph{\&c.}
}\switchcolumn\portugues{
\rlettrine{V}{enerando} nós, Senhor, a perpétua glória dos vossos Santos Apóstolos Simão e Judas, purificai-nos, Vos imploramos, por estes sacrossantos mistérios, a fim de que os celebremos mais dignamente. Por nosso Senhor \emph{\&c.}
}\end{paracol}

\paragraphinfo{Comúnio}{Mt. 19, 28}
\begin{paracol}{2}\latim{
\rlettrine{V}{os,} qui secúti estis me, sedébitis super sedes, judicántes duódecim tribus Israël.
}\switchcolumn\portugues{
\rlettrine{V}{ós,} que me seguistes, assentar-vos-eis sobre tronos e julgareis as doze tribos de Israel.
}\end{paracol}

\paragraph{Postcomúnio}
\begin{paracol}{2}\latim{
\rlettrine{P}{ercéptis,} Dómine, sacraméntis, supplíciter exorámus: ut, intercedéntibus beátis Apóstolis tuis Simóne et Juda, quæ pro illórum veneránda gérimus passióne, nobis profíciant ad medélam. Per Dóminum \emph{\&c.}
}\switchcolumn\portugues{
\rlettrine{H}{avendo} participado destes sacramentos, Vos pedimos humildemente, Senhor, que pela intercessão dos vossos B. B. Apóstolos Simão e Judas, o mistério que celebramos em honra do seu martírio, nos sirva de remédio para os nossos males. Por nosso Senhor \emph{\&c.}
}\end{paracol}

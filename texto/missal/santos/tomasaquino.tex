\subsectioninfo{S. Tomás Aquino, Conf. e Doutor}{7 de Março}\label{tomasaquino}

\textit{Como na Missa In médio Ecclésiae, página \pageref{doutores}, excepto:}

\paragraph{Oração}
\begin{paracol}{2}\latim{
\rlettrine{D}{eus,} qui Ecclésiam tuam beáti Thomæ Confessóris tui mira eruditióne claríficas, et sancta operatióne fecúndas: da nobis, quǽsumus; et quæ dócuit, intelléctu conspícere, et quæ egit, imitatióne complére. Per Dóminum nostrum \emph{\&c.}
}\switchcolumn\portugues{
\slettrine{Ó}{} Deus, que ilustrais a vossa Igreja com a admirável sabedoria do B. Tomás, vosso Confessor, e a fecundais com a santidade das suas acções, concedei-nos, Vos suplicamos, que compreendamos o que ele ensinou e imitemos com as nossas acções o que ele praticou. Por nosso Senhor \emph{\&c.}
}\end{paracol}

\paragraphinfo{Epístola}{Sb. 7, 7-14}
\begin{paracol}{2}\latim{
Léctio libri Sapiéntiæ. 
}\switchcolumn\portugues{
Lição do Livro da Sabedoria.
}\switchcolumn*\latim{
\rlettrine{O}{ptávi,} et datus est mihi sensus: et invocávi, et venit in me spíritus sapiéntiæ: et præpósui illam regnis et sédibus, et divítias nihil esse duxi in comparatióne illíus: nec comparávi illi lápidem pretiósum: quóniam omne aurum in comparatióne illíus arena est exígua, et tamquam lutum æstimábitur argéntum in conspéctu illíus. Super salútem et spéciem diléxi illam, et propósui pro luce habére illam: quóniam inexstinguíbile est lumen illíus. Venérunt autem mihi ómnia bona páriter cum illa, et innumerábilis honéstas per manus illíus, et lætátus sum in ómnibus: quóniam antecedébat me ista sapiéntia, et ignorábam, quóniam horum ómnium mater est. Quam sine fictióne dídici et sine invídia commúnico, et honestátem illíus non abscóndo. Infinítus enim thesáurus est homínibus: quo qui usi sunt, partícipes facti sunt amicítiæ Dei, propter disciplínæ dona commendáti.
}\switchcolumn\portugues{
\rlettrine{D}{esejei} a inteligência, e foi-me dada; invoquei o espírito da sabedoria, e veio a mim. Preferia-a aos reinos e aos tronos; e creio que as riquezas nada são comparadas com ela. Nem mesmo a compararei com as pedras preciosas; pois todo o ouro, comparando-o com ela, é como o grão da areia; e toda a prata, ao pé dela, é desprezível lodo. Amo-a mais do que a saúde e a beleza; e, assim, resolvi tomá-la para minha luz, pois a sabedoria guiava-me, e eu ignorava que ela é a mãe de todos os bens. Conheci a sabedoria sem fingimento e comunico-a sem inveja, não ocultando as suas riquezas. Ela é um tesouro infinito para os homens. Aqueles que a aproveitam tornam-se amigos de Deus e recomendam-se pelos dons da ciência.
}\end{paracol}
\subsectioninfo{S. S. Vito, Modesto e Crescência, Mártires}{15 de Junho}\label{vitomodestocrescencia}

\paragraphinfo{Intróito}{Sl. 33, 20-21}
\begin{paracol}{2}\latim{
\rlettrine{M}{ultæ} tribulationes justórum, et de his ómnibus liberávit eos Dóminus: Dóminus custodit ómnia ossa eórum: unum ex his non conterétur. \emph{Ps. ib., 2} Benedícam Dóminum in omni témpore: semper laus ejus in ore meo.
℣. Gloria Patri \emph{\&c.}
}\switchcolumn\portugues{
\rlettrine{M}{uitas} são as tribulações dos justos, mas de todas elas o Senhor os livrará: o Senhor guarda todos seus ossos e nem um só deles será quebrado. \emph{Sl. ib., 2} Bendirei o Senhor em todo o tempo; o seu louvor estará sempre na minha boca.
℣. Glória ao Pai \emph{\&c.}
}\end{paracol}

\paragraph{Oração}
\begin{paracol}{2}\latim{
\rlettrine{D}{a} Ecclésiæ tuæ, quǽsumus, Dómine, sanctis Martýribus tuis Vito, Modésto atque Crescéntia intercedéntibus, supérbe non sápere, sed tibi plácita humilitáte profícere: ut, prava despíciens, quæcúmque recta sunt, libera exérceat caritáte. Per Dóminum \emph{\&c.}
}\switchcolumn\portugues{
\rlettrine{S}{enhor,} dignai-Vos conceder aos vossos fiéis, pela intercessão dos vossos Santos Mártires Vito, Modesto e Crescência, que não caiam em sentimentos de orgulho, mas pratiquem a humildade, que Vos é tão agradável, a fim de que, desprezando todo o mal, exerçam livremente a caridade em tudo quanto é bom. Por nosso Senhor \emph{\&c.}
}\end{paracol}

\paragraphinfo{Epístola}{Página \pageref{muitosmartires1}}

\paragraphinfo{Gradual}{Sl. 149, 5 \& 1}
\begin{paracol}{2}\latim{
\rlettrine{E}{xsultábunt} Sancti in glória: lætabúntur in cubílibus suis. ℣. Cantáte Dómino cánticum novum: laus ejus in ecclésia sanctórum.
}\switchcolumn\portugues{
\rlettrine{O}{s} Santos exultarão de alegria na sua glória: e rejubilarão nos lugares do seu repouso. ℣. Cantai ao Senhor um cântico novo: que a assembleia dos fiéis cante sempre os seus louvores.
}\switchcolumn*\latim{
Allelúja, allelúja. ℣. \emph{Ps. 144, 10-11} Sancti tui, Dómine, benedícent te: glóriam regni tui dicent. Allelúja.
}\switchcolumn\portugues{
Aleluia, aleluia. ℣. \emph{Sl. 144, 10-11} Que os vossos Santos Vos bendigam: que publiquem a glória do vosso reino. Aleluia.
}\end{paracol}

\paragraphinfo{Evangelho}{Lc. 10, 16-20}
\begin{paracol}{2}\latim{
\cruz Sequéntia sancti Evangélii secúndum Lucam.
}\switchcolumn\portugues{
\cruz Continuação do santo Evangelho segundo S. Lucas.
}\switchcolumn*\latim{
\blettrine{I}{n} illo témpore: Dixit Jesus discípulis suis: Qui vos audit, me audit: et qui vos spernit, me spernit. Qui autem me spernit, spernit eum, qui misit me. Revérsi sunt autem septuagínta duo cum gáudio, dicéntes: Dómine, étiam dæmónia subjiciúntur nobis in nómine tuo. Et ait illis: Vidébam sátanam sicut fulgur de cœlo cadéntem. Ecce, dedi vobis potestátem calcandi supra serpéntes et scorpiónes, et super omnem virtútem inimíci: et nihil vobis nocébit. Verúmtamen in hoc nolíte gaudére, quia spíritus vobis subjiciúntur: gaudéte autem, quod nómina vestra scripta sunt in cœlis.
}\switchcolumn\portugues{
\blettrine{N}{aquele} tempo disse Jesus aos seus discípulos: «Quem vos ouve, ouve-me a mim; e quem vos despreza, despreza-me a mim. Quem me despreza, despreza Aquele que me enviou». Ora os setenta e dous voltaram, depois, cheios de alegria, dizendo: «Senhor, pelo vosso nome até os demónios se submeteram a nós!». E Ele disse: «Eu via Satanás cair do céu, como um relâmpago. Eis que vos dei poder para calcar as serpentes, os escorpiões e ainda todo o poder do inimigo; e nada vos fará dano. Todavia não deveis alegrar-vos, porque os espíritos estão sujeitos a vós; mas porque os vossos nomes estão escritos nos céus».
}\end{paracol}

\paragraphinfo{Ofertório}{Página \pageref{muitosmartires1}}

\paragraph{Secreta}
\begin{paracol}{2}\latim{
\rlettrine{S}{icut} glóriam divínæ poténtiæ múnera pro Sanctis obláta testántur: sic nobis efféctum, Dómine, tuæ salvatiónis impéndant. Per Dóminum nostrum \emph{\&c.}
}\switchcolumn\portugues{
\rlettrine{A}{ssim} como os dons oferecidos em honra dos vossos Santos testemunham gloriosamente o poder divino, assim também, Senhor, permiti que eles nos alcancem a salvação. Por nosso Senhor \emph{\&c.}
}\end{paracol}

\paragraphinfo{Comúnio}{Sb. 3, 1-2 \& 3}
\begin{paracol}{2}\latim{
\qlettrine{J}{ustórum} ánimæ in manu Dei sunt, et non tanget illos torméntum malítiæ: visi sunt óculis insipiéntium mori: illi autem sunt in pace.
}\switchcolumn\portugues{
\rlettrine{A}{s} almas dos justos estão nas mãos de Deus e o tormento da malícia lhes não causará dano. Aos olhos dos insensatos pareciam mortos, mas estão na paz.
}\end{paracol}

\paragraph{Postcomúnio}
\begin{paracol}{2}\latim{
\rlettrine{R}{epléti,} Dómine, benedictióne sollémni: quǽsumus; ut, per intercessiónem sanctórum Mártyrum tuórum Viti, Modésti et Crescéntiæ, medicína sacraménti et corpóribus nostris prosit et méntibus. Per Dóminum \emph{\&c.}
}\switchcolumn\portugues{
\rlettrine{S}{aciados} com vossa bênção nesta solenidade, dignai-Vos conceder-nos, Senhor, pela intercessão dos vossos Santos Mártires Vito, Modesto e Crescência, que este vosso sacramento seja medicina proveitosa para as nossas almas e corpos. Por nosso Senhor \emph{\&c.}
}\end{paracol}

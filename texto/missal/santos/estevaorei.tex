\subsectioninfo{S. Estêvão, Rei e Conf.}{2 de Setembro}

\textit{Como na Missa Os justi, página \pageref{confessoresnaopontifices1}, excepto:}

\paragraph{Oração}
\begin{paracol}{2}\latim{
\rlettrine{C}{oncéde,} quǽsumus, Ecclésiæ tuæ, omnípotens Deus: ut beátum Stéphanum Confessórem tuum, quem regnántem in terris propagatórem hábuit, propugnatórem habére mereátur gloriósum in cœlis. Per Dóminum \emph{\&c.}
}\switchcolumn\portugues{
\rlettrine{C}{oncedei} à vossa Igreja, Vos imploramos, ó Deus omnipotente, que o B. Estêvão, vosso Confessor, que ela possuiu como seu propagador enquanto reinou na terra, seja agora seu defensor na glória dos céus. Por nosso Senhor \emph{\&c.}
}\end{paracol}

\paragraphinfo{Evangelho}{Página \pageref{saoluisrei}}

\paragraph{Secreta}
\begin{paracol}{2}\latim{
\rlettrine{R}{éspice,} quas offérimus, hóstias, omnípotens Deus: et præsta; ut, qui passiónis Dominicae mystéria celebrámus, imitémur quod ágimus. Per eúndem Dóminum \emph{\&c.}
}\switchcolumn\portugues{
\slettrine{Ó}{} Deus omnipotente, dignai-Vos olhar para estas hóstias, que Vos oferecemos, e permiti que, celebrando nós os mistérios da Paixão do Senhor, imitemos o que Vos apresentamos. Por nosso Senhor \emph{\&c.}
}\end{paracol}

\paragraph{Postcomúnio}
\begin{paracol}{2}\latim{
\rlettrine{P}{ræsta,} quǽsumus, omnípotens Deus: ut beáti Stephani Confessóris tui fidem cóngrua devotióne sectémur; qui, pro ejúsdem fídei dilatatióne, de terréno regno ad cœléstis regni glóriam méruit perveníre. Per Dóminum nostrum \emph{\&c.}
}\switchcolumn\portugues{
\rlettrine{C}{oncedei-nos,} Vos rogamos, ó Deus omnipotente, a graça de imitarmos com a devida devoção os exemplos de fé do B. Estêvão, vosso Confessor, o qual pela propagação desta mesma fé mereceu transitar da realeza terrestre para a glória do reino celestial. Por nosso Senhor \emph{\&c.}
}\end{paracol}

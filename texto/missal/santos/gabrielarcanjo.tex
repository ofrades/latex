\subsectioninfo{S. Gabriel, Arcanjo}{24 de Março}\label{gabrielarcanjo}

\paragraphinfo{Intróito}{Sl. 102, 20}
\begin{paracol}{2}\latim{
\rlettrine{B}{enedícite} Dóminum, omnes Angeli ejus: poténtes virtúte, qui fácitis verbum ejus, ad audiéndam vocem sermónum ejus. (T. P. Allelúja, allelúja.) \emph{Ps. ibid., 1} Bénedic, ánima mea, Dómino: et ómnia, quæ intra me sunt, nómini sancto ejus.
℣. Gloria Patri \emph{\&c.}
}\switchcolumn\portugues{
\rlettrine{B}{endizei} o Senhor, ó Anjos do Senhor: Sois cheios de poder e virtude: e fazeis o que Ele vos ordena, obedecendo às suas palavras e ordens. (T. P. Aleluia, aleluia). \emph{Sl. ibid., 1} Bendizei o Senhor, ó minha alma: que tudo quanto houver em mim bendiga o santo nome do Senhor.
℣. Glória ao Pai \emph{\&c.}
}\end{paracol}

\paragraph{Oração}
\begin{paracol}{2}\latim{
\rlettrine{D}{eus,} qui inter céteros Angelos, ad annuntiándum incarnatiónis tuæ mystérium, Gabriélem Archángelum elegísti: concéde propítius; ut, qui festum (commemoratiónem) ejus celebrámus in terris, ipsíus patrocínium sentiámus in cœlis: Qui vivis \emph{\&c.}
}\switchcolumn\portugues{
\slettrine{Ó}{} Deus, que entre os outros Anjos escolhestes o Arcanjo Gabriel para anunciar o mistério da vossa Incarnação, concedei-nos propício que, havendo celebrado a sua festa na terra, experimentemos o seu patrocínio no céu, Ó Vós, que, sendo Deus, viveis e \emph{\&c.}
}\end{paracol}

\paragraphinfo{Epístola}{Dn. 9, 21-26}
\begin{paracol}{2}\latim{
Léctio Daniélis Prophétæ.
}\switchcolumn\portugues{
Lição do Profeta Daniel.
}\switchcolumn*\latim{
\rlettrine{I}{n} diébus illis: Ecce, vir Gábriël, quem víderam m visióne a princípio, cito volans tétigit me in témpore sacrifícii vespertíni. Et dócuit me et locútus est mihi dixítque: Dániel, nunc egréssus sum, ut docérem te et intellégeres. Ab exórdio precum tuárum egréssus est sermo: ego autem veni, ut indicárem tibi, quia vir desideriórum es: tu ergo animadvérte sermónem et intéllege visiónem. Septuagínta hebdómades abbreviátæ sunt super pópulum tuum et super urbem sanctam tuam, ut consummétur prævaricátio, et finem accípiat peccatum, et deleátur iníquitas, et adducátur justítia sempitérna, et impleátur visio et prophetia, et ungátur Sanctus sanctórum. Scito ergo et animadvérte: Ab éxitu sermónis, ut íterum ædificétur Jerúsalem, usque ad Christum ducem, hebdómades septem et hebdómades sexagínta duæ erunt: et rursum ædificábitur platéa et muri in angustia temporum. Et post hebdómades sexagínta duas occidétur Christus: et non erit ejus pópulus, qui eum negatúrus est. Et civitátem et sanctuárium dissipábit populus cum duce ventúro: et finis ejus vástitas, et post finem belli statúta desolátio.
}\switchcolumn\portugues{
\rlettrine{N}{aqueles} dias: Eis que o varão Gabriel, que eu havia visto antes, em visão, veio junto de mim, voando rapidamente, na ocasião do sacrifício da tarde. Então, inspirou-me, falou-me e disse-me: «Daniel, eu vim, agora, para iluminar a tua inteligência. Desde o princípio das tuas preces foi proferida uma palavra. Ora, eu venho para que possas compreendê-la, pois és um varão favorecido por Deus. Medita, portanto, nessa palavra e compreende a visão. Setenta semanas estão marcadas para o teu povo e para a tua cidade, a fim de que a prevaricação seja consumada, o pecado cesse e seja destruída a iniquidade; e apareça a justiça eterna, cumpra-se a visão e a profecia e seja ungido o Santo dos Santos. Fica, pois, sabendo e atende bem: Desde que foi publicado o decreto para reedificar Jerusalém até Cristo-Rei, haverá sete semanas e sessenta e duas semanas; e, então, novamente, será reedificada com praças e muralhas, nos tempos angustiosos. Após sessenta e duas semanas, Cristo será morto e o povo que O negar não mais será considerado como seu povo. Um povo virá conduzido pelo seu chefe, que destruirá a cidade e o santuário; o seu fim será a devastação. Após o fim da guerra virá a desolação decretada».
}\end{paracol}

\paragraphinfo{Gradual}{Sl. 102, 20 \& 1}
\begin{paracol}{2}\latim{
\rlettrine{B}{enedícite} Dóminum, omnes Angeli ejus: poténtes virtúte, qui fácitis verbum ejus. ℣. Benedic, ánima mea, Dóminum, et ómnia interióra mea nomen sanctum ejus.
}\switchcolumn\portugues{
\rlettrine{B}{endizei} o Senhor, vós todos, que sois seus Anjos; ó vós, que sois poderosos e valorosos; e que executais as suas ordens. ℣. Bendizei o Senhor, ó minha alma: tudo o que há no meu íntimo bendiga o Senhor.
}\end{paracol}

\paragraphinfo{Trato}{Lc. 1, 28, 42, 31 \& 35}
\begin{paracol}{2}\latim{
\rlettrine{A}{ve,} María, grátia plena; Dóminus tecum. ℣. Benedícta tu in muliéribus: et benedíctus fructus ventris tui. ℣. Ecce, concípies et páries Fílium, et vocábis nomen ejus Emmánuel. ℣. Spíritus Sanctus supervéniet in te, et virtus Altíssimi obumbrábit tibi. ℣. Ideóque et quod nascétur ex te Sanctum, vocábitur Fílius Dei.
}\switchcolumn\portugues{
\rlettrine{A}{ve,} Maria, cheia de graça, o Senhor é convosco. ℣. Bendita sois vós entre as mulheres: e bendito é o fruto do vosso ventre. ℣. Eis que conceberás, darás à luz um Filho e o seu nome será Emanuel. ℣. O Espírito Santo descerá sobre vós e a virtude do Altíssimo far-vos-á conceber. ℣. E, por isso, o Santo que de vós nascer será chamado Filho de Deus.
}\end{paracol}

\textit{No T. Pascal omite-se o Gradual e o Trato e diz-se:}

\begin{paracol}{2}\latim{
Allelúja, allelúja. ℣. \emph{Ps. 103, 4} Qui facit Angelos suos spíritus: et minístros suos flammam ignis. Allelúja. ℣. \emph{Luc. 1, 28} Ave, María, grátia plena; Dóminus tecum: benedícta tu in muliéribus. Allelúja.
}\switchcolumn\portugues{
Aleluia, aleluia. ℣. \emph{Sl. 103, 4} Fazeis que os vossos Anjos sejam velozes como os ventos e que os vossos ministros sejam activos como a chama do fogo. Aleluia. ℣. \emph{Lc. 1, 28} Ave, Maria, cheia de graça: O Senhor é convosco: bendita sois vós entre as mulheres. Aleluia.
}\end{paracol}

\paragraphinfo{Evangelho}{Lc. 1, 26-38}
\begin{paracol}{2}\latim{
\cruz Sequéntia sancti Evangélii secúndum Lucam.
}\switchcolumn\portugues{
\cruz Continuação do santo Evangelho segundo S. Lucas.
}\switchcolumn*\latim{
\blettrine{I}{n} illo témpore: Missus est Angelus Gábriel a Deo in civitátem Galilǽæ, cui nomen Názareth, ad Vírginem desponsátam viro, cui nomen erat Joseph, de domo David, et nomen Vírginis María. Ei ingréssus Angelus ad eam, dixit: Ave, grátia plena; Dóminus tecum: benedícta tu in muliéribus. Quæ cum audísset, turbáta est in sermóne ejus: et cogitábat, qualis esset ista salutátio. Et ait Angelus ei: Ne tímeas, María, invenísti enim grátiam apud Deum: ecce, concípies in útero et páries fílium, et vocábis nomen ejus Jesum. Hic erit magnus, et Fílius Altíssimi vocábitur, et dabit illi Dóminus Deus sedem David, patris ejus: et regnábit in domo Jacob in ǽtérnum, et regni ejus non erit finis. Dixit autem María ad Angelum: Quómodo fiet istud, quóniam virum non cognósco? Et respóndens Angelus, dixit ei: Spíritus Sanctus supervéniet in te, et virtus Altíssimi obumbrábit tibi. Ideóque et quod nascétur ex te Sanctum, vocábitur Fílius Dei. Et ecce, Elísabeth, cognáta tua, et ipsa concépit fílium in senectúte sua: et hic mensis sextus est illi, quæ vocátur stérilis: quia non erit impossíbile apud Deum omne verbum. Dixit autem María: Ecce ancílla Dómini, fiat mihi secúndum verbum tuum.
}\switchcolumn\portugues{
\blettrine{N}{aquele} tempo, foi mandado por Deus o Anjo Gabriel a uma cidade da Galileia, chamada Nazaré, a uma Virgem, desposada com um varão, cujo nome era José, da casa de David; e o nome da Virgem era Maria. Entrando o Anjo onde ela estava, disse: «Eu te saúdo, cheia de graça: o Senhor é contigo: bendita és tu entre todas as mulheres». Ouvindo ela isto, perturbou-se, e pensava na significação desta saudação. Então, disse-lhe o Anjo: «Não temas, Maria, porquanto alcançaste graça diante do Senhor: eis que conceberás no teu seio, e darás à luz um Filho, e o seu nome será Jesus. Ele será grande e será chamado Filho do Altíssimo; o Senhor Deus Lhe dará o trono de David, seu pai; reinará eternamente na casa de Jacob; e o seu reino não terá fim». Porém Maria disse ao Anjo: «Como acontecerá isso, se não conheço varão?». O Anjo, respondendo, disse-lhe: «O Espírito Santo descerá sobre ti, e a virtude do Altíssimo te tocará com sua sombra. Por isso o Santo que nascer de ti será chamado Filho de Deus. E eis que Isabel, tua parenta, concebeu um filho na sua velhice: este é o sexto mês daquela que é chamada estéril: porque nada é impossível a Deus». Então disse Maria: «Eis aqui a escrava do Senhor, faça-se em mim segundo a tua palavra».
}\end{paracol}

\paragraphinfo{Ofertório}{Ap. 8, 3 \& 4}
\begin{paracol}{2}\latim{
\rlettrine{S}{tetit} Angelus juxta aram templi, habens thuríbulum áureum in manu sua, et data sunt ei incénsa multa: et ascéndit fumus aromátum in conspéctu Dei. (T. P. Allelúja.)
}\switchcolumn\portugues{
\qlettrine{J}{unto} ao altar, no templo, estava de pé um Anjo, tendo na mão um turíbulo de ouro: e deitava-lhe muito incenso, subindo o fumo dos perfumes à presença de Deus. (T. P. Aleluia.)
}\end{paracol}

\paragraph{Secreta}
\begin{paracol}{2}\latim{
\rlettrine{A}{ccéptum} fiat in conspéctu tuo, Dómine, nostræ servitútis munus, et beáti Archángeli Gabriélis orátio: ut, qui a nobis venerátur in terris, sit apud te pro nobis advocátus in cœlis. Per Dóminum nostrum \emph{\&c.}
}\switchcolumn\portugues{
\rlettrine{S}{enhor,} seja agradável a vossos olhos a oferta da nossa servidão, bem como a oração de B. Gabriel Arcanjo, a fim de que, venerando-o nós na terra, seja nosso advogado junto de Vós no céu. Por nosso Senhor \emph{\&c.}
}\end{paracol}

\paragraphinfo{Comúnio}{}
\begin{paracol}{2}\latim{
\rlettrine{B}{enedicite,} omnes Angeli Dómini, Dóminum: hymnum dícite et superexaltáte eum in sǽcula. (T. P. Allelúja.)
}\switchcolumn\portugues{
\rlettrine{A}{njos} todos do Senhor, louvai o Senhor. Cantai hinos em seu louvor e aclamai-O em todos os séculos. (T. P. Aleluia.)
}\end{paracol}

\paragraph{Postcomúnio}
\begin{paracol}{2}\latim{
\rlettrine{C}{órporis} tui et Sánguinis sumptis mystériis, tuam, Dómine, Deus noster, deprecámur cleméntiam: ut, sicut, Gabriéle nuntiánte, incarnatiónem tuam cognóvimus; ita, ipso adjuvante, incarnationis ejúsdem benefícia consequámur: Qui vivis \emph{\&c.}
}\switchcolumn\portugues{
\rlettrine{H}{avendo} já recebido os mistérios do vosso Corpo e Sangue, Senhor, nosso Deus, imploramos a vossa clemência, para que, assim como pela Anunciação do Arcanjo Gabriel conhecemos a vossa Incarnação, assim também com seu auxílio alcancemos os benefícios da mesma Incarnação. Ó Vós, que \emph{\&c.}
}\end{paracol}

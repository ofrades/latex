\subsectioninfo{S. Pedro Crisólogo, B. Conf. e Doutor}{4 de Novembro}

\textit{Como na Missa In médio Ecclésiae, página \pageref{doutores}, excepto:}

\paragraph{Oração}
\begin{paracol}{2}\latim{
\rlettrine{D}{eus,} qui beátum Petrum Chrysólogum Doctorem egrégium, divínitus præmonstrátum, ad regéndam et instruéndam Ecclésiam tuam éligi voluísti: præsta, quǽsumus; ut, quem Doctórem vitæ habúimus in terris, intercessórem habére mereámur in cœlis. Per Dóminum \emph{\&c.}
}\switchcolumn\portugues{
\slettrine{Ó}{} Deus, que para governar e instruir a vossa Igreja Vos dignastes escolher o B. Pedro Crisólogo, egrégio Doutor, o qual nos foi indicado por uma forma divina, concedei-nos, Vos imploramos, que assim como o tivemos como Doutor na terra, assim também mereçamos alcançar a sua intercessão nos céus. Por nosso Senhor \emph{\&c.}
}\end{paracol}

\paragraphinfo{Gradual}{Ecl. 44, 16}
\begin{paracol}{2}\latim{
\rlettrine{E}{cce} sacérdos magnus, qui in diébus suis plácuit Deo. ℣. \emph{ibid., 20} Non est invéntus símilis illi, qui conservaret legem Excélsi.
}\switchcolumn\portugues{
\rlettrine{E}{is} o grande sacerdote que nos dias da sua vida agradou a Deus. ℣. \emph{ibid., 20} Ninguém o igualou na observância das leis do Altíssimo.
}\switchcolumn*\latim{
Allelúja, allelúja. ℣. \emph{Ps. 109. 4} Tu es sacérdos in ætérnum, secúndum órdinem Melchísedech. Allelúja.
}\switchcolumn\portugues{
Aleluia, aleluia. ℣. \emph{Sl. 109. 4} Tu és sacerdote para sempre, segundo a ordem de Melquisedeque. Aleluia.
}\end{paracol}

\paragraphinfo{Comúnio}{Mt. 25, 20 \& 21}
\begin{paracol}{2}\latim{
\rlettrine{D}{ómine,} quinque talénta tradidísti mihi: ecce, ália quinque superlucrátus sum. Euge, serve bone et fidélis, quia in pauca fuísti fidélis, supra multa te constítuam, intra in gáudium Dómini tui.
}\switchcolumn\portugues{
\rlettrine{S}{enhor,} entregastes-me cinco talentos; eis aqui outros cinco que lucrei. «Está bem, servo bom e fiel; visto que foste fiel em pouca coisa, Eu te colocarei sobre muitas; entra na glória do teu Senhor».
}\end{paracol}
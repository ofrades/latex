\subsectioninfo{S. S. Inocentes, Mártires}{28 de Dezembro}

\paragraphinfo{Intróito}{Sl. 8, 3}
\begin{paracol}{2}\latim{
\rlettrine{E}{x} ore infántium, Deus, et lacténtium perfecísti laudem propter inimicos tuos. \emph{Ps. ib., 2} Dómine, Dóminus noster: quam admirábile est nomen tuum in univérsa terra!
℣. Gloria Patri \emph{\&c.}
}\switchcolumn\portugues{
\rlettrine{D}{a} boca dos meninos e das crianças de peito fazeis sair louvores em vossa honra, para confusão dos vossos inimigos. \emph{Sl. ib., 2} Ó Senhor, nosso Deus, como o vosso nome é admirável em todo o universo!
℣. Glória ao Pai \emph{\&c.}
}\end{paracol}

\paragraph{Oração}
\begin{paracol}{2}\latim{
\rlettrine{D}{eus,} cujus hodicrna die præcónium Innocéntes Mártyres non loquéndo, sed moriéndo conféssi sunt: ómnia in nobis vitiórum mala mortífica; ut fidem tuam, quam lingua nostra lóquitur, étiam móribus vita fateátur. Per Dóminum \emph{\&c.}
}\switchcolumn\portugues{
\slettrine{Ó}{} Deus, cuja glória os Inocentes Mártires confessaram neste dia, não com suas palavras, mas com sua morte, mortificai no nosso íntimo as paixões e os vícios, a fim de que a vossa fé, que a nossa boca confessa, seja Confirmada com os nossos costumes. Por nosso Senhor \emph{\&c.}
}\end{paracol}

\paragraphinfo{Epístola}{Ap. 14, 1-5}
\begin{paracol}{2}\latim{
Léctio libri Apocalýpsis beáti Joánnis Apóstoli.
}\switchcolumn\portugues{
Lição do Livro Apocalipse do B. Ap.º João.
}\switchcolumn*\latim{
\rlettrine{I}{n} diébus illis: Vidi supra montem Sion Agnum stantem, et cum eo centum quadragínta quatuor mília, habéntes nomen ejus, et nomen Patris ejus scriptum in fróntibus suis. Et audívi vocem de cœlo, tamquam vocem aquárum multárum, et tamquam vocem tonítrui magni: et vocem, quam audívi, sicut citharœrórum citharizántium in cítharis suis. Et cantábant quasi cánticum novum ante sedem, et ante quátuor animália, et senióres: et nemo póterat dícere cánticum, nisi illa centum quadragínta quátuor mília, qui empti sunt de terra. Hi sunt, qui cum muliéribus non sunt coinquináti: vírgines enim sunt. Hi sequúntur Agnum, quocúmque íerit. Hi empti sunt ex homínibus primítiæ Deo, et Agno: et in ore eórum non est invéntum mendácium: sine mácula enim sunt ante thronum Dei.
}\switchcolumn\portugues{
\rlettrine{N}{aqueles} dias, vi o Cordeiro de pé sobre a montanha de Sião e com ele cento e quarenta e quatro mil pessoas, que tinham o seu nome e o nome de seu Pai escritos na fronte, e ouvi uma voz do céu, semelhante ao murmúrio de muitas águas e ao ruído de um grande trovão. Essa voz era como que o som de muitas harpas dedilhadas por um coro de músicos! E cantavam um cântico novo diante do trono de Deus, em presença dos quatro animais e dos anciãos, não podendo ninguém mais cantar este cântico senão os cento e quarenta e quatro mil que haviam sido resgatados da terra. Estes são os que se não mancharam com mulheres, pois são virgens; estes são os que acompanham o Cordeiro a qualquer lugar onde Ele vá; estes são os que foram resgatados entre os homens, para serem oferecidos como primícias a Deus e ao Cordeiro, não tendo eles nunca procedido falsamente, pois foram julgados sem mancha diante do trono de Deus.
}\end{paracol}

\paragraphinfo{Gradual}{Sl. 123, 7-8}
\begin{paracol}{2}\latim{
\rlettrine{A}{nima} nostra, sicut passer, erépta est de láqueo venántium. ℣. Láqueus contrítus est, et nos liberáti sumus: adjutórium nostrum in nómine Dómini, qui fecit cœlum et terram.
}\switchcolumn\portugues{
\rlettrine{A}{} nossa alma escapou-se, como um pássaro do laço do caçador. O laço quebrou-se, e ficámos livres. ℣. A nossa esperança está no nome do Senhor, que criou o céu e a terra.
}\end{paracol}

\paragraphinfo{Trato}{Sl. 78, 3 \& 10}
\begin{paracol}{2}\latim{
\rlettrine{E}{ffudérunt} sánguinem Sanctórum, velut aquam, in circuitu Jerúsalem. ℣. Et non erat, qui sepelíret. ℣. Víndica, Dómine, sánguinem Sanctórum tuórum, qui effúsus est super terram.
}\switchcolumn\portugues{
\rlettrine{D}{erramaram} o sangue dos Santos, como água, em redor de Jerusalém. ℣. E não houve ninguém que os sepultasse. ℣. Vingai, Senhor, o sangue dos vossos Santos que correu pela terra.
}\end{paracol}

Ao Domingo suprime-se o Trato e diz-se:

\begin{paracol}{2}\latim{
Allelúja, allelúja. ℣. \emph{Ps. 112, 1} Laudáte, púeri, Dóminum, laudáte nomen Dómini. Allelúja.
}\switchcolumn\portugues{
Aleluia, aleluia. ℣. \emph{Sl. 112, 1} Ó meninos, louvai o Senhor; louvai o seu santo nome. Aleluia.
}\end{paracol}

\paragraphinfo{Evangelho}{Mt. 2, 13-18}
\begin{paracol}{2}\latim{
\cruz Sequéntia sancti Evangélii secúndum Matthǽum.
}\switchcolumn\portugues{
\cruz Continuação do santo Evangelho segundo S. Mateus.
}\switchcolumn*\latim{
\blettrine{I}{n} illo témpore: Angelus Dómini appáruit in somnis Joseph, dicens: Surge, et áccipe Púerum et Matrem ejus, et fuge in Ægýptum, et esto ibi, usque dum dicam tibi. Futúrum est enim, ut Heródes quærat Púerum ad perdéndum eum. Qui consúrgens accépit Púerum et Matrem ejus nocte, et secéssit in Ægýptum: et erat ibi usque ad óbitum Heródis: ut adimplerétur quod dictum est a Dómino per Prophétam dicéntem: Ex Ægýpto vocávi Fílium meum. Tunc Heródes videns, quóniam illúsus esset a Magis, irátus est valde, et mittens occídit omnes púeros, qui erant in Béthlehem et in ómnibus fínibus ejus, a bimátu et infra, secúndum tempus, quod exquisíerat a Magis. Tunc adimplétum est, quod dictum est per Jeremíam Prophetam dicéntem: Vox in Rama audíta est, plorátus et ululátus multus: Rachel plorans fílios suos, et nóluit consolári, quia non sunt.
}\switchcolumn\portugues{
\blettrine{N}{aquele} tempo, um Anjo do Senhor apareceu em sonhos a José, dizendo-lhe: «Ergue-te, toma o Menino e a sua Mãe e foge para o Egipto, onde ficarás até que novamente te avise; pois Herodes procura o Menino para O matar». Então José, ainda de noite, levantou-se e tomou o Menino, assim como sua Mãe, conduzindo-os para o Egipto, onde permaneceram até à morte de Herodes. E assim se cumpriu o que o Senhor anunciara pelo Profeta: «Chamei o meu Filho do Egipto». Entretanto, Herodes, vendo que os Magos o tinham iludido, encolerizou-se e mandou matar todos os meninos, que havia em Belém e nos arredores, desde a idade de dous anos para baixo, segundo o tempo que ele inquirira diligentemente dos Magos. Deste modo se cumpriu o que dissera o profeta Jeremias: «Uma voz se ouviu em Rama: muitos soluços e lamentações. É Raquel que chora os seus filhos; e ela não quer ser consolada, porque já não existem!».
}\end{paracol}

\paragraphinfo{Ofertório}{Sl. 123, 7}
\begin{paracol}{2}\latim{
\rlettrine{A}{nima} nostra, sicut passer, erépta est de láqueo venántium: láqueus contrítus est, et nos liberáti sumus.
}\switchcolumn\portugues{
\rlettrine{A}{} nossa alma escapou-se, como um pássaro do laço do caçador. O laço quebrou-se, e ficámos livres.
}\end{paracol}

\paragraph{Secreta}
\begin{paracol}{2}\latim{
\rlettrine{S}{anctórum} tuórum, Dómine, nobis pia non desit orátio: quæ et múnera nostra concíliet, et tuam nobis indulgéntiam semper obtíneat. Per Dóminum \emph{\&c.}
}\switchcolumn\portugues{
\rlettrine{N}{ão} nos falte, Senhor, a pia oração dos vossos Santos e que ela Vos torne agradáveis as nossas ofertas e sempre nos alcance a vossa indulgência. Por nosso Senhor \emph{\&c.}
}\end{paracol}

\paragraphinfo{Comúnio}{Mt. 2, 18}
\begin{paracol}{2}\latim{
\rlettrine{V}{ox} in Rama audíta est, plorátus, et ululátus: Rachel plorans fílios suos, et nóluit consolári, quia non sunt.
}\switchcolumn\portugues{
\rlettrine{U}{ma} voz se ouviu em Rama: muitos soluços e lamentações. É Raquel que chora os seus filhos; e ela não quer ser consolada, porque já não existem!
}\end{paracol}

\paragraph{Postcomúnio}
\begin{paracol}{2}\latim{
\rlettrine{V}{otíva,} Dómine, dona percépimus: quæ Sanctórum nobis précibus, et præséntis, quǽsumus, vitæ páriter et ætérnæ tríbue conférre subsídium. Per Dóminum \emph{\&c.}
}\switchcolumn\portugues{
\rlettrine{H}{avendo} nós participado dos dons que Vos oferecemos, dignai-Vos permitir que as orações dos vossos Santos nos sirvam de auxílio para a vida presente e para a futura. Por nosso Senhor \emph{\&c.}
}\end{paracol}

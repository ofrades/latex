\subsectioninfo{S. Cirilo de Jerusalém, B. Conf. e Doutor}{18 de Março}\label{cirilojerusalem}

\textit{Como na Missa In médio Ecclésiae, página \pageref{doutores}, excepto:}

\paragraph{Oração}
\begin{paracol}{2}\latim{
\rlettrine{D}{a} nobis, quǽsumus, omnípotens Deus, beáto Cyríllo Pontífice intercedénte: te solum verum Deum, et quem misísti Jesum Christum ita cognóscere; ut inter oves, quæ vocem ejus áudiunt, perpétuo connumerári mereámur. Per eúndem Dóminum nostrum \emph{\&c.}
}\switchcolumn\portugues{
\slettrine{Ó}{} Deus omnipotente, Vos suplicamos, permiti que, por intercessão do B. Pontífice Cirilo, conheçamos que sois o único e verdadeiro Deus e Aquele que enviastes ao mundo, Jesus Cristo, de tal sorte que mereçamos ser contados eternamente entre as ovelhas que escutam a vossa voz. Pelo mesmo \emph{\&c.}
}\end{paracol}

\paragraphinfo{Epístola}{Ecl. 39, 6-14}
\begin{paracol}{2}\latim{
Léctio libri Sapiéntiæ.
}\switchcolumn\portugues{
Lição do Livro da Sabedoria.
}\switchcolumn*\latim{
\qlettrine{J}{ustus} cor suum tradet ad vigilándum dilúculo ad Dóminum, qui fecit illum, et in conspéctu Altíssimi deprecábitur. Apériet os suum in oratióne, et pro delíctis suis deprecábitur. Si enim Dóminus magnus volúerit, spíritu intellegéntiæ replébit illum: et ipse tamquam imbres mittet elóquia sapiéntiæ suæ, et in oratióne confitébitur Dómino: et ipse díriget consílium ejus et disciplínam, et in abscónditis suis consiliábitur. Ipse palam fáciet disciplínam doctrínæ suæ, et in lege testaménti Dómini gloriábitur. Collaudábunt multi sapiéntiam ejus, et usque in sǽculum non delébitur. Non recédet memória ejus, et nomen ejus requirétur a generatióne in generatiónem. Sapiéntiam ejus enarrábunt gentes, et laudem ejus enuntiábit ecclésia.
}\switchcolumn\portugues{
\rlettrine{O}{} justo aplicará o seu coração e vigiará desde o romper do dia para se unir ao Senhor, que o criou, e oferecer as suas preces ao Altíssimo. Abrirá a sua boca para orar e implorar o perdão dos seus pecados; pois se o soberano Senhor quiser, enchê-lo-á com o espírito de inteligência. Então ele espalhará, como chuva, as palavras da sabedoria e abençoará o Senhor na sua oração. O Senhor inspirará os seus conselhos e instruções, e ele compreenderá os mistérios divinos. Publicará a doutrina que tiver aprendido, e a sua glória será manter-se na lei da aliança com o Senhor. Sua sabedoria receberá louvor de muitos e não cairá no esquecimento. Sua memória se não apagará. Seu nome será honrado de geração em geração. As nações publicarão a sua sabedoria e a Igreja anunciará os seus louvores.
}\end{paracol}

\paragraphinfo{Evangelho}{Mt. 10, 23-28}
\begin{paracol}{2}\latim{
\cruz Sequéntia sancti Evangélii secúndum Matthǽum.
}\switchcolumn\portugues{
\cruz Continuação do santo Evangelho segundo S. Mateus.
}\switchcolumn*\latim{
\blettrine{I}{n} illo témpore: Dixit Jesus discípulis suis: Cum persequéntur vos in civitáte ista, fúgite in áliam. Amen, dico vobis, non consummábitis civitátes Israël, donec véniat Fílius hóminis. Non est discípulus super magístrum nec servus super dóminum suum. Súfficit discípulo, ut sit sicut magíster ejus: et servo, sicut dóminus ejus. Si patremfamílias Beélzebub vocavérunt; quanto magis domésticos ejus? Ne ergo timuéritis eos. Nihil enim est opértum, quod non revelábitur: et occúltum, quod non sciétur. Quod dico vobis in ténebris, dícite in lúmine: et quod in aure audítis, prædicáte super tecta. Et nolíte timére eos, qui occídunt corpus, ánimam autem non possunt occídere: sed pótius timéte eum, qui potest et ánimam et corpus pérdere in gehénnam.
}\switchcolumn\portugues{
\blettrine{N}{aquele} tempo, disse Jesus aos seus discípulos: «Quando vos perseguirem em uma cidade, fugi para outra. Em verdade vos digo: não acabareis de percorrer as cidades de Israel sem que venha o Filho do homem. O discípulo não é superior ao mestre, nem o servo ao senhor. Basta ao discípulo ser como o mestre e ao servo como o senhor. Se chamaram Belzebute ao pai de família, o que não chamarão aos seus domésticos? Portanto, os não receeis; pois não há nada secreto que não seja revelado, nem oculto que não seja conhecido. Aquilo que vos digo nas trevas, dizei-o à luz; e o que vos digo ao ouvido, pregai-o sobre os tectos. E nunca temais aqueles que matam o corpo, mas não podem matar a alma; temei antes Aquele que pode condenar a alma e o corpo ao inferno».
}\end{paracol}

\paragraph{Secreta}
\begin{paracol}{2}\latim{
\rlettrine{R}{éspice,} Dómine, immaculátam hóstiam, quam tibi offérimus: et præsta; ut, méritis beáti Pontíficis et Confessóris tui Cyrílli, eam mundo corde suscípere studeámus. Per Dóminum \emph{\&c.}
}\switchcolumn\portugues{
\rlettrine{O}{lhai,} Senhor, para a Hóstia Imaculada que Vos oferecemos; e permiti que pelos méritos do B. Cirilo, vosso Confessor e Pontífice, diligenciemos recebê-la com o coração puro. Por nosso Senhor \emph{\&c.}
}\end{paracol}

\paragraph{Postcomúnio}
\begin{paracol}{2}\latim{
\rlettrine{S}{acraménta} Córporis et Sánguinis tui, quæ súmpsimus, Dómine Jesu Christe: beáti Cyrílli Pontíficis précibus, mentes et corda nostra sanctíficent; ut divínæ consórtes natúræ éffici mereámur: Qui vivis \emph{\&c.}
}\switchcolumn\portugues{
\slettrine{Ó}{} Senhor Jesus Cristo, fazei que os sacramentos do vosso Corpo e Sangue, que acabámos de receber, santifiquem, pelas preces do B. Cirilo, os nossos espíritos e os nossos corações, a fim de que mereçamos ser participantes da natureza divina. Ó Vós, que viveis e reinais \emph{\&c.}
}\end{paracol}

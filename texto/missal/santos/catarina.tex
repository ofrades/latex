\subsectioninfo{Santa Catarina de Sena, Virgem}{30 de Abril}

\textit{Como na Missa Dilexísti justitiam, página \pageref{virgemnaomartir1}, excepto:}

\paragraph{Oração}
\begin{paracol}{2}\latim{
\rlettrine{D}{a,} quǽsumus, omnípotens Deus: ut, qui beátæ Catharínæ Vírginis tuæ natalítia cólimus; et ánnua sollemnitáte lætámur, et tantæ virtútis proficiámus exémplo. Per Dóminum \emph{\&c.}
}\switchcolumn\portugues{
\rlettrine{P}{ermiti,} ó Deus omnipotente, Vos rogamos, que, honrando nós o nascimento no céu da B. Catarina, vossa Virgem, nos alegremos nesta solenidade anual e aproveitemos com o exemplo de tão grande virtude. Por nosso Senhor \emph{\&c.}
}\end{paracol}

\paragraph{Secreta}
\begin{paracol}{2}\latim{
\rlettrine{A}{céndant} ad te, Dómine, quas in beátæ Catharínæ sollemnitáte offérimus, preces, et hóstia salutáris, virgíneo fragrans odóre. Per Dóminum \emph{\&c.}
}\switchcolumn\portugues{
\rlettrine{D}{eixai} subir até Vós, Senhor, as preces que Vos oferecemos na solenidade da B. Catarina, e aceitai também esta salutar hóstia, perfumada com seu virginal odor. Por nosso Senhor \emph{\&c.}
}\end{paracol}

\paragraph{Postcomúnio}
\begin{paracol}{2}\latim{
\rlettrine{Æ}{ternitátem} nobis, Dómine, cónferat, qua pasti sumus, mensa cœléstis: quæ beátæ Catharinæ Vírginis vitam étiam áluit temporálem. Per Dóminum \emph{\&c.}
}\switchcolumn\portugues{
\qlettrine{Q}{ue} este celestial banquete em que nos alimentámos, Senhor, nos comunique a vida eterna, como ele alentou também a vida temporal da B. Virgem Catarina. Por nosso Senhor \emph{\&c.}
}\end{paracol}

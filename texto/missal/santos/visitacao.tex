\subsectioninfo{Visitação da B. V. Maria}{2 de Julho}\label{visitacao}

\paragraphinfo{Intróito}{Sedulius}
\begin{paracol}{2}\latim{
\rlettrine{S}{alve,} sancta Parens, eníxa puérpera Regem: qui cælum terrámque regit in sǽcula sæculórum. \emph{Sl. 44, 2} Eructávit cor meum verbum bonum: dico ego ópera mea Regi.
℣. Gloria Patri \emph{\&c.}
}\switchcolumn\portugues{
\rlettrine{S}{alve,} ó Santa Virgem Maria, em cujo seio foi gerado o Rei que governa o céu e a terra, em todos os séculos dos séculos. Meu coração exprimiu uma excelente palavra: Consagro ao Rei as minhas obras!
℣. Glória ao Pai \emph{\&c.}
}\end{paracol}

\paragraph{Oração}
\begin{paracol}{2}\latim{
\rlettrine{F}{ámulis} tuis, quǽsumus, Dómine, cœléstis grátiæ munus impertíre: ut, quibus beátæ Vírginis partus éxstitit salútis exórdium; Visitatiónis ejus votiva sollémnitas, pacis tríbuat increméntum. Per Dóminum \emph{\&c.}
}\switchcolumn\portugues{
\rlettrine{S}{enhor,} Vos suplicamos, concedei aos vossos servos o dom da graça celestial; e, como o parto da B. Virgem foi o início da salvação, permiti que a piedosa solenidade da Visitação lhes proporcione aumento de paz. Por nosso Senhor \emph{\&c.}
}\end{paracol}

\paragraphinfo{Segunda Oração}{S. S. Mártires}
\begin{paracol}{2}\latim{
\rlettrine{D}{eus,} qui nos sanctórum Mártyrum tuórum Proéssi et Martiniáni gloriósis confessiónibus circúmdas et prótegis: da nobis et eórum imitatióne profícere, et intercessióne gaudére. Per Dóminum nostrum \emph{\&c.}
}\switchcolumn\portugues{
\slettrine{Ó}{} Deus, que pelas gloriosas profissões de fé dos vossos Santos Mártires Processo e Martiniano nos defendeis e protegeis, concedei-nos a graça de aproveitarmos com seus exemplos e de nos alegrarmos com sua intercessão. Por nosso Senhor \emph{\&c.}
}\end{paracol}

\paragraphinfo{Epístola}{Ct. 2, 8-14}
\begin{paracol}{2}\latim{
Léctio libri Sapiéntiæ.
}\switchcolumn\portugues{
Lição do Livro da Sabedoria.
}\switchcolumn*\latim{
\rlettrine{E}{cce,} iste venit sáliens in móntibus, transíliens colles; símilis est diléctus meus cápreæ hinnulóque cervórum. En, ipse stat post paríetem nostrum, respíciens per fenéstras, prospíciens per cancéllos. En, diléctus meus lóquitur mihi: Surge, própera, amíca mea, colúmba mea, formósa mea, et veni. Jam enim hiems tránsiit, imber ábiit et recéssit. Flores apparuérunt in terra nostra, tempus putatiónis advénit: vox túrturis audíta est in terra nostra: ficus prótulit grossos suos: víneæ floréntes dedérunt odórem suum. Surge, amíca mea, speciósa mea, et veni: colúmba mea in foramínibus petra, in cavérna macériæ, osténde mihi fáciem tuam, sonet vox tua in áuribus meis: vox enim tua dulcis et fácies tua decóra.
}\switchcolumn\portugues{
\rlettrine{E}{is} que ele vem, galgando montes e transpondo outeiros! Meu amado é semelhante ao gamo e ao filho das corças. Eis que ele vem por detrás da nossa parede, olhando pelas janelas e espreitando pelas frestas. E o meu amado fala-me e diz: «Ergue-te, apressa-te e vem, ó minha amiga, ó minha pomba, ó minha única beleza! Pois já o Inverno acabou; já as chuvas cessaram e se retiraram. As flores brotaram nos nossos jardins; chegou o tempo da poda; ouve-se a voz da rola nos nossos campos; a figueira começa a mostrar os primeiros frutos e as vinhas em flor exalam aromas! Ergue-te e vem, minha amiga, minha única beleza! Ó minha pomba escondida nas fendas das rochas e nas cavernas dos muros em ruínas, mostra-me a tua face e soe tua voz nos meus ouvidos. A tua voz é doce e a tua face graciosa!».
}\end{paracol}

\paragraph{Gradual}
\begin{paracol}{2}\latim{
\rlettrine{B}{enedícta} et venerábilis es, Virgo María: quæ sine tactu pudóris invénta es Mater Salvatóris. ℣. Virgo, Dei Génetrix, quem totus non capit orbis, in tua se clausit víscera factus homo.
}\switchcolumn\portugues{
\rlettrine{B}{endita} e venerável sois, ó Virgem Maria, que fostes Mãe do Salvador sem a vossa pureza sofrer a mais leve mancha. ℣. Ó Virgem, Mãe de Deus, Aquele a quem todo o universo é incapaz de conter, esteve encerrado no vosso seio, quando se fez homem.
}\switchcolumn*\latim{
Allelúja, allelúja. ℣. Felix es, sacra Virgo María, et omni laude digníssima: quia ex te ortus est sol justítiæ, Christus, Deus noster. Allelúja.
}\switchcolumn\portugues{
Aleluia, aleluia. ℣. Bem-aventurada sois, ó santa Virgem Maria: e digna de todos os louvores, pois de vós nasceu o sol da justiça, Cristo, nosso Deus. Aleluia.
}\end{paracol}

\paragraphinfo{Evangelho}{Lc. 1, 39-47}
\begin{paracol}{2}\latim{
\cruz Sequéntia sancti Evangélii secúndum Lucam.
}\switchcolumn\portugues{
\cruz Continuação do santo Evangelho segundo S. Lucas.
}\switchcolumn*\latim{
\blettrine{I}{n} illo témpore: Exsúrgens María ábiit in montána cum festinatióne in civitátem Juda: et intrávit in domum Zacharíæ et salutávit Elísabeth. Et factum est, ut audivit salutatiónem Maríæ Elísabeth, exsultávit infans in útero ejus: et repléta est Spíritu Sancto Elísabeth, et exclamávit voce magna et dixit: Benedícta tu inter mulíeres, et benedíctus fructus ventris tui. Et unde hoc mihi, ut véniat Mater Dómini mei ad me? Ecce enim, ut facta est vox salutatiónis tuæ in áuribus meis, exsultávit in gáudio infans in útero meo. Et beáta, quæ credidísti, quóniam perficiéntur ea, quæ dicta sunt tibi a Dómino. Et ait María: Magníficat ánima mea Dóminum: et exsultávit spíritus meus in Deo, salutári meo.
}\switchcolumn\portugues{
\blettrine{N}{aquele} tempo, levantando-se Maria, foi apressadamente às montanhas de uma cidade de Judá. Aí entrou em casa de Zacarias e saudou Isabel. Logo que Isabel ouviu a saudação de Maria, saltou a criança no seu seio e Isabel ficou cheia do Espírito Santo, exclamando em voz alta: «Bendita sois vós entre todas as mulheres; e bendito é o fruto do vosso ventre. Donde me vem a mim que a Mãe do meu Senhor venha até mim? Pois, desde que a voz da vossa saudação chegou a meus ouvidos, o meu filho exultou de alegria no meu seio! Bem-aventurada sois, porque acreditastes que se hão-de cumprir as cousas que vos foram ditas da parte do Senhor». Maria disse então: «Minha alma glorifica o Senhor: e o meu espírito se alegra em Deus, meu Salvador».
}\end{paracol}

\paragraph{Ofertório}
\begin{paracol}{2}\latim{
\rlettrine{B}{eáta} es, Virgo María, quæ ómnium portásti Creatórem: genuísti, qui te fecit, et in ætérnum pérmanes Virgo.
}\switchcolumn\portugues{
\rlettrine{B}{em-aventurada} és, ó Virgem Maria, pois trouxestes no vosso seio o Criador de todas as cousas. Gerastes Aquele que vos criou: e permaneceis eternamente Virgem.
}\end{paracol}

\paragraph{Secreta}
\begin{paracol}{2}\latim{
\rlettrine{U}{nigéniti} tui, Dómine, nobis succúrrat humánitas: ut, qui, natus de Vírgine, Matris integritátem non mínuit, sed sacrávit; in Visitatiónis ejus sollémniis, nostris nos piáculis éxuens, oblatiónem nostram tibi fáciat accéptam Jesus Christus, Dóminus noster: Qui tecum vivit \emph{\&c.}
}\switchcolumn\portugues{
\qlettrine{Q}{ue} a humanidade do vosso Filho Unigénito, Senhor, nos socorra; e, assim como Ele, nascendo de uma Virgem, não alterou a pureza de sua mãe mas antes a consagrou, assim também, neste dia solene da sua Visitação, desonerando-nos nosso Senhor Jesus Cristo das nossas faltas, Vos torne agradável à nossa oferta. Ele, que, sendo Deus \emph{\&c.}
}\end{paracol}

\paragraphinfo{Segunda Secreta}{S. S. Mártires}
\begin{paracol}{2}\latim{
\rlettrine{S}{úscipe,} Dómine, preces et múnera: quæ ut tuo sint digna conspéctu. Sanctórum tuórum précibus adjuvémur. Per Dóminum \emph{\&c.}
}\switchcolumn\portugues{
\rlettrine{A}{ceitai,} Senhor, as nossas preces e oblatas; e para que elas se tornem dignas de Vos serem apresentadas, fazei que nos auxiliem as preces dos vossos Santos. Por nosso Senhor \emph{\&c.}
}\end{paracol}

\paragraph{Comúnio}
\begin{paracol}{2}\latim{
\rlettrine{B}{eáta} víscera Maríæ Vírginis, quæ portavérunt ætérni Patris Fílium.
}\switchcolumn\portugues{
\rlettrine{B}{em-aventuradas} as entranhas da Virgem Maria, onde esteve encerrado o Filho do Pai Eterno.
}\end{paracol}

\paragraph{Postcomúnio}
\begin{paracol}{2}\latim{
\rlettrine{S}{úmpsimus,} Dómine, celebritátis ánnuæ votiva sacraménta: præsta, quǽsumus; ut et temporális vitæ nobis remédia prǽbeant et ætérnæ. Per Dóminum \emph{\&c.}
}\switchcolumn\portugues{
\rlettrine{R}{ecebemos,} Senhor, os mystérios que Vos são apresentados nesta festa anual; e dignai-Vos permitir, Vos suplicamos, que nos sirvam de remédio durante o tempo presente e na eternidade. Por nosso Senhor \emph{\&c.}
}\end{paracol}

\paragraphinfo{Segundo Postcomúnio}{S. S. Mártires}
\begin{paracol}{2}\latim{
\rlettrine{C}{órporis} sacri et pretiósi Sánguinis repléti libámine, quǽsumus, Dómine, Deus noster: ut, quod pia devotióne gérimus, certa redemptióne capiámus. Per eúndem Dóminum \emph{\&c.}
}\switchcolumn\portugues{
\rlettrine{C}{umulados} de bens pelo sacrifício do sagrado Corpo e do Sangue precioso do Salvador, Vos suplicamos, ó Senhor, nosso Deus, fazei que os mystérios, que com piedade recebemos, nos assegurem os frutos da redenção. Por nosso Senhor \emph{\&c.}
}\end{paracol}

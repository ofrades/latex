\subsectioninfo{Santa Inês, Virgem e Márt.}{21 de Janeiro}

\textit{Como na Missa Loquébar, página \pageref{virgensmartires1}, excepto:}

\paragraphinfo{Intróito}{Sl. 118, 95-96}
\begin{paracol}{2}\latim{
\rlettrine{M}{e} exspectavérunt peccatores, ut pérderent me: testimónia tua, Dómine, intelléxi: omnis consummatiónis vidi finem: latum mandátum tuum nimis. \emph{Ps. ibid., 1} Beáti immaculáti in via: qui ámbulant in lege Dómini. 
℣. Gloria Patri \emph{\&c.}
}\switchcolumn\portugues{
\rlettrine{E}{speraram-me} os pecadores para me perder; mas, Senhor, tinha meditado nos vossos testemunhos. Encontrei limites em tudo quanto existe: só o vosso poder é infinito. \emph{Sl. ibid., 1} Bem-aventurados os que são imaculados em seus caminhos e que cumprem a Lei do Senhor.
℣. Glória ao Pai \emph{\&c.}
}\end{paracol}

\paragraph{Oração}
\begin{paracol}{2}\latim{
\rlettrine{O}{mnipotens} sempitérne Deus, qui infírma mundi éligis, ut fórtia quæque confúndas: concéde propítius; ut, qui beátæ Agnétis Vírginis et Mártyris tuæ sollémnia cólimus, ejus apud te patrocínia sentiámus. Per Dóminum \emph{\&c.}
}\switchcolumn\portugues{
\slettrine{Ó}{} Deus omnipotente e eterno, que escolhestes os fracos para confundir os fortes, concedei-nos benigno que, celebrando a solenidade da B. Inês, vossa Virgem e Mártir, gozemos a sua protecção junto do vosso trono. Por nosso Senhor \emph{\&c.}
}\end{paracol}

\paragraphinfo{Gradual}{Sl. 44, 3}
\begin{paracol}{2}\latim{
\rlettrine{D}{iffúsa} est grátia in lábiis tuis: proptérea 
benedíxit te Deus in ætérnum. ℣. \emph{ibid., 5} Propter veritátem et mansuetúdinem et justítiam: et dedúcet te mirabíliter déxtera tua.
}\switchcolumn\portugues{
\rlettrine{A}{} graça espalhou-se nos vossos lábios; por isso Deus vos abençoou para a eternidade. ℣. \emph{ibid., 5} Por amor da verdade, da mansidão e da justiça a vossa mão direita vos levará a praticar maravilhas. 
}\switchcolumn*\latim{
Allelúja, allelúja. ℣. \emph{Matth. 25, 4 \& 6} Quinque prudéntes vírgines accepérunt óleum in vasis suis cum lampádibus: média autem nocte clamor factus est: Ecce, sponsus venit: exíte óbviam Christo Dómino. Allelúja.
}\switchcolumn\portugues{
Aleluia, aleluia. ℣. \emph{Mt. 25, 4 \& 6} As cinco virgens prudentes tomaram azeite em seus vasos para suas lâmpadas. À meia-noite ouviu-se um clamor dizer: «Eis que chega o esposo: ide ao encontro de Cristo Senhor». Aleluia.
}\end{paracol}

\paragraph{Secreta}
\begin{paracol}{2}\latim{
\rlettrine{H}{óstias,} Dómine, quas tibi offérimus, propítius súscipe: et, intercedénte beáta Agnéte Vírgine et Mártyre tua, víncula peccatórum nostrórum absólve. Per Dóminum nostrum \emph{\&c.}
}\switchcolumn\portugues{
\rlettrine{R}{ecebei} benigno, Senhor, as hóstias que Vos oferecemos; e, por intercessão da B. Inês, vossa Virgem e Mártir, dignai-Vos quebrar as cadeias dos nossos pecados. Por nosso Senhor \emph{\&c.}
}\end{paracol}

\paragraphinfo{Comúnio}{Mt. 25, 4 \& 6}
\begin{paracol}{2}\latim{
\qlettrine{Q}{uinque} prudéntes vírgines accepérunt óleum in vasis suis cum lampádibus: média autem nocte clamor factus est: Ecce, sponsus venit: exíte óbviam Christo Dómino. 
}\switchcolumn\portugues{
\rlettrine{A}{s} cinco virgens prudentes tomaram azeite em seus vasos para suas lâmpadas. À meia-noite ouviu-se um clamor dizer: «Eis o esposo que chega: ide ao encontro de Cristo Senhor».
}\end{paracol}

\paragraph{Postcomúnio}
\begin{paracol}{2}\latim{
\rlettrine{R}{efécti} cibo potúque cœlésti. Deus noster, te súpplices exorámus: ut, in cujus hæc commemoratióne percépimus, ejus muniámur et précibus. Per Dóminum \emph{\&c.}
}\switchcolumn\portugues{
\rlettrine{C}{onfortados} já com o alimento e a bebida celestiais, ó Deus, Vos suplicamos, fazei que aquela em cuja memória os recebemos nos proteja com suas preces. Por nosso Senhor \emph{\&c.}
}\end{paracol}
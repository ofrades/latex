\subsectioninfo{Arcanjo Rafael}{24 de Outubro}

\paragraphinfo{Intróito}{Página \pageref{aparicaoarcanjomiguel}}

\paragraph{Oração}
\begin{paracol}{2}\latim{
\rlettrine{D}{eus,} qui beátum Raphǽlem Archángelum Tobíæ fámulo tuo cómitem dedísti in via: concéde nobis fámulis tuis; ut ejúsdem semper protegámur custódia et muniámur auxílio. Per Dóminum \emph{\&c.}
}\switchcolumn\portugues{
\slettrine{Ó}{} Deus, que designastes o Arcanjo Rafael para guia do vosso servo Tobias, concedei-nos a nós, que somos vossos servos, a graça de sermos sempre protegidos por ele e fortalecidos com seu auxílio. Por nosso Senhor \emph{\&c.}
}\end{paracol}

\paragraphinfo{Epístola}{Tb. 12, 7-15}
\begin{paracol}{2}\latim{
Léctio libri Tobíæ.
}\switchcolumn\portugues{
Lição do Livro de Tobias.
}\switchcolumn*\latim{
\rlettrine{I}{n} diébus illis: Dixit Angelus Ráphaël ad Tobíam: Sacraméntum regis abscóndere bonum est: ópera autem Dei reveláre et confitéri honoríficum est. Bona est orátio cum jejúnio, et eleemósyna magis quam thesáuros auri recóndere: quóniam eleemósyna a morte liberat, et ipsa est, quæ purgat peccáta et facit invenire misericórdiam et vitam ætérnam. Qui autem faciunt peccátum et iniquitátem, hostes sunt ánimæ suæ. Manifésto ergo vobis veritátem, et non abscóndam a vobis occúltum sermónem. Quando orábas cum lácrimis, et sepeliébas mórtuos, et derelinquébas prándium tuum, et mórtuos abscondébas per diem in domo tua, et nocte sepeliébas eos, ego óbtuli oratiónem tuam Dómino. Et quia accéptus eras Deo, necésse fuit, ut tentátio probáret te. Et nunc misit me Dóminus, ut curárem te, et Saram uxórem fílii tui a dæmónio liberárem. Ego enim sum Raphaël Angelus, unus ex septem, qui astámus ante Dóminum.
}\switchcolumn\portugues{
\rlettrine{N}{aqueles} dias, disse o Anjo Rafael a Tobias: «É bom ocultar o segredo de um rei, mas é honroso descobrir e publicar as obras de Deus. A oração, acompanhada com o jejum e com a esmola, vale mais do que os tesouros que se podem juntar; pois a esmola livra da morte, apaga os pecados e faz alcançar a misericórdia e a vida eterna. Aqueles, porém, que cometem o pecado e a iniquidade são inimigos de suas almas. Quero, pois, descobrir-vos a verdade e não vos ocultar uma só cousa secreta: Quando oráveis com lágrimas; sepultáveis os mortos; deixáveis o vosso descanso; guardáveis os mortos em vossa casa de dia e os sepultáveis de noite: então ofereci ao Senhor as vossas orações. E porque éreis agradável a Deus, foi necessário que a tentação vos experimentasse. Mas logo o Senhor me mandou para vos curar e para do demónio livrar Sara, mulher do vosso filho. Pois sou o Anjo Rafael, um dos sete que estão sempre presentes, diante do Senhor».
}\end{paracol}

\paragraphinfo{Gradual}{Tb. 8, 3}
\begin{paracol}{2}\latim{
\rlettrine{A}{ngelus} Dómini Raphaël apprehéndit et ligávit dǽmonem. ℣. \emph{Ps. 146, 5} Magnus Dóminus noster, et magna virtus ejus.
}\switchcolumn\portugues{
\rlettrine{R}{afael,} o Anjo do Senhor, dominou o demónio e amarrou-o. ℣. \emph{Sl. 146, 5} Grande é o Senhor, nosso Deus; grande é o seu poder.
}\switchcolumn*\latim{
Allelúja, allelúja. ℣. \emph{Ps. 137, 1-2} In conspéctu Angelórum psallam tibi: adorábo ad templum sanctum tuum, et confitébor nómini tuo, Dómine. Allelúja.
}\switchcolumn\portugues{
Aleluia, aleluia. ℣. \emph{Sl. 137, 1-2} Cantarei Salmos na presença dos Anjos; adorar-Vos-ei no vosso santo templo e glorificarei o vosso santo nome, ó Senhor. Aleluia.
}\end{paracol}

\paragraphinfo{Evangelho}{Jo. 5, 1-4}
\begin{paracol}{2}\latim{
\cruz Sequéntia sancti Evangélii secúndum Joánnem.
}\switchcolumn\portugues{
\cruz Continuação do santo Evangelho segundo S. João.
}\switchcolumn*\latim{
\blettrine{I}{n} illo témpore: Erat dies festus Judæórum, et ascéndit Jesus Jerosólymam. Est autem Jerosólymis Probática piscína, quæ cognominátur hebráice Bethsaida, quinque pórticus habens. In his jacébat multitúdo magna languéntium, cæcórum, claudórum, aridórum exspectántium aquæ motum. Angelus autem Dómini descendébat secúndum tempus in piscínam, et movebátur aqua. Et, qui prior descendísset in piscínam post motiónem aquæ, sanus fiebat, a quacúmque detinebátur infirmitáte.
}\switchcolumn\portugues{
\blettrine{N}{aquele} tempo, sendo o dia da festa dos judeus, Jesus subiu até Jerusalém. Ora há perto de Jerusalém uma piscina que se chama em hebreu Betsaida, e tem cinco alpendres, debaixo dos quais costumava estar deitada grande multidão de enfermos: cegos, coxos e paralíticos, à espera de que a água se movesse; pois um Anjo do Senhor descia de tempos a tempos à piscina, revolvia a água e o primeiro que descia à piscina, depois do movimento da água, ficava curado de qualquer enfermidade.
}\end{paracol}

\paragraphinfo{Ofertório}{Ap. 8,3 \& 4}
\begin{paracol}{2}\latim{
\rlettrine{S}{tetit} Angelus juxta aram templi, habens thuríbulum áureum in manu sua, et data sunt ei incénsa multa: et ascéndit fumus aromátum in conspéctu Dei.
}\switchcolumn\portugues{
\qlettrine{J}{unto} ao altar, no templo, estava de pé um Anjo, tendo na mão um turíbulo de ouro: e deitaram-lhe muito incenso, subindo o fumo dos perfumes à presença de Deus.
}\end{paracol}

\paragraph{Secreta}
\begin{paracol}{2}\latim{
\rlettrine{H}{óstias} tibi, Dómine, laudis offérimus, supplíciter deprecántes: ut eásdem, angélico pro nobis interveniénte suffrágio, et placátus accípias, et ad salútem nostram proveníre concédas. Per Dóminum \emph{\&c.}
}\switchcolumn\portugues{
\rlettrine{S}{enhor,} Vos oferecemos estas hóstias de louvor, suplicando-Vos humildemente que, por intercessão do santo Anjo, as aceiteis propício e nos concedais que sejam úteis à nossa salvação. Por nosso Senhor \emph{\&c.}
}\end{paracol}

\paragraphinfo{Comúnio}{Dn. 3, 58}
\begin{paracol}{2}\latim{
\rlettrine{B}{enedícite,} omnes Angeli Dómini, Dóminum: hymnum dícite et superexaltáte eum in sǽcula.
}\switchcolumn\portugues{
\rlettrine{A}{njos} do Senhor, bendizei todos o Senhor: cantai hinos em seu louvor e exaltai-O em todos os séculos.
}\end{paracol}

\paragraph{Postcomúnio}
\begin{paracol}{2}\latim{
\rlettrine{D}{irigere} dignáre, Dómine Deus, in adjutórium nostrum sanctum Raphǽlem Archángelum: et, quem tuæ majestáti semper assístere crédimus, tibi nostras exíguas preces benedicéndas assígnet. Per Dóminum \emph{\&c.}
}\switchcolumn\portugues{
\slettrine{Ó}{} Senhor, nosso Deus, dignai-Vos mandar para nosso guarda o santo Arcanjo Rafael; e que as nossas humildes preces Vos sejam apresentadas para serem abençoadas por aquele que sabemos estar sempre na presença de vossa majestade. Por nosso Senhor \emph{\&c.}
}\end{paracol}

\subsectioninfo{Natal de N. S. Jesus Cristo}{no dia da Festa}

\paragraphinfo{Intróito}{Is. 9, 6}

\begin{paracol}{2}\latim{
\rlettrine{P}{uer} natus est nobis, et fílius datus est nobis: cujus impérium super húmerum ejus: et vocábitur nomen ejus magni consílii Angelus. \emph{Ps. 97, 1} Cantáte Dómino cánticum novum, quia mirabília fecit.
℣. Gloria Patri \emph{\&c.}
}\switchcolumn\portugues{
\rlettrine{N}{asceu} para nós um Menino e foi-nos dado um Filho, que traz sobre os seus ombros o manto da realeza, o qual será chamado «Anjo do admirável conselho». \emph{Sl. 97, 1} Cantai ao Senhor um hino novo, pois Ele praticou maravilhas.
℣. Glória ao Pai \emph{\&c.}
}\end{paracol}

\paragraph{Oração}

\begin{paracol}{2}\latim{
\rlettrine{C}{oncéde,} quǽsumus, omnípotens Deus: ut nos Unigéniti tui nova per carnem Natívitas líberet; quos sub peccáti jugo vetústa sérvitus tenet. Per eúndem Dóminum \emph{\&c.}
}\switchcolumn\portugues{
\slettrine{Ó}{} Deus omnipotente, Vos imploramos, dignai-Vos permitir que o novo Nascimento do vosso Filho Unigénito Incarnado nos livre do antigo cativeiro em que nos conserva o jugo do pecado. Pelo mesmo nosso Senhor \emph{\&c.}
}\end{paracol}

\paragraphinfo{Epístola}{Heb. 1, 1-12}

\begin{paracol}{2}\latim{
Léctio Epístolæ beáti Pauli Apóstoli ad Hebrǽos.
}\switchcolumn\portugues{
Lição da Ep.ª do B. Ap.º Paulo aos Hebreus.
}\switchcolumn*\latim{
\rlettrine{M}{ultifáriam,} multísque modis olim Deus loquens pátribus in Prophétis: novíssime diébus istis locútus est nobis in Fílio, quem constítuit herédem universórum, per quem fecit et sǽcula: qui cum sit splendor glóriæ, et figúra substántia? ejus, portánsque ómnia verbo virtútis suæ, purgatiónem peccatórum fáciens, sedet ad déxteram majestátis in excélsis: tanto mélior Angelis efféctus, quanto differéntius præ illis nomen hereditávit. Cui enim dixit aliquándo Ange
lórum: Fílius meus es tu, ego hódie génui te? Et rursum: Ego ero illi in patrem, et ipse erit mihi in fílium? Et cum íterum introdúcit Primogénitum in orbem terræ, dicit: Et adórent eum omnes Angeli Dei. Et ad Angelos quidem dicit: Qui facit Angelos suos spíritus, et minístros suos flammam ignis. Ad Fílium autem: Thronus tuus, Deus, in sǽculum sǽculi: virga æquitátis, virga regni tui. Dilexísti justítiam et odísti iniquitátem: proptérea unxit te Deus, Deus tuus, óleo exsultatiónis præ particípibus tuis. Et: Tu in princípio, Dómine, terram fundásti: et ópera mánuum tuárum sunt cœli. Ipsi períbunt, tu autem permanébis; et omnes ut vestiméntum veteráscent: et velut amíctum mutábis eos, et mutabúntur: tu autem idem ipse es, et anni tui non defícient.
}\switchcolumn\portugues{
\rlettrine{D}{eus,} que falara muitas vezes e de muitas maneiras a nossos pais pelos Profetas, falou-nos nos últimos tempos pelo seu Filho, que constituíra herdeiro de todas as coisas, e por quem, também, criou os séculos, o qual, sendo a irradiação esplendorosa da sua glória e a figura da sua substância e sustentando todas as coisas com sua palavra omnipotente depois de nos haver purificado dos nossos pecados, assentou-se à dextra da majestade divina, no mais alto dos céus, tanto mais superior aos Anjos quanto mais excelente e maior é o nome que herdou. Com efeito, a qual dos Anjos disse Deus estas palavras: «Tu és o meu Filho, gerei-te hoje?». E ainda estoutras: «Eu serei sen Pai e Ele será para mim Filho?». E também, quando mandou o seu primogénito ao mundo, disse: «Que todos os Anjos de Deus O adorem!». E, pelo que respeita aos Anjos, a Escritura diz: «Ele, que aos Anjos faz ventos e aos seus ministros chamas de fogo». Mas a respeito do Filho exprime-se assim: «Vosso trono, ó Deus, será eterno; o ceptro da vossa realeza será um ceptro de justiça e de rectidão. Vós amastes a justiça e odiastes a iniquidade. Por isso, ó Deus, o vosso Deus (o Pai) Vos ungiu com o bálsamo da alegria, mais ainda do que àqueles que convosco participam da glória». E também disse: «Vós, Senhor, no princípio criastes a terra; os céus são obra das vossas mãos. Eles perecerão, mas Vós permanecereis; eles envelhecerão todos, como um vestido, e Vós os mudareis, como se fossem um manto, e ficarão mudados; mas Vós sois sempre o mesmo, e os anos não acabarão para Vós».
}\end{paracol}

\paragraphinfo{Gradual}{Sl. 97, 3 \& 2}
\begin{paracol}{2}\latim{
\rlettrine{V}{idérunt} omnes fines terræ salutare Dei nostri: jubiláte Deo, omnis terra. ℣. Notum fecit Dominus salutare suum: ante conspéctum géntium revelávit justitiam suam.
}\switchcolumn\portugues{
\rlettrine{T}{oda} a terra viu o Salvador, que o nosso Deus enviou: aclamai Deus, ó povos de toda a terra. O Senhor manifestou o Salvador, que havia prometido: e manifestou a sua justiça aos olhos dos povos.
}\switchcolumn*\latim{
Allelúja, allelúja. ℣. Dies sanctificátus illúxit nobis: veníte, gentes, et adoráte Dóminum: quia hódie descéndit lux magna super terram. Allelúja.
}\switchcolumn\portugues{
Aleluia, aleluia. ℣. Um dia de santidade resplandeceu para nós: vinde, ó povos, e adorai o Senhor: pois hoje desceu a grande luz à terra. Aleluia.
}\end{paracol}

\paragraphinfo{Evangelho}{Jo, 1, 1-14}

\begin{paracol}{2}\latim{
\cruz Initium sancti Evangélii secúndum Joánnem.
}\switchcolumn\portugues{
\cruz Início do santo Evangelho segundo S. João.
}\switchcolumn*\latim{
\blettrine{I}{n} princípio erat Verbum, et Verbum erat apud Deum, et Deus erat Verbum. Hoc erat in princípio apud Deum. Omnia per ipsum facta sunt: et sine ipso factum est nihil, quod factum est: in ipso vita erat, et vita erat lux hóminum: et lux in ténebris lucet, et ténebræ eam non comprehendérunt. Fuit homo missus a Deo, cui nomen erat Joánnes. Hic venit in testimónium, ut testimónium perhibéret de lúmine, ut omnes créderent per illum. Non erat ille lux, sed ut testimónium perhibéret de lúmine. Erat lux vera, quæ illúminat omnem hóminem veniéntem in hunc mundum. In mundo erat, et mundus per ipsum factus est, et mundus eum non cognóvit. In própria venit, et sui eum non recepérunt. Quotquot autem recepérunt eum, dedit eis potestátem fílios Dei fíeri, his, qui credunt in nómine ejus: qui non ex sanguínibus, neque ex voluntáte carnis, neque ex voluntáte viri, sed ex Deo nati sunt. \emph{(Hic genuflectitur)} Et Verbum caro factum est, et habitávit in nobis: et vídimus glóriam ejus, glóriam quasi Unigéniti a Patre, plenum grátiæ et veritátis.
}\switchcolumn\portugues{
\blettrine{N}{o} princípio existia o Verbo, e o Verbo estava com Deus, e o Verbo era Deus. Este estava no princípio com Deus. Todas as coisas foram por Ele criadas, e nada daquilo que foi criado teria sido criado sem Ele. N’Ele havia vida, e a vida era a luz dos homens. A luz resplandeceu nas trevas, mas as trevas a não receberam. Apareceu um homem, mandado por Deus, e o seu nome era João, o qual veio como testemunha, para dar testemunho da luz, a fim de que por ele todos acreditassem. Ele não era a luz, mas aquele que havia de dar testemunho da luz. Existia a luz verdadeira, a luz que ilumina todo o homem que vem a este mundo. Ele estava no mundo, e o mundo, embora houvesse sido criado por Ele, O não conheceu. Veio ao que era seu, e os seus O não receberam. Porém, Ele a todos quantos O receberam e aos que acreditaram no seu nome deu o poder de serem filhos de Deus, os quais não nasceram do sangue, nem do desejo da carne, mas somente da vontade de Deus. E o Verbo fez-se carne \emph{(genuflecte-se)} e habitou entre nós; e contemplamos a sua glória, como era própria do Filho Unigénito do Pai, cheio de graça e de verdade.
}\end{paracol}

\paragraphinfo{Ofertório}{Sl. 88, 12 \& 15}

\begin{paracol}{2}\latim{
\rlettrine{T}{ui} sunt cœli et tua est terra: orbem terrárum et plenitúdinem ejus tu fundásti: justítia et judícium præparátio sedis tuæ.
}\switchcolumn\portugues{
\rlettrine{A}{} Vós, Senhor, pertencem os céus e a terra; pois criastes o universo e tudo o que ele encerra. A justiça e a equidade são a base do vosso trono.
}\end{paracol}

\paragraph{Secreta}

\begin{paracol}{2}\latim{
\rlettrine{O}{bláta,} Dómine, múnera, nova Unigéniti tui Nativitáte sanctífica: nosque a peccatórum nostrórum máculis emúnda. Per eúndem Dóminum nostrum \emph{\&c.}
}\switchcolumn\portugues{
\rlettrine{S}{antificai,} Senhor, pelo novo Nascimento do vosso Filho Unigénito, as oblatas que Vos apresentamos, e purificai-nos das manchas dos nossos pecados. Pelo mesmo nosso Senhor \emph{\&c.}
}\end{paracol}

\paragraphinfo{Comúnio}{Sl. 97, 3}

\begin{paracol}{2}\latim{
\rlettrine{V}{idérunt} omnes fines terræ salutáre Dei nostri.
}\switchcolumn\portugues{
\rlettrine{T}{oda} a terra contemplou o Salvador que o nosso Deus enviou,
}\end{paracol}

\paragraph{Postcomúnio}

\begin{paracol}{2}\latim{
\rlettrine{P}{ræsta,} quǽsumus, omnípotens Deus: ut natus hódie Salvátor mundi, sicut divínæ nobis generatiónis est auctor; ita et immortalitátis sit ipse largítor: Qui tecum vivit et regnat \emph{\&c.}
}\switchcolumn\portugues{
\slettrine{Ó}{} Deus omnipotente, dignai-Vos permitir que, assim como o Salvador do mundo, nascendo neste dia, nos comunicou a geração divina, assim também nos conceda a imortalidade. Ele, que, sendo Deus \emph{\&c.}
}\end{paracol}

\paragraphinfo{Último Evangelho}{Mt. 2, 19-23}
\begin{paracol}{2}\latim{
\cruz Sequéntia sancti Evangélii secúndum Matthǽum.
}\switchcolumn\portugues{
\cruz Continuação do santo Evangelho segundo S. Mateus.
}\switchcolumn*\latim{
\blettrine{C}{um} natus esset Jesus in Béthlehem Juda in diébus Heródis regis, ecce, Magi ab Oriénte venerunt Jerosólymam, dicéntes: Ubi est, qui natus est rex Judæórum? Vidimus enim stellam ejus in Oriénte, et vénimus adoráre eum. Audiens autem Heródes rex, turbatus est, et omnis Jerosólyma cum illo. Et cóngregans omnes principes sacerdotum et scribas pópuli, sciscitabátur ab eis, ubi Christus nasceretur. At illi dixérunt ei: In Béthlehem Judae: sic enim scriptum est per Prophétam: Et tu, Béthlehem terra Juda, nequaquam mínima es in princípibus Juda; ex te enim éxiet dux, qui regat pópulum meum Israel. Tunc Heródes, clam vocátis Magis, diligénter dídicit ab eis tempus stellæ, quæ appáruit eis: et mittens illos in Béthlehem, dixit: Ite, et interrogáte diligénter de púero: et cum invenéritis, renuntiáte mihi, ut et ego véniens adórem eum. Qui cum audíssent regem, abiérunt. Et ecce, stella, quam víderant in Oriénte, antecedébat eos, usque dum véniens staret supra, ubi erat Puer. Vidéntes autem stellam, gavísi sunt gáudio magno valde. Et intrántes domum, invenérunt Púerum cum María Matre ejus, \emph{(hic genuflectitur)} ei procidéntes adoravérunt eum. Et, apértis thesáuris suis, obtulérunt ei múnera, aurum, thus et myrrham. Et responso accépto in somnis, ne redírent ad Heródem, per aliam viam revérsi sunt in regiónem suam.
}\switchcolumn\portugues{
\blettrine{C}{ontinuação} do santo Evangelho segundo S. Mateus. Havendo Jesus nascido em Belém, de Judá, no tempo do rei Herodes, eis que vieram a Jerusalém os Magos do Oriente, dizendo: «Onde está o Rei dos Judeus, que acaba de nascer? Pois vimos a sua estrela no Oriente e viemos adorá-l’O». Logo que o rei Herodes ouviu esta notícia, ficou perturbado, assim como toda a gente de Jerusalém, convocando logo todos os príncipes dos sacerdotes e os escribas do povo, para saber deles onde deveria nascer Cristo. Responderam-lhe eles: «Em Belém, de Judá, pois está escrito pelo Profeta: «E tu, Belém, terra de Judá, não serás certamente a menos importante entre as terras principais de Judá, pois em ti nascerá o Rei, que governará o meu povo de Israel». Então Herodes mandou chamar em segredo os Magos, informando-se com eles diligentemente acerca do tempo em que a estrela havia aparecido. E enviando-os a Belém, disse-lhes: «Ide, procurai diligentemente o Menino, e, logo que O houverdes achado, avisai-me, para que eu vá, também, adorá-l’O». Os Magos, tendo ouvido estas palavras, partiram. Ora, a estrela, que tinham visto no Oriente ia adiante deles, até que, chegando ao lugar onde estava o Menino, parou. Quando os Magos viram a estrela, alegraram-se muito. Entrando, então, na casa, encontraram o Menino com Maria, sua mãe; e, de joelhos, O adoraram. \emph{(Todos devem ajoelhar)}. E, tendo aberto os seus tesouros, ofereceram-Lhe presentes de oiro, incenso e mirra. Depois, havendo tido aviso em sonhos de que não deveriam voltar a encontrar Herodes, retiraram-se por outro caminho para o seu país.
}\end{paracol}

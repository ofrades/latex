\subsectioninfo{Natal de N. S. Jesus Cristo}{Segunda Missa: ao romper da aurora}

\paragraphinfo{Intróito}{Is. 9, 2 \& 6}

\begin{paracol}{2}\latim{
\rlettrine{L}{ux} fulgébit hódie super nos: quia natus est nobis Dóminus: et vocábitur Admirábilis, Deus, Princeps pacis, Pater futúri sǽculi: cujus regni non erit finis. \emph{Ps. 92, 1} Dominus regnávit, decorem indutus est: indutus est Dominus fortitudinem, et præcínxit se
℣. Gloria Patri \emph{\&c.}
}\switchcolumn\portugues{
\rlettrine{A}{} luz brilhará hoje sobre nós: pois o Senhor nasceu para nós: e será chamado Admirável, Deus, Príncipe da Paz, Pai da eternidade; seu reino não terá fim. \emph{Sl. 92, 1} O Senhor reinou e revestiu-se de majestade: o Senhor revestiu-se com a túnica da fortaleza e cingiu-se para o combate.
℣. Glória ao Pai \emph{\&c.}
}\end{paracol}

\paragraph{Oração}

\begin{paracol}{2}\latim{
\rlettrine{D}{a} nobis, quǽsumus, omnípotens Deus: ut, qui nova incarnáti Verbi tui luce perfúndimur; hoc in nostro respléndeat ópere, quod per fidem fulget in mente. Per eúndem Dóminum \emph{\&c.}
}\switchcolumn\portugues{
\slettrine{Ó}{} Deus omnipotente, que Vos dignastes esclarecer-nos com a nova luz do Verbo Incarnado, permiti ainda, Vos suplicamos, que o brilho desta mesma luz, que pela fé ilustra as nossas almas, resplandeça nas nossas acções. Pelo mesmo nosso Senhor \emph{\&c.}
}\end{paracol}

\paragraphinfo{Epístola}{Tt. 3, 4-7}

\begin{paracol}{2}\latim{
Lectio Epístolæ beati Pauli Apostoli ad Titum.
}\switchcolumn\portugues{
Lição da Ep.ª do B. Ap.º Paulo a Tito.
}\switchcolumn*\latim{
\rlettrine{C}{aríssime:} Appáruit benígnitas et humánitas Salvatóris nostri Dei: non ex opéribus justítiæ, quæ fécimus nos, sed secúndum suam misericórdiam salvos nos fecit per lavácrum regeneratiónis et renovatiónis Spíritus Sancti, quem effúdit in nos abúnde per Jesum Christum, Salvatorem nostrum: ut, justificáti grátia ipsíus, herédes simus secúndum spem vitæ ætérnæ: in Christo Jesu, Dómino nostro.
}\switchcolumn\portugues{
\rlettrine{C}{aríssimo:} A bondade e o amor de Deus, nosso Salvador, se manifestaram. Ele salvou-nos, não por causa das obras de justiça que houvéssemos praticado, mas pela sua misericórdia, lavando-nos em um banho de regeneração e de renovação do Espírito Santo, que lançou copiosamente sobre nós por Jesus Cristo, nosso Salvador, a fim de que, justificados pela sua graça, nos tornemos herdeiros da vida eterna, segundo a esperança que depositamos em Jesus Cristo, nosso Senhor.
}\end{paracol}

\paragraphinfo{Gradual}{Sl. 117, 26, 27 \& 23}

\begin{paracol}{2}\latim{
\rlettrine{B}{enedíctus,} qui venit in nómine Dómini: Deus Dóminus, et illúxit nobis. ℣. A Dómino factum est istud: et est mirábile in óculis nostris.
}\switchcolumn\portugues{
\rlettrine{B}{endito} seja Aquele que vem em nome do Senhor! O Senhor é Deus e a sua luz resplandeceu sobre nós. ℣. Foi o Senhor quem criou esta maravilha, que brilha aos nossos olhos.
}\switchcolumn*\latim{
Allelúja, allelúja. ℣. \emph{Ps. 92, 1} Dóminus regnávit, decórem índuit: índuit Dóminus fortitúdinem, et præcínxit se virtúte. Allelúja.
}\switchcolumn\portugues{
Aleluia, aleluia. ℣. \emph{Sl. 92, 1} O Senhor reina e revestiu-se de majestade: o Senhor revestiu-se com a túnica da fortaleza e cingiu-se para o combate. Aleluia.
}\end{paracol}

\paragraphinfo{Evangelho}{Lc. 2, 15-20}

\begin{paracol}{2}\latim{
\cruz Sequéntia sancti Evangélii secúndum Lucam.
}\switchcolumn\portugues{
\cruz Continuação do santo Evangelho segundo S. Lucas.
}\switchcolumn*\latim{
\blettrine{I}{n} illo témpore: Pastóres loquebántur ad ínvicem: Transeámus usque Béthlehem, et videámus hoc verbum, quod factum est, quod Dóminus osténdit nobis. Et venérunt festinántes: et invenérunt Maríam et Joseph et Infántem pósitum in præsépio. Vidéntes autem cognovérunt de verbo, quod dictum erat illis de Púero hoc. Et omnes, qui audiérunt, miráti sunt: et de his, quæ dicta erant a pastóribus ad ipsos. María autem conservábat ómnia verba hæc, cónferens in corde suo. Et revérsi sunt pastóres, glorificántes et laudántes Deum in ómnibus, quæ audíerant et víderant, sicut dictum est ad illos.
}\switchcolumn\portugues{
\blettrine{N}{aquele} tempo, disseram os pastores uns aos outros: «Vamos até Belém e vejamos o que foi isto que aconteceu, que o Senhor nos revelou». Vieram, então, a toda a pressa, e encontraram Maria, José e o Menino deitado no presépio. Vendo isto, conheceram a verdade do que lhes havia sido revelado acerca deste Menino. E todos quantos ouviam falar os pastores ficavam admirados do que eles diziam. Ora Maria conservava todas estas cousas e meditava-as no seu íntimo. E os pastores retiraram-se, glorificando e louvando Deus pelo que tinham visto e ouvido, segundo o que lhes havia sido revelado.
}\end{paracol}

\paragraphinfo{Ofertório}{Sl. 92, 1-2}

\begin{paracol}{2}\latim{
\rlettrine{D}{eus} firmávit orbem terræ, qui non commovébitur: paráta sedes tua, Deus, ex tunc, a sǽculo tu es.
}\switchcolumn\portugues{
\rlettrine{D}{eus} firmou de tal modo o globo da terra que nunca mais será destruído: vosso trono, ó Deus, existe desde a eternidade, pois Vós existis antes dos séculos!
}\end{paracol}

\paragraph{Secreta}

\begin{paracol}{2}\latim{
\rlettrine{M}{únera} nostra, quǽsumus, Dómine, Nativitátis hodiérnæ mystériis apta provéniant, et pacem nobis semper infúndant: ut, sicut homo génitus idem refúlsit et Deus, sic nobis hæc terréna substántia cónferat, quod divínum est. Per eúndem Dóminum \emph{\&c.}
}\switchcolumn\portugues{
\qlettrine{Q}{ue} estas nossas ofertas, Senhor, Vos suplicamos, se tornem dignas dos mystérios do Nascimento deste dia, e nos infundam perpetuamente a paz, para que, assim como Aquele que nasceu como homem fez ao mesmo tempo realçar a sua divindade, assim também esta Substância terrena (o pão e o vinho) nos comunique o que é divino. Pelo mesmo nosso Senhor \emph{\&c.}
}\end{paracol}

\paragraphinfo{Comúnio}{Zc. 9, 9}
\begin{paracol}{2}\latim{
\rlettrine{E}{xsúlta,} fília Sion, lauda, fília Jerúsalem: ecce, Rex tuus venit sanctus et Salvátor mundi.
}\switchcolumn\portugues{
\rlettrine{A}{legra-te,} filha de Sião; canta alegremente, filha de Jerusalém: eis que vem o teu Rei, que é o Santo e o Salvador do mundo.
}\end{paracol}

\paragraph{Postcomúnio}

\begin{paracol}{2}\latim{
\rlettrine{H}{ujus} nos, Dómine, sacraménti semper nóvitas natális instáuret: cujus Natívitas singuláris humánam réppulit vetustátem. Per eúndem Dóminum \emph{\&c.}
}\switchcolumn\portugues{
\rlettrine{P}{ermiti,} Senhor, que as nossas almas sejam regeneradas pelo novo Nascimento d’Aquele que se dá neste sacramento, cujo admirável Nascimento destruiu o «homem velho». Pelo mesmo nosso Senhor \emph{\&c.}
}\end{paracol}

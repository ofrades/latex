\subsectioninfo{Natal de N. S. Jesus Cristo}{Primeira Missa: à meia-noite}

\paragraphinfo{Intróito}{Sl. 2, 7}

\begin{paracol}{2}\latim{
\rlettrine{D}{óminus} dixit ad me: Fílius meus es tu, ego hódie génui te. \emph{Ps. ib., 1} Quare fremuérunt gentes: et pópuli meditáti sunt inánia?
℣. Gloria Patri \emph{\&c.}
}\switchcolumn\portugues{
\rlettrine{O}{} Senhor disse-me: Sois o meu Filho: Eu Vos gerei hoje. \emph{Sl. ib., 1} Porque se agitaram as nações? Porque meditaram os povos vãos projectos contra mim?
℣. Glória ao Pai \emph{\&c.}
}\end{paracol}

\paragraph{Oração}

\begin{paracol}{2}\latim{
\rlettrine{D}{eus,} qui hanc sacratíssimam noctem veri lúminis fecísti illustratióne claréscere: da, quǽsumus; ut, cujus lucis mystéria in terra cognóvimus, ejus quoque gáudiis in cœlo perfruámur: Qui tecum vivit \emph{\&c.}
}\switchcolumn\portugues{
\slettrine{Ó}{} Deus, que fizestes brilhar nesta noite santíssima os esplendores da verdadeira luz, permiti, Vos pedimos, que, depois de havermos conhecido esta luz misteriosa neste mundo, possamos gozar no céu as delícias de que é origem Aquele que, sendo Deus, convosco vive e reina em unidade \emph{\&c.}
}\end{paracol}

\paragraphinfo{Epístola}{Tt. 2, 11-15}

\begin{paracol}{2}\latim{
Léctio Epístolæ beati Pauli Apóstoli ad Titum.
}\switchcolumn\portugues{
Lição da Ep.ª do B. Ap.º Paulo a Tito.
}\switchcolumn*\latim{
\rlettrine{C}{aríssime:} Appáruit grátia Dei Salvatóris nostri ómnibus homínibus, erúdiens nos, ut, abnegántes impietátem et sæculária desidéria, sóbrie et juste et pie vivámus in hoc sǽculo, exspectántes beátam spem et advéntum glóriæ magni Dei et Salvatóris nostri Jesu Christi: qui dedit semetípsum pro nobis: ut nos redímeret ab omni iniquitáte, et mundáret sibi pópulum acceptábilem, sectatórem bonórum óperum. Hæc lóquere et exhortáre: in Christo Jesu, Dómino nostro.
}\switchcolumn\portugues{
\rlettrine{C}{aríssimo:} A graça de Deus, nosso Salvador, manifestou-se a todos os homens, ensinando-nos, a fim de que, repudiando a impiedade e os apetites terrenos, vivamos neste mundo com temperança, justiça e piedade, pensando na esperança, na bem-aventurança eterna e na vinda da glória do nosso grande Deus e Salvador, Jesus Cristo: que se ofereceu espontaneamente por nós, para nos resgatar de todas as iniquidades, e tornar-nos numa raça purificada, escolhida e zelosa em suas boas obras. Ensina e prega estas coisas, em Jesus Cristo, nosso Senhor!
}\end{paracol}

\paragraphinfo{Gradual}{Sl. 109, 3 \& 1}

\begin{paracol}{2}\latim{
\rlettrine{T}{ecum} princípium in die virtútis tuæ: in splendóribus Sanctórum, ex útero ante lucíferum génui te. ℣. Dixit Dóminus Dómino meo: Sede a dextris meis: donec ponam inimícos tuos, scabéllum pedum tuórum
}\switchcolumn\portugues{
\rlettrine{E}{m} Vós estará o poder soberano no dia do vosso império, no meio dos esplendores dos escolhidos. Eu Vos gerei no meu seio, antes da aurora. O soberano Senhor disse ao meu Senhor: «Assentai-Vos à minha dextra até que eu torne os meus inimigos em escabelo dos vossos pés».
}\switchcolumn*\latim{
Allelúja, allelúja. ℣. \emph{Ps. 2, 7} Dóminus dixit ad me: Fílius meus es tu, ego hódie génui te. Allelúja.
}\switchcolumn\portugues{
Aleluia, aleluia. ℣. \emph{Sl. 2, 7} O Senhor disse-me: «Vós sois o meu Filho; gerei-Vos hoje». Aleluia.
}\end{paracol}

\paragraphinfo{Evangelho}{Lc. 2, 1-14}
\begin{paracol}{2}\latim{
\cruz Sequéntia sancti Evangélii secúndum Lucam.
}\switchcolumn\portugues{
\cruz Continuação do santo Evangelho segundo S. Lucas.
}\switchcolumn*\latim{
\blettrine{I}{n} illo témpore: Exiit edíctum a Cǽsare Augústo, ut describerétur univérsus orbis. Hæc descríptio prima facta est a prǽside Sýriæ Cyríno: et ibant omnes ut profiteréntur sínguli in suam civitátem. Ascéndit autem et Joseph a Galilǽa de civitáte Názareth, in Judǽam in civitátem David, quæ vocatur Béthlehem: eo quod esset de domo et fámilia David, ut profiterétur cum María desponsáta sibi uxóre prægnánte. Factum est autem, cum essent ibi, impléti sunt dies, ut páreret. Et péperit fílium suum primogénitum, et pannis eum invólvit, et reclinávit eum in præsépio: quia non erat eis locus in diversório. Et pastóres erant in regióne eádem vigilántes, et custodiéntes vigílias noctis super gregem suum. Et ecce, Angelus Dómini stetit juxta illos, et cláritas Dei circumfúlsit illos, et timuérunt timóre magno. Et dixit illis Angelus: Nolíte timére: ecce enim, evangelízo vobis gáudium magnum, quod erit omni pópulo: quia natus est vobis hódie Salvátor, qui est Christus Dóminus, in civitáte David. Et hoc vobis signum: Inveniétis infántem pannis involútum, et pósitum in præsépio. Et súbito facta est cum Angelo multitúdo milítiæ cœléstis, laudántium Deum et dicéntium: Glória in altíssimis Deo, et in terra pax hóminibus bonæ voluntátis.
}\switchcolumn\portugues{
\blettrine{N}{aquele} tempo, foi publicado um édito de César Augusto para que se fizesse o recenseamento de todo o universo. Este primeiro recenseamento foi feito por Cirino, governador da Síria. E todos, pois, iam à sua cidade para se inscreverem. José saiu também da Galileia, da cidade de Nazaré, para a Judeia, e foi a Belém, cidade de David, porque ele era da casa e da família de David, para aí ser recenseado com Maria, sua esposa, que estava próximo a dar à luz. Ora, aconteceu que, enquanto eles estavam em Belém, completaram-se os dias em que ela devia dar à luz. E, com efeito, ela deu à luz o seu Filho primogénito, envolvendo-O em uns panos e colocando-O em um presépio, porque não tiveram lugar na hospedaria. Ora, havia naquele lugar uns pastores, guardando durante a noite os seus gados; e eis que um Anjo do céu lhes apareceu, envolvendo-os com a claridade de Deus, o que os encheu de grande temor. Então, o Anjo disse-lhes: «Não tenhais receio, pois venho anunciar-vos uma grande alegria, que se estenderá a todo o povo: é que nasceu hoje, na cidade de David, um Salvador, que é o Cristo Senhor. Isto vos servirá de sinal: Achareis um Menino, envolvido em uns panos e deitado num presépio», E no mesmo instante reuniu-se com o Anjo uma multidão da milícia celestial, louvando Deus e dizendo: «Glória a Deus no mais alto dos céus, e, na terra, paz aos homens de boa vontade!»
}\end{paracol}

\paragraphinfo{Ofertório}{Sl. 95, 11 \& 13}

\begin{paracol}{2}\latim{
\rlettrine{L}{æténtur} cœli et exsúltet terra ante fáciem Dómini: quóniam venit.
}\switchcolumn\portugues{
\rlettrine{A}{legrem-se} os céus! Regozije-se a terra ante a face do Senhor, pois Ele veio.
}\end{paracol}

\paragraph{Secreta}

\begin{paracol}{2}\latim{
\rlettrine{A}{cépta} tibi sit, Dómine, quǽsumus, hodiérnæ festivitátis oblátio: ut, tua gratia largiénte, per hæc sacrosáncta commércia, in illíus inveniámur forma, in quo tecum est nostra substántia: Qui tecum vivit \emph{\&c.}
}\switchcolumn\portugues{
\rlettrine{D}{ignai-Vos,} Senhor, aceitar a oblação que Vos oferecemos na festividade deste dia, e pela vossa graça permiti que por meio deste sacrossanto comércio nos assemelhemos Àquele em quem a nossa substância humana está unida à vossa divindade. Ele, que, sendo Deus, convosco vive e reina \emph{\&c.}
}\end{paracol}

\paragraphinfo{Comúnio}{Sl. 109, 3}

\begin{paracol}{2}\latim{
\rlettrine{I}{n} splendóribus Sanctórum, ex útero ante lucíferum génui te.
}\switchcolumn\portugues{
\rlettrine{E}{u} Vos gerei no meu seio, antes do romper da aurora, entre os esplendores dos escolhidos.
}\end{paracol}

\paragraph{Postcomúnio}

\begin{paracol}{2}\latim{
\rlettrine{D}{a} nobis, quǽsumus, Dómine, Deus noster: ut, qui Nativitátem Dómini nostri Jesu Christi mystériis nos frequentáre gaudémus; dignis conversatiónibus ad ejus mereámur per veníre consórtium: Qui tecum \emph{\&c.}
}\switchcolumn\portugues{
\rlettrine{S}{enhor,} nosso Deus, permiti, Vos suplicamos, que, celebrando alegremente o Nascimento de nosso Senhor Jesus Cristo, pela frequência destes divinos mistérios, mereçamos, com uma conduta santa, gozar a união perfeita com Aquele: que, sendo Deus \emph{\&c.}
}\end{paracol}

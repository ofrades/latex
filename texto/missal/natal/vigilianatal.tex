\subsectioninfo{Vigília do Natal de N. S. Jesus Cristo}{Dia 24 de Dezembro}

\paragraphinfo{Intróito}{Ex. 16, 6 \& 7.}

\begin{paracol}{2}\latim{
\rlettrine{H}{ódie} sciétis, quia véniet Dóminus et salvábit nos: et mane vidébitis glóriam ejus. \emph{Ps. 23, 1} Dómini est terra, et plenitúdo ejus: orbis terrárum, et univérsi, qui hábitant in eo.
℣. Gloria Patri \emph{\&c.}
}\switchcolumn\portugues{
\rlettrine{S}{abereis} hoje que o Senhor virá e nos salvará: e amanhã vereis a sua glória. \emph{Sl. 23, 1} Pertence ao Senhor a terra e tudo o que ela encerra: o orbe da terra e todos que nele habitam.
℣. Glória ao Pai \emph{\&c.}
}\end{paracol}

\paragraph{Oração}

\begin{paracol}{2}\latim{
\rlettrine{D}{eus,} qui nos redemptiónis nostræ ánnua exspectatióne lætíficas: præsta; ut Unigénitum tuum, quem Redemptórem læti suscípimus, veniéntem quoque Júdicem secúri videámus, Dóminum nostrum Jesum Christum, Fílium tuum: Qui tecum vivit et regnat\emph{\&c.}
}\switchcolumn\portugues{
\slettrine{Ó}{} Deus, que nos alegrais cada ano com a expectação da nossa redenção, dignai-Vos conceder-nos que, recebendo nós com alegria o vosso Filho Unigénito, N. S. Jesus Cristo, quando vem a nós como Redentor, assim possamos igualmente recebê-l’O com tranquilidade quando Ele vier como Juiz: Ele que, sendo Deus \emph{\&c.}
}\end{paracol}

\paragraphinfo{Epístola}{Rm. 1, 1–0}

\begin{paracol}{2}\latim{
Lectio Epístolæ beati Pauli Apostoli ad Romános
}\switchcolumn\portugues{
Lição da Ep.ª do B. Ap.º Paulo aos Romanos.
}\switchcolumn*\latim{
\rlettrine{P}{aulus,} servus Jesu Christi, vocátus Apóstolus, segregátus in Evangélium Dei, quod ante promíserat per Prophétas suos in Scriptúris sanctis de Fílio suo, qui factus est ei ex sémine David secúndum carnem: qui prædestinátus est Fílius Dei in virtúte secúndum spíritum sanctificatiónis ex resurrectióne mortuórum Jesu Christi, Dómi
ni nostri: per quem accépimus grátiam, et apostolátum ad obœdiéndum fídei in ómnibus géntibus pro nómine ejus, in quibus estis et vos vocáti Jesu Christi, Dómini nostri.
}\switchcolumn\portugues{
\rlettrine{P}{aulo,} servo de Jesus Cristo, apóstolo por vocação divina, escolhido para pregar o Evangelho, que Deus havia prometido pelos seus Profetas nas Sagradas Escrituras a respeito de seu Filho (que nasceu da geração de David, segundo a carne, e foi predestinado Filho de Deus, com o poder, segundo o Espírito de santidade, para ressuscitar dos mortos) N. S. Jesus Cristo, de quem recebemos a graça e o apostolado para chamar em seu nome à obediência da fé todas as nações, das quais vós, que também fostes chamados, fazeis parte.
}\end{paracol}

\paragraphinfo{Gradual}{Ex. 16, 6 \& 7. }
\begin{paracol}{2}\latim{
\rlettrine{H}{ódie} sciétis, quia véniet Dóminus et salvábit nos: et mane vidébitis glóriam ejus. ℣. \emph{Ps. 79, 2–3} Qui regis Israël, inténde: qui dedúcis, velut ovem, Joseph: qui sedes super Chérubim, appáre coram Ephraim, Bénjamin, et Manásse.
}\switchcolumn\portugues{
\rlettrine{S}{abereis} hoje que o Senhor virá e nos salvará: e amanhã vereis a sua glória. ℣. \emph{Sl. 79, 2–3} Ouvi, ó pastores de Israel: ó vós, que conduzis José, como um pastor conduz uma ovelha. Manifestai-Vos ante Efraim, Benjamim e Manassés, ó Vós, que tendes um trono acima dos Querubins!
}\end{paracol}

\emph{Se esta Vigília ocorre ao Domingo, acrescenta-se:}

\begin{paracol}{2}\latim{
Allelúja, allelúja. ℣. Crástina die delébitur iníquitas terræ: et regnábit super nos Salvátor mundi. Allelúja.
}\switchcolumn\portugues{
Aleluia, aleluia. Amanhã será apagada a iniquidade da terra e o Salvador do mundo reinará sobre nós. Aleluia.
}\end{paracol}

\paragraphinfo{Evangelho}{Mt. 1, 18–21}

\begin{paracol}{2}\latim{
\cruz Sequéntia sancti Evangélii secúndum Matthǽum.
}\switchcolumn\portugues{
\cruz Continuação do santo Evangelho segundo S. Mateus.
}\switchcolumn*\latim{
\blettrine{C}{um} esset desponsáta Mater Jesu Maria Joseph, ántequam convenírent, inventa est in útero habens de Spiritu Sancto. Joseph autem, vir ejus, cum esset justus et nollet eam tradúcere, vóluit occúlte dimíttere eam. Hæc autem eo cogitánte, ecce, Angelus Dómini appáruit in somnis ei, dicens: Joseph, fili David, noli timére accípere Maríam cónjugem tuam: quod enim in ea natum est, de Spíritu Sancto est. Páriet autem fílium, et vocábis nomen ejus Jesum: ipse enim salvum fáciet pópulum suum a peccátis eórum.
}\switchcolumn\portugues{
\blettrine{E}{stando} já Maria, Mãe de Jesus, desposada com José, notou-se, antes que eles tivessem coabitado, que ela havia concebido do Espírito Santo. Mas José, seu marido, que era homem justo, não queria difamá-la. Resolveu, pois, deixá-la secretamente. Pensando ele nisto, eis que um Anjo do Senhor lhe apareceu em sonhos e lhe disse: «José, filho de David, não temas receber Maria como tua esposa, porquanto o que ela concebeu é obra do Espírito Santo. Ela dará à luz um Filho e ser-Lhe-á dado o nome de Jesus; pois Ele salvará o povo dos seus pecados».
}\end{paracol}

\paragraphinfo{Ofertório}{Sl. 23, 7}

\begin{paracol}{2}\latim{
\rlettrine{T}{óllite} portas, principes, vestras: et elevámini, portæ æternáles, et introíbit Rex glóriæ.
}\switchcolumn\portugues{
\rlettrine{A}{bri-vos} inteiramente, ó portas, e entrará o Rei da glória.
}\end{paracol}

\paragraph{Secreta}

\begin{paracol}{2}\latim{
\rlettrine{D}{a} nobis, quǽsumus, omnípotens Deus: ut, sicut adoránda Fílii tui natalítia prævenímus, sic ejus múnera capiámus sempitérna gaudéntes: Qui tecum \emph{\&c.}
}\switchcolumn\portugues{
\rlettrine{D}{ignai-Vos} permitir, ó Deus omnipotente, Vos suplicamos, que, assim como prevenimos o adorável nascimento do vosso Filho, assim também recebamos com alegria os dons eternos d’Aquele que, sendo Deus, vive e \emph{\&c.}
}\end{paracol}

\paragraphinfo{Comúnio}{Is. 40, 5}

\begin{paracol}{2}\latim{
\rlettrine{R}{evelábitur} glória Dómini: et vidébit omnis caro salutáre Dei nostri.
}\switchcolumn\portugues{
\rlettrine{A}{glória} do Senhor vai manifestar-se: e toda a carne verá o Salvador que o nosso Deus nos manda.
}\end{paracol}

\paragraph{Postcomúnio}

\begin{paracol}{2}\latim{
\rlettrine{D}{a} nobis, quǽsumus, Dómine: unigéniti Fílii tui recensíta nativitáte respiráre; cujus cœlésti mystério páscimur et potámur. Per eúndem Dóminum \emph{\&c.}
}\switchcolumn\portugues{
\rlettrine{C}{oncedei-nos,} Senhor, Vos suplicamos, que possamos respirar com alegria, celebrando o nascimento de vosso Filho Unigénito, cujo celestial mistério nos alimenta e conforta. Pelo mesmo nosso Senhor \emph{\&c.}
}\end{paracol}
\subsection{Domingo dentro do Oitavário do Natal}

\paragraphinfo{Intróito}{Sb. 18, 14-15}

\begin{paracol}{2}\latim{
\rlettrine{D}{um} médium siléntium tenérent ómnia, et nox in suo cursu médium iter háberet, omnípotens Sermo tuus, Dómine, de cœlis a regálibus sédibus venit. \emph{Ps. 92, 1} Dóminus regnávit, decórem indútus est: indútus est Dóminus fortitúdinem, et præcínxit se.

℣. Gloria Patri \emph{\&c.}
}\switchcolumn\portugues{
\rlettrine{E}{nquanto} o mundo repousava em um profundo silêncio: e a noite estava no meio do seu decurso, o vosso Verbo omnipotente, Senhor, desceu dos céus, do seu régio trono. \emph{Sl. 92, 1} O Senhor revestiu-se de glória e reina; o Senhor revestiu-se com a túnica da majestade e cingiu-se com o poder.
℣. Glória ao Pai \emph{\&c.}
}\end{paracol}

\paragraph{Oração}

\begin{paracol}{2}\latim{
\rlettrine{C}{oncéde,} quǽsumus, omnípotens Deus: ut nos Unigéniti tui nova per carnem Natívitas líberet; quos sub peccáti jugo vetústa sérvitus tenet. Per eúndem Dóminum nostrum \emph{\&c.}
}\switchcolumn\portugues{
\slettrine{Ó}{} Deus omnipotente e sempiterno, dirigi as nossas acções segundo a vossa vontade, a fim de que pelo nome do vosso amantíssimo Filho sejamos dignos de praticar abundantes boas obras. Ele, que, sendo Deus \emph{\&c.}
}\end{paracol}

\paragraphinfo{Epístola}{Gl. 4, 1-7}

\begin{paracol}{2}\latim{
Lectio Epístolæ beati Pauli Apostoli ad Gálatas.
}\switchcolumn\portugues{
Lição da Ep.ª do B. Ap.º Paulo aos Gálatas.
}\switchcolumn*\latim{
\rlettrine{P}{atres:} Quanto témpore heres párvulus est, nihil differt a servo, cum sit dóminus ómnium: sed sub tutóribus et actóribus est usque ad præfinítum tempus a patre: ita et nos, cum essémus párvuli, sub eleméntis mundi erámus serviéntes. At ubi venit plenitúdo témporis, misit Deus Fílium suum, factum ex mulíere, factum sub lege, ut eos, qui sub lege erant, redímeret, ut adoptiónem filiórum reciperémus. Quóniam autem estis fílii, misit Deus Spíritum Fílii sui in corda vestra, clamántem: Abba, Pater. Itaque jam non est servus, sed fílius: quod si fílius, et heres per Deum.
}\switchcolumn\portugues{
\rlettrine{M}{eus} irmãos: Enquanto o herdeiro é menor, não se distingue do servo, ainda que seja o senhor de tudo; mas depende dos tutores e administradores, até ao tempo determinado por seu pai. Do mesmo modo, quando nós éramos ainda meninos, estávamos sujeitos aos elementos do mundo; mas, logo que chegou a plenitude dos tempos, Deus enviou o seu Filho, nascido da Mulher e sujeito à lei, para resgatar aqueles que estavam sob a sanção da lei, e se tornarem seus filhos adoptivos. E, porque sois seus filhos, mandou Deus ao vosso coração o espírito de seu Filho, o qual clama: Abba! Pai! E, assim, já nenhum de vós é escravo, mas filho. E, sendo seus filhos, sois também herdeiros, por intervenção de Deus.
}\end{paracol}

\paragraphinfo{Gradual}{Sl. 44, 3 \& 2}

\begin{paracol}{2}\latim{
\rlettrine{S}{peciósus} forma præ filiis hóminum:
diffúsa est gratia in lábiis tuis. ℣. Eructávit cor meum verbum bonum, dico ego ópera mea Regi: lingua mea cálamus scribæ, velóciter scribéntis.
}\switchcolumn\portugues{
\rlettrine{V}{ós} sois mais belo do que todos os filhos dos homens, pois a graça espalhou-se nos vossos lábios. Meu coração exprimiu uma palavra admirável: Consagro ao Rei as minhas obras: Minha língua é como a pena de um escritor expedito.
}\switchcolumn*\latim{
Allelúja, allelúja. ℣. \emph{Ps. 92, 1} Dóminus regnávit, decórem índuit: índuit Dóminus fortitúdinem, et præcínxit se virtúte. Allelúja.
}\switchcolumn\portugues{
Aleluia, aleluia. \emph{Sl. 92, 1} O Senhor revestiu-se de glória e reina: O Senhor revestiu-se de autoridade e cingiu-se com o poder. Aleluia.
}\end{paracol}

\paragraphinfo{Evangelho}{Lc. 2, 33-40}

\begin{paracol}{2}\latim{
\cruz Sequéntia sancti Evangélii secúndum Lucam.
}\switchcolumn\portugues{
\cruz Continuação do santo Evangelho segundo S. Lucas.
}\switchcolumn*\latim{
\blettrine{I}{n} illo témpore: Erat Joseph et Maria Mater Jesu, mirántes super his quæ dicebántur de illo. Et benedíxit illis Símeon, et dixit ad Maríam Matrem ejus: Ecce, pósitus est hic in ruínam et in resurrectiónem multórum in Israël: et in signum, cui contradicétur: et tuam ipsíus ánimam pertransíbit gládius, ut reveléntur ex multis córdibus cogitatiónes. Et erat Anna prophetíssa, fília Phánuel, de tribu Aser: hæc procésserat in diébus multis, et víxerat cum viro suo annis septem a virginitáte sua. Et hæc vídua usque ad annos octogínta quátuor: quæ non discedébat de templo, jejúniis et obsecratiónibus sérviens nocte ac die. Et hæc, ipsa hora supervéniens, confitebátur Dómino, et loquebátur de illo ómnibus, qui exspectábant redemptiónem Israël. Et ut perfecérunt ómnia secúndum legem Dómini, revérsi sunt in Galilǽam in civitátem suam Názareth. Puer autem crescébat, et confortabátur, plenus sapiéntia: et grátia Dei erat in illo.
}\switchcolumn\portugues{
\blettrine{N}{aquele} tempo, José e Maria, Mãe de Jesus, estavam admirados do que se dizia de Jesus. Simeão abençoou-os e disse a Maria, sua Mãe: «Este Menino veio ao mundo para ruína e salvação de muitos de Israel. Ele será como que um sinal de contradição. Uma espada de dor por sua causa traspassará a vossa alma, a fim de que os pensamentos de muitos homens, que estavam ocultos no íntimo do seu coração, sejam revelados». Ora, estava ali a profetiza Ana, filha de Fanuel, da tribo de Aser, que era de idade avançada (a qual vivera sete anos com seu marido, que havia desposado após a sua virgindade), sendo viúva e contando oitenta e quatro anos de idade. Ela não abandonava o templo, servindo a Deus, noite e dia, com jejuns e orações. Tendo ela chegado naquela ocasião, louvava o Senhor e falava de Jesus a todos quantos esperavam a salvação de Israel. Depois que eles cumpriram tudo o que a lei do Senhor preceituava, regressaram à cidade de Nazaré, na Galileia. Entretanto, o Menino crescia, fortalecia-se e era cheio de sabedoria: e a graça de Deus habitava n’Ele.
}\end{paracol}

\paragraphinfo{Ofertório}{Sl. 92, 1-2}
\begin{paracol}{2}\latim{
\rlettrine{D}{eus} firmávit orbem terræ, qui non commovébitur: paráta sedes tua, Deus, ex tunc, a sǽculo tu es.
}\switchcolumn\portugues{
\rlettrine{D}{eus} firmou de tal modo o globo terrestre, que nunca mais será destruído. Ó Deus, o vosso trono estava elevado desde a eternidade, pois Vós existis antes dos séculos.
}\end{paracol}

\paragraph{Secreta}

\begin{paracol}{2}\latim{
\rlettrine{C}{oncéde,} quǽsumus, omnípotens Deus: ut óculis tuæ majestátis munus oblátum, et grátiam nobis piæ devotiónis obtineat, et efféctum beátæ perennitátis acquírat. Per Dóminum \emph{\&c.}
}\switchcolumn\portugues{
\slettrine{Ó}{} Deus omnipotente, Vos pedimos, permiti que as oblatas, que apresentamos ante os olhos de vossa majestade, nos alcancem a graça de uma piedosa devoção e nos assegurem a posse da eterna felicidade. Por nosso Senhor \emph{\&c.}
}\end{paracol}

\paragraphinfo{Comúnio}{Mt. 2, 20}

\begin{paracol}{2}\latim{
\rlettrine{T}{olle} Púerum et Matrem ejus, et vade in terram Israël: defúncti sunt enim, qui quærébant ánimam Púeri.
}\switchcolumn\portugues{
\rlettrine{T}{oma} o Menino e sua Mãe, e volta para a terra de Israel, pois são mortos os que queriam atentar contra a vida do Menino.
}\end{paracol}

\paragraph{Postcomúnio}

\begin{paracol}{2}\latim{
\rlettrine{P}{er} hujus, Dómine, operatiónem mystérii, et vitia nostra purgéntur, et justa desidéria compleántur. Per Dóminum \emph{\&c.}
}\switchcolumn\portugues{
\rlettrine{S}{enhor,} que pela virtude destes divinos mystérios sejamos purificados dos nossos vícios, e que vejamos realizados os nossos justos desejos. Por nosso Senhor \emph{\&c.}
}\end{paracol}

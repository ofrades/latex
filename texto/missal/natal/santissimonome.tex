\subsectioninfo{Santíssimo Nome de Jesus}{Domingo entre a Festa da Circuncisão e a Epifania}

\paragraphinfo{Intróito}{Fl. 2, 10-11}
\begin{paracol}{2}\latim{
\rlettrine{I}{n} nómine Jesu omne genu flectátur, cœléstium, terréstrium et infernórum: et omnis lingua confiteátur, quia Dóminus Jesus Christus in glória est Dei Patris. \emph{Ps. 8, 2} Dómine, Dóminus noster, quam admirábile est nomen tuum in univérsa terra!
℣. Gloria Patri \emph{\&c.}
}\switchcolumn\portugues{
\qlettrine{Q}{ue} ao ser pronunciado o Nome de Jesus se dobrem todos os joelhos dos que estão no céu, na terra e no inferno: e que toda a língua humana confesse que o Senhor Jesus Cristo está na glória de Deus Pai. \emph{Ps. 8, 2} Senhor, nosso Senhor, como o vosso Nome é admirável em todo o universo!
℣. Glória ao Pai \emph{\&c.}
}\end{paracol}

\paragraph{Oração}
\begin{paracol}{2}\latim{
\rlettrine{D}{eus,} qui unigénitum Fílium tuum constituísti hu máni géneris Salvatórem, ei Jesum vocári jussísti: concéde propítius; ut, cujus sanctum nomen venerámur in terris, ejus quoque aspéctu perfruámur in cœlis. Per eúndem Dóminum \emph{\&c.}
}\switchcolumn\portugues{
\slettrine{Ó}{} Deus, que constituístes o vosso Filho Unigénito Salvador do mundo e ordenastes que fosse chamado Jesus, concedei-nos propício que, venerando nós o seu Santo Nome na terra, gozemos também a sua presença nos céus. Pelo mesmo nosso Senhor \emph{\&c.}
}\end{paracol}

\paragraphinfo{Epístola}{Act. 4, 8-12}
\begin{paracol}{2}\latim{
Léctio Actuum Apostolorum.
}\switchcolumn\portugues{
Lição dos Actos dos Apóstolos.
}\switchcolumn*\latim{
\rlettrine{I}{n} diébus illis: Petrus, replétus Spíritu Sancto, dixit: Príncipes pópuli et senióres, audíte: Si nos hódie dijudicámur in benefácto hóminis infírmi, in quo iste salvus factus est, notum sit ómnibus vobis et omni plebi Israël: quia in nómine Dómini nostri Jesu Christi Nazaréni, quem vos crucifixístis, quem Deus suscitávit a mórtuis, in hoc iste astat coram vobis sanus. Hic est lapis, qui reprobátus est a vobis ædificántibus: qui factus est in caput ánguli: et non est in alio áliquo salus. Nec enim aliud nomen est sub cœlo datum homínibus, in quo opórteat nos salvos fíeri.
}\switchcolumn\portugues{
\rlettrine{N}{aqueles} dias, Pedro, cheio de Espírito Santo, disse: «Príncipes do povo e anciãos de Israel, escutai: Visto que somos hoje julgados por causa dum milagre, concedido a um homem enfermo, para saber por quem foi curado, sabei, vós todos e todo o povo de Israel, que foi curado pelo Nome de Jesus Cristo Nazareno, a quem crucificastes, e que Deus ressuscitou dos mortos. É por Ele que este homem se apresenta diante de vós, plenamente curado. Este Jesus é a pedra que vós, querendo edificar, desprezastes, e que se tornou a pedra principal do ângulo. Não há salvação em nenhum outro nome, pois não existe debaixo do céu outro nome, que tenha sido dado aos homens, pelo qual devamos ser salvos».
}\end{paracol}

\paragraphinfo{Gradual}{Sl. 105, 47}
\begin{paracol}{2}\latim{
\rlettrine{S}{alvos} fac nos, Dómine, Deus noster, et cóngrega nos de natiónibus: ut confiteámur nómini sancto tuo, et gloriémur in glória tua. ℣. \emph{Isai. 63, 16} Tu, Dómine, Pater noster et Redémptor noster: a sǽculo nomen tuum.
}\switchcolumn\portugues{
\rlettrine{S}{alvai-nos,}ó Senhor, nosso Deus, e reuni-nos todos no meio das nações, a fim de que confessemos o vosso Santo Nome e nos gloriemos com vossos louvores. ℣. \emph{Isai. 63, 16} Senhor, sois o nosso Pai e Redentor! vosso Nome é eterno!
}\switchcolumn*\latim{
Allelúja, allelúja. ℣. \emph{Ps. 144, 21} Laudem Dómini loquétur os meum, et benedícat omnis caro nomen sanctum ejus. Allelúja.
}\switchcolumn\portugues{
Aleluia, aleluia. ℣. \emph{Sl. 144, 21} Que minha boca anuncie os louvores do Senhor: que toda a carne humana bendiga o seu Santo Nome. Aleluia.
}\end{paracol}

\paragraphinfo{Evangelho}{Lc. 2, 21}

\begin{paracol}{2}\latim{
\cruz Sequéntia sancti Evangélii secúndum Lucam.
}\switchcolumn\portugues{
\cruz Continuação do santo Evangelho segundo S. Lucas.
}\switchcolumn*\latim{
\blettrine{I}{n} illo témpore: Postquam consummáti sunt dies octo, ut circumciderétur Puer: vocátum est nomen ejus Jesus, quod vocátum est ab Angelo, priúsquam in útero conciperétur.
}\switchcolumn\portugues{
\blettrine{N}{aquele} tempo, passados que foram oito dias depois dos quais o Menino devia ser circuncidado, foi-Lhe dado o nome de Jesus, que foi aquele que o Anjo Lhe havia dado, antes de ser concebido no seio de sua Mãe.
}\end{paracol}

 \paragraphinfo{Ofertório}{Sl. 85, 12 \& 5}

\begin{paracol}{2}\latim{
\rlettrine{C}{onfitébor} tibi, Dómine, Deus meus, in toto corde meo, et glorificábo nomen tuum in ætérnum: quóniam tu, Dómine, suávis et mitis es: et multæ misericórdiæ ómnibus invocántibus te, allelúja.
}\switchcolumn\portugues{
\rlettrine{L}{ouvar-Vos-ei} de todo meu coração, ó Senhor, meu Deus: e glorificarei eternamente o vosso Nome; pois Vós, ó Senhor, sois cheio de clemência e de bondade: e as vossas misericórdias estendem-se a todos quantos Vos invocam. Aleluia.
}\end{paracol}

\paragraph{Secreta}

\begin{paracol}{2}\latim{
\rlettrine{B}{enedíctio} tua, clementíssime Deus, qua omnis viget creatúra, sanctíficet, quǽsumus, hoc sacrifícium nostrum, quod ad glóriam nóminis Fílii tui, Dómini nostri Jesu Christi, offérimus tibi: ut majestáti tuæ placére possit ad laudem, et nobis profícere ad salútem. Per eúndem Dóminum \emph{\&c.}
}\switchcolumn\portugues{
\slettrine{Ó}{} Deus clementíssimo, que a vossa bênção, que dá a vida a todas as criaturas, santifique este nosso sacrifício, que oferecemos em glória do Nome do vosso Filho, nosso Senhor Jesus Cristo, a fim de que ele possa honrar a vossa majestade, e, agradando-lhe, seja útil à nossa salvação. Pelo mesmo nosso Senhor \emph{\&c.}
}\end{paracol}

\paragraphinfo{Comúnio}{Sl. 85, 9-10}

\begin{paracol}{2}\latim{
\rlettrine{O}{mnes} gentes, quascúmque fecísti, vénient et adorábunt coram te, Dómine, et glorificábunt nomen tuum: quóniam magnus es tu et fáciens mirabília: tu es Deus solus, allelúja.
}\switchcolumn\portugues{
\rlettrine{T}{odos} os povos, que criastes, virão e se prostrarão diante de Vós, Senhor, adorando-Vos e glorificando o vosso Nome; pois sois grande e praticais prodígios. Só Vós sois Deus. Aleluia.
}\end{paracol}

\paragraph{Postcomúnio}

\begin{paracol}{2}\latim{
\rlettrine{O}{} mnípotens ætérae Deus, qui creásti et redemísti nos, réspice propítius vota nostra: et sacrifícium salutáris hóstiæ, quod in honórem nóminis Fílii tui, Dómini nostri Jesu Christi, majestáti tuæ obtúlimus, plácido et benígno vultu suscípere dignéris; ut grátia tua nobis infúsa, sub glorióso nómine Jesu, ætérnæ prædestinatiónis titulo gaudeámus nómina nostra scripta esse in cœlis. Per eúndem Dóminu \emph{\&c.}
}\switchcolumn\portugues{
\rlettrine{D}{eus} omnipotente e eterno, que nos criastes e resgatastes, olhai benignamente para as nossas ofertas e dignai-Vos aceitar com ânimo suave e piedoso o sacrifício da Hóstia salutar, que oferecemos à vossa majestade em honra do Nome do vosso Filho, nosso Senhor Jesus Cristo, a fim de que, infundida a vossa graça em nossas almas, possamos alegrar-nos, vendo os nossos nomes escritos nos céus, abaixo do glorioso Nome de Jesus, no livro da eterna predestinação. Pelo mesmo Senhor \emph{\&c.}
}\end{paracol}

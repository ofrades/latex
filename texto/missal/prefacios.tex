\subsection{Prefácios}\label{prefacios}

\paragraph{Prefácio do Natal}\label{prefacionatal}

\textit{Diz-se desde o Natal até à Epifania, exclusivamente; na Festa do Corpo de Deus e no seu Oitavário; na festa do Santíssimo Nome de Jesus; na Festa da Transfiguração de Nosso Senhor; na Festa da Purificação de Nossa Senhora; e nas Missas Votivas do Santíssimo Sacramento.}

\begin{paracol}{2}\latim{
\rlettrine{V}{ere} dignum et justum est, æquum et salutáre, nos tibi semper et ubíque grátias ágere: Dómine sancte, Pater omnípotens, ætérne Deus: Quia per incarnáti Verbi mystérium nova mentis nostræ óculis lux tuæ claritátis infúlsit: ut, dum visibíliter Deum cognóscimus, per hunc in invisibílium amorem rapiámur. Et ideo cum Angelis et Archángelis, cum Thronis et Dominatiónibus cumque omni milítia cœléstis exércitus hymnum glóriæ tuæ cánimus, sine fine dicéntes.
}\switchcolumn\portugues{
\rlettrine{V}{erdadeiramente} é digno e justo, racional e salutar dar-Vos graças, Senhor santo, Pai omnipotente, eterno Deus, em todos os lugares e sempre, porque pelo mystério do Verbo Incarnado um novo clarão do vosso esplendor iluminou a nossa alma, a fim de que, ao passo que conhecemos Deus de uma maneira visível, sejamos inspirados por Ele no amor às cousas invisíveis. E, por isso, com os Anjos e Arcanjos, com os Tronos e Dominações e com toda a milícia do exército celestial, cantamos o hino da vossa glória, dizendo incessantemente:
}\end{paracol}

\paragraph{Prefácio da Epifania}\label{prefacioepifania}

\textit{Diz-se na Festa da Epifania e durante o Oitavário.}

\begin{paracol}{2}\latim{
\rlettrine{V}{ere} dignum et justum est, æquum et salutáre, nos tibi semper et ubique grátias agere: Dómine sancte, Pater omnípotens, ætérne Deus: Quia, cum Unigenitus tuus in substántia nostræ mortalitátis appáruit, nova nos immortalitátis suæ luce reparávit. Et ídeo cum Angelis et Archángelis, cum Thronis et Dominatiónibus cumque omni milítia cœléstis exércitus hymnum glóriæ tuæ cánimus, sine fine dicéntes:
}\switchcolumn\portugues{
\rlettrine{V}{erdadeiramente} é digno e justo, racional e salutar dar-Vos graças, Senhor santo, Pai omnipotente, eterno Deus, em todos os lugares e sempre: Pois o vosso Filho Unigénito, revestido com a substancia da nossa carne mortal, reparou as faltas da natureza humana, comunicando-lhe o novo esplendor da sua imortalidade. E, por isso, com os Anjos e Arcanjos, com os Tronos e Dominações e com toda a milícia do exército celestial, cantamos o hino da vossa glória, dizendo incessantemente:
}\end{paracol}

\paragraph{Prefácio da Quaresma}\label{prefacioquaresma}

\textit{Diz-se desde Quarta-Feira das Cinzas até à véspera do Domingo da Paixão, inclusivamente.}

\begin{paracol}{2}\latim{
\rlettrine{V}{ere} dignum et justum est, æquum et salutáre, nos tibi semper et ubíque grátias ágere: Dómine sancte, Pater omnípotens, ætérne Deus: Qui corporáli jejúnio vitia cómprimis, mentem élevas, virtútem largíris et prǽmia: per Christum, Dóminum nostrum. Per quem majestátem tuam laudant Angeli, adórant Dominatiónes, tremunt Potestátes. Cœli cœlorúmque Virtútes ac beáta Séraphim sócia exsultatióne concélebrant. Cum quibus et nostras voces ut admítti júbeas, deprecámur, súpplici confessióne dicéntes:
}\switchcolumn\portugues{
\rlettrine{V}{erdadeiramente} é digno e justo, racional e salutar dar-Vos graças em todos os lugares e sempre, Senhor santo, Pai omnipotente, eterno Deus, pois Vós, pelo jejum natural, reprimis os vícios, elevais o espírito e concedeis-nos a virtude e o prémio por Cristo, nosso Senhor: pelo qual os Anjos louvam a vossa majestade, as Dominações a adoram e as Potestades trémulas a reverenciam; e os Céus, as Virtudes dos Céus e os bem-aventurados Serafins se associam em comum louvor. Dignai-Vos permitir, Senhor, que as nossas vozes suplicantes se unam às deles, dizendo:
}\end{paracol}

\paragraph{Prefácio da Santa Cruz}\label{prefaciosantacruz}

\textit{Diz-se quotidianamente desde Domingo da Paixão até Quinta-Feira Santa, excepto nas Festas de Nossa Senhora e de S. José.}

\begin{paracol}{2}\latim{
\rlettrine{V}{ere} dignum et justum est, æquum et salutáre, nos tibi semper et ubíque grátias ágere: Dómine sancte, Pater omnípotens, ætérne Deus: Qui salútem humáni géneris in ligno Crucis constituísti: ut, unde mors oriebátur, inde vita resúrgeret: et, qui in ligno vincébat, in ligno quoque vincerétur: per Christum, Dóminum nostrum. Per quem majestátem tuam laudant Angeli, adórant Dominatiónes, tremunt Potestátes. Cœli cœlorúmque Virtútes ac beáta Séraphim sócia exsultatióne concélebrant. Cum quibus et nostras voces ut admítti júbeas, deprecámur, súpplici confessióne dicéntes:
}\switchcolumn\portugues{
\slettrine{É}{} verdadeiramente digno e justo, racional e salutar render-Vos graças em todos os lugares e sempre, ó Senhor santo, Pai omnipotente, eterno Deus, que estabelecestes na árvore da Cruz a salvação do género humano, para que renascesse a vida, onde a morte houvera princípio, e que aquilo que outrora vencera na árvore, fosse vencido na árvore também, por Jesus Cristo, nosso Senhor: pelo qual os Anjos louvam a vossa majestade, as Dominações a adoram, as Potestades a reverenciam, os Céus, as Virtudes dos Céus e os bem-aventurados Serafins a celebram em comuns transportes de alegria: aos quais, Vos suplicamos, permiti que se unam as nossas vozes, dizendo em humilde e suplicante confissão:
}\end{paracol}

\paragraph{Prefácio da Páscoa}\label{prefaciopascoa}

\textit{Diz-se desde Domingo de Páscoa até à Vigília da Ascensão, inclusivamente, excepto quando há Prefácio próprio.
Intercala-se: hac potíssimum die (naquele dia); hac potíssimum nocte (naquela noite); hac potíssimum (neste tempo).}

\begin{paracol}{2}\latim{
\rlettrine{V}{ere} dignum et justum est, æquum et salutáre: Te quidem, Dómine, omni témpore, sed in hac potissímum die (vel in hoc potíssimum) gloriósius prædicáre, cum Pascha nostrum immolátus est Christus. Ipse enim verus est Agnus, qui ábstulit peccáta mundi. Qui mortem nostram moriéndo destrúxit et vitam resurgéndo reparávit. Et ídeo cum Angelis et Archángelis, cum Thronis et Dominatiónibus cumque omni milítia cœléstis exércitus hymnum glóriæ tuæ cánimus, sine fine dicéntes:
}\switchcolumn\portugues{
\slettrine{É}{} verdadeiramente digno e justo, racional e salutar, que Vos louvemos sempre, mas principalmente... em que Jesus Cristo foi imolado, como nova Páscoa. Pois Ele é o verdadeiro Cordeiro que tirou os pecados do mundo e que pela sua Ressurreição nos restituiu a vida. Por isso, com os Anjos e Arcanjos, com os Tronos e Dominações e com toda a milícia do exército celestial, cantamos o hino da vossa glória, dizendo incessantemente:
}\end{paracol}

\paragraph{Prefácio da Ascensão}\label{prefacioascensao}

\textit{Diz-se desde a Ascensão até à Vigília de Pentecostes, excepto quando há Prefácio próprio.}

\begin{paracol}{2}\latim{
\rlettrine{V}{ere} dignum et justum est, æquum et salutáre, nos tibi semper et ubíque grátias ágere: Dómine sancte, Pater omnípotens, ætérne Deus: per Christum, Dóminum nostrum. Qui post resurrectiónem suam ómnibus discípulis suis maniféstus appáruit et, ipsis cernéntibus, est elevátus in cœlum, ut nos divinitátis suæ tribúeret esse partícipes. Et ídeo cum Angelis et Archángelis, cum Thronis et Dominatiónibus cumque omni milítia cœléstis exércitus hymnum glóriæ tuæ cánimus, sine fine dicéntes:
}\switchcolumn\portugues{
\rlettrine{V}{erdadeiramente} é digno e justo, racional e salutar que em todos os lugares e sempre Vos rendamos graças, Senhor santo, Pai omnipotente, eterno Deus, por Jesus Cristo, nosso Senhor, que depois da sua Ressurreição apareceu visivelmente a todos seus discípulos, em cuja presença subiu ao céu, a fim de nos tornar participantes da sua divindade. E, por isso, unidos aos Anjos e Arcanjos, aos Tronos e Dominações e a toda a milícia do exército celestial, cantamos um hino em vossa honra, dizendo incessantemente:
}\end{paracol}

\paragraph{Prefácio do SS. Coração de Jesus}\label{prefaciocoracaojesus}

\textit{Diz-se na Missa da festa do Sagrado Coração de Jesus e seu Oitavário e nas Missas votivas do Sagrado Coração de Jesus.}

\begin{paracol}{2}\latim{
\rlettrine{V}{ere} dignum et justum est, æquum et salutáre, nos tibi semper et ubíque grátias ágere: Dómine sancte, Pater omnípotens, ætérne Deus: Qui Unigénitum tuum, in Cruce pendéntem, láncea mílitis transfígi voluísti: ut apértum Cor, divínæ largitátis sacrárium, torréntes nobis fúnderet miseratiónis et grátiæ: et, quod amóre nostri flagráre numquam déstitit, piis esset réquies et pœniténtibus pater et salútis refúgium. Et ídeo cum Angelis et Archángelis, cum Thronis et Dominatiónibus cumque omni milítia cœléstis exércitus hymnum glóriæ tuæ cánimus, sine fine dicéntes:
}\switchcolumn\portugues{
\slettrine{É}{} verdadeiramente digno e justo, racional e salutar que sempre e em todos os lugares Vos demos graças, Senhor santo, Pai omnipotente, eterno Deus, que quisestes que o vosso Filho Unigénito, suspenso na Cruz, fosse trespassado pela lança do soldado, a fim de que, aberto o Coração - sacrário da liberdade divina -, lançasse sobre nós torrentes de misericórdia e de graça; e, ardendo incessantemente de amor por nós, sirva de descanso aos piedosos e de asilo de salvação aos penitentes. E por isso, com os Anjos e Arcanjos, com os Tronos e Dominações e com toda a milícia celestial, cantamos o hino da vossa glória, dizendo sem cessar:
}\end{paracol}

\paragraph{Prefácio de N. S. Jesus Cristo-Rei}\label{prefaciocristorei}

\textit{Diz-se na Missa de N. S. Jesus Cristo-Rei.}

\begin{paracol}{2}\latim{
\rlettrine{V}{ere} dignum et justum est, æquum et salutáre, nos tibi semper et ubíque grátias ágere: Dómine sancte, Pater omnípotens, ætérne Deus: Qui unigénitum Fílium tuum, Dóminum nostrum Jesum Christum, Sacerdótem ætérnum et universórum Regem, óleo exsultatiónis unxísti: ut, seípsum in ara crucis hóstiam immaculátam et pacíficam ófferens, redemptiónis humánæ sacraménta perágeret: et suo subjéctis império ómnibus creatúris, ætérnum et universále regnum, imménsæ tuæ tráderet Majestáti. Regnum veritátis et vitæ: regnum sanctitátis et grátiæ: regnum justítiæ, amóris et pacis. Et ídeo cum Angelis et Archángelis, cum Thronis et Dominatiónibus cumque omni milítia cœléstis exércitus hymnum glóriæ tuæ cánimus, sine fine dicéntes:
}\switchcolumn\portugues{
\slettrine{É}{} verdadeiramente digno e justo, racional e salutar dar-Vos graças em todos os lugares e sempre, Senhor santo, Pai omnipotente, eterno Deus, que ungistes com o óleo da alegria o vosso Filho Unigénito, nosso Senhor Jesus Cristo, como Sacerdote eterno e Rei de todas as cousas, a fim de que, oferecendo-se na ara da Cruz - qual hóstia imaculada e pacífica -, realizasse as maravilhas da redenção humana e, ficando todas as criaturas sujeitas aoseu império, desse à vossa imensa majestade um reino eterno e universal: um reino de verdade e de vida; um reino de santidade e de graça; um reino de justiça, de amor e de paz. E por isso, com os Anjos e Arcanjos, com os Tronos e Dominações e com toda a milícia do exército celestial, cantamos um hino à vossa glória, dizendo incessantemente:
}\end{paracol}

\paragraph{Prefácio do Pentecostes}\label{prefaciopentecostes}

\textit{Diz-se desde a Vigília de Pentecostes até ao fim do Oitavário. Diz-se também nas Missas Votivas do Espírito Santo, mas omitem-se as palavras: neste dia.}

\begin{paracol}{2}\latim{
\rlettrine{V}{ere} dignum et justum est, æquum et salutáre, nos tibi semper et ubíque grátias ágere: Dómine sancte, Pater omnípotens, ætérne Deus: per Christum, Dóminum nostrum. Qui, ascéndens super omnes cœlos sedénsque ad déxteram tuam, promíssum Spíritum Sanctum (hodiérna die) in fílios adoptiónis effúdit. Quaprópter profúsis gáudiis totus in orbe terrárum mundus exsúltat. Sed et supérnæ Virtútes atque angélicæ Potestátes hymnum glóriæ tuæ cóncinunt, sine fine dicéntes:
}\switchcolumn\portugues{
\slettrine{É}{} verdadeiramente digno e justo, racional e salutar dar-Vos graças, em todos os lugares e sempre, Senhor santo, Pai omnipotente, eterno Deus, por nosso Senhor Jesus Cristo, que, subindo ao mais alto dos céus e estando sentado à vossa direita, fez descer, (neste dia) sobre os seus filhos adoptivos o Espírito Santo, como havia prometido. Por isso o mundo inteiro, em transportes de alegria, exulta de contentamento, enquanto as Virtudes do céu e as Potestades angelicais cantam um hino à vossa glória, dizendo incessantemente:
}\end{paracol}

\paragraph{Prefácio da SS. Trindade}\label{prefaciotrindade}

\textit{Diz-se na Festa da Santíssima Trindade e nos Domingos que não tiverem Prefácio próprio.}

\begin{paracol}{2}\latim{
\rlettrine{V}{ere} dignum et justum est, æquum et salutáre, nos tibi semper et ubíque grátias ágere: Dómine sancte, Pater omnípotens, ætérne Deus: Qui cum unigénito Fílio tuo et Spíritu Sancto unus es Deus, unus es Dóminus: non in unius singularitáte persónæ, sed in uníus Trinitáte substántiæ. Quod enim de tua glória, revelánte te, crédimus, hoc de Fílio tuo, hoc de Spíritu Sancto sine differéntia discretiónis sentímus. Ut in confessióne veræ sempiternǽque Deitátis, et in persónis propríetas, et in esséntia únitas, et in majestáte adorétur æquálitas. Quam laudant Angeli atque Archángeli, Chérubim quoque ac Séraphim: qui non cessant clamáre cotídie, una voce dicéntes:
}\switchcolumn\portugues{
\slettrine{É}{} verdadeiramente digno e justo, racional e salutar render-Vos graças em todos os lugares e sempre, ó Senhor santo, Pai omnipotente, eterno Deus, que com vosso Filho Unigénito e com o Espírito Santo sois um só Deus, um só Senhor, não na unidade de uma só pessoa, mas na Trindade de uma só substância. Porquanto, o que acreditamos a respeito da vossa glória, acreditamo-lo também, pela vossa revelação, a respeito do vosso Filho e do Espírito Santo, de tal modo que, confessando a verdadeira e eterna divindade, adoramos nas pessoas a propriedade, na essência a unidade e a igualdade na majestade. É esta majestade, que louvam os Anjos e Arcanjos, os Querubins e Serafins, que não cessam quotidianamente de cantar em uníssono:
}\end{paracol}

\paragraph{Prefácio da B. Virgem Maria}\label{prefaciomaria}

\textit{Diz-se nas Festas da Santíssima Virgem (excepto na da Purificação) e nos seus Oitavários, alterando-se, porém, algumas palavras (segundo a Festa que se celebra), como se diz na nota de rodapé.}

\begin{paracol}{2}\latim{
\rlettrine{V}{ere} dignum et justum est, æquum et salutáre, nos tibi semper et ubique grátias ágere: Dómine sancte, Pater omnípotens, ætérne Deus: Et te in \footnote[2]{Conceptióne immaculáta - na Imaculada Conceição; Præsentatióne - na Apresentação; Visitatióne - na Visitação; Anuntiatióne - na Anunciação; Nativitáte - na Natividade; Festa das Dores diz-se: Transfixióne - na transfixão; Festa de N. S. do Monte Carmelo: Solemnitáte - na Solenidade; Missas Votivas diz-se: Veneratióne - em Veneração.} beátæ Maríæ semper Vírginis collaudáre, benedícere et prædicáre. Quæ et Unigénitum tuum Sancti Spíritus obumbratióne concépit: et, virginitátis glória permanénte, lumen ætérnum mundo effúdit, Jesum Christum, Dóminum nostrum. Per quem majestátem tuam laudant Angeli, adórant Dominatiónes, tremunt Potestátes. Cœli cœlorúmque Virtútes ac beáta Séraphim sócia exsultatióne concélebrant. Cum quibus et nostras voces ut admitti jubeas, deprecámur, súpplici confessióne dicéntes:
}\switchcolumn\portugues{
\slettrine{É}{} verdadeiramente digno e justo, racional e salutar render-Vos graças em todos os lugares e sempre, ó Senhor santo, Pai omnipotente, eterno Deus, e de sempre Vos louvar, bendizer e anunciar \footnote[2]{Conceptióne immaculáta - na Imaculada Conceição; Præsentatióne - na Apresentação; Visitatióne - na Visitação; Anuntiatióne - na Anunciação; Nativitáte - na Natividade; Festa das Dores diz-se: Transfixióne - na transfixão; Festa de N. S. do Monte Carmelo: Solemnitáte - na Solenidade; Missas Votivas diz-se: Veneratióne - em Veneração.} da Bem-aventurada Maria, sempre Virgem. Foi ela quem concebeu o vosso Filho Unigénito por obra do Espírito Santo, e, sem a mais leve perda de glória da virgindade, deu ao mundo a Luz eterna, Jesus Cristo, nosso Senhor, por quem os Anjos louvam a vossa majestade, as Dominações a adoram, as Potestades a reverenciam, os Céus, as Virtudes dos céus e os bem-aventurados Serafins a festejam em transportes de alegria. E, Vos imploramos, permiti que unamos as nossas vozes às de todos estes, dizendo em suplicante confissão:
}\end{paracol}

\paragraph{Prefácio do S. José}\label{prefaciojose}

\textit{Diz-se na Festa de S. José e do seu Patrocínio e Oitavário e nas Missas Votivas de S. José.
Nas Missas Votivas diz-se ...in Veneratióne (em Veneração) em vez de ...in Festivitáte (na Festividade).}

\begin{paracol}{2}\latim{
\rlettrine{V}{ere} dignum et justum est, æquum et salutáre, nos tibi semper et ubíque grátias ágere: Dómine sancte, Pater omnípotens, ætérne Deus: Et te in Festivitáte (Veneratióne) beáti Joseph débitis magnificáre præcóniis, benedícere et prædicáre. Qui et vir justus, a te Deíparæ Vírgini Sponsus est datus: et fidélis servus ac prudens, super Famíliam tuam est constitútus: ut Unigénitum tuum, Sancti Spíritus obumbratióne concéptum, paterna vice custodíret, Jesum Christum, Dóminum nostrum. Per quem majestátem tuam laudant Angeli, adórant Dominatiónes, tremunt Potestátes. Cœli cœlorúmque Virtútes ac beáta Séraphim sócia exsultatióne concélebrant. Cum quibus et nostras voces ut admítti júbeas, deprecámur, súpplici confessióne dicéntes:
}\switchcolumn\portugues{
\slettrine{É}{} verdadeiramente digno e justo, racional e salutar dar-Vos graças em todos os lugares e sempre, ó Senhor santo, Pai omnipotente, eterno Deus: e na festividade do bem-aventurado José devemos proclamar, como convém, as vossas grandezas, bendizer-Vos e louvar-Vos. É ele o homem justo, que destinastes para esposo da Virgem Mãe de Deus; é ele o servo fiel e prudente, que colocastes na Sagrada Família, a fim de que guardasse, como se fora pai, o vosso Filho Unigénito, Jesus Cristo, nosso Senhor, concebido por operação do Espírito Santo; pelo qual os Anjos louvam a vossa majestade, as Dominações a adoram, as Potestades se prostram reverentes, os Céus e os bem-aventurados Serafins a celebram em comuns transportes. Dignai-Vos permitir, Vos suplicamos, que as nossas vozes se unam às deles, dizendo em suplicante confissão:
}\end{paracol}

\paragraph{Prefácio dos Apóstolos}\label{prefacioapostolos}

\textit{Diz-se nas Festas dos Apóstolos e Evangelistas.}

\begin{paracol}{2}\latim{
\rlettrine{V}{ere} dignum et justum est, æquum et salutáre: Te, Dómine, supplíciter exoráre, ut gregem tuum, Pastor ætérne, non déseras: sed per beátos Apóstolos tuos contínua protectióne custódias. Ut iísdem rectóribus gubernétur, quos óperis tui vicários eídem contulísti præésse pastóres. Et ídeo cum Angelis et Archángelis, cum Thronis et Dominatiónibus cumque omni milítia cœléstis exércitus hymnum glóriæ tuæ cánimus, sine fine dicéntes:
}\switchcolumn\portugues{
\slettrine{É}{} verdadeiramente digno e justo, racional e salutar suplicar-Vos humildemente, Senhor, que, como Pastor eterno, que sois, não abandoneis o vosso rebanho, mas antes, por intercessão dos vossos bem-aventurados Apóstolos, o guardeis sob a vossa perpétua protecção, a fim de que seja sempre governado pelos mesmos Directores, que, encarregados como Vigários de perpetuar a vossa obra, designastes seus Pastores. E, por isso, com os Anjos e Arcanjos, com os Tronos e Dominações e com toda a milícia do exército celestial, cantamos um hino à vossa glória, dizendo incessantemente:
}\end{paracol}

\paragraph{Prefácio Comum}\label{prefaciocomum}

\textit{Diz-se em todas as Festas e Férias que não têm Prefácio próprio e até nas Missas dos Domingos, se não celebradas em outro dia da semana.}

\begin{paracol}{2}\latim{
\rlettrine{V}{ere} dignum et justum est, æquum et salutáre, nos tibi semper et ubíque grátias agere: Dómine sancte, Pater omnípotens, ætérne Deus: per Christum, Dóminum nostrum. Per quem majestátem tuam laudant Angeli, adórant Dominatiónes, tremunt Potestátes. Cœli cœlorúmque Virtútes ac beáta Séraphim sócia exsultatióne concélebrant. Cum quibus et nostras voces ut admitti jubeas, deprecámur, súpplici confessione dicéntes:
}\switchcolumn\portugues{
\slettrine{É}{} verdadeiramente digno e justo, racional e salutar render-Vos graças em todos os lugares e sempre, ó Senhor santo, Pai omnipotente, eterno Deus, por nosso Senhor Jesus Cristo. Pelo qual os Anjos louvam a vossa majestade, as Dominações a adoram e as Potestades se prostram reverentes; e os Céus, as Virtudes dos céus e os bem-aventurados Serafins a celebram em comum alegria e louvor. Dignai-Vos permitir, Vos imploramos, que as nossas vozes suplicantes se unam às deles, dizendo:
}\end{paracol}

\paragraph{Prefácio dos Defuntos}\label{prefaciodefuntos}

\textit{Diz-se em todas as Missas de Réquiem.}

\begin{paracol}{2}\latim{
\rlettrine{V}{ere} dignum et justum est, æquum et salutáre, nos tibi semper et ubíque grátias ágere: Dómine sancte, Pater omnípotens, ætérne Deus: per Christum, Dóminum nostrum. In quo nobis spes beátæ resurrectiónis effúlsit, ut, quos contrístat certa moriéndi condício, eósdem consolétur fu-túræ immortalitátis promíssio. Tuis enim fidélibus, Dómine, vita mutátur, non tóllitur: et, dissolúta terréstris hujus incolátus domo, ætérna in cœlis habitátio comparátur. Et ídeo cum Angelis et Archángelis, cum Thronis et Dominatiónibus cumque omni milítia cœléstis exércitus hymnum glóriæ tuæ cánimus, sine fine dicéntes:
}\switchcolumn\portugues{
\slettrine{É}{} verdadeiramente digno e justo, racional e salutar que sempre e em todos os lugares Vos dêmos graças, Senhor santo, Pai omnipotente, eterno Deus, por meio de nosso Senhor Jesus Cristo, em quem nos concedestes a esperança da feliz ressurreição; de sorte que, conquanto a condição certa da nossa morte nos entristeça, fiquemos consolados com a promessa da imortalidade futura. Pois, para os vossos fiéis, Senhor, a vida muda-se, não se acaba; e, desfeita esta Inorada terrena, adquire-se a habitação eterna nos céus. E, por isso, com os Anjos e Arcanjos, com os Tronos e Dominações e com toda a milícia do exército celestial, cantamos o hino da vossa glória, dizendo sem cessar:
}\end{paracol}

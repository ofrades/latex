\subsectioninfo{Um ou Muitos Sumos Pontífices}{Missa Si díligis me}\label{sumospontifices}

\paragraphinfo{Intróito}{Jo. 21, 15, 16 \& 17}
\begin{paracol}{2}\latim{
\rlettrine{S}{i} díligis me, Simon Petre, pasce agnos meos, pasce oves meas. (T. P. Allelúja, allelúja.) \emph{Ps. 29, 2} Exaltábo te, Dómine, quóniam suscepísti me, nec delectásti inimícos meos super me.
℣. Gloria Patri \emph{\&c.}
}\switchcolumn\portugues{
\rlettrine{S}{e} me amas, Simão-Pedro, apascenta os meus cordeiros, apascenta as minhas ovelhas. (T. P. Aleluia, aleluia.) \emph{Sl. 29, 2} Louvar-Vos-ei, Senhor, pois me acolhestes e não permitistes que meus inimigos se rissem de mim.
℣. Glória ao Pai \emph{\&c.}
}\end{paracol}

\paragraph{Oração}
\begin{paracol}{2}\latim{
\rlettrine{G}{regem} tuum, Pastor ætérne, placátus inténde: et, per beátum {\redx N.} (Mártyrem tuum atque) Summum Pontíficem, perpétua protectióne custódi; quem totíus Ecclésiæ præstitísti esse pastórem. Per Dóminum nostrum \emph{\&c.}
}\switchcolumn\portugues{
\slettrine{Ó}{} Pastor eterno, atendei propício ao vosso rebanho; e guardai-o com vossa perpétua protecção por intercessão do bem-aventurado {\redx N.} (Vosso Mártir e) Sumo Pontífice, o qual escolhestes como pastor de toda a Igreja. Por nosso Senhor \emph{\&c.}
}\end{paracol}

\textit{Se, porém, se fizer comemoração doutro Sumo Pontífice nesta mesma Missa, dir-se-á a seguinte Oração, em vez da Precedente:}

\paragraph{Oração}
\begin{paracol}{2}\latim{
\rlettrine{D}{eus,} qui Ecclésiam tuam, in apostólicæ petræ soliditáte fundátam, ab infernárum éruis terróre portárum: præsta, quǽsumus; ut, intercedénte beáto {\redx N.} (Mártyre tuo atque) Summo Pontífice, in tua veritáte persístens, contínua securitáte muniátur. Per Dominum \emph{\&c.}
}\switchcolumn\portugues{
\slettrine{Ó}{} Deus, que do terror das portas do inferno livrastes a vossa Igreja, fundada na solidez da pedra apostólica, concedei-nos, Vos suplicamos, que, por intercessão do bem-aventurado {\redx N.} (Vosso Mártir e) Sumo Pontífice, sempre persista na vossa verdade e seja protegida em contínua segurança. Por nosso Senhor \emph{\&c.}
}\end{paracol}

\paragraphinfo{Epístola}{l. Pe. 5, 1-4 \& 10-11}
\begin{paracol}{2}\latim{
Léctio Epístolæ beáti Petri Apóstoli.
}\switchcolumn\portugues{
Lição da Ep.ª do B. Ap.º Pedro.
}\switchcolumn*\latim{
\rlettrine{C}{aríssimi:} Senióres, qui in vobis sunt, obsécro consénior et testis Christi passiónum, qui et ejus, quæ in futúro revelánda est, glóriæ communicátor: páscite qui in vobis est gregem Dei, providéntes non coácte, sed spontánee secúndum Deum, neque turpis lucri grátia, sed voluntárie; neque ut dominántes in cleris, sed forma facti gregis ex ánimo. Et, cum appáruerit princeps pastórum, percipiétis immarcescíbilem glóriæ corónam. Deus autem
omnis grátiæ, qui vocávit nos in ætérnam suam glóriam in Christo Jesu, módicum passos ipse perfíciet, confirmábit solidabítque. Ipsi glória et impérium in sǽcula sæculórum. Amen.
}\switchcolumn\portugues{
\rlettrine{A}{os} sacerdotes que estão entre vós rogo eu, sacerdote como eles e testemunha dos sofrimentos de Cristo e que tomarei parte com eles naquela glória que será manifestada um dia: apascentai o rebanho de Deus que vos está confiado, cuidai dele não constrangidos, mas de boa vontade, segundo Deus; não por amor de lucro vil, mas por dedicação; não como para dominar sobre a herança (do Senhor), mas fazendo-vos modelos do rebanho. E quando aparecer o príncipe dos pastores, recebereis a coroa imarcescível da glória. Foi Deus de toda a graça que nos chamou em Jesus Cristo à sua eterna glória; e, depois de terdes sofrido um pouco, vos aperfeiçoará, fortificará e consolidará. A Ele: glória e império pelos séculos dos séculos. Amen.
}\end{paracol}

\paragraphinfo{Gradual}{Sl. 106, 32, 31}
\begin{paracol}{2}\latim{
\rlettrine{E}{xáltent} eum in Ecclésia plebis: et in cáthedra seniórum laudent eum. ℣. Confiteántur Dómino misericórdiæ ejus; et mirabília ejus fíliis hóminum.
}\switchcolumn\portugues{
\qlettrine{Q}{ue} seja exaltado na assembleia do povo; que seja louvado no conselho dos anciãos. ℣. Glorifiquem o Senhor pelas suas misericórdias: e pelas suas maravilhas em favor dos filhos dos homens.
}\switchcolumn*\latim{
Allelúja, allelúja. ℣. \emph{Matth. 16, 18} Tu es Petrus, et super hanc petram ædificábo Ecclésiam meam. Allelúja.
}\switchcolumn\portugues{
Aleluia, aleluia. ℣. \emph{Mt. 16, 18} Tu és Pedro, e sobre esta pedra edificarei a minha Igreja. Aleluia.
}\end{paracol}

\textit{Depois da Septuagésima omite-se o Aleluia e o Verso e diz-se:}

\paragraphinfo{Trato}{Sl. 39, 10-11}
\begin{paracol}{2}\latim{
\rlettrine{A}{nnuntiávi} justítiam tuam in ecclésia magna, ecce, lábia mea non prohibébo: Dómine, tu scisti. ℣. Justítiam tuam non abscóndi in corde meo: veritátem tuam et salutáre tuum dixi. ℣. Non abscóndi misericórdiam tuam, et veritátem tuam a concílio multo.
}\switchcolumn\portugues{
\rlettrine{A}{nunciei} a vossa justiça numa grande assembleia: eis, pois, que não cerrareis os meus lábios, Senhor, bem o sabeis. ℣. Não encerrei a vossa justiça no meu coração; mas publiquei a vossa verdade e salvação. ℣. Não ocultei a vossa misericórdia e fidelidade diante da grande assembleia.
}\end{paracol}

\textit{No Tempo Pascal omite-se o Gradual e o Trato e diz-se:}

\begin{paracol}{2}\latim{
Allelúja, allelúja. ℣. \emph{Matth. 16, 18} Tu es Petrus, et super hanc petram ædificábo Ecclésiam meam. Allelúja. ℣. \emph{Ps. 44, 17, 18} Constítues eos príncipes super omnem terram: mémores erunt nóminis tui, Dómine. Allelúja.
}\switchcolumn\portugues{
Aleluia, aleluia. ℣. \emph{Mt. 16, 18} Tu és Pedro, e sobre esta pedra edificarei a minha Igreja. Aleluia. ℣. \emph{Sl. 44, 17, 18} Vós os constituístes príncipes em toda a terra: e eles perpetuarão, Senhor, o vosso nome. Aleluia.
}\end{paracol}

\paragraphinfo{Evangelho}{Mt. 16, 13-19}
\begin{paracol}{2}\latim{
\cruz Sequéntia sancti Evangélii secúndum Matthǽum.
}\switchcolumn\portugues{
\cruz Continuação do santo Evangelho segundo S. Mateus.
}\switchcolumn*\latim{
\blettrine{I}{n} illo témpore: Venit Jesus in partes Cæsaréæ Philíppi, et interrogábat discípulos suos, dicens: Quem dicunt hómines esse Fílium hóminis? At illi dixérunt: Alii Joánnem Baptístam, alii autem Elíam, alii vero Jeremíam aut unum ex prophétis. Dicit illis Jesus: Vos autem quem me esse dícitis? Respóndens Simon Petrus, dixit: Tu es Christus, Fílius Dei vivi. Respóndens autem Jesus, dixit ei: Beátus es, Simon Bar Jona: quia caro et sanguis non revelávit tibi, sed Pater meus, qui in cœlis est. Et ego dico tibi, quia tu es Petrus, et super hanc petram ædificábo Ecclésiam meam, et portæ ínferi non prævalébunt advérsus eam. Et tibi dabo claves regni cœlórum. Et quodcúmque ligáveris super terram, erit ligátum et in cœlis: et quodcúmque sólveris super terram, erit solútum et in cœlis.
}\switchcolumn\portugues{
\blettrine{N}{aquele} tempo, foi Jesus para a região de Cesareia, de Filipe, e interrogou os seus discípulos, dizendo-lhes: «Quem dizem os homens que é o Filho do homem?». Eles responderam: «Uns dizem que é João Baptista, outros que é Elias e outros que é Jeremias ou algum dos Profetas». Jesus disse-lhes: «E quem dizeis vós que eu sou?». Respondendo, Simão-Pedro disse: «Tu és Cristo, Filho de Deus vivo!». E Jesus disse-lhe: «Bem-aventurado és tu, Simão Barjona, porque não foi a carne nem o sangue que te revelaram o que dizes, mas meu Pai, que está nos céus. E Eu digo-te: tu és Pedro, e sobre esta pedra edificarei a minha Igreja; e as portas do inferno não prevalecerão contra ela. Eu te darei as chaves do reino dos céus; e tudo o que ligares sobre a terra será ligado também nos céus; e tudo o que desatares sobre a terra será desatado também nos céus».
}\end{paracol}

\paragraphinfo{Ofertório}{Jr. 1, 9-10}
\begin{paracol}{2}\latim{
\rlettrine{E}{cce,} dedi verba mea in ore tuo: ecce, constítui te super gentes et super regna, ut evéllas et destruas, et ædífices et plantes. (T. P. Allelúja.)
}\switchcolumn\portugues{
\rlettrine{E}{is} que pus as minhas palavras na tua boca: eis que te constituí sobre os povos e sobre os reinos para arrancares e destruíres, e para edificares e plantares. (T. P. Aleluia.)
}\end{paracol}

\paragraph{Secreta}
\begin{paracol}{2}\latim{
\rlettrine{O}{blátis} munéribus, quǽsumus, Dómine, Ecclésiam tuam benígnus illúmina: ut, et gregis tui profíciat ubique succéssus, et grati fiant nómini tuo, te gubernánte, pastóres. Per Dóminum nostrum Jesum Christum, Fílium tuum: Qui tecum vivit et regnat \emph{\&c.}
}\switchcolumn\portugues{
\rlettrine{C}{om} as ofertas destes dons, Vos suplicamos, Senhor, iluminai benignamente a vossa Igreja, a fim de que não só o vosso rebanho triunfe em toda a parte, mas também pelo poder do vosso nome os pastores sejam bem acolhidos. Por nosso Senhor \emph{\&c.}
}\end{paracol}

\textit{Se, porém, se fizer comemoração doutro Sumo Pontífice nesta mesma Missa, dir-se-á a seguinte Secreta, em vez da Precedente:}

\paragraph{Secreta}
\begin{paracol}{2}\latim{
\rlettrine{M}{únera,} quæ tibi, Dómine, lætántes offérimus, súscipe benígnus, et præsta: ut, intercedénte beáto {\redx N.}, Ecclésia tua et fídei integritáte lætétur, et témporum tranquillitáte semper exsúltet. Per Dóminum nostrum \emph{\&c.}
}\switchcolumn\portugues{
\rlettrine{R}{ecebei} benignamente, Senhor, os dons que com alegria Vos oferecemos, e fazei que, por intercessão do bem-aventurado {\redx N.}, a vossa. Igreja se alegre com a integridade da sua fé e sempre exulte com a tranquilidade dos tempos. Por nosso Senhor \emph{\&c.}
}\end{paracol}

\paragraphinfo{Comúnio}{Mt. 16, 18}
\begin{paracol}{2}\latim{
\rlettrine{T}{u} es Petrus, et super hanc petram ædificábo Ecclésiam meam. (T. P. Allelúja.)
}\switchcolumn\portugues{
\rlettrine{T}{u} és Pedro, e sobre esta pedra edificarei a minha Igreja. (T. P. Aleluia.)
}\end{paracol}

\paragraph{Postcomúnio}
\begin{paracol}{2}\latim{
\rlettrine{R}{efectióne} sancta enutrítam gubérna, quǽsumus, Dómine, tuam placátus Ecclésiam: ut, poténti moderatióne dirécta, et increménta libertátis accípiat et in religiónis integritáte persístat. Per Dóminum nostrum \emph{\&c.}
}\switchcolumn\portugues{
\rlettrine{S}{enhor,} Vos suplicamos, governai com mansidão a vossa Igreja, agora que foi alimentada com a sagrada refeição, a fim de que, dirigida com firme suavidade, alcance o incremento da sua liberdade e persista na integridade da sua doutrina. Por nosso Senhor \emph{\&c.}
}\end{paracol}

\textit{Se, porém, se fizer comemoração doutro Sumo Pontífice nesta mesma Missa, dir-se-á o seguinte Postcomúnio, em vez do Precedente:}

\paragraph{Postcomúnio}
\begin{paracol}{2}\latim{
\rlettrine{M}{ultíplica,} quǽsumus, Dómine, in Ecclesia tua spíritum grátiæ, quem dedísti: ut beáti {\redx N.} (Martyris tui atque) Summi Pontíficis deprecatióne nec pastóri obœdiéntia gregis nec gregi desit cura pastóris. Per Dóminum \emph{\&c.}
}\switchcolumn\portugues{
\rlettrine{S}{enhor,} Vos suplicamos, multiplicai na vossa Igreja o espírito da graça, que lhe concedestes, a fim de que, pela oração do bem-aventurado {\redx N.} (Vosso Mártir e) Sumo Pontífice, não falte ao pastor a obediência do rebanho, nem ao rebanho a dedicação do pastor. Por nosso Senhor \emph{\&c.}
}\end{paracol}
\subsectioninfo{Confessores não Pontífices}{Missa Os justi}\label{confessoresnaopontifices1}

\paragraphinfo{Intróito}{Sl. 36, 30-31}
\begin{paracol}{2}\latim{
\rlettrine{O}{s} justi meditábitur sapiéntiam, et lingua ejus loquétur judícium: lex Dei ejus in corde ipsíus. (T. P. Allelúja, allelúja.) \emph{Ps. ibid., 1} Noli æmulári in malignántibus: neque zeláveris faciéntes iniquitátem.
℣. Gloria Patri \emph{\&c.}
}\switchcolumn\portugues{
\rlettrine{A}{} boca do justo fala com sabedoria e a sua língua proclama a justiça. A lei do seu Deus estará no seu coração. (T. P. Aleluia, aleluia.) \emph{Sl. ibid., 1} Não vos irriteis contra os maus, nem tenhais inveja daqueles que cometem iniquidades.
℣. Glória ao Pai \emph{\&c.}
}\end{paracol}

\paragraph{Oração}
\begin{paracol}{2}\latim{
\rlettrine{D}{eus,} qui nos beáti {\redx N.} Confessóris tui ánnua solemnitáte lætíficas: concéde propítius; ut, cujus natalítia cólimus, étiam actiónes imitémur. Per Dóminum \emph{\&c.}
}\switchcolumn\portugues{
\slettrine{Ó}{} Deus, que nos alegrais com a solenidade anual do B. {\redx N.}, vosso Confessor, visto que celebramos o seu nascimento, concedei-nos propício que imitemos também as suas acções. Por nosso Senhor \emph{\&c.}
}\end{paracol}

\paragraphinfo{Epístola}{Ecl. 31, 8-11}
\begin{paracol}{2}\latim{
Léctio libri Sapiéntiæ.
}\switchcolumn\portugues{
Lição do Livro da Sabedoria.
}\switchcolumn*\latim{
\rlettrine{B}{eátus} vir, qui invéntus est sine mácula, et qui post aurum non ábiit, nec sperávit in pecúnia et thesáuris. Quis est hic, et laudábimus eum? fecit enim mirabília in vita sua. Qui probátus est in illo, et perféctus est, erit illi glória ætérna: qui potuit tránsgredi, et non est transgréssus: fácere mala, et non fecit: ídeo stabilíta sunt bona illíus in Dómino, et eleemósynis illíus enarrábit omnis ecclésia sanctórum.
}\switchcolumn\portugues{
\rlettrine{B}{em-aventurado} o homem que foi julgado sem mácula; que não correu após o ouro; e não pôs as esperanças nem no dinheiro, nem nos tesouros! Quem é ele, para que o louvemos? Porquanto operou coisas admiráveis durante a vida. Aquele que foi provado pelo ouro e foi julgado perfeito alcançará a glória eterna; pois poderia ter violado o mandamento, e o não violou; poderia ter praticado acções más, e as não praticou. É por esta razão que seus bens lhe estarão assegurados no Senhor e que toda a assembleia dos justos proclamará as boas acções que praticou.
}\end{paracol}

\paragraphinfo{Gradual}{Sl. 91, 13 \& 14}
\begin{paracol}{2}\latim{
\qlettrine{J}{ustus} ut palma florébit: sicut cedrus Líbani multiplicábitur in domo Dómini. ℣. \emph{ibid., 3} Annuntiándum mane misericórdiam tuam, et veritátem tuam per noctem.
}\switchcolumn\portugues{
\rlettrine{O}{} justo florescerá, como a palmeira, e crescerá, como o cedro do Líbano, na casa do Senhor. ℣. \emph{ibid., 3} Para publicar de manhã a vossa misericórdia; e de noite a vossa verdade.
}\switchcolumn*\latim{
Allelúja, allelúja. ℣. \emph{Jac. 1, 12} Beátus vir, qui suffert tentatiónem: quóniam, cum probátus fúerit, accípiet corónam vitæ. Allelúja.
}\switchcolumn\portugues{
Aleluia, aleluia. ℣. \emph{Tg. 1, 12} Bem-aventurado o varão que sabe sofrer a tentação, porque, quando acabar a tentação, receberá a coroa da vida. Aleluia.
}\end{paracol}

\textit{Após a Septuagésima omite-se o Aleluia e o seguinte e diz-se:}

\paragraphinfo{Trato}{Sl. 111, 1-3}
\begin{paracol}{2}\latim{
\rlettrine{B}{eátus} vir, qui timet Dóminum: in mandátis ejus cupit nimis. ℣. Potens in terra erit semen ejus: generátio rectórum benedicétur. ℣. Glória et divitiæ in domo ejus: et justítia ejus manet in sǽculum sǽculi.
}\switchcolumn\portugues{
\rlettrine{B}{em-aventurado} o varão que teme o Senhor e que põe todo seu zelo em obedecer-Lhe. ℣. Sua descendência será poderosa na terra; pois a geração dos justos será abençoada. ℣. Na sua casa haverá glória e riqueza: e a justiça subsistirá em todos os séculos dos séculos.
}\end{paracol}

\textit{No T. Pascal o Gradual e o Trato e diz-se:}

\begin{paracol}{2}\latim{
Allelúja, allelúja. ℣. \emph{Jac. 1, 12} Beátus vir, qui suffert tentatiónem: quóniam, cum probátus fúerit, accípiet corónam vitæ. Allelúja. ℣. \emph{Eccli. 45, 9} Amávit eum Dóminus et ornávit eum: stolam glóriæ índuit eum. Allelúja.
}\switchcolumn\portugues{
Aleluia, aleluia. \emph{Tg. 1, 12} Bem-aventurado o varão que sabe sofrer a tentação, porque, quando acabar a tentação, receberá a coroa da vida. Aleluia. ℣. \emph{Ecl. 45, 9} O Senhor amou-o, ornou-o e revestiu-o com a túnica da glória. Aleluia.
}\end{paracol}

\paragraphinfo{Evangelho}{Lc. 12, 35-40}\label{evangelhoconfessoresnaopontifices1}
\begin{paracol}{2}\latim{
\cruz Sequéntia sancti Evangélii secúndum Lucam.
}\switchcolumn\portugues{
\cruz Continuação do santo Evangelho segundo S. Lucas.
}\switchcolumn*\latim{
\blettrine{I}{n} illo témpore: Dixit Jesus discípulis suis: Sint lumbi vestri præcíncti, et lucernæ ardéntes in mánibus vestris, et vos símiles homínibus exspectántibus dóminum suum, quando revertátur a núptiis: ut, cum vénerit et pulsáverit, conféstim apériant ei. Beáti servi illi, quos, cum vénerit dóminus, invénerit vigilántes: amen, dico vobis, quod præcínget se, et fáciet illos discúmbere, et tránsiens ministrábit illis. Et si vénerit in secúnda vigília, et si in tértia vigília vénerit, et ita invénerit, beáti sunt servi illi. Hoc autem scitóte, quóniam, si sciret paterfamílias, qua hora fur veníret, vigiláret útique, et non síneret pérfodi domum suam. Et vos estóte paráti, quia, qua hora non putátis, Fílius hóminis véniet.
}\switchcolumn\portugues{
\blettrine{N}{aquele} tempo, disse Jesus aos seus discípulos: «Estejam cingidos os vossos rins; tende nas vossas mãos lâmpadas acesas; sede semelhantes a homens que esperam o seu senhor quando volta das bodas, para que, quando chegar e bater à porta, logo lha abram. Bem-aventurados aqueles servos que seu senhor, quando chegar, os achar vigilantes. Em verdade vos digo que se cingirá, e, mandando-os sentar à mesa, passará por entre eles e os servirá. E, se chegar na segunda ou terceira vigília e assim os achar, bem-aventurados são esses servos. Ora, sabei que, se o pai de família conhecesse a hora em que viria o ladrão, vigiava, sem dúvida, para que sua casa não fosse assaltada. Vós, pois, estai preparados, porque o Filho do homem virá à hora em que menos cuidais».
}\end{paracol}

\paragraphinfo{Ofertório}{Sl. 88, 25.}
\begin{paracol}{2}\latim{
\rlettrine{V}{éritas} mea et misericórdia mea cum ipso: et in nómine meo exaltábitur cornu ejus. (T. P. Allelúja.)
}\switchcolumn\portugues{
\rlettrine{A}{} minha verdade e a minha misericórdia estarão com ele, e, por virtude do meu nome, será exaltado o seu poder. (T. P. Aleluia.)
}\end{paracol}

\paragraph{Secreta}
\begin{paracol}{2}\latim{
\rlettrine{L}{audis} tibi, Dómine, hóstias immolámus in tuórum commemoratióne Sanctórum: quibus nos et præséntibus éxui malis confídimus et futúris. Per Dóminum \emph{\&c.}
}\switchcolumn\portugues{
\rlettrine{S}{enhor,} Vos oferecemos este sacrifício de louvor em memória dos vossos Santos, para que por meio dele nos livremos dos males presentes e futuros. Por nosso Senhor \emph{\&c.}
}\end{paracol}

\paragraphinfo{Comúnio}{Mt. 24, 46-47}
\begin{paracol}{2}\latim{
\rlettrine{B}{eátus} servus, quem, cum vénerit dóminus, invénerit vigilántem: amen, dico vobis, super ómnia bona sua constítuet eum. (T. P. Allelúja.)
}\switchcolumn\portugues{
\rlettrine{B}{em-aventurado} o servo que o Senhor, quando vier, achar vigilante. Em verdade vos digo que lhe dará a administração de todos seus bens. (T. P. Aleluia).
}\end{paracol}

\paragraph{Postcomúnio}
\begin{paracol}{2}\latim{
\rlettrine{R}{efécti} cibo potúque cœlésti, Deus noster, te súpplices exorámus: ut, in cujus hæc commemoratióne percépimus, ejus muniámur et précibus. Per Dóminum \emph{\&c.}
}\switchcolumn\portugues{
\rlettrine{F}{ortalecidos} com o alimento e com a bebida celestiais, Vos suplicamos humildemente, ó nosso Deus, que nos protejam as preces daquele em cuja memória os recebemos. Por nosso Senhor \emph{\&c.}
}\end{paracol}

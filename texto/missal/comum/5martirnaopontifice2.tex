\subsectioninfo{Mártir não Pontífice}{Missa Lætábitur justus}\label{martirnaopontifice2}

\paragraphinfo{Intróito}{Sl. 63, 11}
\begin{paracol}{2}\latim{
\rlettrine{L}{ætábitur} justus in Dómino, et sperábit in eo: et laudabúntur omnes recti corde. \emph{Ps. ibid., 2} Exáudi, Deus, oratiónem meam, cum déprecor: a timóre inimíci éripe ánimam meam.
℣. Gloria Patri \emph{\&c.}
}\switchcolumn\portugues{
\rlettrine{O}{} justo alegrar-se-á no Senhor e porá n’Ele a sua esperança. Todos aqueles que possuem o coração recto serão glorificados. \emph{Ps. ibid., 2} Ouvi, Senhor, a oração com que Vos imploro: livrai a minha alma do temor do inimigo.
℣. Glória ao Pai \emph{\&c.}
}\end{paracol}

\paragraph{Oração}
\begin{paracol}{2}\latim{
\rlettrine{P}{ræsta,} quǽsumus, omnípotens Deus: ut, intercedénte beáto {\redx N.} Mártyre tuo, et a cunctis adversitátibus liberémur in córpore, et a pravis cogitatiónibus mundémur in mente. Per Dóminum nostrum \emph{\&c.}
}\switchcolumn\portugues{
\rlettrine{C}{oncedei-nos,} ó Deus omnipotente, Vos suplicamos, que, por intercessão do B. {\redx N.} vosso Mártir, o nosso corpo seja preservado de todas as adversidades e a nossa alma purificada dos maus pensamentos. Por nosso Senhor \emph{\&c.}
}\end{paracol}

\paragraphinfo{Epístola}{2. Tm. 2, 8-10; 3, 10-12}
\begin{paracol}{2}\latim{
Léctio Epístolæ beáti Pauli Apóstoli ad Timotheum.
}\switchcolumn\portugues{
Lição da Ep.ª do B. Ap.º Paulo a Timóteo.
}\switchcolumn*\latim{
\rlettrine{C}{aríssime:} Memor esto, Dóminum Jesum Christum resurrexísse a mórtuis ex sémine David, secúndum Evangélium meum, in quo labóro usque ad víncula, quasi male óperans: sed verbum Dei non est alligátum. Ideo ómnia sustíneo propter eléctos, ut et ipsi salútem consequántur, quæ est in Christo Jesu, cum glória cœlésti. Tu autem assecútus es meam doctrínam, institutiónem, propósitum, fidem, longanimitátem, dilectiónem, patiéntiam, persecutiónes, passiónes: quália mihi facta sunt Antiochíæ, Icónii et Lystris: quales perseditiónes sustínui, et ex ómnibus erípuit me Dóminus. Et omnes, qui pie volunt vívere in Christo Jesu, persecutiónem patiéntur.
}\switchcolumn\portugues{
\rlettrine{C}{aríssimo:} lembrai-vos de que N. S. Jesus Cristo, descendente da raça de David, ressuscitou dos mortos, segundo o Evangelho que prego, pelo qual tenho sofrido a provação até estar preso com cadeias, como se fora um malfeitor. Mas a palavra de Deus não pode estar presa. Por isso sofro tudo por amor dos escolhidos, a fim de que eles consigam também a salvação, que está na glória celestial com Jesus Cristo. Quanto a vós, lembrai-vos de que estais unidos e compreendestes a minha doutrina, modo de vida, regra de conduta, fé, longanimidade, caridade e paciência, e conheceis as grandes perseguições e vexações que me foram feitas e sofri em Antioquia, Icónia e Lístria, das quais o Senhor me livrou. Do mesmo modo, todos os que querem viver piamente em Jesus Cristo padecerão perseguição.
}\end{paracol}

\paragraphinfo{Gradual}{Sl. 36, 24}
\begin{paracol}{2}\latim{
\qlettrine{J}{ustus} cum cecíderit, non collidétur: quia Dóminus suppónit manum suam. ℣. \emph{ibid., 26} Tota die miserétur, et cómmodat: et semen ejus in benedictióne erit.
}\switchcolumn\portugues{
\qlettrine{Q}{uando} o justo cair, não se magoará, porque o Senhor o amparará com sua mão. ℣. \emph{ibid., 26} Em cada dia ele se emprega em obras de misericórdia, e empresta: e a sua geração será abençoada.
}\switchcolumn*\latim{
Allelúja, allelúja. ℣. \emph{Joann. 8, 12} Qui séquitur me, non ámbulat in ténebris: sed habébit lumen vitæ ætérnæ. Allelúja.
}\switchcolumn\portugues{
Aleluia, aleluia. ℣. \emph{Jo. 8, 12} Aquele que me segue não caminha nas trevas, mas terá a luz da vida eterna. Aleluia.
}\end{paracol}

\textit{Após a Septuagésima omite-se o Aleluia e o seguinte e diz-se:}

\paragraphinfo{Trato}{Sl. 111, 1-3}
\begin{paracol}{2}\latim{
\rlettrine{B}{eátus} vir, qui timet Dóminum: in mandátis ejus cupit nimis. ℣. Potens in terra erit semen ejus: generátio rectórum benedicétur. ℣. Glória et divítiæ in domo ejus: et justítia ejus manet in sǽculum sǽculi.
}\switchcolumn\portugues{
\rlettrine{B}{em-aventurado} o varão que teme o Senhor e que põe todo seu zelo em obedecer-lhe. ℣. Sua descendência será poderosa na terra, pois a geração dos justos será abençoada. ℣. Haverá glória e riqueza em sua casa e a sua justiça subsistirá para sempre.
}\end{paracol}

\paragraphinfo{Evangelho}{Mt. 10, 26-32}
\begin{paracol}{2}\latim{
\cruz Sequéntia sancti Evangélii secúndum Matthǽum.
}\switchcolumn\portugues{
\cruz Continuação do santo Evangelho segundo S. Mateus.
}\switchcolumn*\latim{
\blettrine{I}{n} illo témpore: Dixit Jesus discípulis suis: Nihil est opértum, quod non revelábitur; et occúltum, quod non sciétur. Quod dico vobis in ténebris, dícite in lúmine: et quod in aure audítis, prædicáte super tecta. Et nolíte timére eos, qui occídunt corpus, ánimam autem non possunt occídere; sed pótius timéte eum, qui potest et ánimam et corpus pérdere in gehénnam. Nonne duo pásseres asse véneunt: et unus ex illis non cadet super terram sine Patre vestro? Vestri autem capílli cápitis omnes numerári sunt. Nolíte ergo timére: multis passéribus melióres estis vos. Omnis ergo, qui confitébitur me coram homínibus, confitébor et ego eum coram Patre meo, qui in cœlis est.
}\switchcolumn\portugues{
\blettrine{N}{aquele} tempo, disse Jesus aos discípulos: «Nada há oculto que não haja e ser descoberto, nem segredo que não venha a ser revelado. O que dizeis nas trevas dizei-o às claras; o que dizeis ao ouvido publicai-o em cima dos telhados. Não tenhais medo daqueles que matam o corpo e não podem matar a alma; temei antes aquele que pode condenar a alma e o corpo ao inferno. Porventura se não vendem dous pássaros por um ceitil? E nenhum deles, contudo, cairá no chão sem o consentimento do vosso Pai. Até os cabelos da vossa cabeça estão contados. Nada receeis, pois valeis mais do que muitos pássaros. Portanto, todo aquele que me confessar perante os homens também o confessarei na presença de meu Pai, que está nos céus».
}\end{paracol}

\paragraphinfo{Ofertório}{Sl. 20,4-5}
\begin{paracol}{2}\latim{
\rlettrine{P}{osuísti,} Dómine, in cápite ejus corónam de lápide pretióso: vitam pétiit a te, et tribuísti ei, allelúja.
}\switchcolumn\portugues{
\rlettrine{I}{mpusestes} na sua cabeça, Senhor, uma coroa de pedras preciosas; pediu-Vos a vida e concedestes-lha. Aleluia.
}\end{paracol}

\paragraph{Secreta}
\begin{paracol}{2}\latim{
\rlettrine{A}{ccépta} sit in conspéctu tuo, Dómine, nostra devótio: et ejus nobis fiat supplicatióne salutáris, pro cujus sollemnitáte defértur. Per Dóminum nostrum \emph{\&c.}
}\switchcolumn\portugues{
\rlettrine{R}{ecebei} benigno, Senhor, esta oferta da nossa piedade, e que ela nos alcance a salvação, por intercessão das preces daquele em cuja festa nós Vo-la apresentamos. Por nosso Senhor \emph{\&c.}
}\end{paracol}

\paragraphinfo{Comúnio}{Jo. 12, 26}
\begin{paracol}{2}\latim{
\qlettrine{Q}{ui} mihi mínistrat, me sequátur: et ubi sum ego, illic et miníster meus erit.
}\switchcolumn\portugues{
\rlettrine{S}{e} alguém me serve, siga-me; e onde eu estiver lá estará também o meu servo.
}\end{paracol}

\paragraph{Postcomúnio}
\begin{paracol}{2}\latim{
\rlettrine{R}{efécti} participatióne múneris sacri, quǽsumus, Dómine, Deus noster: ut, cujus exséquimur cultum, intercedénte beáto {\redx N.} Mártyre tuo, sentiámus efféctum. Per Dóminum \emph{\&c.}
}\switchcolumn\portugues{
\rlettrine{F}{ortalecidos} com a participação deste dom sacratíssimo, Vos suplicamos, Senhor, nosso Deus, que, por intercessão do B. {\redx N.}, vosso Mártir, sintamos o efeito do mystério que celebrámos. Por nosso Senhor \emph{\&c.}
}\end{paracol}

Outra Epístola (para certos dias):

\paragraphinfo{Epístola}{Tg. 1, 2-12}
\begin{paracol}{2}\latim{
Léctio Epístolæ beáti Jacóbi Apóstoli.
}\switchcolumn\portugues{
Lição da Ep.ª do B. Ap.º Tiago.
}\switchcolumn*\latim{
\rlettrine{C}{aríssime:} Omne gáudium existimáte, cum in tentatiónes várias incidéritis: sciéntes, quod probátio fídei vestræ patiéntiam operátur. Patiéntia autem opus perféctum habet: ut sitis perfécti et íntegri, in nullo deficiéntes. Si quis autem vestrum índiget sapiéntia, póstulet a Deo, qui dat ómnibus affluénter, et non impróperat: et dábitur ei. Póstulet autem in fide nihil hǽsitans: qui enim hǽsitat, símilis est flúctui maris, qui a vento movétur et circumfértur. Non ergo ǽstimet homo ille, quod accípiat áliquid a Dómino. Vir duplex ánimo incónstans est in ómnibus viis suis. Gloriétur autem frater húmilis in exaltatióne sua: dives autem in humilitáte sua, quóniam sicut flos fæni transíbit: exórtus est enim sol cum ardóre, et arefécit fænum, et flos ejus décidit et decor vultus ejus depériit: ita et dives in itinéribus suis marcéscet. Beátus vir, qui suffert tentatiónem: quóniam, cum probátus fúerit, accípiet corónam vitæ, quam repromísit Deus diligéntibus se.
}\switchcolumn\portugues{
\rlettrine{C}{aríssimo:} considerai como motivo de muita alegria as diversas aflições que vos acometerem; pois deveis saber que a provação da vossa fé produz a paciência. E que a paciência seja acompanhada de obras perfeitas, de maneira que sejais perfeitos e íntegros, nada deixando a desejar. Se algum de vós necessita de sabedoria, peça-a a Deus, que a dá a todos liberalmente, sem a ninguém repreender, e ela lhe será concedida. Peça-a, porém, com fé, sem desconfiança, pois aquele que duvida é semelhante a uma onda do mar, agitada pelo vento e movida de um lado para o outro. Com tais disposições nenhum homem imagine que há-de receber alguma cousa do Senhor. O homem que possui um espírito duplo é inconstante em suas acções. Aquele nosso irmão que é pobre glorie-se com a esperança da sua exaltação; e, ao contrário, aquele que é rico espere a sua humilhação, pois ele passará, como a flor da erva: o sol ergue-se ardente, secando a erva; então a flor da erva emurchece e toda sua beleza desaparece. Assim, o rico secará e murchará em suas empresas. Bem-aventurado o varão que suporta a tentação, porque, quando tiver acabado a provação, receberá a coroa da vida, que Deus prometeu aos que O amam.
}\end{paracol}

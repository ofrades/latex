\subsectioninfo{Muitos Mártires}{Missa Sapiéntiam sanctórum}\label{muitosmartires2}

\paragraphinfo{Intróito}{Ecl. 44,15 \& 14}
\begin{paracol}{2}\latim{
\rlettrine{S}{apiéntiam} Sanctórum narrent pópuli, et laudes eórum núntiet ecclésia: nomina autem eórum vivent in sǽculum sǽculi. \emph{Ps. 32, 1} Exsultáte, justi, in Dómino: rectos decet collaudátio.
℣. Gloria Patri \emph{\&c.}
}\switchcolumn\portugues{
\qlettrine{Q}{ue} os povos publiquem a sabedoria dos santos e que a Igreja celebre os seus louvores: o seu nome subsistirá em todos os séculos! \emph{Sl. 32, 1} Ó justos, rejubilai no Senhor: é àqueles que possuem o coração recto que pertence louvar o Senhor.
℣. Glória ao Pai \emph{\&c.}
}\end{paracol}

\paragraph{Oração}
\begin{paracol}{2}\latim{
\rlettrine{D}{eus,} qui nos concédis sanctórum Mártyrum tuórum {\redx N.} et {\redx N.} natalítia cólere: da nobis in ætérna beatitúdine de eórum societéte gaudére. Per Dóminum \emph{\&c.}
}\switchcolumn\portugues{
\slettrine{Ó}{} Deus, que nos permitistes a graça de celebrarmos o nascimento no céu dos vossos Santos Mártires {\redx N.} e {\redx N.}, concedei-nos ainda a graça de gozarmos na sua companhia a bem-aventurança eterna. Por nosso Senhor \emph{\&c.}
}\end{paracol}

\textit{Se forem Pontífices, não se diz esta Oração mas a da Missa precedente, página \pageref{muitosmartires2}.}

\paragraphinfo{Epístola}{Sb. 5, 16-20}
\begin{paracol}{2}\latim{
Léctio libri Sapiéntiæ.
}\switchcolumn\portugues{
Lição do Livro da Sabedoria.
}\switchcolumn*\latim{
\qlettrine{J}{usti} autem in perpétuum vivent, et apud Dóminum est merces eórum, et cogitátio illórum apud Altíssimum. Ideo accípient regnum decóris, et diadéma speciéi de manu Dómini: quóniam déxtera sua teget eos, et bráchio sancto suo deféndet illos. Accípiet armatúram zelus illíus, et armábit creatúram ad ultiónem inimicórum. Induet pro thoráce justítiam, et accípiet pro gálea judícium certum. Sumet scutum inexpugnábile æquitátem.
}\switchcolumn\portugues{
\rlettrine{O}{s} justos viverão eternamente e alcançarão recompensa junto do Senhor, pois o Altíssimo cuidará deles. Eis porque receberão das mãos do Senhor um reino de glória e um diadema brilhante! O Senhor protegê-los-á com sua dextra, cobrindo-os com seu divino braço, que será como um escudo. Seu zelo o levará a tomar a armadura e a armar as criaturas para se vingar dos seus inimigos. Envergará a justiça como couraça e a integridade do juízo como capacete; e revestir-se-á com a equidade como escudo inexpugnável.
}\end{paracol}

\paragraphinfo{Gradual}{Sl. 123,7-8}
\begin{paracol}{2}\latim{
\rlettrine{A}{nima} nostra, sicut passer, erépta est de láqueo venántium. ℣. Láqueus contrítus est, et nos liberáti sumus: adjutórium nostrum in nómine Dómini, qui fecit cœlum et terram.
}\switchcolumn\portugues{
\rlettrine{A}{} nossa alma livrou-se, como um pássaro do laço dos caçadores! ℣. O laço quebrou-se, e ficámos livres. O nosso auxílio está no nome do Senhor, que criou o céu e a terra.
}\switchcolumn*\latim{
Allelúja, alielúja. ℣. \emph{Ps.67, 4} Justi epuléntur, et exsúltent in conspéctu Dei: et delecténtur in lætítia. Allelúja.
}\switchcolumn\portugues{
Aleluia, aleluia. ℣. \emph{Sl. 67, 4} Que os justos se regozijem e exultem de alegria na presença de Deus, como em um banquete. Que eles se deliciem em transportes de alegria. Aleluia.
}\end{paracol}

\textit{Após a Septuagésima omite-se o Aleluia e o seguinte e diz-se:}

\paragraphinfo{Trato}{Sl. 125, 5-6}
\begin{paracol}{2}\latim{
\qlettrine{Q}{ui} séminant in lácrimis, in gáudio metent. ℣. Eúntes ibant et fiébant, mitténtes sémina sua. ℣. Veniéntes autem vénient cum exsultatióne, portántes manípulos suos
}\switchcolumn\portugues{
\rlettrine{A}{queles} que semeiam com lágrimas ceifarão com júbilo. ℣. Iam, caminhavam e lançavam a semente à terra, chorando. ℣. Porém, quando voltavam, exultavam de alegria, trazendo os seus molhos de trigo.
}\end{paracol}

\paragraphinfo{Evangelho}{Lc. 6, 17-23}
\begin{paracol}{2}\latim{
\cruz Sequéntia sancti Evangélii secúndum Lucam.
}\switchcolumn\portugues{
\cruz Continuação do santo Evangelho segundo S. Lucas.
}\switchcolumn*\latim{
\blettrine{I}{n} illo témpore: Descéndens Jesus de monte, stetit in loco campéstri, et turba discipulórum ejus, et multitúdo copiósa plebis ab omni Judǽa, et Jerúsalem, et marítima, et Tyri, et Sidónis, qui vénerant, ut audírení eum et sanaréntur a languóribus suis. Et, qui vexabántur a spirítibus immúndis, curabántur. Et omnis turba quærébat eum tangere: quia virtus de illo exíbat, et sanábat omnes. Et ipse, elevátis óculis in discípulos suos, dicebat: Beáti, páuperes: quia vestrum est regnum Dei. Beáti, qui nunc esurítis: quia saturabímini. Beáti, qui nunc fletis: quia ridébitis. Beáti eritis, cum vos óderint hómines, et cum separáverint vos et exprobráveriní, et ejécerint nomen vestrum tamquam malum, propter Fílium hóminis. Gaudéte in illa die et exsultáte: ecce enim, merces vestra multa est in cœlo.
}\switchcolumn\portugues{
\blettrine{N}{aquele} tempo, descendo Jesus da montanha, parou em uma planície com a turba dos seus discípulos e grande multidão de povo de toda a Judeia, de Jerusalém, das margens do mar, de Tiro e de Sidónia, que tinha vindo para ouvi-l’O e ser curado de suas enfermidades. E os que estavam possessos de espíritos imundos ficavam sãos. Toda a multidão procurava tocá-l’O, porque saía d’Ele uma virtude que curava a todos. Levantando, então, Jesus os olhos para os seus discípulos, disse: «Bem-aventurados vós, pobres, porque o reino dos céus é vosso; bem-aventurados vós, famintos, porque sereis fartos; bem-aventurados vós, que agora chorais, porque depois rireis; bem-aventurados vós, quando sois odiados e injuriados e quando os homens se aborrecem e rejeitam o vosso nome, como se fora mau, por causa do Filho do homem. Alegrai-vos e rejubilai, pois uma grande recompensa vos está reservada no céu».
}\end{paracol}

\paragraphinfo{Ofertório}{Sl. 149, 5-6}
\begin{paracol}{2}\latim{
\rlettrine{E}{xsultábunt} Sancti in glória, lætabúntur in cubílibus suis: exaltatiónes Dei in fáucibus eórum, allelúja.
}\switchcolumn\portugues{
\rlettrine{O}{s} santos exultarão de alegria na sua glória e deliciar-se-ão de alegria no lugar do seu repouso. Ressoarão em seus lábios louvores a Deus.
}\end{paracol}

\paragraph{Secreta}
\begin{paracol}{2}\latim{
\rlettrine{M}{únera} tibi, Dómine, nostræ devotiónis offérimus: quæ et pro tuórum tibi grata sint honóre Justórum, et nobis salutária, te miseránte, reddántur. Per Dóminum \emph{\&c.}
}\switchcolumn\portugues{
\rlettrine{V}{os} oferecemos, Senhor, estes dons da nossa devoção; e, em atenção aos merecimentos dos vossos justos, dignai-Vos aceitá-los, e pela vossa misericórdia fazei que nos sejam salutares. Por nosso Senhor \emph{\&c.}
}\end{paracol}

\paragraphinfo{Comúnio}{Lc. 12, 4}
\begin{paracol}{2}\latim{
\rlettrine{D}{ico} autem vobis amícis meis: Ne terreámini ab his, qui vos persequúntur.
}\switchcolumn\portugues{
\rlettrine{D}{igo-vos,} pois, a vós, que sois meus amigos: não tenhais medo daqueles que vos perseguem.
}\end{paracol}

\paragraph{Postcomúnio}
\begin{paracol}{2}\latim{
\rlettrine{P}{ræsta} nobis, quǽsumus, Dómine: intercedéntibus sanctis Martýribus tuis {\redx N.} et {\redx N.}; ut, quod ore contíngimus, pura mente capiámus. Per Dóminum \emph{\&c.}
}\switchcolumn\portugues{
\rlettrine{S}{enhor,} por intercessão dos vossos santos Mártires {\redx N.} e {\redx N.}, dignai-Vos conceder-nos a graça de guardarmos com o coração sempre puro o que a nossa boca agora recebeu. Por nosso Senhor \emph{\&c.}
}\end{paracol}

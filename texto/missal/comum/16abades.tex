\subsectioninfo{Abades}{Missa Os justi}\label{abades}

\paragraphinfo{Intróito}{Sl. 36, 30-31}
\begin{paracol}{2}\latim{
\rlettrine{O}{s} justi meditábitur sapiéntiam, et lingua ejus loquétur judícium: lex Dei ejus in corde ipsíus. (T. P. Allelúja allelúja.) \emph{Ps. ibid., 1} Noli æmulári in malignántibus: neque zeláveris faciéntes iniquitátem.
℣. Gloria Patri \emph{\&c.}
}\switchcolumn\portugues{
\rlettrine{A}{} boca do justo fala com sabedoria: e a sua língua proclama a justiça. A lei do seu Deus estará no seu coração. (T. P. Aleluia, aleluia.) \emph{Sl. ibid., 1} Não vos irriteis contra os maus, nem tenhais inveja daqueles que cometem iniquidades.
℣. Glória ao Pai \emph{\&c.}
}\end{paracol}

\paragraph{Oração}
\begin{paracol}{2}\latim{
\rlettrine{I}{ntercéssio} nos, quǽsumus, Dómine, beáti {\redx N.} Abbátis comméndet: ut, quod nostris méritis non valémus, ejus patrocínio assequámur. Per Dóminum nostrum \emph{\&c.}
}\switchcolumn\portugues{
\qlettrine{Q}{ue} a intercessão do B. Abade {\redx N.} nos favoreça junto de Vós, Senhor, Vos suplicamos, a fim de que aquilo que não podemos conseguir com os nossos méritos o alcancemos com seu patrocínio. Por nosso Senhor \emph{\&c.}
}\end{paracol}

\paragraphinfo{Epístola}{Ecl. 45, 1-6}\label{epistolaabades}
\begin{paracol}{2}\latim{
Léctio libri Sapientiæ.
}\switchcolumn\portugues{
Lição do Livro da Sabedoria.
}\switchcolumn*\latim{
\rlettrine{D}{iléctus} Deo et homínibus, cujus memória in benedictióne est. Símilem illum fecit in glória sanctórum, et magnificávit eum in timóre inimicórum, et in verbis suis monstra placávit. Gloríficávit illum in conspéctu regum, et jussit illi coram pópulo suo, et osténdit illi glóriam suam. In fide et lenitáte ipsíus sanctum fecit illum, et elégit eum; ex omni carne. Audívit enim eum et vocem ipsíus, et indúxit illum in nubem. Et dedit illi coram præcépta, et legem vitæ et disciplínæ.
}\switchcolumn\portugues{
\rlettrine{F}{oi} amado de Deus e dos homens; a sua memória é uma bênção. O Senhor deu-lhe uma glória, semelhante à dos Santos; tornou-o temeroso e invencível perante seus inimigos; e com suas palavras aplacou os monstros. O Senhor honrou-o diante dos reis; deu-lhe as suas ordens diante do seu povo; e mostrou-lhe a sua glória. O Senhor santificou-o pela sua fé e mansidão; e escolheu-o entre todos os homens. Deus escutou-o; ouviu a sua voz; e fê-lo entrar na nuvem. Então, deu-lhe face a face os seus preceitos e a lei da vida e da doutrina.
}\end{paracol}

\paragraphinfo{Gradual}{Sl. 20, 4-5}
\begin{paracol}{2}\latim{
\rlettrine{D}{ómine,} prævenísti eum in benedictiónibus dulcédinis: posuísti in cápite ejus corónam de lápide pretióso. ℣. Vitam pétiit a te, et tribuísti ei longitúdinem diérum in sǽculum sǽculi.
}\switchcolumn\portugues{
\rlettrine{S}{enhor,} concedestes-lhe bênçãos escolhidas as mais suaves; e impusestes na sua cabeça uma coroa de pedras preciosas. Concedestes-lhe a vida, que Vos suplicara, e prolongastes-lhe a duração dos seus dias pelos séculos dos séculos.
}\switchcolumn*\latim{
Allelúja, allelúja. ℣. \emph{Ps. 91, 13} Justus ut palma florébit: sicut cedrus Líbani multiplicábitur. Allelúja.
}\switchcolumn\portugues{
Aleluia, aleluia. ℣. \emph{Sl. 91, 13} O justo florescerá, como a palmeira, e crescerá, como o cedro do Líbano. Aleluia.
}\end{paracol}

\textit{Após a Septuagésima omite-se o Aleluia e o seguinte e diz-se:}

\paragraphinfo{Trato}{Sl. 111, 1-3}
\begin{paracol}{2}\latim{
\rlettrine{B}{eátus} vir, qui timet Dóminum: in mandátis ejus cupit nimis. ℣. Potens in terra erit semen ejus: generátio rectórum benedicétur. ℣. Glória et divítiæ in domo ejus: et justítia ejus manet in sǽculum sǽculi.
}\switchcolumn\portugues{
\rlettrine{B}{em-aventurado} o varão que teme o Senhor e que emprega todo o zelo em obedecer-Lhe. ℣. Sua descendência será poderosa na terra; pois a geração dos justos será abençoada. ℣. Na sua casa haverá glória e riqueza, e a sua justiça subsistirá em todos os séculos dos séculos.
}\end{paracol}

\textit{No T. Pascal omite-se o Gradual e o Trato e diz-se:}

\begin{paracol}{2}\latim{
Allelúja, allelúja. ℣. \emph{Ps. 91, 13} Justus ut palma florébit: sicut cedrus Líbani multiplicábitur. Allelúja. ℣. \emph{Osee 14, 6} Justus germinábit sicut lílium: et florébit in ætérnum ante Dóminum. Allelúja.
}\switchcolumn\portugues{
Aleluia, aleluia. ℣. \emph{Sl. 91, 13} O justo florescerá, como a palmeira, e multiplicar-se-á, como o cedro do Líbano. Aleluia. ℣. \emph{Os. 14, 6} O justo germinará, como o lírio, e florescerá eternamente na presença do Senhor. Aleluia.
}\end{paracol}

\paragraphinfo{Evangelho}{Mt. 19, 27-29}\label{evangelhoabades}
\begin{paracol}{2}\latim{
\cruz Sequéntia sancti Evangélii secúndum Matthǽum.
}\switchcolumn\portugues{
\cruz Continuação do santo Evangelho segundo S. Mateus.
}\switchcolumn*\latim{
\blettrine{I}{n} illo témpore: Dixit Petrus ad Jesum: Ecce, nos relíquimus ómnia, et secúti sumus te: quid ergo erit nobis? Jesus autem dixit illis: Amen, dico vobis, quod vos, qui secuti estis me, in regeneratióne, cum séderit Fílius hóminis in sede majestátis suæ, sedébitis et vos super sedes duódecim, judicántes duódecim tribus Israël. Et omnis, qui relíquerit domum, vel fratres, aut soróres, aut patrem, aut matrem, aut uxórem, aut fílios, aut agros, propter nomen meum, céntuplum accípiet, et vitam ætérnam possidébit.
}\switchcolumn\portugues{
\blettrine{N}{aquele} tempo, disse Pedro a Jesus: «Eis que deixámos tudo e Vos seguimos. Que recompensa teremos por isso?». Jesus disse-lhes: «Em verdade vos digo: vós, que me seguistes, quando, no tempo da regeneração, o Filho do homem se sentar no trono da sua glória, também vos sentareis sobre doze tronos para julgar as doze tribos de Israel. Todo aquele que deixar a sua casa, ou os seus irmãos, ou os seus campos, ou o seu pai, ou a sua mãe, ou a sua mulher por causa do meu nome receberá o cêntuplo e possuirá a vida eterna».
}\end{paracol}

\paragraphinfo{Ofertório}{Sl. 20, 3 \& 4}
\begin{paracol}{2}\latim{
\rlettrine{D}{esidérium} ánimæ ejus tribuísti ei, Dómine, et voluntáte labiórum ejus non fraudásti eum: posuísti in cápite ejus corónam de lápide pretióso. (T. P. Allelúja.)
}\switchcolumn\portugues{
\rlettrine{C}{oncedestes-lhe} o desejo da sua alma e não desprezastes a prece dos seus lábios. Impusestes na sua cabeça uma coroa de pedras preciosas. (T. P. Aleluia).
}\end{paracol}

\paragraph{Secreta}
\begin{paracol}{2}\latim{
\rlettrine{S}{acris} altáribus, Dómine, hóstias superpósitas sanctus {\redx N.} Abbas, quǽsumus, in salútem nobis proveníre depóscat. Per Dóminum \emph{\&c.}
}\switchcolumn\portugues{
\rlettrine{V}{os} imploramos, Senhor, que o vosso santo Abade {\redx N.} nos alcance que a hóstia, oferecida no vosso altar, nos proporcione a salvação. Por nosso Senhor \emph{\&c.}
}\end{paracol}

\paragraphinfo{Comúnio}{Lc. 12, 42}
\begin{paracol}{2}\latim{
\rlettrine{F}{idélis} servus et prudens, quem constítuit dóminus super famíliam suam: ut det illis in témpore trítici mensúram. (T. P. Allelúja.)
}\switchcolumn\portugues{
\rlettrine{O}{} servo fiel e prudente é destinado pelo Senhor para distribuir, oportunamente, na sua família a sua medida de trigo. (T. P. Aleluia.)
}\end{paracol}

\paragraph{Postcomúnio}
\begin{paracol}{2}\latim{
\rlettrine{P}{rótegat} nos, Dómine, cum tui perceptióne sacraménti beátus {\redx N.} Abbas, pro nobis intercedéndo: ut et conversatiónis ejus experiámur insígnia, et intercessiónis percipiámus suffrágia. Per Dóminum nostrum \emph{\&c.}
}\switchcolumn\portugues{
\qlettrine{Q}{ue} a recepção do vosso sacramento, Senhor, unida às preces do B. {\redx N.} Abade, nos sirva de protecção, a fim de que, imitando os insignes exemplos da sua vida, sintamos os efeitos da sua intercessão. Por nosso Senhor \emph{\&c.}
}\end{paracol}

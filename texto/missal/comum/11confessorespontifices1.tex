\subsectioninfo{Confessores Pontífices}{Missa Státuit ei Dóminus}\label{confessorespontifices1}

\paragraphinfo{Intróito}{Ecl. 45, 30}
\begin{paracol}{2}\latim{
\rlettrine{S}{tátuit} ei Dóminus testaméntum pacis, et príncipem fecit eum: ut sit illi sacerdótii dígnitas in ætérnum. (T. P. Allelúja, allelúja.) \emph{Ps. 131, 1} Meménto, Dómine, David: et omnis mansuetúdinis ejus.
℣. Gloria Patri \emph{\&c.}
}\switchcolumn\portugues{
\rlettrine{D}{eus} estabeleceu com ele aliança de paz e tornou-o príncipe, para que possuísse eternamente a dignidade sacerdotal. (T. P. Aleluia, aleluia.) \emph{Sl. 131, 1} Lembrai-vos de David, ó Senhor, e da sua grande mansidão.
℣. Glória ao Pai \emph{\&c.}
}\end{paracol}

\paragraph{Oração}
\begin{paracol}{2}\latim{
\rlettrine{D}{a,} quǽsumus, omnípotens Deus: ut beáti {\redx N.} Confessóris tui atque Pontíficis veneránda sollémnitas, et devotiónem nobis áugeat et salútem. Per Dóminum \emph{\&c.}
}\switchcolumn\portugues{
\rlettrine{D}{ignai-Vos} permitir, ó Deus omnipotente, que a veneranda solenidade do vosso Confessor e Pontífice {\redx N.} aumente a nossa piedade e nos assegure a salvação. Por nosso Senhor \emph{\&c.}
}\end{paracol}

\paragraphinfo{Epístola}{Ecl. 44, 16-27; 45, 3-20}
\begin{paracol}{2}\latim{
Léctio libri Sapiéntiæ.
}\switchcolumn\portugues{
Lição do Livro da Sabedoria.
}\switchcolumn*\latim{
\rlettrine{E}{cce} sacérdos magnus, qui in diébus suis plácuit Deo, et invéntus est justus: et in témpore iracúndiæ factus est reconciliátio. Non est invéntus símilis illi, qui conservávit legem Excélsi. Ideo jurejurándo fecit illum Dóminus créscere in plebem suam. Benedictiónem ómnium géntium dedit illi, et testaméntum suum confirmávit super caput ejus. Agnóvit eum in benedictiónibus suis: conservávit illi misericórdiam suam: et invenit grátiam coram óculis Dómini. Magnificávit eum in conspéctu regum: et dedit illi corónam glóriæ. Státuit illi testaméntum ætérnum, et dedit illi sacerdótium magnum: et beatificávit illum in glória. Fungi sacerdótio, et habére laudem in nómine ipsíus, et offérre illi incénsum dignum in odórem suavitátis.
}\switchcolumn\portugues{
\rlettrine{E}{is} o grande sacerdote, que nos dias da sua vida agradou a Deus e foi julgado justo; e no tempo da ira se tornou a reconciliação dos homens. Ninguém o igualou na observância das leis do Altíssimo. Eis porque o Senhor jurou que o tornaria grande no meio do seu povo. O Senhor abençoou nele todos os povos; e com ele ratificou a sua aliança. O Senhor deu-lhe as suas bênçãos e continuou a dispensar-lhe a sua misericórdia, vendo-se bem que este homem achou graça aos olhos do Senhor, que, por isso mesmo, o engrandeceu diante dos reis e lhe deu uma coroa de glória. Estabeleceu com ele uma aliança eterna, elevou-o ao sumo sacerdócio, e tornou-o feliz na glória para exercer o sacerdócio, louvar o seu nome e oferecer-lhe dignamente incenso de odor agradável.
}\end{paracol}

\paragraphinfo{Gradual}{Ecl. 44, 16}
\begin{paracol}{2}\latim{
\rlettrine{E}{cce} sacérdos magnus, qui in diébus suis plácuit Deo. ℣. \emph{ibid., 20} Non st invéntus símilis illi, qui conserváret legem Excélsi.
}\switchcolumn\portugues{
\rlettrine{E}{is} o grande sacerdote que nos dias da sua vida agradou a Deus. ℣. \emph{ibid., 20} Não foi encontrado outrem semelhante a ele na observância das leis do Altíssimo.
}\switchcolumn*\latim{
Allelúja, allelúja. ℣. \emph{Ps. 109, 4} Tu es sacérdos in ætérnum, secúndum órdinem Melchísedech. Allelúja.
}\switchcolumn\portugues{
Aleluia, aleluia. ℣. \emph{Sl. 109, 4} Tu és sacerdote para sempre, segundo a ordem de Melquisedeque. Aleluia.
}\end{paracol}

\textit{Após a Septuagésima omite-se o Aleluia e o seguinte e diz-se:}

\paragraphinfo{Trato}{Sl. 111, 1-3}
\begin{paracol}{2}\latim{
\rlettrine{B}{eátus} vir, qui timet Dóminum: in mandátis ejus cupit nimis. ℣. Potens in terra erit semen ejus: generátio rectórum benedicétur. ℣. Glória et divítiæ in domo ejus: et justítia ejus manet in sǽculum sǽculi.
}\switchcolumn\portugues{
\rlettrine{B}{em-aventurado} o varão que teme o Senhor e cuja vontade é ardente no cumprimento dos seus mandamentos. Sua descendência será poderosa na terra, pois a posteridade dos justos será abençoada. Na sua casa haverá abundância e riqueza, e a sua justiça subsistirá em todos os séculos dos séculos.
}\end{paracol}

\textit{No T. Pascal omite-se Gradual e o Trato e diz-se:}

\begin{paracol}{2}\latim{
Allelúja, allelúja. ℣. \emph{Ps. 109, 4} Tu es sacérdos in ætérnum, secúndum órdinem Melchísedech. Allelúja. ℣. Hic est sacérdos, quem coronávit Dóminus. Allelúja.
}\switchcolumn\portugues{
Aleluia, aleluia. ℣. \emph{Sl. 109, 4} Tu és sacerdote para sempre segundo a ordem de Melquisedeque. Aleluia. ℣. Este é o sacerdote que o Senhor coroou. Aleluia.
}\end{paracol}

\paragraphinfo{Evangelho}{Mt. 25, 14-23}
\begin{paracol}{2}\latim{
\cruz Sequéntia sancti Evangélii secúndum Matthǽum.
}\switchcolumn\portugues{
\cruz Continuação do santo Evangelho segundo S. Mateus.
}\switchcolumn*\latim{
\blettrine{I}{n} illo témpore: Dixit Jesus discípulis suis parábolam hanc: Homo péregre proficíscens vocávit servos suos, et trádidit illis bona sua. Et uni dedit quinque talénta, álii a tem duo, álii vero unum, unicuíque secúndum própriam virtútem, et proféctus est statim. Abiit autem, qui quinque talénta accéperat, et operátus est in eis, et lucrátus est ália quinque. Simíliter et, qui duo accéperat, lucrátus est ália duo. Qui autem unum accéperat, ábiens fodit in terram, et abscóndit pecúniam dómini sui. Post multum vero témporis venit dóminus servórum illórum, et pósuit ratiónem cum eis. Et accédens qui quinque talénta accéperat, óbtulit ália quinque talénta, dicens: Dómine, quinque talénta tradidísti mihi, ecce, ália quinque superlucrátus sum. Ait illi dóminus ejus: Euge, serve bone et fidélis, quia super pauca fuísti fidélis, super multa te constítuam: intra in gáudium dómini tui. Accéssit autem et qui duo talénta accéperat, et ait: Dómine, duo talénta tradidísti mihi, ecce, ália duo lucrátus sum. Ait illi dóminus ejus: Euge, serve bone et fidélis, quia super pauca fuísti fidélis, super multa te constítuam: intra in gáudium dómini tui.
}\switchcolumn\portugues{
\blettrine{N}{aquele} tempo, disse Jesus a seus discípulos esta parábola: Indo um homem viajar para longe, chamou os seus servos e entregou-lhes os bens. A um deu cinco talentos, a outro dous e ao terceiro um. Deu a cada um segundo a sua capacidade; partindo imediatamente. Aquele que havia recebido cinco talentos partiu, e, negociando com este dinheiro, ganhou outros cinco talentos. Semelhantemente, o que recebera dous lucrou outros dous. Mas aquele que havia recebido só um talento foi, cavou a terra e aí ocultou o dinheiro do senhor. Passado muito tempo, veio o senhor daqueles servos e fez contas com eles. Aproximando-se, então, o que recebera cinco talentos, apresentou outros cinco, dizendo: «Senhor, entregastes-me cinco talentos, eis outros cinco, que lucrei». Disse-lhe o seu senhor: «Muito bem, servo bom e fiel: visto que foste fiel em poucas cousas, estabelecer-te-ei acima de muitas cousas: entra no gozo do teu senhor». Aproximando-se também o que recebera dous talentos, disse: «Senhor, entregastes-me dous talentos, eis outros dous que lucrei». E o senhor lhe disse: «Muito bem, servo bom e fiel: visto que foste fiel em poucas cousas, eu te estabelecerei acima de muitas cousas: entra no gozo do teu senhor».
}\end{paracol}

\paragraphinfo{Ofertório}{Sl. 88, 21-22}
\begin{paracol}{2}\latim{
\rlettrine{I}{nvéni} David servum meum, óleo sancto meo unxi eum: manus enim mea auxiliábitur ei, et bráchium meum confortábit eum. (T. P. Allelúja.)
}\switchcolumn\portugues{
\rlettrine{E}{ncontrei} o meu servo David e ungi-o com meu óleo sagrado. Minha mão o socorrerá e o meu braço o fortalecerá. (T. P. Aleluia).
}\end{paracol}

\paragraph{Secreta}
\begin{paracol}{2}\latim{
\rlettrine{S}{ancti} tui, quǽsumus, Dómine, nos ubíque lætíficant: ut, dum eórum mérita recólimus, patrocínia sentiámus. Per Dóminum \emph{\&c.}
}\switchcolumn\portugues{
\qlettrine{Q}{ue} os vossos santos, Senhor, Vos suplicamos, nos alegrem em toda a parte, a fim de que, honrando os seus méritos, sintamos o efeito do seu patrocínio. Por nosso Senhor \emph{\&c.}
}\end{paracol}

\paragraphinfo{Comúnio}{Lc. 12, 42}
\begin{paracol}{2}\latim{
\rlettrine{F}{idélis} servus et prudens, quem constítuit dóminus super famíliam suam: ut det illis in témpore trítici mensúram. (T. P. Allelúja.)
}\switchcolumn\portugues{
\rlettrine{E}{is} o servo fiel e prudente que o Senhor estabeleceu acima da sua família para distribuir, oportunamente, a cada um a sua medida de trigo. (T. P. Aleluia).
}\end{paracol}

\paragraph{Postcomúnio}
\begin{paracol}{2}\latim{
\rlettrine{P}{ræsta,} quǽsumus, omnípotens Deus: ut, de percéptis munéribus grátias exhibéntes, intercedénte beáto {\redx N.} Confessóre tuo atque Pontífice, benefícia potióra sumámus. Per Dóminum \emph{\&c.}
}\switchcolumn\portugues{
\rlettrine{D}{ignai-Vos} permitir, ó Deus omnipotente, que, dando-Vos nós graças pelos benefícios recebidos, alcancemos por intercessão do B. {\redx N.}, vosso Confessor e Pontífice, ainda outros maiores. Por nosso Senhor \emph{\&c.}
}\end{paracol}

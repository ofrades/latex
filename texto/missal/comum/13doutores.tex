\subsectioninfo{Doutores}{Missa In médio Ecclésiae}\label{doutores}

\paragraphinfo{Intróito}{Ecl. 15, 5}
\begin{paracol}{2}\latim{
\rlettrine{I}{n} médio Ecclésiæ apéruit os ejus: et implévit eum Dóminus spíritu sapiéntiæ et intelléctus: stolam glóriæ índuit eum. (T. P. Allelúja, allelúja.) \emph{Ps. 91, 2} Bonum est confitéri Dómino: et psállere nómini tuo, Altíssime.
℣. Gloria Patri \emph{\&c.}
}\switchcolumn\portugues{
\rlettrine{A}{briu-lhe} o Senhor a boca no meio da Igreja, encheu-o com o espírito da sabedoria e da inteligência e revestiu-o com a túnica da glória. (T. P. Aleluia, aleluia.) \emph{Sl. 91, 2} É bom louvar o Senhor: e cantar hinos em honra do vosso nome, ó Altíssimo!
℣. Glória ao Pai \emph{\&c.}
}\end{paracol}

\paragraph{Oração}
\begin{paracol}{2}\latim{
\rlettrine{D}{eus,} qui pópulo tuo ætérnæ salútis beátum {\redx N.} minístrum tribuísti: præsta, quǽsumus; ut, quem Doctórem vitæ habúimus in terris, intercessórem habére mereámur in cœlis. Per Dóminum \emph{\&c.}
}\switchcolumn\portugues{
\slettrine{Ó}{} Deus, que ao vosso povo destinastes o B. {\redx N.} para ministro da eterna salvação, concedei-nos, Vos suplicamos, que, assim como o tivemos como Doutor durante a nossa vida na terra, assim gozemos a sua intercessão no céu. Por nosso Senhor \emph{\&c.}
}\end{paracol}

\paragraphinfo{Epístola}{2. Tm. 4, 1-8}
\begin{paracol}{2}\latim{
Léctio Epístolæ beáti Pauli Apóstoli ad Timotheum.
}\switchcolumn\portugues{
Lição da Ep.ª do B. Ap.º Paulo a Timóteo.
}\switchcolumn*\latim{
\rlettrine{C}{aríssime:} Testíficor coram Deo, et Jesu Christo, qui judicatúrus est vi vos et mórtuos, per advéntum ipsíus et regnum ejus: prǽdica verbum, insta opportúne, importune: árgue, óbsecra, íncrepa in omni patiéntia, et doctrína. Erit enim tempus, cum sanam doctrínam non sustinébunt, sed ad sua desidéria, coacervábunt sibi magistros, pruriéntes áuribus, et a veritáte quidem audítum avértent, ad fábulas autem converténtur. Tu vero vígila, in ómnibus labóra, opus fac Evangelístæ, ministérium tuum ímpie. Sóbrius esto. Ego enim jam delíbor, et tempus resolutiónis meæ instat. Bonum certámen certávi, cursum consummávi, fidem servávi. In réliquo repósita est mihi coróna justítiæ, quam reddet mihi Dóminus in illa die, justus judex: non solum autem mihi, sed et iis, qui díligunt advéntum ejus.
}\switchcolumn\portugues{
\rlettrine{C}{aríssimo:} Conjuro-te diante de Deus e de Jesus Cristo, que há-de julgar vivos e mortos na sua vinda e no seu reino, a que pregues a palavra; instes oportuna e inoportunamente; repreendas; supliques; e ameaces com toda a paciência e doutrina; pois virá tempo em que não suportarão a sã doutrina, mas, indo ao sabor dos seus desejos, procurarão para si muitos mestres, que lhes preguem o que os ouvidos gostam de escutar, e fechem os ouvidos à verdade, para os abrirem às fábulas. Tu, porém, vigia, trabalha em tudo, cumpre o ministério de evangelista e desempenha o teu ministério. Sê sóbrio. Pois quanto a mim sou como uma vítima já aspergida para o sacrifício. O tempo da minha morte já se aproxima. Pelejei o bom combate; acabei a vida; permaneci na fé. Não me falta mais do que esperar a coroa da justiça, que me está reservada, a qual o Senhor, como justo juiz, me dará no grande dia: e não somente a mim, mas também àqueles que amam a sua vinda.
}\end{paracol}

\paragraphinfo{Gradual}{Sl. 36, 30-31}
\begin{paracol}{2}\latim{
\rlettrine{O}{s} justi meditábitur sapiéntiam, et lingua ejus loquétur judícium. ℣. Lex Dei ejus in corde ipsíus: et non supplantabúntur gressus ejus.
}\switchcolumn\portugues{
\rlettrine{A}{} boca do justo falará com sabedoria e a sua língua proclamará a justiça. ℣. A lei do seu Deus está sempre no seu coração e os seus pés não tropeçarão.
}\switchcolumn*\latim{
Allelúja, allelúja. ℣. \emph{Eccli. 45, 9} Amávit eum Dóminus, et ornávit eum: stolam glóriæ índuit eum. Allelúja.
}\switchcolumn\portugues{
Aleluia, aleluia. ℣. \emph{Ecl. 45, 9} Amou-o o Senhor e revestiu-o com a túnica da glória. Aleluia.
}\end{paracol}

\textit{Após a Septuagésima omite-se o Aleluia e o seguinte e diz-se:}

\paragraphinfo{Trato}{Sl. 111, 1-3}
\begin{paracol}{2}\latim{
\rlettrine{B}{eátus} vir, qui timet Dóminum: in mandátis ejus cupit nimis. ℣. Potens in terra erit semen ejus: generátio rectórum benedicétur. ℣. Glória et divítiæ in domo ejus: et justítia ejus manet in sǽculum sǽculi.
}\switchcolumn\portugues{
\rlettrine{B}{em-aventurado} o varão que teme o Senhor e que emprega todo o zelo em obedecer-Lhe. ℣. Sua descendência será poderosa na terra, pois a geração dos justos será abençoada. ℣. Na sua casa haverá glória e riqueza: e a sua justiça subsistirá em todos os séculos dos séculos.
}\end{paracol}

\textit{No T. Pascal omite-se o Gradual e o Trato e diz-se:}

\begin{paracol}{2}\latim{
Allelúja, allelúja. ℣. \emph{Eccli. 45, 9} Amávit eum Dóminus, et ornávit eum: stolam glóriæ índuit eum. Allelúja. ℣. \emph{Osee 14, 6} Justus germinábit sicut lílium: et florébit in ætérnum ante Dóminum. Allelúja.
}\switchcolumn\portugues{
Aleluia, aleluia. ℣. \emph{Ecl. 45, 9} Amou-o o Senhor, ornou-o e revestiu-o com a túnica da glória. Aleluia. ℣. \emph{Os. 14, 6} O justo germinará, como o lírio, e florescerá para sempre diante do Senhor. Aleluia.
}\end{paracol}

\paragraphinfo{Evangelho}{Mt. 5, 13-19}
\begin{paracol}{2}\latim{
\cruz Sequéntia sancti Evangélii secúndum Matthǽum.
}\switchcolumn\portugues{
\cruz Continuação do santo Evangelho segundo S. Mateus.
}\switchcolumn*\latim{
\blettrine{I}{n} illo témpore: Dixit Jesus discípulis suis: Vos estis sal terræ. Quod si sal evanúerit, in quo saliétur? Ad níhilum valet ultra, nisi ut mittátur foras, et conculcétur ab homínibus. Vos estis lux mundi. Non potest cívitas abscóndi supra montem pósita. Neque accéndunt lucérnam, et ponunt eam sub módio, sed super candelábrum, ut lúceat ómnibus qui in domo sunt. Sic lúceat lux vestra coram homínibus, ut vídeant ópera vestra bona, et gloríficent Patrem vestrum, qui in cœlis est. Nolíte putáre, quóniam veni sólvere legem aut prophétas: non veni sólvere, sed adimplére. Amen, quippe dico vobis, donec tránseat cœlum et terra, jota unum aut unus apex non præteríbit a lege, donec ómnia fiant. Qui ergo solvent unum de mandátis istis mínimis, et docúerit sic hómines, mínimus vocábitur in regno cœlórum: qui autem fécerit et docúerit, hic magnus vocábitur in regno cœlórum.
}\switchcolumn\portugues{
\blettrine{N}{aquele} tempo, disse Jesus a seus discípulos: «Sois o sal da terra. Se o sal perde a força, com que salgará? Para nada mais presta, senão para se lançar fora e ser pisado pelos homens. Sois a luz do mundo. Uma cidade situada no cimo de um monte não pode ficar escondida. Nem se acende uma luz para a meter debaixo do alqueire, mas para a colocar no candeeiro, a fim de alumiar todos os que estão em casa. Assim resplandeça a vossa luz diante dos homens, para que vejam as vossas boas obras e glorifiquem vosso Pai, que está nos céus. Não penseis que vim abrogar a Lei ou os Profetas: não vim abrogar, mas aperfeiçoar; porque em verdade vos digo: até que passe o céu e a terra, nem um iota, nem um til se omitirá na Lei. Quem quer, pois, que transgrida, ainda um dos mais pequenos mandamentos, ou ensine os homens a violá-los, será chamado o menor no reino dos céus. Porém, quem os cumprir e ensinar será chamado grande no reino dos céus.
}\end{paracol}

\paragraphinfo{Ofertório}{Sl. 91, 13}
\begin{paracol}{2}\latim{
\qlettrine{J}{ustus} ut palma florébit: sicut cedrus, quæ in Líbano est multiplicábi
tur. (T. P. Allelúja.)
}\switchcolumn\portugues{
\rlettrine{O}{} justo florescerá, como a palmeira, e crescerá, como o cedro do Líbano. (T. P. Aleluia.)
}\end{paracol}

\paragraph{Secreta}
\begin{paracol}{2}\latim{
\rlettrine{S}{ancti} {\redx N.} Pontíficis tui at que Doctóris nobis, Dómine, pia non desit orátio: quæ et múnera nostra concíliet; et tuam nobis indulgéntiam semper obtíneat. Per Dóminum \emph{\&c.}
}\switchcolumn\portugues{
\rlettrine{S}{enhor,} que a piedosa oração de Santo {\redx N.} vosso Pontífice e Doutor, nos não abandone, e que por ela as nossas ofertas Vos sejam agradáveis e alcancemos benigna e continuamente a vossa misericórdia. Por nosso Senhor \emph{\&c.}
}\end{paracol}

\paragraphinfo{Comúnio}{Lc. 12, 42}
\begin{paracol}{2}\latim{
\rlettrine{F}{idélis} servus et prudens, quem constítuit dóminus super famíliam suam: ut det illis in témpore trítici mensúram. (T. P. Allelúja.)
}\switchcolumn\portugues{
\rlettrine{O}{} servo fiel e prudente é destinado pelo Senhor para distribuir, oportunamente, na sua família a cada um a sua medida de trigo. (T. P. Aleluia.)
}\end{paracol}

\paragraph{Postcomúnio}
\begin{paracol}{2}\latim{
\rlettrine{U}{t} nobis, Dómine, tua sacrifícia dent salútem: beátus {\redx N.} Póntifex tuus et Doctor egrégius, quǽsumus, precátor accédat. Per Dóminum nostrum \emph{\&c.}
}\switchcolumn\portugues{
\slettrine{Ó}{} Senhor, dignai-Vos conceder-nos que o B. {\redx N.}, vosso Pontífice e ilustre Doutor, seja nosso intercessor perante Vós, a fim de que este sacrifício nos alcance a salvação. Por nosso Senhor \emph{\&c.}
}\end{paracol}

\textit{Outra Epístola (para certos dias):}\label{doutoresepistola2}

\paragraphinfo{Epístola}{Ecl. 39, 6-14}
\begin{paracol}{2}\latim{
Léctio libri Sapiéntiæ.
}\switchcolumn\portugues{
Lição do Livro da Sabedoria.
}\switchcolumn*\latim{
\qlettrine{J}{ustus} cor suum tradet ad vigilándum dilúculo ad Dóminum, qui fecit illum, et in conspéctu Altíssimi deprecábitur. Apériet os suum in oratióne, et pro delíctis suis deprecábitur. Si enim Dóminus magnus volúerit, spíritu intellegéntias replébit illum: et ipse tamquam imbres mittet elóquia sapiéntiæ suæ, et in oratióne confitébitur Dómino: et ipse díriget consílium ejus et disciplínam, et in abscónditis suis consiliábitur. Ipse palam fáciet disciplínam doctrínæ suæ, et in lege testaménti Dómini gloriábitur. Collaudábunt multi sapiéntiam ejus, et usque in sǽculum non delébitur. Non recédet memória ejus, et nomen ejus requirétur a generatióne in generatiónem. Sapiéntiam ejus enarrábunt gentes, et laudem ejus enuntiábit ecclésia.
}\switchcolumn\portugues{
\rlettrine{O}{} justo aplicará o seu coração e vigiará desde o romper do dia para se unir ao Senhor, que o criou, e oferecer as suas preces ao Altíssimo. Abrirá a sua boca para orar e implorar o perdão dos seus pecados; pois, se o soberano Senhor quiser, enchê-lo-á com o espírito de inteligência. Então ele espalhará, como chuva, as palavras da sua sabedoria e abençoará o Senhor na sua oração. O Senhor inspirará os seus conselhos e instruções; e ele compreenderá os mystérios divinos. Publicará a doutrina que tiver aprendido, e a sua glória será manter-se na lei da aliança com o Senhor. Sua sabedoria receberá louvor de muitos e não cairá no esquecimento. Sua memória se não apagará. Seu nome será honrado de geração em geração. As nações publicarão a sua sabedoria e a Igreja anunciará os seus louvores.
}\end{paracol}

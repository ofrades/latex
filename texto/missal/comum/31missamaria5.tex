\subsectioninfo{5.ª Missa - Desde o Pentecostes até ao Advento}{Missa Salve, sancta Parens da Virgem Maria}\label{missamaria5}

\textit{Como na 3.ª Missa, excepto o seguinte:}

\paragraph{Gradual}
\begin{paracol}{2}\latim{
\rlettrine{B}{enedícta} et venerábilis es, Virgo María: quæ sine tactu pudóris invénia es Mater Salvatóris. ℣. Virgo, Dei Génetrix, quem totus non capit orbis, in tua se clausit víscera factus homo.
}\switchcolumn\portugues{
\rlettrine{B}{endita} e venerável sois, ó Virgem Maria, que fostes Mãe do Salvador, sem que a vossa pureza sofresse a mais leve ofensa. ℣. O Virgem, Mãe de Deus, Aquele que nem todo o universo é capaz de conter, quando se fez homem, esteve encerrado no vosso seio.
}\switchcolumn*\latim{
Allelúja, allelúja. ℣. Post partum, Virgo, invioláta permansísti: Dei Génetrix, intercéde pro nobis. Allelúja.
}\switchcolumn\portugues{
Aleluia, aleluia. ℣. Depois de haverdes dado à luz, permanecestes Virgem imaculada: Intercedei por nós, ó Mãe de Deus. Aleluia.
}\end{paracol}

\paragraphinfo{Ofertório}{Lc. 1, 28 \& 42}
\begin{paracol}{2}\latim{
\rlettrine{A}{ve,} María, grátia plena; Dóminus tecum: benedícta tu in muliéribus, et benedíctus fructus ventris tui.
}\switchcolumn\portugues{
\rlettrine{A}{ve,} Maria, cheia de graça: o Senhor é convosco: bendita sois vós entre as mulheres, e bendito é o fruto do vosso ventre.
}\end{paracol}
\subsectioninfo{Virgem não Mártir}{Missa Vultum tuum}\label{virgemnaomartir2}

\paragraphinfo{Intróito}{Sl. 44, 13, 15 \& 16}
\begin{paracol}{2}\latim{
\rlettrine{V}{ultum} tuum deprecabúntur omnes dívites plebis: adducéntur Regi Vírgines post eam: próximæ ejus adducéntur tibi in lætítia et exsultatióne. (T. P. Allelúja, allelúja.) \emph{Ps. ibid., 2} Eructávit cor meum verbum bonum: dico ego ópera mea Regi.
℣. Gloria Patri \emph{\&c.}
}\switchcolumn\portugues{
\rlettrine{T}{odos} os poderosos da terra implorarão os vossos olhares: após ela, serão apresentadas virgens ao Rei: as suas companheiras serão apresentadas ao Rei com grande alegria e júbilo. (T. P. Aleluia, aleluia.) \emph{Sl. ibid., 2} Meu coração exprimiu uma palavra excelente: Consagro ao Rei as minhas obras!
℣. Glória ao Pai \emph{\&c.}
}\end{paracol}

\paragraph{Oração}
\begin{paracol}{2}\latim{
\rlettrine{E}{xáudi} nos, Deus, salutáris noster: ut, sicut de beátæ {\redx N.} Vírginis tuæ festivitáte gaudémus; ita piæ devotiónis erudiámur affectu. Per Dóminum nostrum \emph{\&c.}
}\switchcolumn\portugues{
\rlettrine{O}{uvi-nos,} ó Deus, nosso salvador, a fim de que, assim como nos alegramos com a festa da vossa B. Virgem {\redx N.}, assim também consigamos alcançar sentimentos de terna devoção. Por nosso Senhor \emph{\&c.}
}\end{paracol}

\paragraphinfo{Epístola}{1. Cor. 7, 25-34}
\begin{paracol}{2}\latim{
Léctio Epístolæ beáti Pauli Apóstoli ad Corínthios.
}\switchcolumn\portugues{
Lição da Ep.ª do B. Ap.º Paulo aos Coríntios.
}\switchcolumn*\latim{
\rlettrine{F}{ratres:} De virgínibus præcéptum Dómini non hábeo: consílium autem do, tamquam misericórdiam consecútus a Dómino, ut sim fidélis. Exístimo ergo hoc bonum esse propter instántem necessitátem, quóniam bonum est hómini sic esse. Alligátus es uxóri? noli quǽrere solutiónem. Solútus es ab uxóre? noli quǽrere uxorem. Si autem acceperis uxorem, non peccásti. Et si núpserit virgo, non peccavit: tribulatiónem tamen carnis habébunt hujúsmodi. Ego autem vobis parco. Hoc ítaque dico, fratres: Tempus breve est: réliquum est, ut, et qui habent uxóres, tamquam non habéntes sint; et qui flent, tamquam non flentes; et qui gaudent, tamquam non gaudéntes; et qui emunt, tamquam non possidéntes; et qui utúntur hoc mundo, tamquam non utántur; prǽterít enim figúra hujus mundi. Volo autem vos sine sollicitúdine esse. Qui sine uxóre est, sollícitus est, quæ Dómini sunt, quómodo pláceat Deo. Qui autem cum uxóre est, sollícitus est, quæ sunt mundi, quómodo pláceat uxóri, et divísus est. Et múlier innúpta et virgo cógitat, quæ Dómini sunt, ut sit sancta córpore et spíritu: in Christo Jesu, Dómino nostro.
}\switchcolumn\portugues{
\rlettrine{M}{eus} irmãos: Quanto às virgens, não recebi preceito do Senhor; mas eis o conselho que dou, para ser fiel à graça que o Senhor misericordiosamente me fez. Creio que é vantajoso ao homem permanecer assim, por causa das instantes necessidades desta vida. Estais unido a uma mulher? Não procureis desligar-vos. Não estais unido a nenhuma mulher? Não procureis mulher. Se, porém, desposastes uma mulher, não pecastes; e, se uma virgem casar, não peca. Contudo, estas pessoas sofrerão as tribulações da carne, o que procuro evitar-vos. Eis, pois, o que vos digo, irmãos: o tempo é breve; o que resta é que os que têm mulheres procedam como se as não tivessem; os que choram, como se não chorassem; os que se regozijam, como se não se regozijassem; os que compram, como se nada possuíssem; e os que usam deste mundo, como se não usassem, pois, a aparência deste mundo passa. Bem quisera que estivésseis sem preocupações. Aquele que não é casado é solícito para com as coisas do Senhor e procura proceder de modo que agrade a Deus. Assim como o que é casado se ocupa solicitamente das coisas deste mundo e do que deverá fazer para agradar a sua mulher; assim ele está dividido. Do mesmo modo a mulher solteira e a virgem pensam nas coisas que são do Senhor, a fim de que sejam santas de corpo e de espírito, em N. S. Jesus Cristo.
}\end{paracol}

\paragraphinfo{Gradual}{Sl. 44, 12 \& 11}
\begin{paracol}{2}\latim{
\rlettrine{C}{oncupívit} Rex decórem tuum, quóniam ipse est Dóminus, Deus tuus. ℣. Audi, fília, et vide, et inclína aurem tuam.
}\switchcolumn\portugues{
\rlettrine{O}{} Rei está cheio de amor por vós, por causa da vossa beleza, pois Ele é o Senhor, vosso Deus. ℣. Ó minha filha, vede e prestai atenção.
}\switchcolumn*\latim{
Allelúja, allelúja. ℣. Hæc est Virgo sápiens, et una de número prudéntum. Allelúja.
}\switchcolumn\portugues{
Aleluia, aleluia. ℣. Esta é a virgem sábia e uma das virgens prudentes. Aleluia.
}\end{paracol}

\textit{Após a Septuagésima omite-se o Aleluia e o seguinte e diz-se:}

\paragraphinfo{Trato}{Sl. 44, 12, 13 \& 10}
\begin{paracol}{2}\latim{
\qlettrine{Q}{uia} concupívit Rex spéciem tuam. ℣. Vultum tuum deprecabúntur omnes divites plebis: fíliæ regum in honóre tuo. ℣. \emph{ibid., 15-16} Adducéntur Regi Vírgines post eam: próximæ ejus afferéntur tibi. ℣. Afferéntur in lætítia et exsultatióne: adducéntur in templum Regis.
}\switchcolumn\portugues{
\rlettrine{P}{ois} o Rei está cheio de amor por vós, por causa da vossa beleza. ℣. Todos os poderosos da terra implorarão os vossos olhares: e as filhas dos reis formam a vossa corte de glória. ℣. \emph{ibid., 15-16} Depois de vós, virão coros de virgens: as suas companheiras serão apresentadas ao Rei. ℣. Serão apresentadas no meio da alegria e do júbilo: e serão introduzidas no templo do Rei.
}\end{paracol}

\textit{No T. Pascal omite-se o Gradual e o Trato e diz-se:}

\begin{paracol}{2}\latim{
Allelúja, allelúja. ℣. Hæc est Virgo sápiens, et una de número prudéntum. Allelúja. ℣. \emph{Sap. 4, 1} O quam pulchra est casta generátio cum claritáte! Allelúja.
}\switchcolumn\portugues{
Aleluia, aleluia. ℣. Esta é a virgem sábia e uma das virgens prudentes. Aleluia. ℣. \emph{Sb. 4, 1} Oh! como é bela a geração casta e gloriosa! Aleluia.
}\end{paracol}

\paragraphinfo{Evangelho}{Página \pageref{virgensmartires1}}

\paragraphinfo{Ofertório}{Sl. 44, 15-16}
\begin{paracol}{2}\latim{
\rlettrine{A}{fferéntur} Regi Vírgines post eam: próximæ ejus afferéntur tibi in lætítia et exsultatióne: adducéntur in templum Regi Dómino. (T. P. Allelúja.)
}\switchcolumn\portugues{
\rlettrine{A}{pós} ela, serão apresentadas virgens ao Rei: as suas companheiras serão introduzidas no meio da alegria e do júbilo: e serão conduzidas ao templo do Rei, seu Senhor. (T. P. Aleluia.)
}\end{paracol}

\paragraph{Secreta}
\begin{paracol}{2}\latim{
\rlettrine{A}{ccépta} tibi sit, Dómine, sacrátæ plebis oblátio pro
tuorum honore Sanctórum: quorum se meritis de tribulatione percepísse cognóscit auxílium. Per Dóminum \emph{\&c.}
}\switchcolumn\portugues{
\rlettrine{A}{ceitai,} Senhor, esta oferta, que Vos consagra o vosso povo fiel em honra dos vossos santos, pelos méritos dos quais reconhece que tem alcançado a vossa assistência nas tribulações. Por nosso Senhor \emph{\&c.}
}\end{paracol}

\paragraphinfo{Comúnio}{Mt. 13, 45-46}
\begin{paracol}{2}\latim{
\rlettrine{S}{ímile} est regnum cœlórum hómini negotiatóri, quærénti bonas margarítas: invénta autem una pretiósa margaríta, dedit ómnia sua, et comparávit eam. (T. P. Allelúja.)
}\switchcolumn\portugues{
\rlettrine{O}{} reino dos céus é semelhante a um homem negociante que procura pérolas boas, e, achando uma de subido valor, vai, vende todos os bens e compra-a. (T. P. Aleluia.)
}\end{paracol}

\paragraph{Postcomúnio}
\begin{paracol}{2}\latim{
\rlettrine{S}{atiásti,} Dómine, famíliam tuam munéribus sacris: ejus, quǽsumus, semper interventióne nos réfove, cujus sollémnia celebrámus. Per Dóminum \emph{\&c.}
}\switchcolumn\portugues{
\rlettrine{H}{avendo} Vós, Senhor, saciado a vossa família com vossos dons sagrados, dignai-Vos favorecer-nos sempre pela intercessão daquela cuja festa celebramos. Por nosso Senhor \emph{\&c.}
}\end{paracol}

\subsectioninfo{Dedicação de uma Igreja}{Missa Terríbilis est}\label{dedicacaoigreja}

\paragraphinfo{Intróito}{Gen. 28, 17}
\begin{paracol}{2}\latim{
\rlettrine{T}{erríbilis} est locus iste: hic domus Dei est et porta cœli: et vocábitur aula Dei. (T. P. Allelúja, allelúja.) \emph{Ps. 83, 2-3} Quam dilécta tabernácula tua, Dómine virtútum! concupíscit, et déficit ánima mea in átria Dómini. ℣. Glória Patri
℣. Gloria Patri \emph{\&c.}
}\switchcolumn\portugues{
\qlettrine{Q}{ue} terrível é este lugar! É verdadeiramente a casa de Deus e a porta do céu: e será chamado o palácio de Deus. (T. P. Aleluia, aleluia.) \emph{Sl. 83, 2-3} Quão dilectos são os vossos tabernáculos, ó Senhor dos exércitos! Minha alma suspira com ardor e em êxtase, em desejos de viver junto dos átrios do Senhor.
℣. Glória ao Pai \emph{\&c.}
}\end{paracol}

\paragraph{Oração}
\begin{paracol}{2}\latim{
\rlettrine{D}{eus,} qui nobis per síngulos annos hujus sancti templi tui consecratiónis réparas diem, et sacris semper mystériis repæséntas incólumes: exáudi preces pópuli tui, et præsta; ut, quisquis hoc templum benefícia petitúrus ingréditur, cuncta se impetrásse lætétur. Per Dóminum \emph{\&c.}
}\switchcolumn\portugues{
\slettrine{Ó}{} Deus, que, anualmente, renovais em nosso favor o dia da consagração deste santo Templo e nos conservais incólumes para podermos sempre celebrar estes sagrados mystérios, ouvi as orações do vosso povo e concedei a todos aqueles que entrarem neste templo, para pedir as vossas graças, a alegria de as alcançarem. Por nosso Senhor \emph{\&c.}
}\end{paracol}

\textit{No dia em que se faz a Dedicação da Igreja e no Oitavário diz-se, em vez da precedente, a seguinte:}

\paragraph{Oração}
\begin{paracol}{2}\latim{
\rlettrine{D}{eus,} qui invisibíliter ómnia cóntines, et tamen pro salúte géneris humáni signa tuæ poténtiæ visibíliter osténdis: templum hoc poténtia tuæ inhabitatiónis illústra, et concéde; ut omnes, qui huc deprecatúri convéniunt, ex quacúmque tribulatióne ad te clamáverint, consolatiónis tuæ benefícia consequántur. Per Dóminum \emph{\&c.}
}\switchcolumn\portugues{
\slettrine{Ó}{} Deus, que, permanecendo invisível, abrangeis, contudo, o universo, e que, entretanto, mostrais visivelmente os milagres do vosso poder para a salvação do género humano, tornai ilustre este Templo, vindo habitar nele com vosso poder, e dignai-Vos conceder a todos aqueles que nele se reunirem para Vos dirigirem suas orações que, em qualquer tribulação em que elevem até Vós os seus clamores, obtenham os benefícios da vossa consolação. Por nosso Senhor \emph{\&c.}
}\end{paracol}

\paragraphinfo{Epístola}{Ap. 21, 2-5.}
\begin{paracol}{2}\latim{
Léctio libri Apocalýpsis beáti Joánnis Apóstoli.
}\switchcolumn\portugues{
Lição do Livro do Apocalipse do B. Ap.º S. João.
}\switchcolumn*\latim{
\rlettrine{I}{n} diébus illis: Vidi sanctam civitátem Jerúsalem novam descendéntem de cœlo a Deo, parátam sicut sponsam ornátam viro suo. Et audívi vocem magnam de throno dicéntem: Ecce tabernáculum Dei cum homínibus, et habitábit cum eis. Et ipsi pópulus ejus erunt, et ipse Deus cum eis erit eórum Deus: et abstérget Deus omnem lácrimam ab óculis eórum: et mors ultra non erit, neque luctus neque clamor neque dolor erit ultra, quia prima abiérunt. Et dixit, qui sedébat in throno: Ecce, nova fácio ómnia.
}\switchcolumn\portugues{
\rlettrine{N}{aqueles} dias, vi a cidade santa, a nova Jerusalém, que vinha de Deus e descia do céu, ornada como uma esposa que se prepara para receber o esposo. E ouvi uma voz forte, que falava do trono, e dizia: «Eis aqui o tabernáculo de Deus no meio dos homens. O Senhor habitará com eles, que serão o seu povo; e o próprio Deus permanecerá com eles e será o seu Deus, enxugando-lhes todas as lágrimas dos seus olhos. Então já não existirão nem a morte, nem as lágrimas, nem os clamores, nem as dores, porque o primeiro estado das cousas terá acabado». E aquele que estava sentado no trono disse: «Eu vou renovar todas as cousas!».
}\end{paracol}

\paragraph{Gradual}
\begin{paracol}{2}\latim{
\rlettrine{L}{ocus} iste a Deo factus est, inæstimábile sacraméntum, irreprehensíbilis est. ℣. Deus, cui astat Angelórum chorus, exáudi preces servórum tuórum.
}\switchcolumn\portugues{
\rlettrine{E}{ste} lugar foi feito por Deus: ele é um mystério inapreciável e isento de qualquer defeito. ℣. Ó Deus, diante de Quem se prostram os caros dos Anjos, ouvi as preces dos vossos servos.
}\switchcolumn*\latim{
Allelúja, allelúja. ℣. \emph{Ps. 137, 2} Adorábo ad templum sanctum tuum: et confitébor nómini tuo. Allelúja.
}\switchcolumn\portugues{
Aleluia, aleluia. ℣. \emph{Sl. 137, 2} Adorar-Vos-ei no vosso santo Templo e louvarei o vosso santo Nome. Aleluia.
}\end{paracol}

\textit{Após a Septuagésima omite-se o Aleluia e o seguinte e diz-se:}

\paragraphinfo{Trato}{Sl. 124, 1-2}
\begin{paracol}{2}\latim{
\qlettrine{Q}{ui} confídunt in Dómino, sicut mons Sion: non commovébitur in ætérnum, qui hábitat in Jerúsalem. ℣. Montes in circúitu ejus, et Dóminus in circúitu pópuli sui, ex hoc nunc, et usque in sǽculum.
}\switchcolumn\portugues{
\rlettrine{A}{queles} que confiam no Senhor são como o monte Sião: aquele que habita em Jerusalém nunca será abalado. ℣. Assim como Jerusalém está rodeada de montanhas, assim o Senhor circunda o seu povo, agora e sempre.
}\end{paracol}

\textit{No T. Pascal omite-se o Gradual e o Trato diz-se:}

\begin{paracol}{2}\latim{
Allelúja, allelúja. ℣. \emph{Ps. 137, 2} Adorábo ad templum sanctum tuum: et confitébor nómini tuo. Allelúja. ℣. Bene fundáta est domus Dómini supra firmam petram. Allelúja.
}\switchcolumn\portugues{
Aleluia, aleluia. ℣. \emph{Sl. 137, 2} Adorar-Vos-ei no vosso santo Templo: e louvarei o vosso santo Nome. ℣. A casa do Senhor está edificada solidamente sobre pedra firme. Aleluia.
}\end{paracol}

\paragraphinfo{Evangelho}{Lc. 19, 1-10}
\begin{paracol}{2}\latim{
\cruz Sequéntia sancti Evangélii secúndum Lucam.
}\switchcolumn\portugues{
\cruz Continuação do santo Evangelho segundo S. Lucas.
}\switchcolumn*\latim{
\blettrine{I}{n} illo témpore: Ingréssus Jesus perambulábat Jéricho. Et ecce, vir nómine Zachǽus: et hic princeps erat publicanórum, et ipse dives: et quærébat vidére Jesum, quis esset: et non póterat præ turba, quia statúra pusíllus erat. Et præcúrrens ascéndit in arborem sycómorum, ut vidéret eum; quia inde erat transitúrus. Et cum venísset ad locum, suspíciens Jesus vidit illum, et dixit ad eum: Zachǽe, féstinans descénde; quia hódie in domo tua opórtet me manére. Et féstinans descéndit, et excépit illum gaudens. Et cum vidérent omnes, murmurábant, dicéntes, quod ad hóminem peccatórem divertísset. Stans au tem Zachǽus, dixit ad Dóminum: Ecce, dimídium bonórum meórum, Dómine, do paupéribus: et si quid áliquem defraudávi, reddo quádruplum. Ait Jesus ad eum: Quia hódie salus dómui huic facta est: eo quod et ipse fílius sit Abrahæ. Venit enim Fílius hóminis quǽrere et salvum fácere, quod períerat.
}\switchcolumn\portugues{
\blettrine{N}{aquele} tempo, havendo Jesus entrado em Jericó, atravessava a cidade. Ora, havia ali um homem, chamado Zaqueu, príncipe dos publicanos e muito rico, que procurava ver Jesus para o conhecer; mas o não conseguia por causa das turbas do povo, pois ele era de baixa estatura. Correu, pois, adiante e subiu para um sicómoro, para ver Jesus, que devia passar por aquele lugar. Chegando Jesus ali, ergueu os olhos e, vendo-o, disse-lhe: «Zaqueu, descei depressa, porque me convém hospedar-me hoje em vossa casa». Imediatamente Zaqueu desceu e recebeu Jesus com alegria. Vendo as pessoas isto, começaram a murmurar, dizendo que Jesus ia hospedar-se em casa de um pecador. Entretanto Zaqueu estava perante o Senhor e dizia-Lhe: «Senhor, eis que vou dar metade dos meus bens aos pobres; e, se defraudei alguém, restituirei o quádruplo». Jesus disse então: «Esta casa recebeu hoje a salvação, pois este também é filho de Abraão. O Filho do homem veio para procurar e salvar o que estava perdido!».
}\end{paracol}

\paragraphinfo{Ofertório}{1. Cr. 29, 17 et 18}
\begin{paracol}{2}\latim{
\rlettrine{D}{ómine} Deus, in simplicitáte cordis mei lætus óbtuli univérsa; et pópulum tuum, qui repertus est, vidi cum ingénti gáudio: Deus Israël, custódi hanc voluntátem, allelúja.
}\switchcolumn\portugues{
\rlettrine{S}{enhor,} meu Deus, foi com simplicidade de coração e com alegria que Vos oferecer todas as cousas: foi com intenso júbilo que vi reunido o vosso povo. Ó Deus de Israel, conservai em minha alma estas boas disposições. Aleluia.
}\end{paracol}

\paragraphinfo{Secreta}{Na Igreja}
\begin{paracol}{2}\latim{
\rlettrine{A}{nnue,} quǽsumus, Dómine, précibus nostris: ut, quicúmque intra templi hujus, cujus anniversárium dedicatiónis diem celebrámus, ámbitum continémur, plena tibi atque perfécta córporis et ánimæ devotióne placeámus; ut, dum hæc vota præséntia réddimus, ad ætérna prǽmia, te adjuvante, perveníre mereámur. Per Dóminum \emph{\&c.}
}\switchcolumn\portugues{
\rlettrine{D}{ignai-Vos,} Senhor, ouvir as nossas orações de modo que todos os que nos encontramos neste santo Templo, de cuja Dedicação celebramos o aniversário, Vos agradecemos com a oferta inteira e perfeita, que Vos fazemos, do nosso corpo e da nossa alma; e permiti que, oferecendo-Vos estes dons, alcancemos a felicidade eterna com vosso auxílio. Por nosso Senhor \emph{\&c.}
}\end{paracol}

\paragraphinfo{Secreta}{Fora da Igreja}
\begin{paracol}{2}\latim{
\rlettrine{A}{nnue,} quǽsumus, Dómine, précibus nostris: ut, dum hæc vota præséntia réddimus, ad ætérna prǽmia, te adjuvánte, perveníre mereámur. Per Dóminum \emph{\&c.}
}\switchcolumn\portugues{
\rlettrine{D}{ignai-Vos,} Senhor, ouvir as nossas orações; e permiti que, oferecendo-Vos estes dons, alcancemos a felicidade eterna com vosso auxílio. Por nosso Senhor \emph{\&c.}
}\end{paracol}

\textit{No dia em que se faz a Dedicação e no seu Oitavário diz-se, em vez da precedente, a seguinte:}

\paragraph{Secreta}
\begin{paracol}{2}\latim{
\rlettrine{D}{eus,} qui sacrandórum tibi auctor es múnerum, effúnde super hanc oratiónis domum benedictiónem tuam: ut ab ómnibus, in ea invocántibus nomen tuum, defensiónis tuæ auxílium se nitátur. Per Dóminum \emph{\&c.}
}\switchcolumn\portugues{
\slettrine{Ó}{} Deus, que sois o autor dos dons que Vos consagramos, lançai a vossa bênção sobre esta casa de oração, a fim de que todos aqueles que aqui invocarem o vosso Nome sintam o auxílio da vossa defesa. Por nosso Senhor \emph{\&c.}
}\end{paracol}

\paragraphinfo{Comúnio}{Mt. 21, 13}
\begin{paracol}{2}\latim{
\rlettrine{D}{omus} mea domus oratiónis vocábitur, dicit Dóminus: in ea omnis, qui pétii, accipit; et qui quærit, invénit; et pulsánti aperiétur. (T. P. Allelúja.)
}\switchcolumn\portugues{
\rlettrine{A}{} minha casa será chamada casa de oração, diz o Senhor: e todo aquele que aí pede, recebe: e o que procura, acha: e ao que bate, abrir-se-lhe-á. (T. P. Aleluia.)
}\end{paracol}

\paragraph{Postcomúnio}
\begin{paracol}{2}\latim{
\rlettrine{D}{eus,} qui de vivis et electis lapídibus ætérnum majestáti tuæ prǽparas habitáculum: auxiliáre pópulo tuo supplicánti; ut, quod Ecclésiæ tuæ corporálibus próficit spátiis, spirituálibus amplificétur augméntis. Per Dóminum nostrum \emph{\&c.}
}\switchcolumn\portugues{
\slettrine{Ó}{} Deus, que preparais para a vossa majestade um Templo de pedras vivas e escolhidas para nele habitardes eternamente, auxiliai o vosso povo suplicante, a fim de que o aumento dos templos materiais faça crescer em proveito da Igreja os seus bens espirituais. Por nosso Senhor \emph{\&c.}
}\end{paracol}

\textit{No dia em que se faz a Dedicação e no seu Oitavário diz-se, em vez do precedente, o seguinte:}

\paragraph{Postcomúnio}
\begin{paracol}{2}\latim{
\qlettrine{Q}{uǽsumus,} omnípotens Deus: ut in hoc loco, quem nómini tuo indígni dedicávimus, cunctis peténtibus aures tuæ pietátis accómmodes. Per Dóminum \emph{\&c.}
}\switchcolumn\portugues{
\rlettrine{D}{ignai-Vos} conceder-nos, ó Deus omnipotente, que neste lugar, que, ainda que indignos, dedicámos ao vosso nome, ouçais benignamente todos os que Vos implorarem. Por nosso Senhor \emph{\&c.}
}\end{paracol}

\textit{Na Festa da Dedicação dum Altar celebra-se a Missa precedente, com excepção do seguinte:}

\paragraph{Oração}
\begin{paracol}{2}\latim{
\rlettrine{D}{eus,} qui ex omni coaptatióne Sanctórum ætérnum tibi condis habitáculum: da ædificatióni tuæ increménta cœléstia; ut, quorum hic relíquias pio amóre compléctimur, eórum semper méritis adjuvémur. Per Dóminum \emph{\&c.}
}\switchcolumn\portugues{
\slettrine{Ó}{} Deus, que, de um composto dos vossos Santos, fundais um templo eterno para vossa habitação, permiti que este palácio celestial tenha um aumento constante, e que os Santos, cujas relíquias honramos neste lugar com amor pio, nos sirvam com seus méritos de perpétuo auxílio. Por nosso Senhor \emph{\&c.}
}\end{paracol}

\paragraph{Secreta}
\begin{paracol}{2}\latim{
\rlettrine{D}{escéndat,} quǽsumus, Dómine, Deus noster, Spíritus tuus Sanctus super hoc altare: qui et pópuli tui dona sanctíficet, et suméntium corda dignánter emúndet. Per Dóminum \emph{\&c.}
}\switchcolumn\portugues{
\qlettrine{Q}{ue} o vosso Espírito Santo desça sobre este altar, Vos suplicamos, ó Senhor, nosso Deus; e que se digne santificar os dons do vosso povo e purificar os corações daqueles que tomarem parte nele. Por nosso Senhor \emph{\&c.}
}\end{paracol}

\paragraph{Postcomúnio}
\begin{paracol}{2}\latim{
\rlettrine{O}{mnípotens} sempitérne Deus, altare hoc, nómini tuo dedicátum, cæléstis virtútis benedictióne sanctífica: et ómnibus in te sperántibus auxílii tui munus osténde; ut et hic sacramentórum viri tus et votórum obtineátur efféctus. Per Dóminum \emph{\&c.}
}\switchcolumn\portugues{
\slettrine{Ó}{} Deus, omnipotente e sempiterno, santificai com a bênção do vosso celestial poder este Altar dedicado ao vosso nome; e a todos aqueles que esperam em Vós concedei o vosso auxílio, a fim de que neste lugar obtenham a virtude dos vossos Sacramentos e o efeito dos seus votos. Por nosso Senhor \emph{\&c.}
}\end{paracol}

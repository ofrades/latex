\subsectioninfo{Festas da B. V. Maria}{Missa Salve, sancta Parens}\label{comumfestasmaria1}

\paragraphinfo{Intróito}{Sedulius}
\begin{paracol}{2}\latim{
\rlettrine{S}{alve,} sancta Parens, eníxa puérpera Regem: qui cœlum terrámque regit in sǽcula sæculórum. (T. P. Allelúja, allelúja.) \emph{Ps. 44, 2} Eructávit cor meum verbum bonum: dico ego ópera mea Regi.
℣. Gloria Patri \emph{\&c.}
}\switchcolumn\portugues{
\rlettrine{S}{alve,} ó Santa Maria, em cujo seio foi gerado o Rei que governa o céu e a terra em todos os séculos dos séculos (T. P. Aleluia, aleluia.) \emph{Sl. 44, 2} Meu coração exprimiu uma excelente palavra: Consagro ao Rei as minhas obras.
℣. Glória ao Pai \emph{\&c.}
}\end{paracol}

\paragraph{Oração}
\begin{paracol}{2}\latim{
\rlettrine{C}{oncéde} nos fámulos tuos, quǽsumus, Dómine Deus, perpátua mentis et córporis sanitáte gaudére: et, gloriósa beátæ Maríæ semper Vírginis intercessióne, a præsénti liberári tristítia et ætérna pérfrui lætítia. Per Dóminum \emph{\&c.}
}\switchcolumn\portugues{
\rlettrine{S}{enhor} Deus, Vos suplicamos, concedei aos vossos servos o gozo da perpétua saúde da alma e do corpo, e pela gloriosa intercessão da B. Maria, sempre Virgem, permiti que sejamos livres das tristezas do tempo presente e alcancemos o gozo da alegria eterna. Por nosso Senhor \emph{\&c.}
}\end{paracol}

\paragraphinfo{Epístola}{Ecl. 24, 14-16}
\begin{paracol}{2}\latim{
Léctio libri Sapiéntiæ.
}\switchcolumn\portugues{
Lição do Livro da Sabedoria.
}\switchcolumn*\latim{
\rlettrine{A}{b} inítio et ante sǽcula creáta sum, et usque ad futúrum sǽculum non désinam, et in habitatióne sancta coram ipso ministrávi. Et sic in Sion firmáta sum, et in civitáte sanctificáta simíliter requiévi, et in Jerúsalem potéstas mea. Et radicávi in pópulo honorificáto, et in parte Dei mei heréditas illíus, et in plenitúdine sanctórum deténtio mea.
}\switchcolumn\portugues{
\rlettrine{F}{ui} criada desde o princípio, antes de todos os séculos, e não deixarei de existir até à eternidade. Exerci perante Ele o meu ministério; e deste modo tenho habitação fixa em Sião. Ele deixa-me descansar na cidade santa e tenho poder em Jerusalém. Arraiguei-me em um povo glorioso, da parte do meu Deus e da sua herança, e permaneço na companhia dos santos.
}\end{paracol}

\paragraph{Gradual}
\begin{paracol}{2}\latim{
\rlettrine{B}{enedícta} et venerábilis es, Virgo María: quæ sine tactu pudóris invénia es Mater Salvatóris. ℣. Virgo, Dei Génetrix, quem totus non capit orbis, in tua se clausit víscera factus homo.
}\switchcolumn\portugues{
\rlettrine{B}{endita} e venerável sois vós, ó Virgem Maria, que fostes Mãe do Salvador, sem que a vossa pureza sofresse a mais leve ofensa. ℣. Ó Virgem Mãe de Deus, Aquele que nem todo o universo é capaz de conter, esteve encerrado, quando se fez homem, no vosso seio.
}\switchcolumn*\latim{
Allelúja, allelúja. ℣. Post partum, Virgo, invioláta perí mansisti: Dei Génetrix, intercéde pro nobis. Allelúja.
}\switchcolumn\portugues{
Aleluia, aleluia. Depois de haverdes dado à luz, permanecestes Virgem Imaculada: Intercedei por nós, ó Mãe de Deus. Aleluia.
}\end{paracol}

\textit{No Advento, em vez do verso precedente, diz-se:}

\begin{paracol}{2}\latim{
Allelúja, allelúja. ℣. \emph{Luc. 1, 28} Ave, María, grátia plena; Dóminus tecum: benedícta tu in muliéribus. Allelúja.
}\switchcolumn\portugues{
Aleluia, aleluia. ℣. \emph{Lc. 1, 28} Ave, Maria, cheia de graça: o Senhor é convosco: bendita sois vós entre as mulheres. Aleluia.
}\end{paracol}

\textit{Após a Septuagésima omite-se o Aleluia e o seguinte e diz-se:}

\paragraph{Trato}
\begin{paracol}{2}\latim{
\rlettrine{G}{aude,} María Virgo, cunctas hǽreses sola interemísti. ℣. Quæ Gabriélis Archángeli dictis credidísti. ℣. Dum Virgo Deum et hóminem genuísti: et post partum, Virgo, invioláta permansísti. ℣. Dei Génetrix, intercéde pro nobis.
}\switchcolumn\portugues{
\rlettrine{R}{egozijai-vos,} ó Virgem Maria, pois só vós fostes capaz de destruir todas as heresias. ℣. Acreditastes nas palavras do Arcanjo Gabriel. ℣. Sendo Virgem, gerastes o Homem-Deus; e, depois de haverdes dado à luz, permanecestes Virgem Imaculada. ℣. Intercedei por nós, ó Mãe de Deus.
}\end{paracol}

\textit{No T. Pascal omite-se o Gradual e o Trato e diz-se:}

\begin{paracol}{2}\latim{
Allelúja, allelúja. ℣. \emph{Num. 17, 8} Virga Jesse flóruit: Virgo Deum et hóminem génuit: pacem Deus réddidit, in se reconcílians ima summis. Allelúja. ℣. \emph{Luc. 1, 28} Ave, María, grátia plena; Dóminus tecum: benedícta tu in muliéribus. Allelúja.
}\switchcolumn\portugues{
Aleluia, aleluia. ℣. \emph{Nm. 17, 8} A vara de Jessé floresceu: a Virgem deu à luz do mundo o Homem-Deus: restabeleceu Deus a paz, reconciliando na sua pessoa a nossa baixeza com a suprema grandeza. Aleluia. \emph{Lc. 1, 28} Ave, Maria, cheia de graça: o Senhor é convosco: bendita sois vós entre as mulheres. Aleluia.
}\end{paracol}

\paragraphinfo{Evangelho}{Lc. 11, 27-28}
\begin{paracol}{2}\latim{
\cruz Sequéntia sancti Evangélii secúndum Lucam.
}\switchcolumn\portugues{
\cruz Continuação do santo Evangelho segundo S. Lucas.
}\switchcolumn*\latim{
\blettrine{I}{n} illo témpore: Loquénte Jesu ad turbas, extóllens vocem quædam múlier de turba, dixit illi: Beátus venter, qui te portávit, et úbera, quæ suxísti. At ille dixit: Quinímmo beáti, qui áudiunt verbum Dei, et custódiunt illud.
}\switchcolumn\portugues{
\blettrine{N}{aquele} tempo, falando Jesus às turbas, eis que uma mulher, elevando a voz no meio da multidão, Lhe disse: «Bem-aventurado o seio que Vos trouxe; bem-aventurados os peitos que Vos amamentaram!». E Jesus respondeu, dizendo: «Bem-aventurados, antes, aqueles que ouvem a palavra de Deus e a cumprem».
}\end{paracol}

\paragraphinfo{Ofertório}{Lc. 1, 28 \& 42}
\begin{paracol}{2}\latim{
\rlettrine{A}{ve,} María, grátia plena; Dóminus tecum: benedícta tu in muliéribus, et benedíctus fructus ventris tui. (T. P. Allelúja.)
}\switchcolumn\portugues{
\rlettrine{A}{ve} Maria, cheia de graça: o Senhor é convosco: bendita sois vós entre as mulheres, e bendito é o fruto do vosso ventre. (T. P. Aleluia.)
}\end{paracol}

\paragraph{Secreta}
\begin{paracol}{2}\latim{
\rlettrine{T}{ua,} Dómine, propitiatióne, et beátæ Maríæ semper Vírginis intercessióne, ad perpétuam atque præséntem hæc oblátio nobis profíciat prosperitátem et pacem. Per Dóminum \emph{\&c.}
}\switchcolumn\portugues{
\rlettrine{P}{ela} vossa misericórdia, Senhor, e pela intercessão da B. Maria, sempre Virgem, fazei que esta oblação nos assegure a prosperidade e a paz, agora e sempre. Por nosso Senhor \emph{\&c.}
}\end{paracol}

\paragraph{Comúnio}
\begin{paracol}{2}\latim{
\rlettrine{B}{eáta} viscera Maríæ Vírginis, quæ portavérunt ætérni Patris Fílium. (T. P. Allelúja.)
}\switchcolumn\portugues{
\rlettrine{B}{em-aventuradas} as entranhas da Virgem Maria, que trouxeram encerrado o Filho do Pai Eterno. (T. P. Aleluia.)
}\end{paracol}

\paragraph{Postcomúnio}
\begin{paracol}{2}\latim{
\rlettrine{S}{umptis,} Dómine, salútis nostræ subsídiis: da, quǽsumus, beátæ Maríæ semper Vírginis patrocíniis nos ubíque prótegi; in cujus veneratióne hæc tuæ obtúlimus majestáti. Per Dóminum \emph{\&c.}
}\switchcolumn\portugues{
\rlettrine{H}{avendo} nós alcançado o poderoso auxílio da vossa salvação, Senhor, fazei, Vos imploramos, que sejamos protegidos com o patrocínio da B. Maria, sempre Virgem, em cuja honra oferecemos este sacrifício à vossa majestade. Por nosso Senhor \emph{\&c.}
}\end{paracol}

\subsectioninfo{Mártir}{Missa Protexísti me}\label{martir}

\paragraphinfo{Intróito}{Sl. 63, 3}
\begin{paracol}{2}\latim{
\rlettrine{P}{rotexísti} me, Deus, a convéntu malignántium, allelúja: a multitúdine operántium iniquitátem, allelúja, allelúja. \emph{Ps. ibid., 2} Exáudi, Deus, oratiónem meam, cum déprecor: a timóre inimíci éripe ánimam meam.
℣. Gloria Patri \emph{\&c.}
}\switchcolumn\portugues{
\rlettrine{P}{rotegestes-me,} ó Deus, contra os conluios dos maus e contra a multidão daqueles que cometem iniquidades. Aleluia, aleluia. \emph{Sl. ibid., 2} Ouvi, ó Deus, a oração que Vos dirijo: livrai a minha alma do temor do inimigo.
℣. Glória ao Pai \emph{\&c.}
}\end{paracol}

\textit{Por um Mártir Pontífice diz-se a seguinte:}

\paragraph{Oração}
\begin{paracol}{2}\latim{
\rlettrine{I}{nfirmitátem} nostram réspice, omnípotens Deus: et, quia pondus própriæ actiónis gravat, beáti {\redx N.} Mártyris tui atque Pontíficis intercéssio gloriósa nos prótegat. Per Dóminum \emph{\&c.}
}\switchcolumn\portugues{
\slettrine{Ó}{} Deus omnipotente, olhai para a nossa fraqueza: e, visto que estamos oprimidos com o peso dos nossos pecados, dignai-Vos permitir que sejamos protegidos pela gloriosa intercessão do B. {\redx N.}, vosso Mártir e Pontífice. Por nosso Senhor \emph{\&c.}
}\end{paracol}

\textit{Outras vezes, em vez da Precedente, diz-se esta:}

\paragraph{Oração}
\begin{paracol}{2}\latim{
\rlettrine{D}{eus,} qui nos beáti {\redx N.} Mártyris tui atque Pontíficis ánnua sollemnitáte lætíficas: concéde propítius; ut, cujus natalítia cólímus, de ejúsdem étiam protectióne gaudeámus. Per Dóminum \emph{\&c.}
}\switchcolumn\portugues{
\slettrine{Ó}{} Deus, que nos alegrais com a solenidade anual do B. {\redx N.} vosso Mártir e Pontífice, concedei-nos propício que nos congratulemos com a protecção daquele cujo nascimento no céu celebramos. Por nosso Senhor \emph{\&c.}
}\end{paracol}

\textit{Por um Mártir não Pontífice diz-se a seguinte:}

\paragraph{Oração}
\begin{paracol}{2}\latim{
\rlettrine{P}{ræsta,} quǽsumus, omnípotens Deus: ut, qui beáti {\redx N.} Mártyris tui natalítia cólimus, intercessióne ejus, in tui nóminis amóre roborémur. Per Dóminum \emph{\&c.}
}\switchcolumn\portugues{
\slettrine{C}{oncedei-nos,} ó Deus omnipotente, Vos suplicamos, que, celebrando nós o nascimento do vosso B. Mártir {\redx N.}, sejamos confirmados pela sua intercessão no amor ao vosso nome. Por nosso Senhor \emph{\&c.}
}\end{paracol}

\textit{Outras vezes, em vez da Precedente, diz-se esta:}

\paragraph{Oração}
\begin{paracol}{2}\latim{
\rlettrine{P}{ræsta,} quǽsumus, omnípotens Deus: ut, intercedénte beáto {\redx N.} Mártyre tuo, et a cunctis adversitátibus liberémur in córpore, et a pravis cogitatiónibus mundémur in mente. Per Dóminum \emph{\&c.}
}\switchcolumn\portugues{
\slettrine{P}{ermiti,} ó Deus omnipotente, Vos imploramos, que, pela intercessão do vosso B. Mártir {\redx N.}, os nossos corpos sejam livres de todas as adversidades e as nossas almas purificadas dos maus pensamentos. Por nosso Senhor \emph{\&c.}
}\end{paracol}

\paragraphinfo{Epístola}{Sb. 5, 1-5}
\begin{paracol}{2}\latim{
Léctio libri Sapiéntiæ.
}\switchcolumn\portugues{
Lição do Livro da Sabedoria.
}\switchcolumn*\latim{
\rlettrine{S}{tabunt} justi in magna constántia advérsus eos, qui se angustiavérunt et qui abstulérunt labóres eórum. Vidéntes turbabúntur timore horríbili, et mirabúntur in subitatióne
insperátæ salútis, dicéntes intra se, pœniténtiam agéntes, et præ angústia spíritus geméntes: Hi sunt, quos habúimus aliquándo in derísum et in similitúdinem impropérii. Nos insensáti vitam illórum æstimabámus insániam, et finem illórum sine honóre: ecce, quómodo computáti sunt inter fílios Dei, et inter Sanctos sors illórum est.
}\switchcolumn\portugues{
\rlettrine{E}{ntão,} os justos erguer-se-ão com grande coragem contra aqueles que os oprimiam e a quem arrebatavam o fruto dos seus trabalhos. Vendo-os assim, os maus perturbar-se-ão, cheios de pavor, e ficarão assombrados com a súbita e inesperada salvação dos justos, dizendo de si para si, arrependidos e angustiados: «Estes são aqueles a quem outrora quisemos injuriar com nossas zombarias e insultos. Insensatos que nós fomos! Pareceu-nos que sua vida era uma loucura, e a sua morte uma vergonha; mas eis que os vemos elevados à dignidade de filhos de Deus e compartilhando da glória dos santos!»
}\end{paracol}

\begin{paracol}{2}\latim{
Allelúja, allelúja. ℣. \emph{Ps. 88, 6} Confitebúntur cœli mirabília tua, Dómine: étenim veritátem tuam in ecclésia sanctórum. Allelúja. ℣. Ps. 20, 4. Posuísti, Dómine, super caput ejus corónam de lápide pretióso. Allelúja.
}\switchcolumn\portugues{
Aleluia, aleluia. ℣. \emph{Sl. 88, 6} Senhor, que os céus festejem as vossas maravilhas; que a vossa verdade seja exaltada na assembleia dos santos. Aleluia. Impusestes na sua cabeça, Senhor, uma coroa de pedras preciosas. Aleluia.
}\end{paracol}

\paragraphinfo{Evangelho}{Jo. 15, 1-7}
\begin{paracol}{2}\latim{
\cruz Sequéntia sancti Evangélii secúndum Joánnem.
}\switchcolumn\portugues{
\cruz Continuação do santo Evangelho segundo S. João.
}\switchcolumn*\latim{
\blettrine{I}{n} illo témpore: Dixit Jesus discípulis suis: Ego sum vitis vera: et Pater meus agrícola est. Omnem pálmitem in me non feréntem fructum, tollet eum: et omnem, qui fert fructum, purgábit eum, ut fructum plus áfferat. Jam vos mundi estis propter sermónem, quem locútus sum vobis. Mane te in me: et ego in vobis. Sicut palmes non potest ferre fructum a semetípso, nisi mánserit in vite: sic nec vos, nisi in me manséritis. Ego sum vitis, vos pálmites: qui manet in me, et ego in eo, hic fert fructum multum: quia sine me nihil potéstis fácere. Si quis in me non mánserit, mittétur foras sicut palmes, et aréscet, et cólligent eum, et in ignem mittent, et ardet. Si manséritis in me, et verba mea in vobis mánserint: quodcúmque voluéritis, petétis, et fiet vobis.
}\switchcolumn\portugues{
\blettrine{E}{u} sou a verdadeira vinha e meu Pai é o vinhateiro. Toda a videira que não der fruto em mim será cortada por Ele, assim como podará a que der fruto, para que o dê com mais abundância. Vós estais já limpos em razão da doutrina que vos tenho pregado. Permanecei em mim, pois eu permaneço em vós. Assim como a videira não pode dar fruto só por si, sem permanecer unida à cepa, assim também não podereis vós dar fruto se não permanecerdes unidos a mim. Eu sou a cepa da vinha, e vós sois as vides. Aquele que permanece em mim, eu permaneço nele, e dará abundante fruto; pois nada podereis fazer sem mim. Se alguém não permanecer em mim, será arrancado e lançado fora, como uma vide seca. Então secará e levá-la-ão para a lançarem no fogo, em que arderá. Se permanecerdes em mim e as minhas palavras permanecerem em vós, tudo o que quiserdes podereis pedir, que vos será concedido.
}\end{paracol}

\paragraphinfo{Ofertório}{Sb. 88, 6}
\begin{paracol}{2}\latim{
\rlettrine{C}{onfitebúntur} cœli mirabília tua, Dómine: et veritátem tuam in ecclésia sanctórum, allelúja, allelúja.
}\switchcolumn\portugues{
\rlettrine{S}{enhor,} que os céus publiquem as vossas maravilhas; que, a vossa verdade seja exaltada na assembleia dos santos. Aleluia.
}\end{paracol}

\textit{Por um Mártir Pontífice diz-se a seguinte:}

\paragraph{Secreta}
\begin{paracol}{2}\latim{
\rlettrine{H}{óstias} tibi, Dómine, beáti {\redx N.} Mártyris tui atque Pontíficis dicátas méritis, benígnus assúme: et ad perpétuum nobis tríbue proveníre subsídium. Per Dóminum \emph{\&c.}
}\switchcolumn\portugues{
\rlettrine{A}{ceitai} benigno, Senhor, as hóstias que Vos oferecemos pelos méritos do B. {\redx N.}, vosso Mártir e Pontífice; e dignai-Vos permitir que em virtude delas alcancemos o vosso perpétuo socorro. Por nosso Senhor \emph{\&c.}
}\end{paracol}

\textit{Outras vezes, em vez da Precedente, diz-se esta:}

\paragraph{Secreta}
\begin{paracol}{2}\latim{
\rlettrine{M}{únera} tibi, Dómine, dicáta sanctífica: et, intercedénte beáto {\redx N.} Mártyre tuo atque Pontífice, per éadem nos placátus inténde. Per Dóminum \emph{\&c.}
}\switchcolumn\portugues{
\rlettrine{S}{antificai,} Senhor, estes dons que Vos são oferecidos, a fim de que pela intercessão do B. {\redx N.}, vosso Mártir e Pontífice, Vos digneis aplacar-Vos, e olhar aplacado para nós. Por nosso Senhor \emph{\&c.}
}\end{paracol}

\textit{Por um Mártir não Pontífice diz-se a seguinte:}

\paragraph{Secreta}
\begin{paracol}{2}\latim{
\rlettrine{M}{unéribus} nostris, quǽsumus, Dómine, precibúsque suscéptis: et cœléstibus nos munda mystériis, et cleménter exáudi. Per Dóminum nostrum \emph{\&c.}
}\switchcolumn\portugues{
\rlettrine{H}{avendo} Vós aceitado os nossos dons e as nossas orações, dignai-Vos purificar-nos com vossos celestiais mystérios e ouvir-nos clementemente. Por nosso Senhor \emph{\&c.}
}\end{paracol}

\textit{Outras vezes, em vez da Precedente, diz-se esta:}

\paragraph{Secreta}
\begin{paracol}{2}\latim{
\rlettrine{A}{ccépta} sit in conspéctu tuo, Dómine, nostra devótio: et ejus nobis fiat supplicatióne salutáris, pro cujus sollemnitáte defértur. Per Dóminum \emph{\&c.}
}\switchcolumn\portugues{
\rlettrine{A}{ceitai} benignamente, Senhor, esta oferta que a nossa devoção Vos apresenta; e permiti que nos alcance a salvação pelas orações daquele em cuja festa Vo-la apresentamos. Por nosso Senhor \emph{\&c.}
}\end{paracol}

\paragraphinfo{Comúnio}{Sl. 63, 11}
\begin{paracol}{2}\latim{
\rlettrine{L}{ætábitur} justus in Dómino, et sperábit in eo: et laudabúntur omnes recti corde, allelúja, allelúja.
}\switchcolumn\portugues{
\rlettrine{O}{} justo rejubilará no Senhor e nele porá a sua confiança, pois todos aqueles que possuem o coração recto serão louvados. Aleluia, aleluia.
}\end{paracol}

\textit{Por um Mártir Pontífice diz-se o seguinte:}

\paragraph{Postcomúnio}
\begin{paracol}{2}\latim{
\rlettrine{R}{efécti} participatióne múneris sacri, quǽsumus, Dómine, Deus noster: ut, cujus exséquimur cultum, intercedénte beáto {\redx N.} Martyre tuo atque Pontifice, sentiámus efféctum. Per Dóminum \emph{\&c.}
}\switchcolumn\portugues{
\qlettrine{S}{aciados} com a participação do dom sacratíssimo, Vos suplicamos, ó Senhor, nosso Deus, fazei-nos sentir pela intercessão do B. {\redx N.}, vosso Mártir e Pontífice, o efeito do mystério, que celebrámos. Por nosso Senhor \emph{\&c.}
}\end{paracol}

\textit{Outras vezes, em vez, do Precedente, diz-se o seguante:}

\paragraph{Postcomúnio}
\begin{paracol}{2}\latim{
\rlettrine{H}{æc} nos communio, Dómine, purget a crimine: et, intercedénte beáto {\redx N.} Mártyre tuo atque Pontifice, cæléstis remédii fáciat esse consortes. Per Dóminum \emph{\&c.}
}\switchcolumn\portugues{
\qlettrine{Q}{ue} esta comunhão, Senhor, nos purifique de nossos crimes, e que, por intercessão do B. {\redx N.}, vosso Mártir e Pontífice, nos torne participantes do remédio celestial. Por nosso Senhor \emph{\&c.}
}\end{paracol}

\textit{Por um Mártir não Pontífice diz-se o seguinte:}

\paragraph{Postcomúnio}
\begin{paracol}{2}\latim{
\rlettrine{D}{a,} quǽsumus, Dómine, Deus noster: ut, sicut tuorum commemoratione Sanctórum temporali gratulámur officio; ita perpetuo lætémur aspéctu. Per Dóminum \emph{\&c.}
}\switchcolumn\portugues{
\rlettrine{C}{oncedei-nos,} Senhor, Vos suplicamos, que, assim como nos alegramos, celebrando na terra a memória dos vossos santos, assim também tenhamos a felicidade de os contemplar na eternidade. Por nosso Senhor \emph{\&c.}
}\end{paracol}

\textit{Outras vezes, ena vez do Precedente, diz-se o seguinte:}

\paragraph{Postcomúnio}
\begin{paracol}{2}\latim{
\rlettrine{R}{efécti} participatióne múneris sacri, quǽsumus, Dómine, Deus noster: ut, cujus exséquimur cultum; intercedénte beáto {\redx N.} Martyre tuo, sentiámus efféctum. Per Dóminum nostrum \emph{\&c.}
}\switchcolumn\portugues{
\rlettrine{C}{onfortados} com a participação do dom sagrado, Vos suplicamos, ó Senhor, nosso Deus, fazei-nos sentir, por intercessão do B. {\redx N.}, vosso Mártir, o efeito do mystério que celebramos. Por nosso Senhor \emph{\&c.}
}\end{paracol}

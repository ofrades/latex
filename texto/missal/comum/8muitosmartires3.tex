\subsectioninfo{Muitos Mártires}{Missa Salus autem}\label{muitosmartires3}

\paragraphinfo{Intróito}{Sl. 36, 39}
\begin{paracol}{2}\latim{
\rlettrine{S}{alus} autem justórum a Dómino: et protéctor eórum est in témpore tribulatiónis. \emph{Ps. ibid., 1} Noli æmulári in malignántibus: neque zeláveris faciéntes iniquitátem.
℣. Gloria Patri \emph{\&c.}
}\switchcolumn\portugues{
\rlettrine{A}{} salvação dos justos, porém, está no Senhor; Ele é o seu refúgio na ocasião da tribulação. \emph{Sl. ibid., 1} Não invejeis os maus, nem tenhais emulação daqueles que cometem iniquidades.
℣. Glória ao Pai \emph{\&c.}
}\end{paracol}

\paragraph{Oração}
\begin{paracol}{2}\latim{
\rlettrine{D}{eus,} qui nos ánnua sanctórum Mártyrum tuórum et {\redx N.} sollemnitáte lætíficas: concéde propítius; ut, quorum gaudémus méritis, accendámur exémplis. Per Dóminum \emph{\&c.}
}\switchcolumn\portugues{
\slettrine{Ó}{} Deus, que nos alegrais com a festividade anual dos vossos santos Mártires {\redx N.} e {\redx N.}, concedei-nos benigno que sejamos afervorados com os exemplos daqueles cujos méritos nos enchem de alegria. Por nosso Senhor \emph{\&c.}
}\end{paracol}

\paragraphinfo{Epístola}{Heb. 10, 32-38}
\begin{paracol}{2}\latim{
Léctio Epístolæ beáti Pauli Apóstoli ad Hebrǽos.
}\switchcolumn\portugues{
Lição da Ep.ª do B. Ap.º Paulo aos Hebreus.
}\switchcolumn*\latim{
\rlettrine{F}{ratres:} Rememorámini prístinos dies, in quibus illumináti magnum certámen sustinuístis passiónum: et in áltero quidem oppróbriis et tribulatiónibus spectáculum facti: in áltero autem sócii táliter conversántium effécti. Nam et vinctis compássi estis, et rapínam bonórum vestrórum cum gáudio suscepístis, cognoscéntes vos habere meliórem et manéntem substántiam. Nolíte itaque amíttere! confidéntiam vestram, quæ magnam habet remuneratiónem. Patiéntia enim vobis necéssaria est: ut, voluntátem Dei faciéntes, reportétis promissiónem. Adhuc enim módicum aliquántulum, qui ventúrus est, véniet, et non tardábit. Justus autem meus ex fide vivit.
}\switchcolumn\portugues{
\rlettrine{M}{eus} irmãos: Lembrai-vos dos primeiros dias em que, depois de haverdes recebido as luzes da fé, sofrestes grandes combates dolorosos, havendo sido, por um lado, expostos, diante de toda a gente, aos opróbrios e aos maus tratos, e, por outro lado, associados aos que sofreram iguais tormentos. Com efeito, vós compartilhastes as agruras dos prisioneiros e suportastes com alegria o despojo dos vossos bens, conhecendo que uma riqueza maior, que nunca seria arrebatada, vos estava reservada. Não percais a confiança que tendes, a qual deve alcançar uma recompensa de elevado valor. Com efeito, tendes necessidade de perseverança para que, depois de haverdes cumprido a vontade de Deus, alcanceis o prémio prometido. Esperai ainda algum tempo, e então chegará aquele que deve vir, o qual não tardará. Meu justo vive, pois, da sua fé.
}\end{paracol}

\paragraphinfo{Gradual}{Sl. 33, 18-19}
\begin{paracol}{2}\latim{
\rlettrine{C}{lamavérunt} justi, et Dóminus exaudívit eos: et ex ómnibus tribulatiónibus eórum liberávit eos. ℣. Juxta est Dóminus his, qui tribuláto sunt corde: et húmiles spíritu salvábit.
}\switchcolumn\portugues{
\rlettrine{C}{lamaram} os justos; então o senhor ouviu-os e livrou-os de todas suas aflições. ℣. O Senhor está próximo daqueles que têm o coração atribulado; e salvará os que têm o espírito humilhado.
}\switchcolumn*\latim{
Allelúja, allelúja. ℣. Te Mártyrum candidátus laudat exércitus, Dómine. Allelúja.
}\switchcolumn\portugues{
Aleluia, aleluia. ℣. O exército cândido dos mártires, ó Senhor, canta louvores em vossa honra! Aleluia.
}\end{paracol}

\textit{Após a Septuagésima omite-se o Aleluia e o seguinte e diz-se:}

\paragraphinfo{Trato}{Sl. 125, 5-6}
\begin{paracol}{2}\latim{
\qlettrine{Q}{ui} séminant in lácrimis, in gáudio metent. ℣. Eúntes ibant et flébant, mitténtes sémina sua. ℣. Veniéntes autem vénient cum exsultatióne, portántes manípulos suos.
}\switchcolumn\portugues{
\rlettrine{A}{queles} que semeiam com lágrimas ceifarão com júbilo. ℣. Iam, caminhavam e lançavam a semente à terra, chorando. ℣. Porém, quando voltavam, exultavam de alegria, trazendo os seus molhos de trigo.
}\end{paracol}

\paragraphinfo{Evangelho}{Lc. 12, 1-8}
\begin{paracol}{2}\latim{
\cruz Sequéntia sancti Evangélii secúndum Lucam.
}\switchcolumn\portugues{
\cruz Continuação do santo Evangelho segundo S. Lucas.
}\switchcolumn*\latim{
\blettrine{I}{n} illo témpore: Dixit Jesus discípulis suis: Atténdite a ferménto pharisæórum, quod est hypócrisis. Nihil autem opértum est, quod non revelétur: neque abscónditum, quod non sciátur. Quóniam, quæ in ténebris dixístis, in lúmine dicéntur: et quod in aurem locuti estis in cubículis, prædicábitur in tectis. Dico autem vobis amícis meis: Ne terreámini ab his, qui occídunt corpus, et post hæc non habent ámplius quid fáciant. Osténdam autem vobis, quem timeátis: timéte eum, qui, postquam occídent, habet potestátem míttere in gehénnam. Ita dico vobis: hunc timéte. Nonne quinque pásseres véneunt dipóndio, et unus ex illis non est in oblivióne coram Deo? Sed et capílli cápitis vestri omnes numerári sunt. Nolíte ergo timére: multis passéribus pluris estis vos. Dico autem vobis: Omnis, quicúmque conféssus fúerit me coram homínibus, et Fílius hóminis confiténtur illum coram Angelis Dei.
}\switchcolumn\portugues{
\blettrine{N}{aquele} tempo, disse Jesus aos seus discípulos: «Acautelai-vos com o fermento dos fariseus, que é a hipocrisia, pois nada há oculto que não chegue a ser descoberto, nem segredo que não venha a ser revelado. Até aquelas cousas que dissestes nas trevas serão publicadas à luz; e o que dissestes ao ouvido, no recôndito dos vossos cubículos, será apregoado sobre os telhados. Digo-vos, porém, a vós, que sois meus amigos: não tenhais receio daqueles que matam o corpo e depois não podem fazer mais nada. Sabeis a quem deveis temer? Temei aquele que, depois de haver dado a morte, tem ainda o poder de lançar no inferno. Sim; eu vo-lo digo: temei este. Porventura se não vendem cinco pássaros por dous ceitis? Contudo, nem um só deles fica em esquecimento diante de Deus! Até os cabelos da vossa cabeça estão contados! Não tenhais, pois, receio. Vós valeis mais do que muitos pássaros. Também vos digo: todo aquele que me confessar diante dos homens, o Filho do homem o reconhecerá igualmente diante dos Anjos de Deus».
}\end{paracol}

\paragraphinfo{Ofertório}{Sb. 3, 1, 2 \& 3}
\begin{paracol}{2}\latim{
\qlettrine{J}{ustórum} ánimæ in manu Dei sunt, et non tanget illos torméntum malítiae: visi sunt óculis insipiéntium mori: illi autem sunt in pace, allelúja.
}\switchcolumn\portugues{
\rlettrine{A}{s} almas dos justos estão nas mãos de Deus e o tormento da malícia as não tocará! Aos olhos dos insensatos pareciam quase a morrer, todavia estão na paz. Aleluia.
}\end{paracol}

\paragraph{Secreta}
\begin{paracol}{2}\latim{
\rlettrine{O}{blátis,} quǽsumus, Dómine, placáre munéribus: et, intercedéntibus sanctis Martýribus tuis {\redx N.} et {\redx N.}, a cunctis nos defénde perículis. Per Dóminum \emph{\&c.}
}\switchcolumn\portugues{
\rlettrine{D}{eixai-Vos} aplacar com os dons que Vos oferecemos, Senhor, e, por intercessão dos vossos santos Mártires {\redx N.} e {\redx N.}, preservai-nos de todos os perigos. Por nosso Senhor \emph{\&c.}
}\end{paracol}

\paragraphinfo{Comúnio}{Mt. 10, 27}
\begin{paracol}{2}\latim{
\qlettrine{Q}{uod} dico vobis in ténebris, dícite in lúmine, dicit Dóminus: et quod in aure audítis, prædicáte super tecta.
}\switchcolumn\portugues{
\rlettrine{O}{} que vos digo nas trevas dizei-o às claras, diz o Senhor; e o que vos disse ao ouvido pregai-o em cima dos telhados.
}\end{paracol}

\paragraph{Postcomúnio}
\begin{paracol}{2}\latim{
\rlettrine{H}{æc} nos commúnio, Dómine, purget a crímine: et, intercedéntibus sanctis Martýribus tuis {\redx N.} et {\redx N.}, cœléstis remédii fáciat esse consórtes. Per Dóminum \emph{\&c.}
}\switchcolumn\portugues{
\qlettrine{Q}{ue} esta comunhão nos purifique dos nossos crimes, Senhor, e que por intercessão dos vossos santos Mártires {\redx N.} e {\redx N.} nos faça participantes do remédio celestial. Por nosso Senhor \emph{\&c.}
}\end{paracol}

\textit{Outro Evangelho (para certos dias):}

\paragraphinfo{Evangelho}{Lc. 12, 1-8}\label{evangelho2muitosmartires3}
\begin{paracol}{2}\latim{
\cruz Sequéntia sancti Evangélii secúndum Matthǽum.
}\switchcolumn\portugues{
\cruz Continuação do santo Evangelho segundo S. Mateus.
}\switchcolumn*\latim{
\blettrine{I}{n} illo témpore: Sedénte Jesu super montem Olivéti, accessérunt ad eum discípuli secréto, dicéntes: Dic nobis, quando hæc erunt? et quod signum advéntus tui et consummatiónis sǽculi? Et respóndens Jesus, dixit eis: Vidéte, ne quis vos sedúcat. Multi enim vénient in nómine meo, dicéntes: Ego sum Christus: et multos sedúcent. Auditúri enim estis prǿlia et opiniónes prœliórum. Vidéte, ne turbémini. Opórtet enim hæc fíeri, sed nondum est finis. Consúrget enim gens in gentem, et regnum in regnum, et erunt pestiléntiæ et fames et terræmótus per loca. Hæc autem ómnia inítia sunt dolórum. Tunc tradent vos in tribulatiónem et occídent vos: et éritis ódio ómnibus géntibus propter nomen meum. Et tunc scandalizabúntur multi, et ínvicem tradent, et odio habébunt ínvicem. Et multi pseudoprophétæ surgent, et sedúcent multos. Et quóniam abundávit iníquitas, refrigéscet cáritas multórum. Qui autem perseveráverit usque in finem, hic salvus erit.
}\switchcolumn\portugues{
\blettrine{N}{aquele} tempo, sentando-se Jesus no Monte das Oliveiras, aproximaram-se d’Ele em particular os seus discípulos, perguntando-Lhe: «Dizei-nos quando acontecerão essas cousas? Que sinal haverá da vossa vinda e da consumação dos séculos?». Então, respondendo, Jesus disse-lhes: «Acautelai-vos, para que ninguém vos seduza; pois virão muitos, dizendo: sou Cristo, seduzindo também muitas pessoas; ouvireis falar de guerras e de rumores de guerras. Tende cuidado em vos não perturbardes, pois convém que tais cousas aconteçam; mas isso não será ainda o fim. Levantar-se-á povo contra povo e reino contra reino; haverá peste, fome e tremores de terra em diversos lugares; mas todas estas cousas serão apenas o começo das aflições. Entregar-vos-ão, para sofrerdes tormentos até à morte; sereis odiados por todos os povos por causa do meu nome; pelo que muitos se deixarão seduzir, se denunciarão e se atraiçoarão uns aos outros. Aparecerão muitos falsos profetas, que seduzirão muitas pessoas. Então, porque a iniquidade terá atingido o cúmulo, resfriará a caridade de muitos. Aquele que perseverar até ao fim será salvo».
}\end{paracol}

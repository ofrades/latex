\subsectioninfo{Mártires}{Missa Sancti tui}\label{martires}

\paragraphinfo{Intróito}{Sl. 144, 10-11}
\begin{paracol}{2}\latim{
\rlettrine{S}{ancti} tui, Dómine, benedícent te: glóriam regni tui dicent, allelúja, allelúja. \emph{Ps. ibid., 1} Exaltábo te, Deus meus, Rex: et benedícam nómini tuo in sǽculum, et in sǽculum sǽculi.
℣. Gloria Patri \emph{\&c.}
}\switchcolumn\portugues{
\qlettrine{Q}{ue} os vossos Santos Vos bendigam, Senhor: e publiquem a glória do vosso reino. Aleluia, aleluia. \emph{Sl. ibid., 1} Exaltarei a vossa glória, ó Deus, o meu Rei: e abençoarei o vosso Nome agora, sempre e em todos os séculos!
℣. Glória ao Pai \emph{\&c.}
}\end{paracol}

\paragraphinfo{Oração, Secreta e Postcomúnio}{Página \pageref{muitosmartires1}}

\paragraphinfo{Epístola}{1 Pe. 1, 3-7.}
\begin{paracol}{2}\latim{
Léctio Epístolæ beáti Petri.
}\switchcolumn\portugues{
Lição da Ep.ª do B. Ap.º Pedro.
}\switchcolumn*\latim{
\rlettrine{B}{enedíctus} Deus et Pater Dómini nostri Jesu Christi, qui secúndum misericórdiam suam magnam regenerávit nos in spem vivam, per resurrectiónem Jesu Christi ex mórtuis, in hereditátem incorruptíbilem et incontaminátam et immarcescíbilem, conservátam in cœlis in vobis, qui in virtúte Dei custodímini per fidem in salútem, parátam revelári in témpore novíssimo. In quo exsultábitis, módicum nunc si opórtet contristári in váriis tentatiónibus: ut probátio vestræ fídei multo pretiósior auro (quod per ignem probátur) inveniátur in laudem et glóriam et honórem, in revelatióne Jesu Christi, Dómini nostri.
}\switchcolumn\portugues{
\rlettrine{B}{endito} seja Deus, Pai de N. S. Jesus Cristo, que, segundo a grandeza da sua misericórdia, nos regenerou para uma esperança viva pela ressurreição dos mortos de Jesus Cristo, para alcançarmos a herança incorruptível, inalterável e imortal que está reservada nos céus para vós, a quem o poder de Deus guarda pela fé, para vos conceder o gozo da salvação, que será manifestada no fim dos tempos. Alegrai-vos com isto, ainda que devais ser perseguidos algumas vezes com diversas provações, a fim de que a manifestação da vossa fé, mais preciosa que o ouro (que é provado pelo fogo), seja julgada digna de louvor, honra e glória na revelação de N. S. Jesus Cristo.
}\end{paracol}

\begin{paracol}{2}\latim{
Allelúja, allelúja. ℣. Sancti tui, Dómine, florébunt sicut lílium: et sicut odor bálsami erunt ante te. Allelúja. ℣. \emph{Ps. 115, 15} Pretiósa in conspéctu Dómini mors Sanctórum ejus. Allelúja.
}\switchcolumn\portugues{
Aleluia, aleluia. ℣. Vossos santos, Senhor, florescerão, como o lírio, e serão, ante Vós, como o odor do bálsamo. Aleluia. ℣. \emph{Sl. 115, 15} É preciosa diante do Senhor a morte dos seus Santos. Aleluia.
}\end{paracol}

\paragraphinfo{Evangelho}{Jo. 15, 5-11}
\begin{paracol}{2}\latim{
\cruz Sequéntia sancti Evangélii secúndum Joánnem.
}\switchcolumn\portugues{
\cruz Continuação do santo Evangelho segundo S. João.
}\switchcolumn*\latim{
\blettrine{I}{n} illo témpore: Dixit Jesus discípulis suis: Ego sum vitis, vos pálmites: qui manet in me, et ego in eo, hic fert fructum multum: quia sine me nihil potéstis fácere. Si quis in me non mánserit, mittétur foras sicut palmes, et aréscet, et cólligent eum, et in ignem mittent, et ardet. Si manséritis in me, et verba mea in vobis mánserint: quodcúmque voluéritis, petétis, et fiet vobis. In hoc clarificátus est Pater meus, ut fructum plúrimum afferátis, et efficiámini mei discípuli. Sicut diléxit me Pater, et ego diléxi vos. Manéte in dilectióne mea. Si præcépta mea servavéritis, manébitis in dilectióne mea, sicut et ego Patris mei præcépta servávi, et máneo in ejus dilectióne. Hæc locútus sum vobis, ut gáudium meum in vobis sit, et gáudium vestrum impleátur.
}\switchcolumn\portugues{
\blettrine{N}{aquele} tempo, disse Jesus aos seus discípulos: «Eu sou a videira, e vós sois as vides. Aquele que permanece em mim, eu permaneço nele, e dará abundante fruto; pois sem mim nada podereis fazer. Se alguém não permanecer em mim, será arrancado e lançado fora, como uma vide seca. Então secará e levá-la-ão para a lançar no fogo, em que arderá. Se permanecerdes em mim e as minhas palavras permanecerem em vós, tudo o que quiserdes podereis pedir, que vos será concedido. Meu Pai será glorificado, se vós derdes muito fruto e vos tornardes meus discípulos. Assim como meu Pai me amou, assim também eu vos amo. Permanecei no meu amor. Se observardes os meus mandamentos, permanecereis no meu amor, como eu, que guardo os mandamentos de meu Pai, permaneço no seu amor. Digo-Vos estas cousas a fim de que minha alegria permaneça convosco e a vossa alegria seja abundante».
}\end{paracol}

\paragraphinfo{Ofertório}{Sl. 31, 11}
\begin{paracol}{2}\latim{
\rlettrine{L}{ætámini} in Dómino et exsultáte, justi: et gloriámini, omnes recti corde, allelúja, allelúja.
}\switchcolumn\portugues{
\rlettrine{A}{legrai-vos} no Senhor, ó justos! Exultai de júbilo! Todos aqueles que possuem o coração recto serão glorificados. Aleluia, aleluia.
}\end{paracol}

\paragraphinfo{Comúnio}{Sl. 32, 1}
\begin{paracol}{2}\latim{
\rlettrine{G}{audéte,} justi, in Dómino, allelúja: rectos decet collaudátio, allelúja.
}\switchcolumn\portugues{
\rlettrine{A}{legrai-vos} no Senhor, ó justos. Aleluia. É aos que são rectos que pertence cantar os vossos louvores. Aleluia.
}\end{paracol}

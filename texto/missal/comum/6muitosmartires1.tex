\subsectioninfo{Muitos Mártires}{Missa Intret in}\label{muitosmartires1}

\paragraphinfo{Intróito}{Sl. 78, 11, 12 \& 10}
\begin{paracol}{2}\latim{
\rlettrine{I}{ntret} in conspéctu tuo, Dómine, gémitus compeditórum: redde vicínis nostris séptuplum in sinu eórum: víndica sánguinem Sanctórum tuórum, qui effúsus est. \emph{Ps. ibid., 1} Deus, venérunt gentes in hereditátem tuam: polluérunt templum sanctum tuum: posuérunt Jerúsalem in pomórum custódiam.
℣. Gloria Patri \emph{\&c.}
}\switchcolumn\portugues{
\qlettrine{Q}{ue} os gemidos dos cativos cheguem à vossa presença, Senhor. Castigai os nossos inimigos sete vezes por cada injúria que nos têm feito: vingai o sangue que os vossos Santos derramaram. \emph{Ps. ibid., 1} Ó Deus, os povos invadiram a vossa herança, profanaram o vosso sagrado templo e reduziram Jerusalém a um monte de ruínas!
℣. Glória ao Pai \emph{\&c.}
}\end{paracol}

\paragraph{Oração}
\begin{paracol}{2}\latim{
\rlettrine{B}{eatórum} Mártyrum paritérque Pontíficum nos, quǽsumus, Dómine, festa tueántur: et eórum comméndet orátio veneránda. Per Dóminum \emph{\&c.}
}\switchcolumn\portugues{
\rlettrine{V}{os} suplicamos, Senhor, que a festa dos vossos B. B. Mártires e Pontífices {\redx N.} e {\redx N.}, nos proteja, e que sua veneranda oração nos sirva de recomendação junto de Vós. Por nosso Senhor \emph{\&c.}
}\end{paracol}

\textit{Não sendo Pontífice, diz-se a Oração da Missa seguinte, página \pageref{muitosmartires2}.}

\paragraphinfo{Epístola}{Sb. 3, 1-8}
\begin{paracol}{2}\latim{
Léctio libri Sapiéntiæ.
}\switchcolumn\portugues{
Lição do Livro da Sabedoria.
}\switchcolumn*\latim{
\qlettrine{J}{ustorum} ánimæ in manu Dei sunt, et non tanget illos torméntum mortis. Visi sunt oculis insipiéntium mori: et æstimála est afflíctio exitus illórum: et quod a nobis est iter, extermínium: illi autem sunt in pace. Et si coram homínibus
torménta passi sunt, spes illórum immortalitáte plena est. In paucis vexáti, in multis bene disponéntur: quóniam Deus tentávit eos, et invenit illos dignos se. Tamquam aurum in fornáce probávit illos, et quasi holocáusti hóstiam accépit illos, et in témpore erit respéctus illorum. Fulgébunt justi, et tamquam scintíllæ in arundinéto discúrrent. Judicábunt natiónes, et dominabúntur pópulis, et regnábit Dóminus illórum in perpétuum.
}\switchcolumn\portugues{
\rlettrine{A}{s} almas dos justos estão nas mãos de Deus; por isso o tormento da morte os não tocará. Pareciam mortos aos olhos dos insensatos: a sua saída do mundo parecia uma aflição; a sua separação de nós urna calamidade; mas, agora, estão em paz; e, ainda que tenham sofrido diante dos homens, a sua esperança está toda na imortalidade. Depois de haverem sofrido uma pena ligeira, receberam uma grande recompensa, pois Deus provou-os e achou-os dignos de si. Provou-os, como ao ouro, na fornalha; recebeu-os, como uma hóstia de holocausto; e para eles olhará benigno, quando vier o seu tempo. Os justos brilharão e resplandecerão, como as chamas, que se ateiam entre os canaviais. Eles julgarão as nações e dominarão os povos; e o Senhor reinará com eles para sempre.
}\end{paracol}

\paragraphinfo{Gradual}{Ex. 15,11}
\begin{paracol}{2}\latim{
\rlettrine{G}{loriósus} Deus in Sanctis suis: mirábilis in majestáte, fáciens prodígia. ℣. \emph{ibid., 6} Déxtera tua, Dómine, glorificáta est in virtúte: déxtera manus tua confrégit inimícos.
}\switchcolumn\portugues{
\rlettrine{D}{eus} é glorioso em seus Santos: e admirável na sua majestade, praticando prodígios. ℣. \emph{ibid., 6} Senhor, a vossa dextra engrandeceu-se pela sua força: a vossa dextra esmagou os inimigos.
}\switchcolumn*\latim{
Allelúja, allelúja. ℣. \emph{Eccli. 44, 14} Córpora Sanctórum in pace sepúlta sunt, et nómina eórum vivent in generatiónem et generatiónem. Allelúja.
}\switchcolumn\portugues{
Aleluia, aleluia. ℣. \emph{Ecl. 44, 14} Senhor, os corpos dos vossos Santos foram sepultados em paz e o seu nome subsistirá de geração em geração. Aleluia.
}\end{paracol}

\textit{Após a Septuagésima omite-se o Aleluia e o seguinte e diz-se:}

\paragraphinfo{Trato}{Sl. 125, 5-6}
\begin{paracol}{2}\latim{
\qlettrine{Q}{ui} séminant in lácrimis, in gáudio metent. ℣. Eúntes ibant et flébant, mitténtes sémina sua. ℣. Veniéntes autem vénient cum exsultatióne, portántes manípulos suos.
}\switchcolumn\portugues{
\rlettrine{A}{queles} que semeiam com lágrimas ceifarão com júbilo. ℣. Iam, caminhavam e lançavam a semente à terra, chorando. ℣. Porém, quando voltavam, exultavam de alegria, trazendo os seus molhos de trigo.
}\end{paracol}

\paragraphinfo{Evangelho}{Lc. 21, 9-19}
\begin{paracol}{2}\latim{
\cruz Sequéntia sancti Evangélii secúndum Lucam.
}\switchcolumn\portugues{
\cruz Continuação do santo Evangelho segundo S. Lucas.
}\switchcolumn*\latim{
\blettrine{I}{n} illo témpore: Dixit Jesus discípulis suis: Cum audieritis prǿlia et seditiónes, nolíte terréri: opórtet primum hæc fíeri, sed nondum statim finis. Tunc dicébat illis: Surget gens contra gentem, et regnum advérsus regnum. Et terræmótus magni erunt per loca, et pestiléntiæ, et fames, terrorésque de cœlo, et signa magna erunt. Sed ante hæc ómnia injícient vobis manus suas, et persequéntur tradéntes in synagógas et custódias, trahéntes ad reges et prǽsides propter nomen meum: contínget autem vobis in testimónium. Pónite ergo in córdibus vestris non præmeditári, quemádmodum respondeátis. Ego enim dabo vobis os et sapiéntiam, cui non potérunt resístere et contradícere omnes adversárii vestri. Tradémini autem a paréntibus, et frátribus, et cognátis, et amícis, et morte affícient ex vobis: et éritis ódio ómnibus propter nomen meum: et capíllus de cápite vestro non períbit. In patiéntia vestra possidébitis ánimas vestras.
}\switchcolumn\portugues{
\blettrine{N}{aquele} tempo, disse Jesus aos discípulos: «Quando ouvirdes falar em guerras e sedições, não vos assusteis; pois é necessário que estas coisas aconteçam, primeiramente; mas isto não será logo o fim». «Então dizia-lhes Ele levantar-se-á povo contra povo e reino contra reino; em diversos lugares haverá tremores de terra, peste, fome e também aparecerão coisas espantosas, grandes sinais no céu e outros prodígios. Mas, antes que tudo isto aconteça, lançar-vos-ão as mãos e perseguir-vos-ão, entregando-vos às sinagogas, lançando-vos nas prisões e conduzindo-vos à força diante dos reis e dos governadores, por causa do meu nome. Isto acontecerá para que deis testemunho da verdade. Gravai, pois, no vosso coração este pensamento: não premediteis de que modo haveis de responder, porque vos darei palavras e sabedoria, a que os vossos inimigos não poderão resistir, nem responder. Sereis entregues pelos vossos próprios pais, irmãos, parentes e amigos, que darão a morte a alguns de vós. Sereis aborrecidos de todos, por causa do meu nome; todavia, não se perderá nem um só cabelo das vossas cabeças. Com a vossa paciência possuireis as vossas almas».
}\end{paracol}

\paragraphinfo{Ofertório}{Sl. 67, 36}
\begin{paracol}{2}\latim{
\rlettrine{M}{irábilis} Deus in Sanctis suis: Deus Israël, ipse dabit virtútem et fortitúdinem plebi suæ: benedíctus Deus, allelúja.
}\switchcolumn\portugues{
\rlettrine{D}{eus} é admirável em seus Santos. Deus de Israel dará ao seu povo a força e a coragem. Bendito Ele seja, pois. Aleluia.
}\end{paracol}

\paragraph{Secreta}
\begin{paracol}{2}\latim{
\rlettrine{A}{désto,} Dómine, supplicatiónibus nostris, quas in Sanctórum tuórum commemoratióne deférimus: ut, qui nostræ justítiæ fidúciam non habémus, eórum, qui tibi placuérunt, méritis adjuvémur. Per Dóminum \emph{\&c.}
}\switchcolumn\portugues{
\rlettrine{A}{tendei,} Senhor, às súplicas que Vos dirigimos em memória dos vossos Santos, a fim de que nós, que não temos confiança na nossa própria justiça, sejamos auxiliados pelos méritos daqueles que Vos agradaram nesta vida. Por nosso Senhor \emph{\&c.}
}\end{paracol}

\paragraphinfo{Comúnio}{Sb. 3, 4, 5 \& 6}
\begin{paracol}{2}\latim{
\rlettrine{E}{t} si coram homínibus torménta passi sunt, Deus tentávit eos: tamquam aurum in fornáce probávit eos, et quasi holocáusta accépit eos.
}\switchcolumn\portugues{
\rlettrine{S}{e} sofreram tormentos diante dos homens, foi para Deus os provar. Deus provou-os na fornalha, como ao ouro, e recebeu-os, como hóstia de holocausto.
}\end{paracol}

\paragraph{Postcomúnio}
\begin{paracol}{2}\latim{
\qlettrine{Q}{uǽsumus,} Dómine, salutáribus repléti mystériis: ut, quorum sollémnia celebrámus, eórum oratiónibus adjuvémur. Per Dóminum \emph{\&c.}
}\switchcolumn\portugues{
\rlettrine{F}{ortificados} com vossos salutares mistérios, dignai-Vos conceder-nos, Senhor, a graça da assistência das orações daqueles cuja festa celebrámos. Por nosso Senhor \emph{\&c.}
}\end{paracol}

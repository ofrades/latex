\subsectioninfo{Virgem não Mártir}{Missa Dilexísti justitiam}\label{virgemnaomartir1}

\paragraphinfo{Intróito}{Sl. 44, 8}
\begin{paracol}{2}\latim{
\rlettrine{D}{ilexísti} justítiam, et odísti iniquitátem: proptérea unxit te Deus, Deus tuus, óleo lætítiae præ consórtibus tuis. (T. P. Allelúja, allelúja.) \emph{Ps. ibid., 2} Eructávit cor meum verbum bonum: dico ego ópera mea Regi.
℣. Gloria Patri \emph{\&c.}
}\switchcolumn\portugues{
\rlettrine{A}{mastes} a justiça e odiastes a iniquidade; pelo que ungiu-vos o Senhor, vosso Deus, com o óleo da alegria, de preferência às vossas companheiras. (T. P. Aleluia, aleluia.) \emph{Sl. ibid., 2} Meu coração exprimiu uma excelente palavra: Consagro ao Rei as minhas obras.
℣. Glória ao Pai \emph{\&c.}
}\end{paracol}

\paragraph{Oração}
\begin{paracol}{2}\latim{
\rlettrine{E}{xáudi} nos, Deus, salutáris noster: ut, sicut de beátæ {\redx N.} Vírginis tuæ festivitáte gaudémus; ita piæ devotiónis erudiámur afféctu. Per Dóminum nostrum \emph{\&c.}
}\switchcolumn\portugues{
\rlettrine{O}{uvi-nos,} ó Deus, nosso salvador, a fim de que, assim como nos alegramos com a festa da vossa B. Virgem {\redx N.} assim também consigamos alcançar piedosos sentimentos de fervorosa devoção. Por nosso Senhor \emph{\&c.}
}\end{paracol}

\paragraphinfo{Epístola}{2. Cor. 10, 17-18; 11, 1-2}
\begin{paracol}{2}\latim{
Léctio Epístolæ beáti Pauli Apóstoli ad Corínthios.
}\switchcolumn\portugues{
Lição da Ep.ª do B. Ap.º Paulo aos Coríntios.
}\switchcolumn*\latim{
\rlettrine{F}{ratres:} Qui gloriátur, in Dómino gloriétur. Non enim, qui seípsum comméndat, ille probátus est; sed quem Deus comméndat. Utinam sustinerétis módicum quid insipiéntiæ meæ, sed et supportáte me: ǽmulor enim vos Dei æmulatióne. Despóndi enim vos uni viro vírginem castam exhibére Christo.
}\switchcolumn\portugues{
\rlettrine{M}{eus} irmãos: Aquele que se gloria, glorie-se no Senhor; pois não é aprovado o que se louva a si mesmo, mas aquele a quem Deus recomenda. Queira Deus que possais suportar um pouco ainda a minha loucura; mas suportai-me ainda, pois estou zeloso de vós da parte do zelo de Deus. Com efeito, desposei-vos com o único esposo, para vos apresentar a Cristo como virgem pura.
}\end{paracol}

\paragraphinfo{Gradual}{Sl. 44, 5}
\begin{paracol}{2}\latim{
\rlettrine{S}{pécie} tua et pulchritúdine tua inténde, próspere procéde et regna. ℣. Propter veritátem et mansuetúdinem et justítiam: et dedúcet te mirabíliter déxtera tua.
}\switchcolumn\portugues{
\rlettrine{C}{aminhai} com beleza e com majestade; ide gozar a vitória e reinai. Por causa da verdade, da mansidão e da justiça, a vossa dextra praticará maravilhas!
}\switchcolumn*\latim{
Allelúja, allelúja. ℣. \emph{ibid., 15 \& 16} Adducántur Regi Vírgines post eam: próximæ ejus afferéntur tibi in lætítia. Allelúja.
}\switchcolumn\portugues{
Aleluia, aleluia. ℣. \emph{ibid., 15 \& 16} Após ela, serão apresentadas virgens ao Rei: as suas companheiras serão introduzidas no meio da alegria. Aleluia.
}\end{paracol}

\textit{Após a Septuagésima omite-se o Aleluia e o seguinte e diz-se:}

\paragraphinfo{Trato}{Sl. 44, 11 \& 12}
\begin{paracol}{2}\latim{
\rlettrine{A}{udi,} fília, et vide, et inclína aurem tuam: quia concupívit Rex spéciem tuam. ℣. \emph{ibid., 13 et 10} Vultum tuum deprecabúntur omnes dívites plebis: fíliæ regum in honóre tuo. ℣. \emph{ibid., 15-16} Adducéntur Regi Vírgines post eam: próximæ ejus afferéntur tibi. ℣. Afferéntur in lætítia et exsultatióne: adducántur in templum Regis.
}\switchcolumn\portugues{
\slettrine{Ó}{} minha filha, ouvi, vede e prestai atenção; pois o Rei está cheio de amor por vós, por causa da vossa beleza. ℣. \emph{ibid., 13 et 10} Todos os poderosos da terra implorarão os vossos olhares: e as filhas dos reis formam a vossa corte de glória. ℣. \emph{ibid., 15-16} Depois de vós, virão coros de virgens: as suas companheiras serão apresentadas ao Rei. ℣. Serão apresentadas no meio da alegria e do júbilo: e serão introduzidas no templo do Rei.
}\end{paracol}

\textit{No T. Pascal omite-se o Gradual e o Trato e diz-se:}

\begin{paracol}{2}\latim{
Allelúia, allelúja. ℣. \emph{Ps. 44, 15 et 16} Adducéntur Regi Vírgines post eam: próximæ ejus afferéntur tibi in lætítia. Allelúja. ℣. \emph{ibid., 5} Spécie tua et pulchritúdine tua inténde, próspere procéde et regna. Allelúja.
}\switchcolumn\portugues{
Aleluia, aleluia. ℣. \emph{Sl. 44, 15 et 16} Após ela, serão apresentadas virgens ao Rei: as suas companheiras serão introduzidas no meio da alegria. Aleluia. ℣. \emph{ibid., 5} Caminhai com beleza e com majestade; ide gozar a vitória e reinai. Aleluia.
}\end{paracol}

\paragraphinfo{Evangelho}{Mt. 25, 1-13}
\begin{paracol}{2}\latim{
\cruz Sequéntia sancti Evangélii secúndum Matthǽum.
}\switchcolumn\portugues{
\cruz Continuação do santo Evangelho segundo S. Mateus.
}\switchcolumn*\latim{
\blettrine{I}{n} illo témpore: Dixit Jesus discípulis suis parábolam hanc: Simile erit regnum cœlórum decem virgínibus: quæ, accipiéntes lámpades suas, exiérunt óbviam sponso et sponsæ. Quinque autem ex eis erant fátuæ, et quinque prudéntes: sed quinque fátuæ, accéptis lampádibus, non sumpsérunt óleum secum: prudéntes vero accepérunt óleum in vasis suis cum lampádibus. Horam autem faciénte sponso, dormitavérunt omnes et dormiérunt. Média autem nocte clamor factus est: Ecce, sponsus venit, exíte óbviam ei. Tunc surrexérunt omnes vírgines illæ, et ornavérunt lámpades suas. Fátuæ autem sapiéntibus dixérunt: Date nobis de óleo vestro: quia lámpades nostræ exstinguúntur. Respondérunt prudéntes, dicéntes: Ne forte non suffíciat nobis et vobis, ite pótius ad vendéntes, et émite vobis. Dum autem irent émere, venit sponsus: et quæ parátæ erant, intravérunt cum eo ad núptias, et clausa est jánua. Novíssime vero véniunt et réliquæ vírgines, dicéntes: Dómine, Dómine, aperi nobis. At ille respóndens, ait: Amen, dico vobis, néscio vos. Vigiláte ítaque, quia nescítis diem neque horam.
}\switchcolumn\portugues{
\blettrine{N}{aquele} tempo, disse Jesus aos seus discípulos esta parábola: «O reino dos céus é semelhante a dez virgens que, empunhando suas lâmpadas, saíram ao encontro do esposo e da esposa. Porém, cinco destas virgens eram loucas e as outras cinco eram prudentes. Ora, as cinco loucas, empunhando as suas lâmpadas, não levaram azeite. Ao contrário, as prudentes tomaram azeite em seus vasos para suas lâmpadas. Como o esposo se demorasse em chegar, tiveram sono e dormiram. Quando era meia-noite, ouviu-se um clamor dizer: «Eis que chega o esposo; ide ao seu encontro». Então todas estas virgens se ergueram e prepararam as suas lâmpadas. As loucas disseram às prudentes: «Dai-nos do vosso azeite, porque as nossas lâmpadas apagam-se». As prudentes responderam-lhes: «Não, porque pode suceder que, como a vós, nos falte o azeite; ide antes aos que o vendem e comprai-o». Ora, enquanto elas foram comprar o azeite, veio o esposo. Então, as que estavam preparadas entraram com ele para as bodas; e fechou-se a porta. Por fim vieram as outras virgens e disseram: «Senhor, senhor, abri-nos a porta». Ele respondeu: «Na verdade vos digo: não vos conheço. Vigiai, pois, visto que não sabeis nem o dia nem a hora».
}\end{paracol}

\paragraphinfo{Ofertório}{Sl. 44, 10}
\begin{paracol}{2}\latim{
\rlettrine{F}{íliæ} regum in honóre tuo, ástitit regína a dextris tuis in vestítu deauráto, circúmdata varietate. (T. P. Allelúja.)
}\switchcolumn\portugues{
\rlettrine{A}{s} filhas dos reis formam a vossa corte de glória: a própria rainha está colocada à vossa direita, envergando um vestido de ouro, recamado da mais rica variedade. (T. P. Aleluia.)
}\end{paracol}

\paragraph{Secreta}
\begin{paracol}{2}\latim{
\rlettrine{A}{ccépta} tibi sit, Dómine, sacrátæ plebis oblátio pro tuórum honóre Sanctórum: quorum se méritis de tribulatióne percepísse cognóscit auxílium. Per Dóminum nostrum \emph{\&c.}
}\switchcolumn\portugues{
\rlettrine{A}{ceitai,} Senhor, esta oferta, que Vos consagra o vosso povo fiel em honra dos vossos santos, pelos méritos dos quais reconhece que tem alcançado a vossa assistência nas tribulações. Por nosso Senhor \emph{\&c.}
}\end{paracol}

\paragraphinfo{Comúnio}{Mt. 25, 4 \& 6}
\begin{paracol}{2}\latim{
\qlettrine{Q}{uinque} prudéntes vírgines accepérunt óleum in vasis suis cum lampádibus: média autem nocte clamor factus est: Ecce, sponsus venit: exite óbviam Christo Dómino. (T. P. Allelúja.)
}\switchcolumn\portugues{
\rlettrine{A}{s} cinco virgens prudentes levaram azeite em seus vasos para suas lâmpadas. Ora, no meio da noite, ouviu-se este clamor: «Eis que chega o esposo; comparecei diante de Cristo, vosso Senhor». (T. P Aleluia.)
}\end{paracol}

\paragraph{Postcomúnio}
\begin{paracol}{2}\latim{
\rlettrine{S}{atiásti,} Dómine, famíliam tuam munéribus sacris: ejus, quǽsumus, semper interventióne nos réfove, cujus sollémnia celebrámus. Per Dóminum \emph{\&c.}
}\switchcolumn\portugues{
\rlettrine{H}{avendo} Vós, Senhor, saciado a vossa família com vossos dons sagrados, dignai-Vos favorecer-nos sempre com a intercessão daquela cuja festa celebrámos. Por nosso Senhor \emph{\&c.}
}\end{paracol}

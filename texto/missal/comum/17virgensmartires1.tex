\subsectioninfo{Virgens Mártires}{Missa Loquébar}\label{virgensmartires1}

\paragraphinfo{Intróito}{Sl. 118, 46-47}
\begin{paracol}{2}\latim{
\rlettrine{L}{oquébar} de testimóniis tuis in conspéctu regum, et non confundébar: et meditábar in mandátis tuis, quæ diléxi nimis. (T. P. Allelúja, allelúja.) \emph{Ps. ibid., 1} Beáti immaculáti in via: qui ámbulant in lege Dómini.
℣. Gloria Patri \emph{\&c.}
}\switchcolumn\portugues{
\rlettrine{F}{alava} dos vossos testemunhos na presença dos reis, sem qualquer vergonha, e meditava nos vossos mandamentos, que amava profundamente. (T. P. Aleluia, aleluia.) \emph{Sl. ibid., 1} Bem-aventurados aqueles que são imaculados nos seus caminhos e cumprem a Lei do Senhor.
℣. Glória ao Pai \emph{\&c.}
}\end{paracol}

\paragraph{Oração}
\begin{paracol}{2}\latim{
\rlettrine{D}{eus,} qui inter cétera poténtiæ tuæ mirácula étiam in sexu frágili victóriam martýrii contulísti: concéde propítius; ut, qui beátæ {\redx N.} Vírginis et Mártyris tuæ natalítia cólimus, per ejus ad te exémpla gradiámur. Per Dóminum \emph{\&c.}
}\switchcolumn\portugues{
\slettrine{Ó}{} Deus, que entre outros milagres do vosso poder permitistes que o sexo frágil alcançasse a vitória do martírio, concedei-nos propício que, venerando o nascimento da vossa B. Virgem e Mártir {\redx N.}, caminhemos para Vós, imitando os seus exemplos. Por nosso Senhor \emph{\&c.}
}\end{paracol}

\paragraphinfo{Epístola}{Ecl. 51, 1-8 et 12}
\begin{paracol}{2}\latim{
Léctio libri Sapiéntiæ.
}\switchcolumn\portugues{
Lição do Livro da Sabedoria.
}\switchcolumn*\latim{
\rlettrine{C}{onfitébor} tibi, Dómine, Rex, et collaudábo te Deum, Salvatórem meum. Confitébor nómini tuo: quóniam adjútor et protéctor factus es mihi, et liberásti corpus meum a perditióne, a láqueo línguæ iníquæ et a lábiis operántium mendácium, et in conspéctu astántium factus es mihi adjutor. Et liberasti me secúndum multitúdinem misericórdiæ nóminis tui a rugiéntibus, præparátis ad escam, de mánibus quæréntium ánimam meam, et de portis tribulatiónum, quæ circumdedérunt me: a pressúra flammæ, quæ circúmdedit me, et in médio ignis non sum æstuáta: de altitúdine ventris inferi, et a lingua coinquináta, et a verbo mendácii, a rege iníquo, et a lingua injústa: laudábit usque ad mortem ánima mea Dóminum: quóniam éruis sustinéntes te, et líberas eos de mánibus géntium, Dómine, Deus noster.
}\switchcolumn\portugues{
\qlettrine{Q}{uero} glorificar-Vos, Senhor e Rei; quero louvar-Vos, ó Deus, meu salvador. Quero glorificar o vosso nome, porque fostes o meu sustentáculo e protector e livrastes o meu corpo da perdição, do laço da língua iníqua e dos lábios daqueles que tramam a mentira. Na presença dos meus adversários fostes o meu auxílio. Livrastes-me, segundo a grandeza da misericórdia do vosso nome, dos que rugiam prestes a devorar-me; livrastes-me das mãos dos que procuravam tirar-me a vida; livrastes-me das aflições, que me cercavam; livrastes-me da violência das chamas, que me rodeavam, no meio das quais não senti o calor do fogo; livrastes-me do abysmo profundo do inferno; da língua impura; das palavras mentirosas; do rei iníquo; e da língua injusta! Minha alma louvará o Senhor até à morte, porque Vós, Senhor, nosso Deus, livrais dos perigos aqueles que confiam em Vós, salvando-os do poder dos inimigos.
}\end{paracol}

\paragraphinfo{Gradual}{Sl. 44, 8}
\begin{paracol}{2}\latim{
\rlettrine{D}{ilexísti} justítiam, et odísti iniquitátem. ℣. Proptérea unxit te Deus, Deus tuus, óleo lætítiæ.
}\switchcolumn\portugues{
\rlettrine{A}{mastes} a justiça e odiastes a iniquidade. ℣. Por isso ungiu-vos o Senhor, vosso Deus, com o óleo da alegria.
}\switchcolumn*\latim{
Allelúja, allelúja. ℣. \emph{ibid., 15 \& 16} Adducántur Regi Vírgines post eam: próximæ ejus afferéntur tibi in lætítia. Allelúja.
}\switchcolumn\portugues{
Aleluia, aleluia. ℣. \emph{ibid., 15 \& 16} Serão apresentadas virgens ao Rei após ela: as suas companheiras serão introduzidas no meio da alegria. Aleluia.
}\end{paracol}

\textit{Após a Septuagésima omite-se o Aleluia e o seguinte e diz-se:}

\paragraph{Trato}
\begin{paracol}{2}\latim{
\rlettrine{V}{eni,} Sponsa Christi, áccipe corónam, quam tibi Dóminus præparávit in ætérnum: pro cujus amóre sánguinem tuum fudísti. ℣. \emph{Ps. 44, 8 \& 5} Dilexísti justítiam, et odísti iniquitátem: proptérea unxit te Deus, Deus tuus, óleo lætítiæ præ consórtibus tuis. ℣. Spécie tua et pulchritúdine tua inténde, próspere procéde et regna.
}\switchcolumn\portugues{
\rlettrine{V}{inde,} ó esposa de Cristo; vinde e recebei a coroa que o Senhor preparou para vós, para a eternidade. Foi por amor para com Ele que derramastes o vosso sangue. ℣. \emph{Sl. 44, 8 \& 5} Amastes a justiça e odiastes a iniquidade: eis porque o Senhor, vosso Deus, vos ungiu com o óleo da alegria, de preferência às vossas companheiras. ℣. Caminhai, pois, com beleza e com majestade; ide gozar a vitória e reinai.
}\end{paracol}

\textit{No T. Pascal omite-se o Gradual e o Trato e diz-se:}

\begin{paracol}{2}\latim{
Allelúja, allelúja. ℣. \emph{Ps. 44, 15 \& 16} Adducántur Regi Vírgines post eam: próximæ ejus afferéntur tibi in lætítia. Allelúja. ℣. \emph{ibid., 5} Spécie tua et pulchritúdine tua inténde, próspere procéde et regna. Allelúja.
}\switchcolumn\portugues{
Aleluia, aleluia. ℣. \emph{Sl. 44, 15 \& 16} Após ela, serão apresentadas virgens ao Rei: as suas companheiras serão introduzidas no meio da alegria. Aleluia. ℣. \emph{ibid., 5} Caminhai, pois, com beleza e com majestade; ide gozar a vitória e reinai. Aleluia.
}\end{paracol}

\paragraphinfo{Evangelho}{Mt. 25, 1-13}
\begin{paracol}{2}\latim{
\cruz Sequéntia sancti Evangélii secúndum Matthǽum.
}\switchcolumn\portugues{
\cruz Continuação do santo Evangelho segundo S. Mateus.
}\switchcolumn*\latim{
\blettrine{I}{n} illo témpore: Dixit Jesus discípulis suis parábolam hanc: Simile erit regnum cœlórum decem virgínibus: quæ, accipiéntes lámpades suas, exiérunt óbviam sponso et sponsæ. Quinque autem ex eis erant fátuæ, et quinque prudéntes: sed quinque fátuæ, accéptis lampádibus, non sumpsérunt óleum secum: prudéntes vero accepérunt óleum in vasis suis cum lampádibus. Horam autem faciénte sponso, dormitavérunt omnes et dormiérunt. Média autem nocte clamor factus est: Ecce, sponsus venit, exíte óbviam ei. Tunc surrexérunt omnes vírgines illae, et ornavérunt lámpades suas. Fátuæ autem sapiéntibus dixérunt: Date nobis de óleo vestro: quia lámpades nostræ exstinguúntur. Respondérunt prudéntes, dicéntes: Ne forte non suffíciat nobis et vobis, ite pótius ad vendéntes, et émite vobis. Dum autem irent émere, venit sponsus: et quæ parátæ erant, intravérunt cum eo ad núptias, et clausa est jánua. Novíssime vero véniunt et réliquæ vírgines, dicéntes: Dómine, Dómine, áperi nobis. At ille respóndens, ait: Amen, dico vobis, néscio vos. Vigiláte ítaque, quia nescítis diem neque horam.
}\switchcolumn\portugues{
\blettrine{N}{aquele} tempo, disse Jesus aos seus discípulos esta parábola: «O reino dos céus é semelhante a dez virgens que, empunhando suas lâmpadas, saíram ao encontro do esposo e da esposa. Porém, cinco destas virgens eram loucas e as outras cinco eram prudentes. Ora, as cinco loucas, empunhando suas lâmpadas, não levaram azeite. Ao contrário, as prudentes tomaram azeite em seus vasos para suas lâmpadas. Como o esposo se demorasse em chegar, tiveram sono e dormiram. Quando era meia-noite, ouviu-se um clamor dizer: «Eis que chega o esposo; ide ao seu encontro». Então todas estas virgens se ergueram e prepararam as suas lâmpadas. As loucas disseram às prudentes: «Dai-nos do vosso azeite, porque as nossas lâmpadas apagam-se». As prudentes responderam-lhes: «Não, porque pode suceder que, como a vós, nos falte o azeite; ide antes aos que o vendem, e comprai-o». Ora, enquanto elas foram comprar o azeite, veio o esposo. Então, as que estavam preparadas entraram com ele para as bodas; e fechou-se a porta. Por fim vieram as outras virgens, e disseram: «Senhor, senhor, abri-nos a porta». Ele respondeu: «Na verdade vos digo: não vos conheço. Vigiai, pois, visto que não sabeis nem o dia nem a hora».
}\end{paracol}

\paragraphinfo{Ofertório}{Sl. 44, 15 \& 16}
\begin{paracol}{2}\latim{
\rlettrine{A}{fferéntur} Regi Vírgines post eam: próximæ ejus afferéntur tibi in lætítia et exsultatióne: adducántur in templum Regi Dómino. (T. P. Allelúja.)
}\switchcolumn\portugues{
\rlettrine{A}{pós} ela, serão apresentadas virgens ao Rei: as suas companheiras serão introduzidas no meio da alegria e do júbilo; elas serão conduzidas ao templo do Rei, seu Senhor. (T. P. Aleluia.)
}\end{paracol}

\paragraph{Secreta}
\begin{paracol}{2}\latim{
\rlettrine{S}{úscipe,} Dómine, múnera, quæ in beátæ {\redx N.} Vírginis et Mártyris tuæ sollemnitáte deférimus: cujus nos confídimus patrocínio liberári. Per Dóminum \emph{\&c.}
}\switchcolumn\portugues{
\rlettrine{R}{ecebei} benigno, Senhor, as ofertas que Vos apresentamos nesta solenidade da B. Virgem {\redx N.}, vossa Mártir, com o patrocínio da qual esperamos ser livres. Por nosso Senhor \emph{\&c.}
}\end{paracol}

\paragraphinfo{Comúnio}{Sl. 118, 78 \& 80}
\begin{paracol}{2}\latim{
\rlettrine{C}{onfundántur} supérbi, quia injúste iniquitátem fecérunt in me: ego autem in mandátis tuis exercébor, in tuis justificatiónibus, ut non confúndar. (T. P. Allelúja.)
}\switchcolumn\portugues{
\rlettrine{S}{ejam} confundidos os soberbos, porque praticaram iniquidades contra mim. Para não ser confundido no último dia, cumprirei os vossos mandamentos e preceitos. (T. P. Aleluia.)
}\end{paracol}

\paragraph{Postcomúnio}
\begin{paracol}{2}\latim{
\rlettrine{A}{uxiliéntur} nobis, Dómine, sumpta mystéria: et, intercedénte beáta {\redx N.} Vírgine et Mártyre tua, sempitérna fáciant protectióne gaudére. Per Dóminum \emph{\&c.}
}\switchcolumn\portugues{
\qlettrine{Q}{ue} nos auxiliem os sagrados mystérios, que acabámos de receber, Senhor, e que, por intercessão da B. Virgem {\redx N.}, vossa Mártir, nos façam gozar continuamente a sua protecção. Por nosso Senhor \emph{\&c.}
}\end{paracol}

\subsectioninfo{Sexta-feira das Têmporas de Pentecostes}{Estação na Igreja dos Doze Apóstolos}\label{sextafeirapentecostes}

\paragraphinfo{Intróito}{Sl. 70, 8 \& 23}
\begin{paracol}{2}\latim{
\rlettrine{R}{epleátur} os meum laude tua, allelúja: ut possim cantáre, allelúja: gaudébunt lábia mea, dum cantávero tibi, allelúja, allelúja. \emph{Ps. ibid., 1-2} In te, Dómine, sperávi, non confúndar in ætérnum: in justítia tua líbera me et éripe me.
℣. Gloria Patri \emph{\&c.}
}\switchcolumn\portugues{
\qlettrine{Q}{ue} a minha boca se encha com vossos louvores, aleluia: para que eu possa cantar, aleluia. E alegrar-se-ão os meus lábios quando cantarem vossos hinos, aleluia, aleluia. \emph{Sl. ibid., 1-2} Senhor, confio em Vós, não serei confundido para sempre; pois a vossa justiça me livrará e salvará.
℣. Glória ao Pai \emph{\&c.}
}\end{paracol}

\paragraph{Oração}
\begin{paracol}{2}\latim{
\rlettrine{D}{a,} quǽsumus, Ecclésiæ tuæ, miséricors Deus: ut, Sancto Spíritu congregáta, hostíli nullaténus incursióne turbétur. Per Dóminum \emph{\&c.}
}\switchcolumn\portugues{
\slettrine{Ó}{} Deus misericordioso, Vos suplicamos, concedei à vossa Igreja que, havendo sido instituída pelo Espírito Santo, não seja de modo algum perturbada pelos ataques dos seus inimigos. Por nosso Senhor \emph{\&c.}
}\end{paracol}

\paragraphinfo{Epístola}{Jl. 2, 23-24 \& 26-27}
\begin{paracol}{2}\latim{
Léctio Joélis Prophétæ.
}\switchcolumn\portugues{
Lição do Profeta Joel.
}\switchcolumn*\latim{
\rlettrine{H}{æc} dicit Dóminus Deus: Exsultáte, fílii Sion, et lætámini in Dómino, Deo vestro: quia dedit vobis doctórem justítiæ, et descéndere fáciet ad vos imbrem matutínum et serótinum, sicut in princípio. Et implebúntur áreæ fruménto et redundábunt torculária vino et óleo Et comedétis vescéntes et saturabímini, et laudábilis nomen Dómini, Dei vestri, qu fecit mirabília vobíscum: et non confundátur pópulus me us in sempitérnum. Et sciétis, quia in médio Israël ego sum: et ego Dóminus, Deus vester, et non est ámplius: et non confundétur pópulus me us in ætérnum: ait Dóminus omnípotens.
}\switchcolumn\portugues{
\rlettrine{I}{sto} diz o Senhor Deus: «Alegrai-vos, ó filhos de Sião, rejubilai no Senhor, vosso Deus, pois deu-vos um Mestre, que vos ensinará a justiça e fará cair sobre vós, como outrora, as chuvas do outono e da primavera. Vossas eiras encher-se-ão de trigo e os lagares de vinho e de azeite. Comereis com abundância e sereis fartos. E louvareis o nome do Senhor, vosso Deus, que tantas maravilhas praticou para vosso benefício. Meu povo não mais será confundido. Então conhecereis que sou Eu que estou no meio de Israel; que sou o Senhor, vosso Deus; e que outro Deus não existe. Meu povo não será mais confundido»: isto diz o Senhor omnipotente.
}\end{paracol}

\begin{paracol}{2}\latim{
Allelúja, allelúja. ℣. \emph{Sap 12, 1} O quam bonus et suávis est, Dómine, Spíritus tuus in nobis! Allelúja. \emph{(Hic genuflectitur)} ℣. Veni, Sancte Spíritus, reple tuórum corda fidélium: et tui amóris in eis ignem accénde.
}\switchcolumn\portugues{
Aleluia, aleluia. ℣. \emph{Sb. 12, 1} Senhor, como é bom e suave o vosso Espírito! Aleluia \emph{(Genuflecte-se)} ℣. Vinde ó Espírito Santo; enchei os corações dos vossos fiéis e acendei neles o fogo do vosso amor.
}\end{paracol}

\paragraphinfo{Evangelho}{Lc. 5, 17-26}
\begin{paracol}{2}\latim{
\cruz Sequéntia sancti Evangélii secúndum Lucam.
}\switchcolumn\portugues{
\cruz Continuação do santo Evangelho segundo S. Lucas.
}\switchcolumn*\latim{
\blettrine{I}{n} illo témpore: Factum est in una diérum, et Jesus sedébat docens. Et erant pharisǽi sedéntes, et legis doctóres, qui vénerant ex omni castéllo Galilǽæ et Judǽæ et Jerúsalem: et virtus Dómini erat ad sanándum eos. Et ecce, viri portántes in lecto hóminem, qui erat paralýticus: et quærébant eum inférre, et pónere ante eum. Et non inveniéntes, qua parte illum inférrent præ turba, ascendérunt supra tectum, et per tégulas summisérunt eum cum lecto in médium ante Jesum. Quorum fidem ut vidit, dixit: Homo, remittúntur tibi peccáta tua. Et cœpérunt cogitáre scribæ et pharisǽi, dicéntes: Quis est hic, qui lóquitur blasphémias? Quis potest dimíttere peccáta nisi solus Deus? Ut cognóvit autem Jesus cogitatiónes eórum, respóndens dixit ad illos: Quid cogitátis in córdibus vestris? Quid est facílius dícere: Dimittúntur tibi peccáta, an dícere: Surge et ámbula? Ut autem sciátis, quia Fílius hóminis habet potestátem in terra dimitténdi peccáta (ait paralýtico): Tibi dico, surge, tolle lectum tuum et vade in domum tuam. Et conféstim consúrgens coram illis, tulit lectum, in quo jacébat: et ábiit in domum suam, magníficans Deum. Et stupor apprehéndit omnes et magnificábant Deum. Et repléti sunt timóre, dicéntes: Quia vídimus mirabília hódie.
}\switchcolumn\portugues{
\blettrine{N}{aquele} tempo, aconteceu que Jesus ensinava. Estavam ao pé dele, assentados também, alguns fariseus e doutores da lei, que tinham vindo das aldeias da Galileia, da Judeia e de Jerusalém. E o poder do Senhor manifestava-se ali pelas suas curas. Então, eis que uns homens trouxeram num leito um paralítico, diligenciando levá-lo presença de Jesus. Não encontrando lugar, pois a multidão era grande, subiram ao telhado e através das telhas desceram o paralítico no leito, no meio de todos e na presença de Jesus. Vendo Jesus a fé destes homens, disse: «Homem, os teus pecados são-te perdoados». Então os escribas e os fariseus começaram a raciocinar nestas palavras e disseram: «Quem é este que blasfema desta maneira? Quem pode perdoar pecados senão só Deus?». Jesus, conhecendo os pensamentos deles, tomou a palavra e disse: «Que pensais no íntimo dos vossos corações? Qual é mais fácil dizer: são-te perdoados os teus pecados, ou levanta-te e caminha? Pois para que conheçais que o Filho do homem tem poder na terra para perdoar pecados: Eu te ordeno, (disse ao paralítico) levanta-te, leva o teu leito e vai para tua casa!». E naquele mesmo instante levantou-se o paralítico, pegou no leito em que viera deitado e foi para sua casa, dando graças a Deus. Todos ficaram atónitos, dizendo: «Hoje vimos maravilhas!».
}\end{paracol}

\paragraphinfo{Ofertório}{Sl. 145, 2}
\begin{paracol}{2}\latim{
\rlettrine{L}{auda,} ánima mea, Dóminum: laudábo Dóminum in vita mea: psallam Deo meo, quámdiu ero, allelúja.
}\switchcolumn\portugues{
\slettrine{Ó}{} minha alma, louva o Senhor! Louvarei o Senhor em toda minha vida: louvarei o Senhor, enquanto eu viver, aleluia.
}\end{paracol}

\paragraph{Secreta}
\begin{paracol}{2}\latim{
\rlettrine{S}{acrifícia,} Dómine, tuis obláta conspéctibus, ignis ille divínus absúmat, qui discipulórum Christi, Fílii tui, per Spíritum Sanctum corda succéndit. Per eúndem Dóminum \emph{\&c.}
}\switchcolumn\portugues{
\qlettrine{Q}{ue} aquele fogo divino, ó Senhor, de que o Espírito Santo abrasou os corações dos discípulos de Jesus Cristo, vosso Filho, consuma o sacrifício que oferecemos diante de vossos olhos. Pelo mesmo nosso Senhor \emph{\&c.}
}\end{paracol}

\paragraphinfo{Comúnio}{Jo. 14, 18}
\begin{paracol}{2}\latim{
\rlettrine{N}{on} vos relínquam órphanos: véniam ad vos íterum, allelúja: et gaudébit cor vestrum, allelúja.
}\switchcolumn\portugues{
\rlettrine{N}{ão} vos deixarei órfãos: de novo virei a vós, aleluia: e o vosso coração se alegrará, aleluia.
}\end{paracol}

\paragraph{Postcomúnio}
\begin{paracol}{2}\latim{
\rlettrine{S}{úmpsimus,} Dómine, sacri dona mystérii: humíliter deprecántes; ut, quæ in tui commemoratiónem nos fácere præcepísti, in nostræ profíciant infirmitátis auxílium: Qui vivis \emph{\&c.}
}\switchcolumn\portugues{
\rlettrine{R}{ecebemos,} Senhor, os dons do divino mistério; e permiti, Vos suplicamos humildemente, que este sacrifício, que mandastes consumar em vossa memória, se torne em auxílio da nossa fraqueza. Vós, que viveis e Tinais \emph{\&c.}
}\end{paracol}

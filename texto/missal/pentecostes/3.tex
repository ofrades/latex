\subsection{Terceiro Domingo depois de Pentecostes}\label{3pentecostes}

\paragraphinfo{Intróito}{Sl. 24, 16 \& 18. }
\begin{paracol}{2}\latim{
\rlettrine{R}{éspice} in me et miserére mei, Dómine: quóniam únicus et pauper sum ego: vide humilitátem meam et labórem meum: et dimítte ómnia peccáta mea, Deus meus. \emph{Ps. ibid., 1-2} Ad te, Dómine, levávi ánimam meam: Deus meus, in te confído, non erubéscam.
℣. Gloria Patri \emph{\&c.}
}\switchcolumn\portugues{
\rlettrine{V}{olvei} propício vossos olhos para mim e tende compaixão de mim, pois estou só e sou pobre. Vede a minha miséria e as minhas dores, ó meu Deus, e perdoai-me todos os pecados. \emph{Sl. ibid., 1-2} A Vós, Senhor, elevei a minha alma. Ó meu Deus, confio em Vós: não permitireis que fique envergonhado.
℣. Glória ao Pai \emph{\&c.}
}\end{paracol}

\paragraph{Oração}
\begin{paracol}{2}\latim{
\rlettrine{P}{rotéctor} in te sperántium, Deus, sine quo nihil est válidum, nihil sanctum: multíplica super nos misericórdiam tuam; ut, te rectóre, te duce, sic transeámus per bona temporália, ut non amittámus ætérna. Per Dóminum \emph{\&c.}
}\switchcolumn\portugues{
\slettrine{Ó}{} Deus, protector daqueles que esperam em Vós, e sem o qual nada existe, nem sólido nem santo, multiplicai a vossa misericórdia sobre nós, de maneira que, sendo sempre governados e guiados por Vós, transitemos de tal modo pelos bens terrenos que não deixemos de gozar os eternos. Por nosso Senhor \emph{\&c.}
}\end{paracol}

\paragraphinfo{Epístola}{1. Pe. 5, 6-11}
\begin{paracol}{2}\latim{
Léctio Epístolæ beáti Petri Apóstoli.
}\switchcolumn\portugues{
Lição da Ep.ª do B. Ap.º Pedro.
}\switchcolumn*\latim{
\rlettrine{C}{aríssimi:} Humiliámini sub coténti manu Dei, ut vos exáltet in témpore visitatiónis: omnem sollicitúdinem vestram projiciéntes in eum, quóniam ipsi cura est de vobis. Sóbrii estote et vigiláte: quia adversárius vester diábolus tamquam leo rúgiens circuit, quærens, quem dévoret: cui resístite fortes in fide: sciéntes eándem passiónem ei, quæ in mundo est, vestræ fraternitáti fíeri. Deus autem omnis grátiæ, qui vocávit nos in ætérnam suam glóriam in Christo Jesu, módicum passos ipse perfíciet, confirmábit solidabítque. Ipsi glória et impérium in sǽcula sæculórum. Amen.
}\switchcolumn\portugues{
\rlettrine{C}{aríssimos:} Humilhai-vos sob a mão poderosa de Deus, para que Ele vos exalte no tempo da sua visita. Confiai a Deus todas as inquietações, porque tomará cuidado delas. Sede sóbrios e vigilantes, pois o vosso adversário, que é o demónio, andará em torno de vós, como um leão, procurando a quem devorar. Resisti-lhe, sendo sempre firmes na fé e sabendo que os vossos irmãos, que estão dispersos pelo mundo, sofrem as mesmas aflições. Deus de toda a graça (que nos chamou à sua glória por meio de Jesus Cristo) depois de haverdes padecido alguns sofrimentos vos aperfeiçoará, confirmará e fortificará. A Ele seja dada glória e soberania em todos os séculos dos séculos! Amen.
}\end{paracol}

\paragraphinfo{Gradual}{Sl. 54, 23, 17 \& 19}
\begin{paracol}{2}\latim{
\qlettrine{J}{acta} cogitátum tuum in Dómino: et ipse te enútriet. ℣. Dum clamárem ad Dóminum, exaudívit vocem meam ab his, qui appropínquant mihi.
}\switchcolumn\portugues{
\rlettrine{C}{onfiai} ao Senhor vossas inquietações: Ele vos sustentará. Logo que invoquei o Senhor, ouviu Ele a minha voz, livrando-me daqueles que me cercavam.
}\switchcolumn*\latim{
Allelúja, allelúja. ℣. \emph{Ps. 7, 12} Deus judex justus, fortis et pátiens, numquid iráscitur per síngulos dies? Allelúja.
}\switchcolumn\portugues{
Aleluia, aleluia. ℣. \emph{Sl. 7, 12} Deus é juiz justo, forte e paciente: sua ira manifesta-se, porventura, todos os dias? Aleluia.
}\end{paracol}

\paragraphinfo{Evangelho}{Lc. 15, 1-10}
\begin{paracol}{2}\latim{
\cruz Sequéntia sancti Evangélii secúndum Lucam.
}\switchcolumn\portugues{
\cruz Continuação do santo Evangelho segundo S. Lucas.
}\switchcolumn*\latim{
\blettrine{I}{n} illo témpore: Erant appropinquántes ad Jesum publicáni et peccatóres, ut audírent illum. Et murmurábant pharisǽi et scribæ, dicéntes: Quia hic peccatóres recipit et mandúcat cum illis. Et ait ad illos parábolam istam, dicens: Quis ex vobis homo, qui habet centum oves: et si perdíderit unam ex illis, nonne dimíttit nonagínta novem in desérto, et vadit ad illam, quæ períerat, donec invéniat eam? Et cum invénerit eam, impónit in húmeros suos gaudens: et véniens domum, cónvocat amícos et vicínos, dicens illis: Congratulámini mihi, quia invéni ovem meam, quæ períerat? Dico vobis, quod ita gáudium erit in cœlo super uno peccatóre pœniténtiam agénte, quam super nonagínta novem justis, qui non índigent pœniténtia. Aut quæ múlier habens drachmas decem, si perdíderit drachmam unam, nonne accéndit lucérnam, et evérrit domum, et quærit diligénter, donec invéniat? Et cum invénerit, cónvocat amícas et vicínas, dicens: Congratulámini mihi, quia invéni drachmam, quam perdíderam? Ita dico vobis: gáudium erit coram Angelis Dei super uno peccatóre pœniténtiam agénte.
}\switchcolumn\portugues{
\blettrine{N}{aquele} tempo, aproximaram-se de Jesus os publicanos e os pecadores para O ouvirem. E os fariseus e os escribas murmuravam, dizendo: «Este homem recebe os pecadores e come com eles». E Jesus propôs-lhes esta parábola e disse-lhes: «Qual é de vós o que, possuindo cem ovelhas, e, perdendo uma, não deixa as noventa e nove no deserto e não vai em procura da que se perdera até achá-la? E, achando-a, a não põe aos ombros com alegria, e, vindo para a sua casa, não reúne os amigos e vizinhos, dizendo-lhes: «congratulai-vos comigo, pois achei a minha ovelha, que se perdera»? Digo-vos que do mesmo modo haverá mais alegria no céu por um pecador que faça penitência do que por noventa e nove justos que não necessitem de fazer penitência. Ou, ainda, que mulher haverá que, possuindo dez dracmas, e, perdendo uma, não acende a candeia, varre a casa e procura a dracma com cuidado até achá-la? E, encontrando-a, não reúne as amigas e vizinhas, dizendo-lhes: «congratulai-vos comigo, pois encontrei a dracma que perdera?». Assim, vos digo: «Haverá muita alegria perante os Anjos de Deus por um só pecador que faça penitência».
}\end{paracol}

\paragraphinfo{Ofertório}{Sl. 9, 11-12 \& 13}
\begin{paracol}{2}\latim{
\rlettrine{S}{perent} in te omnes, qui novérunt nomen tuum, Dómine: quóniam non derelínquis quæréntes te: psállite Dómino, qui hábitat in Sion: quóniam non est oblítus oratiónem páuperum.
}\switchcolumn\portugues{
\rlettrine{C}{onfiem} em Vós, Senhor, aqueles que conhecem o vosso nome; porque nunca abandonais os que a Vós recorrem. Cantai hinos ao Senhor, que habita em Sião. Ele se não esqueceu da oração dos pobres.
}\end{paracol}

\paragraph{Secreta}
\begin{paracol}{2}\latim{
\rlettrine{R}{éspice,} Dómine, múnera supplicántis Ecclésiæ: et salúti credéntium perpétua sanctificatióne suménda concéde. Per Dóminum \emph{\&c.}
}\switchcolumn\portugues{
\rlettrine{O}{lhai} benignamente, Senhor, para estas ofertas da Igreja suplicante, e concedei aos fiéis a graça de as receberem sempre santamente, e com fruto para a sua santificação. Por nosso Senhor \emph{\&c.}
}\end{paracol}

\paragraphinfo{Comúnio}{Lc. 15, 10}
\begin{paracol}{2}\latim{
\rlettrine{D}{ico} vobis: gáudium est Angelis Dei super uno peccatóre pœniténtiam agénte.
}\switchcolumn\portugues{
\rlettrine{V}{os} digo: há muita alegria perante os Anjos de Deus por um só pecador que faça penitência.
}\end{paracol}

\paragraph{Postcomúnio}
\begin{paracol}{2}\latim{
\rlettrine{S}{ancta} tua nos, Dómine, sumpta vivíficent: et misericórdiæ sempitérnæ prǽparent expiátos. Per Dóminum nostrum \emph{\&c.}
}\switchcolumn\portugues{
\rlettrine{S}{enhor,} que a recepção dos vossos sacramentos nos vivifique, e que, servindo de expiação dos nossos pecados, nos torne dignos da misericórdia eterna. Por nosso Senhor \emph{\&c.}
}\end{paracol}

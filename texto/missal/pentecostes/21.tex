\subsection{Vigésimo Primeiro Domingo depois de Pentecostes}

\paragraphinfo{Intróito}{Est. 13, 9 \& 10-11}
\begin{paracol}{2}\latim{
\rlettrine{I}{n} voluntáte tua, Dómine, univérsa sunt pósita, et non est, qui possit resístere voluntáti tuæ: tu enim fecísti ómnia, cœlum et terram et univérsa, quæ cœli ámbitu continéntur: Dominus universórum tu es. \emph{Ps. 118, 1} Beáti immaculáti in via: qui ámbulant in lege Dómini.
℣. Gloria Patri \emph{\&c.}
}\switchcolumn\portugues{
\rlettrine{T}{odas} as cousas, Senhor, estão sujeitas à vossa vontade, e ninguém pode resistir-lhe; pois criastes tudo: o céu, a terra e o que se encerra no âmbito dos céus. Sois o Senhor do universo. \emph{Sl. 118, 1} Bem-aventurados aqueles cuja vida é imaculada: aqueles que praticam a Lei do Senhor.
℣. Glória ao Pai \emph{\&c.}
}\end{paracol}

\paragraph{Oração}
\begin{paracol}{2}\latim{
\rlettrine{F}{amíliam} tuam, quǽsumus, Dómine, contínua pietáte custódi: ut a cunctis adversitátibus, te protegénte, sit líbera, et in bonis áctibus tuo nómini sit devóta. Per Dóminum nostrum \emph{\&c.}
}\switchcolumn\portugues{
\rlettrine{G}{uardai} com vossa contínua misericórdia a vossa família, Senhor, Vos suplicamos, a fim de que sob a vossa protecção seja preservada de todas as adversidades e se dedique à prática das boas obras, em honra do vosso nome. Por nosso Senhor \emph{\&c.}
}\end{paracol}

\paragraphinfo{Epístola}{Ef. 6, 10-17}
\begin{paracol}{2}\latim{
Lectio Epistolæ beáti Pauli Apóstoli ad Ephésios.
}\switchcolumn\portugues{
Lição da Ep.ª do B. Ap.º Paulo aos Efésios.
}\switchcolumn*\latim{
\rlettrine{F}{ratres:} Confortámini in Dómino et in poténtia virtútis ejus. Indúite vos armatúram Dei, ut póssitis stare advérsus insídias diáboli. Quóniam non est nobis colluctátio advérsus carnem et sánguinem: sed advérsus príncipes et potestátes, advérsus mundi rectóres tenebrárum harum, contra spirituália nequítiae, in cœléstibus. Proptérea accípite armatúram Dei, ut póssitis resístere in die malo et in ómnibus perfécti stare. State ergo succíncti lumbos vestros in veritáte, et indúti lorícam justítiæ, et calceáti pedes in præparatióne Evangélii pacis: in ómnibus suméntes scutum fídei, in quo póssitis ómnia tela nequíssimi ígnea exstínguere: et gáleam salútis assúmite: et gládium spíritus, quod est verbum Dei.
}\switchcolumn\portugues{
\rlettrine{M}{eus} irmãos: Confortai-vos no Senhor e no seu poder omnipotente. Revesti-vos de todas armas de Deus, a fim de que possais suportar as ciladas do demónio; pois temos de combater, não só contra a carne e o sangue, mas também contra os príncipes e as potestades do inferno, contra os dominadores deste mundo de trevas e contra os espíritos malignos, espalhados nas regiões do ar. Tomai, então, as armas de Deus, a fim de que, estando preparados para tudo, possais resistir no dia mau e ficar de pé e perfeitos em todas as cousas. Sede firmes, pois! Que a verdade seja o cinto dos vossos rins, e a justiça a vossa couraça. Calçai vossos pés para vos preparardes para seguir o Evangelho da paz. Empunhai sempre o escudo da fé, para poderdes quebrar os dardos inflamados do espírito maligno. Revesti-vos, também, com o capacete da salvação e com a espada espiritual, que é a palavra de Deus.
}\end{paracol}

\paragraphinfo{Gradual}{Sl. 89, 1-2}
\begin{paracol}{2}\latim{
\rlettrine{D}{ómine,} refúgium factus es nobis, a generatióne et progénie. ℣. Priúsquam montes fíerent aut formarétur terra et orbis: a sǽculo et usque in sǽculum tu es, Deus.
}\switchcolumn\portugues{
\rlettrine{T}{endes} sido, Senhor, o nosso refúgio de geração em geração. Antes que as montanhas houvessem sido formadas e que a terra e o mundo tivessem sido criados, já, desde toda a eternidade, Vós éreis Deus.
}\switchcolumn*\latim{
Allelúja, allelúja. ℣. \emph{Ps. 113, 1} In éxitu Israël de Ægýpto, domus Jacob de pópulo bárbaro. Allelúja.
}\switchcolumn\portugues{
Aleluia, aleluia. ℣. \emph{Sl. 113, 1} Quando Israel saiu do Egipto e a casa de Jacob do meio de um povo bárbaro. Aleluia.
}\end{paracol}

\paragraphinfo{Evangelho}{Mt. 18, 23-35}
\begin{paracol}{2}\latim{
\cruz Sequéntia sancti Evangélii secúndum Lucam.
}\switchcolumn\portugues{
\cruz Continuação do santo Evangelho segundo S. Mateus.
}\switchcolumn*\latim{
\blettrine{I}{n} illo témpore: Dixit Jesus discípulis suis parábolam hanc: Assimilátum est regnum cœlórum hómini regi, qui vóluit ratiónem pónere cum servis suis. Et cum cœpísset ratiónem pónere, oblátus est ei unus, qui debébat ei decem mília talénta. Cum autem non habéret, unde rédderet, jussit eum dóminus ejus venúmdari et uxórem ejus et fílios et ómnia, quæ habébat, et reddi. Prócidens autem servus ille, orábat eum, dicens: Patiéntiam habe in me, et ómnia reddam tibi. Misértus autem dóminus servi illíus, dimísit eum et débitum dimísit ei. Egréssus autem servus ille, invénit unum de consérvis suis, qui debébat ei centum denários: et tenens suffocábat eum, dicens: Redde, quod debes. Et prócidens consérvus ejus, rogábat eum, dicens: Patiéntiam habe in me, et ómnia reddam tibi. Ille autem nóluit: sed ábiit, et misit eum in cárcerem, donec rédderet débitum. Vidéntes autem consérvi ejus, quæ fiébant, contristáti sunt valde: et venérunt et narravérunt dómino suo ómnia, quæ facta fúerant. Tunc vocávit illum dóminus suus: et ait illi: Serve nequam, omne débitum dimísi tibi, quóniam rogásti me: nonne ergo opórtuit et te miseréri consérvi tui, sicut et ego tui misértus sum? Et irátus dóminus ejus, trádidit eum tortóribus, quoadúsque rédderet univérsum débitum. Sic et Pater meus cœléstis fáciet vobis, si non remiséritis unusquísque fratri suo de córdibus vestris.
}\switchcolumn\portugues{
\blettrine{N}{aquele} tempo, Jesus disse aos seus discípulos esta parábola: O reino dos céus é semelhante a um rei que quis fazer contas com seus servos. Logo que começou as contas, apresentou-se um servo que lhe devia dez mil talentos. Ora, como este não tinha com que pagar, mandou o senhor que o vendessem, assim como sua mulher, filhos e tudo quanto possuía, para liquidar a dívida. Então este servo, prostrando-se aos pés do rei, pedia-lhe: «Tende paciência para comigo, e pagarei tudo». E o senhor compadeceu-se do servo, deixou-o ir embora e perdoou-lhe a dívida. Apenas o servo saiu, encontrou ele um seu companheiro, que lhe devia cem dinheiros. Logo, agarrou-o até quase o sufocar e disse-lhe: «Paga-me o que me deves». O companheiro prostrou-se a seus pés, suplicando-lhe nestes termos: «Tende paciência para comigo, e pagarei tudo». Mas não quis atendê-lo. Foi dali e mandou metê-lo na cadeia, até que lhe pagasse a dívida. Vendo os outros servos, seus companheiros, o que acontecera, ficaram profundamente tristes, indo narrar tudo ao senhor. Então o senhor mandou chamar o servo e disse-lhe: «Perdoei-te toda tua dívida, porque assim me rogaste; portanto não devias tu, também, ter piedade do teu companheiro, como tive de ti?». Imediatamente se encolerizou o senhor, entregando-o aos algozes da justiça, até que lhe pagasse a sua dívida. Pois bem, terminou Jesus, assim vos tratará meu Pai celeste se cada um não perdoar do íntimo do coração a sua dívida ao seu irmão.
}\end{paracol}

\paragraphinfo{Ofertório}{Jb 1}
\begin{paracol}{2}\latim{
\rlettrine{V}{ir} erat in terra Hus, nómine Job: simplex et rectus ac timens Deum: quem Satan pétiit ut tentáret: et data est ei potéstas a Dómino in facultátes et in carnem ejus: perdidítque omnem substántiam ipsíus et fílios: carnem quoque ejus gravi úlcere vulnerávit.
}\switchcolumn\portugues{
\rlettrine{V}{ivia} no país de Hus um homem chamado Job, que era simples, justo e temente a Deus. Então Satanás pediu licença ao Senhor para o tentar, o que lhe foi permitido, mas somente quanto aos bens e ao corpo: e Satanás fez-lhe perder todos os bens e os filhos, e ainda lhe afligiu o corpo com uma chaga.
}\end{paracol}

\paragraph{Secreta}
\begin{paracol}{2}\latim{
\rlettrine{S}{uscipe,} Dómine, propítius hóstias: quibus et te placári voluísti, et nobis salútem poténti pietáte restítui. Per Dóminum \emph{\&c.}
}\switchcolumn\portugues{
\rlettrine{A}{ceitai} propício, Senhor, estas hóstias com as quais quisestes aplacar-Vos, e, pela vossa infinita bondade, concedei-nos a salvação. Por nosso Senhor \emph{\&c.}
}\end{paracol}

\paragraphinfo{Comúnio}{Sl. 118, 81, 84 \& 86}
\begin{paracol}{2}\latim{
\rlettrine{I}{n} salutári tuo ánima mea, et in verbum tuum sperávi: quando fácies de persequéntibus me judícium? iníqui persecúti sunt me, ádjuva me, Dómine, Deus meus.
}\switchcolumn\portugues{
\rlettrine{A}{} minha alma suspira pela salvação: e pus toda minha esperança na vossa palavra. Quando tratareis com justiça os meus perseguidores? Homens injustos perseguem-me; auxiliai-me, Senhor!
}\end{paracol}

\paragraph{Postcomúnio}
\begin{paracol}{2}\latim{
\rlettrine{I}{mmortalitátis} alimóniam consecúti, quǽsumus, Dómine: ut, quod ore percépimus, pura mente sectémur. Per Dóminum \emph{\&c.}
}\switchcolumn\portugues{
\rlettrine{H}{avendo} recebido o alimento da imortalidade, permiti, Senhor, Vos suplicamos, que guardemos com o coração puro aquilo que a nossa boca recebeu. Por nosso Senhor \emph{\&c.}
}\end{paracol}

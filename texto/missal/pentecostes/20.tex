\subsection{Vigésimo Domingo depois de Pentecostes}

\paragraphinfo{Intróito}{Dn. 3, 31, 29 \& 35}
\begin{paracol}{2}\latim{
\rlettrine{O}{mnia,} quæ fecísti nobis, Dómine, in vero judício fecísti, quia peccávimus tibi et mandátis tuis non obœdívimus: sed da glóriam nómini tuo, et fac nobíscum secúndum multitúdinem misericórdiæ tuæ. \emph{Ps. 118, 1} Beáti immaculáti in via: qui ámbulant in lege Dómini.
℣. Gloria Patri \emph{\&c.}
}\switchcolumn\portugues{
\rlettrine{T}{udo} quanto fizestes contra nós, Senhor, foi com justiça, pois pecámos e não Obedecemos aos vossos mandamentos; mas glorificai o vosso nome e tratai-nos segundo a grandeza da vossa misericórdia. \emph{Sl. 118, 1} Bem-aventurados aqueles cuja vida é imaculada: aqueles que praticam a Lei do Senhor.
℣. Glória ao Pai \emph{\&c.}
}\end{paracol}

\paragraph{Oração}
\begin{paracol}{2}\latim{
\rlettrine{L}{argíre,} quǽsumus, Dómine, fidélibus tuis indulgéntiam placátus et pacem: ut páriter ab ómnibus mundéntur offénsis, et secúra tibi mente desérviant. Per Dóminum \emph{\&c.}
}\switchcolumn\portugues{
\rlettrine{C}{ompadecei-Vos} dos vossos fiéis, Senhor, e dignai-Vos conceder-lhes o perdão e a paz, a fim de que, havendo sido limpos de suas faltas, possam servir-Vos com toda a confiança. Por nosso Senhor \emph{\&c.}
}\end{paracol}

\paragraphinfo{Epístola}{Ef. 5, 15-21}
\begin{paracol}{2}\latim{
Léctio Epístolæ beáti Pauli Apóstoli ad Ephésios.
}\switchcolumn\portugues{
Lição da Ep.ª do B. Ap.º Paulo aos Efésios.
}\switchcolumn*\latim{
\rlettrine{F}{ratres:} Vidéte, quómodo caute ambulétis: non quasi insipiéntes, sed ut sapiéntes, rediméntes tempus, quóniam dies mali sunt. Proptérea nolíte fíeri imprudéntes, sed intellegéntes, quae sit volúntas Dei. Et nolíte inebriári vino, in quo est luxúria: sed implémini Spíritu Sancto, loquéntes vobismetípsis in psalmis et hymnis et cánticis spirituálibus, cantántes et psalléntes in córdibus vestris Dómino: grátias agéntes semper pro ómnibus, in nómine Dómini nostri Jesu Christi, Deo et Patri. Subjecti ínvicem in timóre Christi.
}\switchcolumn\portugues{
\rlettrine{M}{eus} irmãos: Tende cuidado de vos conduzirdes com prudência, não como insensatos, mas como prudentes, aproveitando o tempo, pois os dias são maus. Assim, pois, não sejais imprudentes, mas procurai conhecer qual é a vontade de Deus. Não bebais vinho com excesso, o que é luxúria. Enchei-vos do Espírito Santo, entretendo-vos uns aos outros com Salmos, Hinos, cânticos espirituais e outros cânticos e louvores, saídos do íntimo dos vossos corações, para glória do Senhor, dando graças sempre e em tudo a Deus Pai, em nome de nosso Senhor Jesus Cristo, e submetendo-vos uns aos outros no temor de Cristo.
}\end{paracol}

\paragraphinfo{Gradual}{Sl. 144, 15-16}
\begin{paracol}{2}\latim{
\rlettrine{O}{culi} ómnium in te sperant, Dómine: et tu das illis escam in témpore opportúno. ℣. Aperis tu manum tuam: et imples omne ánimal benedictióne.
}\switchcolumn\portugues{
\rlettrine{T}{odos} os olhares se volvem para Vós, Senhor; a todos os entes dais alimento em tempo conveniente. Abris as mãos e encheis de bênçãos tudo o que tem vida.
}\switchcolumn*\latim{
Allelúja, allelúja. ℣. \emph{Ps. 107, 2} Parátum cor meum, Deus, parátum cor meum: cantábo, et psallam tibi, glória mea. Allelúja.
}\switchcolumn\portugues{
Aleluia, aleluia. ℣. \emph{Sl. 107, 2} Meu coração está preparado, ó Deus; o meu coração está preparado: cantarei vossos louvores, ó Vós, que sois a minha glória. Aleluia.
}\end{paracol}

\paragraphinfo{Evangelho}{Jo. 4, 46-53}
\begin{paracol}{2}\latim{
\cruz Sequéntia sancti Evangélii secúndum Joánnem.
}\switchcolumn\portugues{
\cruz Continuação do santo Evangelho segundo S. João.
}\switchcolumn*\latim{
\blettrine{I}{n} illo témpore: Erat quidam régulus, cujus fílius infirmabátur Caphárnaum. Hic cum audísset, quia Jesus adveníret a Judǽa in Galilǽam, ábiit ad eum, et rogábat eum, ut descénderet et sanáret fílium ejus: incipiébat enim mori. Dixit ergo Jesus ad eum: Nisi signa et prodígia vidéritis, non créditis. Dicit ad eum régulus: Dómine, descénde, priúsquam moriátur fílius meus. Dicit ei Jesus: Vade, fílius tuus vivit. Crédidit homo sermóni, quem dixit ei Jesus, et ibat. Jam autem eo descendénte, servi occurrérunt ei et nuntiavérunt, dicéntes, quia fílius ejus víveret. Interrogábat ergo horam ab eis, in qua mélius habúerit. Et dixérunt ei: Quia heri hora séptima relíquit eum febris. Cognóvit ergo pater, quia illa hora erat, in qua dixit ei Jesus: Fílius tuus vivit: et crédidit ipse et domus ejus tota.
}\switchcolumn\portugues{
\blettrine{N}{aquele} tempo, havia um oficial em Cafarnaum, cujo filho estava enfermo. Tendo ele sabido que Jesus vinha da Judeia para a Galileia, foi ter com Ele, rogando-Lhe que fosse a sua casa curar seu filho, que principiava a agonizar. Disse-lhe Jesus: «Se não vedes prodígios e milagres não acreditais!». Respondeu-Lhe o oficial: «Senhor, vinde, antes que meu filho morra». Jesus disse-lhe: «Vai; o teu filho vive!». Acreditou este homem na palavra de Jesus e partiu. Ia já no caminho, e eis que seus servos foram ao seu encontro, anunciando-lhe que o filho vivia! Então perguntou-lhes qual a hora em que se achara melhor o filho. Os servos responderam: «Ontem, à hora sétima, deixou-o a febre». E o pai reconheceu ter sido aquela a hora em que Jesus lhe dissera: «O teu filho vive». Ele, pois, assim como toda sua família, acreditou.
}\end{paracol}

\paragraphinfo{Ofertório}{Sl. 136, 1}
\begin{paracol}{2}\latim{
\rlettrine{S}{uper} flúmina Babylónis illic sédimus et flévimus: dum recordarémur tui, Sion.
}\switchcolumn\portugues{
\qlettrine{J}{unto} das margens dos rios da Babilónia nos assentámos, chorando as recordações do vosso passado, ó Sião.
}\end{paracol}

\paragraph{Secreta}
\begin{paracol}{2}\latim{
\rlettrine{C}{œléstem} nobis prǽbeant hæc mystéria, quǽsumus,
Dómine, medicínam: et vítia nostri cordis expúrgent. Per Dóminum \emph{\&c.}
}\switchcolumn\portugues{
\qlettrine{Q}{ue} estes mistérios, Senhor, nos sirvam de remédio celestial, Vos suplicamos, e purifiquem os nossos corações de todas nossas iniquidades. Por nosso Senhor \emph{\&c.}
}\end{paracol}

\paragraphinfo{Comúnio}{Sl. 118, 49-50}
\begin{paracol}{2}\latim{
\rlettrine{M}{eménto} verbi tui servo tuo, Dómine, in quo mihi spem dedísti: hæc me consoláta est in humilitáte mea.
}\switchcolumn\portugues{
\rlettrine{L}{embrai-Vos} da vossa promessa ao vosso servo, Senhor, na qual tenho esperado com toda a confiança: é ela que me consola na minha humilhação.
}\end{paracol}

\paragraph{Postcomúnio}
\begin{paracol}{2}\latim{
\rlettrine{U}{t} sacris, Dómine, reddámur digni munéribus: fac nos, quǽsumus, tuis semper obœdíre mandátis. Per Dóminum nostrum \emph{\&c.}
}\switchcolumn\portugues{
\rlettrine{S}{enhor,} permiti que para sermos dignos dos vossos sacratíssimos mistérios obedeçamos sempre aos vossos mandamentos. Por nosso Senhor \emph{\&c.}
}\end{paracol}

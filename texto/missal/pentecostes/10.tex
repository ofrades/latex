\subsection{Décimo Domingo depois de Pentecostes}

\paragraphinfo{Intróito}{Sl. 54, 17, 18, 20 \& 23}
\begin{paracol}{2}\latim{
\rlettrine{C}{um} clamárem ad Dóminum, exaudívit vocem meam, ab his, qui appropínquant mihi: et humiliávit eos, qui est ante sǽcula et manet in ætérnum: jacta cogitátum tuum in Dómino, et ipse te enútriet. \emph{Ps. ibid., 2} Exáudi, Deus, oratiónem meam, et ne despéxeris deprecatiónem meam: inténde mihi et exáudi me.
℣. Gloria Patri \emph{\&c.}
}\switchcolumn\portugues{
\rlettrine{C}{omo} invocasse o Senhor, ouviu Ele a minha voz e protegeu-me contra os que me combatem. Aquele que existia antes dos séculos e subsistirá eternamente humilhou-os. Depositai nas mãos do Senhor todas as preocupações, pois Ele vos sustentará. \emph{Sl. ibid., 2} Ouvi, Senhor, a minha oração e não desprezeis a minha humilde súplica: escutai-me, ouvi-me.
℣. Glória ao Pai \emph{\&c.}
}\end{paracol}

\paragraph{Oração}
\begin{paracol}{2}\latim{
\rlettrine{D}{eus,} qui omnipoténtiam tuam parcéndo máxime et miserándo maniféstas: multíplica super nos misericórdiam tuam; ut, ad tua promíssa curréntes, cœléstium bonórum fácias esse consórtes. Per Dóminum \emph{\&c.}
}\switchcolumn\portugues{
\slettrine{Ó}{} Deus, que manifestais principalmente o vosso poder perdoando ao pecador e compadecendo-Vos das suas misérias, multiplicai sobre nós a abundância da vossa misericórdia, a fim de que, suspirando durante esta vida pelos bens que nos prometestes, nos tornamos participantes deles no céu. Por nosso Senhor \emph{\&c.}
}\end{paracol}

\paragraphinfo{Epístola}{1. Cor 12, 2-11}
\begin{paracol}{2}\latim{
Léctio Epístolæ beáti Pauli Apóstoli ad Corínthios.
}\switchcolumn\portugues{
Lição da Ep.ª do B. Ap.º Paulo aos Coríntios.
}\switchcolumn*\latim{
\rlettrine{F}{ratres:} Scitis, quóniam, cum gentes essétis, ad simulácra muta prout ducebámini eúntes. Ideo notum vobisfacio, quod nemo in Spíritu Dei loquens, dicit anáthema Jesu. Et nemo potest dícere, Dóminus Jesus, nisi in Spíritu Sancto. Divisiónes vero gratiárum sunt, idem autem Spíritus. Et divisiónes ministratiónum sunt, idem autem Dóminus. Et divisiónes operatiónum sunt, idem vero Deus, qui operátur ómnia in ómnibus. Unicuíque autem datur manifestátio Spíritus ad utilitátem. Alii quidem per Spíritum datur sermo sapiéntiæ álii autem sermo sciéntiæ secúndum eúndem Spíritum: álteri fides in eódem Spíritu: álii grátia sanitátum in uno Spíritu: álii operátio virtútum, álii prophétia, álii discrétio spirítuum, álii génera linguárum, álii interpretátio sermónum. Hæc autem ómnia operátur unus atque idem Spíritus, dívidens síngulis, prout vult.
}\switchcolumn\portugues{
\rlettrine{M}{eus} irmãos: Recordai-vos de que, quando éreis pagãos, íeis aos ídolos mudos, sempre que vos conduziam. Ora, eu vos digo que ninguém, inspirado pelo Espírito de Deus, poderá anatematizar Jesus; nem ninguém, também, poderá dizer que Jesus é Senhor, senão inspirado pelo Espírito Santo. Na verdade, há diversas graças; porém o Espírito é o mesmo. Há diversos ministérios; porém o mesmo Senhor. Há diversidade de operações; porém não há senão um só Deus, que opera todas as cousas em nós. Os dons do Espírito Santo, que se manifestam em cada um, são para utilidade comum. A um é dado pelo Espírito o dom de falar com sabedoria; a outro é dado pelo mesmo Espírito o dom de falar com ciência; a um outro é dada a fé pelo mesmo Espírito; este recebe do mesmo Espírito a graça de curar os enfermos; aquele alcança o dom dos milagres; um outro o dom das profecias, aqueloutro o discernimento dos espíritos; este o dom de falar diversas línguas; e ainda aqueloutro o dom de as interpretar. É, porém, só um e o mesmo Espírito que produz todos este dons, distribuindo-os a cada um em particular como Lhe apraz.
}\end{paracol}

\paragraphinfo{Gradual}{Sl. 16, 8 \& 2}
\begin{paracol}{2}\latim{
\rlettrine{C}{ustódi} me, Dómine, ut pupíllam óculi: sub umbra alárum tuárum prótege me. ℣. De vultu tuo judícium meum pródeat: óculi tui vídeant æquitátem.
}\switchcolumn\portugues{
\rlettrine{G}{uardai-me,} Senhor, como a pupila olhos: acolhei-me à sombra das vossas o Senhor. ℣. Que os vossos lábios pronunciem o meu juízo: e que os vossos olhos vejam a minha justiça.
}\switchcolumn*\latim{
Allelúja, allelúja. ℣. \emph{Ps 64, 2} Te decet hymnus, De us, in Sion: et tibi redde tu votum in Jerúsalem. Allelúja.
}\switchcolumn\portugues{
Aleluia, aleluia. ℣. \emph{Ps 64, 2} Diante de Vós, Senhor, é conveniente cantar hinos em Sião: diante de Vós deve cada um cumprir os seus votos em Jerusalém. Aleluia.
}\end{paracol}

\paragraphinfo{Evangelho}{Lc. 18, 9-14}
\begin{paracol}{2}\latim{
\cruz Sequéntia sancti Evangélii secúndum Lucam.
}\switchcolumn\portugues{
\cruz Continuação do santo Evangelho segundo S. Lucas.
}\switchcolumn*\latim{
\blettrine{I}{n} illo témpore: Dixit Jesus ad quosdam, qui in se confidébant tamquam justi et aspernabántur céteros, parábolam istam: Duo hómines ascendérunt in templum, ut orárent: unus pharisǽus, et alter publicánus. Pharisǽus stans, hæc apud se orábat: Deus, grátias ago tibi, quia non sum sicut céteri hóminum: raptóres, injústi, adúlteri: velut étiam hic publicánus. Jejúno bis in sábbato: décimas do ómnium, quæ possídeo. Et publicánus a longe stans nolébat nec óculos ad cœlum leváre: sed percutiébat pectus suum, dicens: Deus, propítius esto mihi peccatóri. Dico vobis: descéndit hic justificátus in domum suam ab illo: quia omnis qui se exáltat, humiliábitur: et qui se humíliat, exaltábitur.
}\switchcolumn\portugues{
\blettrine{N}{aquele} tempo, disse Jesus esta parábola a uns certos que se presumiam justos e punham a confiança em si próprios, desprezando os outros: «dous homens subiram ao templo a orar, sendo um fariseu e o outro publicano. O fariseu, estando de pé, orava assim no seu íntimo: «Ő Deus, dou-Vos graças, porque não sou como os outros homens, que são ladrões, injustos e adúlteros, nem mesmo como este publicano; pois jejuo duas vezes na semana e pago o dízimo dos meus rendimentos». O publicano, ao contrário, conservando-se afastado, não ousava, sequer, levantar os olhos para o céu, e batia no peito, dizendo: «Meu Deus, tende piedade de mim, que sou um pobre pecador». Eu vos declaro, disse Jesus, que este voltou para sua casa justificado, mas não o outro; pois quem se eleva será humilhado, e quem se humilha será exaltado».
}\end{paracol}

\paragraphinfo{Ofertório}{Sl. 24, 1-3}
\begin{paracol}{2}\latim{
\rlettrine{A}{d} te, Dómine, levávi ánimam meam: Deus meus, in te confído, non erubéscam: neque irrídeant me inimíci mei: étenim univérsi, qui te exspéctant, non confundéntur.
}\switchcolumn\portugues{
\rlettrine{A}{} Vós elevei a minha alma. O Deus, confio em Vós: não permitireis que e confundido e envergonhado: nem vencido meus inimigos; porquanto aqueles que em Vós confiam não serão confundidos.
}\end{paracol}

\paragraph{Secreta}
\begin{paracol}{2}\latim{
\rlettrine{T}{ibi,} Dómine, sacrifícia dicáta reddántur: quæ sic ad honórem nóminis tui deferénda tribuísti, ut eadem remédia fíeri nostra præstáres. Per Dóminum \emph{\&c.}
}\switchcolumn\portugues{
\rlettrine{A}{} Vós, Senhor, é devida a homenagem destes sacrifícios, pois fostes Vós que permitindo-nos que os oferecêssemos em glória do vosso nome, os tornastes remédio das nossas almas. Por nosso Senhor \emph{\&c.}
}\end{paracol}

\paragraphinfo{Comúnio}{Sl. 50, 21}
\begin{paracol}{2}\latim{
\rlettrine{A}{cceptábis} sacrificium justítiæ, oblatiónes et holocáusta, super altáre tuum, Dómine.
}\switchcolumn\portugues{
\rlettrine{D}{ignai-Vos,} Senhor, aceitar sobre o vosso Altar o sacrifício da justiça, as ofertas e os holocaustos.
}\end{paracol}

\paragraph{Postcomúnio}
\begin{paracol}{2}\latim{
\qlettrine{Q}{uǽsumus,} Dómine, Deus noster: ut, quos divínis reparáre non désinis sacraméntis, tuis non destítuas benígnus auxíliis. Per Dóminum nostrum \emph{\&c.}
}\switchcolumn\portugues{
\rlettrine{V}{os} suplicamos, ó Senhor, nosso Deus, não priveis do vosso auxílio àqueles a quem não cessais de alimentar com os divinos sacramentos. Por nosso Senhor \emph{\&c.}
}\end{paracol}

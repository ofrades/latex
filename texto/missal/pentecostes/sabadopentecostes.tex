\subsectioninfo{Sábado das Têmporas de Pentecostes}{Estação em São Pedro}

\paragraphinfo{Intróito}{Rm. 5, 5}
\begin{paracol}{2}\latim{
\rlettrine{C}{áritas} Dei diffúsa est in córdibus nostris, allelúja: per inhabitántem Spíritum ejus in nobis, allelúja, allelúja. \emph{Ps. 102, 1} Benedic, anima mea, Dómino: et ómnia, quæ intra me sunt, nómini sancto ejus.
℣. Gloria Patri \emph{\&c.}
}\switchcolumn\portugues{
\rlettrine{O}{} amor de Deus espalhou-se nos nossos corações, aleluia: pelo seu Espírito que habita em nós, aleluia, aleluia. \emph{Sl. 102, 1} Bendize o Senhor, ó minha alma: e que tudo o que existe em mim bendiga o seu santo nome.
℣. Glória ao Pai \emph{\&c.}
}\end{paracol}

\paragraph{Oração}
\begin{paracol}{2}\latim{
\rlettrine{M}{éntibus} nostris, quǽsumus, Dómine, Spíritum Sanctum benígnus infúnde: cujus et sapiéntia cónditi sumus, et providéntia gubernámur. Per Dóminum \emph{\&c.}
}\switchcolumn\portugues{
\rlettrine{S}{enhor,} Vos rogamos, dignai-Vos infundir em nossas almas o Espírito Santo, cuja sabedoria nos criou e cuja providência nos governa. Por nosso Senhor \emph{\&c.}
}\end{paracol}

\paragraphinfo{1.ª Lição}{Jl. 2, 28-32}
\begin{paracol}{2}\latim{
Léctio Joélis Prophétæ.
}\switchcolumn\portugues{
Lição do Profeta Joel.
}\switchcolumn*\latim{
\rlettrine{H}{æc} dicit Dóminus Deus: Effúndam Spíritum meum super omnem carnem: et prophetábunt fílii vestri et fíliæ vestræ: senes vestri sómnia somniábunt, et júvenes vestri visiónes vidébunt. Sed et super servos meos et ancíllas in diébus illis effúndam Spíritum meum. Et dabo prodígia in cœlo et in terra, sánguinem et ignem et vapórem fumi. Sol convertétur in ténebras et luna in sánguinem: ántequam véniat dies Dómini magnus ei horríbilis. Et erit: omnis, qui invocáverit nomen Dómini, salvus erit.
}\switchcolumn\portugues{
\rlettrine{E}{is} o que diz o Senhor Deus: «Espalharei o meu Espírito em toda a carne; os vossos filhos e filhas profetizarão; os vossos anciãos terão sonhos misteriosos: e os vossos jovens visões. Naqueles dias enviarei o meu Espírito aos meus servos e servas; e farei prodígios no céu e na terra: sangue, fogo e nuvens de fumo. O sol converter-se-á em trevas e a lua em sangue, antes que chegue o grande e terrível dia do Senhor. Então, todo aquele que invocar o nome do Senhor será salvo».
}\end{paracol}

\begin{paracol}{2}\latim{
Allelúja. ℣. \emph{Joann. 6, 64} Spíritus est, qui vivíficat: caro autem non prodest quidquam.
}\switchcolumn\portugues{
Aleluia. ℣. \emph{Jo. 6, 64} O Espírito é que dá a vida; pois a carne não dá nenhum proveito.
}\end{paracol}

\paragraph{Oração}
\begin{paracol}{2}\latim{
\rlettrine{I}{llo} nos igne, quǽsumus, Dómine, Spíritus Sanctus inflámmet: quem Dóminus noster Jesus Christus misit in terram, et vóluit veheménter accéndi: Qui tecum \emph{\&c.}
}\switchcolumn\portugues{
\rlettrine{S}{enhor,} Vos suplicamos, fazei que o Espírito Santo nos abrase com aquele fogo que Cristo enviou à terra e quer ardentemente que permaneça aceso. Ele, que convosco vive e reina \emph{\&c.}
}\end{paracol}

\paragraphinfo{2.ª Lição}{Lv. 23, 9-11, 15-17 \& 21}
\begin{paracol}{2}\latim{
Léctio libri Levítici.
}\switchcolumn\portugues{
Lição do Livro Levítico.
}\switchcolumn*\latim{
\rlettrine{I}{n} diébus illis: Locútus est Dóminus ad Móysen, dicens: Lóquere fíliis Israel, et dices ad eos: Cum ingréssi fuéritis terram, quam ego dabo vobis, et messuéritis ségetem, ferétis manípulos spicárum, primítias messis vestræ ad sacerdótem: qui elevábit fascículum coram Dómino, ut acceptábile sit pro vobis, áltero die sábbati, et sanctificábit illum. Numerábitis ergo ab áltero die sábbati, in quo obtulístis manípulum primitiárum, septem hebdómadas plenas, usque ad álteram diem expletiónis hebdómadæ séptiniæ, id est, quinquagínta dies: et sic offerétis sacrifícium novum Dómino ex ómnibus habitáculis vestris, panes primitiárum duos de duábus decimis símilæ fermentátæ, quos coquétis in primítias Dómini. Et vocábitis hunc diem celebérrimum atque sanctíssimum: omne opus servíle non faciétis in eo. Legítimum sempitérnum erit in cunctis habitáculis et generatiónibus vestris: dicit Dóminus omnípotens.
}\switchcolumn\portugues{
\rlettrine{N}{aqueles} dias, disse o Senhor a Moisés: «Fala aos filhos de Israel e diz-lhes: «Quando entrardes na terra, que vos der, e fizerdes a ceifa, apresentareis ao sacerdote um molho de espigas, como primícias da vossa ceifa». O sacerdote no dia seguinte ao sábado o apresentará perante o Senhor e o santificará, para que seja recebido em vosso favor. Então, contareis, desde o dia seguinte ao sábado (dia em que foi oferecido o molho de primícias) sete semanas inteiras até ao dia seguinte da sétima semana, isto é, cinquenta dias, e oferecereis ao Senhor nova oblação. Também trareis de vossas casas dous pães de primícias, que serão fabricados de duas dezenas da flor de farinha fermentada, os quais cozereis para as primícias do Senhor, para um sacrifício novo. A este dia chamareis soleníssimo e santíssimo; e não praticareis durante ele nenhum trabalho servil. Esta ordem será uma lei perpétua em todas vossas casas e para toda vossa descendência»: diz o Senhor omnipotente.
}\end{paracol}

\begin{paracol}{2}\latim{
Allelúja. ℣. \emph{Job 26, 13} Spíritus ejus ornávit cœlos.
}\switchcolumn\portugues{
Aleluia. ℣. \emph{Jb 26, 13} Seu Espírito aformoseou os céus.
}\end{paracol}

\paragraph{Oração}
\begin{paracol}{2}\latim{
\rlettrine{D}{eus,} qui, ad animárum medélam, jejúnii devotióne castigári córpora præcepísti: concéde nobis propítius; et mente et córpore tibi semper esse devótos. Per Dóminum \emph{\&c.}
}\switchcolumn\portugues{
\slettrine{Ó}{} Deus, que para a cura das nossas almas ordenastes que castigássemos os nossos corpos com o devoto exercício do jejum, concedei-nos propício a graça de Vos sermos fielmente dedicados, tanto com o corpo, como com o espírito. Por nosso Senhor \emph{\&c.}
}\end{paracol}

\paragraphinfo{3.ª Lição}{Dt. 26, 1-11}
\begin{paracol}{2}\latim{
Léctio libri Deuteronómii.
}\switchcolumn\portugues{
Lição do Livro Deuteronómio.
}\switchcolumn*\latim{
\rlettrine{I}{n} diébus illis: Dixit Moyses fíliis Israël: Audi, Israël, quæ ego præcípio tibi hódie. Cum intráveris terram, quam Dóminus, Deus tuus, tibi datúrus est possidéndam, et obtinúeris eam atque habitáveris in ea: tolles de cunctis frúgibus tuis primítias, et pones in cartállo, pergésque ad locum, quem Dóminus, Deus tuus, elégerit, ut ibi invocétur nomen ejus: accedésque ad sacerdótem, qui fúerit in diébus illis, et dices ad eum: Profíteor hódie coram Dómino, Deo tuo, qui exaudívit nos, et respexit humilitátem nostram et labórem atque angústiam: et edúxit nos de Ægýpto in manu forti et bráchio exténto, in ingénti pavóre, in signis atque porténtis: et introdúxit ad locum istum, et trádidit nobis terram lacte et melle manántem. Et idcírco nunc óffero primítias frugum terræ, quam Dóminus
dedit mihi. Et dimíttes eas in conspéctu Dómini, Dei tui, et adoráto Dómino, Deo tuo. Et epuláberis in ómnibus bonis, quæ Dóminus, Deus tuus, déderit tibi.
}\switchcolumn\portugues{
\rlettrine{N}{aqueles} dias, disse Moisés aos filhos de Israel: «Ouvi, ó Israel, o que hoje vos ordeno: Quando enfiardes na terra de que o Senhor, vosso Deus, vos dará posse; quando a tiverdes conquistado e nela habitardes, tomareis as primícias de todos vossos frutos e, colocando-os em um cesto, ireis ao lugar que o Senhor, vosso Deus, tiver escolhido para ser honrado, e apresentá-lo-eis ao sacerdote que estiver em funções naquele dia, dizendo: «Declaro, hoje, perante o Senhor Deus, que Ele nos ouviu, atendeu à nossa miséria, trabalho e opressão e nos fez sair do Egipto com sua mão forte e pela força do seu braço, espalhando grande terror entre os povos com muitos milagres e prodígios, conduzindo-nos a este lugar e dando-nos esta terra, onde correm leite e mel. Por isso, agora, Lhe ofereço as primícias da terra, que me deu». Então colocareis as primícias diante do Senhor Deus e O adorareis. Depois preparareis um banquete com todas as coisas boas que o Senhor, vosso Deus, vos houver dado».
}\end{paracol}

\begin{paracol}{2}\latim{
Allelúja. ℣. \emph{Act. 2, 1} Cum compleréntur dies Pentecóstes, erant omnes páriter sedéntes.
}\switchcolumn\portugues{
Aleluia. ℣. \emph{Act. 2, 1} Tendo já chegado o dia de Pentecostes, estavam todos reunidos em assembleia.
}\end{paracol}

\paragraph{Oração}
\begin{paracol}{2}\latim{
\rlettrine{P}{ræsta,} quǽsumus, omnípotens Deus: ut, salutáribus jejúniis erudíti, ab ómnibus etiam vítiis abstinéntes, propitiatiónem tuam facílius impetrémus. Per Dóminum nostrum \emph{\&c.}
}\switchcolumn\portugues{
\slettrine{Ó}{} Deus omnipotente, Vos suplicamos, fazei que sejamos instruídos com estes salutares jejuns, e que nos abstenhamos de todos os vícios, a fim de que mais facilmente alcancemos o vosso perdão. Por nosso Senhor \emph{\&c.}
}\end{paracol}

\paragraphinfo{4.ª Lição}{Lv. 26, 3-12}
\begin{paracol}{2}\latim{
Léctio libri Levítici.
}\switchcolumn\portugues{
Lição do Livro Levítico.
}\switchcolumn*\latim{
\rlettrine{I}{n} diébus illis: Dixit Dóminus ad Móysen: Lóquere fíliis Israël, et dices ad eos: Si in præcéptis meis ambulavéritis, et mandáta mea custodiéritis et fecéritis ea, dabo vobis plúvias tempóribus suis, et terra gignet germen suum, et pomis árbores replebúntur. Apprehéndet méssium tritúra vindémiam, et vindémia occupábit seméntem: et comedétis panem vestrum in saturitáte, et absque pavóre habitábitis in terra vestra. Dabo pacem in fínibus vestris: dormiétis, et non erit, qui extérreat. Auferam malas béstias, et gládius non transíbit términos vestros. Persequémini inimícos vestros, et córruent coram vobis. Persequéntur quinque de vestris centum aliénos et centum de vobis decem mília: cadent inimíci vestri gládio in conspéctu vestro. Respíciam vos et créscere fáciam: multiplicábo mini, et firmábo pactum meum vobíscum. Comedétis vetustíssima véterum, et vétera novis superveniéntibus projiciétis. Ponam tabernáculum meum in médio vestri, et non abjíciet vos ánima mea. Ambulábo inter vos, et ero Deus vester, vosque éritis pópulus meus: dicit Dóminus omnípotens.
}\switchcolumn\portugues{
\rlettrine{N}{aqueles} dias, disse o Senhor a Moisés: «Fala aos filhos de Israel e diz-lhes: «Se seguirdes a vossa vida, segundo os meus preceitos e observando os meus mandamentos, dar-vos-ei chuvas na estação própria, produzindo a terra frutos com abundância, assim como as árvores. A colheita do trigo prolongar-se-á até à vindima, e a vindima até à sementeira. Comereis pão com fartura e habitareis sem temor na vossa terra. Conceder-vos-ei a paz dentro das vossas fronteiras e dormireis sem que ninguém vos perturbe. Afujentarei os animais ferozes e a espada não atravessará as vossas fronteiras. Perseguirei os vossos inimigos, que cairão diante de vós. Cinco de vós terão força para perseguir cem estrangeiros; cem de vós perseguirão dez mil; os vossos inimigos cairão diante de vós, sob o peso das espadas. Olharei por vós e vos farei crescer. Multiplicar-vos-ei e farei aliança convosco. Comereis os frutos das colheitas atrasadas e lançareis fora os frutos antigos por causa da abundância dos novos. Estabelecerei a minha morada no meio de vós. Andarei no meio de vós, serei o vosso Deus e sereis o meu povo»: diz o Senhor omnipotente.
}\end{paracol}

\begin{paracol}{2}\latim{
Allelúja. \emph{(Hic genuflectitur)} ℣. Veni, Sancte Spíritus, reple tuórum corda fidélium: et tui amóris in eis ignem accénde.
}\switchcolumn\portugues{
Aleluia. \emph{(Genuflecte-se)} ℣. Vinde, ó Espírito Santo: enchei os corações dos vossos fiéis e acendei neles o fogo do vosso amor.
}\end{paracol}

\paragraph{Oração}
\begin{paracol}{2}\latim{
\rlettrine{P}{ræsta,} quǽsumus, omnípotens Deus: sic nos ab épulis carnálibus abstinére; ut a vítiis irruéntibus páriter jejunémus. Per Dóminum \emph{\&c.}
}\switchcolumn\portugues{
\rlettrine{C}{oncedei-nos,} Senhor omnipotente, Vos rogamos, que nos abstenhamos de tal modo do uso das carnes, que nos afastemos também dos vícios que nos perseguem. Por nosso Senhor \emph{\&c.}
}\end{paracol}

\paragraphinfo{5.ª Lição}{Dn. 3, 47-51}
\begin{paracol}{2}\latim{
Léctio Daniélis Prophétæ.
}\switchcolumn\portugues{
Lição do Profeta Daniel.
}\switchcolumn*\latim{
\rlettrine{I}{n} diébus illis: Angelus Dómini descéndit cum Azaría et sóciis ejus in fornácem: et excússit flammam ignis de fornáce, et fecit médium fornácis quasi ventum roris flantem. Flamma autem effundebátur super fornácem cúbitis quadragínta novem: et erúpit, et incéndit, quos répperit juxta fornácem de
Chaldǽis, minístros regis, qui eam incendébant. Et non tétigit eos omníno ignis, neque contristávit, nec quidquam moléstiæ íntulit. Tunc hi tres quasi ex uno ore laudábant, et glorificábant, et benedicébant Deum in fornáce, dicéntes:
}\switchcolumn\portugues{
\rlettrine{N}{aqueles} dias, o Anjo do Senhor desceu à fornalha com Azarias e os seus companheiros e afastou da fornalha as chamas do fogo, soprando no meio delas como que um vento de orvalho. As chamas do fogo, porém, cresceram acima da fornalha quarenta e nove côvados; e, saindo fora dela, queimaram os Caldeus, ministros do rei, que estavam perto da fornalha a atiçar o fogo, e não queimaram nenhum dos três jovens, nem os feriram, nem lhes causaram qualquer incómodo! Então, estes três jovens louvavam, glorificavam e bendiziam Deus na fornalha, em voz uníssona, dizendo:
}\end{paracol}

\begin{paracol}{2}\latim{
Allelúja. ℣. \emph{ibid., 52} Benedíctus es, Dómine, Deus patrum nostrórum, et laudábilis in sǽcula.
}\switchcolumn\portugues{
Aleluia. ℣. \emph{ibid., 52} Bendito sois, Senhor, Deus de nossos pais, e digno de louvor em todos os séculos.
}\end{paracol}

\textit{Diz-se o Glória in excélsis... Depois:}

\begin{paracol}{2}\latim{
℣. Dóminus vobíscum.
}\switchcolumn\portugues{
℣. O Senhor seja convosco.
}\switchcolumn*\latim{
℞. Et cum spíritu tuo.
}\switchcolumn\portugues{
℞. E com vosso espírito.
}\end{paracol}

\paragraph{Oração}
\begin{paracol}{2}\latim{
\rlettrine{D}{eus,} qui tribus púeris mitigásti flammas ígnium: concéde propítius; ut nos fámulos tuos non exúrat flamma vitiórum. Per Dóminum \emph{\&c.}
}\switchcolumn\portugues{
\slettrine{Ó}{} Deus, que mitigastes as chamas do fogo aos três jovens, concedei-nos misericordiosamente que nós, vosso servos, não sejamos queimados pelas chamas dos vícios. Por nosso Senhor \emph{\&c.}
}\end{paracol}

\paragraphinfo{Epístola}{Rm. 5, 1-5}
\begin{paracol}{2}\latim{
Léctio Epístolæ beáti Pauli Apóstoli ad Romános.
}\switchcolumn\portugues{
Lição da Ep.ª do B. Ap.º Paulo aos Romanos.
}\switchcolumn*\latim{
\rlettrine{F}{ratres:} Justificáti ex fide, pacem habeámus ad Deum per Dóminum nostrum Jesum Christum: per quem et habémus accéssum per fidem in grátiam istam, in qua stamus, et gloriámur in spe glóriæ filiórum Dei. Non solum autem, sed et gloriámur in tribulatiónibus: scientes, quod tribulátio patiéntiam operátur, patiéntia autem probatiónem, probátio vero spem, spes autem non confúndit: quia cáritas Dei diffúsa est in córdibus nostris per Spíritum Sanctum, qui datus est nobis.
}\switchcolumn\portugues{
\rlettrine{M}{eus} irmãos: Estando já justificados pela fé, tenhamos paz com Deus por nosso Senhor Jesus Cristo, por quem devemos ter acesso pela fé a esta graça, na qual permaneceremos firmes e alegres na esperança de alcançarmos a glória dos filhos de Deus. E não somente na esperança, mas também devemos ter glória nas tribulações, sabendo que a tribulação produz a paciência, a paciência produz a provação, e a provação a esperança. Ora a esperança nunca engana, porque o amor de Deus está espalhado nos nossos corações pelo Espírito Santo, que nos foi dado.
}\end{paracol}

\paragraphinfo{Trato}{Sl. 116, 1-2}
\begin{paracol}{2}\latim{
\rlettrine{L}{audáte} Dóminum, omnes gentes: et collaudáte eum, omnes pópuli. ℣. Quóniam confirmáta est super nos misericórdia ejus: et véritas Dómini manet in ætérnum.
}\switchcolumn\portugues{
\slettrine{Ó}{} nações, louvai todas o Senhor: glorificai-O, ó povos. ℣. Porquanto é grande a sua misericórdia para connosco: e a verdade do Senhor permanece para sempre.
}\end{paracol}

\paragraphinfo{Evangelho}{Lc. 4, 38-44}
\begin{paracol}{2}\latim{
\cruz Sequéntia sancti Evangélii secúndum Lucam.
}\switchcolumn\portugues{
\cruz Continuação do santo Evangelho segundo S. Lucas.
}\switchcolumn*\latim{
\blettrine{I}{n} illo témpore: Surgens Jesus de synagóga, introívit in domum Simónis. Socrus autem Simónis tenebátur magnis fébribus: et rogavérunt illum pro ea. Et stans super illam, imperávit febri: et dimísit illam. Et contínuo surgens, ministrábat illis. Cum autem sol occidísset, omnes qui habébant infírmos váriis languóribus, ducébant illos ad eum. At ille síngulis manus impónens, curábat eos. Exíbant autem dæmónia a multis clamántia et dicéntia: Quia tu es Fílius Dei: et íncrepans non sinébat ea loqui, quia sciébant ipsum esse Christum. Facta autem die egréssus ibat in desértum locum, et turbæ requirébant eum, et venérunt usque ad ipsum: et detinébant illum, ne discéderet ab eis. Quibus ille ait: Quia et áliis civitátibus opórtet me evangelizáre regnum Dei: quia ideo missus sum. Et erat prǽdicans in synagógis Galilǽæ.
}\switchcolumn\portugues{
\blettrine{N}{aquele} tempo, tendo saído Jesus da sinagoga, entrou em casa de Simão. Ora, como a sogra de Simão padecesse duma grande febre, rogaram-Lhe por ela. Indo Ele ao pé dela, impôs o seu poder à febre, que logo deixou a doente, levantando-se e servindo-os. Sendo já sol posto, todos os que tinham enfermos, atacados de diversas doenças, traziam-Lhos. E Jesus, impondo as mãos sobre cada um dos doentes, curava-os. De muitos saíam também demónios, gritando e dizendo: «Sois o Filho de Deus!». Ele repreendia-os, e os não deixava dizer que sabiam que Jesus era o Cristo. Sendo já de manhã, saiu e foi para um lugar deserto. As turbas procuravam-n’O, vieram até onde Ele estava e detiveram-n’O, com medo de que se afastasse. Jesus disse-lhes: «É preciso que anuncie também o reino de Deus às outras cidades, porque para isso fui mandado». E pregava nas sinagogas da Galileia.
}\end{paracol}

\paragraphinfo{Ofertório}{Sl. 87, 2-3}
\begin{paracol}{2}\latim{
\rlettrine{D}{ómine,} Deus salútis meæ, in die clamávi et nocte coram te: intret orátio mea in conspéctu tuo, Dómine, allelúja.
}\switchcolumn\portugues{
\rlettrine{S}{enhor} e Deus, que sois o meu Salvador, dia e noite tenha clamado por Vós! Que minha prece chegue à vossa presença, aleluia!
}\end{paracol}

\paragraph{Secreta}
\begin{paracol}{2}\latim{
\rlettrine{U}{t} accépta tibi sint, Dómine, nostra jejúnia: præsta nobis, quǽsumus; hujus múnere sacraménti purificátum tibi pectus offérre. Per Dóminum \emph{\&c.}
}\switchcolumn\portugues{
\rlettrine{A}{} fim de que os nossos jejuns Vos sejam agradáveis, Senhor, concedei-nos, Vos suplicarmos, que possamos oferecer-Vos o nosso coração já purificado pela virtude deste sacrifício. Por nosso Senhor \emph{\&c.}
}\end{paracol}

\paragraphinfo{Comúnio}{Jo. 3, 8}
\begin{paracol}{2}\latim{
\rlettrine{S}{píritus,} ubi vult, spirat: et vocem ejus audis, allelúja, allelúja: sed nescis, unde véniat aut quo vadat, allelúja, allelúja, allelúja.
}\switchcolumn\portugues{
\rlettrine{O}{} Espírito sopra onde quer: e ouvis a sua voz, aleluia, aleluia: mas ignorais donde Ele Vem e para onde vai, aleluia, aleluia, aleluia.
}\end{paracol}

\paragraph{Postcomúnio}
\begin{paracol}{2}\latim{
\rlettrine{P}{rǽbeant} nobis, Dómine, divínum tua sancta fervórem: quo eórum páriter et actu delectémur et fructu. Per Dóminum \emph{\&c.}
}\switchcolumn\portugues{
\qlettrine{Q}{ue} os vossos divinos mystérios. Senhor, nos inspirem um fervor divino, a fina de que sua celebração e os seus frutos sejam as nossas delícias. Por nosso Senhor \emph{\&c.}
}\end{paracol}

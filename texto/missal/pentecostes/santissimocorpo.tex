\subsectioninfo{Santíssimo Corpo de Cristo}{Dia Festivo de Preceito}\label{santissimocorpocristo}

\paragraphinfo{Intróito}{Sl. 80, 17}
\begin{paracol}{2}\latim{
\rlettrine{C}{ibávit} eos ex ádipe fruménti, allelúja: et de petra, melle saturávit eos, allelúja, allelúja, allelúja. \emph{Ps. ib., 2} Exsultáte Deo, adjutóri nostro: jubiláte Deo Jacob.
℣. Gloria Patri \emph{\&c.}
}\switchcolumn\portugues{
\rlettrine{O}{} Senhor alimentou-os com a flor da farinha, aleluia: e saciou-os com o mel tirado do rochedo, aleluia, aleluia, aleluia. \emph{Sl. ib., 2} Exultai de alegria e louvai a Deus, que é o nosso sustentáculo: Aclamai com júbilo Deus de Jacob.
℣. Glória ao Pai \emph{\&c.}
}\end{paracol}

\paragraph{Oração}
\begin{paracol}{2}\latim{
\rlettrine{D}{eus,} qui nobis sub Sacraménto mirábili passiónis tuæ memóriam reliquísti: tríbue, quǽsumus, ita nos Córporis et Sánguinis tui sacra mystéria venerári; ut redemptiónis tuæ fructum in nobis júgiter sentiámus: Qui vivis et regnas \emph{\&c.}
}\switchcolumn\portugues{
\slettrine{Ó}{} Deus, que neste admirável sacramento nos deixastes o memorial da vossa Paixão, concedei-nos, Vos pedimos, que de tal sorte veneremos os sacratíssimos mystérios do vosso Corpo e Sangue, que sintamos perpetuamente no nosso íntimo o fruto da vossa Redenção. Vós, que, viveis e reinais \emph{\&c.}
}\end{paracol}

\paragraphinfo{Epístola}{1. Cor. 11, 23-29}
\begin{paracol}{2}\latim{
Léctio Epistolæ beáti Pauli Apóstoli ad Corinthios.
}\switchcolumn\portugues{
Lição da Ep.ª do B. Ap.º Paulo aos Coríntios.
}\switchcolumn*\latim{
\rlettrine{F}{ratres:} Ego enim accépi a Dómino quod et trádidí vobis, quóniam Dóminus Jesus, in qua nocte tradebátur, accépit panem, et grátias agens fregit, et dixit: Accípite, et manducáte: hoc est corpus meum, quod pro vobis tradétur: hoc fácite in meam commemoratiónem. Simíliter ei cálicem, postquam cenávit, dicens: Hic calix novum Testaméntum est in meo sánguine. Hoc fácite, quotiescúmque bibétis, in meam commemoratiónem. Quotiescúmque enim manducábitis panem hunc et cálicem bibétis, mortem Dómini annuntiábitis, donec véniat. Itaque quicúmque manducáverit panem hunc vel bíberit cálicem Dómini indígne, reus erit córporis et sánguinis Dómini. Probet autem seípsum homo: et sic de pane illo e dat et de calice bibat. Qui enim mánducat et bibit indígne, judícium sibi mánducat et bibit: non dijúdicans corpus Dómini.
}\switchcolumn\portugues{
\rlettrine{M}{eus} irmãos: Foi com o Senhor que aprendi aquilo que vos ensinei: Que o Senhor Jesus, naquela mesma noite em que foi traído, tomou o pão, e, dando graças, partiu-o e disse: «Tomai e comei: isto é o meu corpo, que será entregue por amor de vós. Fazei isto em minha memória». E, de modo semelhante, após a ceia, tomou também o cálice, dizendo: «Este cálice é a nova aliança no meu sangue. Fazei isto todas as vezes que o beberdes, em minha memória; pois todas as vezes que comerdes este pão e que beberdes este cálice, anunciareis a morte do Senhor, até que Ele venha». Assim, todo aquele que comer o pão ou beber o cálice do Senhor indignamente, será réu do Corpo e do Sangue do Senhor. Examine-se, pois, o homem a si mesmo; e, depois, coma deste pão e beba deste cálice; pois o que come e bebe indignamente, come e bebe a sua própria condenação, não distinguindo o Corpo do Senhor.
}\end{paracol}

\paragraphinfo{Gradual}{Sl. 144, 15-16}
\begin{paracol}{2}\latim{
\rlettrine{O}{culi} ómnium in te sperant, Dómine: et tu das illis escam in témpore opportúno. ℣. Aperis tu manum tuam: et imples omne animal benedictióne.
}\switchcolumn\portugues{
\rlettrine{T}{odos} os olhos, Senhor, estão voltados para Vós, cheios de esperança: Pois dais a cada um, oportunamente, o seu sustento. ℣. Abris as vossas mãos e saciais todos os viventes com vossas bênçãos.
}\switchcolumn*\latim{
Allelúja, allelúja. ℣. \emph{Joann. 6, 56-57} Caro mea vere est cibus, et sanguis meus vere est potus: qui mandúcat meam carnem et bibit meum sánguinem, in me manet et ego in eo.
}\switchcolumn\portugues{
Aleluia, aleluia. ℣. \emph{Jo. 6, 56-57} Minha Carne é verdadeira comida; e o meu Sangue é verdadeira bebida. Aquele, pois, que come a minha carne e bebe o meu sangue permanece em mim e Eu nele.
}\end{paracol}

\paragraphinfo{Sequência}{St. Tomás Aquino}\label{laudasionsalvatorem}
\begin{paracol}{2}\latim{
\rlettrine{L}{áuda} Síon Salvatórem, Láuda dúcem et pastórem, In hýmnis et cánticis.
}\switchcolumn\portugues{
\rlettrine{L}{ouva,}ó Sião, louva o teu Salvador! Louva com hinos e cânticos o teu Principe e Pastor!
}\switchcolumn*\latim{
Quantum pótes, tantum áude: Quia májor ómni láude, Nec laudáre súfficis.
}\switchcolumn\portugues{
Louva-O tanto quanto possas: nem tu podes louvá-l’O dignamente, pois está acima de todos os louvores.
}\switchcolumn*\latim{
Láudis théma speciális, Pánis vívus et vitális Hódie propónitur.
}\switchcolumn\portugues{
O assunto especial dos teus louvores, hoje, é o Pão vivo e vivificante:
}\switchcolumn*\latim{
Quem in sácræ ménsa coénæ Túrbæ frátrum duodénæ Dátum non ambígitur.
}\switchcolumn\portugues{
É aquele mesmo Pão (nós o acreditamos) que foi dado ao grupo dos Doze Irmãos na Última Ceia.
}\switchcolumn*\latim{
Sit laus pléna, sít sonóra, Sit jucúnda, sit decóra Méntis jubilátio.
}\switchcolumn\portugues{
Que o louvor seja perene, sonoro e melodioso: que seja agradável e belo, como a alegria que transporta as nossas almas!
}\switchcolumn*\latim{
Díes enim solémnis ágitur, In qua ménsæ príma recólitur Hújus institútio.
}\switchcolumn\portugues{
Eis o dia solene, que nos recorda a primitiva instituição deste Banquete divino.
}\switchcolumn*\latim{
In hac ménsa nóvi Régis, Nóvum Páscha nóvæ légis, Pháse vétus términat.
}\switchcolumn\portugues{
Nesta mesa do novo Rei a Páscoa da nova Lei acaba com a páscoa antiga.
}\switchcolumn*\latim{
Vetustátem nóvitas, Umbram fúgat véritas, Nóctem lux elíminat.
}\switchcolumn\portugues{
O rito antigo cede o lugar ao novo, como a imagem desaparece diante da realidade, e a luz apaga a noite.
}\switchcolumn*\latim{
Quod in coéna Chrístus géssit, Faciéndum hoc expréssit In súi memóriam.
}\switchcolumn\portugues{
Aquilo que Cristo praticou na Ceia, mandou que fizéssemos, também, em sua memória.
}\switchcolumn*\latim{
Dócti sácris institútis, Pánem, vínum, in salútis Consecrámus hóstiam.
}\switchcolumn\portugues{
E nós, instruídos pelo mandato divino, consagrámos o pão e o vinho em hóstia de salvação.
}\switchcolumn*\latim{
Dógma dátur christiánis, Quod in cárnem tránsit pánis, Et vínum in sánguinem.
}\switchcolumn\portugues{
É um dogma de fé para os cristãos que o pão passa para Carne de Cristo e o vinho para seu Sangue.
}\switchcolumn*\latim{
Quod non cápis, quod non vídes, Animósa fírmat fídes, Præter rérum órdinem.
}\switchcolumn\portugues{
Aquilo que não compreendeis, nem vedes, a fé viva o afirma sem alterar a ordem da natureza.
}\switchcolumn*\latim{
Sub divérsis speciébus, Sígnis tantum, et non rébus, Látent res exímiæ.
}\switchcolumn\portugues{
Debaixo de diversas espécies, distintas somente por sinais exteriores, ocultam-se sublimes realidades.
}\switchcolumn*\latim{
Cáro cíbus, sánguis pótus: Mánet tamen Chrístus tótus, Sub utráque spécie.
}\switchcolumn\portugues{
A Carne de Cristo é alimento, e o Sangue bebida; mas Ele existe inteiro em cada uma das espécies.
}\switchcolumn*\latim{
A suménte non concísus, Non confráctus, non divísus: Integer accípitur.
}\switchcolumn\portugues{
Quem O recebe, nem O parte, nem O corta, nem O divide; porém recebe-O inteiro.
}\switchcolumn*\latim{
Súmit únus, súmunt mille: Quantum ísti, tantum ílle: Nec súmptus consúmitur.
}\switchcolumn\portugues{
Quem O receba uma só pessoa, quer O recebam mil, todas recebem o mesmo: recebem-n’O sem O consumirem.
}\switchcolumn*\latim{
Súmunt bóni, súmunt máli: Sórte tamen inæquáli, Vítæ vel intéritus. Mors est mális, víta bónis:
}\switchcolumn\portugues{
Recebem-n’O bons e maus; porém com efeitos diferentes: uns encontram vida; outros a morte! Para os maus é morte e para os bons é vida.
}\switchcolumn*\latim{
Víde páris sumptiónis Quam sit díspar éxitus. Frácto demum sacraménto,
}\switchcolumn\portugues{
Vede que diferentes são os efeitos que produz o mesmo alimento!
}\switchcolumn*\latim{
Ne vacílles, sed meménto Tantum ésse sub fragménto, Quantum tóto tégitur.
}\switchcolumn\portugues{
Que a vossa fé não vacile quando a hóstia é dividida; mas lembrai-vos de que Jesus tanto está no fragmento, como na hóstia inteira.
}\switchcolumn*\latim{
Núlla réi fit scissúra: Sígni tantum fit fractúra, Qua nec státus, nec statúra Signáti minúitur.
}\switchcolumn\portugues{
A substância não é dividida: somente o sinal é que é partido, mas sem diminuição, nem no estado, nem na grandeza do que está sob as espécies.
}\switchcolumn*\latim{
Ecce Pánis Angelórum, Fáctus cíbus viatórum
}\switchcolumn\portugues{
Eis o Pão dos Anjos que se fez alimento dos homens viadores,
}\switchcolumn*\latim{
Vere pánis filiórum, Non mitténdus cánibus.
}\switchcolumn\portugues{
Verdadeiro pão dos inocentes, que não deve ser dado aos cães!
}\switchcolumn*\latim{
In figúris præsignátur, Cum Isáac immolátur, Agnus Páschæ deputátur, Dátur mánna pátribus.
}\switchcolumn\portugues{
Antigamente foi representado por figuras: imolado com Isaque e significado no cordeiro pascal e no maná do deserto.
}\switchcolumn*\latim{
Bóne pástor, pánis vére, Jésu, nóstri miserére: Tu nos pásce, nos tuére, Tu nos bóna fac vidére In térra vivéntium.
}\switchcolumn\portugues{
Ó bom Pastor, ó Pão verdadeiro, ó Jesus, tende piedade de nós: alimentai-nos, defendei-nos do mal e permiti que gozemos os verdadeiros bens da terra dos vivos.
}\switchcolumn*\latim{
Tu qui cúncta scis et váles, Qui nos páscis hic mortáles: Túos ibi commensáles, Coherédes et sodáles Fac sanctórum cívium. ℟. Amen.
}\switchcolumn\portugues{
Ó Vós, que tudo conheceis e podeis: ó Vós, que nos alimentais nesta vida mortal, tornai-nos co-herdeiros e companheiros dos habitantes da cidade celestial. ℟. Amen.
}\end{paracol}

\paragraphinfo{Evangelho}{Jo. 6, 56-59}
\begin{paracol}{2}\latim{
\cruz Sequéntia sancti Evangéli secúndum Joánnem
}\switchcolumn\portugues{
\cruz Continuação do santo Evangelho segundo S. João.
}\switchcolumn*\latim{
\blettrine{I}{n} illo témpore: Dixit Jesus turbis Judæórum: Caro mea vere est cibus et sanguis meus vere est potus. Qui mandúcat meam carnem e bibit meum sánguinem, in me manet et ego in illo. Sicu misit me vivens Pater, et ego vivo propter Patrem: et qu mandúcat me, et ipse vivet propter me. Hic est panis, qu de cœlo descéndit. Non sicu manducavérunt patres vestri manna, et mórtui sunt. Qu manducat hunc panem, vivet in ætérnum.
}\switchcolumn\portugues{
\blettrine{N}{aquele} tempo, disse Jesus aos judeus: «Minha Carne é verdadeira comida e o meu Sangue é verdadeira bebida. Quem come a minha Carne e bebe o meu Sangue permanece em mim e Eu nele. Assim como meu Pai, que é fonte de vida e me mandou, e Eu vivo de meu Pai, assim também aquele que me comer viverá de mim. Este é o pão que desceu do céu. Porém não é como o maná, que os vossos pais comeram havendo morrido mais tarde. Quem comer este pão viverá eternamente».
}\end{paracol}

\paragraphinfo{Ofertório}{Lv. 21, 6}
\begin{paracol}{2}\latim{
\rlettrine{S}{acerdótes} Dómini incénsum it panes ófferunt Deo: et deo sancti erunt Deo suo, et lon pólluent nomen ejus, allelúja.
}\switchcolumn\portugues{
\rlettrine{O}{s} sacerdotes do Senhor oferecem a Deus incenso e pães: eis porque serão santos diante de Deus e não profanarão o seu nome, aleluia.
}\end{paracol}

\paragraph{Secreta}
\begin{paracol}{2}\latim{
\rlettrine{E}{cclésiæ} tuæ, quǽsumus, Dómine, unitátis et pacis propítius dona concéde: quæ sub oblátis munéribus mýstice designántur. Per Dóminum nostrum \emph{\&c.}
}\switchcolumn\portugues{
\rlettrine{S}{enhor,} Vos imploramos, dignai-Vos conceder à vossa Igreja os dons da unidade e da paz, que misticamente se figuram nas ofertas que Vos apresentamos. Por nosso Senhor \emph{\&c.}
}\end{paracol}

\paragraphinfo{Comúnio}{1. Cor. 11, 26-27}
\begin{paracol}{2}\latim{
\qlettrine{Q}{uotiescúmque} manducábitis panem hunc et cálicem bibétis, mortem Dómini annuntiábitis, donec véniat: itaque quicúmque manducáverit panem vel bíberit calicem Dómini indígne, reus erit córporis et sánguinis Dómini, allelúja.
}\switchcolumn\portugues{
\rlettrine{T}{odas} as vezes que comerdes este Pão e beberdes este Cálice anunciareis a morte do Senhor, até que Ele venha. Assim, todo aquele que comer este Pão ou beber este Cálice indignamente será réu do Corpo e do Sangue do Senhor.
}\end{paracol}

\paragraph{Postcomúnio}
\begin{paracol}{2}\latim{
\rlettrine{F}{ac} nos, quǽsumus, Dómine, divinitátis tuæ sempitérna fruitióne repléri: quam pretiósi Corporis et Sanguinis tui temporalis percéptio præfigúrat: Qui vivis \emph{\&c.}
}\switchcolumn\portugues{
\rlettrine{F}{azei,} Senhor, Vos imploramos, que alcancemos eternamente a vossa divindade na glória eterna, que é figurada neste mundo pela recepção temporal do vosso Corpo e Sangue preciosíssimos. Ó Vós, que viveis e \emph{\&c.}
}\end{paracol}

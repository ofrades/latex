\subsection{Oitavo Domingo depois de Pentecostes}

\paragraphinfo{Intróito}{Sl. 47, 10-11}
\begin{paracol}{2}\latim{
\rlettrine{S}{uscépimus,} Deus, misericórdiam tuam in médio templi tui: secúndum nomen tuum, Deus, ita et laus tua in fines terræ: justítia plena est déxtera tua. \emph{Ps. ibid., 2} Magnus Dóminus, et laudábilis nimis: in civitate Dei nostri, in monte sancto ejus.
℣. Gloria Patri \emph{\&c.}
}\switchcolumn\portugues{
\rlettrine{A}{lcançámos,} ó Deus, a vossa misericórdia no meio do vosso templo. Como o vosso nome, assim a vossa glória, Senhor, se estende até aos confins da terra: e a vossa dextra está cheia de justiça. \emph{Sl. ibid., 2} O Senhor é grande e digno de todos os louvores, tanto na cidade de Deus, como no seu monte santo.
℣. Glória ao Pai \emph{\&c.}
}\end{paracol}

\paragraph{Oração}
\begin{paracol}{2}\latim{
\rlettrine{L}{argíre} nobis, quǽsumus, Dómine, semper spíritum cogitándi quæ recta sunt, propítius et agéndi: ut, qui sine te esse non póssumus, secúndum te vívere valeámus. Per Dóminum \emph{\&c.}
}\switchcolumn\portugues{
\rlettrine{C}{oncedei-nos} propício, Senhor, Vos imploramos, a graça de pensar e de praticar sempre segundo a justiça, a fim de que, não podendo nós existir sem Vós, conformemos sempre a nossa vida com vossa vontade. Por nosso Senhor \emph{\&c.}
}\end{paracol}

\paragraphinfo{Epístola}{Rm. 8, 12-17}
\begin{paracol}{2}\latim{
Léctio Epístolæ beáti Pauli Apóstoli ad Romános.
}\switchcolumn\portugues{
Lição da Ep.ª do B. Ap.º Paulo aos Romanos.
}\switchcolumn*\latim{
\rlettrine{F}{ratres:} Debitóres sumus non carni, ut secúndum carnem vivámus. Si enim secúndum carnem vixéritis, moriémini: si autem spíritu facta carnis mortificavéritis, vivétis. Quicúmque enim spíritu Dei aguntur, ii sunt fílii Dei. Non enim accepístis spíritum servitútis íterum in timóre, sed accepístis spíritum adoptiónis filiórum, in quo clamámus: Abba (Pater). Ipse enim Spíritus testimónium reddit spirítui nostro, quod sumus fílii Dei. Si autem fílii, et herédes: herédes quidem Dei, coherédes autem Christi.
}\switchcolumn\portugues{
\rlettrine{M}{eus} irmãos: Não somos devedores à carne para vivermos segundo a carne. Se, pois, viverdes segundo a carne, morrereis; mas se, pelo contrário, com o Espírito mortificardes as obras da carne, vivereis; porque todos os que são conduzidos pelo Espírito de Deus, são filhos de Deus. Com efeito, não recebestes o espírito de escravidão para vos conduzir pelo temor; mas o espírito de adopção de filhos, pelo qual chamamos: Abba! (Pai). Ora este Espírito dá testemunho ao nosso espírito de que somos filhos de Deus. Se, pois, somos filhos de Deus, somos também herdeiros: herdeiros verdadeiros de Deus e co-herdeiros de Jesus Cristo.
}\end{paracol}

\paragraphinfo{Gradual}{Sl. 30, 3}
\begin{paracol}{2}\latim{
\rlettrine{E}{sto} mihi in Deum protectórem, et in locum refúgii, ut salvum me fácias. ℣. \emph{Ps. 70, 1} Deus, in te sperávi: Dómine, non confúndar in ætérnum.
}\switchcolumn\portugues{
\rlettrine{S}{ede} para mim Deus protector: e um lugar de refúgio para me salvar. ℣. \emph{Sl. 70, 1} Ó Deus, em Vós pus a minha esperança: não serei para sempre confundido, Senhor.
}\switchcolumn*\latim{
Allelúja, allelúja. ℣. \emph{Ps. 47, 2} Magnus Dóminus, et laudábilis valde, in civitáte Dei nostri, in monte sancto ejus. Allelúja.
}\switchcolumn\portugues{
Aleluia, aleluia. ℣. \emph{Sl. 47, 2} O Senhor é grande e digno de todos os louvores, tanto na cidade de Deus, como no seu monte santo. Aleluia.
}\end{paracol}

\paragraphinfo{Evangelho}{Lc. 16, 1-9}
\begin{paracol}{2}\latim{
\cruz Sequéntia sancti Evangélii secúndum Lucam.
}\switchcolumn\portugues{
\cruz Continuação do santo Evangelho segundo S. Lucas.
}\switchcolumn*\latim{
\blettrine{I}{n} illo témpore: Dixit Jesus discípulis suis parábolam hanc: Homo quidam erat dives, qui habébat víllicum: et hic diffamátus est apud illum, quasi dissipásset bona ipsíus. Et vocávit illum et ait illi: Quid hoc audio de te? redde ratiónem villicatiónis tuæ: jam enim non póteris villicáre. Ait autem víllicus intra se: Quid fáciam, quia dóminus meus aufert a me villicatiónem? fódere non váleo, mendicáre erubésco. Scio, quid fáciam, ut, cum amótus fúero a villicatióne, recípiant me in domos suas. Convocátis itaque síngulis debitóribus dómini sui, dicébat primo: Quantum debes dómino meo? At ille dixit: Centum cados ólei. Dixítque illi: Accipe cautiónem tuam: et sede cito, scribe quinquagínta. Deínde álii dixit: Tu vero quantum debes? Qui ait: Centum coros trítici. Ait illi: Accipe lítteras tuas, et scribe octogínta. Et laudávit dóminus víllicum iniquitátis, quia prudénter fecísset: quia fílii hujus sǽculi prudentióres fíliis lucis in generatióne sua sunt. Et ego vobis dico: fácite vobis amicos de mammóna iniquitátis: ut, cum defecéritis, recípiant vos in ætérna tabernácula.
}\switchcolumn\portugues{
\blettrine{N}{aquele} tempo, disse Jesus aos seus discípulos esta parábola: Um homem rico tinha um feitor, que foi acusado diante dele de haver dissipado os seus bens. Então, chamou-o, dizendo-lhe: «Que é isto que ouço dizer de ti? Dá-me conta da tua gerência, pois desde hoje não continuarás a ser meu feitor». Este disse no seu íntimo: «Que será de mim, se o senhor me tira a gerência dos bens?! Pois não posso cultivar a terra e tenho vergonha de mendigar! Eu sei, porém, o que hei-de fazer, a fim de que, quando me seja tirado o emprego, encontre quem me receba em sua casa». Chamando, então, os devedores do senhor, disse ao primeiro: «Quanto deves ao meu senhor?». Ele respondeu: «Cem medidas de azeite». O feitor disse: «Aqui tens a tua obrigação; senta-te depressa e escreve cinquenta». Depois disse ao segundo: «E tu quanto deves?». Ele respondeu: «Cem medidas de trigo». «Toma a tua obrigação disse-lhe o feitor e escreve oitenta». E louvou o senhor o feitor infiel, por haver procedido prudentemente, porque os filhos do mundo são mais hábeis na conduta dos seus negócios do que os filhos da luz. Pois Eu vos digo, acrescentou Jesus: «Granjeai amigos com as riquezas da iniquidade, a fim de que, quando vos encontrardes com necessidade, vos recebam nas suas moradas eternas».
}\end{paracol}

\paragraphinfo{Ofertório}{Sl. 17, 28 \& 32}
\begin{paracol}{2}\latim{
\rlettrine{P}{ópulum} húmilem salvum fácies, Dómine, et óculos superbórum humiliábis: quóniam quis Deus præter te, Dómine?
}\switchcolumn\portugues{
\rlettrine{S}{alvareis,} Senhor, o povo humilde, e humilhareis os soberbos, pois quem é Deus senão Vós, Senhor?
}\end{paracol}

\paragraph{Secreta}
\begin{paracol}{2}\latim{
\rlettrine{S}{úscipe,} quǽsumus, Dómine, múnera, quæ tibi de tua largitáte deférimus: ut hæc sacrosáncta mystéria, grátiæ tuæ operánte virtúte, et præséntis vitæ nos conversatióne sanctíficent, et ad gáudia sempitérna perdúcant. Per Dóminum \emph{\&c.}
}\switchcolumn\portugues{
\rlettrine{A}{ceitai,} Senhor, Vos suplicamos, estes dons, que recebemos da vossa liberalidade, a fim de que pela eficácia da vossa graça estes sacrossantos mistérios nos santifiquem durante a vida presente e nos conduzam à posse das alegrias eternas. Por nosso Senhor \emph{\&c.}
}\end{paracol}

\paragraphinfo{Comúnio}{Sl. 33, 9}
\begin{paracol}{2}\latim{
\rlettrine{G}{ustáte} et vidéte, quóniam suávis est Dóminus: beátus vir, qui sperat in eo.
}\switchcolumn\portugues{
\rlettrine{P}{rovai} e vede como o Senhor é suave: Bem-aventurado o varão que confia n’Ele.
}\end{paracol}

\paragraph{Postcomúnio}
\begin{paracol}{2}\latim{
\rlettrine{S}{it} nobis, Dómine, reparátio mentis et córporis cæléste mystérium: ut, cujus exséquimur cultum, sentiámus efféctum. Per Dóminum \emph{\&c.}
}\switchcolumn\portugues{
\qlettrine{Q}{ue} este celestial mistério, Senhor, renove o nosso espírito e o nosso corpo» a fim de que sintamos os efeitos do sacramento que honrámos. Por nosso Senhor \emph{\&c.}
}\end{paracol}

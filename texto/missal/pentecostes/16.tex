\subsection{Décimo Sexto Domingo depois de Pentecostes}

\paragraphinfo{Intróito}{Sl. 85, 3 \& 5}
\begin{paracol}{2}\latim{
\rlettrine{M}{iserére} mihi, Dómine, quóniam ad te clamávi tota die: quia tu, Dómine, suávis ac mitis es, et copiósus in misericórdia ómnibus invocántibus te. \emph{Ps. ibid., 1} Inclína, Dómine, aurem tuam mihi, et exáudi me: quóniam inops, et pauper sum ego.
℣. Gloria Patri \emph{\&c.}
}\switchcolumn\portugues{
\rlettrine{T}{ende} misericórdia de mim, Senhor, pois por Vós clamei todo o dia, porquanto sois bom, clemente e compassivo para com aqueles que Vos invocam. \emph{Sl. ibid., 1} Inclinai vossos ouvidos para mim, Senhor, e ouvi-me: pois sou necessitado, infeliz e pobre.
℣. Glória ao Pai \emph{\&c.}
}\end{paracol}

\paragraph{Oração}
\begin{paracol}{2}\latim{
\rlettrine{T}{ua} nos, quǽsumus, Dómine, grátia semper et prævéniat et sequátur: ac bonis opéribus júgiter præstet esse inténtos. Per Dóminum \emph{\&c.}
}\switchcolumn\portugues{
\rlettrine{F}{azei,} Senhor, Vos suplicamos, que a vossa graça nos anime e acompanhe sempre, e que nos mantenha incessantemente na prática das boas obras. Por nosso Senhor \emph{\&c.}
}\end{paracol}

\paragraphinfo{Epístola}{Ef. 3, 13-21}
\begin{paracol}{2}\latim{
Léctio Epístolæ beáti Pauli Apóstoli ad Ephésios.
}\switchcolumn\portugues{
Lição da Ep.ª do B. Ap.º Paulo aos Efésios.
}\switchcolumn*\latim{
\rlettrine{F}{ratres:} Obsecro vos, ne deficiátis in tribulatiónibus meis pro vobis: quæ est glória vestra. Hujus rei grátia flecto génua mea ad Patrem Dómini nostri Jesu Christi, ex quo omnis patérnitas in cœlis et in terra nominátur, ut det vobis secúndum divítias glóriæ suæ, virtúte corroborári per Spíritum ejus in interiórem hóminem, Christum habitáre per fidem in córdibus vestris: in caritáte radicáti et fundáti, ut póssitis comprehéndere cum ómnibus sanctis, quæ sit latitúdo et longitúdo et sublímitas et profúndum: scire etiam supereminéntem sciéntiæ caritátem Christi, ut impleámini in omnem plenitúdinem Dei. Ei autem, qui potens est ómnia fácere superabundánter, quam pétimus aut intellégimus, secúndum virtútem, quæ operátur in nobis: ipsi glória in Ecclésia et in Christo Jesu, in omnes generatiónes sǽculi sæculórum. Amen.
}\switchcolumn\portugues{
\rlettrine{M}{eus} irmãos: Peço-vos que não desanimeis com as tribulações que suporto por vossa causa; pois isto constitui a vossa glória. Por causa disto ajoelho diante do Pai de N. S. Jesus Cristo, que é o princípio desta grande família no céu e na terra, para que segundo as riquezas da sua glória sejais fortificados pelo seu Espírito e revestidos com a graça de homens interiores, para que Jesus Cristo habite pela fé nos vossos corações, e que, enraizados e imbuídos na caridade, possais compreender com todos seus santos a largura, o comprimento, a altura e a profundidade deste mistério, e conhecer o amor de Cristo, o qual ultrapassa toda a ciência, a fim de que sejais repletos da plenitude dos dons de Deus. Que Aquele que é poderoso, e que, por isso, pode fazer infinitamente mais do que Lhe pedimos, ou mesmo pensamos, segundo a virtude que opera em nós, seja glorificado na Igreja e em Jesus Cristo, por todas as gerações e em todos os séculos dos séculos. Amen.
}\end{paracol}

\paragraphinfo{Gradual}{Sl. 101, 16-17}
\begin{paracol}{2}\latim{
\rlettrine{T}{imébunt} gentes nomen tuum, Dómine, et omnes reges terræ glóriam tuam. ℣. Quóniam ædificávit Dóminus Sion, et vidébitur in majestáte sua.
}\switchcolumn\portugues{
\rlettrine{A}{s} nações temerão o vosso nome, Senhor, e todos os reis da terra publicarão a vossa glória. ℣. Pois o Senhor edificou Sião, onde fará resplandecer a sua majestade.
}\switchcolumn*\latim{
Allelúja, allelúja. ℣. \emph{Ps. 97, 1} Cantáte Dómino cánticum novum: quia mirabília fecit Dóminus. Allelúja.
}\switchcolumn\portugues{
Aleluia, aleluia. ℣. \emph{Sl. 97, 1} Cantai em honra do Senhor um cântico, pois Ele operou maravilhas. Aleluia.
}\end{paracol}

\paragraphinfo{Evangelho}{Lc. 14, 1-11}
\begin{paracol}{2}\latim{
\cruz Sequéntia sancti Evangélii secúndum Lucam.
}\switchcolumn\portugues{
\cruz Continuação do santo Evangelho segundo S. Lucas.
}\switchcolumn*\latim{
\blettrine{I}{n} illo témpore: Cum intráret Jesus in domum cujúsdam príncipis pharisæórum sábbato manducáre panem, et ipsi observábant eum. Et ecce, homo quidam hydrópicus erat ante illum. Et respóndens Jesus dixit ad legisperítos et pharisǽos, dicens: Si licet sábbato curáre? At illi tacuérunt. Ipse vero apprehénsum sanávit eum ac dimísit. Et respóndens ad illos, dixit: Cujus vestrum ásinus aut bos in púteum cadet, et non contínuo éxtrahet illum die sábbati? Et non póterant ad hæc respóndere illi. Dicebat autem et ad invitátos parábolam, inténdens, quómodo primos accúbitus elígerent, dicens ad illos: Cum invitátus fúeris ad núptias, non discúmbas in primo loco, ne forte honorátior te sit invitátus ab illo, et véniens is, qui te et illum vocávit, dicat tibi: Da huic locum: et tunc incípias cum rubóre novíssimum locum tenére. Sed cum vocátus fúeris, vade, recúmbe in novíssimo loco: ut, cum vénerit, qui te invitávit, dicat tibi: Amíce, ascénde supérius. Tunc erit tibi glória coram simul discumbéntibus: quia omnis, qui se exáltat, humiliábitur: et qui se humíliat, exaltábitur.
}\switchcolumn\portugues{
\blettrine{N}{aquele} tempo, em um sábado, entrou Jesus em casa de um dos principais fariseus, para aí tomar uma refeição, observando os fariseus tudo o que Ele fazia, Estava diante dele um homem hidrópico, Então Jesus tomou a palavra e perguntou aos doutores da Lei e aos fariseus: «É permitido curar os enfermos ao sábado?». Eles se calaram. E Jesus tocou neste homem, curou-o e mandou-o embora. Depois, dirigindo-se aos outros, continuou: «Qual é de vós aquele que, se o seu boi ou o seu jumento cai em um poço no dia de sábado, o não retira logo, mesmo nesse dia?». Eles não puderam responder-Lhe. Ainda Jesus, notando que os convivas escolhiam os principais lugares à mesa, disse: «Quando vos convidarem para assistirdes às bodas, não procureis o primeiro lugar, pois pode acontecer que, tendo sido convidada outra pessoa de maior dignidade do que vós, aquele que vos convidou diga: «Cedei o vosso lugar a este convidado». E, então, tereis de ir, cheios de vergonha, ocupar o último lugar. Pelo contrário, ide ocupar o último lugar, de modo que aquele que vos houver convidado diga: «Amigo, vinde para lugar mais digno», o que será motivo de glória para vós, diante dos outros convivas. Aquele que se exaltar será humilhado; e aquele que se humilhar será exaltado».
}\end{paracol}

\paragraphinfo{Ofertório}{Sl. 39, 14 \& 15}
\begin{paracol}{2}\latim{
\rlettrine{D}{ómine,} in auxílium meum réspice: confundántur et revereántur, qui quærunt ánimam meam, ut áuferant eam: Dómine, in auxílium meum réspice.
}\switchcolumn\portugues{
\rlettrine{V}{olvei} para mim, Senhor, um olhar de protecção. Que fiquem confundidos e cheios de opróbrio aqueles que intentam tirar-me a vida. Senhor, lançai para mim um olhar de protecção!
}\end{paracol}

\paragraph{Secreta}
\begin{paracol}{2}\latim{
\rlettrine{M}{unda} nos, quǽsumus, Dómine, sacrifícii præséntis efféctu: et pérfice miserátus in nobis; ut ejus mereámur esse partícipes. Per Dóminum nostrum \emph{\&c.}
}\switchcolumn\portugues{
\rlettrine{P}{ela} virtude deste sacrifício, Senhor, Vos suplicamos, dignai-Vos purificar-nos; e, usando da vossa misericórdia para connosco, permiti que nos tornemos dignos de participar deste mesmo sacrifício. Por nosso Senhor \emph{\&c.}
}\end{paracol}

\paragraphinfo{Comúnio}{Sl. 70, 16-17 \& 18}
\begin{paracol}{2}\latim{
\rlettrine{D}{ómine,} memorábor justítiæ tuæ solíus: Deus, docuísti me a juventúte mea: et usque in senéctam et sénium, Deus, ne derelínquas me.
}\switchcolumn\portugues{
\rlettrine{S}{enhor,} serão meu único pensamento as obras da vossa justiça: instruístes-me, ó Deus, desde a minha juventude: e não me abandonareis até à minha velhice, até aos meus cabelos brancos, ó meu Deus.
}\end{paracol}

\paragraph{Postcomúnio}
\begin{paracol}{2}\latim{
\rlettrine{P}{urífica,} quǽsumus, Dómine, mentes nostras benígnus, et rénova cœléstibus sacraméntis: ut consequénter et córporum præsens páriter et futúrum capiámus auxílium. Per Dóminum \emph{\&c.}
}\switchcolumn\portugues{
\rlettrine{S}{enhor,} dignai-Vos benignamente purificar e renovar as nossas almas com vossos celestiais sacramentos, a fim de que neles os nossos corpos encontrem auxílio, tanto para a vida presente, como para a futura. Por nosso Senhor \emph{\&c.}
}\end{paracol}

\subsection{Décimo Quarto Domingo depois de Pentecostes}\label{14domingopentecostes}

\paragraphinfo{Intróito}{Sl. 83, 10-11}
\begin{paracol}{2}\latim{
\rlettrine{P}{rotéctor} noster, áspice, Deus, et réspice in fáciem Christi tui: quia mélior est dies una in átriis tuis super mília. \emph{Ps. ibid., 2-3} Quam dilécta tabernácula tua, Dómine virtútum! concupíscit, et déficit ánima mea in átria Dómini.
℣. Gloria Patri \emph{\&c.}
}\switchcolumn\portugues{
\slettrine{Ó}{} Deus, nosso protector, olhai para nós e fitai a face do vosso Cristo: É melhor passar um só dia nos vossos átrios do que mil em outros lugares. \emph{Sl. ibid., 2-3} Como são amáveis os vossos tabernáculos, Senhor dos exércitos! Minha alma suspira pelos átrios do Senhor e desfalece de saudade ao pensar neles.
℣. Glória ao Pai \emph{\&c.}
}\end{paracol}

\paragraph{Oração}
\begin{paracol}{2}\latim{
\rlettrine{C}{ustódi,} Dómine, quǽsumus, Ecclésiam tuam propitiatióne perpétua: et quia sine te lábitur humána mortálitas; tuis semper auxíliis et abstrahátur a nóxiis et ad salutária dirigátur. Per Dóminum \emph{\&c.}
}\switchcolumn\portugues{
\rlettrine{S}{enhor,} guardai misericordiosamente a vossa Igreja para sempre; e, visto que sem Vós a natureza humana, sujeita à morte, não pode subsistir, dignai-Vos com vosso perpétuo auxílio preservá-la de tudo o que lhe pode produzir algum mal e conduzi-la a tudo o que seja salutar à sua salvação. Por nosso Senhor \emph{\&c.}
}\end{paracol}

\paragraphinfo{Epístola}{Gl. 5, 16-24}
\begin{paracol}{2}\latim{
Léctio Epístolæ beáti Pauli Apóstoli ad Gálatas.
}\switchcolumn\portugues{
Lição da Ep.ª do B. Ap.º Paulo aos Gálatas.
}\switchcolumn*\latim{
\rlettrine{F}{ratres:} Spíritu ambuláte, et desidéria carnis non perficiétis. Caro enim concupíscit advérsus spíritum, spíritus autem advérsus carnem: hæc enim sibi ínvicem adversántur,
ut non quæcúmque vultis, illa faciátis. Quod si spíritu ducímini, non estis sub lege. Manifésta sunt autem ópera carnis, quæ sunt fornicátio, immundítia, impudicítia, luxúria, idolórum sérvitus, venefícia, inimicítiæ, contentiónes, æmulatiónes, iræ, rixæ, dissensiónes, sectæ, invídiæ, homicídia, ebrietátes, comessatiónes, et his simília: quæ prædíco vobis, sicut prædíxi: quóniam, qui talia agunt, regnum Dei non consequántur. Fructus autem Spíritus est: cáritas, gáudium, pax, patiéntia, benígnitas, bónitas, longanímitas, mansuetúdo, fides, modéstia, continéntia, cástitas. Advérsus hujúsmodi non est lex. Qui autem sunt Christi, carnem suam crucifixérunt cum vítiis et concupiscéntiis.
}\switchcolumn\portugues{
\rlettrine{M}{eus} irmãos: Andai guiados Pelo espírito, e não tereis os desejos da carne; Pois a carne tem desejos contrários aos do espírito, e o espírito tem desejos contrários aos da carne. São opostos entre si. Deste modo não podeis fazer tudo quanto desejais. Se obedecerdes ao espírito, não estareis sob a lei. Ora as obras da carne são bem manifestas, quais são: a cópula, a imundícia, a impudicícia, a luxúria, a idolatria, os envenenamentos, as inimizades, as contendas, as emulações, as iras, as discórdias, as discussões, as heresias, as invejas, os homicídios, a embriaguez, as comezanas e ainda outras coisas semelhantes. Eu vos declaro (como já vo-lo disse) que aqueles que cometem tais coisas não entrarão no reino de Deus. Porém, os frutos do espírito são os seguintes: a caridade, a alegria, a paz, a paciência, a mansidão, a bondade, a longanimidade, a doçura, a fé, a modéstia, a continência e a castidade. Contra estes frutos não há lei a opor. Aqueles que são de Cristo crucificaram a sua carne, as suas paixões e os seus desejos.
}\end{paracol}

\paragraphinfo{Gradual}{Sl. 117, 8-9}
\begin{paracol}{2}\latim{
\rlettrine{B}{onum} est confidére in Dómino, quam confidére in hómine. ℣. Bonum est speráre in Dómino, quam speráre in princípibus.
}\switchcolumn\portugues{
\slettrine{É}{} melhor confiar no Senhor do que nos homens. ℣. É melhor esperar em Deus do que nos príncipes.
}\switchcolumn*\latim{
Allelúja, allelúja. ℣. \emph{Ps. 94, 1} Veníte, exsultémus Dómino, jubilémus Deo, salutári nostro. Allelúja.
}\switchcolumn\portugues{
Aleluia, aleluia. ℣. \emph{Sl. 94, 1} Vinde, cantemos com alegria ao Senhor, exultemos jubilosamente em Deus, nosso Salvador. Aleluia.
}\end{paracol}

\paragraphinfo{Evangelho}{Mt. 6, 24-33}
\begin{paracol}{2}\latim{
\cruz Sequéntia sancti Evangélii secúndum Matthǽum.
}\switchcolumn\portugues{
\cruz Continuação do santo Evangelho segundo S. Mateus.
}\switchcolumn*\latim{
\blettrine{I}{n} illo témpore: Dixit Jesus discípulis suis: Nemo potest duóbus dóminis servíre: aut enim unum ódio habébit, et álterum díliget: aut unum sustinébit, et álterum contémnet. Non potéstis Deo servíre et mammónæ. Ideo dico vobis, ne sollíciti sitis ánimæ vestræ, quid manducétis, neque córpori vestro, quid induámini. Nonne ánima plus est quam esca: et corpus plus quam vestiméntum? Respícite volatília cœli, quóniam non serunt neque metunt neque cóngregant in hórrea: et Pater vester cœléstis pascit illa. Nonne vos magis pluris estis illis? Quis autem vestrum cógitans potest adjícere ad statúram suam cúbitum unum? Et de vestiménto quid sollíciti estis? Consideráte lília agri, quómodo crescunt: non labórant neque nent. Dico autem vobis, quóniam nec Sálomon in omni glória sua coopértus est sicut unum ex istis. Si autem fænum agri, quod hódie est et cras in clíbanum míttitur, Deus sic vestit: quanto magis vos módicæ fídei? Nolíte ergo sollíciti esse, dicéntes: Quid manducábimus aut quid bibémus aut quo operiémur? Hæc enim ómnia gentes inquírunt. Scit enim Pater vester, quia his ómnibus indigétis. Quǽrite ergo primum regnum Dei et justítiam ejus: et hæc ómnia adjiciéntur vobis.
}\switchcolumn\portugues{
\blettrine{N}{aquele} tempo, disse Jesus aos seus discípulos: «Ninguém pode servir a dois senhores, pois ou há-de odiar um e amar o outro, ou respeitar este e desprezar aquele. Não podeis servir a Deus e às riquezas. Eis porque digo: não vos inquieteis pelas coisas da vossa vida, se tereis o que comer; nem do vosso corpo, se tereis o que vestir. Porventura não é a vida mais do que o alimento e o corpo mais do que o vestido? Reparai nas aves do céu: nem semeiam, nem ceifam, nem arrecadam nos celeiros; contudo o Pai celestial sustenta-as. Acaso não valeis mais do que elas? Qual de vós será capaz, ainda que empregue todas as diligências, de acrescentar um côvado à sua altura? E, quanto ao vestir, porque vos inquietais? Reparai como os lírios dos campos crescem! Eles não trabalham, nem fiam; contudo nem Salomão, apesar de toda sua magnificência, teve um vestido como eles. Ora, se Deus veste assim as ervas dos campos, que hoje existem e amanhã serão lançadas no fogo, quanto mais terá cuidado de vós, homens de pouca fé! Não vos inquieteis, pois, dizendo: «Que havemos de comer e de beber; e como nos vestiremos?». Os pagãos é que têm cuidados com estas coisas. Vosso Pai celestial sabe bem aquilo de que careceis. Procurai primeiramente o reino de Deus e a sua justiça, e todas estas coisas vos serão dadas».
}\end{paracol}

\paragraphinfo{Ofertório}{Sl. 33, 8-9}
\begin{paracol}{2}\latim{
\rlettrine{I}{mmíttet} Angelus Dómini in circúitu
timéntium eum, et erípiet eos: gustáte et vidéte, quóniam suávis est Dóminus.
}\switchcolumn\portugues{
\rlettrine{O}{} Anjo do Senhor circundará aqueles que temem o Senhor e livrá-los-á. Provai e vede como o Senhor é bom!
}\end{paracol}

\paragraph{Secreta}
\begin{paracol}{2}\latim{
\rlettrine{C}{oncéde} nobis, Dómine, quǽsumus, ut hæc hóstia salutáris et nostrórum fiat purgátio delictórum, et tuæ propitiátio potestátis. Per Dóminum \emph{\&c.}
}\switchcolumn\portugues{
\rlettrine{C}{oncedei-nos,} Senhor, Vos suplicamos, que esta salutar hóstia nos purifique das nossas faltas e nos torne propício o vosso poder. Por nosso Senhor \emph{\&c.}
}\end{paracol}

\paragraphinfo{Comúnio}{Mt. 6, 33}
\begin{paracol}{2}\latim{
\rlettrine{P}{rimum} quǽrite regnum Dei, et ómnia adjiciéntur vobis, dicit Dóminus.
}\switchcolumn\portugues{
\rlettrine{P}{rocurai} primeiramente o reino de Deus, e tudo vos será dado por acréscimo, diz o Senhor.
}\end{paracol}

\paragraph{Postcomúnio}
\begin{paracol}{2}\latim{
\rlettrine{P}{uríficent} semper et múniant tua sacraménta nos, Deus: et ad perpétuæ ducant salvatiónis efféctum. Per Dóminum \emph{\&c.}
}\switchcolumn\portugues{
\rlettrine{P}{ermiti,} ó Deus, que os ossos sacramentos nos purifiquem e fortifiquem sempre, e nos façam gozar a salvação eterna. Por nosso Senhor \emph{\&c.}
}\end{paracol}

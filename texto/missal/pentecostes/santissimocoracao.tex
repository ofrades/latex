\subsection{Santíssimo Coração de Jesus}\label{santissimocoracao}

\paragraphinfo{Intróito}{Sl. 32, 11 \& 19}
\begin{paracol}{2}\latim{
\rlettrine{C}{ogitatiónes} Cordis ejus in generatióne et generatiónem: ut éruat a morte ánimas eórum et alat eos in fame. (T. P. Allelúja, allelúja.) \emph{Ps. ibid., 1} Exsultáte, justi, in Dómino: rectos decet collaudátio.
℣. Gloria Patri \emph{\&c.}
}\switchcolumn\portugues{
\rlettrine{O}{s} pensamentos do seu Coração subsistem de geração em geração, para preservar suas almas da morte e alimentá-las quando tiverem fome. \emph{Sl. ibid., 1} Exultai no Senhor, ó justos: Cantem seus louvores aqueles que possuem o coração recto.
℣. Glória ao Pai \emph{\&c.}
}\end{paracol}

\paragraph{Oração}
\begin{paracol}{2}\latim{
\rlettrine{D}{eus,} qui nobis in Corde Fílii tui, nostris vulneráto peccátis, infinítos dilectiónis thesáuros misericórditer largíri dignáris: concéde, quǽsumus; ut, illi devótum pietátis nostræ præstántes obséquium, dignæ quoque satisfactiónis exhibeámus offícium. Per eúndem Dóminum nostrum \emph{\&c.}
}\switchcolumn\portugues{
\slettrine{Ó}{} Deus, que no Coração do vosso Filho, ferido pelos nossos pecados, Vos dignastes prodigalizar-nos os tesouros infinitos do seu amor, concedei-nos, Vos suplicamos, que, rendendo-Lhe a homenagem da nossa filial devoção, Lhe apresentamos também os dons da nossa reparação. Pelo mesmo nosso Senhor \emph{\&c.}
}\end{paracol}

\paragraphinfo{Epístola}{Ef. 3, 8 19}
\begin{paracol}{2}\latim{
Léctio Epístolæ beáti Pauli Apóstoli ad Ephésios.
}\switchcolumn\portugues{
Lição da Ep.ª do B. Ap. Paulo aos Efésios.
}\switchcolumn*\latim{
\rlettrine{F}{ratres:} Mihi, ómnium sanctórum mínimo, data est grátia hæc, in géntibus evangelizáre investigábiles divítias Christi, et illumináre omnes, quæ sit dispensátio sacraménti abscónditi a sǽculis in Deo, qui ómnia creávit: ut innotéscat principátibus et potestátibus in cœléstibus per Ecclésiam multifórmis sapiéntia Dei, secúndum præfinitiónem sæculórum, quam fecit in Christo Jesu, Dómino nostro, in quo habémus fidúciam et accéssum in confidéntia per fidem ejus. Hujus rei grátia flecto génua mea ad Patrem Dómini nostri Jesu Christi, ex quo omnis patérnitas in cœlis ei in terra nominátur, ut det vobis, secúndum divítias glóriæ suæ, virtúte corroborári per Spíritum ejus in interiórem hóminem, Christum habitáre per fidem in córdibus vestris: in caritáte radicáti et fundáti, ut póssitis comprehéndere cum ómnibus sanctis, quæ sit latitúdo, et longitúdo, et sublímitas, et profúndum: scire étiam supereminéntem sciéntiæ caritátem Christi, ut impleámini in omnem plenitúdinem Dei.
}\switchcolumn\portugues{
\rlettrine{M}{eus} irmãos: Fui eu, que sou o mínimo de todos os santos, quem recebeu esta graça de anunciar aos povos as riquezas incompreensíveis de Jesus Cristo e de esclarecer todos Os homens, revelando-lhes a economia do mistério, que desde tantos séculos está oculto em Deus, criador de todas as coisas, a fim de que, Por meio da Igreja, os potentados e as potestades nos céus conheçam a sabedoria infinitamente variada de Deus, segundo o desígnio eterno que Ele realizou por Jesus Cristo, nosso Senhor, em quem, pela fé que n’Ele depositamos, temos a franca segurança de nos aproximarmos de Deus com confiança. Por este motivo ajoelho diante do Pai de nosso Senhor Jesus Cristo, que é o princípio desta grande família no céu e na terra, para que, segundo as riquezas da sua glória, sejais fortificados pelo seu Espírito e revestidos com a graça de homens interiores. Permita Ele que Jesus Cristo habite pela fé nos vossos corações, de modo que, enraizados e imbuídos na caridade, possais compreender com todos os santos a largura, o comprimento, a altura e a profundidade deste mistério e conhecer o amor de Jesus Cristo, o qual ultrapassa toda a ciência; e, assim, fiqueis repletos de plenitude dos dons de Deus.
}\end{paracol}

\paragraphinfo{Gradual}{Sl. 24, 8-9}
\begin{paracol}{2}\latim{
\rlettrine{D}{ulcis} et rectus Dóminus: propter hoc legem dabit delinquéntibus in via. ℣. Díriget mansúetos in judício, docébit mites vias suas.
}\switchcolumn\portugues{
\rlettrine{O}{} Senhor é bom e recto: eis porque dá uma lei aos delinquentes. ℣. Ele guia os pacíficos pela justiça e manifesta os seus caminhos aos humildes.
}\switchcolumn*\latim{
Allelúja, allelúja. ℣. \emph{Matth. 11, 29} Tóllite jugum meum super vos, et díscite a me, quia mitis sum et húmilis Corde, et inveniétis réquiem animábus vestris. Allelúja.
}\switchcolumn\portugues{
Aleluia, aleluia. ℣. \emph{Mt. 11, 29} Tomai o meu jugo sobre vós e aprendei de mim, que sou manso e humilde de Coração, e achareis paz para as vossas almas.
}\end{paracol}

\textit{Após a Septuagésima omite-se o Aleluia e o seguinte e diz-se:}

\paragraphinfo{Trato}{Sl. 102, 8-10}
\begin{paracol}{2}\latim{
\rlettrine{M}{iséricors} et miserátor Dóminus, longánimis, et multum miséricors. ℣. Non in perpétuum irascétur, neque in ætérnum comminábitur. ℣. Non secúndum peccáta nostra fecit nobis, neque secúndum iniquitátes nostras retríbuit nobis.
}\switchcolumn\portugues{
\rlettrine{O}{} Senhor é terno e compassivo, compadecente e misericordiosíssimo: ℣. Não se irará perpetuamente, nem ameaçará eternamente. ℣. Não nos tratará segundo os nossos pecados, nem nos castigará segundo as nossas iniquidades.
}\end{paracol}

\textit{No Tempo Pascal, em vez do Gradual e do Trato diz-se:}

\begin{paracol}{2}\latim{
Allelúja, allelúja. ℣. \emph{Matth. 11, 29 et 28} Tóllite jugum meum super vos, et díscite a me, quia mitis sum et húmilis Corde: et inveniétis réquiem animábus vestris. Allelúja. ℣. Veníte ad me, omnes qui laborátis, et oneráti estis, et ego refíciam vos. Allelúja.
}\switchcolumn\portugues{
Aleluia, aleluia. ℣. \emph{Mt. 11, 29 et 28} Tomai o meu jugo sobre vós e aprendei de mim, que sou manso e humilde de Coração, e achareis paz para as vossas almas. Aleluia. ℣. Vinde a mim, vós todos, que sofreis e estais sobrecarregados, e vos aliviarei. Aleluia.
}\end{paracol}

\paragraphinfo{Evangelho}{Jo. 19 ,11-37}
\begin{paracol}{2}\latim{
\cruz Sequéntia sancti Evangélii secúndum Joánnem.
}\switchcolumn\portugues{
\cruz Continuação do santo Evangelho segundo S. João.
}\switchcolumn*\latim{
\blettrine{I}{n} illo témpore: Judǽi (quóniam Parascéve erat), ut non remanérent in cruce córpora sábbato (erat enim magnus dies ille sábbati), rogavérunt Pilátum, ut frangeréntur eórum crura, et tolleréntur. Venérunt ergo mílites: et primi quidem fregérunt crura et alteríus, qui crucifíxus est cum eo. Ad Jesum autem cum veníssent, ut vidérunt eum jam mórtuum, non fregérunt ejus crura, sed unus mílitum láncea latus ejus apéruit, et contínuo exívit sanguis et aqua. Et qui vidit, testimónium perhíbuit: et verum est testimónium ejus. Et ille scit quia vera dicit, ut et vos credátis. Facta sunt enim hæc ut Scriptúra implerétur: Os non comminuétis ex eo. Et íterum alia Scriptúra dicit: Vidébunt in quem transfixérunt.
}\switchcolumn\portugues{
\blettrine{N}{aquele} tempo, os judeus (porque era o dia da preparação da Páscoa), não desejando que os corpos ficassem na cruz para o sábado (pois o sábado era solene), pediram a Pilatos consentisse que partissem as pernas aos crucificados e os descessem da cruz. Os soldados vieram, pois, e quebraram as pernas do primeiro e do outro que haviam sido crucificados com Jesus. Tendo vindo a Jesus, como O vissem já morto, Lhe não quebraram as pernas, mas um dos soldados abriu-Lhe o lado com a lança, do qual saiu sangue e água imediatamente. E quem isto viu dá testemunho disso, e o seu testemunho é verdadeiro, pois sabe que diz a verdade, para que também vós acrediteis. Aconteceram estas coisas para se cumprir o que dizia a Escritura: «Não quebrareis nenhum dos meus ossos». E ainda a Escritura diz em outro lugar: «Contemplarão aquele que traspassaram».
}\end{paracol}

\paragraphinfo{Ofertório}{Sl. 68, 21}
\begin{paracol}{2}\latim{
\rlettrine{I}{mpropérium} exspectávi Cor meum et misériam: et sustínui, qui simul mecum contristarétur, et non fuit: consolántem me quæsívi, et non invéni.
}\switchcolumn\portugues{
\rlettrine{I}{mpropérios} e misérias afrontaram o meu coração: E procurei quem compartilhasse a minha tristeza, mas não achei; Procurei alguém que me consolasse, mas não encontrei.
}\end{paracol}

\textit{No Tempo Pascal diz-se o seguinte:}

\paragraphinfo{Ofertório}{Sl. 39, 7-9}
\begin{paracol}{2}\latim{
\rlettrine{H}{olocáustum} et pro peccáto non postulásti; tunc dixi: Ecce, vénio. In cápite libri scriptum est de me ut fácerem voluntátem tuam: Deus meus, volui, et legem tuam in médio Cordis mei, allelúja.
}\switchcolumn\portugues{
\rlettrine{N}{ão} pedistes holocaustos nem sacrifícios pelo pecado; e logo dissestes: «Eis que eu venho. No princípio do livro está escrito a meu respeito que vim para fazer a vossa vontade. E Eu, ó meu Deus, assim o quis; a vossa lei está no íntimo do meu coração».
}\end{paracol}

\paragraph{Secreta}
\begin{paracol}{2}\latim{
\rlettrine{R}{éspice,} quǽsumus, Dómine, ad ineffábilem Cordis dilécti Fílii tui caritátem: ut quod offérimus sit tibi munus accéptum et nostrórum expiátio delictórum. Per eúndem Dóminum \emph{\&c.}
}\switchcolumn\portugues{
\rlettrine{S}{enhor,} Vos suplicamos, atendei benigno à caridade inefável do Coração do vosso amado Filho, a fim de que a oferta, que Vos apresentamos, por Vós seja recebida, e nos sirva de expiação dos nossos pecados. Pelo mesmo Senhor \emph{\&c.}
}\end{paracol}

\paragraphinfo{Comúnio}{Jo. 19, 34}
\begin{paracol}{2}\latim{
\rlettrine{U}{nus} mílitum láncea latus ejus apéruit, et contínuo exívit sanguis et aqua.
}\switchcolumn\portugues{
\rlettrine{U}{m} dos soldados abriu-Lhe o lado com uma lança, e logo saiu sangue e água.
}\end{paracol}

\textit{No Tempo Pascal diz-se o seguinte, em vez do Precedente:}

\paragraphinfo{Comúnio}{Jo. 7, 37}
\begin{paracol}{2}\latim{
\rlettrine{S}{i} quis sitit, véniat ad me et bibat, allelúja, allelúja.
}\switchcolumn\portugues{
\rlettrine{S}{e} alguém tem sede, venha a mim e beba. Aleluia, aleluia.
}\end{paracol}

\paragraph{Postcomúnio}
\begin{paracol}{2}\latim{
\rlettrine{P}{rǽbeant} nobis, Dómine Jesu, divínum tua sancta fervórem: quo dulcíssimi Cordis tui suavitáte percépta; discámus terréna despícere, et amáre cœléstia: Qui vivis et regnas \emph{\&c.}
}\switchcolumn\portugues{
\qlettrine{Q}{}ue os vossos sacrossantos mistérios nos comuniquem, ó Senhor Jesus, o divino fervor, a fim de que, sentindo as suavidades do vosso dulcíssimo Coração, aprendamos a desprezar as coisas terrenas e a amar as celestiais. Ó Vós, que, sendo Deus \emph{\&c.}
}\end{paracol}

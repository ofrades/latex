\subsection{Vigésimo Quarto Domingo depois de Pentecostes}

\paragraphinfo{Intróito}{Jr. 29,11,12 \& 14}
\begin{paracol}{2}\latim{
\rlettrine{D}{icit} Dóminus: Ego cógito cogitatiónes pacis, et non afflictiónis: invocábitis me, et ego exáudiam vos: et redúcam captivitátem vestram de cunctis locis. \emph{Ps. 84, 2} Benedixísti, Dómine, terram tuam: avertísti captivitátem Jacob.
℣. Gloria Patri \emph{\&c.}
}\switchcolumn\portugues{
\rlettrine{D}{isse} o Senhor: tenho pensamentos de paz e não de ira: invocar-me-eis e ouvir-Vos-ei; e farei regressar de todos os países os vossos cativos. \emph{Sl. 84, 2} Abençoastes, Senhor, a vossa terra e livrastes Jacob do cativeiro.
℣. Glória ao Pai \emph{\&c.}
}\end{paracol}

\paragraph{Oração}
\begin{paracol}{2}\latim{
\rlettrine{E}{xcita,} quǽsumus. Dómine, tuórum fidélium voluntátes: ut, divíni óperis fructum propénsius exsequéntes; pietátis tuæ remédia majóra percípiant. Per Dóminum nostrum \emph{\&c.}
}\switchcolumn\portugues{
\rlettrine{D}{espertai,} Senhor, Vos imploramos, a vontade dos vossos fiéis, para que, procurando com mais fervor o fruto das obras divinas, alcancem da vossa misericórdia os melhores remédios. Por nosso Senhor \emph{\&c.}
}\end{paracol}

\paragraphinfo{Epístola}{Cl. 1, 9-14}
\begin{paracol}{2}\latim{
Léctio Epístolæ beáti Pauli Apóstoli ad Colossénses.
}\switchcolumn\portugues{
Lição da Ep.ª do B. Ap.º Paulo aos Colossenses.
}\switchcolumn*\latim{
\rlettrine{F}{ratres:} Non cessámus pro vobis orántes et postulántes, ut impleámini agnitióne voluntátis Dei, in omni sapiéntia et intelléctu spiritáli: ut ambulétis digne Deo per ómnia placéntes: in omni ópere bono fructificántes, et crescéntes in scientia Dei: in omni virtúte confortáti secúndum poténtiam claritátis ejus in omni patiéntia, et longanimitáte cum gáudio, grátias agentes Deo Patri, qui dignos nos fecit in partem sortis sanctórum in lúmine: qui erí puit nos de potestáte tenebrárum, et tránstulit in regnum Fílii dilectiónis suæ, in quo habémus redemptiónem per sánguinem ejus, remissiónem peccatórum.
}\switchcolumn\portugues{
\rlettrine{M}{eus} irmãos: Não cessamos de orar a Deus por vós e de pedir-Lhe que vos conceda o verdadeiro conhecimento da sua vontade em toda a plenitude da sabedoria e da inteligência espiritual, para que tenhais uma conduta digna de Deus, agradando-Lhe em todas as coisas, dando frutos em todas as espécies de boas obras e progredindo no conhecimento de Deus. Que sejais fortificados com o poder da glória de Deus para suportardes tudo com paciência, longanimidade e alegria, dando graças a Deus Pai, que nos tornou dignos de tomar parte na herança dos Santos na luz e nos livrou do poder das trevas, para nos transportarmos ao reino do seu amantíssimo Filho, em quem alcançámos pelo seu Sangue a Redenção e a remissão dos pecados.
}\end{paracol}

\paragraphinfo{Gradual}{Sl. 43, 8-9}
\begin{paracol}{2}\latim{
\rlettrine{L}{iberásti} nos, Dómine, ex affligéntibus nos: et eos, qui nos odérunt, confudísti. ℣. In Deo laudábimur tota die, et in nómine tuo confitébimur in sǽcula.
}\switchcolumn\portugues{
\rlettrine{L}{ivrastes-nos,} Senhor, daqueles que nos afligiam: e confundistes os que nos odiavam. Glorificar-nos-emos constantemente em Deus e louvaremos eternamente o vosso nome.
}\switchcolumn*\latim{
Allelúja, allelúja. ℣. \emph{Ps. 129, 12} De profúndis clamávi ad te, Dómine: Dómine, exáudi oratiónem meam. Allelúja.
}\switchcolumn\portugues{
Aleluia, aleluia. ℣. \emph{Sl. 129, 12} Do fundo do abismo Vos invoquei, Senhor: escutai a minha oração. Aleluia.
}\end{paracol}

\paragraphinfo{Evangelho}{Mt. 24, 15-35}
\begin{paracol}{2}\latim{
\cruz Sequéntia sancti Evangélii secúndum Matthǽum.
}\switchcolumn\portugues{
\cruz Continuação do santo Evangelho segundo S. Mateus.
}\switchcolumn*\latim{
\blettrine{I}{n} illo témpore: Dixit Jesus discípulis suis: Cum vidéritis abominatiónem desolatiónis, quæ dicta est a Daniéle Prophéta, stantem in loco sancto: qui legit, intéllegat: tunc qui in Judǽa sunt, fúgiant ad montes: et qui in tecto, non descéndat tóllere áliquid de domo sua: et qui in agro, non revertátur tóllere túnicam suam. Væ autem prægnántibus et nutriéntibus in illis diébus. Oráte autem, ut non fiat fuga vestra in híeme vel sábbato. Erit enim tunc tribulátio magna, qualis non fuit ab inítio mundi usque modo, neque fiet. Et nisi breviáti fuíssent dies illi, non fíeret salva omnis caro: sed propter eléctos breviabúntur dies illi. Tunc si quis vobis díxerit: Ecce, hic est Christus, aut illic: nolíte crédere. Surgent enim pseudochrísti et pseudoprophétæ, et dabunt signa magna et prodígia, ita ut in errórem inducántur (si fíeri potest) étiam elécti. Ecce, prædíxi vobis. Si ergo díxerint vobis: Ecce, in desérto est, nolíte exíre: ecce, in penetrálibus, nolíte crédere. Sicut enim fulgur exit ab Oriénte et paret usque in Occidéntem: ita erit et advéntus Fílii hóminis. Ubicúmque fúerit corpus, illic congregabúntur et áquilæ. Statim autem post tribulatiónem diérum illórum sol obscurábitur, et luna non dabit lumen suum, et stellæ cadent de cælo, et virtútes cœlórum commovebúntur: et tunc parébit signum Fílii hóminis in cœlo: et tunc plangent omnes tribus terræ: et vidébunt Fílium hominis veniéntem in núbibus cæli cum virtúte multa et majestáte. Et mittet Angelos suos cum tuba et voce magna: et congregábunt electos ejus a quátuor ventis, a summis cœlórum usque ad términos eórum. Ab arbóre autem fici díscite parábolam: Dum jam ramus ejus tener túerit et fólia nata, scitis, quia prope est æstas: ita et vos tum vidéritis hæc ómnia, scitóte, quia prope est in jánuis. Amen, dico vobis, quia nonpræteríbit generátio hæc, donec ómnia hæc fiant. Cœlum et terra transíbunt, verba autem mea non præteríbunt.
}\switchcolumn\portugues{
\blettrine{N}{aquele} tempo, disse Jesus aos seus discípulos: Quando virdes a abominação da desolação, anunciada pelo Profeta Daniel, reinando no lugar santo, que aquele que lê entenda: Então, aqueles que estão na Judeia fujam para as montanhas; aquele que se achar em cima do telhado não desça para ir buscar qualquer coisa a casa; e aquele que estiver nos campos não volte a casa para procurar algum vestido. Ai das mulheres que estiverem prestes a ser mães: ou a amamentar seus filhos nesses dias! Rogai ao Senhor que a vossa fuga não seja nem no Inverno, nem ao sábado; pois a aflição será tão grande que não houve coisa semelhante desde o princípio do mundo até ao presente, como não haverá nunca mais; e, se esses dias não fossem abreviados (e sê-lo-ão em atenção aos escolhidos), ninguém seria salvo. Então, se alguém vos disser: «O Cristo está aqui, ou está acolá», não acrediteis; pois aparecerão falsos cristos e falsos profetas, que praticarão grandes maravilhas e prodígios até mesmo seduzirem (se tal fora possível) os próprios escolhidos. Eu vo-lo anuncio, desde já. Se, pois, vos disserem: «Ei-l’O está no deserto», não saiais; ou se vos disserem: «Ei-l’O aqui, no lugar mais retirado de casa», não acrediteis também; pois, assim como o relâmpago parte do Oriente e brilha até ao Ocidente, assim será também a vinda do Filho do homem. Em qualquer lugar em que estiver o cadáver aí se reunirão as águias. Imediatamente após a tribulação destes dias o sol se obscurecerá, a lua não projectará luz, as estrelas cairão do céu e os poderes dos céus serão abalados. E aparecerá no céu o sinal do Filho do homem, e todas as tribos da terra se lamentarão: e verão o Filho do homem sobre as nuvens do céu, revestido de grande poder e majestade! Ele enviará os seus Anjos, que farão retinir a trombeta em som estridente, e logo convocarão os escolhidos dos quatro ventos, desde uma extremidade dos céus até à outra! Compreendei isto por esta parábola, tirada da figueira: «Quando os seus ramos estão tenros e as folhas começam a despontar, conheceis que está próximo o estio; assim também, quando virdes todas estas coisas, sabereis que o Filho do homem está próximo, que está à porta». Em verdade vos digo: «Esta geração não passará sem que isto aconteça. Passarão o céu e a terra; mas as minhas palavras não passarão!».
}\end{paracol}

\paragraphinfo{Ofertório}{Sl. 129, 1-2}
\begin{paracol}{2}\latim{
\rlettrine{D}{e} profúndis clamávi ad te, Dómine: Dómine, exáudi oratiónem meam: de profúndis clamávi ad te. Dómine.
}\switchcolumn\portugues{
\rlettrine{D}{as} profundezas dos abismos Vos invoquei, Senhor; escutai, Senhor, a minha voz: das profundezas dos abismos Vos invoquei.
}\end{paracol}

\paragraph{Secreta}
\begin{paracol}{2}\latim{
\rlettrine{P}{ropítius} esto, Dómine, supplicatiónibus nostris: et, pópuli tui oblatiónibus precibúsque suscéptis, ómnium nostrum ad te corda convérte; ut, a terrenis cupiditátibus liberáti, ad cœléstia desidéria transeámus. Per Dóminum \emph{\&c.}
}\switchcolumn\portugues{
\rlettrine{S}{ede} propício, Senhor, às nossas súplicas, e, depois de haverdes recebido as ofertas e as orações do vosso povo, convertei a Vós os nossos corações, a fim de que, sendo livres das cobiças deste mundo, não tenhamos outra aspiração senão a dos bens do céu. Por nosso Senhor \emph{\&c.}
}\end{paracol}

\paragraphinfo{Comúnio}{Mc. 11, 24}
\begin{paracol}{2}\latim{
\rlettrine{A}{men,} dico vobis, quidquid orántes pétitis, crédite, quia accipiétis, et fiet vobis.
}\switchcolumn\portugues{
\rlettrine{N}{a} verdade vos digo: «Tudo o que pedirdes nas vossas orações, acreditai que o recebereis; e far-se-á como pedirdes».
}\end{paracol}

\paragraph{Postcomúnio}
\begin{paracol}{2}\latim{
\rlettrine{C}{oncéde} nobis, quǽsumus, Dómine: ut per hæc sacraménta quæ súmpsimus, quidquid in nostra mente vitiósum est, ipsorum medicatiónis dono curétur. Per Dóminum nostrum \emph{\&c.}
}\switchcolumn\portugues{
\rlettrine{P}{ermiti,} Senhor, Vos suplicamos, que pela virtude destes sacramentos, que recebemos, tudo quanto houver de vicioso na nossa alma seja curado pelo efeito deste remédio divino. Por nosso Senhor \emph{\&c.}
}\end{paracol}

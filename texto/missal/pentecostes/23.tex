\subsection{Vigésimo Terceiro Domingo depois de Pentecostes}

\paragraphinfo{Intróito}{Jr. 29,11,12 \& 14}
\begin{paracol}{2}\latim{
\rlettrine{D}{icit} Dóminus: Ego cógito cogitatiónes pacis, et non afflictiónis: invocábitis me, et ego exáudiam vos: et redúcam captivitátem vestram de cunctis locis. \emph{Ps. 84, 2} Benedixísti, Dómine, terram tuam: avertísti captivitátem Jacob.
℣. Gloria Patri \emph{\&c.}
}\switchcolumn\portugues{
\rlettrine{D}{isse} o Senhor: tenho pensamentos de paz e não de ira: invocar-me-eis e ouvir-vos-ei; e farei regressar de todos os países os vossos cativos. \emph{Sl. 84, 2} Abençoastes, Senhor, a vossa terra e livrastes Jacob do cativeiro.
℣. Glória ao Pai \emph{\&c.}
}\end{paracol}

\paragraph{Oração}
\begin{paracol}{2}\latim{
\rlettrine{A}{bsólve,} quǽsumus, Dómine, tuórum delícta populórum: ut a peccatórum néxibus, quae pro nostra fraglitáte contráximus, tua benignitáte liberémur. Per Dóminum nostrum \emph{\&c.}
}\switchcolumn\portugues{
\rlettrine{P}{erdoai,} Senhor, os delitos do vosso povo, a fim de que pela vossa benignidade sejamos livres dos liames dos pecados em que a nossa fraqueza nos fez cair. Por nosso Senhor \emph{\&c.}
}\end{paracol}

\paragraphinfo{Epístola}{Fl. 3, 17-21: 4, 1-3}
\begin{paracol}{2}\latim{
Léctio Epístolæ beáti Pauli Apóstoli ad Philippénses.
}\switchcolumn\portugues{
Lição da Ep.ª do B. Ap.º Paulo aos Filipenses.
}\switchcolumn*\latim{
\rlettrine{F}{ratres:} Imitatóres mei estóte, et observáte eos, qui ita ámbulant, sicut habétis formam nostram. Multi enim ámbulant, quos sæpe dicébam vobis (nunc autem et flens dico) inimícos Crucis Christi: quorum finis intéritus: quorum Deus venter est: et glória in confusióne ipsórum, qui terréna sápiunt. Nostra autem conversátio in cœlis est: unde etiam Salvatórem exspectámus, Dóminum nostrum Jesum Christum, qui reformábit corpus humilitátis nostræ, configurátum córpori claritátis suæ, secúndum operatiónem, qua étiam possit subjícere sibi ómnia. Itaque, fratres mei caríssimi et desideratíssimi, gáudium meum et coróna mea: sic state in Dómino, caríssimi. Evódiam rogo et Sýntychen déprecor idípsum sápere in Dómino. Etiam rogo et te, germáne compar, ádjuva illas, quæ mecum laboravérunt in Evangélio cum Cleménte et céteris adjutóribus meis, quorum nómina sunt in libro vitæ.
}\switchcolumn\portugues{
\rlettrine{M}{eus} irmãos: Sede meus imitadores e segui aqueles que se conduzem segundo o modelo que tendes em nós; porque há muitos, de quem vos tenho falado (e ainda falo deles com lágrimas), que se portam como inimigos da cruz de Cristo. Seu fim será a condenação; pois fazem do estômago o seu deus, põem a sua glória naquilo que deveria ser motivo de vergonha e não têm prazer senão nas cousas terrenas. Mas, quanto a nós, o nosso pensamento está nos céus, donde também esperamos o Salvador, N. S. Jesus Cristo, que renovará o nosso corpo vil, tornando-o semelhante aoseu corpo glorioso, por meio daquela virtude, que Ele possui, de sujeitar a si todas as cousas. Portanto, queridos e amados irmãos, que sois a minha alegria e a minha coroa, permanecei firmes no Senhor, meus caríssimos irmãos. Peço a Evódia e suplico a Sintiquene que tenham os mesmos sentimentos no Senhor. Também vos rogo, ó fiel companheiro, que auxilieis aqueles que trabalharam comigo pelo Evangelho, com Clemente e com os outros meus coadjutores, cujos nomes estão no livro da vida.
}\end{paracol}

\paragraphinfo{Gradual}{Sl. 43, 8-9}
\begin{paracol}{2}\latim{
\rlettrine{L}{iberásti} nos, Dómine, ex affligéntibus nos: et eos, qui nos odérunt, confudísti. ℣. In Deo laudábimur tota die, et in nómine tuo confitébimur in sǽcula.
}\switchcolumn\portugues{
\rlettrine{L}{ivrastes-nos,} Senhor, daqueles que nos afligiam: e confundistes os que nos odiavam. ℣. Glorificar-nos-emos constantemente em Deus e louvaremos eternamente o vosso nome.
}\switchcolumn*\latim{
Allelúja, allelúja. ℣. \emph{Ps. 129, 12} De profúndis clamávi ad te, Dómine: Dómine, exáudi oratiónem meam. Allelúja.
}\switchcolumn\portugues{
Aleluia, aleluia. ℣. \emph{Sl. 129, 12} Do fundo do abysmo Vos invoquei, Senhor: escutai a minha oração. Aleluia.
}\end{paracol}

\paragraphinfo{Evangelho}{Mt. 9, 18-26}
\begin{paracol}{2}\latim{
\cruz Sequéntia sancti Evangélii secúndum Matthǽum.
}\switchcolumn\portugues{
\cruz Continuação do santo Evangelho segundo S. Mateus.
}\switchcolumn*\latim{
\blettrine{I}{n} illo témpore: Loquénte Jesu ad turbas, ecce, princeps unus accéssit et adorábat eum, dicens: Dómine, fília mea modo defúncta est: sed veni, impóne manum tuam super eam, et vivet. Et surgens Jesus sequebátur eum et discípuli ejus. Et ecce múlier, quæ sánguinis fluxum patiebátur duódecim annis, accéssit retro et tétigit fímbriam vestiménti ejus. Dicébat enim intra se: Si tetígero tantum vestiméntum ejus, salva ero. At Jesus convérsus et videns eam, dixit: Confíde, fília, fides tua te salvam fecit. Et salva facta est múlier ex illa hora. Et cum venísset Jesus in domum príncipis, et vidísset tibícines et turbam tumultuántem, dicebat: Recédite: non est enim mórtua puélla, sed dormit. Et deridébant eum. Et cum ejécta esset turba, intrávit et ténuit manum ejus. Et surréxit puella. Et éxiit fama hæc in univérsam terram illam.
}\switchcolumn\portugues{
\blettrine{N}{aquele} tempo, falando Jesus ao povo, um dos príncipes da sinagoga aproximou-se d’Ele e, adorando-O, disse-Lhe: «Senhor, a minha filha acaba de morrer, mas vinde, colocai vossas mãos sobre ela e recobrará a vida». Levantando-se então, Jesus segui-o com os discípulos. Logo, uma mulher que padecia dum fluxo de sangue havia doze anos aproximou-se d’Ele por detrás e tocou-Lhe na franja do vestido, pois (dizia de si para si), se eu tocar, somente que seja, no seu vestido, serei curada. Então Jesus, voltando-se e vendo-a, disse-lhe: «Tende confiança, minha filha, a vossa fé salvou-vos». E naquela hora foi curada esta mulher! Quando Jesus chegou a casa do príncipe da sinagoga, vendo os tocadores de flauta e a turba do povo, carpindo muito, disse: «Retirai-vos, porque a menina não está morta, mas adormecida». Eles riam-se de Jesus! Havendo, porém, saído a turba, entrou Jesus e pegou na mão da menina, que logo ressuscitou! E correu a fama deste acontecimento em todo o país.
}\end{paracol}

\paragraphinfo{Ofertório}{Sl. 129, 1-2}
\begin{paracol}{2}\latim{
\rlettrine{D}{e} profúndis clamávi ad te, Dómine: Dómine, exáudi oratiónem meam: de profúndis clamávi ad te. Dómine
}\switchcolumn\portugues{
\rlettrine{D}{as} profundezas dos abysmos Vos invoquei, Senhor; escutai, Senhor, a minha voz: das profundezas dos abysmos Vos invoquei.
}\end{paracol}

\paragraph{Secreta}
\begin{paracol}{2}\latim{
\rlettrine{P}{ro} nostræ servitútis augménto sacrifícium tibi, Dómine, laudis offérimus: ut, quod imméritis contulísti, propítius exsequáris. Per Dóminum nostrum \emph{\&c.}
}\switchcolumn\portugues{
\rlettrine{P}{ara} aumento do nosso zelo em Vos servir, Senhor, Vos oferecemos um sacrifício de louvor, a fim de que pela vossa bondade alcancemos o efeito dos dons que nos concedestes sem nenhuns merecimentos da nossa parte. Por nosso Senhor \emph{\&c.}
}\end{paracol}

\paragraphinfo{Comúnio}{Mc. 11, 24}
\begin{paracol}{2}\latim{
\rlettrine{A}{men,} dico vobis, quidquid orántes pétitis, crédite, quia accipiétis, et fiet vobis.
}\switchcolumn\portugues{
\rlettrine{N}{a} verdade vos digo: «Tudo o que pedirdes nas vossas orações, acreditai que o recebereis; e far-se-á como pedirdes».
}\end{paracol}

\paragraph{Postcomúnio}
\begin{paracol}{2}\latim{
\qlettrine{Q}{uǽsumus,} omnípotens Deus: ut, quos divína tríbuis participatióne gaudére, humánis non sinas subjacére perículis. Per Dóminum \emph{\&c.}
}\switchcolumn\portugues{
\slettrine{Ó}{} Deus omnipotente, Vos suplicamos, não permitais que aqueles a quem concedestes a alegria de participarem dos vossos sagrados mystérios fiquem expostos aos perigos humanos. Por nosso Senhor \emph{\&c.}
}\end{paracol}

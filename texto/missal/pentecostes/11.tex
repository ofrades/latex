\subsection{Décimo Primeiro Domingo depois de Pentecostes}

\paragraphinfo{Intróito}{Sl. 67, 6-7 \& 36 1}
\begin{paracol}{2}\latim{
\rlettrine{D}{eus} in loco sancto suo: Deus qui inhabitáre facit unánimes in domo: ipse dabit virtútem et fortitúdinem plebi suæ. \emph{Ps. ibid., 2} Exsúrgat Deus, et dissipéntur inimíci ejus: et fúgiant, qui odérunt eum, a fácie ejus.
℣. Gloria Patri \emph{\&c.}
}\switchcolumn\portugues{
\rlettrine{D}{eus} está na sua santa morada: Deus dará morada na sua casa àqueles que possuem o mesmo espírito: Ele dará ao seu povo fortaleza e constância. \emph{Sl. ibid., 2} Erga-se Deus e sejam dispersos os seus inimigos: Que aqueles que o odeiam fujam da sua presença.
℣. Glória ao Pai \emph{\&c.}
}\end{paracol}

\paragraph{Oração}
\begin{paracol}{2}\latim{
\rlettrine{O}{mnípotens} sempitérne Deus, qui, abundántia pietátis tuæ, et merita súpplicum excédis et vota: effúnde super nos misericórdiam tuam; ut dimíttas quæ consciéntia metuit, et adjícias quod orátio non præsúmit. Per Dóminum \emph{\&c.}
}\switchcolumn\portugues{
\rlettrine{D}{eus} omnipotente e sempiterno, que pela abundância da vossa bondade excedeis os méritos e os desejos dos suplicantes, espalhai a vossa misericórdia sobre nós, de modo que nos sejam perdoados aqueles castigos, que a nossa consciência teme, e nos sejam concedidas aquelas graças, que não ousamos pedir: Por nosso Senhor \emph{\&c.}
}\end{paracol}

\paragraphinfo{Epístola}{1. Cor. 15, 1-10}
\begin{paracol}{2}\latim{
Léctio Epístolæ beáti Pauli Apóstoli ad Corínthios.
}\switchcolumn\portugues{
Lição da Ep.ª do B. Ap.º Paulo aos Coríntios.
}\switchcolumn*\latim{
\rlettrine{F}{ratres:} Notum vobis fácio Evangélium, quod prædicávi vobis, quod et accepístis, in quo et statis, per quod et salvámini: qua ratione prædicáverim vobis, si tenétis, nisi frustra credidístis. Trádidi enim vobis in primis, quod et accépi: quóniam Christus mortuus est pro peccátis nostris secúndum Scriptúras: et quia sepúltus est, et quia resurréxit tértia die secúndum Scriptúras: et quia visus est Cephæ, et post hoc úndecim. Deinde visus est plus quam quingéntis frátribus simul, ex quibus multi manent usque adhuc, quidam autem dormiérunt. Deinde visus est Jacóbo, deinde Apóstolis ómnibus: novíssime autem ómnium tamquam abortívo, visus est et mihi. Ego enim sum mínimus Apostolórum, qui non sum dignus vocári Apóstolus, quóniam persecútus sum Ecclésiam Dei. Grátia autem Dei sum id quod sum, et grátia ejus in me vácua non fuit.
}\switchcolumn\portugues{
\rlettrine{M}{eus} irmãos: Chamo a vossa atenção para o Evangelho que já vos preguei e recebestes. Nele permanecereis firmes, sendo por ele que sereis salvos, se porventura o cumprirdes tal como vo-lo anunciei; pois, se não fora assim, teríeis abraçado a fé inutilmente. Em primeiro lugar, ensinei-vos o que também aprendi, isto é: Que Cristo morreu por causa dos nossos pecados, segundo as Escrituras; que foi sepultado e ressuscitou ao terceiro dia, segundo as mesmas Escrituras; que apareceu a Cefas e seguidamente aos outros Onze Apóstolos; que, depois, foi visto por mais de quinhentos irmãos, que estavam reunidos, alguns dos quais estão vivos e outros já morreram; que, igualmente, foi visto por Tiago e logo pelos outros Apóstolos. Finalmente, depois dos outros, foi visto por mim, que sou o mais imperfeito de todos. Sim; sou o mínimo dos Apóstolos, assim como sou indigno de ser chamado Apóstolo, pois fui perseguidor da Igreja de Deus. Mas, pela graça de Deus, sou o que sou, e a sua graça não ficou estéril em mim.
}\end{paracol}

\paragraphinfo{Gradual}{Sl. 27, 7 \& 1}
\begin{paracol}{2}\latim{
\rlettrine{I}{n} Deo sperávit cor meum, et adjútus sum: et reflóruit caro mea, et ex voluntáte mea confitébor illi. ℣. Ad te, Dómine, clamávi: Deus meus, ne síleas, ne discédas a me.
}\switchcolumn\portugues{
\rlettrine{O}{} meu coração confiou em Deus, que veio em meu socorro. Então a minha carne remoçou. Eis porque louvarei o Senhor com todo o coração. ℣. Clamei por Vós, Senhor! Meu Deus, não fecheis os ouvidos à minha voz, nem Vos afasteis de mim.
}\switchcolumn*\latim{
Allelúja, allelúja. ℣. \emph{Ps. 80, 2-3} Exsultáte Deo, adjutóri nostro, jubiláte Deo Jacob: súmite psalmum jucúndum cum cíthara. Allelúja.
}\switchcolumn\portugues{
Aleluia, aleluia. ℣. \emph{Sl. 80, 2-3} Exultai de alegria em Deus, que é o nosso protector: Cantai hinos em honra de Deus de Jacob: Tocai em tom alegre o saltério e a cítara.
}\end{paracol}

\paragraphinfo{Evangelho}{Mc. 7, 31-37}
\begin{paracol}{2}\latim{
\cruz Sequéntia sancti Evangélii secúndum Marcum.
}\switchcolumn\portugues{
\cruz Continuação do santo Evangelho segundo S. Marcos.
}\switchcolumn*\latim{
\blettrine{I}{n} illo témpore: Exiens Jesus de fínibus Tyri, venitper Sidónem ad mare Galilǽæ, inter médios fines Decapóleos. Et addúcunt ei surdum et mutum, et deprecabántur eum, ut impónat illi manum. Et apprehéndens eum de turba seórsum, misit dígitos suos in aurículas ejus: et éxspuens, tétigit linguam ejus: et suspíciens in cœlum, ingémuit, et ait illi: Ephphetha, quod est adaperíre. Et statim apértæ sunt aures ejus, et solútum est vínculum linguæ ejus, et loquebátur recte. Et præcépit illis, ne cui dícerent. Quanto autem eis præcipiébat, tanto magis plus prædicábant: et eo ámplius admirabántur, dicéntes: Bene ómnia fecit: et surdos fecit audíre et mutos loqui.
}\switchcolumn\portugues{
\blettrine{N}{aquele} tempo saindo Jesus do território de Tiro, veio por Sidónia, através da Decápole, até ao mar da Galileia. Então, apresentaram-lhe um surdo-mudo e suplicaram-Lhe que pusesse as mãos sobre ele. Jesus, separando-o da multidão, pôs-lhe os dedos nos ouvidos e saliva na língua. Depois, erguendo os olhos ao céu, deu um suspiro e disse: «Ephpheta», isto é, abre-te! E logo se abriram os ouvidos deste homem e se partiu a prisão da sua língua, falando distintamente. Então Jesus impôs-lhes que não dissessem a ninguém o que se passara; mas, quanto mais os proibia, mais o publicavam; e, cheios de admiração, diziam: Ele fez bem tudo, fez ouvir os surdos e falar os mudos.
}\end{paracol}

\paragraphinfo{Ofertório}{Sl. 29, 2-3}
\begin{paracol}{2}\latim{
\rlettrine{E}{xaltábo} te, Dómine, quóniam suscepísti me, nec delectásti inimícos meos super me: Dómine, clamávi ad te, fet sanásti me.
}\switchcolumn\portugues{
\rlettrine{E}{xaltar-Vos-ei,} Senhor, porque me atendestes e não deixastes que meus inimigos escarnecem de mim. Clamei por Vós, Senhor, e curaste-me!
}\end{paracol}

\paragraph{Secreta}
\begin{paracol}{2}\latim{
\rlettrine{R}{éspice,} Dómine, quǽsumus, nostram propítius servitútem: ut, quod offérimus, sit tibi munus accéptum, et sit nostræ fragilitátis subsidium. Per Dóminum \emph{\&c.}
}\switchcolumn\portugues{
\rlettrine{D}{ignai-Vos,} Senhor, Vos suplicamos, olhar propício para a nossa homenagem, a fim de que a nossa oferta Vos seja agradável e sirva de auxílio para a nossa fraqueza. Por nosso Senhor \emph{\&c.}
}\end{paracol}

\paragraphinfo{Comúnio}{Pr. 3, 9-10}
\begin{paracol}{2}\latim{
\rlettrine{H}{ónora} Dóminum de tua substántia, et de prímitus frugum tuárum: et implebúntur hórrea tua saturitáte, et vino torculária redundábunt.
}\switchcolumn\portugues{
\rlettrine{H}{onrai} o Senhor, oferecendo-Lhe os vossos bens e as primícias dos vossos frutos. Então os vossos celeiros ficarão cheios de trigo e os vossos lagares trasbordarão de vinho.
}\end{paracol}

\paragraph{Postcomúnio}
\begin{paracol}{2}\latim{
\rlettrine{S}{entiámus,} quǽsumus, Dómine, tui perceptióne sacraménti, subsídium mentis et córporis: ut, in utróque salváti, cæléstis remédii plenitúdine gloriémur. Per Dóminum nostrum \emph{\&c.}
}\switchcolumn\portugues{
\rlettrine{P}{ermiti,} Senhor, Vos suplicamos, que, pela recepção do vosso sacramento, sintamos o vosso conforto na alma e no corpo, a fim de que, salvando-se ambos, possamos gozar na glória a plenitude deste remédio celestial. Por nosso Senhor \emph{\&c.}
}\end{paracol}

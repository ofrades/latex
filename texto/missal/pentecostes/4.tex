\subsection{Quarto Domingo depois de Pentecostes}

\paragraphinfo{Intróito}{Sl. 26, 1 \& 2}
\begin{paracol}{2}\latim{
\rlettrine{D}{óminus} illuminátio mea et salus mea, quem timebo? Dóminus defensor vitæ meæ, a quo trepidábo? qui tríbulant me inimíci mei, ipsi infirmáti sunt, et cecidérunt. \emph{Ps. ibid., 3} Si consístant advérsum me castra: non timébit cor meum.
℣. Gloria Patri \emph{\&c.}
}\switchcolumn\portugues{
\rlettrine{O}{} Senhor é a minha luz e a minha salvação: a quem, pois, temerei? O Senhor é o defensor da minha vida: quem poderá intimidar-me? Meus inimigos, que me atribulam, enfraqueceram e caíram. \emph{Sl. ibid., 3} Ainda que um exército me cercasse, o meu coração não temeria.
℣. Glória ao Pai \emph{\&c.}
}\end{paracol}

\paragraph{Oração}
\begin{paracol}{2}\latim{
\rlettrine{D}{a} nobis, quǽsumus, Dómine: ut et mundi cursus pacífice nobis tuo órdine dirigátur; et Ecclésia tua tranquílla devotióne lætétur. Per Dóminum \emph{\&c.}
}\switchcolumn\portugues{
\rlettrine{C}{oncedei-nos,} Senhor, Vos suplicamos, que, sob a conduta da vossa providência, os acontecimentos do mundo decorram pacificamente; e que a vossa Igreja goze a alegria de Vos servir com tranquilidade. Por nosso Senhor \emph{\&c.}
}\end{paracol}

\paragraphinfo{Epístola}{Rm. 8, 18-23}
\begin{paracol}{2}\latim{
Léctio Epístolæ beáti Pauli Apóstoli ad Romános.
}\switchcolumn\portugues{
Lição da Ep.ª do B. Ap.º Paulo aos Romanos.
}\switchcolumn*\latim{
\rlettrine{F}{ratres:} Exístimo, quod non sunt condígnæ passiónes hujus témporis ad futúram glóriam, quæ revelábitur in nobis. Nam exspectátio creatúræ revelatiónem filiórum Dei exspéctat. Vanitáti enim creatúra subjécta est, non volens, sed propter eum, qui subjécit eam in spe: quia et ipsa creatúra liberábitur a servitúte corruptiónis, in libertátem glóriæ filiórum Dei. Scimus enim, quod omnis creatúra ingemíscit et párturit usque adhuc. Non solum autem illa, sed et nos ipsi primítias spíritus habéntes: et ipsi intra nos gémimus, adoptiónem filiórum Dei exspectántes, redemptiónem córporis nostri: in Christo Jesu, Dómino nostro.
}\switchcolumn\portugues{
\rlettrine{M}{eus} irmãos: Estou persuadido de que os sofrimentos da vida presente não têm proporção alguma com a glória que nos há-de ser manifestada. Com efeito, as criaturas esperam com viva impaciência a manifestação dos filhos de Deus; pois elas, agora, estão sujeitas à vaidade, não voluntariamente, mas pela vontade daquele que as sujeitou, com a esperança de que seriam livres da servidão da corrupção para participar da liberdade da glória dos filhos de Deus. Na verdade, sabemos que todas as criaturas gemem e estão como que com as dores da maternidade até ao presente. Também nós, que possuímos as primícias do Espírito (que não somente essas), gememos no nosso coração, desejando ardentemente a adopção, como filhos de Deus, e a redenção do corpo, em nosso Senhor Jesus Cristo.
}\end{paracol}

\paragraphinfo{Gradual}{Sl. 78, 9 \& 10}
\begin{paracol}{2}\latim{
\rlettrine{P}{ropítius} esto, Dómine, peccátis nostris: ne quando dicant gentes: Ubi est Deus eórum? ℣. Adjuva nos, Deus, salutáris noster: et propter honórem nóminis tui, Dómine, líbera nos.
}\switchcolumn\portugues{
\rlettrine{P}{erdoai} os nossos pecados, Senhor, para que os povos não digam: Onde está o seu Deus? Auxiliai-nos, ó Senhor, nosso Salvador: e, pela glória do vosso nome, livrai-nos, Senhor.
}\switchcolumn*\latim{
Allelúja, allelúja. ℣. \emph{Ps. 9, 5 \& 10} Deus, qui sedes su per thronum, et júdicas æquitátem: esto refúgium páuperum in tribulatióne. Allelúja.
}\switchcolumn\portugues{
Aleluia, aleluia. ℣. \emph{Sl. 9, 5 \& 10} Ó Deus, que estais assentado no vosso trono e julgais com justiça, sede o refúgio dos pobres na tribulação. Aleluia.
}\end{paracol}

\paragraphinfo{Evangelho}{Lc. 5, 1-11}
\begin{paracol}{2}\latim{
\cruz Sequéntia sancti Evangélii secúndum Lucam.
}\switchcolumn\portugues{
\cruz Continuação do santo Evangelho segundo S. Lucas.
}\switchcolumn*\latim{
\blettrine{I}{n} illo témpore: Cum turbæ irrúerent in Jesum, ut audírent verbum Dei, et ipse stabat secus stagnum Genésareth. Et vidit duas naves stantes secus stagnum: piscatóres autem descénderant et lavábant rétia. Ascéndens autem in unam navim, quæ erat Simónis, rogávit eum a terra redúcere pusíllum. Et sedens docébat de navícula turbas. Ut cessávit autem loqui, dixit ad Simónem: Duc in altum, et laxáte rétia vestra in captúram. Et respóndens Simon, dixit illi: Præcéptor, per totam noctem laborántes, nihil cépimus: in verbo autem tuo laxábo rete. Et cum hoc fecíssent, conclusérunt píscium multitúdinem copiósam: rumpebátur autem rete eórum. Et annuérunt sóciis, qui erant in ália navi, ut venírent et adjuvárent eos. Et venérunt, et implevérunt ambas navículas, ita ut pæne mergeréntur. Quod cum vidéret Simon Petrus, prócidit ad génua Jesu, dicens: Exi a me, quia homo peccátor sum, Dómine. Stupor enim circumdéderat eum et omnes, qui cum illo erant, in captúra píscium, quam céperant: simíliter autem Jacóbum et Joánnem, fílios Zebedǽi, qui erant sócii Simónis. Et ait ad Simónem Jesus: Noli timére: ex hoc jam hómines eris cápiens. Et subdúctis ad terram návibus, relictis ómnibus, secuti sunt eum.
}\switchcolumn\portugues{
\blettrine{N}{aquele} tempo encontrando-se Jesus, nas margens do lago Genesaré, rodeado pela multidão que queria ouvir a sua palavra, viu duas barcas à beira do lago, havendo os pescadores saído delas para lavar as redes. Então, entrou Jesus em uma das barcas, a qual pertencia a Simão, pedindo-lhe que a desviasse um pouco da terra. Depois assentou-se na barca e começou a doutrinar o povo. Quando Jesus acabou de falar, disse a Simão: «Afasta-te para o largo e lança as redes para pescares». Simão respondeu-Lhe: «Mestre, trabalhámos toda a noite e não pescámos nada; mas, obedecendo à vossa palavra, lançarei a rede». E, lançando-a, pescaram tão grande quantidade de peixes, que a rede se rompia! Logo fizeram sinal aos companheiros, que estavam na outra barca, para que fossem auxiliá-los. Foram eles e encheram ambas as barcas; e de tal sorte que quase se afundavam! Simão-Pedro, vendo isto, caiu de joelhos aos pés de Jesus e disse: «Afastai-Vos de mim, Senhor, pois sou um homem pecador!». Simão estava atónito, assim como os companheiros, por causa da pesca que haviam feito! O mesmo acontecia a Tiago e a João, filhos de Zebedeu, que estavam com Simão. Então, Jesus disse a Simão: «Não tenhas receio; doravante serás pescador de homens». E eles, tendo conduzido as barcas para terra, deixaram tudo e seguiram-n’O.
}\end{paracol}

\paragraphinfo{Ofertório}{Sl. 12, 4-5}
\begin{paracol}{2}\latim{
\rlettrine{I}{}llúmina óculos meos, ne umquam obdórmiam in morte: ne quando dicat inimícus meus: Præválui advérsus eum.
}\switchcolumn\portugues{
\rlettrine{I}{luminai} os meus olhos, para que não adormeça na morte e o meu inimigo não diga mais tarde: prevaleci contra ele.
}\end{paracol}

\paragraph{Secreta}
\begin{paracol}{2}\latim{
\rlettrine{O}{blatiónibus} nostris, quǽsumus, Dómine, placáre suscéptis: et ad te nostras étiam rebélles compélle propítius voluntátes. Per Dóminum nostrum \emph{\&c.}
}\switchcolumn\portugues{
\rlettrine{D}{eixai-Vos} aplacar, Senhor, recebendo as nossas oblações; e dignai-Vos propiciamente compelir a nossa vontade rebelde a submeter-se a Vós. Por nosso Senhor \emph{\&c.}
}\end{paracol}

\paragraphinfo{Comúnio}{Sl. 17, 3}
\begin{paracol}{2}\latim{
\rlettrine{D}{óminus} firmaméntum meum, et refúgium meum, et liberátor meus: Deus meus, adjútor meus.
}\switchcolumn\portugues{
\rlettrine{O}{} Senhor é o meu sustentáculo, o meu refúgio e o meu libertador: Ele é o meu Deus e auxílio!
}\end{paracol}

\paragraph{Postcomúnio}
\begin{paracol}{2}\latim{
\rlettrine{M}{ystéria} nos, Dómine, quǽsumus, sumpta puríficent: et suo múnere tueántur. Per Dóminum \emph{\&c.}
}\switchcolumn\portugues{
\rlettrine{V}{os} imploramos, Senhor, que estes mistérios, que recebemos, nos purifiquem; e que pela sua virtude nos sirvam de protecção. Por nosso Senhor \emph{\&c.}
}\end{paracol}

\subsectioninfo{Segunda-feira de Pentecostes}{Estação em São Pedro ad Víncula}

\paragraphinfo{Intróito}{Sl. 80, 17}
\begin{paracol}{2}\latim{
\rlettrine{C}{ibávit} eos ex ádipe fruménti, allelúja: et de petra, melle saturávit eos, allelúja, allelúja. \emph{Ps. ibid., 2} Exsultáte Deo, adjutóri nostro: jubiláte Deo Jacob.
℣. Gloria Patri \emph{\&c.}
}\switchcolumn\portugues{
\rlettrine{A}{limentou-os} Deus com o pão mais puro, aleluia: e saciou-os com o mel que fez brotar duma rocha, aleluia, aleluia. Louvai alegremente o Senhor, que é o nosso sustentáculo; aclamai com hinos de alegria o Deus de Jacob.
℣. Glória ao Pai \emph{\&c.}
}\end{paracol}

\paragraph{Oração}
\begin{paracol}{2}\latim{
\rlettrine{D}{eus,} qui Apóstolis tuis Sanctum dedísti Spíritum: concéde plebi tuæ piæ petitiónis efféctum; ut, quibus dedísti fidem, largiáris et pacem. Per Dóminum \emph{\&c.}
}\switchcolumn\portugues{
\slettrine{Ó}{} Deus, que enviastes o Espírito Santo aos vossos Apóstolos, atendei às pias orações do vosso povo, a fim de que àqueles a quem já destes a fé concedais também a paz. Por nosso Senhor \emph{\&c.}
}\end{paracol}

\paragraphinfo{Epístola}{Act. 10, 34 \& 42-48}
\begin{paracol}{2}\latim{
Léctio Actuum Apostolórum. 
}\switchcolumn\portugues{
Lição dos Actos dos Apóstolos.
}\switchcolumn*\latim{
\rlettrine{I}{n} diébus illis: Apériens Petrus os suum, dixit: Viri fratres, nobis præcépit Dóminus prædicáre pópulo: et testificári, quia ipse est, qui constitútus est a Deo judex vivórum et mortuórum. Huic omnes Prophétæ testimónium pérhibent, remissiónem peccatórum accípere per nomen ejus omnes, qui credunt in eum. Adhuc loquénte Petro verba hæc, cecidit Spíritus Sanctus super omnes, qui audiébant verbum. Et obstupuérunt ex circumcisióne fidéles, qui vénerant cum Petro: quia et in natiónes grátia Spíritus Sancti effúsa est. Audiébant enim illos loquéntes linguis et magnificántes Deum. Tunc respóndit Petrus: Numquid aquam quis prohibére potest, ut non baptizéntur hi, qui Spíritum Sanctum accepérunt sicut et nos? Et jussit eos baptizári in nómine Dómini Jesu Christi.
}\switchcolumn\portugues{
\rlettrine{N}{aqueles} dias, Pedro, tomando a palavra, disse: «Varões, meus irmãos, o Senhor mandou-nos pregar ao povo e testemunhar que foi Ele quem Deus instituiu Juiz dos vivos e dos mortos. Todos os Profetas afirmam a seu respeito que aqueles que acreditarem n’Ele receberão a remissão dos pecados pelo poder do seu nome». Eis que, enquanto Pedro falava, desceu o Espírito Santo sobre aqueles que escutavam as suas palavras. E os fiéis circundados, que tinham vindo com Pedro, admiraram-se de que a graça do Espírito Santo descesse também sobre os pagãos, pois ouviram-nos falar diversas línguas e dar glória a Deus. Então, disse Pedro: «Acaso poder-se-á recusar e Baptismo àqueles que, como vós, receberam o Espírito Santo?». E mandou que fossem baptizados no nome do nosso Senhor Jesus Cristo.
}\end{paracol}

\begin{paracol}{2}\latim{
Allelúja, allelúja. ℣. \emph{Act. 2, 4} Loquebántur váriis linguis Apóstoli magnália Dei. Allelúja. \emph{(Hic genuflectitur)} ℣. Veni, Sancte Spiritus, reple tuorum corda fidélium: et tui amóris in eis ignem accénde.
}\switchcolumn\portugues{
Aleluia, aleluia. ℣. \emph{Act. 2, 4} Os Apóstolos publicavam em diversas línguas as maravilhas de Deus. Aleluia. \emph{(Genuflecte-se)}. Vinde, ó Espírito Santo; enchei os corações dos vossos fiéis e acendei neles o fogo do vosso amor.
}\end{paracol}

\paragraphinfo{Evangelho}{Jo. 3, 16-21}
\begin{paracol}{2}\latim{
\cruz Sequéntia sancti Evangélii secúndum Joánnem.
}\switchcolumn\portugues{
\cruz Continuação do santo Evangelho segundo S. João.
}\switchcolumn*\latim{
\blettrine{I}{n} illo témpore: Dixit Jesus Nicodémo: Sic Deus diléxit mundum, ut Fílium suum unigénitum daret: ut omnis, qui credit in eum, non péreat, sed hábeat vitam ætérnam. Non enim misit Deus Fílium suum in mundum, ut júdicet mundum, sed ut salvétur mundus per ipsum. Qui credit in eum, non judicátur; qui autem non credit, jam judicátus est: quia non credit in nómine unigéniti Fílii Dei. Hoc est autem judícium: quia lux venit in mundum, et dilexérunt hómines magis ténebras quam lucem: erant enim eórum mala ópera. Omnis enim, qui male agit, odit lucem, et non venit ad lucem, ut non arguántur ópera ejus: qui autem facit veritátem, venit ad lucem, ut manifesténtur ópera ejus, quia in Deo sunt facta.
}\switchcolumn\portugues{
\blettrine{N}{aquele} tempo, disse Jesus a Nicodemos: «Deus amou de tal modo o mundo que deu o seu Filho Unigénito, para que todo aquele que acreditar n’Ele não pereça, mas alcance a vida eterna. Porquanto Deus não mandou o seu Filho ao Inundo para condenar o mundo, mas para o salvar por seu intermédio. Aquele que crê n’Elc não é condenado; mas aquele que não crê já está condenado, porque não acreditou no nome do Filho Unigénito de Deus. Eis qual a causa da condenação: é que a luz veio ao mundo, mas os homens amaram mais as trevas do que a luz; e as suas obras foram más. Todo o que procede mal detesta a luz e não aparece à luz, para que suas obras não sejam censuradas. Mas aquele que procede com rectidão aparece à luz, para que se saiba que suas obras são praticadas em nome de Deus».
}\end{paracol}

\paragraphinfo{Ofertório}{Sl. 17, 14 \& 16}
\begin{paracol}{2}\latim{
\rlettrine{I}{ntónuit} de cœlo Dóminus, et Altíssimus dedit vocem suam: et apparuérunt fontes aquárum, allelúja. 
}\switchcolumn\portugues{
\rlettrine{L}{á} do céu o Senhor trovejou: e o Altíssimo fez ouvir a sua voz: então irromperam as fontes das águas, aleluia.
}\end{paracol}

\paragraph{Secreta}
\begin{paracol}{2}\latim{
\rlettrine{P}{ropítius,} Dómine, quǽsumus, hæc dona sanctífica: et, hóstiæ spiritális oblatióne suscépta, nosmetípsos tibi pérfice munus ætérnum. Per Dóminum \emph{\&c.}
}\switchcolumn\portugues{
\rlettrine{D}{ignai-Vos,} Senhor, Vos rogamos, santificar estas ofertas, e, recebendo a oblação desta hóstia espiritual, fazei que nos tornemos em um dom eterno a Vós oferecido. Por nosso Senhor \emph{\&c.}
}\end{paracol}

\paragraphinfo{Comúnio}{Jo. 14, 26}
\begin{paracol}{2}\latim{
\rlettrine{S}{píritus} Sanctus docébit vos, allelúja: quæcúmque díxero vobis, allelúja, allelúja. 
}\switchcolumn\portugues{
\rlettrine{O}{} Espírito Santo vos inspirará, aleluia, tudo o que vos tenho ensinado, aleluia, aleluia.
}\end{paracol}

\paragraph{Postcomúnio}
\begin{paracol}{2}\latim{
\rlettrine{A}{désto,} quǽsumus, Dómine, pópulo tuo: et, quem mystériis cœléstibus imbuísti, ab hóstium furóre defénde. Per Dóminum \emph{\&c.}
}\switchcolumn\portugues{
\rlettrine{A}{ssisti} ao vosso povo, Senhor, Vos suplicamos, e, havendo-o sustentado com os mystérios celestiais, defendei-o contra o furor dos inimigos. Por nosso Senhor \emph{\&c.}
}\end{paracol}

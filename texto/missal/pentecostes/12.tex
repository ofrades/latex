\subsection{Décimo Segundo Domingo depois de Pentecostes}

\paragraphinfo{Intróito}{Sl. 69, 2-3}
\begin{paracol}{2}\latim{
\rlettrine{D}{eus,} in adjutórium meum inténde: Dómine, ad adjuvándum me festína: confundántur et revereántur inimíci mei, qui quærunt ánimam meam. \emph{Ps. ibid., 4} Avertántur retrórsum et erubéscant: qui cógitant mihi mala.
℣. Gloria Patri \emph{\&c.}
}\switchcolumn\portugues{
\slettrine{Ó}{} Deus, vinde em meu auxílio. Apressai-Vos, Senhor, em socorrer-me! Que fiquem confundidos e envergonhados os meus inimigos, que procuram tirar-me a vida. \emph{Sl. ibid., 4} Fujam de mim, cheios de vergonha aqueles que querem a minha perda.
℣. Glória ao Pai \emph{\&c.}
}\end{paracol}

\paragraph{Oração}
\begin{paracol}{2}\latim{
\rlettrine{O}{mnípotens} et miséricors Deus, de cujus múnere venit, ut tibi a fidélibus tuis digne et laudabíliter serviátur: tríbue, quǽsumus, nobis; ut ad promissiónes tuas sine offensióne currámus. Per Dóminum nostrum \emph{\&c.}
}\switchcolumn\portugues{
\slettrine{Ó}{} Deus omnipotente e misericordioso, a quem os fiéis são devedores da felicidade de Vos prestarem culto agradável e digno, concedei-nos a graça, Vos suplicamos, de procurarmos sempre sem qualquer obstáculo os bens que nos prometestes. Por nosso Senhor \emph{\&c.}
}\end{paracol}

\paragraphinfo{Epístola}{2. Cor. 3, 4-9}
\begin{paracol}{2}\latim{
Léctio Epístolæ beáti Pauli Apóstoli ad Corinthios.
}\switchcolumn\portugues{
Lição da Ep.ª do B. Ap.º Paulo aos Coríntios.
}\switchcolumn*\latim{
\rlettrine{F}{ratres:} Fidúciam talem habémus per Christum ad Deum: non quod sufficiéntes simus cogitáre áliquid a nobis, quasi ex nobis: sed sufficiéntia nostra ex Deo est: qui et idóneos nos fecit minístros novi testaménti: non líttera, sed spíritu: líttera enim occídit, spíritus autem vivíficat. Quod si ministrátio mortis, lítteris deformáta in lapídibus, fuit in glória; ita ut non possent inténdere fili Israël in fáciem Moysi, propter glóriam vultus ejus, quæ evacuátur: quómodo non magis ministrátio Spíritus erit in glória? Nam si ministrátio damnátionis glória est multo magis abúndat ministérium justítiæ in glória.
}\switchcolumn\portugues{
\rlettrine{M}{eus} irmãos: Devemos ter toda a confiança em Deus por Cristo, pois não somos capazes nem dum pensamento que tenha origem em nós, mas tudo nos vem de Deus. Com efeito, foi Ele quem nos tornou aptos para sermos ministros da Nova Aliança, não pela letra, mas pelo Espírito; pois a letra mata, mas o Espírito vivifica. Ora, se o ministério da morte, gravado em letras sobre as pedras, foi acompanhado de tal glória que os filhos de Israel não podiam fitar o rosto de Moisés, por causa do brilho que irradiava dele (o qual, contudo, era passageiro), como não será mais glorioso o ministério do Espírito? Se o ministério da condenação foi glorioso, incomparavelmente mais glorioso será o ministério que confere a justiça.
}\end{paracol}

\paragraphinfo{Gradual}{Sl. 33, 2-3}
\begin{paracol}{2}\latim{
\rlettrine{B}{enedícam} Dóminum in omni témpore: semper laus ejus in ore meo. ℣. In Dómino laudábitur ánima mea: áudiant mansuéti, et læténtur.
}\switchcolumn\portugues{
\rlettrine{B}{endirei} o Senhor em todos os tempos! Seus louvores estarão sempre nos meus lábios. ℣. Minha alma será glorificada pelo Senhor. Ouçam isto e alegrem-se aqueles que possuem a mansidão.
}\switchcolumn*\latim{
Allelúja, allelúja. ℣. \emph{Ps. 87, 2} Dómine, Deus salútis meæ, in die clamávi et nocte coram te. Allelúja.
}\switchcolumn\portugues{
Aleluia, aleluia. ℣. \emph{Sl. 87, 2} Senhor, que sois o meu Deus e o meu Salvador, tenho clamado dia e noite diante de Vós.
}\end{paracol}

\paragraphinfo{Evangelho}{Lc. 10, 23-37}
\begin{paracol}{2}\latim{
\cruz Sequéntia sancti Evangélii secúndum Lucam.
}\switchcolumn\portugues{
\cruz Continuação do santo Evangelho segundo S. Lucas.
}\switchcolumn*\latim{
\blettrine{I}{n} illo témpore: Dixit Jesus discípulis suis: Beáti óculi, qui vident quæ vos videtis. Dico enim vobis, quod multi prophétæ et reges voluérunt vidére quæ vos videtis, et non vidérunt: et audire quæ audítis, et non audiérunt. Et ecce, quidam legisperítus surréxit, tentans illum, et dicens: Magister, quid faciéndo vitam ætérnam possidébo? At ille dixit ad eum: In lege quid scriptum est? quómodo legis? Ille respóndens, dixit: Díliges Dóminum, Deum tuum, ex toto corde tuo, et ex tota ánima tua, et ex ómnibus víribus tuis; et ex omni mente tua: et próximum tuum sicut teípsum. Dixítque illi: Recte respondísti: hoc fac, et vives. Ille autem volens justificáre seípsum, dixit ad Jesum: Et quis est meus próximus? Suscípiens autem Jesus, dixit: Homo quidam descendébat ab Jerúsalem in Jéricho, et íncidit in latrónes, qui étiam despoliavérunt eum: et plagis impósitis abiérunt, semivívo relícto. Accidit autem, ut sacerdos quidam descénderet eádem via: et viso illo præterívit. Simíliter et levíta, cum esset secus locum et vidéret eum, pertránsiit. Samaritánus autem quidam iter fáciens, venit secus eum: et videns eum, misericórdia motus est. Et apprópians, alligávit vulnera ejus, infúndens óleum et vinum: et impónens illum in juméntum suum, duxit in stábulum, et curam ejus egit. Et áltera die prótulit duos denários et dedit stabulário, et ait: Curam illíus habe: et quodcúmque supererogáveris, ego cum redíero, reddam tibi. Quis horum trium vidétur tibi próximus fuísse illi, qui íncidit in latrónes? At lle dixit: Qui fecit misericórdiam in illum. Et ait illi Jesus: Vade, et tu fac simíliter.
}\switchcolumn\portugues{
\blettrine{N}{aquele} tempo, disse Jesus aos seus discípulos: «Ditosos os olhos que vêem o que vedes, pois vos digo que muitos profetas e reis quiseram ver o que vedes, e não viram, e desejaram ouvir o que ouvis, e não ouviram». Então um doutor da Lei, ouvindo isto, levantou-se, e, para tentar Jesus, disse-lhe: «Mestre, que hei-de fazer para possuir a vida eterna?». Jesus respondeu-lhe: «Que é que está escrito na Lei? De que modo lês tu?». Respondeu o doutor: «Amareis ao Senhor, vosso Deus, com todo o coração, com toda a alma, com todas as forças e com todo o entendimento; e amar reis ao vosso próximo como a vós mesmos». Jesus disse-lhe: «Respondeste com rectidão. Fale assim, e viverás». Mas o doutor, querendo justificar a sua pergunta, de novo interrogou Jesus: «E quem é o meu próximo?». Então Jesus, tomando a palavra, acrescentou: «Caminhando um homem de Jerusalém para Jericó, caiu em poder dos ladrões, que o despojaram e feriram; depois abandonaram-no semimorto. Passou por aquele caminho um Sacerdote, que viu o ferido e continuou a sua jornada. Passando, também, por ali um Levita, viu igualmente o ferido, aproximou-se um pouco e seguiu a sua viagem. Por fim, viajando por ali um Samaritano, passou perto do ferido, viu-o e encheu-se de compaixão. Logo, aproximou-se dele, deitou óleo e vinho nas feridas, cingiu-as, montou-o no seu próprio jumento, conduziu-o a uma hospedaria e cuidou dele. No outro dia, tirou dois dinheiros da sua bolsa, deu-os ao hospedeiro e disse-lhe: «Cuida deste homem; e tudo o mais que gastares te pagarei eu, quando aqui voltar». «Ora perguntou Jesus ao doutor qual destes três te parece ser o próximo do homem que caiu em poder dos ladrões?». O doutor respondeu: «Aquele que teve misericórdia dele». E Jesus terminou: «Pois tu vai e procede semelhantemente».
}\end{paracol}

\paragraphinfo{Ofertório}{Ex. 32, 11, 13 \& 14}
\begin{paracol}{2}\latim{
\rlettrine{P}{recátus} est Moyses in conspéctu Dómini, Dei sui, et dixit: Quare, Dómine, irascéris in pópulo tuo? Parce iræ ánimæ tuæ: meménto Abraham, Isaac et Jacob, quibus jurásti dare terram fluéntem lac et mel. Et placátus factus est Dóminus de malignitáte, quam dixit fácere pópulo suo.
}\switchcolumn\portugues{
\rlettrine{M}{oisés} orou diante do Senhor, seu Deus, e disse: «Senhor, porque estais irritado contra o vosso povo? Aplacai a vossa ira! Lembrai-Vos de Abraão, de Isaque e de Jacob, a quem jurastes dar a posse da terra, onde. Correm leite e mel». Então o Senhor aplacou-se afastou os males com que ameaçara o povo.
}\end{paracol}

\paragraph{Secreta}
\begin{paracol}{2}\latim{
\rlettrine{H}{óstias,} quǽsumus, Dómine, propítius inténde, quas sacris altáribus exhibémus: ut, nobis indulgéntiam largiéndo, tuo nómini dent honórem. Per Dóminum \emph{\&c.}
}\switchcolumn\portugues{
\rlettrine{D}{ignai-Vos} olhar propício, Senhor, para as hóstias que depomos sobre os vossos sacrossantos altares, a fim de que, obtendo-nos o perdão, sirvam de homenagem ao vosso santo nome. Por nosso Senhor \emph{\&c.}
}\end{paracol}

\paragraphinfo{Comúnio}{Sl. 103, 13 \& 14-15}
\begin{paracol}{2}\latim{
\rlettrine{D}{e} fructu óperum tuórum, Dómine, satiábitur terra: ut edúcas panem de terra, et vinum lætíficet cor hóminis: ut exhílaret fáciem in oleo, et panis cor hóminis confírmet.
}\switchcolumn\portugues{
\rlettrine{A}{} terra ficará saciada com o fruto das vossas obras, Senhor! Fareis brotar da terra o pão, assim como o vinho, que alegra o coração do homem. Dar-lhe-eis azeite, que faz resplandecer o rosto do homem, e pão, que fortifica o seu coração.
}\end{paracol}

\paragraph{Postcomúnio}
\begin{paracol}{2}\latim{
\rlettrine{V}{ivíficet} nos, quǽsumus, Dómine, hujus participátio sancta mystérii: et páriter nobis expiatiónem tríbuat et múnimen. Per Dóminum nostrum \emph{\&c.}
}\switchcolumn\portugues{
\rlettrine{F}{azei,} Senhor, Vos suplicamos, que a comparticipação deste sacrossanto mistério nos sirva igualmente de expiação e de protecção. Por nosso Senhor \emph{\&c.}
}\end{paracol}

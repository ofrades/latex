\subsectioninfo{Terça-feira de Pentecostes}{Estação em Santa Anastácia}

\paragraphinfo{Intróito}{4 Esd. 2, 36 \& 37}
\begin{paracol}{2}\latim{
\rlettrine{A}{ccípite} jucunditátem glóriæ vestræ, allelúja: grátias agéntes Deo, allelúja: qui vos ad cœléstia regna vocávit, allelúja, allelúja, allelúja. \emph{Ps. 77, 1} Atténdite, pópule meus, legem meam: inclináte aurem vestram in verba oris mei.
℣. Gloria Patri \emph{\&c.}
}\switchcolumn\portugues{
\rlettrine{R}{ecebei} com alegria as delícias da vossa glória, aleluia: dai graças a Deus, aleluia: que vos chamou aos reinos celestiais, aleluia, aleluia. \emph{Sl. 77, 1} Escutai a minha lei, ó meu povo: escutai as palavras que saem da minha boca.
℣. Glória ao Pai \emph{\&c.}
}\end{paracol}

\paragraph{Oração}
\begin{paracol}{2}\latim{
\rlettrine{A}{dsit} nobis, quǽsumus, Dómine, virtus Spíritus Sancti: quæ et corda nostra cleménter expúrget, et ab ómnibus tueátur advérsis. Per Dóminum \emph{\&c.}
}\switchcolumn\portugues{
\rlettrine{S}{enhor,} Vos suplicamos, fazei que a virtude do Espírito Santo nos assista; e que pela sua clemência purifique os nossos corações e nos defenda de todas as adversidades. Por nosso Senhor \emph{\&c.}
}\end{paracol}

\paragraphinfo{Epístola}{Act. 8, 14-17}
\begin{paracol}{2}\latim{
Léctio Actuum Apostolórum.
}\switchcolumn\portugues{
Lição dos Actos dos Apóstolos.
}\switchcolumn*\latim{
\rlettrine{I}{n} diébus illis: Cum audíssent Apóstoli, qui erant Jerosólymis, quod recepísset Samaría verbum Dei, misérunt ad eos Petrum et Joánnem. Qui cum veníssent, oravérunt pro ipsis, ut accíperent Spíritum Sanctum: nondum enim in quemquam illórum vénerat, sed baptizáti tantum erant in nómine Dómini Jesu. Tunc imponébant manus super illos, et accipiébant Spíritum Sanctum.
}\switchcolumn\portugues{
\rlettrine{N}{aqueles} dias, quando os Apóstolos estavam em Jerusalém, souberam que a Samaria recebera a palavra de Deus; e por isso enviaram para lá Pedro e João, os quais, logo que chegaram, oraram por eles, para que recebessem o Espírito Santo, que não havia ainda descido sobre nenhum deles, pois somente haviam sido baptizados no nome do Senhor Jesus. Então impuseram-lhes as mãos, recebendo o Espírito Santo.
}\end{paracol}

\begin{paracol}{2}\latim{
Allelúja, allelúja. ℣. \emph{Joann. 14, 26} Spíritus Sanctus docébit vos, quæcúmque díxero vobis. Allelúja. \emph{(Hic genuflectitur)} ℣. Veni, Sancte Spíritus, reple tuórum corda fidélium: et tui amóris in eis ignem accénde.
}\switchcolumn\portugues{
Aleluia, aleluia. ℣. \emph{Jo. 14, 26} O Espírito Santo vos inspirará tudo o que vos tenho ensinado, aleluia. \emph{(Genuflecte-se)} ℣. Vinde, Espírito Santo, enchei os corações dos vossos fiéis e acendei neles o fogo do vosso amor.
}\end{paracol}

\paragraphinfo{Evangelho}{Jo. 10, 1-10}
\begin{paracol}{2}\latim{
\cruz Sequéntia sancti Evangélii secúndum Joánnem.
}\switchcolumn\portugues{
\cruz Continuação do santo Evangelho segundo S. João.
}\switchcolumn*\latim{
\blettrine{I}{n} illo témpore: Dixit Jesus pharisǽis: Amen, amen, dico vobis: qui non intrat per óstium in ovíle óvium, sed ascéndit aliúnde, ille fur est et latro. Qui autem intrat per óstium, pastor est óvium. Huic ostiárius áperit, et oves vocem ejus áudiunt, et próprias oves vocat nominátim et e ducit eas. Et cum próprias oves emíserit, ante eas vadit: et oves illum sequúntur, quia sciunt vocem ejus. Aliénum autem non sequúntur, sed fúgiunt ab eo; quia non novérunt vocem alienórum. Hoc provérbium dixit eis Jesus. Illi autem non cognovérunt, quid loquerétur eis. Dixit ergo eis íterum Jesus: Amen, amen, dico vobis, quia ego sum óstium óvium. Omnes, quotquot venérunt, fures sunt et latrónes, et non audiérunt eos oves. Ego sum. óstium. Per me si quis introíerit, salvábitur: et ingrediétur et egrediátur et páscua invéniet. Fur non venit, nisi ut furétur et mactet et perdat. Ego veni, ut vitam hábeant et abundántius hábeant.
}\switchcolumn\portugues{
\blettrine{N}{aquele} tempo, disse Jesus aos fariseus: «Em verdade, em verdade vos digo: aquele que não entra pela porta no estábulo das ovelhas, mas entra por outro lugar, é ladrão e salteador. O pastor das ovelhas entra pela porta. A este abre o porteiro a porta e as ovelhas ouvem a sua voz. Ele chama as ovelhas pelos seus próprios nomes e fá-las sair. E, quando sai com elas, vai adiante e elas seguem-no, pois conhecem a sua voz. Mas não acompanharão um estranho; antes fugirão dele, porque não conhecem a voz dos estranhos». Jesus disse-lhes esta parábola, mas eles não compreenderam o seu sentido. E Jesus disse-lhes novamente: «Em verdade, em verdade vos digo: Eu sou a porta das ovelhas. Todos quantos vieram são ladrões e salteadores, por isso as minhas ovelhas os não ouviram. Eu sou a porta. Se alguém entrar por mim, será salvo: entrará e sairá e encontrará alimento. O ladrão não vem senão para devorar, roubar e destruir. Eu, porém, venho para que minhas ovelhas possuam a vida, e a possuam com abundância».
}\end{paracol}

\paragraphinfo{Ofertório}{Sl. 77, 23-25}
\begin{paracol}{2}\latim{
\rlettrine{P}{ortas} cœli aperuit Dóminus: et pluit illis manna, ut éderent: panem cœli dedit eis, panem Angelórum manducávit homo, allelúja.
}\switchcolumn\portugues{
\rlettrine{O}{} Senhor abriu as portas do céu: fez chover maná, para que o comessem: e deu-lhes o pão do céu. O homem comeu o pão dos Anjos, aleluia,
}\end{paracol}

\paragraph{Secreta}
\begin{paracol}{2}\latim{
\rlettrine{P}{uríficet} nos, quǽsumus. Dómine, múneris præséntis oblátio: et dignos sacra participatióne effíciat. Per Dóminum nostrum \emph{\&c.}
}\switchcolumn\portugues{
\rlettrine{S}{enhor,} Vos suplicamos, fazei que a oferta deste sacrifício nos purifique e nos torne dignos de participarmos deste mistério sagrado. Por nosso Senhor \emph{\&c.}
}\end{paracol}

\paragraphinfo{Comúnio}{Jo. 15, 26; 16, 14; 17, 1 \& 5}
\begin{paracol}{2}\latim{
\rlettrine{S}{píritus} qui a Patre procédit, allelúja: ille me clarificábit, allelúja, allelúja.
}\switchcolumn\portugues{
\rlettrine{O}{} Espírito, que procede do Pai, aleluia, glorificar-me-á, aleluia, aleluia.
}\end{paracol}

\paragraph{Postcomúnio}
\begin{paracol}{2}\latim{
\rlettrine{M}{entes} nostras, quǽsumus, Dómine, Spíritus Sanctus divínis réparet sacraméntis: quia ipse est remíssio ómnium peccatórum. Per Dóminum \emph{\&c.}
}\switchcolumn\portugues{
\rlettrine{S}{enhor,} Vos suplicamos, permiti que o Espírito Santo restaure com estes divinos sacramentos as nossas almas, pois Ele é a remissão de todos os pecados. Por nosso Senhor \emph{\&c.}
}\end{paracol}

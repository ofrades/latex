\subsection{Domingo de Pentecostes}\label{domingopentecostes}

\paragraphinfo{Intróito}{Sb. 1, 7}
\begin{paracol}{2}\latim{
\rlettrine{S}{píritus} Dómini replévit orbem terrárum, allelúja: et hoc quod cóntinet ómnia, sciéntiam habet vocis, allelúja, allelúja, allelúja. \emph{Ps. 67, 2} Exsúrgat Deus, et dissipéntur inimíci ejus: et fúgiant, qui odérunt eum, a fácie ejus.
℣. Gloria Patri \emph{\&c.}
}\switchcolumn\portugues{
\rlettrine{O}{} Espírito do Senhor encheu todo o universo, aleluia. Ele, que abrange todas as coisas, sabe tudo quanto se diz: aleluia, aleluia. \emph{Sl. 67, 2} Erga-se Deus e sejam dispersos os seus inimigos: fujam da sua presença os seus inimigos!
℣. Glória ao Pai \emph{\&c.}
}\end{paracol}

\paragraph{Oração}
\begin{paracol}{2}\latim{
\rlettrine{D}{eus,} qui hodiérna die corda fidélium Sancti Spíritus illustratióne docuísti: da nobis in eódem Spíritu recta sápere; et de ejus semper consolatióne gaudére. Per Dóminum \emph{\&c.}
}\switchcolumn\portugues{
\slettrine{Ó}{} Deus, que neste dia instruístes os corações dos fiéis, infundindo-lhes os dons do Espírito Santo, concedei-nos pelo mesmo Espírito a graça de amarmos o que é recto e de gozarmos incessantemente as suas consolações. Pelo mesmo \emph{\&c.}
}\end{paracol}

\paragraphinfo{Epístola}{Act. 2, 1-11}
\begin{paracol}{2}\latim{
Léctio Actuum Apostolórum.
}\switchcolumn\portugues{
Lição dos Actos dos Apóstolos.
}\switchcolumn*\latim{
\rlettrine{C}{um} compleréntur dies Pentecóstes, erant omnes discípuli pariter in eódem loco: et factus est repéente de cœlo sonus, tamquam adveniéntis spíritus veheméntis: et replévit totam domum, ubi erant sedentes. Et apparuérunt illis dispertítæ linguæ tamquam ignis, sedítque supra síngulos eórum: et repléti sunt omnes Spíritu Sancto, et cœpérunt loqui váriis linguis, prout Spíritus Sanctus dabat éloqui illis. Erant autem in Jerúsalem habitántes Judǽi, viri religiósi ex omni natióne, quæ sub cœlo est. Facta autem hac voce, convénit multitúdo, et mente confúsa est, quóniam audiébat unusquísque lingua sua illos loquéntes. Stupébant autem omnes et mirabántur, dicéntes: Nonne ecce omnes isti, qui loquúntur, Galilǽi sunt? Et quómodo nos audívimus unusquísque linguam nostram, in qua nati sumus? Parthi et Medi et Ælamítæ et qui hábitant Mesopotámiam, Judǽam et Cappadóciam, Pontum et Asiam, Phrýgiam et Pamphýliam, Ægýptum et partes Líbyæ, quæ est circa Cyrénen, et ádvenæ Románi, Judǽi quoque et Prosélyti, Cretes et Arabes: audívimus eos loquéntes nostris linguis magnália Dei.
}\switchcolumn\portugues{
\qlettrine{Q}{uando} chegou o dia de Pentecostes, os discípulos estavam todos reunidos no mesmo lugar. De repente, ouviu-se um ruído, vindo do céu, semelhante a um vento impetuoso, que encheu toda a casa onde estavam reunidos. Então, apareceram-lhes línguas de fogo, que se dividiram umas das outras e pousaram sobre cada um deles, ficando ao mesmo tempo cheios de Espírito Santo e começando a falar várias línguas, corno o Espírito Santo lhes concedia que falassem. Estavam, então, em Jerusalém, judeus e homens religiosos de todas as nações, que existem na terra. Logo que este ruído foi ouvido correram muitas pessoas em multidão ao lugar e ficaram admiradas, porque cada uma ouvia os discípulos falar na sua própria língua. E as pessoas estavam de tal modo fora de si e maravilhadas que diziam: «Porventura estes, que nos falam, não são galileus? Como é, pois, que os ouvimos falar, a cada um de nós, na língua do nosso país natal? Partos, Medos, Elamitas, habitantes da Mesopotâmia, da Judeia, da Capadócia, do Ponto e da Ásia, da Frígia e da Panfília, do Egipto, dos confins da Líbia, vizinha da Cirene, até os romanos de passagem e também os judeus e os prosélitos, os de Creta e da Arábia ouvimos os Apóstolos, nas nossas próprias línguas, encarecer as grandezas de Deus».
}\end{paracol}

\begin{paracol}{2}\latim{
Allelúja, allelúja. ℣. \emph{Ps. 103, 30} Emítte Spíritum tuum, et creabúntur, et renovábis fáciem terræ. Allelúja. \emph{(Hic genuflectitur)} ℣. Veni, Sancte Spíritus, reple tuórum corda fidélium: et tui amóris in eis ignem accénde.
}\switchcolumn\portugues{
Aleluia, aleluia. ℣. \emph{Sl. 103, 30} Enviai o vosso Espírito, e uma nova criação se operará: e renovareis a face da terra. Aleluia. \emph{(Genuflecte-se)} ℣. Vinde, ó Espírito Santo; enchei os corações dos vossos fiéis e acendei neles o fogo do vosso amor.
}\end{paracol}

\subsubsection{Sequência}
\begin{paracol}{2}\latim{
\rlettrine{V}{eni,} Sancte Spíritus, et emítte cǽlitus lucis tuæ rádium. Veni, pater páuperum; veni, dator rnúnerum; veni, lumen córdium. Consolátor óptime, dulcis hospes ánimæ, dulce refrigérium. In labóre réquies, in æstu tempéries, in fletu solácium. O lux beatíssima, reple cordis íntima tuórum fidélium. Sine tuo númine nihil est in hómine, nihil est innóxium. Lava quod est sórdidum, riga quod est áridum, sana quod est sáucium. Flecte quod est rígidum, fove quod est frígidum, rege quod est dévium. Da tuis fidélibus, in te confidéntibus, sacrum septenárium. Da virtútis méritum, da salútis éxitum, da perénne gáudium. Amen. Allelúja.
}\switchcolumn\portugues{
\rlettrine{V}{inde,} ó Espírito Santo, e enviai do céu um raio da vossa divina luz. Vinde, ó pai dos pobres; vinde, distribuidor de todos os dons; vinde, luz dos corações. Consolador supremo, doce hóspede da alma, suave refrigério! No trabalho sois repouso; sois calma no ardor; e consolação no pranto. Ó luz felicíssima, enchei até ao íntimo os corações dos vossos fiéis. Sem a vossa assistência nada há bom no homem, nada há que seja puro. Lavai, pois, o que está manchado; regai o que está seco; curai o que está enfermo. Abrandai o que é duro; aquecei o que está frio; guiai o que anda errado. Concedei aos fiéis, que em Vós confiam, os sete dons sagrados. Dai-lhes o mérito da virtude; dai-lhes um fim feliz; dai-lhes o gozo eterno. Ameno Aleluia.
}\end{paracol}

\paragraphinfo{Evangelho}{Jo. 14, 23-31}
\begin{paracol}{2}\latim{
\cruz Sequéntia sancti Evangélii secúndum Joánnem.
}\switchcolumn\portugues{
\cruz Continuação do santo Evangelho segundo S. João.
}\switchcolumn*\latim{
\blettrine{I}{n} illo témpore: Dixit Jesus discípulis suis: Si quis díligit me, sermónem meum servábit, et Pater meus díliget eum, et ad eum veniémus et mansiónem apud eum faciémus: qui non díligit me, sermónes meos non servat. Et sermónem quem audístis, non est meus: sed ejus, qui misit me, Patris. Hæc locútus sum vobis, apud vos manens. Paráclitus autem Spíritus Sanctus, quem mittet Pater in nómine meo, ille vos docébit ómnia et súggeret vobis ómnia, quæcúmque díxero vobis. Pacem relínquo vobis, pacem meam do vobis: non quómodo mundus dat, ego do vobis. Non turbátur cor vestrum neque formídet. Audístis, quia ego dixi vobis: Vado et vénio ad vos. Si diligere tis me, gaudere tis utique, quia vado ad Patrem: quia Pater major me est. Et nunc dixi vobis, priúsquam fiat: ut, cum factum fúerit, credátis. Jam non multa loquar vobíscum. Venit enim princeps mundi hujus, et in me non habet quidquam. Sed ut cognóscat mundus, quia díligo Patrem, et sicut mandátum dedit mihi Pater, sic fácio.
}\switchcolumn\portugues{
\blettrine{N}{aquele} tempo, disse Jesus aos discípulos: «Se alguém me ama, guardará a minha doutrina; e meu Pai o amará. E viremos a ele e nele faremos nossa morada. Quem me não ama, não guarda os meus ensinos». A doutrina que ouvis não é minha, mas daquele que me enviou, isto é, do Pai. Disse-vos todas estas coisas, enquanto estive convosco. Porém o Paráclito o Espírito Santo que o Pai vos enviará em meu nome, ensinar-vos-á todas as coisas e vos inspirará tudo o que vos ensinei. Deixo-vos a paz; dou-vos a minha paz; mas a paz que vos dou não é como a que o mundo dá. Que o vosso coração não se perturbe, nem se intimide. Ouvistes o que disse: «Eu vou, mas depois volto». Se me amais regozijar-vos-eis por Eu ir ao Pai, pois o Pai é maior do que Eu. Digo-vos estas coisas, antes de acontecerem, para que, quando acontecerem, acrediteis nelas. Já não falarei mais convosco, pois eis que vem o príncipe deste mundo. Ele não tem poder em mim, mas é para que o mundo conheça que amo o Pai e que procedo conforme o que o Pai me mandou.
}\end{paracol}

\paragraphinfo{Ofertório}{Sl. 67, 29-30}
\begin{paracol}{2}\latim{
\rlettrine{C}{onfírma} hoc, Deus, quod operátus es in nobis: a templo tuo, quod est in Jerúsalem, tibi ófferent reges múnera, allelúja.
}\switchcolumn\portugues{
\rlettrine{C}{onfirmai,} ó Deus, o que em nós começastes. Lá, no vosso templo, em Jerusalém, os reis oferecer-Vos-ão suas dádivas, aleluia.
}\end{paracol}

\paragraph{Secreta}
\begin{paracol}{2}\latim{
\rlettrine{M}{únera,} quǽsumus, Dómine, obláta sanctífica: et corda nostra Sancti Spíritus illustratióne emúnda. Per Dóminum \emph{\&c.} in unitáte ejusdem \emph{\&c.}
}\switchcolumn\portugues{
\rlettrine{S}{antificai,} Senhor, Vos suplicamos, os dons que Vos oferecemos, e purificai os nossos corações com a luz do Espírito Santo. Por nosso Senhor \emph{\&c.} em unidade do mesmo \emph{\&c.}
}\end{paracol}

\paragraphinfo{Comúnio}{Act. 2, 2 \& 4}
\begin{paracol}{2}\latim{
\rlettrine{F}{actus} est repénte de cœlo sonus, tamquam adveniéntis spíritus veheméntis, ubi erant sedéntes, allelúja: et repléti sunt omnes Spíritu Sancto, loquéntes magnália Dei, allelúja, allelúja.
}\switchcolumn\portugues{
\rlettrine{D}{e} repente, ouviu-se, vindo do céu, um ruído, semelhante a um vento impetuoso, que encheu a casa onde estavam reunidos, aleluia: e ficaram cheios do Espírito Santo, anunciando as maravilhas de Deus, aleluia.
}\end{paracol}

\paragraph{Postcomúnio}
\begin{paracol}{2}\latim{
\rlettrine{S}{ancti} Spíritus, Dómine, corda nostra mundet infúsio: et sui roris íntima aspersióne fecúndet. Per Dóminum \emph{\&c.} in unitáte ejusdem \emph{\&c.}
}\switchcolumn\portugues{
\rlettrine{S}{enhor,} que a efusão do Espírito Santo purifique os nossos corações e que, penetrando neles, a aspersão do seu orvalho fecunde o íntimo das nossas almas. Por nosso Senhor \emph{\&c.} em unidade do mesmo \emph{\&c.}
}\end{paracol}

\subsection{Décimo Oitavo Domingo depois de Pentecostes}

\paragraphinfo{Intróito}{Ecl. 36, 18}
\begin{paracol}{2}\latim{
\rlettrine{D}{a} pacem, Dómine, sustinéntibus te, ut prophétæ tui fidéles inveniántur: exáudi preces servi tui et plebis tuæ Israël. \emph{Ps. 121, 1} Lætátus sum in his, quæ dicta sunt mihi: in domum Dómini íbimus.
℣. Gloria Patri \emph{\&c.}
}\switchcolumn\portugues{
\rlettrine{D}{ai} paz, Senhor, àqueles que em Vós esperam, a fim de que os vossos Profetas sejam julgados fiéis: Ouvi as preces de Israel, vosso servo e vosso povo. \emph{Sl. 121, 1} Alegrei-me com aqueles que me disseram estas palavras: iremos para a casa do Senhor.
℣. Glória ao Pai \emph{\&c.}
}\end{paracol}

\paragraph{Oração}
\begin{paracol}{2}\latim{
\rlettrine{D}{írigat} corda nostra, quǽsumus, Dómine, tuæ miseratiónis operátio: quia tibi sine te placére non póssumus. Per Dóminum \emph{\&c.}
}\switchcolumn\portugues{
\rlettrine{D}{ignai-Vos,} Senhor, Vos imploramos, dirigir os nossos corações segundo as inspirações da vossa misericórdia, pois sem Vós não podemos agradar-Vos. Por nosso Senhor \emph{\&c.}
}\end{paracol}

\paragraphinfo{Epístola}{1. Cor. 1, 4-8}
\begin{paracol}{2}\latim{
Léctio Epístolæ beáti Pauli Apóstoli ad Corinthios.
}\switchcolumn\portugues{
Lição da Ep.ª do B. Ap.º Paulo aos Coríntios.
}\switchcolumn*\latim{
\rlettrine{F}{ratres:} Grátias ago Deo meo semper pro vobis in grátia Dei, quæ data est vobis in Christo Jesu: quod in ómnibus dívites facti estis in illo, in omni verbo et in omni sciéntia: sicut testimónium Christi confirmátum est in vobis: ita ut nihil vobis desit in ulla grátia, exspectántibus revelatiónem Dómini nostri Jesu Christi, qui et confirmábit vos usque in finem sine crímine, in die advéntus Dómini nostri Jesu Christi.
}\switchcolumn\portugues{
\rlettrine{M}{eus} irmãos: Rendo contínuas acções de graças a Deus por vós, visto que vos foi dada a graça de Deus, em Jesus Cristo. Pois pela vossa união com Cristo ficastes repletos de riquezas espirituais, particularmente pelo que respeita ao dom da palavra e da ciência. O testemunho, que vos tinha sido dado de Jesus Cristo, foi assim confirmado em vós, de modo que não cedeis a ninguém em nenhum dom da graça, esperando com confiança a manifestação de nosso Senhor. E ainda até ao fim Deus vos confirmará, para que vos encontre irrepreensíveis no dia da vinda de N. S. Jesus Cristo.
}\end{paracol}

\paragraphinfo{Gradual}{Sl. 121, 1 \& 7}
\begin{paracol}{2}\latim{
\rlettrine{L}{ætátus} sum in his, quæ dicta sunt mihi: in domum Dómini íbimus. ℣. Fiat pax in virtúte tua: et abundántia in túrribus tuis.
}\switchcolumn\portugues{
\rlettrine{A}{legro-me} com aqueles que me disseram estas palavras: iremos para a casa do Senhor. Que a paz reine nas tuas fortalezas e a abundância nos teus palácios.
}\switchcolumn*\latim{
Allelúja, allelúja. ℣. \emph{Ps. 101, 16} Timébunt gentes nomen tuum, Dómine, et omnes reges terræ glóriam tuam. Allelúja.
}\switchcolumn\portugues{
Aleluia, aleluia. ℣. \emph{Sl. 101, 16} As nações, Senhor, temerão o vosso nome e os reis da terra contemplarão a vossa glória. Aleluia.
}\end{paracol}

\paragraphinfo{Evangelho}{Mt. 9, 1-8}
\begin{paracol}{2}\latim{
\cruz Sequéntia sancti Evangélii secúndum Matthǽum.
}\switchcolumn\portugues{
\cruz Continuação do santo Evangelho segundo S. Mateus.
}\switchcolumn*\latim{
\blettrine{I}{n} illo témpore: Ascéndens Jesus in navículam, transfretávit et venit in civitátem suam. Et ecce, offerébant ei paralýticum jacéntem in lecto. Et videns Jesus fidem illórum, dixit paralýtico: Confíde, fili, remittúntur tibi peccáta tua. Et ecce, quidam de scribis dixérunt intra se: Hic blasphémat. Et cum vidísset Jesus cogitatiónes eórum, dixit: Ut quid cogitátis mala in córdibus vestris? Quid est facílius dícere: Dimittúntur tibi peccáta tua; an dícere: Surge et ámbula? Ut autem sciátis, quia Fílius hóminis habet potestátem in terra dimitténdi peccáta, tunc ait paralýtico: Surge, tolle lectum tuum, et vade in domum tuam. Et surréxit et ábiit in domum suam. Vidéntes autem turbæ timuérunt, et glorificavérunt Deum, qui dedit potestátem talem homínibus.
}\switchcolumn\portugues{
\blettrine{N}{aquele} tempo, subindo Jesus para uma barca, passou o lago e foi para a cidade de Cafarnaum, onde Lhe apresentaram um paralítico, deitado num leito. Vendo Jesus a fé de todos, disse ao paralítico: «Tem confiança, meu filho, os teus pecados ficam-te perdoados». Logo alguns dos escribas disseram intimamente: «Este homem blasfema!». Jesus, conhecendo os seus pensamentos, disse-lhes: «Para que cogitais mal nos vossos corações? Qual é mais fácil dizer: Os teus pecados ficam-te perdoados, ou levanta-te e caminha? Pois, para que conheçais que o Filho do homem tem na terra poder de perdoar os pecados, levanta-te (continuou Ele, dirigindo-se ao paralítico), leva a tua cama e vai para tua casal». E o paralítico levantou-se e foi para casa. Vendo as turbas do povo este acontecimento, ficaram cheias de temor e deram glória a Deus, que havia dado tal poder aos homens!
}\end{paracol}

\paragraphinfo{Ofertório}{Ex. 24, 4 \& 5}
\begin{paracol}{2}\latim{
\rlettrine{S}{anctificávit} Móyses altáre Dómino, ófferens super illud holocáusta et ímmolans víctimas: fecit sacrifícium vespertínum in odórem suavitátis Dómino Deo, in conspéctu filiórum Israël.
}\switchcolumn\portugues{
\rlettrine{M}{oisés} consagrou um altar ao Senhor, sobre o qual ofereceu holocaustos e imolou vítimas: e na presença dos filhos de Israel celebrou o sacrifício da tarde em odor de suavidade ao Senhor.
}\end{paracol}

\paragraph{Secreta}
\begin{paracol}{2}\latim{
\rlettrine{D}{eus,} qui nos, per hujus sacrifícii veneránda commércia, uníus summæ divinitátis partícipes éfficis: præsta, quǽsumus; ut, sicut tuam cognóscimus veritátem, sic eam dignis móribus assequámur. Per Dóminum \emph{\&c.}
}\switchcolumn\portugues{
\slettrine{Ó}{} Deus, que, pela augusta mudança que este sacrifício estabelece entre Vós e nós, nos fazeis participantes da vossa divindade una e soberana, permiti, Vos imploramos, que, conhecendo a vossa verdade, mostremos possuí-la nos nossos costumes dignos. Por nosso Senhor \emph{\&c.}
}\end{paracol}

\paragraphinfo{Comúnio}{Sl. 95. 8-9}
\begin{paracol}{2}\latim{
\rlettrine{T}{óllite} hóstias, et introíte in átria ejus: adoráte Dóminum in aula sancta ejus.
}\switchcolumn\portugues{
\rlettrine{T}{razei} vossas hóstias e entrai nos átrios do Senhor: adorai o Senhor no seu santo templo.
}\end{paracol}

\paragraph{Postcomúnio}
\begin{paracol}{2}\latim{
\rlettrine{G}{rátias} tibi reférimus, Dómine, sacro múnere vegetáti: tuam misericórdiam deprecántes; ut dignos nos ejus participatióne perfícias. Per Dóminum \emph{\&c.}
}\switchcolumn\portugues{
\rlettrine{V}{os} damos graças, Senhor, por nos haverdes alimentado com o sacratíssimo dom; e imploramos da vossa misericórdia que nos torneis dignos desta participação. Por nosso Senhor \emph{\&c.}
}\end{paracol}

\subsection{Quarto Domingo depois da Epifania}

\paragraphinfo{Intróito}{Jr. 29,11,12 \& 14}
\begin{paracol}{2}\latim{
\rlettrine{D}{icit} Dóminus: Ego cógito cogitatiónes pacis, et non afflictiónis: invocábitis me, et ego exáudiam vos: et redúcam captivitátem vestram de cunctis locis. \emph{Ps. 84, 2} Benedixísti, Dómine, terram tuam: avertísti captivitátem Jacob.
℣. Gloria Patri \emph{\&c.}
}\switchcolumn\portugues{
\rlettrine{D}{isse} o Senhor: tenho pensamentos de paz e não de ira: invocar-me-eis e ouvir-vos-ei; e farei regressar de todos os países os vossos cativos. \emph{Sl. 84, 2} Abençoastes, Senhor, a vossa terra e livrastes Jacob do cativeiro.
℣. Glória ao Pai \emph{\&c.}
}\end{paracol}

\paragraph{Oração}
\begin{paracol}{2}\latim{
\rlettrine{D}{eus,} qui nos, in tantis perículis constitútos, pro humána scis fragilitáte non posse subsístere: da nobis salútem mentis et córporis; ut ea, quæ pro peccátis nostris pátimur, te adjuvánte vincámus. Per Dóminum \emph{\&c.}
}\switchcolumn\portugues{
\slettrine{Ó}{} Deus, que conheceis não poder a fraqueza humana subsistir no meio de tantos perigos, que nos cercam, concedei-nos a saúde da alma e do corpo, a fim de que com vosso auxílio possamos vencer os males que devemos sofrer em castigo dos pecados. Por nosso Senhor \emph{\&c.}
}\end{paracol}

\paragraphinfo{Epístola}{Rm. 13, 8-10}
\begin{paracol}{2}\latim{
Léctio Epístolæ beáti Pauli Apóstoli ad Romános.
}\switchcolumn\portugues{
Lição da Ep.ª do B. Ap.º Paulo aos Romanos.
}\switchcolumn*\latim{
\rlettrine{F}{ratres:} Némini quidquam debeátis, nisi ut ínvicem diligátis: qui enim díligit próximum, legem implévit. Nam: Non adulterábis, Non occídes, Non furáberis, Non falsum testimónium dices, Non concupísces: et si quod est áliud mandátum, in hoc verbo instaurátur: Díliges próximum tuum sicut teípsum. Diléctio próximi malum non operátur. Plenitúdo ergo legis est diléctio.
}\switchcolumn\portugues{
\rlettrine{M}{eus} irmãos: Não sejais devedores a ninguém de cousa alguma, senão do amor que deveis uns aos outros; pois aquele que ama o seu próximo, cumpre a lei. Com efeito, estes Mandamentos de Deus: não cometerás adultério; não matarás; não roubarás; não levantarás falso testemunho; não cobiçarás as cousas alheias; e todos os outros Mandamentos, que há, resumem-se nestas palavras: «Amarás ao teu próximo como a ti próprio». O amor ao próximo não permite que se lhe faça mal. O amor é, portanto, a plenitude da lei.
}\end{paracol}

\paragraphinfo{Gradual}{Sl. 43, 8-9}
\begin{paracol}{2}\latim{
\rlettrine{L}{iberásti} nos, Dómine, ex affligéntibus nos: et eos, qui nos odérunt, confudísti. ℣. In Deo laudábimur tota die, et in nómine tuo confitébimur in sǽcula.
}\switchcolumn\portugues{
\rlettrine{L}{ivrastes-nos,} Senhor, daqueles que nos afligiam: e confundistes os que nos odiavam. ℣. Glorificar-nos-emos constantemente em Deus e louvaremos eternamente o vosso nome.
}\switchcolumn*\latim{
Allelúja, allelúja. ℣. \emph{Ps. 129, 12} De profúndis clamávi ad te, Dómine: Dómine, exáudi oratiónem meam. Allelúja.
}\switchcolumn\portugues{
Aleluia, aleluia. ℣. \emph{Sl. 129, 12} Do fundo do abysmo Vos invoquei, Senhor: escutai a minha oração. Aleluia.
}\end{paracol}

\paragraphinfo{Evangelho}{Mt. 8, 23-27}
\begin{paracol}{2}\latim{
\cruz Sequéntia sancti Evangélii secúndum Matthǽum.
}\switchcolumn\portugues{
\cruz Continuação do santo Evangelho segundo S. Mateus.
}\switchcolumn*\latim{
\blettrine{I}{n} illo témpore: Ascendénte Jesu in navículam, secúti sunt eum discípuli ejus: et ecce, motus magnus factus est in mari, ita ut navícula operirétur flúctibus, ipse vero dormiébat. Et accessérunt ad eum discípuli ejus, et suscitavérunt eum, dicéntes: Dómine, salva nos, perímus. Et dicit eis Jesus: Quid tímidi estis, módicæ fídei? Tunc surgens, imperávit ventis et mari, et facta est tranquíllitas magna. Porro hómines miráti sunt, dicéntes: Qualis est hic, quia venti et mare obœdiunt ei?
}\switchcolumn\portugues{
\blettrine{N}{aquele} tempo, Jesus entrou em uma barca, sendo acompanhado pelos seus discípulos. E eis que uma grande tempestade surgiu no mar, de modo que as ondas cobriam a barca. Jesus dormia. Seus discípulos aproximaram-se, então, de Jesus, dizendo: «Senhor, salvai-nos, pois perecemos!». Jesus disse-lhes: «Porque receais, homens de pouca fé?». E, erguendo-se, impôs a sua vontade aos ventos e ao mar; e fez-se uma grande bonança. E aqueles homens admiraram-se, dizendo: «Quem é Este que até os ventos e o mar Lhe obedecem?
}\end{paracol}

\paragraphinfo{Ofertório}{Sl. 129, 1-2}
\begin{paracol}{2}\latim{
\rlettrine{D}{e} profúndis clamávi ad te, Dómine: Dómine, exáudi oratiónem meam: de profúndis clamávi ad te. Dómine.
}\switchcolumn\portugues{
\rlettrine{D}{as} profundezas dos abysmos Vos invoquei, Senhor; escutai, Senhor, a minha voz: das profundezas dos abysmos Vos invoquei.
}\end{paracol}

\paragraph{Secreta}
\begin{paracol}{2}\latim{
\rlettrine{C}{oncéde,} quǽsumus, omnípotens Deus: ut hujus sacrifícii munus oblátum fragilitátem nostram ab omni malo purget semper et múniat. Per Dóminum \emph{\&c.}
}\switchcolumn\portugues{
\slettrine{Ó}{} Deus omnipotente, Vos suplicamos, fazei que a hóstia oferecida neste sacrifício livre a nossa fraqueza de todo o mal e a fortifique para o futuro. Por nosso Senhor \emph{\&c.}
}\end{paracol}

\paragraphinfo{Comúnio}{Mc. 11, 24}
\begin{paracol}{2}\latim{
\rlettrine{A}{men,} dico vobis, quidquid orántes pétitis, crédite, quia accipiétis, et fiet vobis.
}\switchcolumn\portugues{
\rlettrine{N}{a} verdade vos digo: «Tudo o que pedirdes nas vossas orações, acreditai que o recebereis; e far-se-á como pedirdes».
}\end{paracol}

\paragraph{Postcomúnio}
\begin{paracol}{2}\latim{
\rlettrine{M}{únera} tua nos, Deus, a delectatiónibus terrenis expédiant: et cœléstibus semper instáurent aliméntis. Per Dóminum \emph{\&c.}
}\switchcolumn\portugues{
\slettrine{Ó}{} Deus, que estes vossos dons nos afastem dos gozos terrenos e nos restaurem com seu alimento celestial. Por nosso Senhor \emph{\&c.}
}\end{paracol}

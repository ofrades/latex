\subsection{Sexto Domingo depois de Pentecostes}

\paragraphinfo{Intróito}{Sl. 27, 8-9}
\begin{paracol}{2}\latim{
\rlettrine{D}{óminus} fortitudo plebis suæ, et protéctor salutárium Christi sui est: salvum fac pópulum tuum, Dómine, et benedic hereditáti tuæ, et rege eos usque in sǽculum. \emph{Ps. ibid., 1} Ad te, Dómine, clamábo, Deus meus, ne síleas a me: ne quando táceas a me, et assimilábor descendéntibus in lacum.
℣. Gloria Patri \emph{\&c.}
}\switchcolumn\portugues{
\rlettrine{O}{} Senhor é a fortaleza do seu povo e o protector e salvador do seu Cristo. Salvai o vosso povo, Senhor: abençoai a vossa herança: e governai-os até à eternidade. \emph{Sl. ibid., 1} Clamei por Vós, Senhor: Ó meu Deus, não fecheis os ouvidos, afastando-Vos de mim, pois, se me não escutais, tornar-me-ei semelhante àqueles que caem no abysmo.
℣. Glória ao Pai \emph{\&c.}
}\end{paracol}

\paragraph{Oração}
\begin{paracol}{2}\latim{
\rlettrine{D}{eus} virtútum, cujus est totum quod est óptimum: ínsere pectóribus nostris amórem tui nóminis, et præsta in nobis religiónis augméntum; ut, quæ sunt bona, nútrias, ac pietátis stúdio, quæ sunt nutríta, custódias. Per Dóminum \emph{\&c.}
}\switchcolumn\portugues{
\slettrine{Ó}{} Deus das virtudes, origem de tudo o que é verdadeiramente bom, infundi nos nossos corações o amor ao vosso santo nome e aumentai na nossa alma o espírito da religião, a fim de que avigoreis nas nossas almas aquilo que é bom, e que, pelo fervor da piedade, aquilo que é avigorado seja conservado. Por nosso Senhor \emph{\&c.}
}\end{paracol}

\paragraphinfo{Epístola}{Rm. 6, 3-11}
\begin{paracol}{2}\latim{
Léctio Epístolæ beáti Pauli Apóstoli ad Romános.
}\switchcolumn\portugues{
Lição da Ep.ª do B. Ap.º Paulo aos Romanos.
}\switchcolumn*\latim{
\rlettrine{F}{ratres:} Quicúmque baptizáti sumus in Christo Jesu, in morte ipsíus baptizáti sumus. Consepúlti enim sumus cum illo per baptísmum in mortem: ut, quómodo Christus surréxit a mórtuis per glóriam Patris, ita et nos in novitáte vitæ ambulémus. Si enim complantáti facti sumus similitúdini mortis ejus: simul et resurrectiónis érimus. Hoc sciéntes, quia vetus homo noster simul crucifíxus est: ut destruátur corpus peccáti, et ultra non serviámus peccáto. Qui enim mórtuus est, justificátus est a peccáto. Si autem mórtui sumus cum Christo: crédimus, quia simul étiam vivémus cum Christo: sciéntes, quo d Christus re surgens ex mórtuis, jam non móritur, mors illi ultra non dominábitur. Quod enim mórtuus est peccáto, mórtuus est semel: quod autem vivit, vivit Deo. Ita et vos existimáte, vos mórtuos quidem esse peccáto, vivéntes autem Deo, in Christo Jesu, Dómino nostro.
}\switchcolumn\portugues{
\rlettrine{M}{eus} irmãos: Nós todos, que fomos baptizados em J. Cristo, fomos baptizados na sua morte. Com efeito, fomos sepultados com Ele pelo baptismo para morrermos para o pecado, a fim de que, assim como J. Cristo ressuscitou dos mortos para a glória do Pai, assim também caminhemos para uma vida nova; pois, com efeito, se, sendo plantados n’Ele, somos semelhantes a Ele na morte, também teremos uma ressurreição semelhante à sua, porquanto sabemos que o nosso homem velho foi crucificado com Ele, para que o corpo de pecado seja destruído e doravante não sejamos mais escravos do pecado; visto que aquele que está morto já está livre do pecado. De facto, se morrermos com Cristo, devemos crer que viveremos com Ele, recordando-nos de que Cristo, havendo ressuscitado dos mortos, já não morre, nem a morte terá domínio sobre Ele; pois bastou que morresse uma só vez, para destruir o pecado. Considerai-vos também mortos para sempre para o pecado, e vivos somente para Deus, em nosso Senhor Jesus Cristo.
}\end{paracol}

\paragraphinfo{Gradual}{Sl. 89, 13 \& 1}
\begin{paracol}{2}\latim{
\rlettrine{C}{onvértere,} Dómine, aliquántulum, et deprecáre super servos tuos. ℣. Dómine, refúgium factus es nobis, a generatióne et progénie.
}\switchcolumn\portugues{
\rlettrine{V}{olvei-Vos} um pouco para nós, Senhor: tende piedade dos vossos servos. ℣. De geração em geração, fostes, ó Senhor, o nosso refúgio.
}\switchcolumn*\latim{
Allelúja, allelúja. ℣. \emph{Ps. 30, 2-3} In te, Dómine, sperávi, non confúndar in ætérnum: in justítia tua líbera me et éripe me: inclína ad me aurem tuam, accélera, ut erípias me. Allelúja.
}\switchcolumn\portugues{
Aleluia, aleluia. ℣. \emph{Sl. 30, 2-3} Em Vós pus a esperança, e não esperarei em vão: Pela vossa justiça, salvai-me e livrai-me: inclinai vossos ouvidos para mim e apressai-Vos em socorrer-me. Aleluia.
}\end{paracol}

\paragraphinfo{Evangelho}{Mc. 8, 1-9}
\begin{paracol}{2}\latim{
\cruz Sequéntia sancti Evangélii secúndum Marcum.
}\switchcolumn\portugues{
\cruz Continuação do santo Evangelho segundo S. Marcos.
}\switchcolumn*\latim{
\blettrine{I}{n} illo témpore: Cum turba multa esset cum Jesu, nec haberent, quod manducárent, convocatis discípulis, ait illis: Miséreor super turbam: quia ecce jam tríduo sústinent me, nec habent quod mandúcent: et si dimísero eos jejúnos in domum suam, defícient in via: quidam enim ex eis de longe venérunt. Et respondérunt ei discípuli sui: Unde illos quis póterit hic saturáre pánibus in solitúdine? Et interrogávit eos: Quot panes habétis? Qui dixérunt: Septem. Et præcépit turbæ discúmbere super terram. Et accípiens septem panes, grátias agens fregit, et dabat discípulis suis, ut appónerent, et apposuérunt turbæ. Et habébant piscículos paucos: et ipsos benedíxit, et jussit appóni. Et manducavérunt, et saturáti sunt, et sustulérunt quod superáverat de fragméntis, septem sportas. Erant autem qui manducáverant, quasi quatuor mília: et dimísit eos.
}\switchcolumn\portugues{
\blettrine{N}{aquele} tempo, estando Jesus acompanhado por grande multidão de povo, que não tinha o que comer, chamou os discípulos e disse-lhes: «Tenho compaixão deste povo, que há três dias, já, está comigo e não tem o que comer. Se os deixo ir em jejum para casa, cairão de fraqueza pelo caminho, porque alguns vieram de longe». Os discípulos responderam: «Como poderemos encontrar neste deserto bastantes pães para os saciar?». Jesus interrogou-os: «Quantos pães tendes vós?». Eles responderam: «Temos sete». Então Jesus ordenou à multidão que se assentasse no chão. Depois recebeu os sete pães em suas mãos, deu graças a Deus, partiu-os e deu-os aos discípulos, para que os distribuíssem pelo povo. Havia ali, também, alguns poucos peixinhos. Ele os abençoou; e mandou que os discípulos os distribuíssem. Então todos comeram, até ficarem saciados; e, sendo recolhidos os sobejos, ficaram cheios sete cestos. Eram cerca de quatro mil aqueles que comeram! Depois Jesus mandou retirá-los.
}\end{paracol}

\paragraphinfo{Ofertório}{Sl. 16, 5 \& 6-7}
\begin{paracol}{2}\latim{
\rlettrine{P}{érfice} gressus meos in sémitis tuis, ut non moveántur vestígia mea: inclína aurem tuam, et exáudi verba mea: mirífica misericórdias tuas, qui salvos facis sperántes in te, Dómine.
}\switchcolumn\portugues{
\rlettrine{F}{irmai} meus passos nos vossos caminhos, a fim de que meus pés não vacilem: Inclinai os ouvidos para mim e ouvi as minhas palavras: Fazei brilhar a vossa misericórdia, ó Senhor, que salvais os que em Vós confiam.
}\end{paracol}

\paragraph{Secreta}
\begin{paracol}{2}\latim{
\rlettrine{P}{ropitiáre,} Dómine, supplicatiónibus nostris, et has pópuli tui oblatiónes benígnus assúme: et, ut nullíus sit írritum votum, nullíus vácua postulátio, præsta; ut, quod fidéliter pétimus, efficáciter consequámur. Per Dóminum \emph{\&c.}
}\switchcolumn\portugues{
\rlettrine{A}{colhei} propício, Senhor, as nossas súplicas e recebei benignamente estas ofertas do vosso povo; e, para que ninguém Vos apresente votos inúteis, nem súplicas vãs, permiti que obtenhamos eficazmente aquilo que Vos pedimos com fé. Por nosso Senhor \emph{\&c.}
}\end{paracol}

\paragraphinfo{Comúnio}{Sl. 26, 6}
\begin{paracol}{2}\latim{
\rlettrine{C}{ircuíbo} et immolábo in tabernáculo ejus hóstiam jubilatiónis: cantábo et psalmum dicam Dómino.
}\switchcolumn\portugues{
\rlettrine{A}{ndarei} em redor do altar: depositarei diante do seu tabernáculo uma hóstia de júbilo: e cantarei hinos ao Senhor.
}\end{paracol}

\paragraph{Postcomúnio}
\begin{paracol}{2}\latim{
\rlettrine{R}{epléti} sumus, Dómine, munéribus tuis: tríbue, quǽsumus; ut eórum et mundémur efféctu et muniámur auxílio. Per Dóminum \emph{\&c.}
}\switchcolumn\portugues{
\rlettrine{H}{avendo} nós sido saciados com vossos dons, Vos suplicamos, Senhor, nos façais a graça de, pela sua virtude, sermos purificados, e de, pelo seu socorro, sermos fortalecidos. Por nosso Senhor \emph{\&c.}
}\end{paracol}

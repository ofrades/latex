\subsection{Terceiro Domingo depois da Epifania}

\paragraphinfo{Intróito}{Jr. 29,11,12 \& 14}
\begin{paracol}{2}\latim{
\rlettrine{D}{icit} Dóminus: Ego cógito cogitatiónes pacis, et non afflictiónis: invocábitis me, et ego exáudiam vos: et redúcam captivitátem vestram de cunctis locis. \emph{Ps. 84, 2} Benedixísti, Dómine, terram tuam: avertísti captivitátem Jacob.
℣. Gloria Patri \emph{\&c.}
}\switchcolumn\portugues{
\rlettrine{D}{isse} o Senhor: tenho pensamentos de paz e não de ira: invocar-me-eis e ouvir-vos-ei; e farei regressar de todos os países os vossos cativos. \emph{Sl. 84, 2} Abençoastes, Senhor, a vossa terra e livrastes Jacob do cativeiro.
℣. Glória ao Pai \emph{\&c.}
}\end{paracol}

\paragraph{Oração}
\begin{paracol}{2}\latim{
\rlettrine{O}{mnípotens} sempitérne Deus, infirmitatem nostram propítius réspice: atque, ad protegéndum nos, déxteram tuæ majestátis exténde. Per Dóminum \emph{\&c.}
}\switchcolumn\portugues{
\slettrine{Ó}{} Deus omnipotente e eterno, olhai benigno para a nossa fraqueza e estendei-nos a dextra da vossa majestade para nos proteger continuamente. Por nosso Senhor \emph{\&c.}
}\end{paracol}

\paragraphinfo{Epístola}{Rm. 12, 16-21}
\begin{paracol}{2}\latim{
Léctio Epístolæ beáti Pauli Apóstoli ad Romános.
}\switchcolumn\portugues{
Lição da Ep.ª do B. Ap.º Paulo aos Romanos.
}\switchcolumn*\latim{
\rlettrine{F}{ratres:} Nolíte esse prudéntes apud vosmetípsos: nulli malum pro malo reddéntes: providéntes bona non tantum coram Deo, sed étiam coram ómnibus homínibus. Si fíeri potest, quod ex vobis est, cum ómnibus homínibus pacem habéntes: Non vosmetípsos defendéntes, caríssimi, sed date locum iræ. Scriptum est enim: Mihi vindícta: ego retríbuam, dicit Dóminus. Sed si esuríerit inimícus tuus, ciba illum: si sitit, potum da illi: hoc enim fáciens, carbónes ignis cóngeres super caput ejus. Noli vinci a malo, sed vince in bono malum.
}\switchcolumn\portugues{
\rlettrine{M}{eus} irmãos: Não presumais que sois prudentes; não retribuais a ninguém o mal com o mal; antes praticai cuidadosamente o bem, tanto diante de Deus, como perante os homens. Vivei em paz com todos os homens, tanto quanto seja possível e dependa de vós. Não vos vingueis vós próprios, caríssimos, mas deixai a justiça de Deus, pois está escrito: «A vingança pertence-me; Eu é que a exercerei, diz o Senhor». Se o vosso inimigo tem fome, dai-lhe de comer; se tem sede, dai-lhe de beber; pois, se assim procederdes, reunireis brasas de fogo sobre a sua cabeça. Não vos deixeis vencer pelo mal; mas triunfai do mal, praticando o bem.
}\end{paracol}

\paragraphinfo{Gradual}{Sl. 43, 8-9}
\begin{paracol}{2}\latim{
\rlettrine{L}{iberásti} nos, Dómine, ex affligéntibus nos: et eos, qui nos odérunt, confudísti. ℣. In Deo laudábimur tota die, et in nómine tuo confitébimur in sǽcula.
}\switchcolumn\portugues{
\rlettrine{L}{ivrastes-nos,} Senhor daqueles que nos afligiam: e confundistes os que nos odiavam. Glorificar-nos-emos constantemente em Deus e louvaremos eternamente o vosso nome.
}\switchcolumn*\latim{
Allelúja, allelúja. ℣. \emph{Ps. 129, 12} De profúndis clamávi ad te, Dómine: Dómine, exáudi oratiónem meam. Allelúja.
}\switchcolumn\portugues{
Aleluia, aleluia. ℣. \emph{Sl. 129, 12} Do fundo do abysmo Vos invoquei, Senhor: escutai a minha oração. Aleluia.
}\end{paracol}

\paragraphinfo{Evangelho}{Mt. 8, 1-13}
\begin{paracol}{2}\latim{
\cruz Sequéntia sancti Evangélii secúndum Matthǽum.
}\switchcolumn\portugues{
\cruz Continuação do santo Evangelho segundo S. Mateus.
}\switchcolumn*\latim{
\blettrine{I}{n} illo témpore: Cum descendísset Jesus de monte, secútæ sunt eum turbæ multæ: et ecce, leprósus véniens adorábat eum, dicens: Dómine, si vis, potes me mundáre. Et exténdens Jesus manum, tétigit eum, dicens: Volo. Mundáre. Et conféstim mundáta est lepra ejus. Et ait illi Jesus: Vide, némini díxeris: sed vade, osténde te sacerdóti, et offer munus, quod præcépit Móyses, in testimónium illis. Cum autem introísset Caphárnaum, accéssit ad eum centúrio, rogans eum et dicens: Dómine, puer meus jacet in domo paralýticus, et male torquetur. Et ait illi Jesus: Ego véniam, et curábo eum. Et respóndens centúrio, ait: Dómine, non sum dignus, ut intres sub tectum meum: sed tantum dic verbo, et sanábitur puer meus. Nam et ego homo sum sub potestáte constitútus, habens sub me mílites, et dico huic: Vade, et vadit; et alii: Veni, et venit; et servo meo: Fac hoc, et facit. Audiens autem Jesus, mirátus est, et sequéntibus se dixit: Amen, dico vobis, non inveni tantam fidem in Israël. Dico autem vobis, quod multi ab Oriénte et Occidénte vénient, et recúmbent cum Abraham et Isaac et Jacob in regno cœlórum: fílii autem regni ejiciéntur in ténebras exterióres: ibi erit fletus et stridor déntium. Et dixit Jesus centurióni: Vade et, sicut credidísti, fiat tibi. Et sanátus est puer in illa hora.
}\switchcolumn\portugues{
\blettrine{N}{aquele} tempo, descendo Jesus do monte, era acompanhado por numerosas pessoas. Eis que foi ter com Ele um leproso, adorando-O e dizendo-Lhe: «Senhor, se quiserdes, podeis curar-me». Então Jesus, estendendo a mão, tocou-o e disse: «Quero; sê curado». E no mesmo instante ficou limpo da lepra. E Jesus disse-lhe: «Ouve: não digas nada a ninguém; mas vai, mostra-te aos sacerdotes e oferece-lhes a dádiva que Moisés prescreveu, a fim de que ela seja testemunha da tua cura». E entrando Jesus em Cafarnaum, aproximou-se d’Ele um centurião, pedindo e dizendo: «Senhor, o meu servo está na cama paralítico e sofre cruelmente». E Jesus disse-lhe: «Eu irei e o curarei». Mas o centurião respondeu-Lhe: «Senhor, não sou digno de que entreis sob o meu telhado; dizei somente uma palavra e o meu servo será curado. Pois eu, que sou um homem submisso à autoridade dos meus superiores, tenho soldados sob as minhas ordens e digo a um deles: Vai. E ele vai. E digo a um outro: Vem, E ele vem. E digo a meu servo: Faz isto. E ele faz». Ouvindo Jesus isto, mostrou-se admirado, e disse: «Em verdade vos digo, que não encontrei, ainda, uma fé tão grande em Israel! Assim vos digo: que muitos virão do Oriente e do Ocidente e terão lugar no banquete com Abraão, Isaque e Jacob no reino dos céus; mas os filhos do reino serão lançados nas trevas exteriores, onde haverá choro e ranger de dentes». Então Jesus disse ao centurião: «Vai; e que se faça segundo a tua crença». E naquela hora ficou curado o servo.
}\end{paracol}

\paragraphinfo{Ofertório}{Sl. 129, 1-2}
\begin{paracol}{2}\latim{
\rlettrine{D}{e} profúndis clamávi ad te, Dómine: Dómine, exáudi oratiónem meam: de profúndis clamávi ad te. Dómine.
}\switchcolumn\portugues{
\rlettrine{D}{as} profundezas dos abysmos Vos invoquei, Senhor; escutai, Senhor, a minha voz: das profundezas dos abysmos Vos invoquei.
}\end{paracol}

\paragraph{Secreta}
\begin{paracol}{2}\latim{
\rlettrine{H}{æc} hóstia, Dómine, quǽsumus, emúndet nostra delícta: et, ad sacrifícium celebrándum, subditórum tibi córpora mentésque sanctíficet. Per Dóminum \emph{\&c.}
}\switchcolumn\portugues{
\qlettrine{Q}{ue} esta hóstia, Senhor, Vos suplicamos, nos purifique dos nossos delitos e santifique as almas e os corpos dos vossos súbditos, para que dignamente celebremos este sacrifício. Por nosso Senhor \emph{\&c.}
}\end{paracol}

\paragraphinfo{Comúnio}{Mc. 11, 24}
\begin{paracol}{2}\latim{
\rlettrine{A}{men,} dico vobis, quidquid orántes pétitis, crédite, quia accipiétis, et fiet vobis.
}\switchcolumn\portugues{
\rlettrine{N}{a} verdade vos digo: «Tudo o que pedirdes nas vossas orações, acreditai que o recebereis; e far-se-á como pedirdes».
}\end{paracol}

\paragraph{Postcomúnio}
\begin{paracol}{2}\latim{
\qlettrine{Q}{uos} tantis, Dómine, largíris uti mystériis: quǽsumus; ut efféctibus nos eórum veráciter aptáre dignéris. Per Dóminum \emph{\&c.}
}\switchcolumn\portugues{
\rlettrine{S}{enhor,} já que nos concedestes a graça de participarmos destes tão augustos mystérios, tornai-nos dignos, Vos suplicamos, de receberdes com eficácia os seus efeitos. Por nosso Senhor \emph{\&c.}
}\end{paracol}

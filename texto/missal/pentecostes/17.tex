\subsection{Décimo Sétimo Domingo depois de Pentecostes}\label{17pentecostes}

\paragraphinfo{Intróito}{Sl. 118, 137 \& 124}
\begin{paracol}{2}\latim{
\qlettrine{J}{ustus} es, Dómine, et rectum judicium tuum: fac cum servo tuo secúndum misericórdiam tuam. \emph{Ps. ibid., 1} Beáti immaculáti in via: qui ámbulant in lege Dómini.
℣. Gloria Patri \emph{\&c.}
}\switchcolumn\portugues{
\rlettrine{S}{ois} justo, ó Senhor, e os vossos juízos são equitativos: tratai o vosso servo segundo a vossa misericórdia. \emph{Ps. ibid., 1} Bem-aventurados os que são imaculados na sua vida: os que o caminho da Lei do Senhor.
℣. Glória ao Pai \emph{\&c.}
}\end{paracol}

\paragraph{Oração}
\begin{paracol}{2}\latim{
\rlettrine{D}{a,} quǽsumus, Dómine, pópulo tuo diabólica vitáre contágia: et te solum Deum pura mente sectári. Per Dóminum \emph{\&c.}
}\switchcolumn\portugues{
\rlettrine{C}{oncedei} ao vosso povo, Senhor, Vos suplicamos, a graça de evitar todo o contágio do demónio; e que o mesmo vosso povo procure com o coração puro servir-Vos só a Vós, que sois o seu Deus. Por nosso Senhor \emph{\&c.}
}\end{paracol}

\paragraphinfo{Epístola}{Ef. 4, 1-6}
\begin{paracol}{2}\latim{
Léctio Epístolæ beáti Pauli Apóstoli ad Ephésios.
}\switchcolumn\portugues{
Lição da Ep.ª do B. Ap.º Paulo aos Efésios.
}\switchcolumn*\latim{
\rlettrine{F}{atres:} Obsecro vos ego vinctus in Dómino, ut digne ambulétis vocatióne, qua vocáti estis, cum omni humilitáte et mansuetúdine, cum patiéntia, supportántes ínvicem in caritáte, sollíciti serváre unitátem spíritus in vínculo pacis. Unum corpus et unus spíritus, sicut vocáti estis in una spe vocatiónis vestræ. Unus Dóminus, una fides, unum baptísma. Unus Deus et Pater ómnium, qui est super omnes et per ómnia et in ómnibus nobis. Qui est benedíctus in sǽcula sæculórum. Amen.
}\switchcolumn\portugues{
\rlettrine{M}{eus} irmãos: Peço-vos (eu, que me encontro em prisão, nas cadeias, pelo Senhor) que vivais de modo digno da vocação a que fostes chamados, praticando a humildade, a mansidão e a paciência, suportando-vos uns aos outros com caridade e esforçando-vos em conservar a unidade do espírito pelo laço da paz. Sede um só corpo e um só espírito, como, pela vossa vocação, sois todos chamados à mesma esperança. Não há senão um só Senhor, uma só fé, um só baptismo e um só Deus que é Pai de todos os homens, e está acima de todos, penetra em todos, vive em todos e é bendito em todos os séculos dos séculos. Amen.
}\end{paracol}

\paragraphinfo{Gradual}{Sl. 32, 12 \& 6}
\begin{paracol}{2}\latim{
\rlettrine{B}{eáta} gens, cujus est Dóminus Deus eórum: pópulus, quem elégit Dóminus in hereditátem sibi. ℣. Verbo Dómini cœli firmáti sunt: et spíritu oris ejus omnis virtus eórum.
}\switchcolumn\portugues{
\rlettrine{B}{em-aventurada} a nação que tem o Senhor como seu Deus: bem-aventurado o povo que o Senhor escolheu para sua herança. A palavra do Senhor criou os céus: e o sopro da sua boca criou os espíritos celestiais.
}\switchcolumn*\latim{
Allelúja, allelúja. ℣. \emph{Ps. 101, 2} Dómine, exáudi oratiónem meam, et clamor meus ad te pervéniat. Allelúja.
}\switchcolumn\portugues{
Aleluia, aleluia. ℣. \emph{Sl. 101, 2} Ouvi, Senhor, a minha oração: e que meu clamor chegue até vós. Aleluia.
}\end{paracol}

\paragraphinfo{Evangelho}{Mt. 22, 34-46}
\begin{paracol}{2}\latim{
\cruz Sequéntia sancti Evangélii secúndum Matthǽum.
}\switchcolumn\portugues{
\cruz Continuação do santo Evangelho segundo S. Mateus.
}\switchcolumn*\latim{
\blettrine{I}{n} illo témpore: Accessérunt ad Jesum pharisǽi: et interrogávit eum unus ex eis legis doctor, tentans eum: Magíster, quod est mandátum magnum in lege? Ait illi Jesus: Díliges Dóminum, Deum tuum, ex toto corde tuo et in tota ánima tua et in tota mente tua. Hoc est máximum et primum mandátum. Secúndum autem símile est huic: Díliges próximum tuum sicut teípsum. In his duóbus mandátis univérsa lex pendet et prophétæ. Congregátis autem pharisǽis, interrogávit eos Jesus, dicens: Quid vobis vidétur de Christo? cujus fílius est? Dicunt ei: David. Ait illis: Quómodo ergo David in spíritu vocat eum Dóminum, dicens: Dixit Dóminus Dómino meo, sede a dextris meis, donec ponam inimícos tuos scabéllum pedum tuórum? Si ergo David vocat eum Dóminum, quómodo fílius ejus est? Et nemo poterat ei respóndere verbum: neque ausus fuit quisquam ex illa die eum ámplius interrogáre.
}\switchcolumn\portugues{
\blettrine{N}{aquele} tempo aproximaram-se de Jesus os fariseus. Um deles que era doutor da Lei, perguntou-Lhe para D tentar: «Mestre, qual é o principal mandamento da Lei?». Jesus respondeu-lhe: «Amarás ao Senhor, teu Deus, com todo o coração, com toda a alma e com todo o entendimento». Este é o maior e o principal mandamento. Porém, eis o segundo, que é semelhante: «Amarás ao próximo como a ti mesmo». Estes dous mandamentos encerram toda a Lei e os Profetas». E, como os fariseus estivessem reunidos, perguntou-lhes Jesus, dizendo: «Que pensais vós de Cristo? De quem é Ele Filho?». Eles responderam: «De David». E Jesus replicou: «Como é, então, que David (que havia sido inspirado pelo Espírito) Lhe chama seu Senhor, quando diz: «O Senhor diz ao meu Senhor: assenta-te à minha dextra até que reduza os teus inimigos a escabelo dos teus pés»? Se, pois, David O trata como Senhor, como é Ele seu Filho?». E ninguém pôde responder-Lhe uma palavra, nem, desde aquele dia em diante, ninguém mais ousou interrogá-l’O!
}\end{paracol}

\paragraphinfo{Ofertório}{Dn. 9, 17, 18 et 19}
\begin{paracol}{2}\latim{
\rlettrine{O}{rávi} Deum meum ego Dániel, dicens: Exáudi, Dómine, preces servi tui: illúmina fáciem tuam super sanctuárium tuum: et propítius inténde pópulum istum, super quem invocátum est nomen tuum, Deus.
}\switchcolumn\portugues{
\rlettrine{E}{u,} Daniel, orei ao Senhor, meu Deus, dizendo: Ouvi, Senhor, as orações do vosso servo: fazei resplandecer o brilho da vossa face sobre o vosso santuário: e olhai propício para este povo, em favor do qual, ó Deus, foi invocado o vosso nome.
}\end{paracol}

\paragraph{Secreta}
\begin{paracol}{2}\latim{
\rlettrine{M}{ajestátem} tuam, Dómine, supplíciter deprecámur: ut hæc sancta, quæ gérimus, et a prætéritis nos delictis éxuant
et futúris. Per Dóminum \emph{\&c.}
}\switchcolumn\portugues{
\rlettrine{H}{umildemente} rogamos à vossa majestade, Senhor, permitais que estes sacrossantos mystérios que celebramos nos livrem dos nossos delitos passados e futuros. Por nosso Senhor \emph{\&c.}
}\end{paracol}

\paragraphinfo{Comúnio}{Sl. 75, 12-13}
\begin{paracol}{2}\latim{
\rlettrine{V}{ovéte} et réddite Dómino, Deo vestro, omnes, qui in circúitu ejus affértis múnera: terríbili, et ei qui aufert spíritum príncipum: terríbili apud omnes reges terræ.
}\switchcolumn\portugues{
\rlettrine{F}{azei} votos ao Senhor, vosso Deus, e cumpri-os, ó vós, que habitais em redor d’Ele. Fazei votos a este Deus tremendo que arrebata a vida aos príncipes e esmaga os reis da terra.
}\end{paracol}

\paragraph{Postcomúnio}
\begin{paracol}{2}\latim{
\rlettrine{S}{anctificatiónibus} tuis, omnípotens Deus, et vítia nostra curéntur, et remédia nobis ætérna provéniant. Per Dóminum \emph{\&c.}
}\switchcolumn\portugues{
\qlettrine{Q}{ue} os vossos sacrossantos mystérios, ó Deus omnipotente, nos curem dos nossos vícios e nos sirvam de remédio para a eternidade. Por nosso Senhor \emph{\&c.}
}\end{paracol}

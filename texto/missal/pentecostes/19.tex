\subsection{Décimo Nono Domingo depois de Pentecostes}\label{19domingopentecostes}

\paragraph{Intróito}
\begin{paracol}{2}\latim{
\rlettrine{S}{alus} pópuli ego sum, dicit Dóminus: de quacúmque tribulatióne clamáverint ad me, exáudiam eos: et ero illórum Dóminus in perpétuum. \emph{Ps. 77, 1} Attendite, pópule meus, legem meam: inclináte aurem vestram in verba oris mei.
℣. Gloria Patri \emph{\&c.}
}\switchcolumn\portugues{
\rlettrine{S}{ou} a salvação do povo, diz o Senhor; em qualquer tribulação clamarão por mim e ouvi-los-ei, pois serei o seu Senhor eternamente. \emph{Sl. 77, 1} Sede atentos à minha Lei, ó meu povo; escutai as palavras que saem da minha boca.
℣. Glória ao Pai \emph{\&c.}
}\end{paracol}

\paragraph{Oração}
\begin{paracol}{2}\latim{
\rlettrine{O}{mnípotens} et miséricors Deus, univérsa nobis adversántia propitiátus exclúde: ut mente et córpore páriter expedíti, quæ tua sunt, líberis méntibus exsequámur. Per Dóminum \emph{\&c.}
}\switchcolumn\portugues{
\slettrine{Ó}{} Deus omnipotente e misericordioso, afastai benignamente de nós tudo quanto se opõe à salvação, a fim de que, estando a nossa alma e o nosso corpo livres de todos os obstáculos, possamos com inteira liberdade de espírito cumprir o que diz respeito ao vosso serviço. Por nosso Senhor \emph{\&c.}
}\end{paracol}

\paragraphinfo{Epístola}{Ef. 4, 23-28}
\begin{paracol}{2}\latim{
Léctio Epístolæ beáti Pauli Apóstoli ad Ephésios.
}\switchcolumn\portugues{
Lição da Ep.ª do B. Ap.º Paulo aos Efésios.
}\switchcolumn*\latim{
\rlettrine{F}{ratres:} Renovámini spíritu mentis vestræ, et indúite novum hóminem, qui secúndum Deum creátus est in justítia et sanctitáte veritátis. Propter quod deponéntes mendácium, loquímini veritátem unusquísque cum próximo suo: quóniam sumus ínvicem membra. Irascímini, et nolíte peccáre: sol non occídat super iracúndiam vestram. Nolíte locum dare diábolo: qui furabátur, jam non furétur; magis autem labóret, operándo mánibus suis, quod bonum est, ut hábeat, unde tríbuat necessitátem patiénti.
}\switchcolumn\portugues{
\rlettrine{M}{eus} irmãos: Renovai-vos no íntimo da vossa alma e revesti-vos do «homem novo», que foi criado à semelhança de Deus na justiça e santidade verdadeiras. Eis porque deveis renunciar à mentira, e falar a cada um, nas relações com o próximo, segundo a verdade, pois somos todos membros uns dos outros. Se vos irardes, que seja sem pecar; e que o sol se não esconda sem que a vossa ira desapareça. Não deis lugar no vosso coração ao demónio. Aquele que furtava não torne a furtar, mas trabalhe, empregando as mãos em alguma obra boa e útil, para socorrer os que padecem de necessidade.
}\end{paracol}

\paragraphinfo{Gradual}{Sl. 140, 2}
\begin{paracol}{2}\latim{
\rlettrine{D}{irigátur} orátio mea, sicut Incénsum in conspéctu tuo, Dómine. ℣. Elevatio mánuum meárum sacrifícium vespertínum.
}\switchcolumn\portugues{
\qlettrine{Q}{ue} a minha oração, Senhor, chegue até Vós, como perfume de incenso: ℣. E que minhas mãos erguidas Vos sejam agradáveis, como o sacrifício vespertino.
}\switchcolumn*\latim{
Allelúja, allelúja. ℣. \emph{Ps. 104, 1} Confitémini Dómino, et invocáte nomen ejus: annuntiáte inter gentes ópera ejus. Allelúja.
}\switchcolumn\portugues{
Aleluia, aleluia. ℣. \emph{Sl. 104, 1} Louvai o Senhor e invocai o seu nome: publicai as suas obras em todos os povos. Aleluia.
}\end{paracol}

\paragraphinfo{Evangelho}{Mt. 22, 1-14}
\begin{paracol}{2}\latim{
\cruz Sequéntia sancti Evangélii secúndum Matthǽum.
}\switchcolumn\portugues{
\cruz Continuação do santo Evangelho segundo S. Mateus.
}\switchcolumn*\latim{
\blettrine{I}{n} illo témpore: Loquebátur Jesus princípibus sacerdótum et pharisǽis in parábolis, dicens: Símile factum est regnum cœlórum hómini regi, qui fecit núptias fílio suo. Et misit servos suos vocáre invitátos ad nuptias, et nolébant veníre. Iterum misit álios servos, dicens: Dícite invitátis: Ecce, prándium meum parávi, tauri mei et altília occísa sunt, et ómnia paráta: veníte ad núptias. Illi autem neglexérunt: et abiérunt, álius in villam suam, álius vero ad negotiatiónem suam: réliqui vero tenuérunt servos ejus, et contuméliis afféctos occidérunt. Rex autem cum audísset, iratus est: et, missis exercítibus suis, pérdidit homicídas illos et civitátem illórum succéndit. Tunc ait servis suis: Núptiæ quidem parátæ sunt, sed, qui invitáti erant, non fuérunt digni. Ite ergo ad exitus viárum et, quoscúmque invenéritis, vocáte ad núptias. Et egréssi servi ejus in vias, congregavérunt omnes, quos invenérunt, malos et bonos: et implétæ sunt núptiæ discumbéntium. Intrávit autem rex, ut vidéret discumbéntes, et vidit ibi hóminem non vestítum veste nuptiáli. Et ait illi: Amíce, quómodo huc intrásti non habens vestem nuptiálem? At ille obmútuit. Tunc dixit rex minístris: Ligátis mánibus et pédibus ejus, míttite eum in ténebras exterióres: ibi erit fletus et stridor déntium. Multi enim sunt vocáti, pauci vero elécti.
}\switchcolumn\portugues{
\blettrine{N}{aquele} tempo, falando Jesus aos príncipes dos sacerdotes e aos fariseus por meio de parábolas, disse-lhes: O reino dos céus é semelhante a um rei que, que rendo celebrar as bodas do filho, mandou os servos chamar aqueles que haviam sido convidados para assistir. Eles, porém, não quiseram vir. Novamente mandou outros servos com esta participação: «Preparei o banquete; estão já mortos os bois e os animais gordos; tudo está preparado; vinde, pois, às bodas». Mas nenhuma atenção lhes dispensaram; e um foi para a sua casa de campo, outro para os seus negócios, e ainda outros detiveram os servos, ultrajando-os e matando-os. Então o rei, tendo conhecimento do que se passara, ficou cheio de ira e mandou os seus exércitos matar os assassinos e queimar a sua cidade. Depois disse aos servos: «As bodas estão preparadas, mas aqueles que haviam sido convidados não eram dignos. Ide vós, pois, pelas encruzilhadas dos caminhos e chamai para as bodas todos quantos encontrardes». Saíram os servos pelos caminhos e reuniram todos quantos encontraram, quer bons, quer maus, ficando cheia de convivas a sala das bodas. Entretanto o rei entrou para ver os que estavam. E tendo visto um dos convivas sem as vestes das bodas, disse-lhe: «Amigo, como te atreveste a entrar sem a veste das bodas?». O conviva guardou silêncio. Então o rei disse aos servos: «Amarrai-o de pés e mãos e lançai-o fora nas trevas, nesse lugar de choro e de ranger de dentes; pois muitos são os chamados, mas poucos os escolhidos».
}\end{paracol}

\paragraphinfo{Ofertório}{Sl. 137, 7}
\begin{paracol}{2}\latim{
\rlettrine{S}{i} ambulávero in médio tribulatiónis, vivificábis me, Dómine: et super iram inimicórum meórum exténdes manum tuam, et salvum me fáciet déxtera tua.
}\switchcolumn\portugues{
\qlettrine{Q}{uando} estiver na tribulação, dar-me-eis a vida, Senhor! Estendereis a vossa mão contra o furor dos meus inimigos: e serei salvo pela vossa dextra.
}\end{paracol}

\paragraph{Secreta}
\begin{paracol}{2}\latim{
\rlettrine{H}{æc} múnera, quǽsumus, Dómine, quæ óculis tuæ majestátis offérimus, salutária nobis esse concéde. Per Dóminum \emph{\&c.}
}\switchcolumn\portugues{
\rlettrine{S}{enhor,} permiti que estes dons, que colocamos diante da vossa majestade, sejam úteis à nossa salvação. Por nosso Senhor \emph{\&c.}
}\end{paracol}

\paragraphinfo{Comúnio}{Sl. 118,4-5}
\begin{paracol}{2}\latim{
\rlettrine{T}{u} mandásti mandáta tua custodíri nimis: útinam dirigántur viæ meæ, ad custodiéndas justificatiónes tuas.
}\switchcolumn\portugues{
\rlettrine{O}{rdenastes} que os vossos mandamentos fossem cabalmente cumpridos: dignai-Vos, pois, dirigir os meus passos, para que sigam o caminho das vossas ordens.
}\end{paracol}

\paragraph{Postcomúnio}
\begin{paracol}{2}\latim{
\rlettrine{T}{ua} nos, Dómine, medicinális operátio, et a nostris perversitátibus cleménter expédiat, et tuis semper fáciat inhærére mandátis. Per Dóminum \emph{\&c.}
}\switchcolumn\portugues{
\rlettrine{P}{ermiti,} Senhor, que as operações da vossa graça nos sirvam de remédio contra as perversidades; e, pela vossa clemência, dignai-Vos livrar-nos das más inclinações e manter-nos no cumprimento dos vossos mandamentos. Por nosso Senhor \emph{\&c.}
}\end{paracol}

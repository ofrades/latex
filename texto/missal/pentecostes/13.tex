\subsection{Décimo Terceiro Domingo depois de Pentecostes}

\paragraphinfo{Intróito}{Sl. 73, 20,19 \& 23}
\begin{paracol}{2}\latim{
\rlettrine{R}{éspice,} Dómine, in testaméntum tuum, et ánimas páuperum tuórum ne derelínquas in finem: exsúrge, Dómine, et júdica causam tuam, et ne obliviscáris voces quæréntium te. \emph{Ps. ibid., 1} Ut quid, Deus, reppulísti in finem: irátus est furor tuus super oves páscuæ tuæ?
℣. Gloria Patri \emph{\&c.}
}\switchcolumn\portugues{
\rlettrine{L}{embrai-vos,} Senhor, da vossa aliança connosco e não esqueçais para sempre as almas dos vossos servos. Erguei-Vos, Senhor, julgai a vossa causa: e não desprezeis as orações daqueles que a Vós recorrem. Porque, ó Deus, nos repelistes para sempre? \emph{Sl. ibid., 1} Porque, ó Deus, cresceu a vossa ira contra as ovelhas dos vossos pastos?
℣. Glória ao Pai \emph{\&c.}
}\end{paracol}

\paragraph{Oração}
\begin{paracol}{2}\latim{
\rlettrine{O}{mnípotens} sempitérne Deus, da nobis fídei, spei et caritátis augméntum: et, ut mereámur asséqui quod promíttis, fac nos amáre quod prǽcipis. Per Dóminum \emph{\&c.}
}\switchcolumn\portugues{
\slettrine{Ó}{} Deus omnipotente e sempiterno, concedei-nos o aumento da fé, da esperança e da caridade; e, a fim de merecermos alcançar o que nos prometestes, permiti que amemos o que nos preceituais. Por nosso Senhor \emph{\&c.}
}\end{paracol}

\paragraphinfo{Epístola}{Gl. 3, 16-22}
\begin{paracol}{2}\latim{
Léctio Epístolæ beáti Pauli Apóstoli ad Gálatas.
}\switchcolumn\portugues{
Lição da Ep.ª do B. Ap.º Paulo aos Gálatas.
}\switchcolumn*\latim{
\rlettrine{F}{ratres:} Abrahæ dictæ sunt promissiónes, et sémini ejus. Non dicit: Et semínibus, quasi in multis; sed quasi in uno: Et sémini tuo, qui est Christus. Hoc autem dico: testaméntum confirmátum a Deo, quæ post quadringéntos et trigínta annos facta est lex, non írritum facit ad evacuándam promissiónem. Nam si ex lege heréditas, jam non ex promissióne. Abrahæ autem per repromissiónem donávit Deus. Quid igitur lex? Propter transgressiónes pósita est, donec veníret semen, cui promíserat, ordináta per Angelos in manu mediatóris. Mediátor autem uníus non est: Deus autem unus est. Lex ergo advérsus promíssa Dei? Absit. Si enim data esset lex, quæ posset vivificáre, vere ex lege esset justítia. Sed conclúsit Scriptúra ómnia sub peccáto, ut promíssio ex fide Jesu Christi darétur credéntibus.
}\switchcolumn\portugues{
\rlettrine{M}{eus} irmãos: As promessas foram feitas a Abraão e à sua descendência. Pois a Escritura não diz «aos seus descendentes», como se se referisse a muitos, mas diz, referindo-se a um só: «e Àquele que nascerá de vós, isto é, Cristo». Portanto digo: Deus, tendo feito e confirmado uma aliança, a Lei, que foi dada somente quatrocentos e trinta anos depois, não poderá anular nem destruir a promessa. Ora, se pela Lei é que somos herdeiros da bênção, já esta não vem da promessa, ainda que tenha sido em virtude da promessa que a bênção foi dada a Abraão. Para que serve, pois, a Lei? Ela foi dada por causa das transgressões, até que viesse «a descendência», à qual havia sido feita a promessa. Ela foi promulgada pelos Anjos por meio dum mediador. Ora, um mediador o não é para um só, e Deus é um só. Logo é a Lei contrária à promessa de Deus? De nenhum modo. Se a Lei tivesse podido dar a vida verdadeira, então a justiça viria também, verdadeiramente, da Lei; mas a Escritura encerrou tudo sob o pecado, a fim de que a promessa fosse dada pela fé em Jesus Cristo àqueles que acreditassem.
}\end{paracol}

\paragraphinfo{Gradual}{Sl. 73, 20, 19 et 22}
\begin{paracol}{2}\latim{
\rlettrine{R}{éspice,} Dómine, in testaméntum tuum: et ánimas páuperum tuórum ne obliviscáris in finem. ℣. Exsúrge, Dómine, et júdica causam tuam: memor esto oppróbrii servórum tuórum.
}\switchcolumn\portugues{
\rlettrine{L}{embrai-Vos,} Senhor, da vossa aliança connosco e não esqueçais para sempre as almas dos vossos servos. ℣. Erguei-Vos, Senhor, julgai a vossa causa e lembrai-Vos dos opróbrios que sofreram os vossos servos.
}\switchcolumn*\latim{
Allelúja, allelúja. ℣. \emph{Ps. 89, 1} Dómine, refúgium factus es nobis a generatióne et progénie. Allelúja.
}\switchcolumn\portugues{
Aleluia, aleluia. ℣. \emph{Sl. 89, 1} De geração em geração, Senhor tendes sido o nosso refúgio. Aleluia.
}\end{paracol}

\paragraphinfo{Evangelho}{Lc. 17, 11-19}
\begin{paracol}{2}\latim{
\cruz Sequéntia sancti Evangélii secúndum Lucam.
}\switchcolumn\portugues{
\cruz Continuação do santo Evangelho segundo S. Lucas.
}\switchcolumn*\latim{
\blettrine{I}{n} illo témpore: Dum iret Jesus in Jerúsalem, transíbat per médiam Samaríam et Galilǽam. Et cum ingrederétur quoddam castéllum, occurrérunt ei decem viri leprósi, qui stetérunt a longe; et levavérunt vocem dicéntes: Jesu præcéptor, miserére nostri. Quos ut vidit, dixit: Ite, osténdite vos sacerdótibus. Et factum est, dum irent, mundáti sunt. Unus autem ex illis, ut vidit quia mundátus est, regréssus est, cum magna voce magníficans Deum, et cecidit in fáciem ante pedes ejus, grátias agens: et hic erat Samaritánus. Respóndens autem Jesus, dixit: Nonne decem mundáti sunt? et novem ubi sunt? Non est invéntus, qui redíret et daret glóriam Deo, nisi hic alienígena. Et ait illi: Surge, vade; quia fides tua te salvum fecit.
}\switchcolumn\portugues{
\blettrine{N}{aquele} tempo, indo Jesus para Jerusalém, atravessou a Samaria e a Galileia. Entrando, então, numa aldeia, foram ao seu encontro (ficando, contudo, a certa distância) dez leprosos, que clamavam: «Jesus, Mestre, tende misericórdia de nós!». Quando Jesus os viu, disse-lhes: «Ide e mostrai-vos aos sacerdotes». Enquanto eles iam, foram curados, Então um deles, vendo-se curado, voltou para trás, glorificando Deus em voz alta; e, prostrando-se com o rosto no chão, aos pés de Jesus, deu-Lhe graças! Este era samaritano. Jesus disse: «Porventura não foram os dez curados? Onde estão, pois, os outros nove? Não houve senão este estrangeiro que viesse e rendesse glória a Deus?». E, dirigindo-se a ele, acrescentou: «Levanta-te e vai: a tua fé te salvo.
}\end{paracol}

\paragraphinfo{Ofertório}{Sl. 30, 15-16}
\begin{paracol}{2}\latim{
\rlettrine{I}{n} te sperávi, Dómine; dixi: Tu es Deus meus, in mánibus tuis témpora mea.
}\switchcolumn\portugues{
\rlettrine{E}{m} Vós, Senhor, pus toda minha esperança; e disse: Vós sois o meu Deus; a minha vida está nas vossas mãos.
}\end{paracol}

\paragraph{Secreta}
\begin{paracol}{2}\latim{
\rlettrine{P}{ropitiáre,} Dómine, pópulo tuo, propitiáre munéribus: ut, hac oblatióne placátus, et indulgéntiam nobis tríbuas et postuláta concedas. Per Dóminum \emph{\&c.}
}\switchcolumn\portugues{
\rlettrine{O}{lhai} propício para o vosso povo, Senhor, e aceitai benignamente as nossas ofertas, a fim de que, deixando-Vos aplacar com esta oferta, nos concedais o perdão e atendais às nossas súplicas. Por nosso Senhor \emph{\&c.}
}\end{paracol}

\paragraphinfo{Comúnio}{Sb. 16, 20}
\begin{paracol}{2}\latim{
\rlettrine{P}{anem} de cœlo dedísti nobis, Dómine, habéntem omne delectaméntum et omnem sapórem suavitátis.
}\switchcolumn\portugues{
\rlettrine{D}{estes-nos,} Senhor, um pão do céu, que contém todas as delícias e o mais suave sabor.
}\end{paracol}

\paragraph{Postcomúnio}
\begin{paracol}{2}\latim{
\rlettrine{S}{umptis,} Dómine, cœléstibus sacraméntis: ad redemptiónis ætérnæ, quǽsumus, proficiámus augméntum. Per Dóminum \emph{\&c.}
}\switchcolumn\portugues{
\rlettrine{H}{avendo} nós recebido estes celestiais sacramentos, Senhor, concedei-nos a graça, Vos imploramos, de progredirmos sempre no caminho da salvação eterna. Por nosso Senhor \emph{\&c.}
}\end{paracol}

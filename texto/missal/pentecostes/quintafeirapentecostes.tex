\subsectioninfo{Quinta-feira de Pentecostes}{Estação em S. Lourenço fora de Muros}\label{quintafeirapentecostes}

\textit{Como no dia da Festa de Pentecostes, excepto:}

\paragraphinfo{Epístola}{Act. 8, 5-8}
\begin{paracol}{2}\latim{
Léctio Actuum Apostolórum. 
}\switchcolumn\portugues{
Lição dos Actos dos Apóstolos.
}\switchcolumn*\latim{
\rlettrine{I}{n} diébus illis: Philíppus descéndens in civitátem Samaríæ, prædicábat illis Christum. Intendébant autem turbæ his, quæ a Philíppo dicebántur, unanímiter audiéntes et vidéntes signa, quæ faciébat. Multi enim eórum, qui habébant spíritus immúndos, clamántes voce magna, exíbant. Multi autem paralýtici et claudi curáti sunt. Factum est ergo gáudium magnum in illa civitáte.
}\switchcolumn\portugues{
\rlettrine{N}{aqueles} dias, descendo Filipe à cidade de Samaria, pregava Cristo aos seus habitantes. As multidões estavam atentas ao que dizia, escutando-o unanimemente e vendo os milagres que operava. Muitos espíritos imundos saíam de várias pessoas, que estavam possessas, gritando em voz alta. E também muitos paralíticos e coxos eram curados, havendo grande alegria naquela cidade.
}\end{paracol}

\paragraphinfo{Evangelho}{Lc. 9, 1-6}
\begin{paracol}{2}\latim{
\cruz Sequéntia sancti Evangélii secúndum Lucam. 
}\switchcolumn\portugues{
\cruz Continuação do santo Evangelho segundo S. Lucas.
}\switchcolumn*\latim{
\blettrine{I}{n} illo témpore: Convocátis. Jesus duódecim Apóstolis, dedit illis virtútem et potestátem super ómnia dæmónia, et ut languóres curárent. Et misit illos prædicáre regnum Dei et sanáre infírmos. Et ait ad illos: Nihil tuléritis in via, neque virgam, neque peram, neque panem, neque pecúniam, neque duas túnicas habeátis. Et in quamcúmque domum intravéritis, ibi manéte et inde ne exeátis. Et quicúmque non recéperint vos: exeúntes de civitáte illa, etiam púlverem pedum vestrórum excútite in testimónium supra illos. Egréssi autem circuíbant per castélla, evangelizántes et curántes ubíque.
}\switchcolumn\portugues{
\blettrine{N}{aquele} tempo, havendo Jesus reunido os Doze Apóstolos, deu-lhes poder e autoridade sobre todos os demónios e também o poder de curar os enfermos. Depois, mandou-os pregar o reino de Deus e curar os enfermos, e disse-lhes: «Não leveis nada pelo caminho: nem bordão, nem saco, nem pão, nem dinheiro; não tenhais duas túnicas; em qualquer casa em que entrardes, permanecei aí e não queirais sair; quando encontrardes pessoas que não queiram receber-vos, saí dessa cidade e sacudi até o pó dos vossos pés, a fim de que isso sirva de testemunho contra elas». Então partiram; e andavam de aldeia em aldeia, evangelizando e curando os enfermos em toda a parte.
}\end{paracol}
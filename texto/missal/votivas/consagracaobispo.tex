\subsection{No Aniversário da Eleição ou Consagração do Bispo}

\textit{Como Missa Sacerdótes tui, página \pageref{confessorespontifices2}, excepto:}

\paragraphinfo{Oração, Secreta e Postcomúnio}{Página \pageref{sumopontifice}}

\paragraphinfo{Epístola}{Heb. 5, 1-4}
\begin{paracol}{2}\latim{
Léctio Epístolæ beáti Pauli Apóstoli ad Hebrǽos.
}\switchcolumn\portugues{
Lição da Ep.ª do B. Ap.º Paulo aos Hebreus.
}\switchcolumn*\latim{
\rlettrine{F}{ratres:} Omnis póntifex ex homínibus as sumptus, pro homínibus constitúitur in iis, quæ sunt ad Deum, ut ófferat dona, et sacrifícia pro peccátis: qui condolére possit iis, qui ígnorant et errant: quóniam et ipse circúmdatus est infirmitáte: et proptérea debet, quemádmodum pro pópulo, ita étiam et pro semetípso offérre pro peccátis. Nec quisquam sumit sibi honórem, sed qui vocátur a Deo, tamquam Aaron.
}\switchcolumn\portugues{
\rlettrine{M}{eus} irmãos: Todo o Pontífice é escolhido entre os homens e instituído para os homens naquilo que diz respeito ao culto de Deus, a fim de que ofereça dons e sacrifícios pelos pecados e se compadeça daqueles que vivem na ignorância e no erro, pois ele também está cercado de fraqueza. É por causa desta fraqueza que ele deve oferecer por si e pelo povo sacrifícios pelos pecados. Ninguém tome para si esta honra, mas espere que seja chamado por Deus, como Aarão.
}\end{paracol}

\paragraphinfo{Evangelho}{Mc. 13, 33-37}
\begin{paracol}{2}\latim{
\cruz Sequéntia sancti Evangélii secúndum Marcum.
}\switchcolumn\portugues{
\cruz Continuação do santo Evangelho segundo S. Marcos.
}\switchcolumn*\latim{
\blettrine{I}{n} illo témpore: Dixit Jesus discípulis suis: Vidéte, vigiláte et oráte: nescítis enim, quando tempus sit. Sicut homo, qui péregre proféctus réliquit domum suam, et dedit servis suis potestátem cujúsque óperis, et janitóri præcépit, ut vígilet. Vigiláte ergo (nescítis enim, quando dóminus domus véniat: sero, an média nocte, an galli cantu, an mane) ne, cum vénerit repénte, invéniat vos dormiéntes. Quod autem vobis dico, ómnibus dico: Vigilate.
}\switchcolumn\portugues{
\blettrine{N}{aquele} tempo, disse Jesus aos seus discípulos: «Sede atentos; vigiai e orai, pois não sabeis quando virá esse tempo. Assim como um homem que vai para uma viagem deixa a sua casa, entrega o seu domínio aos servos, marca a cada um deles a sua ocupação e encarrega o porteiro da vigilância, assim também vós deveis vigiar (pois não sabeis se o senhor da casa virá de tarde, ou à meia-noite, ou ao cantar do galo, ou de manhã), para que não aconteça que, regressando ele repentinamente, vos encontre a dormir. O que a vós digo, a todos o digo: Vigiai!».
}\end{paracol}

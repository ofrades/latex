\subsectioninfo{Missa do Espírito Santo}{Para a Quinta-feira}\label{votivaespiritosanto}

\textit{Como na Missa do Domingo de Pentecostes, página \pageref{domingopentecostes}, excepto:}

\paragraphinfo{Epístola}{Act. 8, 14-17}
\begin{paracol}{2}\latim{
Léctio Actuum Apostólorum.
}\switchcolumn\portugues{
	Lição dos Actos dos Apóstolos.
}\switchcolumn*\latim{
\rlettrine{I}{n} diébus illis: Cum audíssent Apóstoli, qui erant Jerosólymis, quod recepísset Samaría verbum Dei, misérunt ad eos Petrum et Joánnem. Qui cum veníssent, oravérunt pro ipsis, ut accíperent Spíritum Sanctum: nondum enim in quemquam illórum vénerat, sed baptizáti tantum erant in nómine Dómini Jesu. Tunc imponébant manus super illos, et accipiébant Spíritum Sanctum.
}\switchcolumn\portugues{
\rlettrine{N}{aqueles} dias, quando os Apóstolos, que estavam em Jerusalém, souberam que a Samaria recebera a palavra de Deus, enviaram lá Pedro e João, os quais, apenas lá chegaram, oraram por aqueles, para que recebessem o Espírito Santo que não havia descido sobre nenhum deles; porquanto haviam sido baptizados somente em nome do Senhor Jesus. Então impuseram-lhes as mãos e eles receberam o Espírito Santo.
}\end{paracol}

\paragraphinfo{Gradual}{Sl. 32, 12 \& 6}
\begin{paracol}{2}\latim{
\rlettrine{B}{eáta} gens, cujus est Dóminus Deus eórum: pópulus, quem elégit Dóminus in hereditátem sibi. ℣. Verbo Dómini cœli firmáti sunt: et Spíritu oris ejus omnis virtus eórum.
}\switchcolumn\portugues{
\rlettrine{B}{em-aventurado} o povo cujo Deus é o Senhor! Bem-aventurado o povo que, escolheu o Senhor para sua herança. A palavra do Senhor criou os céus; e o sopro dos seus lábios criou toda a milícia celestial.
}\switchcolumn*\latim{
Allelúja, allelúja. \emph{(Hic genuflectitur)} ℣. Veni, Sancte Spíritus,
reple tuórum corda fidélium: et tui amóris in eis ignem accénde. Allelúja.
}\switchcolumn\portugues{
Aleluia, aleluia. \emph{(Genuflecte-se)} Vinde, Espírito Santo, enchei os corações dos vossos fiéis e acendei neles o fogo do vosso amor. Aleluia.
}\end{paracol}

\textit{Depois da Septuagésima omite-se o Aleluia e o Verso que se segue, e diz-se o:}

\paragraphinfo{Trato}{Sl. 103, 30}
\begin{paracol}{2}\latim{
\rlettrine{E}{mítte} Spíritum tuum, et creabúntur: et renovábis fáciem terræ. ℣. O quam bonus et suávis est, Dómine, Spíritus tuus in nobis! \emph{(Hic genuflectitur)} ℣. Veni, Sancte Spíritus, reple tuórum corda fidélium: et tui amóris in eis ignem accénde.
}\switchcolumn\portugues{
\rlettrine{E}{nviai} o vosso Espírito e eles serão criados: e renovarão a face da terra. ℣. Ó Senhor, como é bom e suave o vosso Espírito dentro de nós! \emph{(Genuflecte-se)} ℣. Vinde, Espírito Santo, enchei os corações dos vossos fiéis e acendei neles o fogo do vosso amor.
}\end{paracol}

\textit{No Tempo Pascal omite-se o Gradual e o Trato, e diz-se:}

\begin{paracol}{2}\latim{
Allelúja, allelúja. ℣. \emph{Ps. 103, 30} Emítte Spíritum tuum, et creabúntur: et renovábis fáciem terræ. Allelúja. \emph{(Hic genuflectitur)} ℣. Veni, Sancte Spíritus, reple tuórum corda fidélium: et tui amóris in eis ignem accénde. Allelúja.
}\switchcolumn\portugues{
Aleluia, aleluia. ℣. \emph{Sl. 103, 30} Enviai o vosso Espírito e eles serão criados: e renovarão a face da terra. Aleluia. \emph{(Genuflecte-se)} ℣. Vinde, Espírito Santo, enchei os corações dos vossos fiéis e acendei neles o fogo do vosso amor. Aleluia.
}\end{paracol}

\subsectioninfo{Missa do Santíssimo Sacramento}{Para a Quinta-feira}

\textit{Como na Missa do Santíssimo Corpo de Cristo, página \pageref{santissimocorpocristo}, excepto:}

\textit{Depois da Septuagésima omite-se o Aleluia o Verso que se segue, e diz-se o:}

\paragraphinfo{Trato}{Ml. 1, 11}
\begin{paracol}{2}\latim{
\rlettrine{A}{b} ortu solis usque ad occásum, magnum est nomen meum in géntibus. ℣. Et in omni loco sacrificátur, et offértur nómini meo oblátio munda: quia magnum est nomen meum in géntibus. ℣. \emph{Prov. 9, 5} Veníte, comédite panem meum: et bíbite vinum, quod míscui vobis.
}\switchcolumn\portugues{
\rlettrine{D}{esde} o nascente até ao poente o meu nome é grande entre as nações. ℣. Em todos os lugares fazem-se sacrifícios e oferece-se em honra do meu nome uma vítima pura; pois o meu nome é grande entre as nações. ℣. \emph{Pr. 9, 5} Vinde, comei o meu pão e bebei o vinho, que vos preparei.
}\end{paracol}

\textit{No Tempo Pascal omite-se O Gradual e o Trato, e diz-se:}

\begin{paracol}{2}\latim{
Allelúja, allelúja. ℣. \emph{Luc. 24, 35} Cognovérunt discípuli Dóminum Jesum in fractióne panis. Allelúja. ℣. \emph{Joann. 6, 56-57} Caro mea vere est cibus, et sanguis meus vere est potus: qui mánducat meam carnem, et bibit meum sánguinem, in me manet, et ego in eo. Allelúja.
}\switchcolumn\portugues{
Aleluia, aleluia. ℣. \emph{Lc. 24, 35} Os discípulos reconheceram o Senhor pela fracção do pão. Aleluia. ℣. \emph{Jo. 6, 56-57} Minha Carne é verdadeira comida e o meu Sangue verdadeira bebida. Aquele que come a minha Carne e bebe o meu Sangue, permanece em mim e Eu nele. Aleluia.
}\end{paracol}

\subsection{Para a Propagação da Fé}

\paragraphinfo{Intróito}{Sl. 66, 2-3}
\begin{paracol}{2}\latim{
\rlettrine{D}{eus} misereátur nostri, et benedícat nobis: illúminet vultum suum super nos, et misereátur nostri: ut cognoscámus in terra viam tuam, in ómnibus géntibus salutáre tuum. (T. P. Allelúja, allelúja.) \emph{Ps. ibid., 4} Confiteántur tibi pópuli, Deus: confiteántur tibi pópuli omnes.
℣. Gloria Patri \emph{\&c.}
}\switchcolumn\portugues{
\qlettrine{Q}{ue} Deus tenha misericórdia de nós e nos abençoe: que brilhe sobre nós a luz da sua face e tenha misericórdia de nós: para que seja conhecido no universo o seu caminho e todos os povos encontrem a sua salvação. (T. P. Aleluia, aleluia.) \emph{Sl. ibid., 4} Que todos os povos Vos prestem homenagem, ó Deus; que todos os povos Vos glorifiquem.
℣. Glória ao Pai \emph{\&c.}
}\end{paracol}

\paragraph{Oração}
\begin{paracol}{2}\latim{
\rlettrine{D}{eus,} qui omnes hómines vis salvos fíeri et ad agnitiónem veritátis veníre: mitte, quǽsumus, operários in messem tuam, et da eis cum omni fidúcia loqui verbum tuum; ut sermo tuus currat et clarificétur, et omnes gentes cognóscant te solum Deum verum, et quem misísti Jesum Christum, Fílium tuum, Dóminum nostrum: Qui tecum vivit \emph{\&c.}
}\switchcolumn\portugues{
\slettrine{Ó}{} Deus, que quereis que todos os homens sejam salvos e venham ao conhecimento da verdade, enviai operários para a vossa messe, Vos suplicamos, e fazei que preguem confiadamente a vossa palavra, a fim de que a vossa doutrina se propague e seja manifestada, e todas as nações Vos conheçam como único e verdadeiro Deus, e Aquele que mandastes ao mundo Jesus Cristo, vosso Filho e nosso Senhor: Que convosco vive e reina \emph{\&c.}
}\end{paracol}

\paragraphinfo{Epístola}{Ecl. 36, 1-10 et 17-19}
\begin{paracol}{2}\latim{
Léctio libri Sapiéntiæ.
}\switchcolumn\portugues{
Lição do Livro da Sabedoria.
}\switchcolumn*\latim{
\rlettrine{M}{iserére} nostri, Deus ómnium, et réspice nos, et osténde nobis lucem miseratiónum tuárum: et immítte timórem tuum super gentes, quæ non exquisiérunt te, ut cognóscant, quia non est Deus nisi tu, et enárrent magnália tua. Alleva manum tuam super gentes aliénas, ut vídeant poténtiam tuam. Sicut enim in conspéctu eórum sanctificátus es in nobis, sic in conspéctu nostro magnificáberis in eis, ut cognóscant te, sicut et nos cognóvimus, quóniam non est Deus præter te, Dómine. Innova signa et immúta mirabília. Glorífica manum et bráchium dextrum. Excita furórem et effúnde iram. Tolle adversárium et afflíge inimícum. Festína tempus et meménto finis, ut enárrent mirabília tua. Da testimónium his, qui ab inítio creatúræ tuæ sunt, et súscita prædicatiónes, quas locúti sunt in nómine tuo prophétæ prióres. Da mercédem sustinéntibus te, ut prophétæ tui fidéles inveniántur: et exáudi oratiónes servórum tuórum, secúndum benedictiónem Aaron de pópulo tuo, et dírige nos in viam justítiæ, et sciant omnes, qui hábitant terram, quia tu es Deus conspéctor sæculórum.
}\switchcolumn\portugues{
\rlettrine{T}{ende} misericórdia de nós, ó Deus de tudo quanto existe; olhai para nós e mostrai-nos a luz das vossas misericórdias. Espalhai o vosso temor sobre os povos que Vos não procuram, Para que reconheçam que não há outro Deus senão Vós e proclamem vossas maravilhas. Erguei vossa mão contra as nações estrangeiras, Para que reconheçam vosso poder; porque, assim como diante dos seus olhos mostrastes a vossa santidade, assim também à nossa vista mostrai nelas a vossa grandeza, a fim de que reconheçam, como nós, que além de Vós, Senhor, não há outro Deus. Renovai vossos prodígios e operai novas maravilhas. Glorificai a vossa mão e o vosso braço direito. Excitai o vosso furor e manifestai a vossa ira. Destruí o adversário e afligi o inimigo. Apressai o tempo e lembrai-Vos do fim, para que se narrem as vossas maravilhas. Dai testemunho em favor daqueles que desde o princípio são vossas criaturas e realizai as predições que em vosso nome proferiram os primeiros Profetas. Concedei recompensa aos que esperam em Vós, para que os vossos profetas sejam considerados verdadeiros, e ouvi as súplicas dos vossos servos, segundo a bênção de Aarão ao vosso povo, e encaminhai-nos pela estrada da justiça, a fim de que todos os que habitam a terra saibam que sois o Senhor, que contempla os séculos!
}\end{paracol}

\paragraphinfo{Gradual}{Sl. 66, 6-8}
\begin{paracol}{2}\latim{
\rlettrine{C}{onfiteántur} tibi pópuli, Deus, confiteántur tibi pópuli omnes: terra dedit fructum suum. ℣. Benedícat nos Deus, Deus noster, benedícat nos Deus: et métuant eum omnes fines terræ.
}\switchcolumn\portugues{
\qlettrine{Q}{ue} todos os povos Vos prestem homenagem, ó Deus; que todos os povos Vos glorifiquem. ℣. A terra deu o seu fruto. Que o nosso Deus nos abençoe, ó Deus! Que em todos os confins da terra O temam.
}\switchcolumn*\latim{
Allelúja, allelúja. ℣. \emph{Ps. 99, 1} Jubiláte Deo, omnis terra: servíte Dómino in lætítia: introíte in conspéctu ejus, in exsultatióne. Allelúja.
}\switchcolumn\portugues{
Aleluia, aleluia. ℣. \emph{Sl. 99, 1} Ó terra universal, rejubilai diante de Deus. Servi o Senhor com alegria. Comparecei diante d’Ele, exultando de contentamento. Aleluia.
}\end{paracol}

\textit{Após a Septuagésima omite-se o Aleluia e o que segue e diz-se:}

\paragraphinfo{Trato}{Sl. 95, 3-5}
\begin{paracol}{2}\latim{
\rlettrine{A}{nnuntiáte} inter gentes glóriam Dómini, in ómnibus pópulis mirabília ejus. ℣. Quóniam magnus Dóminus, et laudábilis nimis: terríbilis est super omnes deos. ℣. Quóniam omnes dii géntium dæmónia: Dóminus autem cœlos fecit.
}\switchcolumn\portugues{
\rlettrine{A}{nunciai} diante dos povos a glória do Senhor: tomai conhecidas dos povos as suas maravilhas: ℣. Porquanto grande é o Senhor e infinitamente digno de todos os louvores. Ele é mais temível que todos os deuses: ℣. Pois todos os deuses dos povos são demónios: Enquanto que o Senhor criou os céus.
}\end{paracol}

\textit{No T. Pascal omite-se o Gradual e o Trato e diz-se:}

\begin{paracol}{2}\latim{
Allelúja, allelúja. ℣. \emph{Ps. 99, 1-2} Jubiláte Deo, omnis terra: servíte Dómino in lætítia: introíte in conspéctu ejus, in exsultatióne. Allelúja. ℣. Scitóte quóniam Dóminus ipse est Deus: ipse fecit nos, et non ipsi nos. Allelúja.
}\switchcolumn\portugues{
Aleluia, aleluia. ℣. \emph{Sl. 99, 1-2} Ó terra universal, rejubilai diante de Deus. Servi o Senhor com alegria. Comparecei diante d’Ele, exultando de contentamento. Aleluia. Sabei que o Senhor é o próprio Deus; e que nos criou a nós, e não nós a nós mesmos. Aleluia.
}\end{paracol}

\paragraphinfo{Evangelho}{Mt. 9, 35-38}
\begin{paracol}{2}\latim{
\cruz Sequéntia sancti Evangélii secúndum Matthǽum.
}\switchcolumn\portugues{
\cruz Continuação do santo Evangelho segundo S. Mateus.
}\switchcolumn*\latim{
\blettrine{I}{n} illo témpore: Circuíbat Jesus omnes civitátes et castélla, docens in synagógis eórum, et prǽdicans Evangélium regni, et curans omnem languórem et omnem infirmitátem. Videns autem turbas, misértus est eis: quia erant vexáti, et jacéntes sicut oves non habéntes pastórem. Tunc dicit discípulis suis: Messis quidem multa, operárii autem pauci. Rogáte ergo Dóminum messis, ut mittat operários in messem suam.
}\switchcolumn\portugues{
\blettrine{N}{aquele} tempo, andava Jesus por todas as cidades e aldeias, ensinando nas suas sinagogas, pregando o Evangelho do reino e curando todas as doenças e enfermidades. E, vendo as turbas, compadeceu-se delas, pois estavam desgarradas e abandonadas, como ovelhas sem pastor. Então, disse aos seus discípulos: «A messe, na verdade, é abundante, mas os operários são poucos. Rogai, pois, ao Senhor da messe que mande operários para a sua messe».
}\end{paracol}

\paragraphinfo{Ofertório}{Sl. 95, 7-9}
\begin{paracol}{2}\latim{
\rlettrine{A}{fférte} Dómino, pátriæ géntium, afférte Dómino glóriam et honórem, afférte Dómino glóriam nómini ejus: tóllite hóstias, et introíte in átria ejus: adoráte Dóminum in átrio sancto ejus. (T. P. Allelúja.)
}\switchcolumn\portugues{
\rlettrine{R}{endei} ao Senhor, ó gentes das nações, rendei ao Senhor glória e honra! Rendei ao Senhor a glória devida ao seu nome! Procurai vítimas e entrai no átrio do seu santuário. Adorai o Senhor no seu sagrado santuário. (T. P. Aleluia.)
}\end{paracol}

\paragraph{Secreta}
\begin{paracol}{2}\latim{
\rlettrine{P}{rotéctor} noster, áspice, Deus, et réspice in fáciem Christi tui, qui dedit redemptiónem semetípsum pro ómnibus: et fac; ut ab ortu solis usque ad occásum magnificétur nomen tuum in géntibus, ac in omni loco sacrificétur et offerátur nómini tuo oblátio munda. Per eúndem Dóminum \emph{\&c.}
}\switchcolumn\portugues{
\rlettrine{O}{lhai} para nós, ó Deus, nosso Protector, e fixai os olhos na face do vosso Cristo, que a si próprio se entregou por todos em redenção; e permiti que do oriente até ao ocidente seja glorificado entre as nações o vosso nome, e que em todo o orbe se sacrifique e ofereça ao vosso Nome a oblação pura. Pelo mesmo nosso Senhor \emph{\&c.}
}\end{paracol}

\paragraphinfo{Comúnio}{Sl. 116, 1-2}
\begin{paracol}{2}\latim{
\rlettrine{L}{audáte} Dóminum, omnes gentes: laudáte eum, omnes pópuli: quóniam confirmáta est super nos misericordia ejus, et véritas Dómini manet in ætérnum. (T. P. Allelúja.)
}\switchcolumn\portugues{
\slettrine{Ó}{} nações, louvai todas o Senhor! Ó povos, louvai todos em uníssono o Senhor! Pois a sua misericórdia para connosco confirmou-se e a verdade do Senhor permanecerá eternamente. (T. P. Aleluia.)
}\end{paracol}

\paragraph{Postcomúnio}
\begin{paracol}{2}\latim{
\rlettrine{R}{edemptiónis} nostræ múnere vegetáti: quǽsumus, Dómine; ut, hoc perpétuæ salútis auxílio, fides semper vera profíciat. Per Dóminum \emph{\&c.}
}\switchcolumn\portugues{
\rlettrine{F}{ortalecidos} com o dom da nossa Redenção, Vos suplicamos, Senhor, que, por este auxílio da salvação perpétua, sempre aumente a verdadeira fé. Por nosso Senhor \emph{\&c.}
}\end{paracol}

\subsection{Para a Eleição do Sumo Pontífice}

\paragraphinfo{Intróito}{1. Sm. 2, 35}
\begin{paracol}{2}\latim{
\rlettrine{S}{uscitábo} mihi sacerdótem fidélem, qui juxta cor meum et ánimam meam fáciet: et ædificábo ei domum fidélem, et ambulábit coram Christo meo cunctis diébus. (T. P. Allelúja, allelúja.) \emph{Ps. 131, 1} Meménto, Dómine, David: et omnis mansuetúdinis ejus.
℣. Gloria Patri \emph{\&c.}
}\switchcolumn\portugues{
\rlettrine{F}{arei} aparecer um sacerdote fiel que tudo fará segundo o meu coração e a minha alma: e por ele edificarei uma casa fiel e estará sempre na presença do meu Cristo em todos os dias. (T. P. Aleluia, aleluia.) \emph{Sl. 131, 1} Lembrai-Vos, Senhor, de David e de toda sua mansidão.
℣. Glória ao Pai \emph{\&c.}
}\end{paracol}

\paragraph{Oração}
\begin{paracol}{2}\latim{
\rlettrine{S}{úpplici,} Dómine, humilitáte depóscimus: ut sacrosánctæ Románæ Ecclésiæ concédat Pontíficem illum tua imménsa píetas; qui et pio in nos stúdio semper tibi plácitus, et tuo pópulo pro salúbri regímine sit assidue ad glóriam tui nóminis reveréndus. Per Dóminum \emph{\&c.}
}\switchcolumn\portugues{
\rlettrine{D}{a} vossa misericórdia, Senhor, imploramos, súplices, que, pela vossa infinita bondade, à vossa Igreja Romana concedais um tal Pontífice que sempre Vos agrade pelo seu zelo e piedade e que pelo seu sábio governo em honra do vosso nome atraia a assídua veneração do vosso povo. Por nosso Senhor \emph{\&c.}
}\end{paracol}

\paragraphinfo{Epístola}{Heb. 4, 16; 5, 1-7.}
\begin{paracol}{2}\latim{
Léctio Epístolæ beáti Pauli Apóstoli ad Hebrǽos.
}\switchcolumn\portugues{
Lição da Epístola do bem-aventurado Apóstolo Paulo aos Hebreus.
}\switchcolumn*\latim{
\rlettrine{F}{ratres:} Adeámus cum fidúcia ad thronum grátiæ, ut misericórdiam conséquimur et grátiam inveniámus in auxílio opportúno. Omnis namque póntifex ex homínibus assúmptus, pro homínibus constitúitur in iis, quæ sunt ad Deum, ut ófferat dona, et sacrifícia pro peccátis: qui condolére possit iis, qui ígnorant et errant: quóniam et ipse circúmdatus est infirmitáte: et proptérea debet, quemádmodum pro pópulo, ita étiam et pro semetípso offérre pro peccátis. Nec quisquam sumit sibi honórem, sed qui vocátur a Deo, tamquam Aaron. Sic et Christus non semetípsum clarificávit, ut póntifex fíeret, sed qui locútus est ad eum: Fílius meus es tu, ego hódie génui te. Quemádmodum et in álio loco dicit: Tu es sacérdos in ætérnum, secúndum órdinem Melchísedech. Qui in diébus carnis suæ preces supplicationésque ad eum, qui possit illum salvum fácere a morte, cum clamóre válido et lácrimis ófferens, exaudítus est pro sua reveréntia.
}\switchcolumn\portugues{
\rlettrine{A}{cerquemo-nos} confiadamente do trono da graça, a fim de alcançarmos misericórdia e encontrarmos graça para nosso auxílio em ocasião oportuna. Porquanto todo o Pontífice é escolhido entre os homens, em favor dos quais é instituído para o que respeita a Deus, Para oferecer dons e sacrifícios pelos pecados, para se compadecer dos ignorantes e extraviados; e porque ele está também cercado de fraqueza, e por causa dela, deve oferecer sacrifícios, tanto por si Próprio como pelo povo. Ninguém procure para si esta honra, mas aguarde que seja chamado como Aarão. E assim aconteceu com Cristo, que se não exaltou a Si próprio, fazendo-se Pontífice, mas foi glorificado por Aquele que Lhe disse: «Tu és o meu Filho; hoje te gerei». E do mesmo modo disse em outra ocasião: «Tu és Sacerdote, eternamente, segundo a ordem de Melquisedeque». O qual, havendo oferecido nos dias da sua vida mortal orações e súplicas com fortes clamores e lágrimas Àquele que podia salvá-l’O da morte, foi atendido pela devida reverência.
}\end{paracol}

\paragraphinfo{Gradual}{Lv. 21, 10}
\begin{paracol}{2}\latim{
\rlettrine{P}{óntifex} sacérdos magnus inter fratres suos, super cujus caput fusum est unctiónis óleum, et cujus manus in sacerdótio consecrátæ sunt, vestitúsque est sanctis véstibus: débuit per ómnia frátribus similári. ℣. \emph{Hebr. 2, 17} Ut miséricors fíeret, et fidélis póntifex ad Deum: ut repropitiáret delícta pópuli.
}\switchcolumn\portugues{
\rlettrine{O}{} Pontífice é o sacerdote magno entre os seus irmãos, sobre cuja cabeça foi lançado o óleo da unção e cujas mãos foram consagradas para o sacerdócio: está revestido com as vestes sagradas e em tudo deverá assemelhar-se a seus irmãos: ℣. \emph{Heb. 2, 17} A fim de ser um Pontífice misericordioso e fiel junto de Deus, para expiar os pecados do povo.
}\switchcolumn*\latim{
Allelúja, allelúja. ℣. \emph{Levit. 21, 8} Sacerdos sit sanctus, sicut et ego sanctus sum, Dóminus, qui sanctífico vos. Allelúja.
}\switchcolumn\portugues{
Aleluia, aleluia. ℣. \emph{Lv. 21, 8} Que o Sacerdote seja santo, como Eu, o Senhor, que vos santifico, sou santo. Aleluia
}\end{paracol}

\textit{Após a Septuagésima, omite-se o Aleluia e o seguinte e diz-se:}

\paragraphinfo{Trato}{Sl. 131, 8-10}
\begin{paracol}{2}\latim{
\rlettrine{S}{urge,} Dómine, in réquiem tuam: tu et arca sanctificatiónis tuæ. ℣. Sacerdótes tui induántur justítiam, et sancti tui exsúltent. ℣. Propter David, servum tuum, non avértas fáciem Christi tui.
}\switchcolumn\portugues{
\rlettrine{E}{rguei-Vos,} Senhor, e ide onde Vós repousais: Vós e a arca da vossa santificação. ℣. Que os vossos Sacerdotes se revistam de justiça e que os vossos Santos rejubilem. ℣. Por causa de David, vosso servo, não afasteis a face do vosso Cristo.
}\end{paracol}

\textit{No Tempo Pascal omite-se o Gradual e o Trato e diz-se:}

\begin{paracol}{2}\latim{
Allelúja, allelúja. ℣. \emph{Levit. 21, 8} Sacérdos sit sanctus, sicut et ego sanctus sum, Dóminus, qui sanctífico vos. Allelúja. ℣. \emph{Joann. 10, 14} Ego sum pastor bonus: et cognósco oves meas, et cognóscunt me meæ. Allelúja.
}\switchcolumn\portugues{
Aleluia, aleluia. ℣. \emph{Lv. 21, 8} Que o Sacerdote seja santo, como Eu, o Senhor, que vos santifico, sou santo. Aleluia. ℣. \emph{Jo. 10, 14} Eu sou o bom pastor: e conheço as minhas ovelhas e elas conhecem-me. Aleluia.
}\end{paracol}

\paragraphinfo{Evangelho}{Página \pageref{vigiliapentecostes}}

\paragraphinfo{Ofertório}{3. Esd. 5, 40}
\begin{paracol}{2}\latim{
\rlettrine{N}{on} participéntur sancta, donec exsúrgat póntifex in ostensiónem et veritátem. (T. P. Allelúja.)
}\switchcolumn\portugues{
\qlettrine{Q}{ue} se não celebre o sacrifício enquanto não surgir o Pontífice que há-de mostrar a verdade. (T. P. Aleluia.)
}\end{paracol}

\paragraph{Secreta}
\begin{paracol}{2}\latim{
\rlettrine{T}{uæ} nobis, Dómine, abundántia pietátis indúlgeat: ut per sacra múnera, quæ tibi reverénter offérimus, gratum majestáti tuæ Pontíficem sanctæ matris Ecclésiæ regímini præésse gaudeámus. Per Dóminum \emph{\&c.}
}\switchcolumn\portugues{
\qlettrine{Q}{ue} a vossa grande clemência seja indulgente para connosco, Senhor, e, pela reverente oferta destes sagrados dons, permiti que gozemos a alegria de ver presidir ao governo da santa Madre Igreja um Pontífice agradável à vossa majestade. Por nosso Senhor \emph{\&c.}
}\end{paracol}

\paragraphinfo{Comúnio}{Ex. 29,29-30}
\begin{paracol}{2}\latim{
\rlettrine{V}{este} sancta utétur póntifex, qui fúerit constitútus, et ingrediétur tabernáculum testimónii, ut minístret in sanctuário. (T. P. Allelúja.)
}\switchcolumn\portugues{
\rlettrine{O}{} Pontífice que for escolhido use vestes sagradas e entre no tabernáculo da aliança para ministrar no santuário. (T. P. Aleluia.)
}\end{paracol}

\paragraph{Postcomúnio}
\begin{paracol}{2}\latim{
\rlettrine{P}{retiósi} Córporis et Sánguinis tui nos, Dómine, sacraménto reféctos, mirífica tuæ majestátis grátia de illíus Summi Pontíficis concessióne lætíficet: qui et plebem tuam virtútibus ínstruat, et fidélium mentes spirituálium aromátum odóre perfúndat: Qui vivis et regnas \emph{\&c.}
}\switchcolumn\portugues{
\rlettrine{H}{avendo} sido saciados com vosso precioso corpo e sangue, Senhor, alegre-nos a admirável graça da vossa majestade, concedendo-nos um Pontífice que instrua o vosso povo na virtude e embalsame as almas dos fiéis com o perfume das graças espirituais. Ó Vós, que viveis e reinais \emph{\&c.}
}\end{paracol}

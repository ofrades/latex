\subsectioninfo{Missa da Santíssima Trindade}{Segunda-feira}\label{votivasantissimatrindade}

\textit{Como na Missa do Domingo da Santíssima Trindade, página \pageref{domingosantissimatrindade}, excepto:}

\paragraphinfo{Epístola}{2. Cor. 13, 11 \& 13}
\begin{paracol}{2}\latim{
Léctio Epístolæ beáti Pauli Apóstoli ad Corínthios.
}\switchcolumn\portugues{
Lição da Ep.ª do B. Ap.º Paulo aos Coríntios.
}\switchcolumn*\latim{
\rlettrine{F}{ratres:} Gaudéte, perfécti estóte, exhortámini, idem sápite, pacem habéte, et Deus pacis et dilectiónis erit vobíscum. Grátia Dómini nostri Jesu Christi, et cáritas Dei, et communicátio Sancti Spíritus sit cum ómnibus vobis. Amen.
}\switchcolumn\portugues{
\rlettrine{M}{eus} irmãos: Regozijai-vos, sede perfeitos» confortai-vos reciprocamente, sede unidos em vossos sentimentos, vivei em paz; e Deus da paz e do amor permanecerá convosco. Que a graça de nosso Senhor Jesus Cristo, o amor de Deus e a comunicação do Espírito Santo estejam sempre convosco. Amen.
}\end{paracol}

\paragraphinfo{Gradual}{Dan. 3, 55-56}
\begin{paracol}{2}\latim{
\rlettrine{B}{enedíctus} es, Dómine, qui íntuens abýssos, et sedes super Chérubim. ℣. Benedíctus es, Dómine, in firmaménto cœli, et laudábilis in sǽcula.
}\switchcolumn\portugues{
\rlettrine{S}{enhor,} que estais assentado acima dos Querubins e cujo olhar penetra na profundeza dos abismos, sois bendito! ℣. Sim, Senhor, sois bendito no firmamento do céu; e sois digno de louvor em todos os séculos.
}\switchcolumn*\latim{
Allelúja, allelúja. ℣. \emph{ibid., 52} Benedíctus es, Dómine, Deus patrum nostrórum, et laudábilis in sǽcula. Allelúja.
}\switchcolumn\portugues{
Aleluia, aleluia. ℣. \emph{ibid., 52} Sois bendito, Senhor, Deus dos nossos pais: e sois digno de louvor em todos os séculos.
}\end{paracol}

\textit{Depois da Septuagésima omite-se o Aleluia e o que se segues, e diz-se:}

\paragraph{Trato}
\begin{paracol}{2}\latim{
\rlettrine{T}{e} Deum, Patrem ingénitum, te, Fílium unigénitum, te, Spíritum Sanctum Paráclitum, sanctam et indivíduam Trinitátem, toto corde confitémur, laudámus atque benedícimus. ℣. Quóniam magnus es tu, et fáciens mirabília: tu es Deus solus. ℣. Tibi laus, tibi glória, tibi gratiárum áctio in sǽcula sempitérna, o beáta Trinitas.
}\switchcolumn\portugues{
\rlettrine{A}{} Vós, ó Deus Pai, que fostes criado; a Vós, ó Filho Unigénito; a Vós, ó Espírito Santo Paráclito; a Vós, ó Santa e indivisível Trindade: adoramos com todo o coração, louvamos bendizemos. ℣. Pois Vós sois imenso, Senhor, praticais prodígios e sois o único Deus. ℣. A Vós, ó beatíssima Trindade, louvor, glória e acção de graças em todos os séculos.
}\end{paracol}

\textit{No tempo Pascal omite-se o Gradual e o Trato, e diz-se:}

\begin{paracol}{2}\latim{
Allelúja, allelúja. ℣. \emph{Dan. 3, 52} Benedíctus es, Dómine, Deus patrum nostrórum, et laudábilis in sǽcula. Allelúja. ℣. Benedicámus Patrem et Fílium cum Sancto Spíritu. Allelúja.
}\switchcolumn\portugues{
Aleluia, aleluia. ℣. \emph{Dn. 3, 52} Sois bendito, Senhor, Deus dos nossos e digno de louvor em todos os séculos. Aleluia. ℣. Bendigamos o Pai, e o Filho com o Espírito Santo. Aleluia.
}\end{paracol}

\paragraphinfo{Evangelho}{Jo. 15, 26-27; 16, 1-4.}
\begin{paracol}{2}\latim{
\cruz Sequéntia sancti Evangélii secúndum Joánnem.
}\switchcolumn\portugues{
\cruz Continuação do santo Evangelho segundo S. João.
}\switchcolumn*\latim{
\blettrine{I}{n} illo témpore: Dixit Jesus discípulis suis: Cum vénerit Paráclitus, quem ego mittam vobis a Patre, Spíritum veritátis, qui a Patre procédit, ille testimónium perhibébit de me: et vos testimónium perhibébitis, quia ab inítio mecum estis. Hæc locútus sum vobis, ut non scandalizémini. Absque synagógis fácient vos: sed venit hora, ut omnis, qui intérficit vos, arbitrétur obséquium se præstáre Deo. Et hæc fácient vobis, quia non novérunt Patrem neque me. Sed hæc locútus sum vobis, ut, cum vénerit hora eórum, reminiscámini, quia ego dixi vobis.
}\switchcolumn\portugues{
\blettrine{N}{aquele} tempo, disse Jesus aos seus discípulos: Quando vier o Paráclito, o Espírito da verdade que procede do Pai, que Eu vos enviarei do Pai. Ele dará testemunho de mim, e vós, que estais comigo desde o princípio, Lhe dareis também testemunho de mim. Digo-vos estas coisas para que vos não escandalizeis: expulsar-vos-ão das sinagogas; e vem a hora em que qualquer que vos mate, julgará que presta um serviço a Deus. Tratar-vos-ão deste modo, porque não conhecem nem o Pai, nem me conhecem a mim. Digo-vos estas coisas para que, quando chegar a hora, vos lembreis de que vo-las disse.
}\end{paracol}

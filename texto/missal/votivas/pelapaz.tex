\subsection{Pela Paz}

\paragraphinfo{Intróito}{Ecl. 36, 18}
\begin{paracol}{2}\latim{
\rlettrine{D}{a} pacem, Dómine, sustinéntibus te, ut prophétæ tui fidéles inveniántur: exáudi preces servi tui, et plebis tuæ Israël. (T. P. Allelúja, allelúja.) \emph{Ps. 121, 1} Lætátus sum in his, quæ dicta sunt mihi: in domum Dómini íbimus.
℣. Gloria Patri \emph{\&c.}
}\switchcolumn\portugues{
\rlettrine{D}{ai} a paz, Senhor, aos que em Vós esperam, para que os vossos Profetas sejam encontrados verdadeiros; ouvi as súplicas do vosso servo e de Israel, vosso povo. (T. P. Aleluia, aleluia.) \emph{Sl. 121, 1} Regozijei-me quando me disseram: iremos para a casa do Senhor.
℣. Glória ao Pai \emph{\&c.}
}\end{paracol}

\paragraph{Oração}
\begin{paracol}{2}\latim{
\rlettrine{D}{eus,} a quo sancta desidéria, recta consília et justa sunt ópera: da servis tuis illam, quam mundus dare non potest, pacem; ut et corda nostra mandátis tuis dédita, et, hóstium subláta formídine, témpora sint, tua protectióne, tranquílla. Per Dóminum nostrum \emph{\&c.}
}\switchcolumn\portugues{
\slettrine{Ó}{} Deus, de quem nascem os santos desejos, os bons pensamentos e as obras justas, concedei aos vossos servos aquela paz que o mundo não pode dar, a fim de que os nossos corações amem intimamente os vossos mandamentos, e livres do temor dos inimigos e sustentados com vossa graça, gozemos dias tranquilos. Por nosso Senhor \emph{\&c.}
}\end{paracol}

\paragraphinfo{Epístola}{2. Mach. 1, 1-5.}
\begin{paracol}{2}\latim{
Léctio libri Machabæórum.
}\switchcolumn\portugues{
Lição do Livro dos Macabeus.
}\switchcolumn*\latim{
\rlettrine{F}{rátribus,} qui sunt per Ægýptum, Judǽis, salutem dicunt fratres, qui sunt in Jerosólymis, Judǽi et qui in regióne Judǽæ, et pacem bonam. Benefáciat vobis Deus et memínerit testaménti sui, quod locútus est ad Abraham et Isaac et Jacob, servórum suórum fidélium; et det vobis cor ómnibus, ut colátis eum et faciátis ejus voluntátem corde magno et ánimo volénti. Adapériat cor vestrum in lege sua et in præcéptis suis et fáciat pacem. Exáudiat oratiónes vestras et reconciliétur vobis nec vos déserat in témpore malo, Dóminus, Deus noster.
}\switchcolumn\portugues{
\rlettrine{O}{s} judeus de Jerusalém e do país da Judeia desejam aos seus Irmãos, que estão dispersos peto Egipto, saúde e boa paz. Que Deus vos cumule de bênçãos, que se lembre da aliança que fez com seus fiéis servos Abraão, Isaque e Jacob; que vos dê disposições rectas, a fim de que O adoreis e cumprais a sua vontade com coração generoso e ânimo dócil; que ilumine o vosso coração com sua lei e com seus preceitos, para que gozeis a paz; que Ele o Senhor, nosso Deus atenda às vossas orações, se reconcilie convosco e vos não abandone nos tempos adversos.
}\end{paracol}

\paragraphinfo{Gradual}{Sl. 121, 6-7}
\begin{paracol}{2}\latim{
\rlettrine{R}{ogáte} quæ ad pacem sunt Jerúsalem: et abundántia diligéntibus te. ℣. Fiat pax in virtúte tua, et abundántia in túrribus tuis.
}\switchcolumn\portugues{
\rlettrine{R}{ogai} ao Senhor que dê a paz aos que estão em Jerusalém. Que vivam na abundância dos bens aqueles que Vos amam, ó Sião. ℣. Que haja paz nas tuas fortalezas e abundância nas tuas torres.
}\switchcolumn*\latim{
Allelúja, allelúja. ℣. \emph{Ps. 147, 12} Lauda, Jerúsalem, Dóminum: lauda Deum tuum, Sion. Allelúja.
}\switchcolumn\portugues{
Aleluia, aleluia. ℣. \emph{Sl. 147, 12} Louvai o Senhor, vosso Deus, ó Sião. Aleluia.
}\end{paracol}

\textit{Após a Septuagésima, omite-se o Aleluia e o seguinte, e diz-se:}

\paragraphinfo{Trato}{Sl. 75, 2-4}
\begin{paracol}{2}\latim{
\rlettrine{N}{otus} in Judǽa Deus, in Israël magnum nomen ejus. ℣. Et factus est in pace locus ejus, et habitátio ejus in Sion. ℣. Ibi confrégit poténtias árcuum, scutum, gládium, et bellum.
}\switchcolumn\portugues{
\rlettrine{D}{eus} é conhecido na Judeia; o seu nome é grande em Israel; ℣. Escolheu em Sião o seu lugar de repouso e aí permaneceu; ℣. E contra os seus muros quebraram-se os poderes dos arcos, dos escudos, das espadas e da própria guerra.
}\end{paracol}

\textit{No Tempo Pascal omite-se, o Gradual e o Trato e diz-se:}

\begin{paracol}{2}\latim{
Allelúja, allelúja. ℣. \emph{Ps. 147, 12} Lauda, Jerúsalem, Dóminum: lauda Deum tuum, Sion. Allelúja. ℣. \emph{ibid., 14} Qui pósuit fines tuos pacem, et ádipe fruménti sátiat te. Allelúja.
}\switchcolumn\portugues{
Aleluia, aleluia. ℣. \emph{Sl. 147, 12} Louvai o Senhor, ó Jerusalém; louvai o vosso Deus, ó Sião. Aleluia. ℣. \emph{ibid., 14} Ele estabeleceu a paz nas vossas fronteiras e sustentou-vos com o pão mais fino. Aleluia.
}\end{paracol}

\paragraphinfo{Evangelho}{Jo. 20, 19-23}
\begin{paracol}{2}\latim{
\cruz Sequéntia sancti Evangélii secúndum Joánnem.
}\switchcolumn\portugues{
\cruz Continuação do santo Evangelho segundo S. João.
}\switchcolumn*\latim{
\blettrine{I}{n} illo témpore: Cum sero esset die illo, una sabbatórum, et fores essent clausæ, ubi erant discípuli congregáti propter metum Judæórum: venit Jesus, et stetit in médio, et dixit eis: Pax vobis. Et cum hoc dixísset, osténdit eis manus et latus. Gavísi sunt ergo discípuli, viso Dómino. Dixit ergo eis íterum: Pax vobis. Sicut misit me Pater, et ego mitto vos. Hæc cum dixísset, insufflávit, et dixit eis: Accípite Spíritum Sanctum: quorum remiseritis peccáta, remittúntur eis; et quorum retinuéritis, reténta sunt.
}\switchcolumn\portugues{
\blettrine{N}{aquele} tempo, chegada a tarde daquele dia, que era o primeiro da semana, encontrando-se os discípulos reunidos em um lugar, cujas portas estavam fechadas, por causa do medo que tinham dos judeus, veio Jesus, e, estando no meio deles, disse: «A paz seja convosco!». Depois de dizer isto, mostrou as suas mãos e o seu lado. E alegraram-se os discípulos, vendo o Senhor. E disse-lhes novamente: «A paz seja convosco! Assim como meu Pai me enviou, assim também vos envio». Ditas estas palavras, soprou sobre eles, dizendo: «Recebei o Espírito Santo. Àqueles a quem perdoardes os pecados, ser-lhes-ão perdoados, e àqueles a quem os retiverdes, ser-lhes-ão retidos».
}\end{paracol}

\paragraphinfo{Ofertório}{Sl. 134, 3 \& 6}
\begin{paracol}{2}\latim{
\rlettrine{L}{audáte} Dóminum, quia benígnus est: psállite nómini ejus, quóniam suávis est: ómnia quæcúmque vóluit, fecit in cœlo et in terra. (T. P. Allelúja.)
}\switchcolumn\portugues{
\rlettrine{L}{ouvai} o Senhor, pois Ele é benigno: cantai hinos em louvor do seu nome, porque Ele é suave. Ele fez tudo quanto quis no céu e na terra. (T. P. Aleluia.)
}\end{paracol}

\paragraph{Secreta}
\begin{paracol}{2}\latim{
\rlettrine{D}{eus,} qui credéntes in te pópulos nullis sinis cóncuti terróribus: dignáre preces et hóstias dicátæ tibi plebis suscípere; ut pax, a tua pietáte concéssa, Christianórum fines ab omni hoste fáciat esse secúros. Per Dóminum nostrum \emph{\&c.}
}\switchcolumn\portugues{
\slettrine{Ó}{} Deus, que não deixais abalar pelo terror os povos que confiam em Vós, dignai-Vos aceitar as preces e as oblatas deste devoto povo, para que, havendo-nos a vossa bondade concedido a paz, os países cristãos vivam em segurança contra seus inimigos. Por nosso Senhor \emph{\&c.}
}\end{paracol}

\paragraphinfo{Comúnio}{Jo. 14, 27}
\begin{paracol}{2}\latim{
\rlettrine{P}{acem} relínquo vobis: pacem meam do vobis, dicit Dóminus. (T. P. Allelúja.)
}\switchcolumn\portugues{
\rlettrine{E}{u} vos deixo a paz: Eu vos dou a minha paz, diz o Senhor. (T. P. Aleluia.)
}\end{paracol}

\paragraph{Postcomúnio}
\begin{paracol}{2}\latim{
\rlettrine{D}{eus,} auctor pacis et amátor, quem nosse vívere, cui servíre regnáre est: prótege ab ómnibus impugnatiónibus súpplices tuos; ut, qui in defensióne tua confídimus, nullíus hostilitátis arma timeámus. Per Dóminum \emph{\&c.}
}\switchcolumn\portugues{
\slettrine{Ó}{} Deus, que sois autor e amante da paz, aqueles que Vos conhecem vivem, e aqueles que Vos servem reinam; protegei, pois, contra todas as adversidades aqueles que Vos dirigem as suas súplicas, para que, havendo posto toda a confiança em Vós, nunca mais temamos as armas de nenhum inimigo. Por nosso Senhor \emph{\&c.}
}\end{paracol}

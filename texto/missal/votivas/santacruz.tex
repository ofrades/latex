\subsectioninfo{Missa de Santa Cruz}{Para a Sexta-feira}

\textit{Como na Missa da Exaltação da Santa Cruz, página \pageref{exaltacaocruz}, excepto:}

\paragraph{Oração}
\begin{paracol}{2}\latim{
\rlettrine{D}{eus,} qui unigéniti Fílii tui pretióso sanguine vivíficæ Crucis vexíllum sanctificáre voluísti: concéde, quǽsumus; eos, qui ejusdem sanctæ Crucis gaudent honóre, tua quoque ubíque protectióne gaudére. Per eúndem Dóminum nostrum \emph{\&c.}
}\switchcolumn\portugues{
\slettrine{Ó}{} Deus, que pelo precioso Sangue do vosso Filho Unigénito quisestes santificar o estandarte vivificante da Cruz, Vos pedimos, concedei àqueles que se regozijam, honrando esta mesma santa Cruz, que gozem sempre e em todos os lugares a vossa protecção. Pelo mesmo \emph{\&c.}
}\end{paracol}

\textit{No Tempo Pascal em vez da Oração Precedente diz-se a seguinte:}

\paragraph{Oração}
\begin{paracol}{2}\latim{
\rlettrine{D}{eus,} qui pro nobis Fílium tuum Crucis patíbulum subíre voluísti, ut inimíci a nobis expélleres potestátem: concéde nobis, fámulis tuis; ut resurrectiónis grátiam consequámur. Per eúndem Dóminum nostrum \emph{\&c.}
}\switchcolumn\portugues{
\slettrine{Ó}{} Deus, que quisestes que o vosso Filho sofresse o suplício da Cruz para nos livrar do poder do inimigo, concedei aos vossos servos que alcancem a graça de tomar parte na sua ressurreição. Pelo mesmo nosso Senhor \emph{\&c.}
}\end{paracol}

\paragraphinfo{Epístola}{Fl. 2, 8-11}
\begin{paracol}{2}\latim{
Léctio Epístolæ beáti Pauli Apóstoli ad Philippénses.
}\switchcolumn\portugues{
Lição da Ep.ª do B. Ap.º Paulo aos Filipenses.
}\switchcolumn*\latim{
\rlettrine{F}{ratres:} Christus factus est pro nobis obǿdiens usque ad mortem, mortem autem crucis. Propter quod et Deus exaltávit illum, et donávit illi nomen, quod est super omne nomen: \emph{(hic genuflectitur)} ut in nómine Jesu omne genu flectátur cœléstium, terréstrium et infernórum, et omnis lingua confiteátur, quia Dóminus Jesus Christus in glória est Dei Patris.
}\switchcolumn\portugues{
\rlettrine{M}{eus} irmãos: Cristo, por causa de nós, tornou-se obediente até à morte, e morte na cruz. Eis porque Deus o exaltou e lhe deu um nome que é superior a todos os outros. \emph{(Genuflecte-se)} De sorte que ao ser pronunciado o nome de Jesus todos os joelhos se devem dobrar no céu, na terra e nos infernos: e todas as línguas devem confessar que o senhor Jesus está na glória de Deus Pai.
}\end{paracol}

\textit{Depois da Septuagésima omite-se o Aleluia e o Verso que se segue, e diz-se o:}

\paragraph{Trato}
\begin{paracol}{2}\latim{
\rlettrine{A}{dorámus} te, Christe, et benedícimus tibi: quia per Crucem tuam redemísti mundum. ℣. Tuam Crucem adorámus, Dómine, tuam gloriósam recólimus passiónem: miserére nostri, qui passus es pro nobis. ℣. O Crux benedícta, quæ sola fuisti digna portáre Regem cœlórum et Dóminum.
}\switchcolumn\portugues{
\rlettrine{V}{os} adoramos e bendizemos, ó Cristo, porque salvastes o mundo pela vossa santa Cruz. ℣. Adoramos a vossa Cruz, Senhor, e honramos a vossa gloriosa paixão. O Vós, que sofrestes por nós, compadecei-Vos de nós. ℣. Ó Cruz bendita, só tu foste digna de sustentar o Senhor, que é o Rei dos céus.
}\end{paracol}

\textit{No Tempo Pascal omite-se o Gradual o Trato, e diz-se:}

\begin{paracol}{2}\latim{
Allelúja, allelúja. ℣. \emph{Ps. 95, 10} Dícite in géntibus, quia Dóminus regnávit a ligno. Allelúja. ℣. Dulce lignum, dulces clavos, dúlcia ferens póndera: quæ sola fuísti digna sustinére Regem cœlórum et Dóminum. Allelúja.
}\switchcolumn\portugues{
Aleluia, aleluia. ℣. \emph{Sl. 95, 10} Anunciai aos povos que Deus reinou pela Cruz. Aleluia. ℣. Ó lenho querido, que, preso aos benditos cravos, suportaste o doce fardo! Só tu foste digno de sustentar o Senhor, que é o Rei dos céus. Aleluia.
}\end{paracol}

\paragraphinfo{Evangelho}{Mt. 20, 17-19}
\begin{paracol}{2}\latim{
\cruz Sequéntia sancti Evangélii secúndum Matthǽum.
}\switchcolumn\portugues{
\cruz Continuação do santo Evangelho segundo S. Mateus.
}\switchcolumn*\latim{
\blettrine{I}{n} illo témpore: Assúmpsit Jesus duódecim discípulos secréto, et ait illis: Ecce, ascéndimus Jerosólymam, et Fílius hóminis tradétur princípibus sacerdótum et scribis, et condemnábunt eum morte, et tradent eum Géntibus ad illudéndum, et flagellándum, et crucifigéndum, et tértia die resúrget.
}\switchcolumn\portugues{
\blettrine{N}{aquele} tempo, subindo Jesus para Jerusalém, chamou de parte os seus doze discípulos e disse-lhes: «Eis que subimos; e aí o Filho do homem será entregue aos príncipes dos sacerdotes e aos escribas, que O condenarão à morte e O entregarão aos pagãos para zombarem d’Ele, e depois será flagelado e crucificado; porém ao terceiro dia ressuscitará».
}\end{paracol}

\paragraph{Secreta}
\begin{paracol}{2}\latim{
\rlettrine{H}{æc} oblátio, Dómine, quǽsumus, ab ómnibus nos purget offénsis: quæ in ara Crucis étiam totíus mundi tulit offénsam. Per eúndem Dóminum nostrum \emph{\&c.}
}\switchcolumn\portugues{
\rlettrine{S}{enhor,} Vos suplicamos, que esta oblação, que no altar da Cruz apagou os pecados do universo, nos purifique dos nossos pecados. Pelo mesmo \emph{\&c.}
}\end{paracol}

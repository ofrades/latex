\subsection{Por Qualquer Necessidade}

\paragraph{Intróito}
\begin{paracol}{2}\latim{
\rlettrine{S}{alus} pópuli ego sum, dicit Dóminus: de quacúmque tribulatióne clamáverint ad me, exáudiam eos: et ero illórum Dóminus in perpétuum. (T. P. Allelúja, allelúja.) \emph{Ps. 77, 1} Atténdite, pópule meus, legem meam: inclináte aurem vestram in verba oris mei.
℣. Gloria Patri \emph{\&c.}
}\switchcolumn\portugues{
\rlettrine{E}{u} sou a salvação do povo, diz o Senhor: quando em qualquer tribulação chamarem por mim, Eu os ouvirei: e serei perpetuamente o seu Senhor. (T. P. Aleluia, aleluia.) \emph{Sl. 77, 1} Ó meu povo, escutai a minha lei: inclinai os vosso ouvidos às palavras da minha boca.
℣. Glória ao Pai \emph{\&c.}
}\end{paracol}

\paragraph{Oração}
\begin{paracol}{2}\latim{
\rlettrine{I}{neffábilem} misericórdiam tuam, Dómine, nobis cleménter osténde: ut simul nos et a peccátis ómnibus éxuas; et a pœnis, quas pro his merémur, erípias. Per Dóminum \emph{\&c.}
}\switchcolumn\portugues{
\rlettrine{M}{ostrai-nos} clementemente, Senhor, a vossa inefável misericórdia; e, livrando-nos dos nossos pecados, aliviai-nos dos castigos em que incorremos por causa deles. Por nosso Senhor \emph{\&c.}
}\end{paracol}

\paragraphinfo{Epístola}{Jr. 14, 7-8 \& 9}
\begin{paracol}{2}\latim{
Léctio Jeremíæ Prophétæ.
}\switchcolumn\portugues{
Lição do Profeta Jeremias.
}\switchcolumn*\latim{
\rlettrine{S}{i} iniquitátes nostræ respónderint nobis: Dómine, fac propter nomen tuum, quóniam multæ sunt aversiónes nostræ: tibi peccávimus. Exspectátio Israël, salvátor ejus in témpore tribulatiónis. Tu autem in nobis es, Dómine, et nomen tuum invocátum est super nos, ne derelínquas nos, Dómine, Deus noster.
}\switchcolumn\portugues{
\rlettrine{S}{e} as nossas iniquidades servem de nossa acusação diante de Vós, Senhor, sede propício para nós pela glória do vosso nome; porque numerosas são as nossas revoltas e os nossos pecados contra Vós! Sois a esperança de Israel e o seu Salvador, no tempo da tribulação. Vós estais no meio de nós, Senhor, e o vosso nome é invocado por nós. Não nos abandoneis, pois, ó Senhor, nosso Deus.
}\end{paracol}

\paragraphinfo{Gradual}{Sl. 43, 3-9}
\begin{paracol}{2}\latim{
\rlettrine{L}{iberásti} nos, Dómine, ex affligéntibus nos: et eos, qui nos oderunt, confudísti. ℣. In Deo laudábimur tota die: et in nómine tuo confitébimur in sǽcula.
}\switchcolumn\portugues{
\rlettrine{L}{ivrastes-nos,} Senhor, daqueles que nos afligiam: e confundistes aqueles que nos odiavam. ℣. Alegrar-nos-emos em Deus continuamente: glorificaremos sempre o vosso nome.
}\switchcolumn*\latim{
Allelúja, allelúja. ℣. \emph{Ps. 78, 9-10} Propítius esto, Dómine, peccátis nostris: ne quando dicant gentes: Ubi est Deus eórum? Allelúja.
}\switchcolumn\portugues{
Aleluia, aleluia. ℣. \emph{Sl. 78, 9-10} Perdoai-nos, Senhor, os nossos pecados: para que os povos não digam: «Onde está o seu Deus»?». Aleluia.
}\end{paracol}

\textit{Depois da Septuagésima, omite-se o Aleluia e o Seguinte, e diz-se o:}

\paragraphinfo{Trato}{Sl. 24, 17-18 \& 1-4}
\begin{paracol}{2}\latim{
\rlettrine{D}{e} necessitátibus meis éripe me, Dómine: vide humilitátem meam et labórem meum: et dimitte ómnia peccáta mea. ℣. Ad te, Dómine, levávi ánimam meam: Deus meus, in te confído, non erubéscam: neque irrídeant me inimíci mei. ℣. Etenim univérsi, qui te exspéctant, non confundéntur: confundántur omnes faciéntes vana.
}\switchcolumn\portugues{
\rlettrine{L}{ivrai-me,} Senhor, das minhas tribulações: vede a minha miséria e as minhas penas: e perdoai todos meus pecados. ℣. A Vós, Senhor, elevei a minha alma: meu Deus, confio em Vós: não ficarei envergonhado; pois os meus inimigos não triunfarão de mim. ℣. Não serão confundidos, Senhor, os que confiam em Vós: mas serão confundidos os que procedem em vão.
}\end{paracol}

\textit{No Tempo Pascal omite-se o Gradual e o Trato, e diz-se:}

\begin{paracol}{2}\latim{
Allelúja, allelúja. ℣. \emph{Ps. 78, 9-10} Propítius esto, Dómine, peccátis nostris: ne quando dicant gentes: Ubi est Deus eórum? Allelúja. ℣. \emph{Ps. 30, 8} Exsultábo et lætábor in misericórdia tua, quóniam respexísti humilitátem meam: salvasti de necessitátibus ánimam meam. Allelúja.
}\switchcolumn\portugues{
Aleluia, aleluia. ℣. \emph{Sl. 78, 9-10} Perdoai-nos, Senhor, Os nossos pecados: para que os povos não digam: «Onde está o seu Deus?». Aleluia. ℣. \emph{Sl. 30, 8} Exultarei de alegria por causa da vossa misericórdia, pois Vós tivestes compaixão da minha desgraça: e salvastes a minha alma das minhas tribulações. Aleluia.
}\end{paracol}

\paragraphinfo{Evangelho}{Mc. 11, 22-26}
\begin{paracol}{2}\latim{
\cruz Sequéntia sancti Evangélii secúndum Marcum.
}\switchcolumn\portugues{
\cruz Continuação do santo Evangelho segundo S. Marcos.
}\switchcolumn*\latim{
\blettrine{I}{n} illo témpore: Dixit Jesus discípulis suis: Habete fidem Dei. Amen, dico vobis, quia, quicúmque díxerit huic monti: Tóllere et míttere in mare, et non hæsitáverit in corde suo, sed credíderit, quia, quodcúmque díxerit, fiat, fiet ei. Proptérea dico vobis: Omnia quæcúmque orántes pétitis, crédite quia accipiétis, et evénient vobis. Et cum stábitis ad orándum, dimíttite, si quid habétis advérsus áliquem: ut et Pater vester, qui in cælis est, dimíttat vobis peccáta vestra. Quod si vos non dimiséritis: nec Pater vester, qui in cœlis est, dimíttet vobis peccáta vestra.
}\switchcolumn\portugues{
\blettrine{N}{aquele} tempo, disse Jesus aos seus discípulos: «Tende fé em Deus! Em verdade vos digo: todo aquele que disser a este monte: «tira-te e lança-te ao mar», se ele não hesitar no seu coração e, antes, acreditar que acontecerá tudo quanto ele tiver dito, tudo será feito, como ele tiver dito. Por isso vos digo: tudo quanto pedirdes, orando, crede que o recebereis, e que assim acontecerá. Quando fordes orar, se tiverdes alguma coisa contra alguém, perdoai-lhe, para que o vosso Pai, que está nos céus, perdoe também os vossos pecados; pois se vós não perdoardes, também o vosso Pai, que está nos céus, não perdoará os vossos pecados.
}\end{paracol}

\paragraphinfo{Ofertório}{Sl. 137, 7}
\begin{paracol}{2}\latim{
\rlettrine{S}{i} ambulávero in médio tribulatiónis, vivificábis me, Dómine: et super iram inimicórum meórum
exténdes manum tuam, et salvum me fáciet déxtera lua. (T. P. Allelúja.)
}\switchcolumn\portugues{
\rlettrine{S}{e} me, encontrar no meio das tribulações, Vós me dareis a vida, Senhor: Vós impusestes a vossa mão contra o furor dos meus inimigos e a vossa dextra salvou-me. (T. P. Aleluia.)
}\end{paracol}

\paragraph{Secreta}
\begin{paracol}{2}\latim{
\rlettrine{P}{uríficet} nos, Dómine, quǽsumus, múneris præséntis oblátio: et dignos sacra participatióne perfíciat. Per Dóminum \emph{\&c.}
}\switchcolumn\portugues{
\qlettrine{Q}{ue} a oferta deste sacrifício nos purifique, Senhor, e nos torne verdadeiramente dignos de participarmos santamente destes mistérios. Por \emph{\&c.}
}\end{paracol}

\paragraphinfo{Comúnio}{Sl. 118, 49-50}
\begin{paracol}{2}\latim{
\rlettrine{M}{eménto} verbi tui servo tuo, Dómine, in quo mihi spemdedísti: hæc me consoláta est in humilitáte mea. (T. P. Allelúja.)
}\switchcolumn\portugues{
\rlettrine{L}{embrai-Vos,} Senhor, da palavra que dissestes ao vosso servo, a qual me encheu de esperança. Foi ela que me serviu de consolação nas minhas penas. (T. P. Aleluia.)
}\end{paracol}

\paragraph{Postcomúnio}
\begin{paracol}{2}\latim{
\rlettrine{P}{ræsta,} quǽsumus, Dómine: ut, terrenis afféctibus expiáti, ad superni plenitúdinem sacraménti, cujus libávimus sancta, tendámus. Per Dóminum \emph{\&c.}
}\switchcolumn\portugues{
\rlettrine{D}{ignai-Vos} conceder-nos, Senhor, que, livres nós de todos os afectos terrenos, procuremos a posse completa do sacramento, que acabámos de receber, nas divinas espécies. Por \emph{\&c.}
}\end{paracol}

\subsection{Suplicando Graça da Boa Morte}

\paragraphinfo{Intróito}{Sl. 12, 4}
\begin{paracol}{2}\latim{
\rlettrine{I}{llúmina} óculos meos, ne umquam obdórmiam in morte, ne quando dicat inimícus meus: Præválui advérsus eum. (T. P. Allelúja, allelúja.) \emph{Ps. ibid., 1} Usquequo, Dómine, oblivísceris me in finem? usquequo avertis fáciem tuam a me?
℣. Gloria Patri \emph{\&c.}
}\switchcolumn\portugues{
\rlettrine{I}{luminai} os meus olhos, a fim de que eu não adormeça na morte para sempre e para que meu inimigo não diga a meu respeito: prevaleci contra ele. (T. P. Aleluia, aleluia.) \emph{Sl. ibid., 1} Até quando, Senhor, me esquecereis? Esquecer-me-eis até ao fim? Até quando conservareis afastada de mim a vossa face?
℣. Glória ao Pai \emph{\&c.}
}\end{paracol}

\paragraph{Oração}
\begin{paracol}{2}\latim{
\rlettrine{O}{mnípotens} et miséricors Deus, qui humáno géneri et salútis remédia et vitæ ætérnæ múnera contulísti: réspice propítius nos fámulos tuos, et ánimas réfove, quas creásti; ut in hora éxitus eárum, absque peccáti mácula tibi, Creatóri suo, per manus sanctórum Angelórum repræsentári mereántur. Per Dóminum nostrum \emph{\&c.}
}\switchcolumn\portugues{
\slettrine{Ó}{} Deus omnipotente e misericordioso, que concedestes ao género humano os remédios para a sua salvação e as recompensas da vida eterna, olhai benigno para o vosso servo, cujo corpo está oprimido pela enfermidade, e fortalecei a sua alma que criastes, a fim de que à hora da sua morte ele compareça, acompanhado pelos santos Anjos, sem mácula alguma de pecado diante de Vós, que sois o seu Criador. Por nosso Senhor \emph{\&c.}
}\end{paracol}

\paragraphinfo{Epístola}{Rm. 14, 7-12}
\begin{paracol}{2}\latim{
Léctio Epístolæ beáti Pauli Apóstoli ad Romános.
}\switchcolumn\portugues{
Lição da Ep.ª do B. Ap.º Paulo aos Romanos.
}\switchcolumn*\latim{
\rlettrine{F}{ratres:} Nemo nostrum sibi vivit, et nemo sibi móritur. Sive enim vívimus, Dómino vívimus: sive mórimur, Dómino mórimur. Sive ergo vívimus sive mórimur, Dómini sumus. In hoc enim Christus mórtuus est et resurréxit: ut et mortuórum et vivórum dominétur. Tu autem quid júdicas fratrem tuum? aut tu quare spernis fratrem tuum? Omnes enim stábimus ante tribúnal Christi. Scriptum est enim: Vivo ego, dicit Dóminus, quóniam mihi flectétur omne genu: et omnis lingua confitébitur Deo. Itaque unusquísque nostrum pro se ratiónem reddet Deo.
}\switchcolumn\portugues{
\rlettrine{M}{eus} irmãos: Nenhum de nós vive para si, nem morre para si; mas, vivendo, para o Senhor vivemos; e, se morremos, para o Senhor morremos. Portanto, quer vivamos, quer morramos, somos do Senhor. Foi por isto que Cristo morreu e ressuscitou: para que fosse Senhor tanto dos mortos, como dos vivos. Porque julgais, então, o vosso irmão? Certamente todos comparecemos ante o tribunal de Cristo; pois está escrito: «Por minha vida, diz o Senhor: pois ante mim se dobrará todo o joelho, e toda a língua confessará Deus». Assim, portanto, cada um de nós dará conta a Deus de si próprio.
}\end{paracol}

\paragraphinfo{Gradual}{Sl. 22, 4}
\begin{paracol}{2}\latim{
\rlettrine{S}{i} ámbulem in médio umbræ mortis, non timébo mala: quóniam tu mecum es, Dómine. ℣. Virga tua et báculus tuus, ipsa me consoláta sunt.
}\switchcolumn\portugues{
\rlettrine{A}{inda} mesmo que eu caminhe no meio das sombras da morte, não temerei os males; pois Vós, Senhor, estais comigo. ℣. Vossa vara e o vosso báculo servem-me de consolação.
}\switchcolumn*\latim{
Allelúja, allelúja. ℣. \emph{Ps. 30, 2-3} In te, Dómine, sperávi, non confúndar in ætérnum: in justítia tua líbera me et éripe me: inclína ad me aurem tuam, accélera, ut erípias me. Allelúja.
}\switchcolumn\portugues{
Aleluia, aleluia. ℣. \emph{Sl. 30, 2-3} Tenho esperança em Vós, Senhor: não ficarei confundido para sempre. Pela vossa justiça, livrai-me e salvai-me: Volvei os vossos ouvidos para mim e apressai-Vos em salvar-me. Aleluia.
}\end{paracol}

\textit{Depois da Septuagésima, omite-se o Aleluia e o Verso seguinte e diz-se o:}

\paragraphinfo{Trato}{Sl. 24, 17-18 \& 1-4}
\begin{paracol}{2}\latim{
\rlettrine{D}{e} necessitátibus meis éripe me, Dómine: vide humilitátem meam et labórem meum: et dimítte ómnia peccáta mea. ℣. Ad te, Dómine, levávi ánimam meam: Deus meus, in te confído, non erubéscam: neque irrídeant me inimíci mei. ℣. Etenim univérsi, qui te exspéctant, non confundéntur: confundántur omnes faciéntes vana.
}\switchcolumn\portugues{
\rlettrine{L}{ivrai-me,} Senhor, das minhas tribulações: vede a minha miséria e as minhas penas: e perdoai todos meus pecados. ℣. A Vós, Senhor, elevei a minha alma: meu Deus, confio em Vós: não ficarei envergonhado, pais os meus inimigos não triunfarão de mim. ℣. Não serão confundidos, Senhor, os que confiam em Vós; mas serão confundidos os que procedem em vão.
}\end{paracol}

\textit{No Tempo Pascal omite-se O Gradual e O Trato e diz-se:}

\begin{paracol}{2}\latim{
Allelúja, allelúja. ℣. \emph{Ps. 113, 1} In éxitu Israël de Ægýpto, domus Jacob de pópulo bárbaro. Allelúja. ℣. \emph{Ps. 107, 2} Parátum cor meum, Deus, parátum cor meum: cantábo et psallam tibi, glória mea. Allelúja.
}\switchcolumn\portugues{
Aleluia, aleluia. ℣. \emph{Sl. 113, 1} Quando Israel saiu do Egipto e a casa de Jacob no meio de um povo bárbaro. Aleluia. ℣. \emph{Sl. 107, 2} Meu coração está preparado. Eu cantarei Salmos em vossa honra na minha glória. Aleluia.
}\end{paracol}

\paragraphinfo{Evangelho}{Lc. 21, 34-36}
\begin{paracol}{2}\latim{
\cruz Sequéntia sancti Evangélii secúndum Lucam.
}\switchcolumn\portugues{
\cruz Continuação do santo Evangelho segundo S. Lucas.
}\switchcolumn*\latim{
\blettrine{I}{n} illo témpore: Dixit Jesus discípulis suis: Atténdite vobis, ne forte gravéntur corda vestra in crápula et ebrietáte et curis hujus vitæ: et supervéniat in vos repentína dies illa: tamquam láqueus enim supervéniet in omnes, qui sedent super fáciem omnis terræ. Vigiláte ítaque, omni témpore orántes, ut digni habeámini fúgere ista ómnia, quæ futúra sunt, et stare ante Fílium hóminis.
}\switchcolumn\portugues{
\blettrine{N}{aquele} tempo, disse Jesus aos seus discípulos: «Tende cuidado convosco para que não aconteça agravarem-se os vossos corações com os excessos das comidas e das bebidas e com os cuidados desta vida, a fim de que não caia sobre vós aquele dia repentino; porquanto ele virá, à semelhança de um laço, sobre todos aqueles que estão à face da terra. Vigiai, pois, orando em todo o tempo, para que sejais dignos de evitar todas estas cousas, que hão-de acontecer, e de comparecerdes ante o Filho do homem».
}\end{paracol}

\paragraphinfo{Ofertório}{Sl. 30, 15-16}
\begin{paracol}{2}\latim{
\rlettrine{I}{n} te sperávi, Dómine; dixi: Tu es Deus meus, in mánibus tuis témpora mea. (T. P. Allelúja)
}\switchcolumn\portugues{
\rlettrine{T}{enho} esperança em Vós, Senhor; e por isso eu disse: «Vós sois o meu Deus e nas vossas mãos estão os meus destinos». (T. P. Aleluia.)
}\end{paracol}

\paragraph{Secreta}
\begin{paracol}{2}\latim{
\rlettrine{S}{úscipe,} quǽsumus, Dómine, hóstiam, quam tibi offérimus pro extrémo vitæ nostræ, et concéde: ut per eam univérsa nostra purgéntur delícta; ut, qui tuæ dispositiónis flagéllis in hac vita attérimur, in futúra réquiem consequámur ætérnam. Per Dóminum nostrum \emph{\&c.}
}\switchcolumn\portugues{
\rlettrine{A}{ceitai,} Senhor Vos suplicamos, esta hóstia que Vos oferecemos pelo vosso servo, que se encontra no fim da vida, e dignai-Vos permitir que, em virtude dela, ele seja purificado das suas faltas, a fim de que, havendo sido provado nesta vida com os flagelos da vossa providência, alcance o repouso eterno na vida futura. Por nosso Senhor \emph{\&c.}
}\end{paracol}

\paragraphinfo{Comúnio}{Sl. 70, 16-17 et 18}
\begin{paracol}{2}\latim{
\rlettrine{D}{ómine,} memorábor justítiæ tuæ solíus: Deus, docuísti me a juventúte mea: et usque in senéctam et sénium, Deus, ne derelínquas me. (T. P. Allelúja.)
}\switchcolumn\portugues{
\rlettrine{L}{embrar-me-ei,} Senhor, da vossa justiça, pois só Vós a possuís! Instruístes-me, ó Deus, desde a minha juventude: e até à velhice e aos últimos suspiros me não abandonareis, ó Deus! (T. P. Aleluia.)
}\end{paracol}

\paragraph{Postcomúnio}
\begin{paracol}{2}\latim{
\qlettrine{Q}{uǽsumus} cleméntiam tuam, omnípotens Deus, ut per hujus virtútem sacraménti nos fámulos tuos grátia tua confirmáre dignéris: ut in hora mortis nostræ non præváleat contra nos adversárius; sed cum Angelis tuis tránsitum habére mereámur ad vitam. Per Dóminum \emph{\&c.}
}\switchcolumn\portugues{
\rlettrine{I}{mploramos} a vossa clemência, ó Deus omnipotente, a fim de que, pela virtude deste sacramento, Vos digneis fortalecer o vosso servo com vossa graça; e que à hora da sua morte o inimigo não prevaleça contra ele, mas mereça transitar para a vida eterna, acompanhado pelos vossos Anjos. Por nosso Senhor \emph{\&c.}
}\end{paracol}

\subsection{Pelos Enfermos}

\paragraphinfo{Intróito}{Sl. 54, 2-3}
\begin{paracol}{2}\latim{
\rlettrine{E}{xáudi,} Deus, oratiónem meam, et ne despéxeris deprecatiónem meam: inténde in me et exáudi me. (T. P. Allelúja, allelúja.) \emph{Ps. ibid., 3-4} Contristátus sum in exercitatióne mea: et conturbátus sum a voce inimíci et a tribulatióne peccatóris.
℣. Gloria Patri \emph{\&c.}
}\switchcolumn\portugues{
\rlettrine{O}{uvi,} ó Deus, a minha oração e não desprezeis a minha humilde súplica: atendei-me e ouvi-me. (T. P. Aleluia, aleluia.) \emph{Sl. ibid., 3-4} Estou cheio de angústia por causa das aflições, que me oprimem. Estou perturbado por causa da voz do meu inimigo e das tribulações que me infligem os pecadores.
℣. Glória ao Pai \emph{\&c.}
}\end{paracol}

\paragraph{Oração}
\begin{paracol}{2}\latim{
\rlettrine{O}{mnípotens} sempitérne Deus, salus ætérna credéntium: exáudi nos pro fámulis tuis infírmis, pro quibus misericórdiæ tuæ implorámus auxílium; ut, reddíta sibi sanitáte, gratiárum tibi in Ecclésia tua réferant actiónes. Per Dóminum \emph{\&c.}
}\switchcolumn\portugues{
\rlettrine{O}{mnipotente} e sempiterno Deus, que salvais eternamente os que crêem em Vós, ouvi as orações que Vos dirigimos pelos vossos servos enfermos, em favor dos quais imploramos o socorro da vossa misericórdia, a fim de que, readquirindo a saúde, Vos rendam acções de graças na vossa Igreja. Por nosso Senhor \emph{\&c.}
}\end{paracol}

\paragraphinfo{Epístola}{Tg. 5, 13-16}
\begin{paracol}{2}\latim{
Léctio Epístolæ beáti Jacóbi Apóstoli.
}\switchcolumn\portugues{
Lição da Ep.ª do B. Ap.º Tiago.
}\switchcolumn*\latim{
\rlettrine{C}{aríssimi:} Tristátur áliquis vestrum? oret.Æquo ánimo est? psallat. Infirmátur quis in vobis? indúcat presbýteros Ecclésiæ, et orent super eum, ungéntes eum óleo in nómine Dómini: et orátio fídei salvábit infírmum, et alleviábit eum Dóminus: et si in peccátis sit, remitténtur ei. Confitémini ergo altérutrum peccáta vestra, et oráte pro ínvicem, ut salvémini.
}\switchcolumn\portugues{
\rlettrine{C}{aríssimos:} Está triste algum de vós? Que ore. Está alegre? Que cante Salmos. Está enfermo algum de vós? Que sejam chamados os Sacerdotes da Igreja, para que orem por ele, ungindo-o com o óleo em nome do Senhor. Então a oração da fé salvará o enfermo; o Senhor o aliviará e, se tiver pecados, ser-lhe-ão perdoados. Confessai, portanto, os vossos pecados um ao outro e orai uns pelos outros, a fim de que sejais salvos.
}\end{paracol}

\paragraphinfo{Gradual}{Sl. 6, 3-4}
\begin{paracol}{2}\latim{
\rlettrine{M}{iserére} mihi, Dómine, quóniam infírmus sum: sana me, Dómine. ℣. Conturbáta sunt ómnia ossa mea: et ánima mea turbáta est valde.
}\switchcolumn\portugues{
\rlettrine{C}{ompadecei-Vos} de mim, Senhor, pois estou enfermo: curai-me, Senhor. ℣. Estou oprimido em todo meu corpo: estou perturbado até ao íntimo da minha alma.
}\switchcolumn*\latim{
Allelúja, allelúja. ℣. \emph{Ps. 101, 2} Dómine, exáudi oratiónem meam: et clamor meus ad te pervéniat. Allelúja.
}\switchcolumn\portugues{
Aleluia, aleluia. ℣. \emph{Sl. 101, 2} Ouvi, Senhor, a minha oração: e que meu clamor chegue até Vós. Aleluia.
}\end{paracol}

\textit{Após a Septuagésima, omite-se o Aleluia e o seguinte, e diz-se:}

\paragraphinfo{Trato}{Sl. 30, 10-11}
\begin{paracol}{2}\latim{
\rlettrine{M}{iserére} mei, Dómine, quóniam tríbulor: conturbátus est in ira óculus meus, ánima mea et venter meus. ℣. Quóniam defécit in dolóre vita mea, et anni mei in gemítibus. ℣. Infirmáta est in paupertáte virtus mea: et ossa mea conturbáta sunt.
}\switchcolumn\portugues{
\rlettrine{C}{ompadecei-Vos} de mim, Senhor, pois estou atribulado: os meus olhos, a minha alma e até as minhas entranhas estão atribuladas. ℣. Pois a minha vida consome-se no meio da dor: e os meus anos em gemidos! ℣. Minhas forças debilitaram-se por causa da minha pobreza: e os meus ossos estão abalados.
}\end{paracol}

\textit{No Tempo Pascal omite-se o Gradual- e a Trato, e diz-se:}

\begin{paracol}{2}\latim{
Allelúja, allelúja. ℣. \emph{Ps. 101, 2} Dómine, exáudi oratiónem meam: et clamor meus ad te pervéniat. Allelúja. ℣. \emph{Ps. 27, 7} In Deo sperávit cor meum, et adjútus sum: et reflóruit caro mea, et ex voluntáte mea confitébor ei. Allelúja.
}\switchcolumn\portugues{
Aleluia, aleluia. ℣. \emph{Sl. 101, 2} Senhor, ouvi a minha oração: e que meu clamor chegue até Vós. Aleluia. ℣. \emph{Sl. 27, 7} Meu coração teve esperança em Deus e foi socorrido: e a minha carne refloresceu: eis porque O louvarei de todo meu coração. Aleluia.
}\end{paracol}

\paragraphinfo{Evangelho}{Mt. 8, 5-13}
\begin{paracol}{2}\latim{
\cruz Sequéntia sancti Evangélii secúndum Matthǽum.
}\switchcolumn\portugues{
\cruz Continuação do santo Evangelho segundo S. Mateus.
}\switchcolumn*\latim{
\blettrine{I}{n} illo témpore: Cum introísset Jesus Caphárnaum, accessit ad eum centúrio, rogans eum et dicens: Dómine, puer meus jacet in domo paralýticus, et male torquétur. Et ait illi Jesus: Ego véniam et curábo eum. Et respóndens centúrio, ait: Dómine, non sum dignus, ut intres sub tectum meum: sed tantum dic verbo, et sanábitur puer meus. Nam et ego homo sum sub potestáte constitútus, habens sub me mílites, et dico huic: Vade, et vadit; et alii: Veni, et venit; et servo meo: Fac hoc, ei facit. Audiens autem Jesus, mirátus est et sequéntibus se dixit: Amen, dico vobis, non inveni tantam fidem in Israël. Dico autem vobis, quod multi ab Oriénte et Occidénte vénient, et recúmbent cum Abraham et Isaac et Jacob in regno cœlórum: filii autem regni ejiciéntur in ténebras exterióres: ibi erit fletus et stridor déntium. Et dixit Jesus centurióni: Vade, et, sicut credidísti, fiat tibi. Et sanátus est puer in illa hora.
}\switchcolumn\portugues{
\blettrine{N}{aquele} tempo, entrando Jesus em Cafarnaum, aproximou-se dele um centurião, pedindo-Lhe e dizendo: «Senhor, o meu servo jaz em casa paralítico e sofre gravemente». Jesus disse-lhe: «Eu irei e o curarei». Mas o centurião respondeu: «Senhor, não sou digno de que entreis em minha casa; dizei somente uma palavra e o meu servo será curado. Pois eu, posto que seja um homem sujeito a outros superiores, tenho soldados debaixo das minhas ordens. E digo a um: vai; e ele vai. E digo a outro: vem; e ele vem. E digo ao meu servo: faz isto; e ele faz». Ouvindo Jesus isto, ficou admirado e disse aos que O seguiam: «Em verdade vos digo que nunca encontrei tão grande fé em Israel! Declaro-vos que muitos virão do Oriente e do Ocidente e tomarão lugar no banquete com Abraão, Isaque e Jacob, no reino dos céus; mas os filhos do reino serão lançados nas trevas exteriores, onde só haverá pranto e ranger de dentes». Então Jesus disse ao centurião: «Vai; e, assim como acreditastes, assim acontecerá». E naquela hora o servo foi curado.
}\end{paracol}

\paragraphinfo{Ofertório}{Sl. 54, 2-3}
\begin{paracol}{2}\latim{
\rlettrine{E}{xáudi,} Deus, oratiónem meam, et ne despéxeris deprecatiónem meam: inténde in me et exáudi me. (T. P. Allelúja.)
}\switchcolumn\portugues{
\rlettrine{S}{enhor,} ouvi a minha oração e não desprezeis as minhas súplicas. Volvei-Vos para mim e ouvi-me. (T. P. Aleluia.)
}\end{paracol}

\paragraph{Secreta}
\begin{paracol}{2}\latim{
\rlettrine{D}{eus,} cujus nútibus vitæ nostræ moménta decúrrunt: súscipe preces et hóstias famulórum tuórum, pro quibus ægrotántibus misericórdiam tuam implorámus; ut, de quorum perículo metúimus, de eórum salúte lætémur. Per Dóminum \emph{\&c.}
}\switchcolumn\portugues{
\slettrine{Ó}{} Deus, cuja vontade governa o decurso dos instantes da nossa vida, recebei as preces e as oblatas dos vossos servos enfermos, em favor dos quais imploramos a vossa misericórdia, a fim de que, depois de havermos temido o perigo em que eles se encontravam, nos regozijemos de os ver sãos e salvos. Por nosso Senhor \emph{\&c.}
}\end{paracol}

\paragraphinfo{Comúnio}{Sl. 30, 17-18}
\begin{paracol}{2}\latim{
\rlettrine{I}{llúmina} fáciem tuam super servum tuum, et salvum me fac in tua misericórdia: Dómine, non confúndar, quóniam invocávi te. (T. P. Allelúja.)
}\switchcolumn\portugues{
\rlettrine{L}{ançai} o esplendor da vossa face sobre o vosso servo: e salvai-me consoante a vossa misericórdia! Senhor, visto que Vos invoquei, fazei que não seja confundido. (T. P. Aleluia.)
}\end{paracol}

\paragraph{Postcomúnio}
\begin{paracol}{2}\latim{
\rlettrine{D}{eus,} infirmitátis humánæ singuláre præsídium: auxílii tui super infírmos fámulos tuos osténde virtútem; ut, ope misericórdiæ tuæ adjúti, Ecclésiæ tuæ sanctæ incólumes repræsentári mereántur. Per Dóminum \emph{\&c.}
}\switchcolumn\portugues{
\slettrine{Ó}{} Deus, que sois o único apoio da fraqueza humana, mostrai aos vossos servos enfermos o poder do vosso socorro, a fim de que, socorridos pela vossa misericórdia, sejam restituídos sãos e salvos ao seio da vosso Igreja. Por nosso Senhor \emph{\&c.}
}\end{paracol}

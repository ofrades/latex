\section{Comum dos Santos}

Os Santos caem em classes distintas: Mártires, Confessores, Bispos, Confessores não Bispos, Doutores, Virgens Mártires, Virgens e Mulheres Santas que não são nem Virgens nem Mártires. Para além das Festas em honra dos Santos destas diferentes classes há a Festa que comemora o aniversário da Dedicação de uma Igreja. Todas estas Festas têm uma Missa própria. Essas Festas estão na seguinte secção deste Missal e são usadas para Santos que não têm uma Missa especialmente dedicada a eles. As Missas de Nossa Senhora foram também trazidas para esta secção.

\subsectioninfo{Um ou Muitos Sumos Pontífices}{Missa Si díligis me}\label{sumospontifices}

\paragraphinfo{Intróito}{Jo. 21, 15, 16 \& 17}
\begin{paracol}{2}\latim{
\rlettrine{S}{i} díligis me, Simon Petre, pasce agnos meos, pasce oves meas. (T. P. Allelúja, allelúja.) \emph{Ps. 29, 2} Exaltábo te, Dómine, quóniam suscepísti me, nec delectásti inimícos meos super me.
℣. Gloria Patri \emph{\&c.}
}\switchcolumn\portugues{
\rlettrine{S}{e} me amas, Simão-Pedro, apascenta os meus cordeiros, apascenta as minhas ovelhas. (T. P. Aleluia, aleluia.) \emph{Sl. 29, 2} Louvar-Vos-ei, Senhor, pois me acolhestes e não permitistes que meus inimigos se rissem de mim.
℣. Glória ao Pai \emph{\&c.}
}\end{paracol}

\paragraph{Oração}
\begin{paracol}{2}\latim{
\rlettrine{G}{regem} tuum, Pastor ætérne, placátus inténde: et, per beátum {\redx N.} (Mártyrem tuum atque) Summum Pontíficem, perpétua protectióne custódi; quem totíus Ecclésiæ præstitísti esse pastórem. Per Dóminum nostrum \emph{\&c.}
}\switchcolumn\portugues{
\slettrine{Ó}{} Pastor eterno, atendei propício ao vosso rebanho; e guardai-o com vossa perpétua protecção por intercessão do bem-aventurado {\redx N.} (Vosso Mártir e) Sumo Pontífice, o qual escolhestes como pastor de toda a Igreja. Por nosso Senhor \emph{\&c.}
}\end{paracol}

\textit{Se, porém, se fizer comemoração doutro Sumo Pontífice nesta mesma Missa, dir-se-á a seguinte Oração, em vez da Precedente:}

\paragraph{Oração}
\begin{paracol}{2}\latim{
\rlettrine{D}{eus,} qui Ecclésiam tuam, in apostólicæ petræ soliditáte fundátam, ab infernárum éruis terróre portárum: præsta, quǽsumus; ut, intercedénte beáto {\redx N.} (Mártyre tuo atque) Summo Pontífice, in tua veritáte persístens, contínua securitáte muniátur. Per Dominum \emph{\&c.}
}\switchcolumn\portugues{
\slettrine{Ó}{} Deus, que do terror das portas do inferno livrastes a vossa Igreja, fundada na solidez da pedra apostólica, concedei-nos, Vos suplicamos, que, por intercessão do bem-aventurado {\redx N.} (Vosso Mártir e) Sumo Pontífice, sempre persista na vossa verdade e seja protegida em contínua segurança. Por nosso Senhor \emph{\&c.}
}\end{paracol}

\paragraphinfo{Epístola}{l. Pe. 5, 1-4 \& 10-11}
\begin{paracol}{2}\latim{
Léctio Epístolæ beáti Petri Apóstoli.
}\switchcolumn\portugues{
Lição da Ep.ª do B. Ap.º Pedro.
}\switchcolumn*\latim{
\rlettrine{C}{aríssimi:} Senióres, qui in vobis sunt, obsécro consénior et testis Christi passiónum, qui et ejus, quæ in futúro revelánda est, glóriæ communicátor: páscite qui in vobis est gregem Dei, providéntes non coácte, sed spontánee secúndum Deum, neque turpis lucri grátia, sed voluntárie; neque ut dominántes in cleris, sed forma facti gregis ex ánimo. Et, cum appáruerit princeps pastórum, percipiétis immarcescíbilem glóriæ corónam. Deus autem
omnis grátiæ, qui vocávit nos in ætérnam suam glóriam in Christo Jesu, módicum passos ipse perfíciet, confirmábit solidabítque. Ipsi glória et impérium in sǽcula sæculórum. Amen.
}\switchcolumn\portugues{
\rlettrine{A}{os} sacerdotes que estão entre vós rogo eu, sacerdote como eles e testemunha dos sofrimentos de Cristo e que tomarei parte com eles naquela glória que será manifestada um dia: apascentai o rebanho de Deus que vos está confiado, cuidai dele não constrangidos, mas de boa vontade, segundo Deus; não por amor de lucro vil, mas por dedicação; não como para dominar sobre a herança (do Senhor), mas fazendo-vos modelos do rebanho. E quando aparecer o príncipe dos pastores, recebereis a coroa imarcescível da glória. Foi Deus de toda a graça que nos chamou em Jesus Cristo à sua eterna glória; e, depois de terdes sofrido um pouco, vos aperfeiçoará, fortificará e consolidará. A Ele: glória e império pelos séculos dos séculos. Amen.
}\end{paracol}

\paragraphinfo{Gradual}{Sl. 106, 32, 31}
\begin{paracol}{2}\latim{
\rlettrine{E}{xáltent} eum in Ecclésia plebis: et in cáthedra seniórum laudent eum. ℣. Confiteántur Dómino misericórdiæ ejus; et mirabília ejus fíliis hóminum.
}\switchcolumn\portugues{
\qlettrine{Q}{ue} seja exaltado na assembleia do povo; que seja louvado no conselho dos anciãos. ℣. Glorifiquem o Senhor pelas suas misericórdias: e pelas suas maravilhas em favor dos filhos dos homens.
}\switchcolumn*\latim{
Allelúja, allelúja. ℣. \emph{Matth. 16, 18} Tu es Petrus, et super hanc petram ædificábo Ecclésiam meam. Allelúja.
}\switchcolumn\portugues{
Aleluia, aleluia. ℣. \emph{Mt. 16, 18} Tu és Pedro, e sobre esta pedra edificarei a minha Igreja. Aleluia.
}\end{paracol}

\textit{Depois da Septuagésima omite-se o Aleluia e o Verso e diz-se:}

\paragraphinfo{Trato}{Sl. 39, 10-11}
\begin{paracol}{2}\latim{
\rlettrine{A}{nnuntiávi} justítiam tuam in ecclésia magna, ecce, lábia mea non prohibébo: Dómine, tu scisti. ℣. Justítiam tuam non abscóndi in corde meo: veritátem tuam et salutáre tuum dixi. ℣. Non abscóndi misericórdiam tuam, et veritátem tuam a concílio multo.
}\switchcolumn\portugues{
\rlettrine{A}{nunciei} a vossa justiça numa grande assembleia: eis, pois, que não cerrareis os meus lábios, Senhor, bem o sabeis. ℣. Não encerrei a vossa justiça no meu coração; mas publiquei a vossa verdade e salvação. ℣. Não ocultei a vossa misericórdia e fidelidade diante da grande assembleia.
}\end{paracol}

\textit{No Tempo Pascal omite-se o Gradual e o Trato e diz-se:}

\begin{paracol}{2}\latim{
Allelúja, allelúja. ℣. \emph{Matth. 16, 18} Tu es Petrus, et super hanc petram ædificábo Ecclésiam meam. Allelúja. ℣. \emph{Ps. 44, 17, 18} Constítues eos príncipes super omnem terram: mémores erunt nóminis tui, Dómine. Allelúja.
}\switchcolumn\portugues{
Aleluia, aleluia. ℣. \emph{Mt. 16, 18} Tu és Pedro, e sobre esta pedra edificarei a minha Igreja. Aleluia. ℣. \emph{Sl. 44, 17, 18} Vós os constituístes príncipes em toda a terra: e eles perpetuarão, Senhor, o vosso nome. Aleluia.
}\end{paracol}

\paragraphinfo{Evangelho}{Mt. 16, 13-19}
\begin{paracol}{2}\latim{
\cruz Sequéntia sancti Evangélii secúndum Matthǽum.
}\switchcolumn\portugues{
\cruz Continuação do santo Evangelho segundo S. Mateus.
}\switchcolumn*\latim{
\blettrine{I}{n} illo témpore: Venit Jesus in partes Cæsaréæ Philíppi, et interrogábat discípulos suos, dicens: Quem dicunt hómines esse Fílium hóminis? At illi dixérunt: Alii Joánnem Baptístam, alii autem Elíam, alii vero Jeremíam aut unum ex prophétis. Dicit illis Jesus: Vos autem quem me esse dícitis? Respóndens Simon Petrus, dixit: Tu es Christus, Fílius Dei vivi. Respóndens autem Jesus, dixit ei: Beátus es, Simon Bar Jona: quia caro et sanguis non revelávit tibi, sed Pater meus, qui in cœlis est. Et ego dico tibi, quia tu es Petrus, et super hanc petram ædificábo Ecclésiam meam, et portæ ínferi non prævalébunt advérsus eam. Et tibi dabo claves regni cœlórum. Et quodcúmque ligáveris super terram, erit ligátum et in cœlis: et quodcúmque sólveris super terram, erit solútum et in cœlis.
}\switchcolumn\portugues{
\blettrine{N}{aquele} tempo, foi Jesus para a região de Cesareia, de Filipe, e interrogou os seus discípulos, dizendo-lhes: «Quem dizem os homens que é o Filho do homem?». Eles responderam: «Uns dizem que é João Baptista, outros que é Elias e outros que é Jeremias ou algum dos Profetas». Jesus disse-lhes: «E quem dizeis vós que eu sou?». Respondendo, Simão-Pedro disse: «Tu és Cristo, Filho de Deus vivo!». E Jesus disse-lhe: «Bem-aventurado és tu, Simão Barjona, porque não foi a carne nem o sangue que te revelaram o que dizes, mas meu Pai, que está nos céus. E Eu digo-te: tu és Pedro, e sobre esta pedra edificarei a minha Igreja; e as portas do inferno não prevalecerão contra ela. Eu te darei as chaves do reino dos céus; e tudo o que ligares sobre a terra será ligado também nos céus; e tudo o que desatares sobre a terra será desatado também nos céus».
}\end{paracol}

\paragraphinfo{Ofertório}{Jr. 1, 9-10}
\begin{paracol}{2}\latim{
\rlettrine{E}{cce,} dedi verba mea in ore tuo: ecce, constítui te super gentes et super regna, ut evéllas et destruas, et ædífices et plantes. (T. P. Allelúja.)
}\switchcolumn\portugues{
\rlettrine{E}{is} que pus as minhas palavras na tua boca: eis que te constituí sobre os povos e sobre os reinos para arrancares e destruíres, e para edificares e plantares. (T. P. Aleluia.)
}\end{paracol}

\paragraph{Secreta}
\begin{paracol}{2}\latim{
\rlettrine{O}{blátis} munéribus, quǽsumus, Dómine, Ecclésiam tuam benígnus illúmina: ut, et gregis tui profíciat ubique succéssus, et grati fiant nómini tuo, te gubernánte, pastóres. Per Dóminum nostrum Jesum Christum, Fílium tuum: Qui tecum vivit et regnat \emph{\&c.}
}\switchcolumn\portugues{
\rlettrine{C}{om} as ofertas destes dons, Vos suplicamos, Senhor, iluminai benignamente a vossa Igreja, a fim de que não só o vosso rebanho triunfe em toda a parte, mas também pelo poder do vosso nome os pastores sejam bem acolhidos. Por nosso Senhor \emph{\&c.}
}\end{paracol}

\textit{Se, porém, se fizer comemoração doutro Sumo Pontífice nesta mesma Missa, dir-se-á a seguinte Secreta, em vez da Precedente:}

\paragraph{Secreta}
\begin{paracol}{2}\latim{
\rlettrine{M}{únera,} quæ tibi, Dómine, lætántes offérimus, súscipe benígnus, et præsta: ut, intercedénte beáto {\redx N.}, Ecclésia tua et fídei integritáte lætétur, et témporum tranquillitáte semper exsúltet. Per Dóminum nostrum \emph{\&c.}
}\switchcolumn\portugues{
\rlettrine{R}{ecebei} benignamente, Senhor, os dons que com alegria Vos oferecemos, e fazei que, por intercessão do bem-aventurado {\redx N.}, a vossa. Igreja se alegre com a integridade da sua fé e sempre exulte com a tranquilidade dos tempos. Por nosso Senhor \emph{\&c.}
}\end{paracol}

\paragraphinfo{Comúnio}{Mt. 16, 18}
\begin{paracol}{2}\latim{
\rlettrine{T}{u} es Petrus, et super hanc petram ædificábo Ecclésiam meam. (T. P. Allelúja.)
}\switchcolumn\portugues{
\rlettrine{T}{u} és Pedro, e sobre esta pedra edificarei a minha Igreja. (T. P. Aleluia.)
}\end{paracol}

\paragraph{Postcomúnio}
\begin{paracol}{2}\latim{
\rlettrine{R}{efectióne} sancta enutrítam gubérna, quǽsumus, Dómine, tuam placátus Ecclésiam: ut, poténti moderatióne dirécta, et increménta libertátis accípiat et in religiónis integritáte persístat. Per Dóminum nostrum \emph{\&c.}
}\switchcolumn\portugues{
\rlettrine{S}{enhor,} Vos suplicamos, governai com mansidão a vossa Igreja, agora que foi alimentada com a sagrada refeição, a fim de que, dirigida com firme suavidade, alcance o incremento da sua liberdade e persista na integridade da sua doutrina. Por nosso Senhor \emph{\&c.}
}\end{paracol}

\textit{Se, porém, se fizer comemoração doutro Sumo Pontífice nesta mesma Missa, dir-se-á o seguinte Postcomúnio, em vez do Precedente:}

\paragraph{Postcomúnio}
\begin{paracol}{2}\latim{
\rlettrine{M}{ultíplica,} quǽsumus, Dómine, in Ecclesia tua spíritum grátiæ, quem dedísti: ut beáti {\redx N.} (Martyris tui atque) Summi Pontíficis deprecatióne nec pastóri obœdiéntia gregis nec gregi desit cura pastóris. Per Dóminum \emph{\&c.}
}\switchcolumn\portugues{
\rlettrine{S}{enhor,} Vos suplicamos, multiplicai na vossa Igreja o espírito da graça, que lhe concedestes, a fim de que, pela oração do bem-aventurado {\redx N.} (Vosso Mártir e) Sumo Pontífice, não falte ao pastor a obediência do rebanho, nem ao rebanho a dedicação do pastor. Por nosso Senhor \emph{\&c.}
}\end{paracol}

\subsectioninfo{Vigília dos Apóstolos}{Missa Ego autem}\label{1vigiliaapostolos}

\paragraphinfo{Intróito}{Sl. 51, 10 \& 11}
\begin{paracol}{2}\latim{
\rlettrine{E}{go} autem, sicut olíva fructífera in domo Dómini, sperávi in misericórdia Dei mei: et exspectábo nomen tuum, quóniam bonum est ante conspéctum sanctórum tuorum. \emph{Ps. ibid., 3} Quid gloriáris in malítia: qui potens es in iniquitáte?
℣. Gloria Patri \emph{\&c.}
}\switchcolumn\portugues{
\rlettrine{E}{u,} porém, sou como urna oliveira fértil na casa do Senhor: confio na misericórdia do meu Deus e esperarei o poder do seu nome, porque sois cheio de bondade diante de vossos santos. \emph{Sl. ibid., 3} Para que te vanglorias com a maldade, ó tu, que és poderoso em iniquidades?
℣. Glória ao Pai \emph{\&c.}
}\end{paracol}

\paragraph{Oração}
\begin{paracol}{2}\latim{
\rlettrine{D}{a,} quǽsumus, omnípotens Deus: ut beáti {\redx N.} Apóstoli tui, quam prævenímus, veneránda sollémnitas, et devotiónem nobis áugeat et salútem. Per Dóminum \emph{\&c.}
}\switchcolumn\portugues{
\rlettrine{C}{oncedei-nos,} ó Deus omnipotente, Vos suplicamos, que a solene festa do vosso B. Apóstolo {\redx N.}, cuja celebração antecipamos, nos aumente a piedade e o desejo da salvação. Por nosso Senhor \emph{\&c.}
}\end{paracol}

Se nesta Missa a Oração Precedente é recitada em honra de outro Santo, substitui-se pela seguinte:

\paragraph{Oração}
\begin{paracol}{2}\latim{
\qlettrine{Q}{uǽsumus,} omnípotens Deus: ut beátus Apóstolus, cujus prævenímus festivitátem, tuum pro nobis implóret auxílium; ut, a nostris reátibus absolúti, a cunctis étiam perículis eruámur. Per Dóminum \emph{\&c.}
}\switchcolumn\portugues{
\rlettrine{V}{os} suplicamos, ó Deus omnipotente, que o B. Apóstolo {\redx N.}, cuja festa antecipamos, implore o vosso socorro em nosso favor, a fim de que, absolvidos de nossas culpas, sejamos também livres de todos os perigos. Por nosso Senhor \emph{\&c.}
}\end{paracol}

\paragraphinfo{Epístola}{Ecl. 44, 25-27; 45, 2-4 \& 6-9}
\begin{paracol}{2}\latim{
Léctio libri Sapiéntiæ.
}\switchcolumn\portugues{
Lição do Livro da Sabedoria.
}\switchcolumn*\latim{
\rlettrine{B}{enedíctio} Dómini super caput justi. Ideo dedit illi Dóminus hereditátem, et divísit illi partem in tríbubus duódecim: et invénit grátiam in conspéctu omnis carnis. Et magnificávit eum in timóre inimicórum, et in verbis suis monstra placávit. Glorificávit illum in conspéctu regum, et jussit illi coram pópulo suo, et osténdit illi glóriam suam. In fide et lenitáte ipsíus sanctum fecit illum, et elégit eum ex omni carne. Et dedit illi coram præcépta, et legem vitæ et disciplínæ, et excélsum fecit illum. Státuit ei testaméntum ætérnum, et circumcínxit eum zona justítiæ: et índuit eum Dóminus corónam glóriæ.
}\switchcolumn\portugues{
\rlettrine{A}{} bênção do Senhor repousa sobre a cabeça do justo. Por isso o Senhor lhe deu a terra em herança e a dividiu entre as doze tribos. Ele achou graça aos olhos de todos os viventes. O Senhor engrandeceu-o e tornou-o admirável diante dos seus inimigos; e com suas palavras aplacou os monstros. O Senhor glorificou-o diante dos reis, deu-lhe os seus mandamentos diante do seu povo e manifestou-lhe a sua glória. Por causa da sua fé e mansidão, santificou-o e escolheu-o entre todos os mortais. Deu-lhe face a face os seus preceitos e a lei da vida e da sabedoria. Estabeleceu com ele uma aliança eterna; cingiu-o com a túnica da justiça; e ornou-o com a coroa da glória.
}\end{paracol}

\paragraphinfo{Gradual}{Sl. 91, 13 \& 14}
\begin{paracol}{2}\latim{
\qlettrine{J}{ustus} ut palma florébit: sicut cedrus Líbani multiplicábitur in domo Dómini. ℣. \emph{ibid., 3} Ad annuntiándum mane misericórdiam tuam, et veritátem tuam per noctem.
}\switchcolumn\portugues{
\rlettrine{O}{} justo florescerá, como a palmeira, e multiplicar-se-á, como o cedro do Líbano, plantado na casa do Senhor. ℣. \emph{ibid., 3} Para publicar de manhã a vossa misericórdia, Senhor, e durante a noite a vossa doutrina.
}\end{paracol}

\paragraphinfo{Evangelho}{Jo. 15. 12-16}
\begin{paracol}{2}\latim{
\cruz Sequéntia sancti Evangélii secúndum Joánnem.
}\switchcolumn\portugues{
\cruz Continuação do santo Evangelho segundo S. João.
}\switchcolumn*\latim{
\blettrine{I}{n} illo témpore: Dixit Jesus discípulis suis: Hoc est præcéptum meum, ut diligátis ínvicem, sicut diléxi vos. Majórem hac dilectiónem nemo habet, ut ánimam suam ponat quis pro amícis suis. Vos amíci mei estis, si fecéritis quæ ego præcípio vobis. Jam non dicam vos servos: quia servus nescit, quid fáciat dóminus ejus. Vos autem dixi amícos: quia ómnia, quæcúmque audivi a Patre meo, nota feci vobis. Non vos me elegístis: sed ego elégi vos, et posui vos, ut eátis, et fructum afferátis: et fructus vester maneat: ut, quodcúmque petiéritis Patrem in nómine meo, det vobis.
}\switchcolumn\portugues{
\blettrine{N}{aquele} tempo, disse Jesus aos discípulos: «Este é o meu mandamento: que vos ameis uns aos outros, como vos amei. Ninguém pode ter maior amor do que dar a sua vida pelos seus amigos. Vós sereis meus amigos, se fizerdes o que vos mando. Já vos não chamarei servos, porque o servo ignora o que faz o seu senhor. Chamo-vos meus amigos porque tudo quanto ouvi a meu Pai vo-lo tenho feito conhecer. Não fostes vós que me escolhestes a mim, mas eu é que vos escolhi e vos estabeleci, para que caminheis e alcanceis fruto. Que este fruto, pois, permaneça, para que meu Pai vos conceda tudo quanto lhe pedirdes em meu nome».
}\end{paracol}

\paragraphinfo{Ofertório}{Sl. 8, 6-7}
\begin{paracol}{2}\latim{
\rlettrine{G}{lória} et honore coronásti eum: et constituísti eum super ópera mánuum tuárum, Dómine.
}\switchcolumn\portugues{
\rlettrine{V}{ós} o coroastes, Senhor, com glória e com honras e lhe concedestes o domínio sobre as obras das vossas mãos.
}\end{paracol}

\paragraph{Secreta}
\begin{paracol}{2}\latim{
\rlettrine{A}{postólici} reveréntia cúlminis offeréntes tibi sacra mystéria, Dómine, quǽsumus: ut beáti {\redx N.} Apóstoli tui suffrágiis, cujus natalícia prævenímus; plebs tua semper et sua vota deprómat, et desideráta percípiat. Per Dóminum nostrum \emph{\&c.}
}\switchcolumn\portugues{
\rlettrine{O}{ferecendo-Vos} estes sagrados mystérios em reverência da dignidade apostólica, Vos suplicamos, Senhor, que, pelos rogos do B. {\redx N.}, vosso Apóstolo, cuja festa antecipamos, o vosso povo possa sempre apresentar-Vos seus votos e alcançar a realização de seus desejos. Por nosso Senhor \emph{\&c.}
}\end{paracol}

\paragraphinfo{Comúnio}{Sl. 20, 6}
\begin{paracol}{2}\latim{
\rlettrine{M}{agna} est glória ejus in salutári tuo: glóriam et magnum decórem impónes super eum, Dómine.
}\switchcolumn\portugues{
\rlettrine{G}{rande} é, Senhor, a sua glória, graças à vossa protecção. Vós o rodeastes de glória e de magnificência.
}\end{paracol}

\paragraph{Postcomúnio}
\begin{paracol}{2}\latim{
\rlettrine{S}{ancti} Apóstoli tui {\redx N.}, quǽsumus. Dómine, supplicatióne placátus: et veniam nobis tríbue, et remédia sempitérna concéde. Per Dóminum \emph{\&c.}
}\switchcolumn\portugues{
\rlettrine{V}{os} suplicamos, Senhor, deixai-Vos aplacar pelas orações do vosso santo Apóstolo {\redx N.}; concedei-nos ainda o perdão das nossas faltas e o remédio sempiterno dos nossos males. Por nosso Senhor \emph{\&c.}
}\end{paracol}


\subsectioninfo{Comum dos Mártires}{Fora do Tempo Pascal}

\subsectioninfo{Mártir Pontífice}{Missa Státuit ei Dóminus}\label{martirpontificeforapascal}

\paragraphinfo{Intróito}{Ecl. 45, 30}
\begin{paracol}{2}\latim{
\rlettrine{S}{tátuit} ei Dóminus testaméntum pacis, et príncipem fecit eum: ut sit illi sacerdótii dígnitas in ætérnum. \emph{Ps. 131, 1} Meménto, Dómine, David: et omnis mansuetúdinis ejus.
℣. Gloria Patri \emph{\&c.}
}\switchcolumn\portugues{
\rlettrine{O}{} Senhor fez com ele uma aliança de paz e proclamou-o príncipe, para que a dignidade sacerdotal lhe pertencesse eternamente. \emph{Sl. 131, 1} Lembrai-Vos de David, ó Senhor, e da sua grande solicitude.
℣. Glória ao Pai \emph{\&c.}
}\end{paracol}

\paragraph{Oração}
\begin{paracol}{2}\latim{
\rlettrine{I}{nfirmitátem} nostram réspice, omnípotens Deus: et, quia pondus própriæ actiónis gravat, beáti {\redx N.} Martyris tui atque Pontíficis intercéssio gloriósa nos prótegat. Per Dóminum \emph{\&c.}
}\switchcolumn\portugues{
\rlettrine{O}{lhai} para a nossa fraqueza, ó Deus omnipotente; e, visto que estamos oprimidos sob o peso dos nossos pecados, fazei que nos proteja a gloriosa intercessão do B. {\redx N.}, vosso Pontífice e Mártir. Por nosso Senhor \emph{\&c.}
}\end{paracol}

\paragraphinfo{Epístola}{Tg. 1, 12-18}
\begin{paracol}{2}\latim{
Léctio Epístolæ beáti Jacóbi Apóstoli.
}\switchcolumn\portugues{
Lição da Ep.ª do B. Ap.º Tiago.
}\switchcolumn*\latim{
\rlettrine{C}{aríssimi:} Beátus vir, qui suffert tentatiónem: quóniam, cum probátus fúerit, accípiet corónam vitæ, quam repromísit Deus diligéntibus se. Nemo, cum tentátur, dicat, quóniam a Deo tentátur: Deus enim intentátor malórum est: ipse autem néminem tentat. Unusquísque vero tentátur a concupiscéntia sua abstráctus et illéctus. Deinde Concupiscéntia cum concéperit, parit peccátum: peccátum vero cum consummátum fúerit, génerat mortem. Nolíte itaque erráre, fratres mei dilectíssimi. Omne datum óptimum et omne donum perféctum desúrsum est, descéndens a Patre lúminum, apud quem non est transmutátio nec vicissitúdinis obumbrátio. Voluntárie enim génuit nos verbo veritátis, ut simus inítium aliquod creatúræ ejus.
}\switchcolumn\portugues{
\rlettrine{C}{aríssimos:} bem-aventurado o varão que sofre a tentação, porque, quando acabar a provação, receberá a coroa da vida, que o Senhor prometeu aos que O amam. Ninguém, quando for tentado, diga que é Deus quem o tenta, pois Deus não é tentador que arraste para o mal, nem tenta ninguém. Porém, cada um é tentado pela sua própria concupiscência, que o atrai e solicita; e depois, quando a concupiscência já concebeu, gera o pecado, e o pecado, logo que é consumado, gera a morte. Não vos enganeis, pois, irmãos dilectíssimos. Todo o dom excelente e todo o dom perfeito vêm do alto e derivam do Pai das luzes, em quem não há mudanças, nem sombra de alteração; pois foi Ele quem por sua espontânea vontade nos gerou pela palavra da verdade, a fim de que fôssemos como primícias das suas criaturas.
}\end{paracol}

\paragraphinfo{Gradual}{Sl. 88, 21-23}
\begin{paracol}{2}\latim{
\rlettrine{I}{nvéni} David servum meum, óleo sancto meo unxi eum: manus enim mea auxiliábitur ei, et bráchium meum confortábit eum. ℣. Nihil profíciet inimícus in eo, et fílius iniquitátis non nocébit ei.
}\switchcolumn\portugues{
\rlettrine{E}{ncontrei} o meu servo David e ungi-o com meu óleo sagrado; a minha mão o auxiliará e o meu braço o fortificará. ℣. O inimigo nada poderá contra ele e o filho da iniquidade nenhum mal lhe fará.
}\switchcolumn*\latim{
Allelúja, allelúja. ℣. \emph{Ps. 109, 4} Tu es sacérdos in ætérnum, secúndum órdinem Melchísedech. Allelúja.
}\switchcolumn\portugues{
Aleluia, aleluia. ℣. \emph{Sl. 109, 4} Tu és sacerdote para sempre, segundo a ordem de Melquisedeque. Aleluia.
}\end{paracol}

\textit{Após a Septuagésima omite-se o Aleluia e o seguinte, dizendo-se:}

\paragraphinfo{Trato}{Sl. 20, 3-4}
\begin{paracol}{2}\latim{
\rlettrine{D}{esidérium} ánimæ ejus tribuísti ei: et voluntáte labiórum ejus non fraudásti eum. ℣. Quóniam prævenísti eum in benedictiónibus dulcédinis. ℣. Posuísti in cápite ejus corónam de lápide pretióso.
}\switchcolumn\portugues{
\rlettrine{C}{oncedestes-lhe} o desejo da sua alma: lhe não negastes o que seus lábios Vos pediram. ℣. Premuniste-lo com bênçãos de doçura. ℣. Impusestes na sua cabeça uma coroa de pedras preciosas.
}\end{paracol}

\paragraphinfo{Evangelho}{Lc. 14, 26-33}
\begin{paracol}{2}\latim{
\cruz Sequéntia sancti Evangélii secúndum Lucam.
}\switchcolumn\portugues{
\cruz Continuação do santo Evangelho segundo S. Lucas.
}\switchcolumn*\latim{
\blettrine{I}{n} illo témpore: Dixit Jesus turbis: Si quis venit ad me, et non odit patrem suum, et matrem, et uxórem, et fílios, et fratres, et soróres, adhuc autem et ánimam suam, non potest meus esse discípulus. Et qui non bájulat crucem suam, et venit post me, non potest meus esse discípulus. Quis enim ex vobis volens turrim ædificáre, non prius sedens cómputat sumptus, qui necessárii sunt, si hábeat ad perficiéndum; ne, posteáquam posúerit fundaméntum, et non potúerit perfícere, omnes, qui vident, incípiant illúdere ei, dicéntes: Quia hic homo cœpit ædificáre, et non pótuit consummáre? Aut quis rex iturus commíttere bellum advérsus álium regem, non sedens prius cógitat, si possit cum decem mílibus occúrrere ei, qui cum vigínti mílibus venit ad se? Alióquin, adhuc illo longe agénte, legatiónem mittens, rogat ea, quæ pacis sunt. Sic ergo omnis ex vobis, qui non renúntiat ómnibus, quæ póssidet, non potest meus esse discípulus.
}\switchcolumn\portugues{
\blettrine{N}{aquele} tempo, disse Jesus às turbas: «Se alguém vem a mim e não despreza seu pai, sua mãe, sua mulher e filhos, seus irmãos e irmãs e até mesmo a sua própria vida, não pode ser meu discípulo. E todo aquele que não leva a sua cruz não pode ser meu discípulo. Com efeito, qual é de vós que, querendo edificar uma torre, não calcula primeiramente com cuidado os gastos necessários, para ver se possui meios para a acabar? Pois poderá acontecer que, depois de haver lançado os alicerces e não podendo acabar a torre, comecem a zombar dele aqueles que o vêem, dizendo: «Este homem começou a edificar e não pôde acabar!». Ou qual é o rei que, preparando-se para pelejar com outro rei, não considera primeiramente se com um exército de dez mil homens poderá fazer frente ao inimigo, que avança contra ele com vinte mil homens? Se vê que não pode combater, estando ainda o outro longe, manda-lhe uma embaixada a pedir-lhe a paz. Assim, pois, todo aquele de vós que não renunciar a tudo quanto possui não pode ser meu discípulo».
}\end{paracol}

\paragraphinfo{Ofertório}{Sl. 88, 25}
\begin{paracol}{2}\latim{
\rlettrine{V}{éritas} mea et misericórdia mea cum
ipso: et in nómine meo exaltábitur cornu ejus.
}\switchcolumn\portugues{
\rlettrine{A}{} minha fidelidade e a minha misericórdia estarão com ele: e o seu poder elevar-se-á pelo meu nome.
}\end{paracol}

\paragraph{Secreta}
\begin{paracol}{2}\latim{
\rlettrine{H}{óstias} tibi, Dómine, beáti {\redx N.} Mártyris tui atque Pontíficis dicátas méritis, benígnus assúme: et ad perpétuum nobis tríbue proveníre subsídium. Per Dóminum \emph{\&c.}
}\switchcolumn\portugues{
\rlettrine{R}{ecebei} benigno, Senhor, as hóstias que Vos oferecemos pelos merecimentos do B. {\redx N.}, vosso Mártir e Pontífice, e fazei que elas nos alcancem o vosso perpétuo socorro. Por nosso Senhor \emph{\&c.}
}\end{paracol}

\paragraphinfo{Comúnio}{Sl. 88, 36 \& 37-38}
\begin{paracol}{2}\latim{
\rlettrine{S}{emel} jurávi in sancto meo: Semen ejus in ætérnum manébit: et sedes ejus sicut sol in conspéctu meo, et sicut luna perfécta in ætérnum, et testis in cœlo fidélis.
}\switchcolumn\portugues{
\qlettrine{J}{urei} uma vez por minha santidade: sua descendência durará eternamente e o seu trono brilhará perante mim, como o sol, e como a lua permanecerá para sempre e será testemunho fiel no céu.
}\end{paracol}

\paragraph{Postcomúnio}
\begin{paracol}{2}\latim{
\rlettrine{R}{efécti} participatióne múneris sacri, quǽsumus, Dómine, Deus noster: ut, cujus exséquimur cultum, intercedénte beáto {\redx N.} Mártyre tuo atque Pontífice, sentiámus efféctum. Per Dóminum \emph{\&c.}
}\switchcolumn\portugues{
\rlettrine{F}{ortalecidos} com a participação do dom sacratíssimo, Vos pedimos, Senhor, nosso Deus, que, por intercessão do B. {\redx N.}, vosso Mártir e Pontífice, sintamos o efeito do mistério que hoje celebrámos. Por nosso Senhor \emph{\&c.}
}\end{paracol}

\subsectioninfo{Mártir Pontífice}{Missa Sacerdótes Dei}\label{martirpontifice}

\paragraphinfo{Intróito}{Dn. 3, 84 \& 87}
\begin{paracol}{2}\latim{
\rlettrine{S}{acerdótes} Dei, benedícite Dóminum: sancti et húmiles corde, laudáte Deum. \emph{ibid., 57} Benedícite, ómnia ópera Dómini, Dómino: laudáte et superexaltáte eum in sǽcula.
℣. Gloria Patri \emph{\&c.}
}\switchcolumn\portugues{
\rlettrine{B}{endizei} o Senhor, ó sacerdotes de Deus: louvai o Senhor, ó vós, santos e humildes de coração! \emph{ibid., 57} Bendizei o Senhor, todas as obras do Senhor: louvai-O e glorificai-O em todos os séculos!
℣. Glória ao Pai \emph{\&c.}
}\end{paracol}

\paragraph{Oração}
\begin{paracol}{2}\latim{
\rlettrine{D}{eus,} qui nos beáti {\redx N.} Mártyris tui atque Pontíficis ánnua sollemnitáte lætíficas: concéde propítius; ut, cujus natalítia cólimus, de ejúsdem étiam protectióne gaudeámus. Per Dóminum \emph{\&c.}
}\switchcolumn\portugues{
\slettrine{Ó}{} Deus, que nos alegrais com a solenidade anual do B. {\redx N.}, vosso Mártir e Pontífice, concedei-nos propício que, assim como celebramos o seu nascimento, assim também gozemos a sua protecção. Por nosso Senhor \emph{\&c.}
}\end{paracol}

\paragraphinfo{Epístola}{2. Cor. 1, 3-7}
\begin{paracol}{2}\latim{
Léctio Epístolæ beáti Pauli Apóstoli ad Corínthios.
}\switchcolumn\portugues{
Lição da Ep.ª do B. Ap.º Paulo aos Coríntios.
}\switchcolumn*\latim{
\rlettrine{F}{ratres:} Benedíctus Deus et Pater Dómini nostri Jesu Christi, Pater misericordiárum, et Deus totíus consolatiónis, qui consolátur nos in omni tribulatióne nostra: ut póssimus et ipsi consolári eos, qui in omni pressúra sunt, per exhortatiónem, qua exhortámur et ipsi a Deo. Quóniam sicut abúndant passiónes Christi in nobis: ita et per Christum abúndat consolátio nostra. Sive autem tribulámur pro vestra exhortatióne et salúte, sive consolámur pro vestra consolatióne, sive exhortámur pro vestra exhortatióne et salúte, quæ operátur tolerántiam earúndem passiónum, quas et nos pátimur: ut spes nostra firma sit pro vobis: sciéntes, quod, sicut sócii passiónum estis, sic éritis et consolatiónis: in Christo Jesu, Dómino nostro.
}\switchcolumn\portugues{
\rlettrine{M}{eus} irmãos: bendito seja Deus, Pai de N. S. Jesus Cristo e das misericórdias e Deus de toda a consolação, que nos consola em todas nossas tribulações, para que pela mesma consolação, que recebemos de Deus, possamos consolar os que estão oprimidos. Pois, assim como abundam em nós as aflições de Cristo, assim também em Cristo abundem as consolações. Se, portanto, somos atribulados, é para vossa consolação e salvação; se somos consolados, é também para vossa consolação; e se somos confortados, é ainda para vosso conforto e salvação, a qual mostra a sua eficácia em suportar os mesmos males que nos afligem. A nossa confiança a vosso respeito é firme, sabendo que, assim como participais das aflições, assim também participareis da consolação em Jesus Cristo, nosso Senhor.
}\end{paracol}

\paragraphinfo{Gradual}{Sl. 8, 6-7}
\begin{paracol}{2}\latim{
\rlettrine{G}{lória} et honóre coronásti eum. ℣. Et constituísti eum super ópera mánuum tuárum, Dómine.
}\switchcolumn\portugues{
\rlettrine{V}{ós} o coroastes, Senhor, com glória e com honras. ℣. Vós lhe destes o domínio sobre as obras das vossas mãos.
}\switchcolumn*\latim{
 Allelúja, allelúja. ℣. Hic est Sacérdos, quem coronávit Dóminus. Allelúja.
}\switchcolumn\portugues{
Aleluia, aleluia. ℣. Eis o sacerdote que o Senhor coroou. Aleluia.
}\end{paracol}

\textit{Após a Septuagésima omite-se o Aleluia e o seguinte e diz-se:}

\paragraphinfo{Trato}{Sl. 111, 1-3}
\begin{paracol}{2}\latim{
\rlettrine{B}{eátus} vir, qui timet Dóminum: in mandátis ejus cupit nimis. ℣. Potens in terra erit semen ejus: generátio rectórum benedicétur. ℣. Glória et divítiæ in domo ejus: et justítia ejus manet in sǽculum sǽculi.
}\switchcolumn\portugues{
\rlettrine{B}{em-aventurado} o varão que teme o Senhor e que põe todo seu zelo em obedecer-Lhe. ℣. Sua descendência será poderosa na terra; pois a geração dos justos será abençoada. ℣. Na sua casa haverá glória e riqueza: e a sua justiça permanecerá em todos os séculos.
}\end{paracol}

\paragraphinfo{Evangelho}{Mt. 16, 24-27}
\begin{paracol}{2}\latim{
\cruz Sequéntia sancti Evangélii secúndum Matthǽum.
}\switchcolumn\portugues{
\cruz Continuação do santo Evangelho segundo S. Mateus.
}\switchcolumn*\latim{
\blettrine{I}{n} illo témpore: Dixit Jesus discípulis suis: Si quis vult post me veníre, ábneget semetípsum, et tollat crucem suam, et sequátur me. Qui enim voluerit ánimam suam salvam fácere, perdet eam: qui autem perdíderit ánimam suam propter me, invéniet eam. Quid enim prodest hómini, si mundum univérsum lucrétur, ánimæ vero suæ detriméntum patiátur? Aut quam dabit homo commutatiónem pro ánima sua? Fílius enim hóminis ventúrus est in glória Patris sui cum Angelis suis: et tunc reddet unicuíque secúndum ópera ejus.
}\switchcolumn\portugues{
\blettrine{N}{aquele} tempo, disse Jesus aos discípulos: «Se alguém quer vir após mim, negue-se a si próprio, tome a sua cruz e siga-me. Porque aquele que quiser salvar a sua vida perdê-la-á; e aquele que tiver perdido a sua vida por mim encontrá-la-á. De que serve ao homem ganhar todo o mundo, se isto vier em prejuízo da sua alma? Ou o que dará um homem em troca de sua alma? Porque o Filho do homem há-de vir na glória de seu Pai com os Anjos, e então dará a cada um segundo as suas obras».
}\end{paracol}

\paragraphinfo{Ofertório}{Sl. 88, 21-22}
\begin{paracol}{2}\latim{
\rlettrine{I}{nvéni} David servum meum, oleo sancto meo unxi eum: manus enim mea auxiliábitur ei, et bráchium meum confortábit eum.
}\switchcolumn\portugues{
\rlettrine{E}{ncontrei} o meu servo David: e ungi-o com meu óleo sagrado: a minha mão o socorrerá e o meu braço o fortalecerá.
}\end{paracol}

\paragraph{Secreta}
\begin{paracol}{2}\latim{
\rlettrine{M}{únera} tibi, Dómine, dicáta sanctífica: et, intercedénte beáto {\redx N.} Mártyre tuo atque Pontífice, per éadem nos placátus inténde. Per Dóminum \emph{\&c.}
}\switchcolumn\portugues{
\rlettrine{S}{antificai,} Senhor, estes dons que Vos são oferecidos, e, por intercessão do B. {\redx N.} vosso Mártir e Pontífice, olhai aplacado para nós. Por nosso Senhor \emph{\&c.}
}\end{paracol}

\paragraphinfo{Comúnio}{Sl. 20, 4}
\begin{paracol}{2}\latim{
\rlettrine{P}{osuísti,} Dómine, in cápite ejus corónam de lápide pretióso.
}\switchcolumn\portugues{
\rlettrine{S}{enhor,} impusestes na sua cabeça uma coroa de pedras preciosas.
}\end{paracol}

\paragraph{Postcomúnio}
\begin{paracol}{2}\latim{
\rlettrine{H}{æc} nos commúnio, Dómine, purget a crímine: et, intercedénte beáto {\redx N.} Mártyre tuo atque Pontífice, cœléstis remédii fáciat esse consórtes. Per Dóminum nostrum \emph{\&c.}
}\switchcolumn\portugues{
\qlettrine{Q}{ue} esta comunhão, Senhor, nos purifique de todos nossos crimes, e, por intercessão do B. {\redx N.}, vosso Mártir e Pontífice, nos torne participante do remédio celestial. Por nosso Senhor \emph{\&c.}
}\end{paracol}

\subsectioninfo{Mártir não Pontífice}{Missa In virtúte tua}\label{martirnaopontifice1}

\paragraphinfo{Intróito}{Sl. 20, 2-3}
\begin{paracol}{2}\latim{
\rlettrine{I}{n} virtúte tua, Dómine, lætábitur justus: et super salutáre tuum exsultábit veheménter: desidérium ánimæ ejus tribuísti ei. \emph{Ps. ibid., 4} Quóniam prævenísti eum in benedictiónibus dulcédinis: posuísti in cápite ejus corónam de lápide pretióso.
℣. Gloria Patri \emph{\&c.}
}\switchcolumn\portugues{
\rlettrine{O}{} justo rejubilará com vosso poder, Senhor, e exultará de alegria, vendo-se salvo por Vós; pois concedestes-lhe o que seu coração desejava. \emph{Sl. ibid., 4} Com efeito, Vós o premunistes com bênçãos de doçura: e impusestes na sua cabeça uma coroa de pedras preciosas.
℣. Glória ao Pai \emph{\&c.}
}\end{paracol}

\paragraph{Oração}
\begin{paracol}{2}\latim{
\rlettrine{P}{ræsta,} quǽsumus, omnípotens Deus: ut, qui beáti {\redx N.} Mártyris tui natalícia cólimus, intercessióne ejus, in tui nóminis amóre roborémur. Per Dóminum \emph{\&c.}
}\switchcolumn\portugues{
\slettrine{Ó}{} Deus omnipotente, permiti que, celebrando nós o nascimento do B. {\redx N.}, vosso Mártir, e pela sua intercessão, alcancemos a graça de sermos confirmados no amor ao vosso Nome, Por nosso Senhor \emph{\&c.}
}\end{paracol}

\paragraphinfo{Epístola}{Sb. 10, 10-14}
\begin{paracol}{2}\latim{
Léctio libri Sapiéntiæ.
}\switchcolumn\portugues{
Lição do Livro da Sabedoria.
}\switchcolumn*\latim{
\qlettrine{J}{ustum} dedúxit Dóminus per vias rectas, et ostendit illi regnum Dei, et dedit illi sciéntiam sanctórum: honestávit illum in labóribus, et complévit labores illíus. In fraude circumveniéntium illum áffuit illi, et honéstum fecit illum. Custodívit illum ab inimícis, et a seductóribus tutávit illum, et certámen forte dedit illi, ut vínceret et sciret, quóniam ómnium poténtior est sapiéntia. Hæc vénditum jusíum non derelíquit, sed a peccatóribus liberávit eum: descendítque cum illo in fóveam, et in vínculis non derelíquit illum, donec afférret illi sceptrum regni, et poténtiam advérsus eos, qui eum deprimébant: et mendáces osténdit, qui maculavérunt illum, et dedit illi claritátem ætérnam, Dóminus, Deus noster.
}\switchcolumn\portugues{
\rlettrine{O}{} Senhor conduziu o justo por caminhos direitos; mostrou-lhe o reino de Deus; transmitiu-lhe a ciência das coisas santas; enriqueceu-o nos seus trabalhos; e fez frutificar esses seus labores. O Senhor auxiliou-o contra os que queriam enganá-lo com suas fraudes e fê-lo adquirir riquezas. Protegeu-o contra os seus inimigos; defendeu-o de seus sedutores; e alcançou a vitória em um rude combate em seu favor, para lhe ensinar que a sabedoria é a mais poderosa de todas as coisas. O Senhor não abandonou o justo quando este foi vendido, mas até o preservou das mãos dos pecadores; desceu com ele á prisão; e o não abandonou nas cadeias, enquanto lhe não entregou o ceptro do império e o poder sobre os seus opressores. O Senhor, nosso Deus, provou que eram mentirosos aqueles que o desacreditaram e tornou-o ilustre para sempre.
}\end{paracol}

\paragraphinfo{Gradual}{Sl. 111, 1-2}
\begin{paracol}{2}\latim{
\rlettrine{B}{eátus} vir, qui timet Dóminum: in mandátis ejus cupit nimis. ℣. Potens in terra erit semen ejus: generátio rectórum benedicétur.
}\switchcolumn\portugues{
\rlettrine{B}{em-aventurado} o varão que teme o Senhor e que põe todo seu zelo em obedecer-Lhe. ℣. Sua descendência será poderosa na terra; pois a geração dos justos será abençoada.
}\switchcolumn*\latim{
 Allelúja, allelúja. ℣. \emph{Ps. 20, 4} Posuísti, Dómine, super caput ejus corónam de lápide pretióso. Allelúja.
}\switchcolumn\portugues{
Aleluia, aleluia. ℣. \emph{Sl. 20, 4} Senhor, impusestes na sua cabeça uma coroa de pedras preciosas. Aleluia.
}\end{paracol}

\textit{Após a Septuagésima omite-se o Aleluia e o seguinte e diz-se:}

\paragraphinfo{Trato}{Sl. 20, 3-4}
\begin{paracol}{2}\latim{
\rlettrine{D}{esidérium} ánimæ ejus tribuísti ei: et voluntáte labiórum ejus non fraudásti eum. ℣. Quóniam prævenísti eum in benedictiónibus dulcédinis. ℣. Posuísti in cápite ejus corónam de lápide pretióso.
}\switchcolumn\portugues{
\rlettrine{C}{oncedestes-lhe} o desejo da sua alma; lhe não negastes o que seus lábios Vos pediram. ℣. Premuniste-lo com bênçãos de doçura. ℣. Impusestes na sua cabeça uma coroa de pedras preciosas.
}\end{paracol}

\paragraphinfo{Evangelho}{Mt. 10, 34-42}
\begin{paracol}{2}\latim{
\cruz Sequéntia sancti Evangélii secúndum Matthǽum.
}\switchcolumn\portugues{
\cruz Continuação do santo Evangelho segundo S. Mateus.
}\switchcolumn*\latim{
\blettrine{I}{n} illo témpore: Dixit Jesus discípulis suis: Nolíte arbitrári, quia pacem vénerim míttere in terram: non veni pacem míttere, sed gládium. Veni enim separáre hóminem advérsus patrem suum, et fíliam advérsus matrem suam, et nurum advérsus socrum suam: et inimíci hóminis doméstici ejus. Qui amat patrem aut matrem plus quam me, non est me dignus: et qui amat fílium aut fíliam super me, non est me dignus. Et qui non áccipit crucem suam, et séquitur me, non est me dignus. Qui invénit ánimam suam, perdet illam: et qui perdíderit ánimam suam propter me, invéniet eam. Qui récipit vos, me récipit: et qui me récipit, récipit eum, qui me misit. Qui récipit prophétam in nómine prophétæ, mercédem prophétæ accípiet: et qui récipit justum in nómine justi, mercédem justi accípiet. Et quicúmque potum déderit uni ex mínimis istis cálicem aquæ frígidæ tantum in nómine discípuli: amen, dico vobis, non perdet mercédem suam.
}\switchcolumn\portugues{
\blettrine{N}{aquele} tempo, disse Jesus aos discípulos: «Não penseis que vim trazer a paz à terra; não vim trazer a paz, mas o gládio; pois vim separar o homem de seu pai; a filha de sua mãe; e a nora de sua sogra. O homem terá como inimigos os seus próprios criados. Aquele que ama seu pai ou sua mãe mais do que a mim não é digno de mim; e aquele que ama seu filho ou filha mais do que a mim não é digno de mim. Quem não toma a sua cruz e me não segue não é digno de mim. Aquele que conserva a sua vida perdê-la-á; e aquele que por amor de mim a perder achá-la-á. Aquele que vos recebe recebe-me a mim; e o que me recebe, recebe Aquele que me enviou. Aquele que recebe um profeta na qualidade de profeta receberá a recompensa de profeta; e aquele que recebe um justo na qualidade de justo receberá a recompensa de justo. Todo aquele que der de beber, mesmo que seja um copo de água fria, a um destes pequenos, como sendo meu discípulo, eu vos digo, na verdade, que não perderá a recompensa.
}\end{paracol}

\paragraphinfo{Ofertório}{Sl. 8, 6-7}
\begin{paracol}{2}\latim{
\rlettrine{G}{lória} et honóre coronásti eum: et constituísti eum super ópera mánuum tuárum, Dómine.
}\switchcolumn\portugues{
\rlettrine{V}{ós} o coroastes, Senhor, com glória e com honras; Vós lhe concedestes o domínio sobre as obras das vossas mãos.
}\end{paracol}

\paragraph{Secreta}
\begin{paracol}{2}\latim{
\rlettrine{M}{unéribus} nostris, quǽsumus, Dómine, precibúsque suscéptis: et cœléstibus nos munda mystériis, et cleménter exáudi. Per Dóminum nostrum \emph{\&c.}
}\switchcolumn\portugues{
\rlettrine{H}{avendo} Vós, Senhor, aceitado as nossas ofertas e orações, dignai-Vos purificar-nos com vossos celestiais mistérios e ouvir-nos benignamente. Por nosso Senhor \emph{\&c.}
}\end{paracol}

\paragraphinfo{Comúnio}{Mt. 16, 24}
\begin{paracol}{2}\latim{
\qlettrine{Q}{ui} vult veníre post me, ábneget semetípsum, et tollat crucem suam, et sequátur me.
}\switchcolumn\portugues{
\rlettrine{S}{e} alguém quer vir após mim, negue-se a si mesmo, tome a sua cruz e siga-me!
}\end{paracol}

\paragraph{Postcomúnio}
\begin{paracol}{2}\latim{
\rlettrine{D}{a,} quǽsumus, Dómine, Deus noster: ut, sicut tuórum commemoratióne Sanctórum temporáli gratulámur offício; ita perpétuo lætámur aspéctu. Per Dóminum nostrum \emph{\&c.}
}\switchcolumn\portugues{
\rlettrine{P}{ermiti,} ó Senhor, nosso Deus, Vos suplicamos, que, assim como nos alegramos, comemorando nesta vida pelo nosso ministério a memória dos vossos Santos, assim também tenhamos a felicidade de os contemplar na eternidade. Por nosso Senhor \emph{\&c.}
}\end{paracol}

\subsectioninfo{Mártir não Pontífice}{Missa Lætábitur justus}\label{martirnaopontifice2}

\paragraphinfo{Intróito}{Sl. 63, 11}
\begin{paracol}{2}\latim{
\rlettrine{L}{ætábitur} justus in Dómino, et sperábit in eo: et laudabúntur omnes recti corde. \emph{Ps. ibid., 2} Exáudi, Deus, oratiónem meam, cum déprecor: a timóre inimíci éripe ánimam meam.
℣. Gloria Patri \emph{\&c.}
}\switchcolumn\portugues{
\rlettrine{O}{} justo alegrar-se-á no Senhor e porá n’Ele a sua esperança. Todos aqueles que possuem o coração recto serão glorificados. \emph{Ps. ibid., 2} Ouvi, Senhor, a oração com que Vos imploro: livrai a minha alma do temor do inimigo.
℣. Glória ao Pai \emph{\&c.}
}\end{paracol}

\paragraph{Oração}
\begin{paracol}{2}\latim{
\rlettrine{P}{ræsta,} quǽsumus, omnípotens Deus: ut, intercedénte beáto {\redx N.} Mártyre tuo, et a cunctis adversitátibus liberémur in córpore, et a pravis cogitatiónibus mundémur in mente. Per Dóminum nostrum \emph{\&c.}
}\switchcolumn\portugues{
\rlettrine{C}{oncedei-nos,} ó Deus omnipotente, Vos suplicamos, que, por intercessão do B. {\redx N.} vosso Mártir, o nosso corpo seja preservado de todas as adversidades e a nossa alma purificada dos maus pensamentos. Por nosso Senhor \emph{\&c.}
}\end{paracol}

\paragraphinfo{Epístola}{2. Tm. 2, 8-10; 3, 10-12}
\begin{paracol}{2}\latim{
Léctio Epístolæ beáti Pauli Apóstoli ad Timotheum.
}\switchcolumn\portugues{
Lição da Ep.ª do B. Ap.º Paulo a Timóteo.
}\switchcolumn*\latim{
\rlettrine{C}{aríssime:} Memor esto, Dóminum Jesum Christum resurrexísse a mórtuis ex sémine David, secúndum Evangélium meum, in quo labóro usque ad víncula, quasi male óperans: sed verbum Dei non est alligátum. Ideo ómnia sustíneo propter eléctos, ut et ipsi salútem consequántur, quæ est in Christo Jesu, cum glória cœlésti. Tu autem assecútus es meam doctrínam, institutiónem, propósitum, fidem, longanimitátem, dilectiónem, patiéntiam, persecutiónes, passiónes: quália mihi facta sunt Antiochíæ, Icónii et Lystris: quales perseditiónes sustínui, et ex ómnibus erípuit me Dóminus. Et omnes, qui pie volunt vívere in Christo Jesu, persecutiónem patiéntur.
}\switchcolumn\portugues{
\rlettrine{C}{aríssimo:} lembrai-vos de que N. S. Jesus Cristo, descendente da raça de David, ressuscitou dos mortos, segundo o Evangelho que prego, pelo qual tenho sofrido a provação até estar preso com cadeias, como se fora um malfeitor. Mas a palavra de Deus não pode estar presa. Por isso sofro tudo por amor dos escolhidos, a fim de que eles consigam também a salvação, que está na glória celestial com Jesus Cristo. Quanto a vós, lembrai-vos de que estais unidos e compreendestes a minha doutrina, modo de vida, regra de conduta, fé, longanimidade, caridade e paciência, e conheceis as grandes perseguições e vexações que me foram feitas e sofri em Antioquia, Icónia e Lístria, das quais o Senhor me livrou. Do mesmo modo, todos os que querem viver piamente em Jesus Cristo padecerão perseguição.
}\end{paracol}

\paragraphinfo{Gradual}{Sl. 36, 24}
\begin{paracol}{2}\latim{
\qlettrine{J}{ustus} cum cecíderit, non collidétur: quia Dóminus suppónit manum suam. ℣. \emph{ibid., 26} Tota die miserétur, et cómmodat: et semen ejus in benedictióne erit.
}\switchcolumn\portugues{
\qlettrine{Q}{uando} o justo cair, não se magoará, porque o Senhor o amparará com sua mão. ℣. \emph{ibid., 26} Em cada dia ele se emprega em obras de misericórdia, e empresta: e a sua geração será abençoada.
}\switchcolumn*\latim{
Allelúja, allelúja. ℣. \emph{Joann. 8, 12} Qui séquitur me, non ámbulat in ténebris: sed habébit lumen vitæ ætérnæ. Allelúja.
}\switchcolumn\portugues{
Aleluia, aleluia. ℣. \emph{Jo. 8, 12} Aquele que me segue não caminha nas trevas, mas terá a luz da vida eterna. Aleluia.
}\end{paracol}

\textit{Após a Septuagésima omite-se o Aleluia e o seguinte e diz-se:}

\paragraphinfo{Trato}{Sl. 111, 1-3}
\begin{paracol}{2}\latim{
\rlettrine{B}{eátus} vir, qui timet Dóminum: in mandátis ejus cupit nimis. ℣. Potens in terra erit semen ejus: generátio rectórum benedicétur. ℣. Glória et divítiæ in domo ejus: et justítia ejus manet in sǽculum sǽculi.
}\switchcolumn\portugues{
\rlettrine{B}{em-aventurado} o varão que teme o Senhor e que põe todo seu zelo em obedecer-lhe. ℣. Sua descendência será poderosa na terra, pois a geração dos justos será abençoada. ℣. Haverá glória e riqueza em sua casa e a sua justiça subsistirá para sempre.
}\end{paracol}

\paragraphinfo{Evangelho}{Mt. 10, 26-32}
\begin{paracol}{2}\latim{
\cruz Sequéntia sancti Evangélii secúndum Matthǽum.
}\switchcolumn\portugues{
\cruz Continuação do santo Evangelho segundo S. Mateus.
}\switchcolumn*\latim{
\blettrine{I}{n} illo témpore: Dixit Jesus discípulis suis: Nihil est opértum, quod non revelábitur; et occúltum, quod non sciétur. Quod dico vobis in ténebris, dícite in lúmine: et quod in aure audítis, prædicáte super tecta. Et nolíte timére eos, qui occídunt corpus, ánimam autem non possunt occídere; sed pótius timéte eum, qui potest et ánimam et corpus pérdere in gehénnam. Nonne duo pásseres asse véneunt: et unus ex illis non cadet super terram sine Patre vestro? Vestri autem capílli cápitis omnes numerári sunt. Nolíte ergo timére: multis passéribus melióres estis vos. Omnis ergo, qui confitébitur me coram homínibus, confitébor et ego eum coram Patre meo, qui in cœlis est.
}\switchcolumn\portugues{
\blettrine{N}{aquele} tempo, disse Jesus aos discípulos: «Nada há oculto que não haja e ser descoberto, nem segredo que não venha a ser revelado. O que dizeis nas trevas dizei-o às claras; o que dizeis ao ouvido publicai-o em cima dos telhados. Não tenhais medo daqueles que matam o corpo e não podem matar a alma; temei antes aquele que pode condenar a alma e o corpo ao inferno. Porventura se não vendem dois pássaros por um ceitil? E nenhum deles, contudo, cairá no chão sem o consentimento do vosso Pai. Até os cabelos da vossa cabeça estão contados. Nada receeis, pois valeis mais do que muitos pássaros. Portanto, todo aquele que me confessar perante os homens também o confessarei na presença de meu Pai, que está nos céus».
}\end{paracol}

\paragraphinfo{Ofertório}{Sl. 20,4-5}
\begin{paracol}{2}\latim{
\rlettrine{P}{osuísti,} Dómine, in cápite ejus corónam de lápide pretióso: vitam pétiit a te, et tribuísti ei, allelúja.
}\switchcolumn\portugues{
\rlettrine{I}{mpusestes} na sua cabeça, Senhor, uma coroa de pedras preciosas; pediu-Vos a vida e concedestes-lha. Aleluia.
}\end{paracol}

\paragraph{Secreta}
\begin{paracol}{2}\latim{
\rlettrine{A}{ccépta} sit in conspéctu tuo, Dómine, nostra devótio: et ejus nobis fiat supplicatióne salutáris, pro cujus sollemnitáte defértur. Per Dóminum nostrum \emph{\&c.}
}\switchcolumn\portugues{
\rlettrine{R}{ecebei} benigno, Senhor, esta oferta da nossa piedade, e que ela nos alcance a salvação, por intercessão das preces daquele em cuja festa nós Vo-la apresentamos. Por nosso Senhor \emph{\&c.}
}\end{paracol}

\paragraphinfo{Comúnio}{Jo. 12, 26}
\begin{paracol}{2}\latim{
\qlettrine{Q}{ui} mihi mínistrat, me sequátur: et ubi sum ego, illic et miníster meus erit.
}\switchcolumn\portugues{
\rlettrine{S}{e} alguém me serve, siga-me; e onde eu estiver lá estará também o meu servo.
}\end{paracol}

\paragraph{Postcomúnio}
\begin{paracol}{2}\latim{
\rlettrine{R}{efécti} participatióne múneris sacri, quǽsumus, Dómine, Deus noster: ut, cujus exséquimur cultum, intercedénte beáto {\redx N.} Mártyre tuo, sentiámus efféctum. Per Dóminum \emph{\&c.}
}\switchcolumn\portugues{
\rlettrine{F}{ortalecidos} com a participação deste dom sacratíssimo, Vos suplicamos, Senhor, nosso Deus, que, por intercessão do B. {\redx N.}, vosso Mártir, sintamos o efeito do mistério que celebrámos. Por nosso Senhor \emph{\&c.}
}\end{paracol}

Outra Epístola (para certos dias):

\paragraphinfo{Epístola}{Tg. 1, 2-12}
\begin{paracol}{2}\latim{
Léctio Epístolæ beáti Jacóbi Apóstoli.
}\switchcolumn\portugues{
Lição da Ep.ª do B. Ap.º Tiago.
}\switchcolumn*\latim{
\rlettrine{C}{aríssime:} Omne gáudium existimáte, cum in tentatiónes várias incidéritis: sciéntes, quod probátio fídei vestræ patiéntiam operátur. Patiéntia autem opus perféctum habet: ut sitis perfécti et íntegri, in nullo deficiéntes. Si quis autem vestrum índiget sapiéntia, póstulet a Deo, qui dat ómnibus affluénter, et non impróperat: et dábitur ei. Póstulet autem in fide nihil hǽsitans: qui enim hǽsitat, símilis est flúctui maris, qui a vento movétur et circumfértur. Non ergo ǽstimet homo ille, quod accípiat áliquid a Dómino. Vir duplex ánimo incónstans est in ómnibus viis suis. Gloriétur autem frater húmilis in exaltatióne sua: dives autem in humilitáte sua, quóniam sicut flos fæni transíbit: exórtus est enim sol cum ardóre, et arefécit fænum, et flos ejus décidit et decor vultus ejus depériit: ita et dives in itinéribus suis marcéscet. Beátus vir, qui suffert tentatiónem: quóniam, cum probátus fúerit, accípiet corónam vitæ, quam repromísit Deus diligéntibus se.
}\switchcolumn\portugues{
\rlettrine{C}{aríssimo:} considerai como motivo de muita alegria as diversas aflições que vos acometerem; pois deveis saber que a provação da vossa fé produz a paciência. E que a paciência seja acompanhada de obras perfeitas, de maneira que sejais perfeitos e íntegros, nada deixando a desejar. Se algum de vós necessita de sabedoria, peça-a a Deus, que a dá a todos liberalmente, sem a ninguém repreender, e ela lhe será concedida. Peça-a, porém, com fé, sem desconfiança, pois aquele que duvida é semelhante a uma onda do mar, agitada pelo vento e movida de um lado para o outro. Com tais disposições nenhum homem imagine que há-de receber alguma coisa do Senhor. O homem que possui um espírito duplo é inconstante em suas acções. Aquele nosso irmão que é pobre glorie-se com a esperança da sua exaltação; e, ao contrário, aquele que é rico espere a sua humilhação, pois ele passará, como a flor da erva: o sol ergue-se ardente, secando a erva; então a flor da erva emurchece e toda sua beleza desaparece. Assim, o rico secará e murchará em suas empresas. Bem-aventurado o varão que suporta a tentação, porque, quando tiver acabado a provação, receberá a coroa da vida, que Deus prometeu aos que O amam.
}\end{paracol}

\subsectioninfo{Muitos Mártires}{Missa Intret in}\label{muitosmartires1}

\paragraphinfo{Intróito}{Sl. 78, 11, 12 \& 10}
\begin{paracol}{2}\latim{
\rlettrine{I}{ntret} in conspéctu tuo, Dómine, gémitus compeditórum: redde vicínis nostris séptuplum in sinu eórum: víndica sánguinem Sanctórum tuórum, qui effúsus est. \emph{Ps. ibid., 1} Deus, venérunt gentes in hereditátem tuam: polluérunt templum sanctum tuum: posuérunt Jerúsalem in pomórum custódiam.
℣. Gloria Patri \emph{\&c.}
}\switchcolumn\portugues{
\qlettrine{Q}{ue} os gemidos dos cativos cheguem à vossa presença, Senhor. Castigai os nossos inimigos sete vezes por cada injúria que nos têm feito: vingai o sangue que os vossos Santos derramaram. \emph{Ps. ibid., 1} Ó Deus, os povos invadiram a vossa herança, profanaram o vosso sagrado templo e reduziram Jerusalém a um monte de ruínas!
℣. Glória ao Pai \emph{\&c.}
}\end{paracol}

\paragraph{Oração}
\begin{paracol}{2}\latim{
\rlettrine{B}{eatórum} Mártyrum paritérque Pontíficum nos, quǽsumus, Dómine, festa tueántur: et eórum comméndet orátio veneránda. Per Dóminum \emph{\&c.}
}\switchcolumn\portugues{
\rlettrine{V}{os} suplicamos, Senhor, que a festa dos vossos B. B. Mártires e Pontífices {\redx N.} e {\redx N.}, nos proteja, e que sua veneranda oração nos sirva de recomendação junto de Vós. Por nosso Senhor \emph{\&c.}
}\end{paracol}

\textit{Não sendo Pontífice, diz-se a Oração da Missa seguinte, página \pageref{muitosmartires2}.}

\paragraphinfo{Epístola}{Sb. 3, 1-8}
\begin{paracol}{2}\latim{
Léctio libri Sapiéntiæ.
}\switchcolumn\portugues{
Lição do Livro da Sabedoria.
}\switchcolumn*\latim{
\qlettrine{J}{ustorum} ánimæ in manu Dei sunt, et non tanget illos torméntum mortis. Visi sunt oculis insipiéntium mori: et æstimála est afflíctio exitus illórum: et quod a nobis est iter, extermínium: illi autem sunt in pace. Et si coram homínibus
torménta passi sunt, spes illórum immortalitáte plena est. In paucis vexáti, in multis bene disponéntur: quóniam Deus tentávit eos, et invenit illos dignos se. Tamquam aurum in fornáce probávit illos, et quasi holocáusti hóstiam accépit illos, et in témpore erit respéctus illorum. Fulgébunt justi, et tamquam scintíllæ in arundinéto discúrrent. Judicábunt natiónes, et dominabúntur pópulis, et regnábit Dóminus illórum in perpétuum.
}\switchcolumn\portugues{
\rlettrine{A}{s} almas dos justos estão nas mãos de Deus; por isso o tormento da morte os não tocará. Pareciam mortos aos olhos dos insensatos: a sua saída do mundo parecia uma aflição; a sua separação de nós urna calamidade; mas, agora, estão em paz; e, ainda que tenham sofrido diante dos homens, a sua esperança está toda na imortalidade. Depois de haverem sofrido uma pena ligeira, receberam uma grande recompensa, pois Deus provou-os e achou-os dignos de si. Provou-os, como ao ouro, na fornalha; recebeu-os, como uma hóstia de holocausto; e para eles olhará benigno, quando vier o seu tempo. Os justos brilharão e resplandecerão, como as chamas, que se ateiam entre os canaviais. Eles julgarão as nações e dominarão os povos; e o Senhor reinará com eles para sempre.
}\end{paracol}

\paragraphinfo{Gradual}{Ex. 15,11}
\begin{paracol}{2}\latim{
\rlettrine{G}{loriósus} Deus in Sanctis suis: mirábilis in majestáte, fáciens prodígia. ℣. \emph{ibid., 6} Déxtera tua, Dómine, glorificáta est in virtúte: déxtera manus tua confrégit inimícos.
}\switchcolumn\portugues{
\rlettrine{D}{eus} é glorioso em seus Santos: e admirável na sua majestade, praticando prodígios. ℣. \emph{ibid., 6} Senhor, a vossa dextra engrandeceu-se pela sua força: a vossa dextra esmagou os inimigos.
}\switchcolumn*\latim{
Allelúja, allelúja. ℣. \emph{Eccli. 44, 14} Córpora Sanctórum in pace sepúlta sunt, et nómina eórum vivent in generatiónem et generatiónem. Allelúja.
}\switchcolumn\portugues{
Aleluia, aleluia. ℣. \emph{Ecl. 44, 14} Senhor, os corpos dos vossos Santos foram sepultados em paz e o seu nome subsistirá de geração em geração. Aleluia.
}\end{paracol}

\textit{Após a Septuagésima omite-se o Aleluia e o seguinte e diz-se:}

\paragraphinfo{Trato}{Sl. 125, 5-6}
\begin{paracol}{2}\latim{
\qlettrine{Q}{ui} séminant in lácrimis, in gáudio metent. ℣. Eúntes ibant et flébant, mitténtes sémina sua. ℣. Veniéntes autem vénient cum exsultatióne, portántes manípulos suos.
}\switchcolumn\portugues{
\rlettrine{A}{queles} que semeiam com lágrimas ceifarão com júbilo. ℣. Iam, caminhavam e lançavam a semente à terra, chorando. ℣. Porém, quando voltavam, exultavam de alegria, trazendo os seus molhos de trigo.
}\end{paracol}

\paragraphinfo{Evangelho}{Lc. 21, 9-19}
\begin{paracol}{2}\latim{
\cruz Sequéntia sancti Evangélii secúndum Lucam.
}\switchcolumn\portugues{
\cruz Continuação do santo Evangelho segundo S. Lucas.
}\switchcolumn*\latim{
\blettrine{I}{n} illo témpore: Dixit Jesus discípulis suis: Cum audieritis prǿlia et seditiónes, nolíte terréri: opórtet primum hæc fíeri, sed nondum statim finis. Tunc dicébat illis: Surget gens contra gentem, et regnum advérsus regnum. Et terræmótus magni erunt per loca, et pestiléntiæ, et fames, terrorésque de cœlo, et signa magna erunt. Sed ante hæc ómnia injícient vobis manus suas, et persequéntur tradéntes in synagógas et custódias, trahéntes ad reges et prǽsides propter nomen meum: contínget autem vobis in testimónium. Pónite ergo in córdibus vestris non præmeditári, quemádmodum respondeátis. Ego enim dabo vobis os et sapiéntiam, cui non potérunt resístere et contradícere omnes adversárii vestri. Tradémini autem a paréntibus, et frátribus, et cognátis, et amícis, et morte affícient ex vobis: et éritis ódio ómnibus propter nomen meum: et capíllus de cápite vestro non períbit. In patiéntia vestra possidébitis ánimas vestras.
}\switchcolumn\portugues{
\blettrine{N}{aquele} tempo, disse Jesus aos discípulos: «Quando ouvirdes falar em guerras e sedições, não vos assusteis; pois é necessário que estas coisas aconteçam, primeiramente; mas isto não será logo o fim». «Então dizia-lhes Ele levantar-se-á povo contra povo e reino contra reino; em diversos lugares haverá tremores de terra, peste, fome e também aparecerão coisas espantosas, grandes sinais no céu e outros prodígios. Mas, antes que tudo isto aconteça, lançar-vos-ão as mãos e perseguir-vos-ão, entregando-vos às sinagogas, lançando-vos nas prisões e conduzindo-vos à força diante dos reis e dos governadores, por causa do meu nome. Isto acontecerá para que deis testemunho da verdade. Gravai, pois, no vosso coração este pensamento: não premediteis de que modo haveis de responder, porque vos darei palavras e sabedoria, a que os vossos inimigos não poderão resistir, nem responder. Sereis entregues pelos vossos próprios pais, irmãos, parentes e amigos, que darão a morte a alguns de vós. Sereis aborrecidos de todos, por causa do meu nome; todavia, não se perderá nem um só cabelo das vossas cabeças. Com a vossa paciência possuireis as vossas almas».
}\end{paracol}

\paragraphinfo{Ofertório}{Sl. 67, 36}
\begin{paracol}{2}\latim{
\rlettrine{M}{irábilis} Deus in Sanctis suis: Deus Israël, ipse dabit virtútem et fortitúdinem plebi suæ: benedíctus Deus, allelúja.
}\switchcolumn\portugues{
\rlettrine{D}{eus} é admirável em seus Santos. Deus de Israel dará ao seu povo a força e a coragem. Bendito Ele seja, pois. Aleluia.
}\end{paracol}

\paragraph{Secreta}
\begin{paracol}{2}\latim{
\rlettrine{A}{désto,} Dómine, supplicatiónibus nostris, quas in Sanctórum tuórum commemoratióne deférimus: ut, qui nostræ justítiæ fidúciam non habémus, eórum, qui tibi placuérunt, méritis adjuvémur. Per Dóminum \emph{\&c.}
}\switchcolumn\portugues{
\rlettrine{A}{tendei,} Senhor, às súplicas que Vos dirigimos em memória dos vossos Santos, a fim de que nós, que não temos confiança na nossa própria justiça, sejamos auxiliados pelos méritos daqueles que Vos agradaram nesta vida. Por nosso Senhor \emph{\&c.}
}\end{paracol}

\paragraphinfo{Comúnio}{Sb. 3, 4, 5 \& 6}
\begin{paracol}{2}\latim{
\rlettrine{E}{t} si coram homínibus torménta passi sunt, Deus tentávit eos: tamquam aurum in fornáce probávit eos, et quasi holocáusta accépit eos.
}\switchcolumn\portugues{
\rlettrine{S}{e} sofreram tormentos diante dos homens, foi para Deus os provar. Deus provou-os na fornalha, como ao ouro, e recebeu-os, como hóstia de holocausto.
}\end{paracol}

\paragraph{Postcomúnio}
\begin{paracol}{2}\latim{
\qlettrine{Q}{uǽsumus,} Dómine, salutáribus repléti mystériis: ut, quorum sollémnia celebrámus, eórum oratiónibus adjuvémur. Per Dóminum \emph{\&c.}
}\switchcolumn\portugues{
\rlettrine{F}{ortificados} com vossos salutares mistérios, dignai-Vos conceder-nos, Senhor, a graça da assistência das orações daqueles cuja festa celebrámos. Por nosso Senhor \emph{\&c.}
}\end{paracol}

\subsectioninfo{Muitos Mártires}{Missa Sapiéntiam sanctórum}\label{muitosmartires2}

\paragraphinfo{Intróito}{Ecl. 44,15 \& 14}
\begin{paracol}{2}\latim{
\rlettrine{S}{apiéntiam} Sanctórum narrent pópuli, et laudes eórum núntiet ecclésia: nomina autem eórum vivent in sǽculum sǽculi. \emph{Ps. 32, 1} Exsultáte, justi, in Dómino: rectos decet collaudátio.
℣. Gloria Patri \emph{\&c.}
}\switchcolumn\portugues{
\qlettrine{Q}{ue} os povos publiquem a sabedoria dos santos e que a Igreja celebre os seus louvores: o seu nome subsistirá em todos os séculos! \emph{Sl. 32, 1} Ó justos, rejubilai no Senhor: é àqueles que possuem o coração recto que pertence louvar o Senhor.
℣. Glória ao Pai \emph{\&c.}
}\end{paracol}

\paragraph{Oração}
\begin{paracol}{2}\latim{
\rlettrine{D}{eus,} qui nos concédis sanctórum Mártyrum tuórum {\redx N.} et {\redx N.} natalítia cólere: da nobis in ætérna beatitúdine de eórum societéte gaudére. Per Dóminum \emph{\&c.}
}\switchcolumn\portugues{
\slettrine{Ó}{} Deus, que nos permitistes a graça de celebrarmos o nascimento no céu dos vossos Santos Mártires {\redx N.} e {\redx N.}, concedei-nos ainda a graça de gozarmos na sua companhia a bem-aventurança eterna. Por nosso Senhor \emph{\&c.}
}\end{paracol}

\textit{Se forem Pontífices, não se diz esta Oração mas a da Missa precedente, página \pageref{muitosmartires2}.}

\paragraphinfo{Epístola}{Sb. 5, 16-20}
\begin{paracol}{2}\latim{
Léctio libri Sapiéntiæ.
}\switchcolumn\portugues{
Lição do Livro da Sabedoria.
}\switchcolumn*\latim{
\qlettrine{J}{usti} autem in perpétuum vivent, et apud Dóminum est merces eórum, et cogitátio illórum apud Altíssimum. Ideo accípient regnum decóris, et diadéma speciéi de manu Dómini: quóniam déxtera sua teget eos, et bráchio sancto suo deféndet illos. Accípiet armatúram zelus illíus, et armábit creatúram ad ultiónem inimicórum. Induet pro thoráce justítiam, et accípiet pro gálea judícium certum. Sumet scutum inexpugnábile æquitátem.
}\switchcolumn\portugues{
\rlettrine{O}{s} justos viverão eternamente e alcançarão recompensa junto do Senhor, pois o Altíssimo cuidará deles. Eis porque receberão das mãos do Senhor um reino de glória e um diadema brilhante! O Senhor protegê-los-á com sua dextra, cobrindo-os com seu divino braço, que será como um escudo. Seu zelo o levará a tomar a armadura e a armar as criaturas para se vingar dos seus inimigos. Envergará a justiça como couraça e a integridade do juízo como capacete; e revestir-se-á com a equidade como escudo inexpugnável.
}\end{paracol}

\paragraphinfo{Gradual}{Sl. 123,7-8}
\begin{paracol}{2}\latim{
\rlettrine{A}{nima} nostra, sicut passer, erépta est de láqueo venántium. ℣. Láqueus contrítus est, et nos liberáti sumus: adjutórium nostrum in nómine Dómini, qui fecit cœlum et terram.
}\switchcolumn\portugues{
\rlettrine{A}{} nossa alma livrou-se, como um pássaro do laço dos caçadores! ℣. O laço quebrou-se, e ficámos livres. O nosso auxílio está no nome do Senhor, que criou o céu e a terra.
}\switchcolumn*\latim{
Allelúja, alielúja. ℣. \emph{Ps.67, 4} Justi epuléntur, et exsúltent in conspéctu Dei: et delecténtur in lætítia. Allelúja.
}\switchcolumn\portugues{
Aleluia, aleluia. ℣. \emph{Sl. 67, 4} Que os justos se regozijem e exultem de alegria na presença de Deus, como em um banquete. Que eles se deliciem em transportes de alegria. Aleluia.
}\end{paracol}

\textit{Após a Septuagésima omite-se o Aleluia e o seguinte e diz-se:}

\paragraphinfo{Trato}{Sl. 125, 5-6}
\begin{paracol}{2}\latim{
\qlettrine{Q}{ui} séminant in lácrimis, in gáudio metent. ℣. Eúntes ibant et fiébant, mitténtes sémina sua. ℣. Veniéntes autem vénient cum exsultatióne, portántes manípulos suos
}\switchcolumn\portugues{
\rlettrine{A}{queles} que semeiam com lágrimas ceifarão com júbilo. ℣. Iam, caminhavam e lançavam a semente à terra, chorando. ℣. Porém, quando voltavam, exultavam de alegria, trazendo os seus molhos de trigo.
}\end{paracol}

\paragraphinfo{Evangelho}{Lc. 6, 17-23}
\begin{paracol}{2}\latim{
\cruz Sequéntia sancti Evangélii secúndum Lucam.
}\switchcolumn\portugues{
\cruz Continuação do santo Evangelho segundo S. Lucas.
}\switchcolumn*\latim{
\blettrine{I}{n} illo témpore: Descéndens Jesus de monte, stetit in loco campéstri, et turba discipulórum ejus, et multitúdo copiósa plebis ab omni Judǽa, et Jerúsalem, et marítima, et Tyri, et Sidónis, qui vénerant, ut audírení eum et sanaréntur a languóribus suis. Et, qui vexabántur a spirítibus immúndis, curabántur. Et omnis turba quærébat eum tangere: quia virtus de illo exíbat, et sanábat omnes. Et ipse, elevátis óculis in discípulos suos, dicebat: Beáti, páuperes: quia vestrum est regnum Dei. Beáti, qui nunc esurítis: quia saturabímini. Beáti, qui nunc fletis: quia ridébitis. Beáti eritis, cum vos óderint hómines, et cum separáverint vos et exprobráveriní, et ejécerint nomen vestrum tamquam malum, propter Fílium hóminis. Gaudéte in illa die et exsultáte: ecce enim, merces vestra multa est in cœlo.
}\switchcolumn\portugues{
\blettrine{N}{aquele} tempo, descendo Jesus da montanha, parou em uma planície com a turba dos seus discípulos e grande multidão de povo de toda a Judeia, de Jerusalém, das margens do mar, de Tiro e de Sidónia, que tinha vindo para ouvi-l’O e ser curado de suas enfermidades. E os que estavam possessos de espíritos imundos ficavam sãos. Toda a multidão procurava tocá-l’O, porque saía d’Ele uma virtude que curava a todos. Levantando, então, Jesus os olhos para os seus discípulos, disse: «Bem-aventurados vós, pobres, porque o reino dos céus é vosso; bem-aventurados vós, famintos, porque sereis fartos; bem-aventurados vós, que agora chorais, porque depois rireis; bem-aventurados vós, quando sois odiados e injuriados e quando os homens se aborrecem e rejeitam o vosso nome, como se fora mau, por causa do Filho do homem. Alegrai-vos e rejubilai, pois uma grande recompensa vos está reservada no céu».
}\end{paracol}

\paragraphinfo{Ofertório}{Sl. 149, 5-6}
\begin{paracol}{2}\latim{
\rlettrine{E}{xsultábunt} Sancti in glória, lætabúntur in cubílibus suis: exaltatiónes Dei in fáucibus eórum, allelúja.
}\switchcolumn\portugues{
\rlettrine{O}{s} santos exultarão de alegria na sua glória e deliciar-se-ão de alegria no lugar do seu repouso. Ressoarão em seus lábios louvores a Deus.
}\end{paracol}

\paragraph{Secreta}
\begin{paracol}{2}\latim{
\rlettrine{M}{únera} tibi, Dómine, nostræ devotiónis offérimus: quæ et pro tuórum tibi grata sint honóre Justórum, et nobis salutária, te miseránte, reddántur. Per Dóminum \emph{\&c.}
}\switchcolumn\portugues{
\rlettrine{V}{os} oferecemos, Senhor, estes dons da nossa devoção; e, em atenção aos merecimentos dos vossos justos, dignai-Vos aceitá-los, e pela vossa misericórdia fazei que nos sejam salutares. Por nosso Senhor \emph{\&c.}
}\end{paracol}

\paragraphinfo{Comúnio}{Lc. 12, 4}
\begin{paracol}{2}\latim{
\rlettrine{D}{ico} autem vobis amícis meis: Ne terreámini ab his, qui vos persequúntur.
}\switchcolumn\portugues{
\rlettrine{D}{igo-vos,} pois, a vós, que sois meus amigos: não tenhais medo daqueles que vos perseguem.
}\end{paracol}

\paragraph{Postcomúnio}
\begin{paracol}{2}\latim{
\rlettrine{P}{ræsta} nobis, quǽsumus, Dómine: intercedéntibus sanctis Martýribus tuis {\redx N.} et {\redx N.}; ut, quod ore contíngimus, pura mente capiámus. Per Dóminum \emph{\&c.}
}\switchcolumn\portugues{
\rlettrine{S}{enhor,} por intercessão dos vossos santos Mártires {\redx N.} e {\redx N.}, dignai-Vos conceder-nos a graça de guardarmos com o coração sempre puro o que a nossa boca agora recebeu. Por nosso Senhor \emph{\&c.}
}\end{paracol}

\subsectioninfo{Muitos Mártires}{Missa Salus autem}\label{muitosmartires3}

\paragraphinfo{Intróito}{Sl. 36, 39}
\begin{paracol}{2}\latim{
\rlettrine{S}{alus} autem justórum a Dómino: et protéctor eórum est in témpore tribulatiónis. \emph{Ps. ibid., 1} Noli æmulári in malignántibus: neque zeláveris faciéntes iniquitátem.
℣. Gloria Patri \emph{\&c.}
}\switchcolumn\portugues{
\rlettrine{A}{} salvação dos justos, porém, está no Senhor; Ele é o seu refúgio na ocasião da tribulação. \emph{Sl. ibid., 1} Não invejeis os maus, nem tenhais emulação daqueles que cometem iniquidades.
℣. Glória ao Pai \emph{\&c.}
}\end{paracol}

\paragraph{Oração}
\begin{paracol}{2}\latim{
\rlettrine{D}{eus,} qui nos ánnua sanctórum Mártyrum tuórum et {\redx N.} sollemnitáte lætíficas: concéde propítius; ut, quorum gaudémus méritis, accendámur exémplis. Per Dóminum \emph{\&c.}
}\switchcolumn\portugues{
\slettrine{Ó}{} Deus, que nos alegrais com a festividade anual dos vossos santos Mártires {\redx N.} e {\redx N.}, concedei-nos benigno que sejamos afervorados com os exemplos daqueles cujos méritos nos enchem de alegria. Por nosso Senhor \emph{\&c.}
}\end{paracol}

\paragraphinfo{Epístola}{Heb. 10, 32-38}
\begin{paracol}{2}\latim{
Léctio Epístolæ beáti Pauli Apóstoli ad Hebrǽos.
}\switchcolumn\portugues{
Lição da Ep.ª do B. Ap.º Paulo aos Hebreus.
}\switchcolumn*\latim{
\rlettrine{F}{ratres:} Rememorámini prístinos dies, in quibus illumináti magnum certámen sustinuístis passiónum: et in áltero quidem oppróbriis et tribulatiónibus spectáculum facti: in áltero autem sócii táliter conversántium effécti. Nam et vinctis compássi estis, et rapínam bonórum vestrórum cum gáudio suscepístis, cognoscéntes vos habere meliórem et manéntem substántiam. Nolíte itaque amíttere! confidéntiam vestram, quæ magnam habet remuneratiónem. Patiéntia enim vobis necéssaria est: ut, voluntátem Dei faciéntes, reportétis promissiónem. Adhuc enim módicum aliquántulum, qui ventúrus est, véniet, et non tardábit. Justus autem meus ex fide vivit.
}\switchcolumn\portugues{
\rlettrine{M}{eus} irmãos: Lembrai-vos dos primeiros dias em que, depois de haverdes recebido as luzes da fé, sofrestes grandes combates dolorosos, havendo sido, por um lado, expostos, diante de toda a gente, aos opróbrios e aos maus tratos, e, por outro lado, associados aos que sofreram iguais tormentos. Com efeito, vós compartilhastes as agruras dos prisioneiros e suportastes com alegria o despojo dos vossos bens, conhecendo que uma riqueza maior, que nunca seria arrebatada, vos estava reservada. Não percais a confiança que tendes, a qual deve alcançar uma recompensa de elevado valor. Com efeito, tendes necessidade de perseverança para que, depois de haverdes cumprido a vontade de Deus, alcanceis o prémio prometido. Esperai ainda algum tempo, e então chegará aquele que deve vir, o qual não tardará. Meu justo vive, pois, da sua fé.
}\end{paracol}

\paragraphinfo{Gradual}{Sl. 33, 18-19}
\begin{paracol}{2}\latim{
\rlettrine{C}{lamavérunt} justi, et Dóminus exaudívit eos: et ex ómnibus tribulatiónibus eórum liberávit eos. ℣. Juxta est Dóminus his, qui tribuláto sunt corde: et húmiles spíritu salvábit.
}\switchcolumn\portugues{
\rlettrine{C}{lamaram} os justos; então o senhor ouviu-os e livrou-os de todas suas aflições. ℣. O Senhor está próximo daqueles que têm o coração atribulado; e salvará os que têm o espírito humilhado.
}\switchcolumn*\latim{
Allelúja, allelúja. ℣. Te Mártyrum candidátus laudat exércitus, Dómine. Allelúja.
}\switchcolumn\portugues{
Aleluia, aleluia. ℣. O exército cândido dos mártires, ó Senhor, canta louvores em vossa honra! Aleluia.
}\end{paracol}

\textit{Após a Septuagésima omite-se o Aleluia e o seguinte e diz-se:}

\paragraphinfo{Trato}{Sl. 125, 5-6}
\begin{paracol}{2}\latim{
\qlettrine{Q}{ui} séminant in lácrimis, in gáudio metent. ℣. Eúntes ibant et flébant, mitténtes sémina sua. ℣. Veniéntes autem vénient cum exsultatióne, portántes manípulos suos.
}\switchcolumn\portugues{
\rlettrine{A}{queles} que semeiam com lágrimas ceifarão com júbilo. ℣. Iam, caminhavam e lançavam a semente à terra, chorando. ℣. Porém, quando voltavam, exultavam de alegria, trazendo os seus molhos de trigo.
}\end{paracol}

\paragraphinfo{Evangelho}{Lc. 12, 1-8}
\begin{paracol}{2}\latim{
\cruz Sequéntia sancti Evangélii secúndum Lucam.
}\switchcolumn\portugues{
\cruz Continuação do santo Evangelho segundo S. Lucas.
}\switchcolumn*\latim{
\blettrine{I}{n} illo témpore: Dixit Jesus discípulis suis: Atténdite a ferménto pharisæórum, quod est hypócrisis. Nihil autem opértum est, quod non revelétur: neque abscónditum, quod non sciátur. Quóniam, quæ in ténebris dixístis, in lúmine dicéntur: et quod in aurem locuti estis in cubículis, prædicábitur in tectis. Dico autem vobis amícis meis: Ne terreámini ab his, qui occídunt corpus, et post hæc non habent ámplius quid fáciant. Osténdam autem vobis, quem timeátis: timéte eum, qui, postquam occídent, habet potestátem míttere in gehénnam. Ita dico vobis: hunc timéte. Nonne quinque pásseres véneunt dipóndio, et unus ex illis non est in oblivióne coram Deo? Sed et capílli cápitis vestri omnes numerári sunt. Nolíte ergo timére: multis passéribus pluris estis vos. Dico autem vobis: Omnis, quicúmque conféssus fúerit me coram homínibus, et Fílius hóminis confiténtur illum coram Angelis Dei.
}\switchcolumn\portugues{
\blettrine{N}{aquele} tempo, disse Jesus aos seus discípulos: «Acautelai-vos com o fermento dos fariseus, que é a hipocrisia, pois nada há oculto que não chegue a ser descoberto, nem segredo que não venha a ser revelado. Até aquelas cousas que dissestes nas trevas serão publicadas à luz; e o que dissestes ao ouvido, no recôndito dos vossos cubículos, será apregoado sobre os telhados. Digo-vos, porém, a vós, que sois meus amigos: não tenhais receio daqueles que matam o corpo e depois não podem fazer mais nada. Sabeis a quem deveis temer? Temei aquele que, depois de haver dado a morte, tem ainda o poder de lançar no inferno. Sim; eu vo-lo digo: temei este. Porventura se não vendem cinco pássaros por dous ceitis? Contudo, nem um só deles fica em esquecimento diante de Deus! Até os cabelos da vossa cabeça estão contados! Não tenhais, pois, receio. Vós valeis mais do que muitos pássaros. Também vos digo: todo aquele que me confessar diante dos homens, o Filho do homem o reconhecerá igualmente diante dos Anjos de Deus».
}\end{paracol}

\paragraphinfo{Ofertório}{Sb. 3, 1, 2 \& 3}
\begin{paracol}{2}\latim{
\qlettrine{J}{ustórum} ánimæ in manu Dei sunt, et non tanget illos torméntum malítiae: visi sunt óculis insipiéntium mori: illi autem sunt in pace, allelúja.
}\switchcolumn\portugues{
\rlettrine{A}{s} almas dos justos estão nas mãos de Deus e o tormento da malícia as não tocará! Aos olhos dos insensatos pareciam quase a morrer, todavia estão na paz. Aleluia.
}\end{paracol}

\paragraph{Secreta}
\begin{paracol}{2}\latim{
\rlettrine{O}{blátis,} quǽsumus, Dómine, placáre munéribus: et, intercedéntibus sanctis Martýribus tuis {\redx N.} et {\redx N.}, a cunctis nos defénde perículis. Per Dóminum \emph{\&c.}
}\switchcolumn\portugues{
\rlettrine{D}{eixai-Vos} aplacar com os dons que Vos oferecemos, Senhor, e, por intercessão dos vossos santos Mártires {\redx N.} e {\redx N.}, preservai-nos de todos os perigos. Por nosso Senhor \emph{\&c.}
}\end{paracol}

\paragraphinfo{Comúnio}{Mt. 10, 27}
\begin{paracol}{2}\latim{
\qlettrine{Q}{uod} dico vobis in ténebris, dícite in lúmine, dicit Dóminus: et quod in aure audítis, prædicáte super tecta.
}\switchcolumn\portugues{
\rlettrine{O}{} que vos digo nas trevas dizei-o às claras, diz o Senhor; e o que vos disse ao ouvido pregai-o em cima dos telhados.
}\end{paracol}

\paragraph{Postcomúnio}
\begin{paracol}{2}\latim{
\rlettrine{H}{æc} nos commúnio, Dómine, purget a crímine: et, intercedéntibus sanctis Martýribus tuis {\redx N.} et {\redx N.}, cœléstis remédii fáciat esse consórtes. Per Dóminum \emph{\&c.}
}\switchcolumn\portugues{
\qlettrine{Q}{ue} esta comunhão nos purifique dos nossos crimes, Senhor, e que por intercessão dos vossos santos Mártires {\redx N.} e {\redx N.} nos faça participantes do remédio celestial. Por nosso Senhor \emph{\&c.}
}\end{paracol}

\textit{Outro Evangelho (para certos dias):}

\paragraphinfo{Evangelho}{Lc. 12, 1-8}\label{evangelho2muitosmartires3}
\begin{paracol}{2}\latim{
\cruz Sequéntia sancti Evangélii secúndum Matthǽum.
}\switchcolumn\portugues{
\cruz Continuação do santo Evangelho segundo S. Mateus.
}\switchcolumn*\latim{
\blettrine{I}{n} illo témpore: Sedénte Jesu super montem Olivéti, accessérunt ad eum discípuli secréto, dicéntes: Dic nobis, quando hæc erunt? et quod signum advéntus tui et consummatiónis sǽculi? Et respóndens Jesus, dixit eis: Vidéte, ne quis vos sedúcat. Multi enim vénient in nómine meo, dicéntes: Ego sum Christus: et multos sedúcent. Auditúri enim estis prǿlia et opiniónes prœliórum. Vidéte, ne turbémini. Opórtet enim hæc fíeri, sed nondum est finis. Consúrget enim gens in gentem, et regnum in regnum, et erunt pestiléntiæ et fames et terræmótus per loca. Hæc autem ómnia inítia sunt dolórum. Tunc tradent vos in tribulatiónem et occídent vos: et éritis ódio ómnibus géntibus propter nomen meum. Et tunc scandalizabúntur multi, et ínvicem tradent, et odio habébunt ínvicem. Et multi pseudoprophétæ surgent, et sedúcent multos. Et quóniam abundávit iníquitas, refrigéscet cáritas multórum. Qui autem perseveráverit usque in finem, hic salvus erit.
}\switchcolumn\portugues{
\blettrine{N}{aquele} tempo, sentando-se Jesus no Monte das Oliveiras, aproximaram-se d’Ele em particular os seus discípulos, perguntando-Lhe: «Dizei-nos quando acontecerão essas cousas? Que sinal haverá da vossa vinda e da consumação dos séculos?». Então, respondendo, Jesus disse-lhes: «Acautelai-vos, para que ninguém vos seduza; pois virão muitos, dizendo: sou Cristo, seduzindo também muitas pessoas; ouvireis falar de guerras e de rumores de guerras. Tende cuidado em vos não perturbardes, pois convém que tais cousas aconteçam; mas isso não será ainda o fim. Levantar-se-á povo contra povo e reino contra reino; haverá peste, fome e tremores de terra em diversos lugares; mas todas estas cousas serão apenas o começo das aflições. Entregar-vos-ão, para sofrerdes tormentos até à morte; sereis odiados por todos os povos por causa do meu nome; pelo que muitos se deixarão seduzir, se denunciarão e se atraiçoarão uns aos outros. Aparecerão muitos falsos profetas, que seduzirão muitas pessoas. Então, porque a iniquidade terá atingido o cúmulo, resfriará a caridade de muitos. Aquele que perseverar até ao fim será salvo».
}\end{paracol}


\subsectioninfo{Comum dos Mártires}{Dentro do Tempo Pascal}

\subsectioninfo{Mártir}{Missa Protexísti me}\label{martir}

\paragraphinfo{Intróito}{Sl. 63, 3}
\begin{paracol}{2}\latim{
\rlettrine{P}{rotexísti} me, Deus, a convéntu malignántium, allelúja: a multitúdine operántium iniquitátem, allelúja, allelúja. \emph{Ps. ibid., 2} Exáudi, Deus, oratiónem meam, cum déprecor: a timóre inimíci éripe ánimam meam.
℣. Gloria Patri \emph{\&c.}
}\switchcolumn\portugues{
\rlettrine{P}{rotegestes-me,} ó Deus, contra os conluios dos maus e contra a multidão daqueles que cometem iniquidades. Aleluia, aleluia. \emph{Sl. ibid., 2} Ouvi, ó Deus, a oração que Vos dirijo: livrai a minha alma do temor do inimigo.
℣. Glória ao Pai \emph{\&c.}
}\end{paracol}

\textit{Por um Mártir Pontífice diz-se a seguinte:}

\paragraph{Oração}
\begin{paracol}{2}\latim{
\rlettrine{I}{nfirmitátem} nostram réspice, omnípotens Deus: et, quia pondus própriæ actiónis gravat, beáti {\redx N.} Mártyris tui atque Pontíficis intercéssio gloriósa nos prótegat. Per Dóminum \emph{\&c.}
}\switchcolumn\portugues{
\slettrine{Ó}{} Deus omnipotente, olhai para a nossa fraqueza: e, visto que estamos oprimidos com o peso dos nossos pecados, dignai-Vos permitir que sejamos protegidos pela gloriosa intercessão do B. {\redx N.}, vosso Mártir e Pontífice. Por nosso Senhor \emph{\&c.}
}\end{paracol}

\textit{Outras vezes, em vez da Precedente, diz-se esta:}

\paragraph{Oração}
\begin{paracol}{2}\latim{
\rlettrine{D}{eus,} qui nos beáti {\redx N.} Mártyris tui atque Pontíficis ánnua sollemnitáte lætíficas: concéde propítius; ut, cujus natalítia cólímus, de ejúsdem étiam protectióne gaudeámus. Per Dóminum \emph{\&c.}
}\switchcolumn\portugues{
\slettrine{Ó}{} Deus, que nos alegrais com a solenidade anual do B. {\redx N.} vosso Mártir e Pontífice, concedei-nos propício que nos congratulemos com a protecção daquele cujo nascimento no céu celebramos. Por nosso Senhor \emph{\&c.}
}\end{paracol}

\textit{Por um Mártir não Pontífice diz-se a seguinte:}

\paragraph{Oração}
\begin{paracol}{2}\latim{
\rlettrine{P}{ræsta,} quǽsumus, omnípotens Deus: ut, qui beáti {\redx N.} Mártyris tui natalítia cólimus, intercessióne ejus, in tui nóminis amóre roborémur. Per Dóminum \emph{\&c.}
}\switchcolumn\portugues{
\slettrine{C}{oncedei-nos,} ó Deus omnipotente, Vos suplicamos, que, celebrando nós o nascimento do vosso B. Mártir {\redx N.}, sejamos confirmados pela sua intercessão no amor ao vosso nome. Por nosso Senhor \emph{\&c.}
}\end{paracol}

\textit{Outras vezes, em vez da Precedente, diz-se esta:}

\paragraph{Oração}
\begin{paracol}{2}\latim{
\rlettrine{P}{ræsta,} quǽsumus, omnípotens Deus: ut, intercedénte beáto {\redx N.} Mártyre tuo, et a cunctis adversitátibus liberémur in córpore, et a pravis cogitatiónibus mundémur in mente. Per Dóminum \emph{\&c.}
}\switchcolumn\portugues{
\slettrine{P}{ermiti,} ó Deus omnipotente, Vos imploramos, que, pela intercessão do vosso B. Mártir {\redx N.}, os nossos corpos sejam livres de todas as adversidades e as nossas almas purificadas dos maus pensamentos. Por nosso Senhor \emph{\&c.}
}\end{paracol}

\paragraphinfo{Epístola}{Sb. 5, 1-5}
\begin{paracol}{2}\latim{
Léctio libri Sapiéntiæ.
}\switchcolumn\portugues{
Lição do Livro da Sabedoria.
}\switchcolumn*\latim{
\rlettrine{S}{tabunt} justi in magna constántia advérsus eos, qui se angustiavérunt et qui abstulérunt labóres eórum. Vidéntes turbabúntur timore horríbili, et mirabúntur in subitatióne
insperátæ salútis, dicéntes intra se, pœniténtiam agéntes, et præ angústia spíritus geméntes: Hi sunt, quos habúimus aliquándo in derísum et in similitúdinem impropérii. Nos insensáti vitam illórum æstimabámus insániam, et finem illórum sine honóre: ecce, quómodo computáti sunt inter fílios Dei, et inter Sanctos sors illórum est.
}\switchcolumn\portugues{
\rlettrine{E}{ntão,} os justos erguer-se-ão com grande coragem contra aqueles que os oprimiam e a quem arrebatavam o fruto dos seus trabalhos. Vendo-os assim, os maus perturbar-se-ão, cheios de pavor, e ficarão assombrados com a súbita e inesperada salvação dos justos, dizendo de si para si, arrependidos e angustiados: «Estes são aqueles a quem outrora quisemos injuriar com nossas zombarias e insultos. Insensatos que nós fomos! Pareceu-nos que sua vida era uma loucura, e a sua morte uma vergonha; mas eis que os vemos elevados à dignidade de filhos de Deus e compartilhando da glória dos santos!»
}\end{paracol}

\begin{paracol}{2}\latim{
Allelúja, allelúja. ℣. \emph{Ps. 88, 6} Confitebúntur cœli mirabília tua, Dómine: étenim veritátem tuam in ecclésia sanctórum. Allelúja. ℣. Ps. 20, 4. Posuísti, Dómine, super caput ejus corónam de lápide pretióso. Allelúja.
}\switchcolumn\portugues{
Aleluia, aleluia. ℣. \emph{Sl. 88, 6} Senhor, que os céus festejem as vossas maravilhas; que a vossa verdade seja exaltada na assembleia dos santos. Aleluia. Impusestes na sua cabeça, Senhor, uma coroa de pedras preciosas. Aleluia.
}\end{paracol}

\paragraphinfo{Evangelho}{Jo. 15, 1-7}
\begin{paracol}{2}\latim{
\cruz Sequéntia sancti Evangélii secúndum Joánnem.
}\switchcolumn\portugues{
\cruz Continuação do santo Evangelho segundo S. João.
}\switchcolumn*\latim{
\blettrine{I}{n} illo témpore: Dixit Jesus discípulis suis: Ego sum vitis vera: et Pater meus agrícola est. Omnem pálmitem in me non feréntem fructum, tollet eum: et omnem, qui fert fructum, purgábit eum, ut fructum plus áfferat. Jam vos mundi estis propter sermónem, quem locútus sum vobis. Mane te in me: et ego in vobis. Sicut palmes non potest ferre fructum a semetípso, nisi mánserit in vite: sic nec vos, nisi in me manséritis. Ego sum vitis, vos pálmites: qui manet in me, et ego in eo, hic fert fructum multum: quia sine me nihil potéstis fácere. Si quis in me non mánserit, mittétur foras sicut palmes, et aréscet, et cólligent eum, et in ignem mittent, et ardet. Si manséritis in me, et verba mea in vobis mánserint: quodcúmque voluéritis, petétis, et fiet vobis.
}\switchcolumn\portugues{
\blettrine{E}{u} sou a verdadeira vinha e meu Pai é o vinhateiro. Toda a videira que não der fruto em mim será cortada por Ele, assim como podará a que der fruto, para que o dê com mais abundância. Vós estais já limpos em razão da doutrina que vos tenho pregado. Permanecei em mim, pois eu permaneço em vós. Assim como a videira não pode dar fruto só por si, sem permanecer unida à cepa, assim também não podereis vós dar fruto se não permanecerdes unidos a mim. Eu sou a cepa da vinha, e vós sois as vides. Aquele que permanece em mim, eu permaneço nele, e dará abundante fruto; pois nada podereis fazer sem mim. Se alguém não permanecer em mim, será arrancado e lançado fora, como uma vide seca. Então secará e levá-la-ão para a lançarem no fogo, em que arderá. Se permanecerdes em mim e as minhas palavras permanecerem em vós, tudo o que quiserdes podereis pedir, que vos será concedido.
}\end{paracol}

\paragraphinfo{Ofertório}{Sb. 88, 6}
\begin{paracol}{2}\latim{
\rlettrine{C}{onfitebúntur} cœli mirabília tua, Dómine: et veritátem tuam in ecclésia sanctórum, allelúja, allelúja.
}\switchcolumn\portugues{
\rlettrine{S}{enhor,} que os céus publiquem as vossas maravilhas; que, a vossa verdade seja exaltada na assembleia dos santos. Aleluia.
}\end{paracol}

\textit{Por um Mártir Pontífice diz-se a seguinte:}

\paragraph{Secreta}
\begin{paracol}{2}\latim{
\rlettrine{H}{óstias} tibi, Dómine, beáti {\redx N.} Mártyris tui atque Pontíficis dicátas méritis, benígnus assúme: et ad perpétuum nobis tríbue proveníre subsídium. Per Dóminum \emph{\&c.}
}\switchcolumn\portugues{
\rlettrine{A}{ceitai} benigno, Senhor, as hóstias que Vos oferecemos pelos méritos do B. {\redx N.}, vosso Mártir e Pontífice; e dignai-Vos permitir que em virtude delas alcancemos o vosso perpétuo socorro. Por nosso Senhor \emph{\&c.}
}\end{paracol}

\textit{Outras vezes, em vez da Precedente, diz-se esta:}

\paragraph{Secreta}
\begin{paracol}{2}\latim{
\rlettrine{M}{únera} tibi, Dómine, dicáta sanctífica: et, intercedénte beáto {\redx N.} Mártyre tuo atque Pontífice, per éadem nos placátus inténde. Per Dóminum \emph{\&c.}
}\switchcolumn\portugues{
\rlettrine{S}{antificai,} Senhor, estes dons que Vos são oferecidos, a fim de que pela intercessão do B. {\redx N.}, vosso Mártir e Pontífice, Vos digneis aplacar-Vos, e olhar aplacado para nós. Por nosso Senhor \emph{\&c.}
}\end{paracol}

\textit{Por um Mártir não Pontífice diz-se a seguinte:}

\paragraph{Secreta}
\begin{paracol}{2}\latim{
\rlettrine{M}{unéribus} nostris, quǽsumus, Dómine, precibúsque suscéptis: et cœléstibus nos munda mystériis, et cleménter exáudi. Per Dóminum nostrum \emph{\&c.}
}\switchcolumn\portugues{
\rlettrine{H}{avendo} Vós aceitado os nossos dons e as nossas orações, dignai-Vos purificar-nos com vossos celestiais mistérios e ouvir-nos clementemente. Por nosso Senhor \emph{\&c.}
}\end{paracol}

\textit{Outras vezes, em vez da Precedente, diz-se esta:}

\paragraph{Secreta}
\begin{paracol}{2}\latim{
\rlettrine{A}{ccépta} sit in conspéctu tuo, Dómine, nostra devótio: et ejus nobis fiat supplicatióne salutáris, pro cujus sollemnitáte defértur. Per Dóminum \emph{\&c.}
}\switchcolumn\portugues{
\rlettrine{A}{ceitai} benignamente, Senhor, esta oferta que a nossa devoção Vos apresenta; e permiti que nos alcance a salvação pelas orações daquele em cuja festa Vo-la apresentamos. Por nosso Senhor \emph{\&c.}
}\end{paracol}

\paragraphinfo{Comúnio}{Sl. 63, 11}
\begin{paracol}{2}\latim{
\rlettrine{L}{ætábitur} justus in Dómino, et sperábit in eo: et laudabúntur omnes recti corde, allelúja, allelúja.
}\switchcolumn\portugues{
\rlettrine{O}{} justo rejubilará no Senhor e nele porá a sua confiança, pois todos aqueles que possuem o coração recto serão louvados. Aleluia, aleluia.
}\end{paracol}

\textit{Por um Mártir Pontífice diz-se o seguinte:}

\paragraph{Postcomúnio}
\begin{paracol}{2}\latim{
\rlettrine{R}{efécti} participatióne múneris sacri, quǽsumus, Dómine, Deus noster: ut, cujus exséquimur cultum, intercedénte beáto {\redx N.} Martyre tuo atque Pontifice, sentiámus efféctum. Per Dóminum \emph{\&c.}
}\switchcolumn\portugues{
\qlettrine{S}{aciados} com a participação do dom sacratíssimo, Vos suplicamos, ó Senhor, nosso Deus, fazei-nos sentir pela intercessão do B. {\redx N.}, vosso Mártir e Pontífice, o efeito do mistério, que celebrámos. Por nosso Senhor \emph{\&c.}
}\end{paracol}

\textit{Outras vezes, em vez, do Precedente, diz-se o seguante:}

\paragraph{Postcomúnio}
\begin{paracol}{2}\latim{
\rlettrine{H}{æc} nos communio, Dómine, purget a crimine: et, intercedénte beáto {\redx N.} Mártyre tuo atque Pontifice, cæléstis remédii fáciat esse consortes. Per Dóminum \emph{\&c.}
}\switchcolumn\portugues{
\qlettrine{Q}{ue} esta comunhão, Senhor, nos purifique de nossos crimes, e que, por intercessão do B. {\redx N.}, vosso Mártir e Pontífice, nos torne participantes do remédio celestial. Por nosso Senhor \emph{\&c.}
}\end{paracol}

\textit{Por um Mártir não Pontífice diz-se o seguinte:}

\paragraph{Postcomúnio}
\begin{paracol}{2}\latim{
\rlettrine{D}{a,} quǽsumus, Dómine, Deus noster: ut, sicut tuorum commemoratione Sanctórum temporali gratulámur officio; ita perpetuo lætémur aspéctu. Per Dóminum \emph{\&c.}
}\switchcolumn\portugues{
\rlettrine{C}{oncedei-nos,} Senhor, Vos suplicamos, que, assim como nos alegramos, celebrando na terra a memória dos vossos santos, assim também tenhamos a felicidade de os contemplar na eternidade. Por nosso Senhor \emph{\&c.}
}\end{paracol}

\textit{Outras vezes, ena vez do Precedente, diz-se o seguinte:}

\paragraph{Postcomúnio}
\begin{paracol}{2}\latim{
\rlettrine{R}{efécti} participatióne múneris sacri, quǽsumus, Dómine, Deus noster: ut, cujus exséquimur cultum; intercedénte beáto {\redx N.} Martyre tuo, sentiámus efféctum. Per Dóminum nostrum \emph{\&c.}
}\switchcolumn\portugues{
\rlettrine{C}{onfortados} com a participação do dom sagrado, Vos suplicamos, ó Senhor, nosso Deus, fazei-nos sentir, por intercessão do B. {\redx N.}, vosso Mártir, o efeito do mistério que celebramos. Por nosso Senhor \emph{\&c.}
}\end{paracol}

\subsectioninfo{Mártires}{Missa Sancti tui}\label{martires}

\paragraphinfo{Intróito}{Sl. 144, 10-11}
\begin{paracol}{2}\latim{
\rlettrine{S}{ancti} tui, Dómine, benedícent te: glóriam regni tui dicent, allelúja, allelúja. \emph{Ps. ibid., 1} Exaltábo te, Deus meus, Rex: et benedícam nómini tuo in sǽculum, et in sǽculum sǽculi.
℣. Gloria Patri \emph{\&c.}
}\switchcolumn\portugues{
\qlettrine{Q}{ue} os vossos Santos Vos bendigam, Senhor: e publiquem a glória do vosso reino. Aleluia, aleluia. \emph{Sl. ibid., 1} Exaltarei a vossa glória, ó Deus, o meu Rei: e abençoarei o vosso Nome agora, sempre e em todos os séculos!
℣. Glória ao Pai \emph{\&c.}
}\end{paracol}

\paragraphinfo{Oração, Secreta e Postcomúnio}{Página \pageref{muitosmartires1}}

\paragraphinfo{Epístola}{1 Pe. 1, 3-7.}
\begin{paracol}{2}\latim{
Léctio Epístolæ beáti Petri.
}\switchcolumn\portugues{
Lição da Ep.ª do B. Ap.º Pedro.
}\switchcolumn*\latim{
\rlettrine{B}{enedíctus} Deus et Pater Dómini nostri Jesu Christi, qui secúndum misericórdiam suam magnam regenerávit nos in spem vivam, per resurrectiónem Jesu Christi ex mórtuis, in hereditátem incorruptíbilem et incontaminátam et immarcescíbilem, conservátam in cœlis in vobis, qui in virtúte Dei custodímini per fidem in salútem, parátam revelári in témpore novíssimo. In quo exsultábitis, módicum nunc si opórtet contristári in váriis tentatiónibus: ut probátio vestræ fídei multo pretiósior auro (quod per ignem probátur) inveniátur in laudem et glóriam et honórem, in revelatióne Jesu Christi, Dómini nostri.
}\switchcolumn\portugues{
\rlettrine{B}{endito} seja Deus, Pai de N. S. Jesus Cristo, que, segundo a grandeza da sua misericórdia, nos regenerou para uma esperança viva pela ressurreição dos mortos de Jesus Cristo, para alcançarmos a herança incorruptível, inalterável e imortal que está reservada nos céus para vós, a quem o poder de Deus guarda pela fé, para vos conceder o gozo da salvação, que será manifestada no fim dos tempos. Alegrai-vos com isto, ainda que devais ser perseguidos algumas vezes com diversas provações, a fim de que a manifestação da vossa fé, mais preciosa que o ouro (que é provado pelo fogo), seja julgada digna de louvor, honra e glória na revelação de N. S. Jesus Cristo.
}\end{paracol}

\begin{paracol}{2}\latim{
Allelúja, allelúja. ℣. Sancti tui, Dómine, florébunt sicut lílium: et sicut odor bálsami erunt ante te. Allelúja. ℣. \emph{Ps. 115, 15} Pretiósa in conspéctu Dómini mors Sanctórum ejus. Allelúja.
}\switchcolumn\portugues{
Aleluia, aleluia. ℣. Vossos santos, Senhor, florescerão, como o lírio, e serão, ante Vós, como o odor do bálsamo. Aleluia. ℣. \emph{Sl. 115, 15} É preciosa diante do Senhor a morte dos seus Santos. Aleluia.
}\end{paracol}

\paragraphinfo{Evangelho}{Jo. 15, 5-11}
\begin{paracol}{2}\latim{
\cruz Sequéntia sancti Evangélii secúndum Joánnem.
}\switchcolumn\portugues{
\cruz Continuação do santo Evangelho segundo S. João.
}\switchcolumn*\latim{
\blettrine{I}{n} illo témpore: Dixit Jesus discípulis suis: Ego sum vitis, vos pálmites: qui manet in me, et ego in eo, hic fert fructum multum: quia sine me nihil potéstis fácere. Si quis in me non mánserit, mittétur foras sicut palmes, et aréscet, et cólligent eum, et in ignem mittent, et ardet. Si manséritis in me, et verba mea in vobis mánserint: quodcúmque voluéritis, petétis, et fiet vobis. In hoc clarificátus est Pater meus, ut fructum plúrimum afferátis, et efficiámini mei discípuli. Sicut diléxit me Pater, et ego diléxi vos. Manéte in dilectióne mea. Si præcépta mea servavéritis, manébitis in dilectióne mea, sicut et ego Patris mei præcépta servávi, et máneo in ejus dilectióne. Hæc locútus sum vobis, ut gáudium meum in vobis sit, et gáudium vestrum impleátur.
}\switchcolumn\portugues{
\blettrine{N}{aquele} tempo, disse Jesus aos seus discípulos: «Eu sou a videira, e vós sois as vides. Aquele que permanece em mim, eu permaneço nele, e dará abundante fruto; pois sem mim nada podereis fazer. Se alguém não permanecer em mim, será arrancado e lançado fora, como uma vide seca. Então secará e levá-la-ão para a lançar no fogo, em que arderá. Se permanecerdes em mim e as minhas palavras permanecerem em vós, tudo o que quiserdes podereis pedir, que vos será concedido. Meu Pai será glorificado, se vós derdes muito fruto e vos tornardes meus discípulos. Assim como meu Pai me amou, assim também eu vos amo. Permanecei no meu amor. Se observardes os meus mandamentos, permanecereis no meu amor, como eu, que guardo os mandamentos de meu Pai, permaneço no seu amor. Digo-Vos estas cousas a fim de que minha alegria permaneça convosco e a vossa alegria seja abundante».
}\end{paracol}

\paragraphinfo{Ofertório}{Sl. 31, 11}
\begin{paracol}{2}\latim{
\rlettrine{L}{ætámini} in Dómino et exsultáte, justi: et gloriámini, omnes recti corde, allelúja, allelúja.
}\switchcolumn\portugues{
\rlettrine{A}{legrai-vos} no Senhor, ó justos! Exultai de júbilo! Todos aqueles que possuem o coração recto serão glorificados. Aleluia, aleluia.
}\end{paracol}

\paragraphinfo{Comúnio}{Sl. 32, 1}
\begin{paracol}{2}\latim{
\rlettrine{G}{audéte,} justi, in Dómino, allelúja: rectos decet collaudátio, allelúja.
}\switchcolumn\portugues{
\rlettrine{A}{legrai-vos} no Senhor, ó justos. Aleluia. É aos que são rectos que pertence cantar os vossos louvores. Aleluia.
}\end{paracol}


\subsection{Comum dos Confessores}

\subsectioninfo{Confessores Pontífices}{Missa Státuit ei Dóminus}\label{confessorespontifices1}

\paragraphinfo{Intróito}{Ecl. 45, 30}
\begin{paracol}{2}\latim{
\rlettrine{S}{tátuit} ei Dóminus testaméntum pacis, et príncipem fecit eum: ut sit illi sacerdótii dígnitas in ætérnum. (T. P. Allelúja, allelúja.) \emph{Ps. 131, 1} Meménto, Dómine, David: et omnis mansuetúdinis ejus.
℣. Gloria Patri \emph{\&c.}
}\switchcolumn\portugues{
\rlettrine{D}{eus} estabeleceu com ele aliança de paz e tornou-o príncipe, para que possuísse eternamente a dignidade sacerdotal. (T. P. Aleluia, aleluia.) \emph{Sl. 131, 1} Lembrai-vos de David, ó Senhor, e da sua grande mansidão.
℣. Glória ao Pai \emph{\&c.}
}\end{paracol}

\paragraph{Oração}
\begin{paracol}{2}\latim{
\rlettrine{D}{a,} quǽsumus, omnípotens Deus: ut beáti {\redx N.} Confessóris tui atque Pontíficis veneránda sollémnitas, et devotiónem nobis áugeat et salútem. Per Dóminum \emph{\&c.}
}\switchcolumn\portugues{
\rlettrine{D}{ignai-Vos} permitir, ó Deus omnipotente, que a veneranda solenidade do vosso Confessor e Pontífice {\redx N.} aumente a nossa piedade e nos assegure a salvação. Por nosso Senhor \emph{\&c.}
}\end{paracol}

\paragraphinfo{Epístola}{Ecl. 44, 16-27; 45, 3-20}
\begin{paracol}{2}\latim{
Léctio libri Sapiéntiæ.
}\switchcolumn\portugues{
Lição do Livro da Sabedoria.
}\switchcolumn*\latim{
\rlettrine{E}{cce} sacérdos magnus, qui in diébus suis plácuit Deo, et invéntus est justus: et in témpore iracúndiæ factus est reconciliátio. Non est invéntus símilis illi, qui conservávit legem Excélsi. Ideo jurejurándo fecit illum Dóminus créscere in plebem suam. Benedictiónem ómnium géntium dedit illi, et testaméntum suum confirmávit super caput ejus. Agnóvit eum in benedictiónibus suis: conservávit illi misericórdiam suam: et invenit grátiam coram óculis Dómini. Magnificávit eum in conspéctu regum: et dedit illi corónam glóriæ. Státuit illi testaméntum ætérnum, et dedit illi sacerdótium magnum: et beatificávit illum in glória. Fungi sacerdótio, et habére laudem in nómine ipsíus, et offérre illi incénsum dignum in odórem suavitátis.
}\switchcolumn\portugues{
\rlettrine{E}{is} o grande sacerdote, que nos dias da sua vida agradou a Deus e foi julgado justo; e no tempo da ira se tornou a reconciliação dos homens. Ninguém o igualou na observância das leis do Altíssimo. Eis porque o Senhor jurou que o tornaria grande no meio do seu povo. O Senhor abençoou nele todos os povos; e com ele ratificou a sua aliança. O Senhor deu-lhe as suas bênçãos e continuou a dispensar-lhe a sua misericórdia, vendo-se bem que este homem achou graça aos olhos do Senhor, que, por isso mesmo, o engrandeceu diante dos reis e lhe deu uma coroa de glória. Estabeleceu com ele uma aliança eterna, elevou-o ao sumo sacerdócio, e tornou-o feliz na glória para exercer o sacerdócio, louvar o seu nome e oferecer-lhe dignamente incenso de odor agradável.
}\end{paracol}

\paragraphinfo{Gradual}{Ecl. 44, 16}
\begin{paracol}{2}\latim{
\rlettrine{E}{cce} sacérdos magnus, qui in diébus suis plácuit Deo. ℣. \emph{ibid., 20} Non st invéntus símilis illi, qui conserváret legem Excélsi.
}\switchcolumn\portugues{
\rlettrine{E}{is} o grande sacerdote que nos dias da sua vida agradou a Deus. ℣. \emph{ibid., 20} Não foi encontrado outrem semelhante a ele na observância das leis do Altíssimo.
}\switchcolumn*\latim{
Allelúja, allelúja. ℣. \emph{Ps. 109, 4} Tu es sacérdos in ætérnum, secúndum órdinem Melchísedech. Allelúja.
}\switchcolumn\portugues{
Aleluia, aleluia. ℣. \emph{Sl. 109, 4} Tu és sacerdote para sempre, segundo a ordem de Melquisedeque. Aleluia.
}\end{paracol}

\textit{Após a Septuagésima omite-se o Aleluia e o seguinte e diz-se:}

\paragraphinfo{Trato}{Sl. 111, 1-3}
\begin{paracol}{2}\latim{
\rlettrine{B}{eátus} vir, qui timet Dóminum: in mandátis ejus cupit nimis. ℣. Potens in terra erit semen ejus: generátio rectórum benedicétur. ℣. Glória et divítiæ in domo ejus: et justítia ejus manet in sǽculum sǽculi.
}\switchcolumn\portugues{
\rlettrine{B}{em-aventurado} o varão que teme o Senhor e cuja vontade é ardente no cumprimento dos seus mandamentos. Sua descendência será poderosa na terra, pois a posteridade dos justos será abençoada. Na sua casa haverá abundância e riqueza, e a sua justiça subsistirá em todos os séculos dos séculos.
}\end{paracol}

\textit{No T. Pascal omite-se Gradual e o Trato e diz-se:}

\begin{paracol}{2}\latim{
Allelúja, allelúja. ℣. \emph{Ps. 109, 4} Tu es sacérdos in ætérnum, secúndum órdinem Melchísedech. Allelúja. ℣. Hic est sacérdos, quem coronávit Dóminus. Allelúja.
}\switchcolumn\portugues{
Aleluia, aleluia. ℣. \emph{Sl. 109, 4} Tu és sacerdote para sempre segundo a ordem de Melquisedeque. Aleluia. ℣. Este é o sacerdote que o Senhor coroou. Aleluia.
}\end{paracol}

\paragraphinfo{Evangelho}{Mt. 25, 14-23}
\begin{paracol}{2}\latim{
\cruz Sequéntia sancti Evangélii secúndum Matthǽum.
}\switchcolumn\portugues{
\cruz Continuação do santo Evangelho segundo S. Mateus.
}\switchcolumn*\latim{
\blettrine{I}{n} illo témpore: Dixit Jesus discípulis suis parábolam hanc: Homo péregre proficíscens vocávit servos suos, et trádidit illis bona sua. Et uni dedit quinque talénta, álii a tem duo, álii vero unum, unicuíque secúndum própriam virtútem, et proféctus est statim. Abiit autem, qui quinque talénta accéperat, et operátus est in eis, et lucrátus est ália quinque. Simíliter et, qui duo accéperat, lucrátus est ália duo. Qui autem unum accéperat, ábiens fodit in terram, et abscóndit pecúniam dómini sui. Post multum vero témporis venit dóminus servórum illórum, et pósuit ratiónem cum eis. Et accédens qui quinque talénta accéperat, óbtulit ália quinque talénta, dicens: Dómine, quinque talénta tradidísti mihi, ecce, ália quinque superlucrátus sum. Ait illi dóminus ejus: Euge, serve bone et fidélis, quia super pauca fuísti fidélis, super multa te constítuam: intra in gáudium dómini tui. Accéssit autem et qui duo talénta accéperat, et ait: Dómine, duo talénta tradidísti mihi, ecce, ália duo lucrátus sum. Ait illi dóminus ejus: Euge, serve bone et fidélis, quia super pauca fuísti fidélis, super multa te constítuam: intra in gáudium dómini tui.
}\switchcolumn\portugues{
\blettrine{N}{aquele} tempo, disse Jesus a seus discípulos esta parábola: Indo um homem viajar para longe, chamou os seus servos e entregou-lhes os bens. A um deu cinco talentos, a outro dous e ao terceiro um. Deu a cada um segundo a sua capacidade; partindo imediatamente. Aquele que havia recebido cinco talentos partiu, e, negociando com este dinheiro, ganhou outros cinco talentos. Semelhantemente, o que recebera dous lucrou outros dous. Mas aquele que havia recebido só um talento foi, cavou a terra e aí ocultou o dinheiro do senhor. Passado muito tempo, veio o senhor daqueles servos e fez contas com eles. Aproximando-se, então, o que recebera cinco talentos, apresentou outros cinco, dizendo: «Senhor, entregastes-me cinco talentos, eis outros cinco, que lucrei». Disse-lhe o seu senhor: «Muito bem, servo bom e fiel: visto que foste fiel em poucas cousas, estabelecer-te-ei acima de muitas cousas: entra no gozo do teu senhor». Aproximando-se também o que recebera dous talentos, disse: «Senhor, entregastes-me dous talentos, eis outros dous que lucrei». E o senhor lhe disse: «Muito bem, servo bom e fiel: visto que foste fiel em poucas cousas, eu te estabelecerei acima de muitas cousas: entra no gozo do teu senhor».
}\end{paracol}

\paragraphinfo{Ofertório}{Sl. 88, 21-22}
\begin{paracol}{2}\latim{
\rlettrine{I}{nvéni} David servum meum, óleo sancto meo unxi eum: manus enim mea auxiliábitur ei, et bráchium meum confortábit eum. (T. P. Allelúja.)
}\switchcolumn\portugues{
\rlettrine{E}{ncontrei} o meu servo David e ungi-o com meu óleo sagrado. Minha mão o socorrerá e o meu braço o fortalecerá. (T. P. Aleluia).
}\end{paracol}

\paragraph{Secreta}
\begin{paracol}{2}\latim{
\rlettrine{S}{ancti} tui, quǽsumus, Dómine, nos ubíque lætíficant: ut, dum eórum mérita recólimus, patrocínia sentiámus. Per Dóminum \emph{\&c.}
}\switchcolumn\portugues{
\qlettrine{Q}{ue} os vossos santos, Senhor, Vos suplicamos, nos alegrem em toda a parte, a fim de que, honrando os seus méritos, sintamos o efeito do seu patrocínio. Por nosso Senhor \emph{\&c.}
}\end{paracol}

\paragraphinfo{Comúnio}{Lc. 12, 42}
\begin{paracol}{2}\latim{
\rlettrine{F}{idélis} servus et prudens, quem constítuit dóminus super famíliam suam: ut det illis in témpore trítici mensúram. (T. P. Allelúja.)
}\switchcolumn\portugues{
\rlettrine{E}{is} o servo fiel e prudente que o Senhor estabeleceu acima da sua família para distribuir, oportunamente, a cada um a sua medida de trigo. (T. P. Aleluia).
}\end{paracol}

\paragraph{Postcomúnio}
\begin{paracol}{2}\latim{
\rlettrine{P}{ræsta,} quǽsumus, omnípotens Deus: ut, de percéptis munéribus grátias exhibéntes, intercedénte beáto {\redx N.} Confessóre tuo atque Pontífice, benefícia potióra sumámus. Per Dóminum \emph{\&c.}
}\switchcolumn\portugues{
\rlettrine{D}{ignai-Vos} permitir, ó Deus omnipotente, que, dando-Vos nós graças pelos benefícios recebidos, alcancemos por intercessão do B. {\redx N.}, vosso Confessor e Pontífice, ainda outros maiores. Por nosso Senhor \emph{\&c.}
}\end{paracol}

\subsectioninfo{Confessores Pontífices}{Missa Sacerdótes tui}\label{confessorespontifices2}

\paragraphinfo{Intróito}{Sl. 131, 9-10}
\begin{paracol}{2}\latim{
\rlettrine{S}{acerdótes} tui, Dómine, índuant justítiam, et sancti tui exsúltent: propter David servum tuum, non avértas fáciem Christi tui. (T. P. Allelúja, allelúja.) \emph{Ps. ibid., 1} Meménto, Dómine, David: et omnis mansuetúdinis ejus.
℣. Gloria Patri \emph{\&c.}
}\switchcolumn\portugues{
\qlettrine{Q}{ue} os vossos sacerdotes, Senhor, se revistam de santidade; e que os vossos santos exultem de alegria! Por amor do vosso servo David não afasteis os olhos do vosso Cristo. (T. Aleluia, aleluia.) \emph{Ps. ibid., 1} Lembrai-Vos, Senhor, de David e da sua grande mansidão.
℣. Glória ao Pai \emph{\&c.}
}\end{paracol}

\paragraph{Oração}
\begin{paracol}{2}\latim{
\rlettrine{E}{xáudi,} quǽsumus, Dómine, preces nostras, quas in beáti {\redx N.} Confessóris tui atque Pontíficis sollemnitáte deférimus: et, qui tibi digne méruit famulári, ejus intercedéntibus méritis, ab ómnibus nos absólve peccátis Per Dóminum \emph{\&c.}
}\switchcolumn\portugues{
\rlettrine{O}{uvi,} Senhor, Vos suplicamos, as preces que Vos dirigimos na solenidade do B. {\redx N.} vosso Confessor e Pontífice, e, pelos méritos e intercessão daquele que tão dignamente Vos serviu, concedei-nos o perdão dos nossos pecados. Por nosso Senhor \emph{\&c.}
}\end{paracol}

\paragraphinfo{Epístola}{Heb. 7, 23-27}
\begin{paracol}{2}\latim{
Léctio Epístolæ beáti Pauli Apóstoli ad Hebrǽos.
}\switchcolumn\portugues{
Lição da Ep.ª do B. Ap.º Paulo aos Hebreus.
}\switchcolumn*\latim{
\rlettrine{F}{ratres:} Plures facti sunt sacerdótes, idcírco quod morte prohiberéntur permanére: Jesus autem, eo quod máneat in ætérnum, sempitérnum habet sacerdótium. Unde et salváre in perpétuum potest accedéntes per semetípsum ad Deum: semper vivens ad interpellándum pro nobis. Talis enim decébat, ut nobis esset póntifex, sanctus, ínnocens, impollútus, segregátus a peccatóribus, et excélsior cœlis factus: qui non habet necessitátem cotídie, quemádmodum sacerdótes, prius pro suis delíctis hóstias offérre, deínde pro pópuli: hoc enim fecit semel, seípsum offeréndo, Jesus Christus, Dóminus noster.
}\switchcolumn\portugues{
\rlettrine{M}{eus} irmãos: Entre eles, além disso, muitos outros foram feitos sacerdotes, porque a morte os impedia de viverem sempre. Mas Jesus, que permanece eternamente, tem um sacerdócio eterno, por isso Ele pode salvar perpetuamente aqueles que se aproximam de Deus por seu intermédio, pois está sempre vivo para interceder por nós. Convinha, pois, que tivéssemos um Pontífice santo, inocente, imaculado, afastado dos pecadores e mais elevado que os céus; que não tivesse necessidade, como os outros sacerdotes, de oferecer, quotidianamente, vítimas, primeiro pelos seus próprios pecados e depois pelos pecados do povo; pois isto N. S. Jesus Cristo fez uma vez, oferecendo-se a Si mesmo.
}\end{paracol}

\paragraphinfo{Gradual}{Sl. 131, 16-17}
\begin{paracol}{2}\latim{
\rlettrine{S}{acerdótes} ejus índuam salutári: et sancti ejus exsultatióne exsultábunt. ℣. Illuc prodúcam cornu David: parávi lucérnam Christo meo.
}\switchcolumn\portugues{
\rlettrine{R}{evestirei} os seus sacerdotes de salvação e os seus santos exultarão em transportes de alegria. ℣. Em Sião farei aparecer o poder de David: prepararei uma lâmpada ao meu Cristo.
}\switchcolumn*\latim{
Allelúja, allelúja. ℣. \emph{Ps. 109, 4} Jurávit Dóminus, et non pœnitébit eum: Tu es sacérdos in ætérnum, secúndum órdinem Melchísedech. Allelúja.
}\switchcolumn\portugues{
Aleluia, aleluia. ℣. \emph{Sl. 109, 4} O Senhor jurou e não se arrependerá: Tu és sacerdote para sempre, segundo a ordem de Melquisedeque. Aleluia.
}\end{paracol}

\textit{Após a Septuagésima omite-se o Aleluia e o seguinte e diz-se:}

\paragraphinfo{Trato}{Sl. 111, 1-3}
\begin{paracol}{2}\latim{
\rlettrine{B}{eátus} vir, qui timet Dóminum: in mandátis ejus cupit nimis. ℣. Potens in terra erit semen ejus: generátio rectórum benedicétur. ℣. Glória et divítiæ in domo ejus: et justítia ejus manet in sǽculum sǽculi.
}\switchcolumn\portugues{
\rlettrine{B}{em-aventurado} o varão que teme o Senhor e que põe todo seu zelo em obedecer-Lhe. ℣. Sua descendência será poderosa na terra, pois a geração dos justos será abençoada. ℣. Na sua casa haverá glória e riqueza, e a sua justiça subsistirá em todos os séculos.
}\end{paracol}

\textit{No T. Pascal omite-se o Gradual e o Trato e diz-se:}

\begin{paracol}{2}\latim{
Allelúja, allelúja. ℣. \emph{Ps. 109, 4} Jurávit Dóminus, et non pœnitébit eum: Tu es sacérdos in ætérnum, secúndum órdinem Melchísedech. Allelúja. ℣. \emph{Eccli. 45, 9} Amávit eum Dóminus, et ornávit eum: stolam glóriæ índuit eum. Allelúja.
}\switchcolumn\portugues{
Aleluia, aleluia. ℣. \emph{Sl. 109, 4} O Senhor jurou e não se arrependerá: Tu és sacerdote para sempre, segundo a ordem de Melquisedeque. Aleluia. ℣. \emph{Ecl. 45, 9} O Senhor amou-o, ornou-o e revestiu-o com a túnica da glória. Aleluia.
}\end{paracol}

\paragraphinfo{Evangelho}{Mt. 24, 42-47}
\begin{paracol}{2}\latim{
\cruz Sequéntia sancti Evangélii secúndum Matthǽum.
}\switchcolumn\portugues{
\cruz Continuação do santo Evangelho segundo S. Mateus.
}\switchcolumn*\latim{
\blettrine{I}{n} illo témpore: Dixit Jesus discípulis suis: Vigilate, quia nescítis, qua hora Dóminus vester ventúrus sit. Illud autem scitóte, quóniam, si sciret paterfamílias, qua hora fur ventúrus esset, vigiláret útique, et non síneret pérfodi domum suam. Ideo et vos estóte parati: quia qua nescítis hora Fílius hóminis ventúrus est. Quis, putas, est fidélis servus et prudens, quem constítuit dóminus suus super famíliam suam, ut det illis cibum in témpore? Beátus ille servus, quem, cum vénerit dóminus ejus, invénerit sic faciéntem. Amen, dico vobis, quóniam super ómnia bona sua constítuet eum.
}\switchcolumn\portugues{
\blettrine{N}{aquele} tempo, disse Jesus a seus discípulos: «Vigiai, porque não sabeis a que hora virá o vosso Senhor. Pois sabei que, se o pai de família conhecesse a que horas viria o ladrão, certamente velaria e não deixaria violar a sua casa. Portanto vós, também, estai preparados, porque o Filho do homem virá durante a hora em que não pensais. Qual é, segundo a vossa opinião, o servo fiel e prudente que o Senhor estabeleceu como superior na sua família para distribuir-lhe o sustento em tempo oportuno? Bem-aventurado aquele servo a quem, quando o seu senhor vier, o achar assim ocupado. Em verdade vos digo que o encarregará de administrar todos seus bens».
}\end{paracol}

\paragraphinfo{Ofertório}{Sl. 88, 25}
\begin{paracol}{2}\latim{
\rlettrine{V}{éritas} mea et misericórdia mea cum ipso: et in nómine meo exaltábitur cornu ejus. (T. P. Allelúja.)
}\switchcolumn\portugues{
\rlettrine{A}{} minha fidelidade e a minha misericórdia estarão com ele: e o seu poder elevar-se-á pelo meu nome. (T. P. Aleluia).
}\end{paracol}

\paragraph{Secreta}
\begin{paracol}{2}\latim{
\rlettrine{S}{ancti} {\redx N.} Confessóris tui atque Pontíficis, quǽsumus, Dómine, ánnua sollémnitas pietáti tuæ nos reddat accéptos: ut, per hæc piæ placatiónis offícia, et illum beáta retribútio comitétur, et nobis grátiæ tuæ dona concíliet. Per Dóminum \emph{\&c.}
}\switchcolumn\portugues{
\qlettrine{Q}{ue} a festa anual do vosso santo Confessor e Pontífice {\redx N.} nos torne agradáveis à vossa bondade, Vos suplicamos, Senhor, a fim de que a piedosa oferta desta vítima de expiação lhe aumente a felicidade, que goza como recompensa, e nos obtenha os dons da vossa graça. Por nosso Senhor \emph{\&c.}
}\end{paracol}

\paragraphinfo{Comúnio}{Mt. 24,46-47}
\begin{paracol}{2}\latim{
\rlettrine{B}{eátus} servus, quem, cum vénerit dóminus, invénerit vigilántem: amen, dico vobis, super ómnia bona sua constítuet eum. (T. P. Allelúja.)
}\switchcolumn\portugues{
\rlettrine{B}{em-aventurado} o servo que, quando o seu senhor vier, o encontrar vigilante. Em verdade vos digo que o encarregará de administrar todos seus bens. (T. P. Aleluia).
}\end{paracol}

\paragraph{Postcomúnio}
\begin{paracol}{2}\latim{
\rlettrine{D}{eus,} fidélium remunerátor animárum: præsta; ut beáti {\redx N.} Confessóris tui atque Pontíficis, cujus venerándam celebrámus festivitátem, précibus indulgéntiam consequámur. Per Dóminum \emph{\&c.}
}\switchcolumn\portugues{
\slettrine{Ó}{} Deus, remunerador das almas fiéis, dignai-Vos permitir que pelas orações do B. Pontífice e Confessor {\redx N.}, cuja veneranda festa celebramos, obtenhamos o perdão dos nossos pecados. Por nosso Senhor \emph{\&c.}
}\end{paracol}


\subsectioninfo{Doutores}{Missa In médio Ecclésiae}\label{doutores}

\paragraphinfo{Intróito}{Ecl. 15, 5}
\begin{paracol}{2}\latim{
\rlettrine{I}{n} médio Ecclésiæ apéruit os ejus: et implévit eum Dóminus spíritu sapiéntiæ et intelléctus: stolam glóriæ índuit eum. (T. P. Allelúja, allelúja.) \emph{Ps. 91, 2} Bonum est confitéri Dómino: et psállere nómini tuo, Altíssime.
℣. Gloria Patri \emph{\&c.}
}\switchcolumn\portugues{
\rlettrine{A}{briu-lhe} o Senhor a boca no meio da Igreja, encheu-o com o espírito da sabedoria e da inteligência e revestiu-o com a túnica da glória. (T. P. Aleluia, aleluia.) \emph{Sl. 91, 2} É bom louvar o Senhor: e cantar hinos em honra do vosso nome, ó Altíssimo!
℣. Glória ao Pai \emph{\&c.}
}\end{paracol}

\paragraph{Oração}
\begin{paracol}{2}\latim{
\rlettrine{D}{eus,} qui pópulo tuo ætérnæ salútis beátum {\redx N.} minístrum tribuísti: præsta, quǽsumus; ut, quem Doctórem vitæ habúimus in terris, intercessórem habére mereámur in cœlis. Per Dóminum \emph{\&c.}
}\switchcolumn\portugues{
\slettrine{Ó}{} Deus, que ao vosso povo destinastes o B. {\redx N.} para ministro da eterna salvação, concedei-nos, Vos suplicamos, que, assim como o tivemos como Doutor durante a nossa vida na terra, assim gozemos a sua intercessão no céu. Por nosso Senhor \emph{\&c.}
}\end{paracol}

\paragraphinfo{Epístola}{2. Tm. 4, 1-8}
\begin{paracol}{2}\latim{
Léctio Epístolæ beáti Pauli Apóstoli ad Timotheum.
}\switchcolumn\portugues{
Lição da Ep.ª do B. Ap.º Paulo a Timóteo.
}\switchcolumn*\latim{
\rlettrine{C}{aríssime:} Testíficor coram Deo, et Jesu Christo, qui judicatúrus est vi vos et mórtuos, per advéntum ipsíus et regnum ejus: prǽdica verbum, insta opportúne, importune: árgue, óbsecra, íncrepa in omni patiéntia, et doctrína. Erit enim tempus, cum sanam doctrínam non sustinébunt, sed ad sua desidéria, coacervábunt sibi magistros, pruriéntes áuribus, et a veritáte quidem audítum avértent, ad fábulas autem converténtur. Tu vero vígila, in ómnibus labóra, opus fac Evangelístæ, ministérium tuum ímpie. Sóbrius esto. Ego enim jam delíbor, et tempus resolutiónis meæ instat. Bonum certámen certávi, cursum consummávi, fidem servávi. In réliquo repósita est mihi coróna justítiæ, quam reddet mihi Dóminus in illa die, justus judex: non solum autem mihi, sed et iis, qui díligunt advéntum ejus.
}\switchcolumn\portugues{
\rlettrine{C}{aríssimo:} Conjuro-te diante de Deus e de Jesus Cristo, que há-de julgar vivos e mortos na sua vinda e no seu reino, a que pregues a palavra; instes oportuna e inoportunamente; repreendas; supliques; e ameaces com toda a paciência e doutrina; pois virá tempo em que não suportarão a sã doutrina, mas, indo ao sabor dos seus desejos, procurarão para si muitos mestres, que lhes preguem o que os ouvidos gostam de escutar, e fechem os ouvidos à verdade, para os abrirem às fábulas. Tu, porém, vigia, trabalha em tudo, cumpre o ministério de evangelista e desempenha o teu ministério. Sê sóbrio. Pois quanto a mim sou como uma vítima já aspergida para o sacrifício. O tempo da minha morte já se aproxima. Pelejei o bom combate; acabei a vida; permaneci na fé. Não me falta mais do que esperar a coroa da justiça, que me está reservada, a qual o Senhor, como justo juiz, me dará no grande dia: e não somente a mim, mas também àqueles que amam a sua vinda.
}\end{paracol}

\paragraphinfo{Gradual}{Sl. 36, 30-31}
\begin{paracol}{2}\latim{
\rlettrine{O}{s} justi meditábitur sapiéntiam, et lingua ejus loquétur judícium. ℣. Lex Dei ejus in corde ipsíus: et non supplantabúntur gressus ejus.
}\switchcolumn\portugues{
\rlettrine{A}{} boca do justo falará com sabedoria e a sua língua proclamará a justiça. ℣. A lei do seu Deus está sempre no seu coração e os seus pés não tropeçarão.
}\switchcolumn*\latim{
Allelúja, allelúja. ℣. \emph{Eccli. 45, 9} Amávit eum Dóminus, et ornávit eum: stolam glóriæ índuit eum. Allelúja.
}\switchcolumn\portugues{
Aleluia, aleluia. ℣. \emph{Ecl. 45, 9} Amou-o o Senhor e revestiu-o com a túnica da glória. Aleluia.
}\end{paracol}

\textit{Após a Septuagésima omite-se o Aleluia e o seguinte e diz-se:}

\paragraphinfo{Trato}{Sl. 111, 1-3}
\begin{paracol}{2}\latim{
\rlettrine{B}{eátus} vir, qui timet Dóminum: in mandátis ejus cupit nimis. ℣. Potens in terra erit semen ejus: generátio rectórum benedicétur. ℣. Glória et divítiæ in domo ejus: et justítia ejus manet in sǽculum sǽculi.
}\switchcolumn\portugues{
\rlettrine{B}{em-aventurado} o varão que teme o Senhor e que emprega todo o zelo em obedecer-Lhe. ℣. Sua descendência será poderosa na terra, pois a geração dos justos será abençoada. ℣. Na sua casa haverá glória e riqueza: e a sua justiça subsistirá em todos os séculos dos séculos.
}\end{paracol}

\textit{No T. Pascal omite-se o Gradual e o Trato e diz-se:}

\begin{paracol}{2}\latim{
Allelúja, allelúja. ℣. \emph{Eccli. 45, 9} Amávit eum Dóminus, et ornávit eum: stolam glóriæ índuit eum. Allelúja. ℣. \emph{Osee 14, 6} Justus germinábit sicut lílium: et florébit in ætérnum ante Dóminum. Allelúja.
}\switchcolumn\portugues{
Aleluia, aleluia. ℣. \emph{Ecl. 45, 9} Amou-o o Senhor, ornou-o e revestiu-o com a túnica da glória. Aleluia. ℣. \emph{Os. 14, 6} O justo germinará, como o lírio, e florescerá para sempre diante do Senhor. Aleluia.
}\end{paracol}

\paragraphinfo{Evangelho}{Mt. 5, 13-19}
\begin{paracol}{2}\latim{
\cruz Sequéntia sancti Evangélii secúndum Matthǽum.
}\switchcolumn\portugues{
\cruz Continuação do santo Evangelho segundo S. Mateus.
}\switchcolumn*\latim{
\blettrine{I}{n} illo témpore: Dixit Jesus discípulis suis: Vos estis sal terræ. Quod si sal evanúerit, in quo saliétur? Ad níhilum valet ultra, nisi ut mittátur foras, et conculcétur ab homínibus. Vos estis lux mundi. Non potest cívitas abscóndi supra montem pósita. Neque accéndunt lucérnam, et ponunt eam sub módio, sed super candelábrum, ut lúceat ómnibus qui in domo sunt. Sic lúceat lux vestra coram homínibus, ut vídeant ópera vestra bona, et gloríficent Patrem vestrum, qui in cœlis est. Nolíte putáre, quóniam veni sólvere legem aut prophétas: non veni sólvere, sed adimplére. Amen, quippe dico vobis, donec tránseat cœlum et terra, jota unum aut unus apex non præteríbit a lege, donec ómnia fiant. Qui ergo solvent unum de mandátis istis mínimis, et docúerit sic hómines, mínimus vocábitur in regno cœlórum: qui autem fécerit et docúerit, hic magnus vocábitur in regno cœlórum.
}\switchcolumn\portugues{
\blettrine{N}{aquele} tempo, disse Jesus a seus discípulos: «Sois o sal da terra. Se o sal perde a força, com que salgará? Para nada mais presta, senão para se lançar fora e ser pisado pelos homens. Sois a luz do mundo. Uma cidade situada no cimo de um monte não pode ficar escondida. Nem se acende uma luz para a meter debaixo do alqueire, mas para a colocar no candeeiro, a fim de alumiar todos os que estão em casa. Assim resplandeça a vossa luz diante dos homens, para que vejam as vossas boas obras e glorifiquem vosso Pai, que está nos céus. Não penseis que vim abrogar a Lei ou os Profetas: não vim abrogar, mas aperfeiçoar; porque em verdade vos digo: até que passe o céu e a terra, nem um iota, nem um til se omitirá na Lei. Quem quer, pois, que transgrida, ainda um dos mais pequenos mandamentos, ou ensine os homens a violá-los, será chamado o menor no reino dos céus. Porém, quem os cumprir e ensinar será chamado grande no reino dos céus.
}\end{paracol}

\paragraphinfo{Ofertório}{Sl. 91, 13}
\begin{paracol}{2}\latim{
\qlettrine{J}{ustus} ut palma florébit: sicut cedrus, quæ in Líbano est multiplicábi
tur. (T. P. Allelúja.)
}\switchcolumn\portugues{
\rlettrine{O}{} justo florescerá, como a palmeira, e crescerá, como o cedro do Líbano. (T. P. Aleluia.)
}\end{paracol}

\paragraph{Secreta}
\begin{paracol}{2}\latim{
\rlettrine{S}{ancti} {\redx N.} Pontíficis tui at que Doctóris nobis, Dómine, pia non desit orátio: quæ et múnera nostra concíliet; et tuam nobis indulgéntiam semper obtíneat. Per Dóminum \emph{\&c.}
}\switchcolumn\portugues{
\rlettrine{S}{enhor,} que a piedosa oração de Santo {\redx N.} vosso Pontífice e Doutor, nos não abandone, e que por ela as nossas ofertas Vos sejam agradáveis e alcancemos benigna e continuamente a vossa misericórdia. Por nosso Senhor \emph{\&c.}
}\end{paracol}

\paragraphinfo{Comúnio}{Lc. 12, 42}
\begin{paracol}{2}\latim{
\rlettrine{F}{idélis} servus et prudens, quem constítuit dóminus super famíliam suam: ut det illis in témpore trítici mensúram. (T. P. Allelúja.)
}\switchcolumn\portugues{
\rlettrine{O}{} servo fiel e prudente é destinado pelo Senhor para distribuir, oportunamente, na sua família a cada um a sua medida de trigo. (T. P. Aleluia.)
}\end{paracol}

\paragraph{Postcomúnio}
\begin{paracol}{2}\latim{
\rlettrine{U}{t} nobis, Dómine, tua sacrifícia dent salútem: beátus {\redx N.} Póntifex tuus et Doctor egrégius, quǽsumus, precátor accédat. Per Dóminum nostrum \emph{\&c.}
}\switchcolumn\portugues{
\slettrine{Ó}{} Senhor, dignai-Vos conceder-nos que o B. {\redx N.}, vosso Pontífice e ilustre Doutor, seja nosso intercessor perante Vós, a fim de que este sacrifício nos alcance a salvação. Por nosso Senhor \emph{\&c.}
}\end{paracol}

\textit{Outra Epístola (para certos dias):}\label{doutoresepistola2}

\paragraphinfo{Epístola}{Ecl. 39, 6-14}
\begin{paracol}{2}\latim{
Léctio libri Sapiéntiæ.
}\switchcolumn\portugues{
Lição do Livro da Sabedoria.
}\switchcolumn*\latim{
\qlettrine{J}{ustus} cor suum tradet ad vigilándum dilúculo ad Dóminum, qui fecit illum, et in conspéctu Altíssimi deprecábitur. Apériet os suum in oratióne, et pro delíctis suis deprecábitur. Si enim Dóminus magnus volúerit, spíritu intellegéntias replébit illum: et ipse tamquam imbres mittet elóquia sapiéntiæ suæ, et in oratióne confitébitur Dómino: et ipse díriget consílium ejus et disciplínam, et in abscónditis suis consiliábitur. Ipse palam fáciet disciplínam doctrínæ suæ, et in lege testaménti Dómini gloriábitur. Collaudábunt multi sapiéntiam ejus, et usque in sǽculum non delébitur. Non recédet memória ejus, et nomen ejus requirétur a generatióne in generatiónem. Sapiéntiam ejus enarrábunt gentes, et laudem ejus enuntiábit ecclésia.
}\switchcolumn\portugues{
\rlettrine{O}{} justo aplicará o seu coração e vigiará desde o romper do dia para se unir ao Senhor, que o criou, e oferecer as suas preces ao Altíssimo. Abrirá a sua boca para orar e implorar o perdão dos seus pecados; pois, se o soberano Senhor quiser, enchê-lo-á com o espírito de inteligência. Então ele espalhará, como chuva, as palavras da sua sabedoria e abençoará o Senhor na sua oração. O Senhor inspirará os seus conselhos e instruções; e ele compreenderá os mystérios divinos. Publicará a doutrina que tiver aprendido, e a sua glória será manter-se na lei da aliança com o Senhor. Sua sabedoria receberá louvor de muitos e não cairá no esquecimento. Sua memória se não apagará. Seu nome será honrado de geração em geração. As nações publicarão a sua sabedoria e a Igreja anunciará os seus louvores.
}\end{paracol}

\subsectioninfo{Confessores não Pontífices}{Missa Os justi}\label{confessoresnaopontifices1}

\paragraphinfo{Intróito}{Sl. 36, 30-31}
\begin{paracol}{2}\latim{
\rlettrine{O}{s} justi meditábitur sapiéntiam, et lingua ejus loquétur judícium: lex Dei ejus in corde ipsíus. (T. P. Allelúja, allelúja.) \emph{Ps. ibid., 1} Noli æmulári in malignántibus: neque zeláveris faciéntes iniquitátem.
℣. Gloria Patri \emph{\&c.}
}\switchcolumn\portugues{
\rlettrine{A}{} boca do justo fala com sabedoria e a sua língua proclama a justiça. A lei do seu Deus estará no seu coração. (T. P. Aleluia, aleluia.) \emph{Sl. ibid., 1} Não vos irriteis contra os maus, nem tenhais inveja daqueles que cometem iniquidades.
℣. Glória ao Pai \emph{\&c.}
}\end{paracol}

\paragraph{Oração}
\begin{paracol}{2}\latim{
\rlettrine{D}{eus,} qui nos beáti {\redx N.} Confessóris tui ánnua solemnitáte lætíficas: concéde propítius; ut, cujus natalítia cólimus, étiam actiónes imitémur. Per Dóminum \emph{\&c.}
}\switchcolumn\portugues{
\slettrine{Ó}{} Deus, que nos alegrais com a solenidade anual do B. {\redx N.}, vosso Confessor, visto que celebramos o seu nascimento, concedei-nos propício que imitemos também as suas acções. Por nosso Senhor \emph{\&c.}
}\end{paracol}

\paragraphinfo{Epístola}{Ecl. 31, 8-11}
\begin{paracol}{2}\latim{
Léctio libri Sapiéntiæ.
}\switchcolumn\portugues{
Lição do Livro da Sabedoria.
}\switchcolumn*\latim{
\rlettrine{B}{eátus} vir, qui invéntus est sine mácula, et qui post aurum non ábiit, nec sperávit in pecúnia et thesáuris. Quis est hic, et laudábimus eum? fecit enim mirabília in vita sua. Qui probátus est in illo, et perféctus est, erit illi glória ætérna: qui potuit tránsgredi, et non est transgréssus: fácere mala, et non fecit: ídeo stabilíta sunt bona illíus in Dómino, et eleemósynis illíus enarrábit omnis ecclésia sanctórum.
}\switchcolumn\portugues{
\rlettrine{B}{em-aventurado} o homem que foi julgado sem mácula; que não correu após o ouro; e não pôs as esperanças nem no dinheiro, nem nos tesouros! Quem é ele, para que o louvemos? Porquanto operou coisas admiráveis durante a vida. Aquele que foi provado pelo ouro e foi julgado perfeito alcançará a glória eterna; pois poderia ter violado o mandamento, e o não violou; poderia ter praticado acções más, e as não praticou. É por esta razão que seus bens lhe estarão assegurados no Senhor e que toda a assembleia dos justos proclamará as boas acções que praticou.
}\end{paracol}

\paragraphinfo{Gradual}{Sl. 91, 13 \& 14}
\begin{paracol}{2}\latim{
\qlettrine{J}{ustus} ut palma florébit: sicut cedrus Líbani multiplicábitur in domo Dómini. ℣. \emph{ibid., 3} Annuntiándum mane misericórdiam tuam, et veritátem tuam per noctem.
}\switchcolumn\portugues{
\rlettrine{O}{} justo florescerá, como a palmeira, e crescerá, como o cedro do Líbano, na casa do Senhor. ℣. \emph{ibid., 3} Para publicar de manhã a vossa misericórdia; e de noite a vossa verdade.
}\switchcolumn*\latim{
Allelúja, allelúja. ℣. \emph{Jac. 1, 12} Beátus vir, qui suffert tentatiónem: quóniam, cum probátus fúerit, accípiet corónam vitæ. Allelúja.
}\switchcolumn\portugues{
Aleluia, aleluia. ℣. \emph{Tg. 1, 12} Bem-aventurado o varão que sabe sofrer a tentação, porque, quando acabar a tentação, receberá a coroa da vida. Aleluia.
}\end{paracol}

\textit{Após a Septuagésima omite-se o Aleluia e o seguinte e diz-se:}

\paragraphinfo{Trato}{Sl. 111, 1-3}
\begin{paracol}{2}\latim{
\rlettrine{B}{eátus} vir, qui timet Dóminum: in mandátis ejus cupit nimis. ℣. Potens in terra erit semen ejus: generátio rectórum benedicétur. ℣. Glória et divitiæ in domo ejus: et justítia ejus manet in sǽculum sǽculi.
}\switchcolumn\portugues{
\rlettrine{B}{em-aventurado} o varão que teme o Senhor e que põe todo seu zelo em obedecer-Lhe. ℣. Sua descendência será poderosa na terra; pois a geração dos justos será abençoada. ℣. Na sua casa haverá glória e riqueza: e a justiça subsistirá em todos os séculos dos séculos.
}\end{paracol}

\textit{No T. Pascal o Gradual e o Trato e diz-se:}

\begin{paracol}{2}\latim{
Allelúja, allelúja. ℣. \emph{Jac. 1, 12} Beátus vir, qui suffert tentatiónem: quóniam, cum probátus fúerit, accípiet corónam vitæ. Allelúja. ℣. \emph{Eccli. 45, 9} Amávit eum Dóminus et ornávit eum: stolam glóriæ índuit eum. Allelúja.
}\switchcolumn\portugues{
Aleluia, aleluia. \emph{Tg. 1, 12} Bem-aventurado o varão que sabe sofrer a tentação, porque, quando acabar a tentação, receberá a coroa da vida. Aleluia. ℣. \emph{Ecl. 45, 9} O Senhor amou-o, ornou-o e revestiu-o com a túnica da glória. Aleluia.
}\end{paracol}

\paragraphinfo{Evangelho}{Lc. 12, 35-40}\label{evangelhoconfessoresnaopontifices1}
\begin{paracol}{2}\latim{
\cruz Sequéntia sancti Evangélii secúndum Lucam.
}\switchcolumn\portugues{
\cruz Continuação do santo Evangelho segundo S. Lucas.
}\switchcolumn*\latim{
\blettrine{I}{n} illo témpore: Dixit Jesus discípulis suis: Sint lumbi vestri præcíncti, et lucernæ ardéntes in mánibus vestris, et vos símiles homínibus exspectántibus dóminum suum, quando revertátur a núptiis: ut, cum vénerit et pulsáverit, conféstim apériant ei. Beáti servi illi, quos, cum vénerit dóminus, invénerit vigilántes: amen, dico vobis, quod præcínget se, et fáciet illos discúmbere, et tránsiens ministrábit illis. Et si vénerit in secúnda vigília, et si in tértia vigília vénerit, et ita invénerit, beáti sunt servi illi. Hoc autem scitóte, quóniam, si sciret paterfamílias, qua hora fur veníret, vigiláret útique, et non síneret pérfodi domum suam. Et vos estóte paráti, quia, qua hora non putátis, Fílius hóminis véniet.
}\switchcolumn\portugues{
\blettrine{N}{aquele} tempo, disse Jesus aos seus discípulos: «Estejam cingidos os vossos rins; tende nas vossas mãos lâmpadas acesas; sede semelhantes a homens que esperam o seu senhor quando volta das bodas, para que, quando chegar e bater à porta, logo lha abram. Bem-aventurados aqueles servos que seu senhor, quando chegar, os achar vigilantes. Em verdade vos digo que se cingirá, e, mandando-os sentar à mesa, passará por entre eles e os servirá. E, se chegar na segunda ou terceira vigília e assim os achar, bem-aventurados são esses servos. Ora, sabei que, se o pai de família conhecesse a hora em que viria o ladrão, vigiava, sem dúvida, para que sua casa não fosse assaltada. Vós, pois, estai preparados, porque o Filho do homem virá à hora em que menos cuidais».
}\end{paracol}

\paragraphinfo{Ofertório}{Sl. 88, 25.}
\begin{paracol}{2}\latim{
\rlettrine{V}{éritas} mea et misericórdia mea cum ipso: et in nómine meo exaltábitur cornu ejus. (T. P. Allelúja.)
}\switchcolumn\portugues{
\rlettrine{A}{} minha verdade e a minha misericórdia estarão com ele, e, por virtude do meu nome, será exaltado o seu poder. (T. P. Aleluia.)
}\end{paracol}

\paragraph{Secreta}
\begin{paracol}{2}\latim{
\rlettrine{L}{audis} tibi, Dómine, hóstias immolámus in tuórum commemoratióne Sanctórum: quibus nos et præséntibus éxui malis confídimus et futúris. Per Dóminum \emph{\&c.}
}\switchcolumn\portugues{
\rlettrine{S}{enhor,} Vos oferecemos este sacrifício de louvor em memória dos vossos Santos, para que por meio dele nos livremos dos males presentes e futuros. Por nosso Senhor \emph{\&c.}
}\end{paracol}

\paragraphinfo{Comúnio}{Mt. 24, 46-47}
\begin{paracol}{2}\latim{
\rlettrine{B}{eátus} servus, quem, cum vénerit dóminus, invénerit vigilántem: amen, dico vobis, super ómnia bona sua constítuet eum. (T. P. Allelúja.)
}\switchcolumn\portugues{
\rlettrine{B}{em-aventurado} o servo que o Senhor, quando vier, achar vigilante. Em verdade vos digo que lhe dará a administração de todos seus bens. (T. P. Aleluia).
}\end{paracol}

\paragraph{Postcomúnio}
\begin{paracol}{2}\latim{
\rlettrine{R}{efécti} cibo potúque cœlésti, Deus noster, te súpplices exorámus: ut, in cujus hæc commemoratióne percépimus, ejus muniámur et précibus. Per Dóminum \emph{\&c.}
}\switchcolumn\portugues{
\rlettrine{F}{ortalecidos} com o alimento e com a bebida celestiais, Vos suplicamos humildemente, ó nosso Deus, que nos protejam as preces daquele em cuja memória os recebemos. Por nosso Senhor \emph{\&c.}
}\end{paracol}

\subsectioninfo{Confessores não Pontífices}{Missa Justus ut palma}\label{confessoresnaopontifices2}

\paragraphinfo{Intróito}{Sl. 91, 13-14}
\begin{paracol}{2}\latim{
\qlettrine{J}{ustus} ut palma florébit: sicut cedrus Líbani multiplicábitur: plantátus in domo Dómini: in átriis domus Dei nostri. (T. P. Allelúja, allelúja.) \emph{Ps. ibid., 2} Bonum est confitéri Dómino: et psállere nómini tuo, Altíssime.
℣. Gloria Patri \emph{\&c.}
}\switchcolumn\portugues{
\rlettrine{O}{} justo florescerá, como a palmeira, e multiplicar-se-á, como o cedro do Líbano, plantado na casa do Senhor e nos átrios da casa do nosso Deus. (T. P. Aleluia, aleluia.) \emph{Sl. ibid., 2} É bom louvar o Senhor: e cantar hinos em honra do vosso nome, ó Altíssimo!
℣. Glória ao Pai \emph{\&c.}
}\end{paracol}

\paragraph{Oração}
\begin{paracol}{2}\latim{
\rlettrine{A}{désto,} Dómine, supplicatiónibus nostris, quas in beáti {\redx N.} Confessóris tui sollemnitáte deférimus: ut, qui nostræ justítiæ fidúciam non habémus, ejus, qui tibi plácuit, précibus adjuvémur. Per Dóminum \emph{\&c.}
}\switchcolumn\portugues{
\rlettrine{O}{uvi} benigno, Senhor, as súplicas que Vos apresentamos na solenidade do vosso B. Confessor {\redx N.}, a fim de que, já que não podemos ter confiança na nossa justiça, sejamos auxiliados pelas preces daquele que Vos foi agradável neste mundo. Por nosso Senhor \emph{\&c.}
}\end{paracol}

\paragraphinfo{Epístola}{1. Cor. 4, 9-14}
\begin{paracol}{2}\latim{
Léctio Epístolæ beáti Pauli Apóstoli ad Corínthios.
}\switchcolumn\portugues{
Lição da Ep.ª do B. Ap.º Paulo aos Coríntios.
}\switchcolumn*\latim{
\rlettrine{F}{ratres:} Spectáculum facti sumus mundo et Angelis et homínibus. Nos stulti propter Christum, vos autem prudéntes in Christo: nos infírmi, vos autem fortes: vos nóbiles, nos autem ignóbiles. Usque in hanc horam et esurímus, et sitímus, et nudi sumus, et cólaphis cǽdimur, et instábiles sumus, et laborámus operántes mánibus nostris: maledícimur, et benedícimus: persecutiónem pátimur, et sustinémus: blasphemámur, et obsecrámus: tamquam purgaménta hujus mundi facti sumus, ómnium peripséma usque adhuc. Non ut confúndant vos, hæc scribo, sed ut fílios meos caríssimos móneo: in Christo Jesu, Dómino nostro.
}\switchcolumn\portugues{
\rlettrine{M}{eus} irmãos: Tornamo-nos espectáculo do mundo, dos Anjos e dos homens. Somos loucos por amor de Cristo; mas vós sois prudentes em Cristo. Nós somos fracos; mas vós sois fortes. Vós sois honrados; mas nós somos desprezados. Até agora padecemos a fome, a sede e a nudez; fomos maltratados; andamos errantes; trabalhamos penosamente com nossas próprias mãos. Somos amaldiçoados, mas abençoamos; somos perseguidos, e sofremos; somos blasfemados, e correspondemos com orações. Temos sido considerados até ao presente como o refugo deste mundo, como a escória de todos! Escrevo estas cousas, não para vos envergonhar, mas para vos admoestar, como meus filhos caríssimos em N. S. Jesus Cristo.
}\end{paracol}

\paragraphinfo{Gradual}{Sl. 36, 30-31}
\begin{paracol}{2}\latim{
\rlettrine{O}{s} justi meditábitur sapiéntiam, et lingua ejus loquétur judícium. ℣. Lex Dei ejus in corde ipsíus: et non supplantabúntur gressus ejus.
}\switchcolumn\portugues{
\rlettrine{A}{} boca do justo fala com sabedoria e a sua língua proclama a justiça! ℣. A lei do seu Deus está no seu coração e os seus pés não tropeçarão.
}\switchcolumn*\latim{
Allelúja, allelúja. ℣. \emph{Ps. 111, 1} Beátus vir, qui timet Dóminum: in mandátis ejus cupit nimis. Allelúja.
}\switchcolumn\portugues{
Aleluia, aleluia. ℣. \emph{Sl. 111, 1} Bem-aventurado o varão que teme o Senhor e que põe todo seu zelo em obedecer-Lhe. Aleluia.
}\end{paracol}

\textit{Após a Septuagésima omite-se o Aleluia e o seguinte e diz-se:}

\paragraphinfo{Trato}{Sl. 111, 1-3}
\begin{paracol}{2}\latim{
\rlettrine{B}{eátus} vir, qui timet Dóminum: in mandátis ejus cupit nimis. ℣. Potens in terra erit semen ejus: generátio rectórum benedicétur. ℣. Glória et divítiæ in domo ejus: et justítia ejus manet in sǽculum sǽculi.
}\switchcolumn\portugues{
\rlettrine{B}{em-aventurado} o verão que teme o Senhor e que põe todo o zelo em obedecer-Lhe. ℣. Sua descendência será poderosa na terra; pois a geração dos justos será abençoada. ℣. Na sua casa haverá glória e riqueza; e a sua justiça subsistirá em todos os séculos dos séculos.
}\end{paracol}

\textit{No T. Pascal omite-se o Gradual e o Trato e diz-se:}

\begin{paracol}{2}\latim{
Allelúja, allelúja. ℣. \emph{Ps. 111, 1} Beátus vir, qui timet Dóminum: in mandátis ejus cupit nimis. Allelúja. ℣. \emph{Osee 14, 6} Justus germinábit sicut lílium: et florébit in ætérnum ante Dóminum. Allelúja.
}\switchcolumn\portugues{
Aleluia, aleluia. ℣. \emph{Sl. 111, 1} Bem-aventurado o varão que teme o Senhor e que põe todo o zelo em obedecer-Lhe. Aleluia. ℣. \emph{Os. 14, 6} O justo germinará, como o lírio, e florescerá para sempre diante do Senhor. Aleluia.
}\end{paracol}

\paragraphinfo{Evangelho}{Lc. 12, 32-34}
\begin{paracol}{2}\latim{
\cruz Sequéntia sancti Evangélii secúndum Lucam.
}\switchcolumn\portugues{
\cruz Continuação do santo Evangelho segundo S. Lucas.
}\switchcolumn*\latim{
\blettrine{I}{n} illo témpore: Dixit Jesus discípulis suis: Nolíte timére, pusíllus grex, quia complácuit Patri vestro dare vobis regnum. Véndite quæ possidétis, et date eleemósynam. Fácite vobis sácculos, qui non veteráscunt, thesáurum non deficiéntem in cœlis: quo fur non apprópiat, neque tínea corrúmpit. Ubi enim thesáurus vester est, ibi et cor vestrum erit.
}\switchcolumn\portugues{
\blettrine{N}{aquele} tempo, disse Jesus aos seus discípulos: «Não temais, pequeno rebanho, pois agradou ao vosso Pai dar-vos o seu reino. Vendei tudo quanto possuís e dai-o em esmolas. Fazei para vós bolsas que se não estraguem com o tempo; ajuntai um tesouro no céu, onde o ladrão não pode chegar, nem a traça o pode corromper; pois, onde estiver o vosso tesouro, aí estará o vosso coração».
}\end{paracol}

\paragraphinfo{Ofertório}{Sl. 20, 2-3}
\begin{paracol}{2}\latim{
\rlettrine{I}{n} virtúte tua, Dómine, lætábitur justus, et super salutáre tuum exsultábit veheménter: desidérium ánimæ ejus tribuísti ei. (T. P. Allelúja.)
}\switchcolumn\portugues{
\rlettrine{C}{om} o vosso poder, Senhor, se alegrará o justo, o qual exultará de alegria, vendo-se salvo por Vós. Concedestes-lhe, Senhor, o desejo da sua alma. (T. P. Aleluia.)
}\end{paracol}

\paragraph{Secreta}
\begin{paracol}{2}\latim{
\rlettrine{P}{ræsta} nobis, quǽsumus, omnípotens Deus: ut nostræ humilitátis oblátio et pro tuórum tibi grata sit honóre Sanctórum, et nos córpore páriter et mente puríficet. Per Dóminum \emph{\&c.}
}\switchcolumn\portugues{
\rlettrine{D}{ignai-Vos} conceder-nos, Deus omnipotente, que esta oferta da nossa humildade, em honra dos vossos Santos, Vos seja agradável; e permiti que nos purifique ao mesmo tempo o corpo e a alma. Por nosso Senhor \emph{\&c.}
}\end{paracol}

\paragraphinfo{Comúnio}{Mt. 19, 28 \& 29}
\begin{paracol}{2}\latim{
\rlettrine{A}{men,} dico vobis: quod vos, qui reliquístis ómnia et secúti estis me, céntuplum accipiétis, et vitam ætérnam possidébitis, (T. P. Allelúja.)
}\switchcolumn\portugues{
\rlettrine{E}{m} verdade vos digo: vós, que abandonastes tudo e me seguistes, recebereis o cêntuplo e possuireis a vida eterna. (T. P. Aleluia.)
}\end{paracol}

\paragraph{Postcomúnio}
\begin{paracol}{2}\latim{
\qlettrine{Q}{uǽsumus,} omnípotens Deus: ut, qui cœléstia aliménta percépimus, intercedénte beáto {\redx N.} Confessóre tuo, per hæc contra ómnia advérsa muniámur. Per Dóminum nostrum \emph{\&c.}
}\switchcolumn\portugues{
\slettrine{Ó}{} Deus omnipotente, Vos imploramos, havendo nós recebido o alimento celestial, fazei que, por intercessão do B. Confessor {\redx N.} sejamos fortificados contra todas as adversidades. Por nosso Senhor \emph{\&c.}
}\end{paracol}

\subsectioninfo{Abades}{Missa Os justi}\label{abades}

\paragraphinfo{Intróito}{Sl. 36, 30-31}
\begin{paracol}{2}\latim{
\rlettrine{O}{s} justi meditábitur sapiéntiam, et lingua ejus loquétur judícium: lex Dei ejus in corde ipsíus. (T. P. Allelúja allelúja.) \emph{Ps. ibid., 1} Noli æmulári in malignántibus: neque zeláveris faciéntes iniquitátem.
℣. Gloria Patri \emph{\&c.}
}\switchcolumn\portugues{
\rlettrine{A}{} boca do justo fala com sabedoria: e a sua língua proclama a justiça. A lei do seu Deus estará no seu coração. (T. P. Aleluia, aleluia.) \emph{Sl. ibid., 1} Não vos irriteis contra os maus, nem tenhais inveja daqueles que cometem iniquidades.
℣. Glória ao Pai \emph{\&c.}
}\end{paracol}

\paragraph{Oração}
\begin{paracol}{2}\latim{
\rlettrine{I}{ntercéssio} nos, quǽsumus, Dómine, beáti {\redx N.} Abbátis comméndet: ut, quod nostris méritis non valémus, ejus patrocínio assequámur. Per Dóminum nostrum \emph{\&c.}
}\switchcolumn\portugues{
\qlettrine{Q}{ue} a intercessão do B. Abade {\redx N.} nos favoreça junto de Vós, Senhor, Vos suplicamos, a fim de que aquilo que não podemos conseguir com os nossos méritos o alcancemos com seu patrocínio. Por nosso Senhor \emph{\&c.}
}\end{paracol}

\paragraphinfo{Epístola}{Ecl. 45, 1-6}\label{epistolaabades}
\begin{paracol}{2}\latim{
Léctio libri Sapientiæ.
}\switchcolumn\portugues{
Lição do Livro da Sabedoria.
}\switchcolumn*\latim{
\rlettrine{D}{iléctus} Deo et homínibus, cujus memória in benedictióne est. Símilem illum fecit in glória sanctórum, et magnificávit eum in timóre inimicórum, et in verbis suis monstra placávit. Gloríficávit illum in conspéctu regum, et jussit illi coram pópulo suo, et osténdit illi glóriam suam. In fide et lenitáte ipsíus sanctum fecit illum, et elégit eum; ex omni carne. Audívit enim eum et vocem ipsíus, et indúxit illum in nubem. Et dedit illi coram præcépta, et legem vitæ et disciplínæ.
}\switchcolumn\portugues{
\rlettrine{F}{oi} amado de Deus e dos homens; a sua memória é uma bênção. O Senhor deu-lhe uma glória, semelhante à dos Santos; tornou-o temeroso e invencível perante seus inimigos; e com suas palavras aplacou os monstros. O Senhor honrou-o diante dos reis; deu-lhe as suas ordens diante do seu povo; e mostrou-lhe a sua glória. O Senhor santificou-o pela sua fé e mansidão; e escolheu-o entre todos os homens. Deus escutou-o; ouviu a sua voz; e fê-lo entrar na nuvem. Então, deu-lhe face a face os seus preceitos e a lei da vida e da doutrina.
}\end{paracol}

\paragraphinfo{Gradual}{Sl. 20, 4-5}
\begin{paracol}{2}\latim{
\rlettrine{D}{ómine,} prævenísti eum in benedictiónibus dulcédinis: posuísti in cápite ejus corónam de lápide pretióso. ℣. Vitam pétiit a te, et tribuísti ei longitúdinem diérum in sǽculum sǽculi.
}\switchcolumn\portugues{
\rlettrine{S}{enhor,} concedestes-lhe bênçãos escolhidas as mais suaves; e impusestes na sua cabeça uma coroa de pedras preciosas. Concedestes-lhe a vida, que Vos suplicara, e prolongastes-lhe a duração dos seus dias pelos séculos dos séculos.
}\switchcolumn*\latim{
Allelúja, allelúja. ℣. \emph{Ps. 91, 13} Justus ut palma florébit: sicut cedrus Líbani multiplicábitur. Allelúja.
}\switchcolumn\portugues{
Aleluia, aleluia. ℣. \emph{Sl. 91, 13} O justo florescerá, como a palmeira, e crescerá, como o cedro do Líbano. Aleluia.
}\end{paracol}

\textit{Após a Septuagésima omite-se o Aleluia e o seguinte e diz-se:}

\paragraphinfo{Trato}{Sl. 111, 1-3}
\begin{paracol}{2}\latim{
\rlettrine{B}{eátus} vir, qui timet Dóminum: in mandátis ejus cupit nimis. ℣. Potens in terra erit semen ejus: generátio rectórum benedicétur. ℣. Glória et divítiæ in domo ejus: et justítia ejus manet in sǽculum sǽculi.
}\switchcolumn\portugues{
\rlettrine{B}{em-aventurado} o varão que teme o Senhor e que emprega todo o zelo em obedecer-Lhe. ℣. Sua descendência será poderosa na terra; pois a geração dos justos será abençoada. ℣. Na sua casa haverá glória e riqueza, e a sua justiça subsistirá em todos os séculos dos séculos.
}\end{paracol}

\textit{No T. Pascal omite-se o Gradual e o Trato e diz-se:}

\begin{paracol}{2}\latim{
Allelúja, allelúja. ℣. \emph{Ps. 91, 13} Justus ut palma florébit: sicut cedrus Líbani multiplicábitur. Allelúja. ℣. \emph{Osee 14, 6} Justus germinábit sicut lílium: et florébit in ætérnum ante Dóminum. Allelúja.
}\switchcolumn\portugues{
Aleluia, aleluia. ℣. \emph{Sl. 91, 13} O justo florescerá, como a palmeira, e multiplicar-se-á, como o cedro do Líbano. Aleluia. ℣. \emph{Os. 14, 6} O justo germinará, como o lírio, e florescerá eternamente na presença do Senhor. Aleluia.
}\end{paracol}

\paragraphinfo{Evangelho}{Mt. 19, 27-29}\label{evangelhoabades}
\begin{paracol}{2}\latim{
\cruz Sequéntia sancti Evangélii secúndum Matthǽum.
}\switchcolumn\portugues{
\cruz Continuação do santo Evangelho segundo S. Mateus.
}\switchcolumn*\latim{
\blettrine{I}{n} illo témpore: Dixit Petrus ad Jesum: Ecce, nos relíquimus ómnia, et secúti sumus te: quid ergo erit nobis? Jesus autem dixit illis: Amen, dico vobis, quod vos, qui secuti estis me, in regeneratióne, cum séderit Fílius hóminis in sede majestátis suæ, sedébitis et vos super sedes duódecim, judicántes duódecim tribus Israël. Et omnis, qui relíquerit domum, vel fratres, aut soróres, aut patrem, aut matrem, aut uxórem, aut fílios, aut agros, propter nomen meum, céntuplum accípiet, et vitam ætérnam possidébit.
}\switchcolumn\portugues{
\blettrine{N}{aquele} tempo, disse Pedro a Jesus: «Eis que deixámos tudo e Vos seguimos. Que recompensa teremos por isso?». Jesus disse-lhes: «Em verdade vos digo: vós, que me seguistes, quando, no tempo da regeneração, o Filho do homem se sentar no trono da sua glória, também vos sentareis sobre doze tronos para julgar as doze tribos de Israel. Todo aquele que deixar a sua casa, ou os seus irmãos, ou os seus campos, ou o seu pai, ou a sua mãe, ou a sua mulher por causa do meu nome receberá o cêntuplo e possuirá a vida eterna».
}\end{paracol}

\paragraphinfo{Ofertório}{Sl. 20, 3 \& 4}
\begin{paracol}{2}\latim{
\rlettrine{D}{esidérium} ánimæ ejus tribuísti ei, Dómine, et voluntáte labiórum ejus non fraudásti eum: posuísti in cápite ejus corónam de lápide pretióso. (T. P. Allelúja.)
}\switchcolumn\portugues{
\rlettrine{C}{oncedestes-lhe} o desejo da sua alma e não desprezastes a prece dos seus lábios. Impusestes na sua cabeça uma coroa de pedras preciosas. (T. P. Aleluia).
}\end{paracol}

\paragraph{Secreta}
\begin{paracol}{2}\latim{
\rlettrine{S}{acris} altáribus, Dómine, hóstias superpósitas sanctus {\redx N.} Abbas, quǽsumus, in salútem nobis proveníre depóscat. Per Dóminum \emph{\&c.}
}\switchcolumn\portugues{
\rlettrine{V}{os} imploramos, Senhor, que o vosso santo Abade {\redx N.} nos alcance que a hóstia, oferecida no vosso altar, nos proporcione a salvação. Por nosso Senhor \emph{\&c.}
}\end{paracol}

\paragraphinfo{Comúnio}{Lc. 12, 42}
\begin{paracol}{2}\latim{
\rlettrine{F}{idélis} servus et prudens, quem constítuit dóminus super famíliam suam: ut det illis in témpore trítici mensúram. (T. P. Allelúja.)
}\switchcolumn\portugues{
\rlettrine{O}{} servo fiel e prudente é destinado pelo Senhor para distribuir, oportunamente, na sua família a sua medida de trigo. (T. P. Aleluia.)
}\end{paracol}

\paragraph{Postcomúnio}
\begin{paracol}{2}\latim{
\rlettrine{P}{rótegat} nos, Dómine, cum tui perceptióne sacraménti beátus {\redx N.} Abbas, pro nobis intercedéndo: ut et conversatiónis ejus experiámur insígnia, et intercessiónis percipiámus suffrágia. Per Dóminum nostrum \emph{\&c.}
}\switchcolumn\portugues{
\qlettrine{Q}{ue} a recepção do vosso sacramento, Senhor, unida às preces do B. {\redx N.} Abade, nos sirva de protecção, a fim de que, imitando os insignes exemplos da sua vida, sintamos os efeitos da sua intercessão. Por nosso Senhor \emph{\&c.}
}\end{paracol}


\subsection{Comum das Virgens}

\subsectioninfo{Virgens Mártires}{Missa Loquébar}\label{virgensmartires1}

\paragraphinfo{Intróito}{Sl. 118, 46-47}
\begin{paracol}{2}\latim{
\rlettrine{L}{oquébar} de testimóniis tuis in conspéctu regum, et non confundébar: et meditábar in mandátis tuis, quæ diléxi nimis. (T. P. Allelúja, allelúja.) \emph{Ps. ibid., 1} Beáti immaculáti in via: qui ámbulant in lege Dómini.
℣. Gloria Patri \emph{\&c.}
}\switchcolumn\portugues{
\rlettrine{F}{alava} dos vossos testemunhos na presença dos reis, sem qualquer vergonha, e meditava nos vossos mandamentos, que amava profundamente. (T. P. Aleluia, aleluia.) \emph{Sl. ibid., 1} Bem-aventurados aqueles que são imaculados nos seus caminhos e cumprem a Lei do Senhor.
℣. Glória ao Pai \emph{\&c.}
}\end{paracol}

\paragraph{Oração}
\begin{paracol}{2}\latim{
\rlettrine{D}{eus,} qui inter cétera poténtiæ tuæ mirácula étiam in sexu frágili victóriam martýrii contulísti: concéde propítius; ut, qui beátæ {\redx N.} Vírginis et Mártyris tuæ natalítia cólimus, per ejus ad te exémpla gradiámur. Per Dóminum \emph{\&c.}
}\switchcolumn\portugues{
\slettrine{Ó}{} Deus, que entre outros milagres do vosso poder permitistes que o sexo frágil alcançasse a vitória do martírio, concedei-nos propício que, venerando o nascimento da vossa B. Virgem e Mártir {\redx N.}, caminhemos para Vós, imitando os seus exemplos. Por nosso Senhor \emph{\&c.}
}\end{paracol}

\paragraphinfo{Epístola}{Ecl. 51, 1-8 et 12}
\begin{paracol}{2}\latim{
Léctio libri Sapiéntiæ.
}\switchcolumn\portugues{
Lição do Livro da Sabedoria.
}\switchcolumn*\latim{
\rlettrine{C}{onfitébor} tibi, Dómine, Rex, et collaudábo te Deum, Salvatórem meum. Confitébor nómini tuo: quóniam adjútor et protéctor factus es mihi, et liberásti corpus meum a perditióne, a láqueo línguæ iníquæ et a lábiis operántium mendácium, et in conspéctu astántium factus es mihi adjutor. Et liberasti me secúndum multitúdinem misericórdiæ nóminis tui a rugiéntibus, præparátis ad escam, de mánibus quæréntium ánimam meam, et de portis tribulatiónum, quæ circumdedérunt me: a pressúra flammæ, quæ circúmdedit me, et in médio ignis non sum æstuáta: de altitúdine ventris inferi, et a lingua coinquináta, et a verbo mendácii, a rege iníquo, et a lingua injústa: laudábit usque ad mortem ánima mea Dóminum: quóniam éruis sustinéntes te, et líberas eos de mánibus géntium, Dómine, Deus noster.
}\switchcolumn\portugues{
\qlettrine{Q}{uero} glorificar-Vos, Senhor e Rei; quero louvar-Vos, ó Deus, meu salvador. Quero glorificar o vosso nome, porque fostes o meu sustentáculo e protector e livrastes o meu corpo da perdição, do laço da língua iníqua e dos lábios daqueles que tramam a mentira. Na presença dos meus adversários fostes o meu auxílio. Livrastes-me, segundo a grandeza da misericórdia do vosso nome, dos que rugiam prestes a devorar-me; livrastes-me das mãos dos que procuravam tirar-me a vida; livrastes-me das aflições, que me cercavam; livrastes-me da violência das chamas, que me rodeavam, no meio das quais não senti o calor do fogo; livrastes-me do abysmo profundo do inferno; da língua impura; das palavras mentirosas; do rei iníquo; e da língua injusta! Minha alma louvará o Senhor até à morte, porque Vós, Senhor, nosso Deus, livrais dos perigos aqueles que confiam em Vós, salvando-os do poder dos inimigos.
}\end{paracol}

\paragraphinfo{Gradual}{Sl. 44, 8}
\begin{paracol}{2}\latim{
\rlettrine{D}{ilexísti} justítiam, et odísti iniquitátem. ℣. Proptérea unxit te Deus, Deus tuus, óleo lætítiæ.
}\switchcolumn\portugues{
\rlettrine{A}{mastes} a justiça e odiastes a iniquidade. ℣. Por isso ungiu-vos o Senhor, vosso Deus, com o óleo da alegria.
}\switchcolumn*\latim{
Allelúja, allelúja. ℣. \emph{ibid., 15 \& 16} Adducántur Regi Vírgines post eam: próximæ ejus afferéntur tibi in lætítia. Allelúja.
}\switchcolumn\portugues{
Aleluia, aleluia. ℣. \emph{ibid., 15 \& 16} Serão apresentadas virgens ao Rei após ela: as suas companheiras serão introduzidas no meio da alegria. Aleluia.
}\end{paracol}

\textit{Após a Septuagésima omite-se o Aleluia e o seguinte e diz-se:}

\paragraph{Trato}
\begin{paracol}{2}\latim{
\rlettrine{V}{eni,} Sponsa Christi, áccipe corónam, quam tibi Dóminus præparávit in ætérnum: pro cujus amóre sánguinem tuum fudísti. ℣. \emph{Ps. 44, 8 \& 5} Dilexísti justítiam, et odísti iniquitátem: proptérea unxit te Deus, Deus tuus, óleo lætítiæ præ consórtibus tuis. ℣. Spécie tua et pulchritúdine tua inténde, próspere procéde et regna.
}\switchcolumn\portugues{
\rlettrine{V}{inde,} ó esposa de Cristo; vinde e recebei a coroa que o Senhor preparou para vós, para a eternidade. Foi por amor para com Ele que derramastes o vosso sangue. ℣. \emph{Sl. 44, 8 \& 5} Amastes a justiça e odiastes a iniquidade: eis porque o Senhor, vosso Deus, vos ungiu com o óleo da alegria, de preferência às vossas companheiras. ℣. Caminhai, pois, com beleza e com majestade; ide gozar a vitória e reinai.
}\end{paracol}

\textit{No T. Pascal omite-se o Gradual e o Trato e diz-se:}

\begin{paracol}{2}\latim{
Allelúja, allelúja. ℣. \emph{Ps. 44, 15 \& 16} Adducántur Regi Vírgines post eam: próximæ ejus afferéntur tibi in lætítia. Allelúja. ℣. \emph{ibid., 5} Spécie tua et pulchritúdine tua inténde, próspere procéde et regna. Allelúja.
}\switchcolumn\portugues{
Aleluia, aleluia. ℣. \emph{Sl. 44, 15 \& 16} Após ela, serão apresentadas virgens ao Rei: as suas companheiras serão introduzidas no meio da alegria. Aleluia. ℣. \emph{ibid., 5} Caminhai, pois, com beleza e com majestade; ide gozar a vitória e reinai. Aleluia.
}\end{paracol}

\paragraphinfo{Evangelho}{Mt. 25, 1-13}
\begin{paracol}{2}\latim{
\cruz Sequéntia sancti Evangélii secúndum Matthǽum.
}\switchcolumn\portugues{
\cruz Continuação do santo Evangelho segundo S. Mateus.
}\switchcolumn*\latim{
\blettrine{I}{n} illo témpore: Dixit Jesus discípulis suis parábolam hanc: Simile erit regnum cœlórum decem virgínibus: quæ, accipiéntes lámpades suas, exiérunt óbviam sponso et sponsæ. Quinque autem ex eis erant fátuæ, et quinque prudéntes: sed quinque fátuæ, accéptis lampádibus, non sumpsérunt óleum secum: prudéntes vero accepérunt óleum in vasis suis cum lampádibus. Horam autem faciénte sponso, dormitavérunt omnes et dormiérunt. Média autem nocte clamor factus est: Ecce, sponsus venit, exíte óbviam ei. Tunc surrexérunt omnes vírgines illae, et ornavérunt lámpades suas. Fátuæ autem sapiéntibus dixérunt: Date nobis de óleo vestro: quia lámpades nostræ exstinguúntur. Respondérunt prudéntes, dicéntes: Ne forte non suffíciat nobis et vobis, ite pótius ad vendéntes, et émite vobis. Dum autem irent émere, venit sponsus: et quæ parátæ erant, intravérunt cum eo ad núptias, et clausa est jánua. Novíssime vero véniunt et réliquæ vírgines, dicéntes: Dómine, Dómine, áperi nobis. At ille respóndens, ait: Amen, dico vobis, néscio vos. Vigiláte ítaque, quia nescítis diem neque horam.
}\switchcolumn\portugues{
\blettrine{N}{aquele} tempo, disse Jesus aos seus discípulos esta parábola: «O reino dos céus é semelhante a dez virgens que, empunhando suas lâmpadas, saíram ao encontro do esposo e da esposa. Porém, cinco destas virgens eram loucas e as outras cinco eram prudentes. Ora, as cinco loucas, empunhando suas lâmpadas, não levaram azeite. Ao contrário, as prudentes tomaram azeite em seus vasos para suas lâmpadas. Como o esposo se demorasse em chegar, tiveram sono e dormiram. Quando era meia-noite, ouviu-se um clamor dizer: «Eis que chega o esposo; ide ao seu encontro». Então todas estas virgens se ergueram e prepararam as suas lâmpadas. As loucas disseram às prudentes: «Dai-nos do vosso azeite, porque as nossas lâmpadas apagam-se». As prudentes responderam-lhes: «Não, porque pode suceder que, como a vós, nos falte o azeite; ide antes aos que o vendem, e comprai-o». Ora, enquanto elas foram comprar o azeite, veio o esposo. Então, as que estavam preparadas entraram com ele para as bodas; e fechou-se a porta. Por fim vieram as outras virgens, e disseram: «Senhor, senhor, abri-nos a porta». Ele respondeu: «Na verdade vos digo: não vos conheço. Vigiai, pois, visto que não sabeis nem o dia nem a hora».
}\end{paracol}

\paragraphinfo{Ofertório}{Sl. 44, 15 \& 16}
\begin{paracol}{2}\latim{
\rlettrine{A}{fferéntur} Regi Vírgines post eam: próximæ ejus afferéntur tibi in lætítia et exsultatióne: adducántur in templum Regi Dómino. (T. P. Allelúja.)
}\switchcolumn\portugues{
\rlettrine{A}{pós} ela, serão apresentadas virgens ao Rei: as suas companheiras serão introduzidas no meio da alegria e do júbilo; elas serão conduzidas ao templo do Rei, seu Senhor. (T. P. Aleluia.)
}\end{paracol}

\paragraph{Secreta}
\begin{paracol}{2}\latim{
\rlettrine{S}{úscipe,} Dómine, múnera, quæ in beátæ {\redx N.} Vírginis et Mártyris tuæ sollemnitáte deférimus: cujus nos confídimus patrocínio liberári. Per Dóminum \emph{\&c.}
}\switchcolumn\portugues{
\rlettrine{R}{ecebei} benigno, Senhor, as ofertas que Vos apresentamos nesta solenidade da B. Virgem {\redx N.}, vossa Mártir, com o patrocínio da qual esperamos ser livres. Por nosso Senhor \emph{\&c.}
}\end{paracol}

\paragraphinfo{Comúnio}{Sl. 118, 78 \& 80}
\begin{paracol}{2}\latim{
\rlettrine{C}{onfundántur} supérbi, quia injúste iniquitátem fecérunt in me: ego autem in mandátis tuis exercébor, in tuis justificatiónibus, ut non confúndar. (T. P. Allelúja.)
}\switchcolumn\portugues{
\rlettrine{S}{ejam} confundidos os soberbos, porque praticaram iniquidades contra mim. Para não ser confundido no último dia, cumprirei os vossos mandamentos e preceitos. (T. P. Aleluia.)
}\end{paracol}

\paragraph{Postcomúnio}
\begin{paracol}{2}\latim{
\rlettrine{A}{uxiliéntur} nobis, Dómine, sumpta mystéria: et, intercedénte beáta {\redx N.} Vírgine et Mártyre tua, sempitérna fáciant protectióne gaudére. Per Dóminum \emph{\&c.}
}\switchcolumn\portugues{
\qlettrine{Q}{ue} nos auxiliem os sagrados mystérios, que acabámos de receber, Senhor, e que, por intercessão da B. Virgem {\redx N.}, vossa Mártir, nos façam gozar continuamente a sua protecção. Por nosso Senhor \emph{\&c.}
}\end{paracol}

\subsectioninfo{Virgens Mártires}{Missa Me exspectavérunt}\label{virgensmartires2}

\paragraphinfo{Intróito}{Sl. 118, 95-96}
\begin{paracol}{2}\latim{
\rlettrine{M}{e} exspectavérunt peccatóres, ut pérderent me: testimónia Jua, Dómine, intelléxi: omnis consummatiónis vidi finem: latum mandátum tuum nimis. (T. P. Allelúja, allelúja.) \emph{Ps. ibid., 1} Beáti immaculáti in via: qui ámbulant in lege Dómini.
℣. Gloria Patri \emph{\&c.}
}\switchcolumn\portugues{
\rlettrine{E}{speraram-me} os pecadores para me perder; mas eu, Senhor, tinha meditado nos vossos testemunhos. Encontrei limites em tudo quanto existe: e só o vosso poder é infinito. (T, P. Aleluia, aleluia.) \emph{Sl. ibid., 1} Bem-aventurados os que são imaculados em seus caminhos e cumprem a Lei do Senhor.
℣. Glória ao Pai \emph{\&c.}
}\end{paracol}

\paragraph{Oração}
\begin{paracol}{2}\latim{
\rlettrine{I}{ndulgéntiam} nobis, quǽsumus, Dómine, beáta {\redx N.} Virgo et Martyr implóret: quæ tibi grata semper éxstitit, et mérito castitátis, et tuæ professióne virtútis. Per Dóminum nostrum \emph{\&c.}
}\switchcolumn\portugues{
\rlettrine{D}{ignai-Vos} conceder-nos, Senhor, que alcancemos o perdão dos nossos pecados pela intercessão da B. Virgem e Mártir {\redx N.}, que sempre Vos agradou, não só pelo mérito da castidade, mas também pela prática da vossa virtude. Por nosso Senhor \emph{\&c.}
}\end{paracol}

\paragraphinfo{Epístola}{Ecl. 51, 13-17}
\begin{paracol}{2}\latim{
Léctio libri Sapiéntiæ.
}\switchcolumn\portugues{
Lição do Livro da Sabedoria.
}\switchcolumn*\latim{
\rlettrine{D}{ómine,} Deus meus, exaltásti super terram habitatiónem meam, et pro morre defluénte deprecáta sum. Invocávi Dóminum, Patrem Dómini mei, ut non derelínquat me in die tribulatiónis meæ, et in témpore superbórum sine adjutório. Laudábo nomen tuum assídue, et collaudábo illud in confessióne, et exaudíta est orátio mea. Et liberásti me de perditióne, et eripuísti me de témpore iníquo. Proptérea confitébor et laudem dicam tibi, Dómine, Deus noster.
}\switchcolumn\portugues{
\rlettrine{S}{enhor,} meu Deus, elevastes a minha morada acima da terra. Roguei-Vos que me livrásseis da morte, que me ameaça. Supliquei ao Senhor, Pai do meu Senhor, que me não abandone sem socorro no dia da minha aflição e durante o poderio dos soberbos. Louvarei incessantemente o vosso nome e glorificá-lo-ei na minha acção de graças, porque ouvistes a minha oração, livrastes-me da perdição e arrancastes-me da ocasião iníqua! Eis porque Vos glorificarei, cantando louvores em vossa honra, ó Senhor, nosso Deus.
}\end{paracol}

\paragraphinfo{Gradual}{Sl. 45, 6 \& 5}
\begin{paracol}{2}\latim{
\rlettrine{A}{djuvábit} eam Deus vultu suo: Deus in médio ejus, non commovébitur. ℣. Flúminis ímpetus lætíficat civitátem Dei: sanctificávit tabernáculum suum Altíssimus.
}\switchcolumn\portugues{
\rlettrine{A}{uxiliá-la-á} o Senhor com seu olhar: Deus está ao pé dela e a não deixará fraquejar. ℣. Um rio com a torrente das suas águas alegra a cidade de Deus. O Altíssimo santificou o seu tabernáculo.
}\switchcolumn*\latim{
Allelúja, allelúja. ℣. Hæc est Virgo sápiens, et una de número prudéntum. Allelúja.
}\switchcolumn\portugues{
Aleluia, aleluia. ℣. Esta é a virgem sábia e uma das prudentes. Aleluia.
}\end{paracol}

\textit{Após a Septuagésima omite-se o Aleluia e o seguinte e diz-se:}

\paragraph{Trato}
\begin{paracol}{2}\latim{
\rlettrine{V}{eni,} Sponsa Christi, áccipe corónam, quam tibi Dóminus præparávit in ætérnum: pro cujus amóre sánguinem tuum fudísti. ℣. \emph{Ps. 44, 8 \& 5} Dilexísti justítiam, et odísti iniquitátem proptérea unxit te Deus, Deui tuus, óleo lætítiæ præ consórtibus tuis. ℣. Spécie tua et pulchritúdine tua inténde, próspere procéde et regna.
}\switchcolumn\portugues{
\rlettrine{V}{inde,} ó esposa de Cristo; vinde e recebei a coroa que o Senhor preparou para vós, para a eternidade. Foi por amor dele que derramastes o vosso sangue. ℣. \emph{Sl. 44, 8 \& 5} Amastes a justiça e odiastes a iniquidade: eis porque o Senhor, vosso Deus, vos ungiu com o óleo da alegria, de preferência às vossas companheiras. ℣. Caminhai, pois, com beleza e majestade; ide gozar a vitória e reinai.
}\end{paracol}

\textit{No T. Pascal omite-se o Gradual e o Trato diz-se:}

\begin{paracol}{2}\latim{
Allelúja, allelúja. ℣. Hæc est Virgo sápiens, et una de número prudéntum. Allelúja. ℣. \emph{Sap. 4, 1} O quam pulchra et casta generátio cum claritáte! Allelúja.
}\switchcolumn\portugues{
Aleluia, aleluia. ℣. Esta é a virgem sábia e uma das prudentes. Aleluia. ℣. \emph{Sb. 4, 1} Oh! como é bela a geração casta e gloriosa! Aleluia.
}\end{paracol}

\paragraphinfo{Evangelho}{Mt. 13, 44-52}\label{evangelhovirgensmartires2}
\begin{paracol}{2}\latim{
\cruz Sequéntia sancti Evangélii secúndum Matthǽum.
}\switchcolumn\portugues{
\cruz Continuação do santo Evangelho segundo S. Mateus.
}\switchcolumn*\latim{
\blettrine{I}{n} illo témpore: Dixit Jesus discípulis suis parábolam me: Símile est regnum cœlórum thesáuro abscóndito in agro: quem qui invénit homo, abscóndit, et præ gáudio illíus vadit, et vendit univérsa, quæ habet, et emit agrum illum. Iterum símile est regnum cœlórum hómini negotiatóri, quærénti bonas margarítas. Invénta autem una pretiósa margaríta, ábiit, et véndidit ómnia, quæ hábuit, et emit eam. Iterum símile est regnum cœlórum sagénse, missæ in mare et ex omni génere píscium congregánti. Quam, cum impléta esset, educéntes, et secus litus sedéntes, elegérunt bonos in vasa, malos autem foras misérunt. Sic erit in consummatióne sǽculi: exíbunt Angeli, et separábunt malos de médio justórum, et mittent eos in camínum ignis: ibi erit fletus et stridor déntium. Intellexístis hæc ómnia? Dicunt ei: Etiam. Ait illis: Ideo omnis scriba doctus in regno cœlórum símilis est hómini patrifamílias, qui profert de thesáuro suo nova et vétera.
}\switchcolumn\portugues{
\blettrine{N}{aquele} tempo, disse Jesus aos seus discípulos esta parábola: «O reino dos céus é semelhante a um tesouro escondido no campo, o qual um homem acha e esconde; e, alegre com o achado, vai, vende tudo o que tem e compra o campo. Igualmente o reino dos céus é semelhante a um negociante, que busca pedras preciosas; e, achando urna de grande preço, vai, vende tudo o que tem e compra-a. O reino dos céus é ainda semelhante a uma rede que, lançada ao mar, prende toda a casta de peixes, e, estando já cheia, os pescadores a puxam para a praia, escolhem os bons peixes para os vasos e deitam fora os maus. Assim será no fim dos séculos: virão os Anjos e separarão os maus do meio dos justos e os lançarão na fornalha do fogo, onde não haverá senão fogo e ranger de dentes. Haveis compreendido tudo isto?». «Sim», responderam eles. E Jesus disse-lhes: «Por esta razão todo o escriba douto no que diz respeito ao reino dos céus é semelhante a um pai de família, que tira dos seus tesouros cousas novas e velhas».
}\end{paracol}

\paragraphinfo{Ofertório}{Sl. 44. 3}
\begin{paracol}{2}\latim{
\rlettrine{D}{iffúsa} est grátia in lábiis tuis: proptérea benedíxit te Deus in ætérnum, et in sǽculum sǽculi. (T. P. Allelúja.)
}\switchcolumn\portugues{
\rlettrine{A}{} graça espalhou-se nos vossos lábios: por isso Deus vos abençoou para a eternidade e para todos os séculos. (T. P. Aleluia.)
}\end{paracol}

\paragraph{Secreta}
\begin{paracol}{2}\latim{
\rlettrine{H}{óstias} tibi, Dómine, beátæ {\redx N.} Vírginis et Mártyris tuæ dicátas méritis, benígnus assúme: et ad perpétuum nobis tríbue proveníre subsídium. Per Dóminum nostrum \emph{\&c.}
}\switchcolumn\portugues{
\rlettrine{A}{ceitai} benignamente, Senhor, as hóstias que Vos oferecemos pelos méritos da B. Virgem e Mártir {\redx N.}, e dignai-Vos permitir que ela nos sirva de perpétuo auxílio. Por nosso Senhor \emph{\&c.}
}\end{paracol}

\paragraphinfo{Comúnio}{Sl. 118, 121, 122 \& 128}
\begin{paracol}{2}\latim{
\rlettrine{F}{eci} judícium et justítiam, Dómine, non calumniéntur mihi supérbi: ad ómnia mandáta tua dirigébar, omnem viam iniquitátis ódio hábui. (T. P. Allelúja.)
}\switchcolumn\portugues{
\rlettrine{T}{enho} procedido com equidade e com justiça, Senhor! Que os soberbos, pois, não me caluniem. Tenho-me guiado em tudo pelos vossos Mandamentos e tenho odiado todas as regras iníquas. (T. P. Aleluia.)
}\end{paracol}

\paragraph{Postcomúnio}
\begin{paracol}{2}\latim{
\rlettrine{D}{ivíni} múneris largitáte satiáti, quǽsumus, Dómine, Deus noster: ut, intercedénte beáta {\redx N.} Vírgine et Mártyre tua, in ejus semper participatióne vivámus. Per Dóminum \emph{\&c.}
}\switchcolumn\portugues{
\rlettrine{S}{aciados} com a liberalidade do dom divino, ó Senhor, nosso Deus, Vos suplicamos pela intercessão da B. Virgem {\redx N.}, vossa Mártir, que durante a nossa vida comparticipemos sempre deste dom divino. Por nosso Senhor \emph{\&c.}
}\end{paracol}

Outro Evangelho (para certos dias):

\paragraphinfo{Evangelho}{Mt. 19, 3-12}
\begin{paracol}{2}\latim{
\cruz Sequéntia sancti Evangélii secúndum Matthǽum.
}\switchcolumn\portugues{
\cruz Continuação do santo Evangelho segundo S. Mateus.
}\switchcolumn*\latim{
\blettrine{I}{n} illo témpore: Accessérunt ad Jesum pharisǽi, tentántes eum et dicéntes: Si licet hómini dimíttere uxórem suam quacúmque ex causa? Qui respóndens, ait eis: Non legístis, quia, qui fecit hóminem ab inítio, másculum et féminam fecit eos? et dixit: Propter hoc dimíttet homo patrem, et matrem, et adhærébit uxóri suæ, et erunt duo in carne una. Itaque jam non sunt duo, sed una caro. Quod ergo Deus conjúnxit, homo non séparet. Dicunt illi: Quid ergo Móyses mandávit dare libéllum repúdii, et dimíttere? Ait illis: Quóniam Móyses ad durítiam cordis vestri permísit vobis dimíttere uxóres vestras: ab inítio autem non fuit sic. Dico autem vobis, quia, quicúmque dimíserit uxórem suam, nisi ob fornicatiónem, et áliam dúxerit, mœchátur: et qui dimíssam dúxerit, mœchátur. Dicunt ei discípuli ejus: Si ita est causa hóminis cum uxóre, non expédit núbere. Qui dixit illis: Non omnes cápiunt verbum istud, sed quibus datum est. Sunt enim eunúchi, qui de matris útero sic nati sunt; et sunt eunúchi, qui facti sunt ab homínibus; et sunt eunúchi, qui seípsos castravérunt propter regnum cœlórum. Qui potest cápere, cápiat.
}\switchcolumn\portugues{
\blettrine{N}{aquele} tempo, aproximaram-se os fariseus de Jesus para O tentar e disseram-Lhe: «É lícito ao homem repudiar sua mulher por qualquer causa?». Respondendo Jesus, disse-lhes: «Não lestes: «Aquele que criou o homem no princípio do mundo criou um homem e uma mulher, e disse que por causa disto o homem deixará seu pai e sua mãe, e se unirá com sua mulher; e serão dous em uma só carne»? Assim, não serão mais dous; mas uma só carne. Que o homem, pois, não separe o que Deus uniu». Eles disseram-Lhe: «Porque mandou, então, Moisés dar carta de repúdio, e deixá-la?». Ele respondeu: «Foi por causa da dureza do vosso coração que Moisés permitiu que repudiásseis vossas mulheres; mas no princípio não foi assim. E eu vos digo: todo aquele que deixar sua mulher, a não ser por adultério, e casar com outra, comete adultério; e aquele que casar com uma mulher repudiada, também comete adultério». Disseram-Lhe, então, os discípulos: «Se tal é a situação do homem diante da mulher, melhor é não se casar». E Ele disse-lhes: «Nem todos são capazes de compreender esta palavra, mas só aqueles a quem isso é dado; pois há eunucos que já assim vieram do seio de sua mãe; há outros que foram feitos pelos homens; e há ainda outros que se fizeram a si mesmo, por causa do reino dos céus. Quem pode compreender isto, compreenda».
}\end{paracol}

\subsection{Muitas Virgens Mártires}\label{19muitasvirgensmartires}

\textit{Como na Missa Me exspectavérunt, página \pageref{virgensmartires2}, excepto o seguinte:}

\paragraph{Oração}
\begin{paracol}{2}\latim{
\rlettrine{D}{a} nobis, quǽsumus, Dómine, Deus noster, sanctárum Vírginum et Mártyrum tuárum {\redx N.} et {\redx N.} palmas incessábili devotióne venerári: ut, quas digna mente non póssumus celebráre, humílibus saltem frequentémus obséquiis. Per Dóminum nostrum \emph{\&c.}
}\switchcolumn\portugues{
\slettrine{Ó}{} Senhor, nosso Deus, dignai-Vos conceder-nos a graça de incessantemente venerarmos com devoção a vitória das vossas santas Virgens e Mártires {\redx N.} e {\redx N.}, a fim de que, já que não podemos celebrar dignamente os seus méritos, ao menos possamos oferecer-lhes as nossas humildes homenagens. Por nosso Senhor \emph{\&c.}
}\end{paracol}

\paragraphinfo{Epístola}{Página \pageref{virgemnaomartir2}}

\paragraph{Secreta}
\begin{paracol}{2}\latim{
\rlettrine{I}{nténde,} quǽsumus, Dómine, múnera altáribus tuis pro sanctárum Vírginum et Mártyrum tuárum {\redx N.} et {\redx N.} festivitáte propósita: ut, sicut per hæc beáta mystéria illis glóriam contulísti; ita nobis indulgéntiam largiáris. Per Dóminum \emph{\&c.}
}\switchcolumn\portugues{
\rlettrine{S}{enhor,} dignai-Vos volver os vossos olhares para estas ofertas que depositamos nos vossos altares para celebrar a festa das vossas santas Virgens e Mártires {\redx N.} e {\redx N.}, a fim de que, assim como lhes concedestes a glória, assim também nos concedais o perdão dos nossos pecados. Por nosso Senhor \emph{\&c.}
}\end{paracol}

\paragraph{Postcomúnio}
\begin{paracol}{2}\latim{
\rlettrine{P}{ræsta} nobis, quǽsumus, Dómine, intercedéntibus sanctis Virgínibus et Martýribus tuis {\redx N.} et {\redx N.}: ut, quod ore contíngimus, pura mente capiámus. Per Dóminum \emph{\&c.}
}\switchcolumn\portugues{
\rlettrine{C}{oncedei-nos,} Senhor, Vos suplicamos, que, por intercessão das vossas santas Virgens e Mártires {\redx N.} e {\redx N.}, guardemos com o coração puro o que nossa boca acaba de receber. Por nosso Senhor \emph{\&c.}
}\end{paracol}

\subsectioninfo{Virgem não Mártir}{Missa Dilexísti justitiam}\label{virgemnaomartir1}

\paragraphinfo{Intróito}{Sl. 44, 8}
\begin{paracol}{2}\latim{
\rlettrine{D}{ilexísti} justítiam, et odísti iniquitátem: proptérea unxit te Deus, Deus tuus, óleo lætítiae præ consórtibus tuis. (T. P. Allelúja, allelúja.) \emph{Ps. ibid., 2} Eructávit cor meum verbum bonum: dico ego ópera mea Regi.
℣. Gloria Patri \emph{\&c.}
}\switchcolumn\portugues{
\rlettrine{A}{mastes} a justiça e odiastes a iniquidade; pelo que ungiu-vos o Senhor, vosso Deus, com o óleo da alegria, de preferência às vossas companheiras. (T. P. Aleluia, aleluia.) \emph{Sl. ibid., 2} Meu coração exprimiu uma excelente palavra: Consagro ao Rei as minhas obras.
℣. Glória ao Pai \emph{\&c.}
}\end{paracol}

\paragraph{Oração}
\begin{paracol}{2}\latim{
\rlettrine{E}{xáudi} nos, Deus, salutáris noster: ut, sicut de beátæ {\redx N.} Vírginis tuæ festivitáte gaudémus; ita piæ devotiónis erudiámur afféctu. Per Dóminum nostrum \emph{\&c.}
}\switchcolumn\portugues{
\rlettrine{O}{uvi-nos,} ó Deus, nosso salvador, a fim de que, assim como nos alegramos com a festa da vossa B. Virgem {\redx N.} assim também consigamos alcançar piedosos sentimentos de fervorosa devoção. Por nosso Senhor \emph{\&c.}
}\end{paracol}

\paragraphinfo{Epístola}{2. Cor. 10, 17-18; 11, 1-2}
\begin{paracol}{2}\latim{
Léctio Epístolæ beáti Pauli Apóstoli ad Corínthios.
}\switchcolumn\portugues{
Lição da Ep.ª do B. Ap.º Paulo aos Coríntios.
}\switchcolumn*\latim{
\rlettrine{F}{ratres:} Qui gloriátur, in Dómino gloriétur. Non enim, qui seípsum comméndat, ille probátus est; sed quem Deus comméndat. Utinam sustinerétis módicum quid insipiéntiæ meæ, sed et supportáte me: ǽmulor enim vos Dei æmulatióne. Despóndi enim vos uni viro vírginem castam exhibére Christo.
}\switchcolumn\portugues{
\rlettrine{M}{eus} irmãos: Aquele que se gloria, glorie-se no Senhor; pois não é aprovado o que se louva a si mesmo, mas aquele a quem Deus recomenda. Queira Deus que possais suportar um pouco ainda a minha loucura; mas suportai-me ainda, pois estou zeloso de vós da parte do zelo de Deus. Com efeito, desposei-vos com o único esposo, para vos apresentar a Cristo como virgem pura.
}\end{paracol}

\paragraphinfo{Gradual}{Sl. 44, 5}
\begin{paracol}{2}\latim{
\rlettrine{S}{pécie} tua et pulchritúdine tua inténde, próspere procéde et regna. ℣. Propter veritátem et mansuetúdinem et justítiam: et dedúcet te mirabíliter déxtera tua.
}\switchcolumn\portugues{
\rlettrine{C}{aminhai} com beleza e com majestade; ide gozar a vitória e reinai. Por causa da verdade, da mansidão e da justiça, a vossa dextra praticará maravilhas!
}\switchcolumn*\latim{
Allelúja, allelúja. ℣. \emph{ibid., 15 \& 16} Adducántur Regi Vírgines post eam: próximæ ejus afferéntur tibi in lætítia. Allelúja.
}\switchcolumn\portugues{
Aleluia, aleluia. ℣. \emph{ibid., 15 \& 16} Após ela, serão apresentadas virgens ao Rei: as suas companheiras serão introduzidas no meio da alegria. Aleluia.
}\end{paracol}

\textit{Após a Septuagésima omite-se o Aleluia e o seguinte e diz-se:}

\paragraphinfo{Trato}{Sl. 44, 11 \& 12}
\begin{paracol}{2}\latim{
\rlettrine{A}{udi,} fília, et vide, et inclína aurem tuam: quia concupívit Rex spéciem tuam. ℣. \emph{ibid., 13 et 10} Vultum tuum deprecabúntur omnes dívites plebis: fíliæ regum in honóre tuo. ℣. \emph{ibid., 15-16} Adducéntur Regi Vírgines post eam: próximæ ejus afferéntur tibi. ℣. Afferéntur in lætítia et exsultatióne: adducántur in templum Regis.
}\switchcolumn\portugues{
\slettrine{Ó}{} minha filha, ouvi, vede e prestai atenção; pois o Rei está cheio de amor por vós, por causa da vossa beleza. ℣. \emph{ibid., 13 et 10} Todos os poderosos da terra implorarão os vossos olhares: e as filhas dos reis formam a vossa corte de glória. ℣. \emph{ibid., 15-16} Depois de vós, virão coros de virgens: as suas companheiras serão apresentadas ao Rei. ℣. Serão apresentadas no meio da alegria e do júbilo: e serão introduzidas no templo do Rei.
}\end{paracol}

\textit{No T. Pascal omite-se o Gradual e o Trato e diz-se:}

\begin{paracol}{2}\latim{
Allelúia, allelúja. ℣. \emph{Ps. 44, 15 et 16} Adducéntur Regi Vírgines post eam: próximæ ejus afferéntur tibi in lætítia. Allelúja. ℣. \emph{ibid., 5} Spécie tua et pulchritúdine tua inténde, próspere procéde et regna. Allelúja.
}\switchcolumn\portugues{
Aleluia, aleluia. ℣. \emph{Sl. 44, 15 et 16} Após ela, serão apresentadas virgens ao Rei: as suas companheiras serão introduzidas no meio da alegria. Aleluia. ℣. \emph{ibid., 5} Caminhai com beleza e com majestade; ide gozar a vitória e reinai. Aleluia.
}\end{paracol}

\paragraphinfo{Evangelho}{Mt. 25, 1-13}
\begin{paracol}{2}\latim{
\cruz Sequéntia sancti Evangélii secúndum Matthǽum.
}\switchcolumn\portugues{
\cruz Continuação do santo Evangelho segundo S. Mateus.
}\switchcolumn*\latim{
\blettrine{I}{n} illo témpore: Dixit Jesus discípulis suis parábolam hanc: Simile erit regnum cœlórum decem virgínibus: quæ, accipiéntes lámpades suas, exiérunt óbviam sponso et sponsæ. Quinque autem ex eis erant fátuæ, et quinque prudéntes: sed quinque fátuæ, accéptis lampádibus, non sumpsérunt óleum secum: prudéntes vero accepérunt óleum in vasis suis cum lampádibus. Horam autem faciénte sponso, dormitavérunt omnes et dormiérunt. Média autem nocte clamor factus est: Ecce, sponsus venit, exíte óbviam ei. Tunc surrexérunt omnes vírgines illæ, et ornavérunt lámpades suas. Fátuæ autem sapiéntibus dixérunt: Date nobis de óleo vestro: quia lámpades nostræ exstinguúntur. Respondérunt prudéntes, dicéntes: Ne forte non suffíciat nobis et vobis, ite pótius ad vendéntes, et émite vobis. Dum autem irent émere, venit sponsus: et quæ parátæ erant, intravérunt cum eo ad núptias, et clausa est jánua. Novíssime vero véniunt et réliquæ vírgines, dicéntes: Dómine, Dómine, aperi nobis. At ille respóndens, ait: Amen, dico vobis, néscio vos. Vigiláte ítaque, quia nescítis diem neque horam.
}\switchcolumn\portugues{
\blettrine{N}{aquele} tempo, disse Jesus aos seus discípulos esta parábola: «O reino dos céus é semelhante a dez virgens que, empunhando suas lâmpadas, saíram ao encontro do esposo e da esposa. Porém, cinco destas virgens eram loucas e as outras cinco eram prudentes. Ora, as cinco loucas, empunhando as suas lâmpadas, não levaram azeite. Ao contrário, as prudentes tomaram azeite em seus vasos para suas lâmpadas. Como o esposo se demorasse em chegar, tiveram sono e dormiram. Quando era meia-noite, ouviu-se um clamor dizer: «Eis que chega o esposo; ide ao seu encontro». Então todas estas virgens se ergueram e prepararam as suas lâmpadas. As loucas disseram às prudentes: «Dai-nos do vosso azeite, porque as nossas lâmpadas apagam-se». As prudentes responderam-lhes: «Não, porque pode suceder que, como a vós, nos falte o azeite; ide antes aos que o vendem e comprai-o». Ora, enquanto elas foram comprar o azeite, veio o esposo. Então, as que estavam preparadas entraram com ele para as bodas; e fechou-se a porta. Por fim vieram as outras virgens e disseram: «Senhor, senhor, abri-nos a porta». Ele respondeu: «Na verdade vos digo: não vos conheço. Vigiai, pois, visto que não sabeis nem o dia nem a hora».
}\end{paracol}

\paragraphinfo{Ofertório}{Sl. 44, 10}
\begin{paracol}{2}\latim{
\rlettrine{F}{íliæ} regum in honóre tuo, ástitit regína a dextris tuis in vestítu deauráto, circúmdata varietate. (T. P. Allelúja.)
}\switchcolumn\portugues{
\rlettrine{A}{s} filhas dos reis formam a vossa corte de glória: a própria rainha está colocada à vossa direita, envergando um vestido de ouro, recamado da mais rica variedade. (T. P. Aleluia.)
}\end{paracol}

\paragraph{Secreta}
\begin{paracol}{2}\latim{
\rlettrine{A}{ccépta} tibi sit, Dómine, sacrátæ plebis oblátio pro tuórum honóre Sanctórum: quorum se méritis de tribulatióne percepísse cognóscit auxílium. Per Dóminum nostrum \emph{\&c.}
}\switchcolumn\portugues{
\rlettrine{A}{ceitai,} Senhor, esta oferta, que Vos consagra o vosso povo fiel em honra dos vossos santos, pelos méritos dos quais reconhece que tem alcançado a vossa assistência nas tribulações. Por nosso Senhor \emph{\&c.}
}\end{paracol}

\paragraphinfo{Comúnio}{Mt. 25, 4 \& 6}
\begin{paracol}{2}\latim{
\qlettrine{Q}{uinque} prudéntes vírgines accepérunt óleum in vasis suis cum lampádibus: média autem nocte clamor factus est: Ecce, sponsus venit: exite óbviam Christo Dómino. (T. P. Allelúja.)
}\switchcolumn\portugues{
\rlettrine{A}{s} cinco virgens prudentes levaram azeite em seus vasos para suas lâmpadas. Ora, no meio da noite, ouviu-se este clamor: «Eis que chega o esposo; comparecei diante de Cristo, vosso Senhor». (T. P Aleluia.)
}\end{paracol}

\paragraph{Postcomúnio}
\begin{paracol}{2}\latim{
\rlettrine{S}{atiásti,} Dómine, famíliam tuam munéribus sacris: ejus, quǽsumus, semper interventióne nos réfove, cujus sollémnia celebrámus. Per Dóminum \emph{\&c.}
}\switchcolumn\portugues{
\rlettrine{H}{avendo} Vós, Senhor, saciado a vossa família com vossos dons sagrados, dignai-Vos favorecer-nos sempre com a intercessão daquela cuja festa celebrámos. Por nosso Senhor \emph{\&c.}
}\end{paracol}

\subsectioninfo{Virgem não Mártir}{Missa Vultum tuum}\label{virgemnaomartir2}

\paragraphinfo{Intróito}{Sl. 44, 13, 15 \& 16}
\begin{paracol}{2}\latim{
\rlettrine{V}{ultum} tuum deprecabúntur omnes dívites plebis: adducéntur Regi Vírgines post eam: próximæ ejus adducéntur tibi in lætítia et exsultatióne. (T. P. Allelúja, allelúja.) \emph{Ps. ibid., 2} Eructávit cor meum verbum bonum: dico ego ópera mea Regi.
℣. Gloria Patri \emph{\&c.}
}\switchcolumn\portugues{
\rlettrine{T}{odos} os poderosos da terra implorarão os vossos olhares: após ela, serão apresentadas virgens ao Rei: as suas companheiras serão apresentadas ao Rei com grande alegria e júbilo. (T. P. Aleluia, aleluia.) \emph{Sl. ibid., 2} Meu coração exprimiu uma palavra excelente: Consagro ao Rei as minhas obras!
℣. Glória ao Pai \emph{\&c.}
}\end{paracol}

\paragraph{Oração}
\begin{paracol}{2}\latim{
\rlettrine{E}{xáudi} nos, Deus, salutáris noster: ut, sicut de beátæ {\redx N.} Vírginis tuæ festivitáte gaudémus; ita piæ devotiónis erudiámur affectu. Per Dóminum nostrum \emph{\&c.}
}\switchcolumn\portugues{
\rlettrine{O}{uvi-nos,} ó Deus, nosso salvador, a fim de que, assim como nos alegramos com a festa da vossa B. Virgem {\redx N.}, assim também consigamos alcançar sentimentos de terna devoção. Por nosso Senhor \emph{\&c.}
}\end{paracol}

\paragraphinfo{Epístola}{1. Cor. 7, 25-34}
\begin{paracol}{2}\latim{
Léctio Epístolæ beáti Pauli Apóstoli ad Corínthios.
}\switchcolumn\portugues{
Lição da Ep.ª do B. Ap.º Paulo aos Coríntios.
}\switchcolumn*\latim{
\rlettrine{F}{ratres:} De virgínibus præcéptum Dómini non hábeo: consílium autem do, tamquam misericórdiam consecútus a Dómino, ut sim fidélis. Exístimo ergo hoc bonum esse propter instántem necessitátem, quóniam bonum est hómini sic esse. Alligátus es uxóri? noli quǽrere solutiónem. Solútus es ab uxóre? noli quǽrere uxorem. Si autem acceperis uxorem, non peccásti. Et si núpserit virgo, non peccavit: tribulatiónem tamen carnis habébunt hujúsmodi. Ego autem vobis parco. Hoc ítaque dico, fratres: Tempus breve est: réliquum est, ut, et qui habent uxóres, tamquam non habéntes sint; et qui flent, tamquam non flentes; et qui gaudent, tamquam non gaudéntes; et qui emunt, tamquam non possidéntes; et qui utúntur hoc mundo, tamquam non utántur; prǽterít enim figúra hujus mundi. Volo autem vos sine sollicitúdine esse. Qui sine uxóre est, sollícitus est, quæ Dómini sunt, quómodo pláceat Deo. Qui autem cum uxóre est, sollícitus est, quæ sunt mundi, quómodo pláceat uxóri, et divísus est. Et múlier innúpta et virgo cógitat, quæ Dómini sunt, ut sit sancta córpore et spíritu: in Christo Jesu, Dómino nostro.
}\switchcolumn\portugues{
\rlettrine{M}{eus} irmãos: Quanto às virgens, não recebi preceito do Senhor; mas eis o conselho que dou, para ser fiel à graça que o Senhor misericordiosamente me fez. Creio que é vantajoso ao homem permanecer assim, por causa das instantes necessidades desta vida. Estais unido a uma mulher? Não procureis desligar-vos. Não estais unido a nenhuma mulher? Não procureis mulher. Se, porém, desposastes uma mulher, não pecastes; e, se uma virgem casar, não peca. Contudo, estas pessoas sofrerão as tribulações da carne, o que procuro evitar-vos. Eis, pois, o que vos digo, irmãos: o tempo é breve; o que resta é que os que têm mulheres procedam como se as não tivessem; os que choram, como se não chorassem; os que se regozijam, como se não se regozijassem; os que compram, como se nada possuíssem; e os que usam deste mundo, como se não usassem, pois, a aparência deste mundo passa. Bem quisera que estivésseis sem preocupações. Aquele que não é casado é solícito para com as coisas do Senhor e procura proceder de modo que agrade a Deus. Assim como o que é casado se ocupa solicitamente das coisas deste mundo e do que deverá fazer para agradar a sua mulher; assim ele está dividido. Do mesmo modo a mulher solteira e a virgem pensam nas coisas que são do Senhor, a fim de que sejam santas de corpo e de espírito, em N. S. Jesus Cristo.
}\end{paracol}

\paragraphinfo{Gradual}{Sl. 44, 12 \& 11}
\begin{paracol}{2}\latim{
\rlettrine{C}{oncupívit} Rex decórem tuum, quóniam ipse est Dóminus, Deus tuus. ℣. Audi, fília, et vide, et inclína aurem tuam.
}\switchcolumn\portugues{
\rlettrine{O}{} Rei está cheio de amor por vós, por causa da vossa beleza, pois Ele é o Senhor, vosso Deus. ℣. Ó minha filha, vede e prestai atenção.
}\switchcolumn*\latim{
Allelúja, allelúja. ℣. Hæc est Virgo sápiens, et una de número prudéntum. Allelúja.
}\switchcolumn\portugues{
Aleluia, aleluia. ℣. Esta é a virgem sábia e uma das virgens prudentes. Aleluia.
}\end{paracol}

\textit{Após a Septuagésima omite-se o Aleluia e o seguinte e diz-se:}

\paragraphinfo{Trato}{Sl. 44, 12, 13 \& 10}
\begin{paracol}{2}\latim{
\qlettrine{Q}{uia} concupívit Rex spéciem tuam. ℣. Vultum tuum deprecabúntur omnes divites plebis: fíliæ regum in honóre tuo. ℣. \emph{ibid., 15-16} Adducéntur Regi Vírgines post eam: próximæ ejus afferéntur tibi. ℣. Afferéntur in lætítia et exsultatióne: adducéntur in templum Regis.
}\switchcolumn\portugues{
\rlettrine{P}{ois} o Rei está cheio de amor por vós, por causa da vossa beleza. ℣. Todos os poderosos da terra implorarão os vossos olhares: e as filhas dos reis formam a vossa corte de glória. ℣. \emph{ibid., 15-16} Depois de vós, virão coros de virgens: as suas companheiras serão apresentadas ao Rei. ℣. Serão apresentadas no meio da alegria e do júbilo: e serão introduzidas no templo do Rei.
}\end{paracol}

\textit{No T. Pascal omite-se o Gradual e o Trato e diz-se:}

\begin{paracol}{2}\latim{
Allelúja, allelúja. ℣. Hæc est Virgo sápiens, et una de número prudéntum. Allelúja. ℣. \emph{Sap. 4, 1} O quam pulchra est casta generátio cum claritáte! Allelúja.
}\switchcolumn\portugues{
Aleluia, aleluia. ℣. Esta é a virgem sábia e uma das virgens prudentes. Aleluia. ℣. \emph{Sb. 4, 1} Oh! como é bela a geração casta e gloriosa! Aleluia.
}\end{paracol}

\paragraphinfo{Evangelho}{Página \pageref{virgensmartires1}}

\paragraphinfo{Ofertório}{Sl. 44, 15-16}
\begin{paracol}{2}\latim{
\rlettrine{A}{fferéntur} Regi Vírgines post eam: próximæ ejus afferéntur tibi in lætítia et exsultatióne: adducéntur in templum Regi Dómino. (T. P. Allelúja.)
}\switchcolumn\portugues{
\rlettrine{A}{pós} ela, serão apresentadas virgens ao Rei: as suas companheiras serão introduzidas no meio da alegria e do júbilo: e serão conduzidas ao templo do Rei, seu Senhor. (T. P. Aleluia.)
}\end{paracol}

\paragraph{Secreta}
\begin{paracol}{2}\latim{
\rlettrine{A}{ccépta} tibi sit, Dómine, sacrátæ plebis oblátio pro
tuorum honore Sanctórum: quorum se meritis de tribulatione percepísse cognóscit auxílium. Per Dóminum \emph{\&c.}
}\switchcolumn\portugues{
\rlettrine{A}{ceitai,} Senhor, esta oferta, que Vos consagra o vosso povo fiel em honra dos vossos santos, pelos méritos dos quais reconhece que tem alcançado a vossa assistência nas tribulações. Por nosso Senhor \emph{\&c.}
}\end{paracol}

\paragraphinfo{Comúnio}{Mt. 13, 45-46}
\begin{paracol}{2}\latim{
\rlettrine{S}{ímile} est regnum cœlórum hómini negotiatóri, quærénti bonas margarítas: invénta autem una pretiósa margaríta, dedit ómnia sua, et comparávit eam. (T. P. Allelúja.)
}\switchcolumn\portugues{
\rlettrine{O}{} reino dos céus é semelhante a um homem negociante que procura pérolas boas, e, achando uma de subido valor, vai, vende todos os bens e compra-a. (T. P. Aleluia.)
}\end{paracol}

\paragraph{Postcomúnio}
\begin{paracol}{2}\latim{
\rlettrine{S}{atiásti,} Dómine, famíliam tuam munéribus sacris: ejus, quǽsumus, semper interventióne nos réfove, cujus sollémnia celebrámus. Per Dóminum \emph{\&c.}
}\switchcolumn\portugues{
\rlettrine{H}{avendo} Vós, Senhor, saciado a vossa família com vossos dons sagrados, dignai-Vos favorecer-nos sempre pela intercessão daquela cuja festa celebramos. Por nosso Senhor \emph{\&c.}
}\end{paracol}


\subsection{Comum das Mulheres Santas}

\subsectioninfo{Mártires não Virgens}{Missa Me exspectavérunt}\label{martiresnaovirgens}

\paragraphinfo{Intróito}{Sl. 118, 95-96}
\begin{paracol}{2}\latim{
\rlettrine{M}{e} exspectavérunt peccatóres, ut pérderent me: testimónia tua. Dómine, intelléxi: omnis consummatiónis vidi finem: latum mandátum tuum nimis. (T. P. Allelúja, allelúja.) \emph{Ps. ibid., 1} Beáti immaculáti in via: qui ámbulant in lege Dómini.
℣. Gloria Patri \emph{\&c.}
}\switchcolumn\portugues{
\rlettrine{E}{speraram-me} os pecadores para me perderem; mas eu, Senhor, tinha meditado nos vossos avisos. Encontrei limites em tudo quanto existe: e só os vossos Mandamentos são infinitos. (T. P. Aleluia, aleluia). \emph{Sl. ibid., 1} Bem-aventurados os que são imaculados em seus caminhos e que cumprem a Lei do Senhor.
℣. Glória ao Pai \emph{\&c.}
}\end{paracol}

\paragraph{Oração}
\begin{paracol}{2}\latim{
\rlettrine{D}{eus,} qui inter cétera poténtiæ tuæ mirácula etiam in sexu frágili victóriam martýrii contulísti: concéde propítius;
ut, qui beátæ {\redx N.} Martyris tuæ natalítia cólimus, per ejus ad te exémpla gradiámur. Per Dóminum \emph{\&c.}
}\switchcolumn\portugues{
\slettrine{Ó}{} Deus, que entre outros milagres do vosso poder permitistes que o sexo frágil alcançasse a vitória do martírio, concedei-nos propício que, venerando nós o nascimento no céu da B Mártir {\redx N.}, caminhemos para Vós, imitando os seus exemplos. Por nosso Senhor \emph{\&c.}
}\end{paracol}

\paragraphinfo{Epístola}{Ecl. 51, 1-8 \& 12}
\begin{paracol}{2}\latim{
Lectio Epístolæ beati Pauli Apostoli ad Corinthios.
}\switchcolumn\portugues{
Lição do Livro da Sabedoria.
}\switchcolumn*\latim{
\rlettrine{C}{onfitébor} tibi, Dómine, Rex, et collaudábo te Deum, Salvatórem meum. Confitébor nómini tuo: quóniam adjútor et protéctor factus es mihi, et liberásti corpus meum a perditióne, a láqueo linguæ iníquæ et a lábiis operántium mendácium, et in conspéctu astántium factus es mihi adjútor. Et liberásti me secúndum multitúdinem misericórdiæ nóminis tui a rugiéntibus, præparátis ad escam, de mánibus quæréntium ánimam meam, et de portis tribulatiónum, quæ circumdedérunt me: a pressúra flammæ, quæ circúmdedit me, et in médio ignis non sum æstuáta: de altitúdine ventris ínferi, et a lingua coinquináta, et a verbo mendácii, a rege iníquo, et a lingua injústa: laudábit usque ad mortem ánima mea Dóminum: quóniam éruis sustinéntes te, et líberas eos de mánibus géntium, Dómine, Deus noster.
}\switchcolumn\portugues{
\qlettrine{Q}{uero} glorificar-Vos, ó Senhor e Rei; quero louvar-Vos, ó Deus, meu salvador. Quero glorificar o vosso nome, porque fostes o meu sustentáculo e protector, e livrastes o meu corpo da perdição, do laço da língua iníqua e dos lábios daqueles que tramam a mentira; e na presença dos meus adversários fostes o meu auxílio. Livrastes-me, segundo a grandeza da misericórdia do vosso Nome, dos que rugiam, prestes a devorar-me; das mãos dos que procuravam tirar-me a vida; e das aflições, que me cercavam. Livrastes-me da violência das chamas, que me cercavam, no meio das quais não senti o calor do fogo. Livrastes-me do abismo profundo do inferno; da língua impura; das palavras mentirosas; do rei iníquo e da língua injusta. Minha alma louvará o Senhor até à morte, porque Vós, Senhor, nosso Deus, livrais dos perigos aqueles que confiam em Vós, salvando-os do poder dos inimigos.
}\end{paracol}

\paragraphinfo{Gradual}{Sl. 44, 8}
\begin{paracol}{2}\latim{
\rlettrine{D}{ilexísti} justítiam, et odísti iniquitátem. ℣. Proptérea unxit te Deus, Deus tuus, óleo lætítiae.
}\switchcolumn\portugues{
\rlettrine{A}{mastes} a justiça e odiastes a iniquidade. Por essa razão, o Senhor, vosso Deus, vos ungiu com o óleo da alegria.
}\switchcolumn*\latim{
Allelúja, allelúja. ℣. \emph{ibid., 5} Spécie tua et pulchritúdine tua inténde, próspere procéde et regna. Allelúja.
}\switchcolumn\portugues{
Aleluia, aleluia. ℣. \emph{ibid., 5} Caminhai, pois, com beleza e com majestade; ide gozar a vitória e reinai. Aleluia.
}\end{paracol}

\textit{Após a Septuagésima omite-se o Aleluia e o seguinte e diz-se:}

\paragraph{Trato}
\begin{paracol}{2}\latim{
\rlettrine{V}{eni,} Sponsa Christi, áccipe corónam, quam tibi Dóminus præparávit in æternum: pro cujus amóre sánguinem tuum fudísti. ℣. \emph{Ps. 44, 8 et 5} Diléxisti justítiam, et odísti iniquitátem: proptérea unxit te Deus, Deus tuus, óleo lætítiae præ consórtibus tuis. ℣. Spécie tua et pulchritúdine tua inténde, próspere procéde et regna.
}\switchcolumn\portugues{
\rlettrine{V}{inde,} ó esposa de Cristo; vinde e recebei a coroa que o Senhor preparou para vós, para a eternidade. Foi por amor dele que derramastes o vosso sangue. ℣. \emph{Sl. 44, 8 et 5} Amastes a justiça e odiastes a iniquidade: eis porque o Senhor, vosso Deus, vos ungiu com o óleo da alegria, de preferência às vossas companheiras. ℣. Caminhai, pois, com beleza e com majestade; ide gozar a vitória e reinai.
}\end{paracol}

\textit{No T. Pascal omite-se o Gradual e o Trato e diz-se:}

\begin{paracol}{2}\latim{
Allelúja, allelúja. ℣. \emph{Ps. 44, 5} Spécie tua et pulchritúdine tua inténde, próspere procéde et regna. Allelúja. ℣. Propter veritátem et mansuetúdinem et justítiam: et dedúcet te mirabíliter déxtera tua. Allelúja.
}\switchcolumn\portugues{
Aleluia, aleluia. ℣. \emph{Sl. 44, 5} Caminhai, pois, com beleza e com majestade; ide gozar a vitória e reinai. Aleluia. ℣. Por causa da vossa verdade, mansidão e justiça, a vossa dextra operará admiráveis prodígios. Aleluia.
}\end{paracol}

\paragraphinfo{Evangelho}{Mt. 13, 44-52}
\begin{paracol}{2}\latim{
\cruz Sequéntia sancti Evangélii secúndum Matthǽum.
}\switchcolumn\portugues{
\cruz Continuação do santo Evangelho segundo S. Mateus
}\switchcolumn*\latim{
\blettrine{I}{n} illo témpore: Dixit Jesus discípulis suis parábolam hanc: Símile est regnum cœlórum thesáuro abscóndito in agro: quem qui invénit homo, abscóndit, et præ gáudio illíus vadit, et vendit univérsa, quæ habet, et emit agrum illum. Iterum símile est regnum cœlórum homini negotiatóri, quærénti bonas margarítas. Invénta autem una pretiósa margaríta, ábiit, et véndidit ómnia, quæ hábuit, et emit eam. Iterum símile est regnum cœlórum sagénæ, missæ in mare et ex omni génere píscium cóngreganti. Quam, cum impléta esset educéntes, et secus litus sedéntes, elegérunt bonos in vasa, malos autem foras misérunt. Sic erit in consummatióne sǽculi: exíbunt Angeli, et separábunt malos de médio justórum, et mittent eos in camínum ignis: ibi erit fletus et stridor déntium. Intellexístis hæc ómnia? Dicunt ei: Etiam. Ait illis: Ideo omnis scriba doctus in regno cœlórum símilis est hómini patrifamílias, qui profert de thesáuro suo nova et vétera.
}\switchcolumn\portugues{
\blettrine{N}{aquele} tempo, disse Jesus aos seus discípulos esta parábola: «O reino dos céus é semelhante a um tesouro escondido no campo, o qual um homem achou e esconde; e, alegre com o achado, vai, vende tudo o que tem e compra o campo. Igualmente o reino dos céus é semelhante a um negociante, que busca pedras preciosas; e, achando uma de grande preço, vai, vende tudo o que tem e compra-a. O reino dos céus é ainda semelhante a uma rede que, lançada ao mar, prende toda a casta de peixes, e, estando já cheia, os pescadores a puxam para a praia, escolhem os bons peixes para os vasos e deitam fora os maus. Assim será no fim dos séculos: virão os Anjos e separarão os maus do meio dos justos e os lançarão na fornalha do fogo, onde não haverá senão fogo e ranger de dentes. Haveis compreendido tudo isto?». «Sim» , responderam eles. E Jesus disse-lhes: «Por esta razão todo o escriba douto, no que diz respeito ao reino dos céus, é semelhante a um pai de família, que tira dos seus tesouros coisas novas e velhas».
}\end{paracol}

\paragraphinfo{Ofertório}{Sl. 44, 3}
\begin{paracol}{2}\latim{
\rlettrine{D}{iffúsa} est grátia in lábiis tuis: proptérea benedíxit te Deus in ætérnum, et in sǽculum sǽculi, allelúja.
}\switchcolumn\portugues{
\rlettrine{A}{} graça espalhou-se nos vossos lábios: eis porque Deus vos abençoou para a eternidade e para todos os séculos dos séculos. (T. P. Aleluia).
}\end{paracol}

\paragraph{Secreta}
\begin{paracol}{2}\latim{
\rlettrine{S}{úscipe,} Dómine, múnera, quæ in beátæ {\redx N.} Martyris tuæ sollemnitáte deférimus: cujus nos confídimus patrocínio liberári. Per Dóminum \emph{\&c.}
}\switchcolumn\portugues{
\rlettrine{R}{ecebei} benigno, Senhor, as ofertas que Vos apresentamos nesta solenidade da vossa B. Mártir {\redx N.}, com o patrocínio da qual esperamos ser livres. Por nosso Senhor \emph{\&c.}
}\end{paracol}

\paragraphinfo{Comúnio}{Sl. 118, 161-162}
\begin{paracol}{2}\latim{
\rlettrine{P}{ríncipes} persecúti sunt me gratis, et a verbis tuis formidávit cor meum: lætábor ego super elóquia tua, quasi qui invénit spólia multa. (T. P. Allelúja.)
}\switchcolumn\portugues{
\rlettrine{O}{s} príncipes perseguiram-me injustamente, mas o meu coração não temeu senão as vossas palavras. Regozijar-me-ei com vossas palavras, como se um homem houvera achado ricos despojos. (T. P. Aleluia.)
}\end{paracol}

\paragraph{Postcomúnio}
\begin{paracol}{2}\latim{
\rlettrine{A}{uxiliéntur} nobis, Dómine, sumpta mystéria: et, intercedénte beáta {\redx N.} Mártyre tua, sempitérna fáciant protectióne gaudére. Per Dóminum nostrum.
\emph{\&c.}
}\switchcolumn\portugues{
\qlettrine{Q}{ue} nos auxiliem os sagrados mistérios que acabámos de receber, Senhor, e que, por intercessão da B. {\redx N.}, vossa Mártir, nos façam gozar continuamente a sua protecção. Por nosso Senhor \emph{\&c.}
}\end{paracol}

\subsectioninfo{Muitas Mártires não Virgens}{Missa Me exspectavérunt}\label{muitasmartiresnaovirgens}

\textit{Como a Missa Precedente, página \pageref{martiresnaovirgens}, excepto o seguinte:}

\paragraph{Oração}
\begin{paracol}{2}\latim{
\rlettrine{D}{a} nobis, quǽsumus, Dómine, Deus noster, sanctárum Mártyrum tuárum {\redx N.} et {\redx N.} palmas incessábili devotióne venerári: ut, quas digna mente non póssumus celebráre, humílibus saltem frequentémus obséquiis. Per Dóminum nostrum \emph{\&c.}
}\switchcolumn\portugues{
\slettrine{Ó}{} Senhor, nosso Deus, dignai-Vos conceder-nos a graça de incessantemente venerarmos com devoção a vitória das vossas santas Mártires {\redx N.} e {\redx N.}, a fim de que, já que não podemos celebrar dignamente os seus méritos, possamos, ao menos, oferecer-lhes as nossas humildes homenagens. Por nosso Senhor \emph{\&c.}
}\end{paracol}

\paragraph{Secreta}
\begin{paracol}{2}\latim{
\rlettrine{I}{nténde,} quǽsumus, Dómine, múnera altáribus tuis pro sanctárum Mártyrum tuárum {\redx N.} et {\redx N.} festivitáte propósita: ut, sicut per hæc beáta mystéria illis glóriam contulísti; ita nobis indulgéntiam largiáris. Per Dóminum \emph{\&c.}
}\switchcolumn\portugues{
\rlettrine{S}{enhor,} dignai-Vos volver os olhares para estas ofertas, que depositamos nos vossos altares para comemorar a festa das vossas santas Mártires {\redx N.} e {\redx N.}, a fim de que, assim como lhes concedestes a glória, assim também nos concedais o perdão dos nossos pecados. Por nosso Senhor \emph{\&c.}
}\end{paracol}

\paragraph{Postcomúnio}
\begin{paracol}{2}\latim{
\rlettrine{P}{ræsta} nobis, quǽsumus, Dómine, intercedéntibus sanctis Martýribus tuis {\redx N.} et {\redx N.}: ut, quod ore contíngimus, pura mente capiámus. Per Dóminum \emph{\&c.}
}\switchcolumn\portugues{
\rlettrine{C}{oncedei-nos,} Senhor, Vos suplicamos, que, por intercessão das vossas santas Mártires {\redx N.} e {\redx N.}, guardemos com o coração puro o que a nossa boca acaba de receber. Por nosso Senhor \emph{\&c.}
}\end{paracol}

\subsectioninfo{Nem Virgens nem Mártires}{Missa Cognóvi, Dómine}\label{nemvirgensnemmartires}

\paragraphinfo{Intróito}{Sl. 118, 75 \& 120}
\begin{paracol}{2}\latim{
\rlettrine{C}{ognóvi,} Dómine, quia ǽquitas judícia tua, et in veritáte tua humiliásti me: confíge timóre tuo carnes meas, a mandátis tuis tímui. (T. P. Allelúja, allelúja.) \emph{Ps. ibid., 1} Beáti immaculáti in via: qui ámbulant in lege Dómini.
℣. Gloria Patri \emph{\&c.}
}\switchcolumn\portugues{
\rlettrine{C}{onheço,} Senhor, que os vossos juízos são equitativos e que me humilhastes com justiça. Esmagai as minhas carnes com vosso temor; os vossos Mandamentos inspiram-me temor. (T. P. Aleluia, aleluia.) \emph{Sl. ibid., 1} Bem-aventurados aqueles que são imaculados nos seus caminhos e cumprem a lei do Senhor.
℣. Glória ao Pai \emph{\&c.}
}\end{paracol}

\paragraph{Oração}
\begin{paracol}{2}\latim{
\rlettrine{E}{xáudi} nos, Deus, salutáris noster: ut, sicut de beátæ {\redx N.} festivitáte gaudémus; ita piæ devotiónis erudiámur afféctu. Per Dóminum \emph{\&c.}
}\switchcolumn\portugues{
\rlettrine{O}{uvi-nos,} ó Deus, nosso salvador, a fim de que, assim como nos alegramos com a festa da vossa B. {\redx N.}, assim também consigamos os afectos duma pia devoção. Por nosso Senhor \emph{\&c.}
}\end{paracol}

\paragraphinfo{Epístola}{Pr. 31, 10-31}
\begin{paracol}{2}\latim{
Léctio libri Sapiéntiæ.
}\switchcolumn\portugues{
Lição do Livro da Sabedoria.
}\switchcolumn*\latim{
\rlettrine{M}{ulíerem} fortem quis invéniet? Procul et de últimis fínibus prétium ejus. Confídit in ea cor viri sui, et spóliis non indigébit. Reddet ei bonum, et non malum, ómnibus diébus vitæ suæ. Quæsívit lanam et linum, et operáta est consílio mánuum suárum. Facta est quasi navis institóris, de longe portans panem suum. Et de nocte surréxit, dedítque prædam domésticis suis, et cibária ancíllis suis. Considerávit agrum, et emit eum: de fructu mánuum suárum plantávit víneam. Accínxit fortitúdine lumbos suos, et roborávit bráchium suum. Gustávit, et vidit, quia bona est negotiátio ejus: non exstinguétur in nocte lucérna ejus. Manum suam misit ad fórtia, et dígiti ejus apprehénderent fusum. Manum suam apéruit ínopi, et palmas suas exténdit ad páuperem. Non timébit dómui suæ a frigóribus nivis: omnes enim doméstici ejus vestíti sunt duplícibus. Stragulátam vestem fecit sibi: byssus et púrpura induméntum ejus. Nóbilis in portis vir ejus, quando séderit cum senatóribus terræ. Síndonem fecit et véndidit, et cíngulum tradidit Chananǽo. Fortitúdo et decor induméntum ejus, et ridébit in die novíssimo. Os suum apéruit sapiéntiæ, et lex cleméntiæ in lingua ejus. Considerávit sémitas domus suæ, et panem otiósa non comédit. Surrexérunt fílii ejus, et beatíssimam prædicavérunt: vir ejus, et laudávit eam. Multæ fíliæ congregavérunt divítias, tu supergréssa es univérsas. Fallax grátia, et vana est pulchritúdo: mulier timens Dóminum, ipsa laudábitur. Date ei de fructu mánuum suárum, et laudent eam in portis ópera ejus.
}\switchcolumn\portugues{
\qlettrine{Q}{uem} encontrará uma mulher forte? Seu valor é maior do que o das pérolas que vêm dos confins do mundo. Nela confia o coração do marido, que por isso lhe não faltará proveito. Ela procurará praticar o bem, e não o mal, em todos os dias da sua vida. Ela procurará a lã e o linho, e, diligentemente, trabalhará neles com suas mãos. Ela é como um navio de mercador que traz de longe o seu pão! Levanta-se quando ainda é noite, e dá alimento à família e trabalho aos servos. Ela pensa em um campo, compra-o e planta uma vinha com o trabalho de suas mãos. Cinge com vigor os seus rins e esforça os seus braços. Examina e vê que seu negócio é bom e que sua lâmpada se não apagará durante a noite. Emprega suas mãos em trabalhos rudes e seus dedos no fuso. Estende a sua mão ao indigente e o seu braço ao pobre. Não receia na sua casa nem o frio, nem a neve, porque todos os criados têm roupa em duplo. Fabrica para si tapeçarias e faz os seus vestidos de bom linho e de púrpura. Seu marido será enobrecido, quando se sentar às Portas da cidade com os senadores da terra. Faz vestidos, que vende, e cintas, que entrega aos mercadores. Revestiu-se de coragem e de glória, e, alegre, aguarda o futuro. Fala com sabedoria e a sua língua é clemente. Vigia os caminhos de sua casa e não come o pão na ociosidade. Erguem-se seus filhos e a proclamam bem-aventurada, e seu marido a louva também: «Muitas filhas reuniram virtudes, mas tu excedeste-las». Enganadora é a graça e vã é a formosura. Só a mulher, que teme o Senhor, será louvada. Dai-lhe o produto de suas mãos, e que a louvem às portas da cidade, por causa das suas obras.
}\end{paracol}

\paragraphinfo{Gradual}{Sl. 44, 3 \& 5}
\begin{paracol}{2}\latim{
\rlettrine{D}{iffúsa} est grátia in labiis tuis: proptérea benedíxit te Deus in ætérnum. ℣. Propter veritátem et mansuetúdinem et justítiam: et de ducet te mirabíliter déxtera tua.
}\switchcolumn\portugues{
\rlettrine{A}{} graça espalhou-se nos vossos lábios: eis porque Deus vos abençoou para a eternidade. ℣. Por causa da vossa verdade, mansidão e justiça, a vossa dextra praticará maravilhas.
}\switchcolumn*\latim{
Allelúja, allelúja. ℣. \emph{ibid., 5} Spécie tua et pulchritúdine tua inténde, próspere procéde et regna. Allelúja.
}\switchcolumn\portugues{
Aleluia, aleluia. ℣. \emph{ibid., 5} Caminhai, pois, com beleza e com majestade; ide gozar a vitória e reinai. Aleluia.
}\end{paracol}

\textit{Após a Septuagésima omite-se o Aleluia e o seguinte e diz-se:}

\paragraph{Trato}
\begin{paracol}{2}\latim{
\rlettrine{V}{eni,} Sponsa Christi, áccipe coronam, quam tibi Dóminus præparávit in ætérnum. ℣. \emph{Ps. 44, 8 \& 5} Dilexísti justítiam, et odísti iniquitátem: proptérea unxit te Deus, Deus tuus, oleo lætítiæ præ consórtibus tuis. ℣. Spécie tua et pulchritúdine tua inténde, próspere procéde et regna.
}\switchcolumn\portugues{
\rlettrine{V}{inde,} ó esposa de Cristo; vinde e recebei a coroa que o Senhor vos preparou para a eternidade. ℣. \emph{Sl. 44, 8 \& 5} Amastes a justiça e odiastes a iniquidade: eis porque o Senhor, vosso Deus, vos ungiu com o óleo da alegria, de preferências às vossas semelhantes. ℣. Caminhai, pois, com beleza e com majestade; ide gozar a vitória e reinai.
}\end{paracol}

\textit{No T. Pascal omite-se o Gradual e o Trato e diz-se:}

\begin{paracol}{2}\latim{
Allelúja, allelúja. ℣. \emph{Ps. 44, 5} Spécie tua et pulchritúdine tua inténde, próspere procéde et regna. Allelúja. ℣. Propter veritátem et mansuetúdinem et justítiam: et dedúcet te mirabíliter déxtera tua. Allelúja.
}\switchcolumn\portugues{
Aleluia, aleluia. ℣. \emph{Sl. 44, 5} Caminhai, pois, com beleza e com majestade; ide gozar a vitória e reinai. ℣. Por causa da vossa verdade, mansidão e justiça, a vossa dextra praticará maravilhas. Aleluia.
}\end{paracol}

\paragraphinfo{Evangelho}{Mt. 13, 44-52}
\begin{paracol}{2}\latim{
\cruz Sequéntia sancti Evangélii secúndum Matthǽum.
}\switchcolumn\portugues{
\cruz Continuação do santo Evangelho segundo S. Mateus.
}\switchcolumn*\latim{
\blettrine{I}{n} illo témpore: Dixit Jesus discípulis suis parábolam hanc: Símile est regnum cœlórum thesáuro abscóndito in agro: quem qui invénit homo, abscóndit, et præ gáudio illíus vadit, et vendit univérsa, quæ habet, et emit agrum illum. Iterum símile est regnum cœlórum homini negotiatóri, quærénti bonas margarítas. Invénta autem una pretiósa margaríta, ábiit, et véndidit ómnia, quæ hábuit, et emit eam. Iterum símile est regnum cœlórum sagénæ, missæ in mare et ex omni génere píscium cóngreganti. Quam, cum impléta esset educéntes, et secus litus sedéntes, elegérunt bonos in vasa, malos autem foras misérunt. Sic erit in consummatióne sǽculi: exíbunt Angeli, et separábunt malos de médio justórum, et mittent eos in camínum ignis: ibi erit fletus et stridor déntium. Intellexístis hæc ómnia? Dicunt ei: Etiam. Ait illis: Ideo omnis scriba doc
tus in regno cœlórum símilis est hómini patrifamílias, qui profert de thesáuro suo nova et vétera.
}\switchcolumn\portugues{
\blettrine{N}{aquele} tempo, disse Jesus aos seus discípulos esta parábola: «O reino dos céus é semelhante a um tesouro escondido no campo, o qual um homem achou e esconde; e, alegre com o achado, vai, vende tudo o que tem e compra o campo. Igualmente o reino dos céus é semelhante a um negociante, que busca pedras preciosas; e, achando uma de grande preço, vai, vende tudo o que tem e compra-a. O reino dos céus é ainda semelhante a uma rede que, lançada ao mar, prende toda a casta de peixes, e, estando já cheia, os pescadores a puxam para a praia, escolhem os bons peixes para os vasos e deitam fora os maus. Assim será no fim dos séculos: virão os Anjos e separarão os maus do meio dos justos e os lançarão na fornalha do fogo, onde não haverá senão fogo e ranger de dentes. Haveis compreendido tudo isto?». «Sim», responderam eles. E Jesus disse-lhes: «Por esta razão todo o escriba douto, a respeito do reino dos céus, é semelhante a um pai de família, que tira dos seus tesouros cousas novas e velhas».
}\end{paracol}

\paragraphinfo{Ofertório}{Sl. 44, 3}
\begin{paracol}{2}\latim{
\rlettrine{D}{iffúsa} est grátia in lábiis tuis: proptérea benedíxit te Deus in ætérnum, et in sǽculum sǽculi, allelúja. (T. P. Allelúja.)
}\switchcolumn\portugues{
\rlettrine{A}{} graça espalhou-se nos vossos lábios; eis porque Deus vos abençoou para a eternidade e para os séculos dos séculos. (T. P. Aleluia.)
}\end{paracol}

\paragraph{Secreta}
\begin{paracol}{2}\latim{
\rlettrine{A}{ccépta} tibi sit, Dómine, sacrátæ plebis oblátio pro tuórum honóre Sanctórum: quorum se méritis de tribulatióne percepísse cognóscit auxílium. Per Dóminum nostrum \emph{\&c.}
}\switchcolumn\portugues{
\rlettrine{A}{ceitai,} Senhor, esta oferta que Vos consagra o vosso povo fiel, em honra dos vossos Santos, pelos méritos dos quais reconhece que tem alcançado a vossa assistência nas tribulações. Por nosso Senhor \emph{\&c.}
}\end{paracol}

\paragraphinfo{Comúnio}{Sl. 44, 8}
\begin{paracol}{2}\latim{
\rlettrine{D}{ilexísti} justítiam, et odísti iniquitátem: proptérea unxit te Deus, Deus tuus, óleo lætítiæ præ consórtibus tuis. (T. P. Allelúja.)
}\switchcolumn\portugues{
\rlettrine{A}{mastes} a justiça e odiastes a iniquidade: eis porque o Senhor, vosso Deus, vos ungiu com o óleo da alegria, de preferência às vossas companheiras. (T. P. Aleluia.)
}\end{paracol}

\paragraph{Postcomúnio}
\begin{paracol}{2}\latim{
\rlettrine{S}{atiásti,} Dómine, famíliam tuam munéribus sacris: ejus, quǽsumus, semper interventióne nos réfove, cujus sollémnia celebrámus. Per Dóminum \emph{\&c.}
}\switchcolumn\portugues{
\rlettrine{H}{avendo} saciado, Senhor, a vossa família com os dons sagrados, dignai-Vos favorecer-nos sempre pela intercessão daquela cuja Solenidade celebrámos. Por nosso Senhor \emph{\&c.}
}\end{paracol}

Outra Epístola (em certos dias):

\paragraphinfo{Epístola}{Pr. 31, 10-31}
\begin{paracol}{2}\latim{
Léctio libri Sapiéntiæ.
}\switchcolumn\portugues{
Lição da Ep.º do B. Ap.º Paulo a Timóteo.
}\switchcolumn*\latim{
\rlettrine{C}{aríssime:} Víduas hónora, quæ vere víduæ sunt. Si qua autem vídua fílios aut nepótes habet, discat primum domum suam régere, et mútuam vicem réddere paréntibus: hoc enim accéptum est coram Deo. Quæ autem vere vídua est et desoláta, speret in Deum, et instet obsecratiónibus et oratiónibus nocte ac die. Nam quæ in delíciis est, vivens mórtua est. Et hoc prǽcipe, ut irreprehensíbiles sint. Si quis autem suórum, et máxime domesticórum curam non habet, fidem negávit, et est infidéli detérior. Vídua eligátur non minus sexagínta annórum, quæ fúerit uníus viri uxor, in opéribus bonis testimónium habens, si fílios educávit, si hospítio recépit, si sanctórum pedes lavit, si tribulatiónem patiéntibus subministrávit, si omne opus bonum subsecúta est.
}\switchcolumn\portugues{
\rlettrine{C}{aríssimos:} Honrai as, viúvas que são verdadeiramente viúvas. Se alguma viúva tem filhos ou netos, ensine-os, primeiramente, a governar a sua casa e a retribuir a seus pais conforme o que havia recebido deles; porque tal é a vontade de Deus. Aquela viúva, que é verdadeiramente viúva e vive só, espere em Deus e persevere noite e dia em suas súplicas e preces. Porém, aquela viúva que vive nas delícias, não está viva, mas sim morta. Fazei-lhes, pois, saber isto, a fim de que sejam irrepreensíveis. Se alguém se não interessa pelos seus, e principalmente pelos de sua casa, nega a fé e é pior do que um infiel. Que a viúva, que for escolhida, não tenha menos de sessenta anos, nem haja tido mais do que um marido, e que tenha reputação de ter praticado boas obras; que tenha educado os seus filhos; praticado a hospitalidade; lavado os pés aos santos; socorrido os aflitos; e, enfim, praticado toda a espécie de boas obras.
}\end{paracol}

\subsectioninfo{Dedicação de uma Igreja}{Missa Terríbilis est}\label{dedicacaoigreja}

\paragraphinfo{Intróito}{Gen. 28, 17}
\begin{paracol}{2}\latim{
\rlettrine{T}{erríbilis} est locus iste: hic domus Dei est et porta cœli: et vocábitur aula Dei. (T. P. Allelúja, allelúja.) \emph{Ps. 83, 2-3} Quam dilécta tabernácula tua, Dómine virtútum! concupíscit, et déficit ánima mea in átria Dómini. ℣. Glória Patri
℣. Gloria Patri \emph{\&c.}
}\switchcolumn\portugues{
\qlettrine{Q}{ue} terrível é este lugar! É verdadeiramente a casa de Deus e a porta do céu: e será chamado o palácio de Deus. (T. P. Aleluia, aleluia.) \emph{Sl. 83, 2-3} Quão dilectos são os vossos tabernáculos, ó Senhor dos exércitos! Minha alma suspira com ardor e em êxtase, em desejos de viver junto dos átrios do Senhor.
℣. Glória ao Pai \emph{\&c.}
}\end{paracol}

\paragraph{Oração}
\begin{paracol}{2}\latim{
\rlettrine{D}{eus,} qui nobis per síngulos annos hujus sancti templi tui consecratiónis réparas diem, et sacris semper mystériis repæséntas incólumes: exáudi preces pópuli tui, et præsta; ut, quisquis hoc templum benefícia petitúrus ingréditur, cuncta se impetrásse lætétur. Per Dóminum \emph{\&c.}
}\switchcolumn\portugues{
\slettrine{Ó}{} Deus, que, anualmente, renovais em nosso favor o dia da consagração deste santo Templo e nos conservais incólumes para podermos sempre celebrar estes sagrados mistérios, ouvi as orações do vosso povo e concedei a todos aqueles que entrarem neste templo, para pedir as vossas graças, a alegria de as alcançarem. Por nosso Senhor \emph{\&c.}
}\end{paracol}

\textit{No dia em que se faz a Dedicação da Igreja e no Oitavário diz-se, em vez da precedente, a seguinte:}

\paragraph{Oração}
\begin{paracol}{2}\latim{
\rlettrine{D}{eus,} qui invisibíliter ómnia cóntines, et tamen pro salúte géneris humáni signa tuæ poténtiæ visibíliter osténdis: templum hoc poténtia tuæ inhabitatiónis illústra, et concéde; ut omnes, qui huc deprecatúri convéniunt, ex quacúmque tribulatióne ad te clamáverint, consolatiónis tuæ benefícia consequántur. Per Dóminum \emph{\&c.}
}\switchcolumn\portugues{
\slettrine{Ó}{} Deus, que, permanecendo invisível, abrangeis, contudo, o universo, e que, entretanto, mostrais visivelmente os milagres do vosso poder para a salvação do género humano, tornai ilustre este Templo, vindo habitar nele com vosso poder, e dignai-Vos conceder a todos aqueles que nele se reunirem para Vos dirigirem suas orações que, em qualquer tribulação em que elevem até Vós os seus clamores, obtenham os benefícios da vossa consolação. Por nosso Senhor \emph{\&c.}
}\end{paracol}

\paragraphinfo{Epístola}{Ap. 21, 2-5.}
\begin{paracol}{2}\latim{
Léctio libri Apocalýpsis beáti Joánnis Apóstoli.
}\switchcolumn\portugues{
Lição do Livro do Apocalipse do B. Ap.º S. João.
}\switchcolumn*\latim{
\rlettrine{I}{n} diébus illis: Vidi sanctam civitátem Jerúsalem novam descendéntem de cœlo a Deo, parátam sicut sponsam ornátam viro suo. Et audívi vocem magnam de throno dicéntem: Ecce tabernáculum Dei cum homínibus, et habitábit cum eis. Et ipsi pópulus ejus erunt, et ipse Deus cum eis erit eórum Deus: et abstérget Deus omnem lácrimam ab óculis eórum: et mors ultra non erit, neque luctus neque clamor neque dolor erit ultra, quia prima abiérunt. Et dixit, qui sedébat in throno: Ecce, nova fácio ómnia.
}\switchcolumn\portugues{
\rlettrine{N}{aqueles} dias, vi a cidade santa, a nova Jerusalém, que vinha de Deus e descia do céu, ornada como uma esposa que se prepara para receber o esposo. E ouvi uma voz forte, que falava do trono, e dizia: «Eis aqui o tabernáculo de Deus no meio dos homens. O Senhor habitará com eles, que serão o seu povo; e o próprio Deus permanecerá com eles e será o seu Deus, enxugando-lhes todas as lágrimas dos seus olhos. Então já não existirão nem a morte, nem as lágrimas, nem os clamores, nem as dores, porque o primeiro estado das coisas terá acabado». E aquele que estava sentado no trono disse: «Eu vou renovar todas as coisas!».
}\end{paracol}

\paragraph{Gradual}
\begin{paracol}{2}\latim{
\rlettrine{L}{ocus} iste a Deo factus est, inæstimábile sacraméntum, irreprehensíbilis est. ℣. Deus, cui astat Angelórum chorus, exáudi preces servórum tuórum.
}\switchcolumn\portugues{
\rlettrine{E}{ste} lugar foi feito por Deus: ele é um mistério inapreciável e isento de qualquer defeito. ℣. Ó Deus, diante de Quem se prostram os caros dos Anjos, ouvi as preces dos vossos servos.
}\switchcolumn*\latim{
Allelúja, allelúja. ℣. \emph{Ps. 137, 2} Adorábo ad templum sanctum tuum: et confitébor nómini tuo. Allelúja.
}\switchcolumn\portugues{
Aleluia, aleluia. ℣. \emph{Sl. 137, 2} Adorar-Vos-ei no vosso santo Templo e louvarei o vosso santo Nome. Aleluia.
}\end{paracol}

\textit{Após a Septuagésima omite-se o Aleluia e o seguinte e diz-se:}

\paragraphinfo{Trato}{Sl. 124, 1-2}
\begin{paracol}{2}\latim{
\qlettrine{Q}{ui} confídunt in Dómino, sicut mons Sion: non commovébitur in ætérnum, qui hábitat in Jerúsalem. ℣. Montes in circúitu ejus, et Dóminus in circúitu pópuli sui, ex hoc nunc, et usque in sǽculum.
}\switchcolumn\portugues{
\rlettrine{A}{queles} que confiam no Senhor são como o monte Sião: aquele que habita em Jerusalém nunca será abalado. ℣. Assim como Jerusalém está rodeada de montanhas, assim o Senhor circunda o seu povo, agora e sempre.
}\end{paracol}

\textit{No T. Pascal omite-se o Gradual e o Trato diz-se:}

\begin{paracol}{2}\latim{
Allelúja, allelúja. ℣. \emph{Ps. 137, 2} Adorábo ad templum sanctum tuum: et confitébor nómini tuo. Allelúja. ℣. Bene fundáta est domus Dómini supra firmam petram. Allelúja.
}\switchcolumn\portugues{
Aleluia, aleluia. ℣. \emph{Sl. 137, 2} Adorar-Vos-ei no vosso santo Templo: e louvarei o vosso santo Nome. ℣. A casa do Senhor está edificada solidamente sobre pedra firme. Aleluia.
}\end{paracol}

\paragraphinfo{Evangelho}{Lc. 19, 1-10}
\begin{paracol}{2}\latim{
\cruz Sequéntia sancti Evangélii secúndum Lucam.
}\switchcolumn\portugues{
\cruz Continuação do santo Evangelho segundo S. Lucas.
}\switchcolumn*\latim{
\blettrine{I}{n} illo témpore: Ingréssus Jesus perambulábat Jéricho. Et ecce, vir nómine Zachǽus: et hic princeps erat publicanórum, et ipse dives: et quærébat vidére Jesum, quis esset: et non póterat præ turba, quia statúra pusíllus erat. Et præcúrrens ascéndit in arborem sycómorum, ut vidéret eum; quia inde erat transitúrus. Et cum venísset ad locum, suspíciens Jesus vidit illum, et dixit ad eum: Zachǽe, féstinans descénde; quia hódie in domo tua opórtet me manére. Et féstinans descéndit, et excépit illum gaudens. Et cum vidérent omnes, murmurábant, dicéntes, quod ad hóminem peccatórem divertísset. Stans au tem Zachǽus, dixit ad Dóminum: Ecce, dimídium bonórum meórum, Dómine, do paupéribus: et si quid áliquem defraudávi, reddo quádruplum. Ait Jesus ad eum: Quia hódie salus dómui huic facta est: eo quod et ipse fílius sit Abrahæ. Venit enim Fílius hóminis quǽrere et salvum fácere, quod períerat.
}\switchcolumn\portugues{
\blettrine{N}{aquele} tempo, havendo Jesus entrado em Jericó, atravessava a cidade. Ora, havia ali um homem, chamado Zaqueu, príncipe dos publicanos e muito rico, que procurava ver Jesus para o conhecer; mas o não conseguia por causa das turbas do povo, pois ele era de baixa estatura. Correu, pois, adiante e subiu para um sicómoro, para ver Jesus, que devia passar por aquele lugar. Chegando Jesus ali, ergueu os olhos e, vendo-o, disse-lhe: «Zaqueu, descei depressa, porque me convém hospedar-me hoje em vossa casa». Imediatamente Zaqueu desceu e recebeu Jesus com alegria. Vendo as pessoas isto, começaram a murmurar, dizendo que Jesus ia hospedar-se em casa de um pecador. Entretanto Zaqueu estava perante o Senhor e dizia-Lhe: «Senhor, eis que vou dar metade dos meus bens aos pobres; e, se defraudei alguém, restituirei o quádruplo». Jesus disse então: «Esta casa recebeu hoje a salvação, pois este também é filho de Abraão. O Filho do homem veio para procurar e salvar o que estava perdido!».
}\end{paracol}

\paragraphinfo{Ofertório}{1. Cr. 29, 17 et 18}
\begin{paracol}{2}\latim{
\rlettrine{D}{ómine} Deus, in simplicitáte cordis mei lætus óbtuli univérsa; et pópulum tuum, qui repertus est, vidi cum ingénti gáudio: Deus Israël, custódi hanc voluntátem, allelúja.
}\switchcolumn\portugues{
\rlettrine{S}{enhor,} meu Deus, foi com simplicidade de coração e com alegria que Vos oferecer todas as coisas: foi com intenso júbilo que vi reunido o vosso povo. Ó Deus de Israel, conservai em minha alma estas boas disposições. Aleluia.
}\end{paracol}

\paragraphinfo{Secreta}{Na Igreja}
\begin{paracol}{2}\latim{
\rlettrine{A}{nnue,} quǽsumus, Dómine, précibus nostris: ut, quicúmque intra templi hujus, cujus anniversárium dedicatiónis diem celebrámus, ámbitum continémur, plena tibi atque perfécta córporis et ánimæ devotióne placeámus; ut, dum hæc vota præséntia réddimus, ad ætérna prǽmia, te adjuvante, perveníre mereámur. Per Dóminum \emph{\&c.}
}\switchcolumn\portugues{
\rlettrine{D}{ignai-Vos,} Senhor, ouvir as nossas orações de modo que todos os que nos encontramos neste santo Templo, de cuja Dedicação celebramos o aniversário, Vos agradecemos com a oferta inteira e perfeita, que Vos fazemos, do nosso corpo e da nossa alma; e permiti que, oferecendo-Vos estes dons, alcancemos a felicidade eterna com vosso auxílio. Por nosso Senhor \emph{\&c.}
}\end{paracol}

\paragraphinfo{Secreta}{Fora da Igreja}
\begin{paracol}{2}\latim{
\rlettrine{A}{nnue,} quǽsumus, Dómine, précibus nostris: ut, dum hæc vota præséntia réddimus, ad ætérna prǽmia, te adjuvánte, perveníre mereámur. Per Dóminum \emph{\&c.}
}\switchcolumn\portugues{
\rlettrine{D}{ignai-Vos,} Senhor, ouvir as nossas orações; e permiti que, oferecendo-Vos estes dons, alcancemos a felicidade eterna com vosso auxílio. Por nosso Senhor \emph{\&c.}
}\end{paracol}

\textit{No dia em que se faz a Dedicação e no seu Oitavário diz-se, em vez da precedente, a seguinte:}

\paragraph{Secreta}
\begin{paracol}{2}\latim{
\rlettrine{D}{eus,} qui sacrandórum tibi auctor es múnerum, effúnde super hanc oratiónis domum benedictiónem tuam: ut ab ómnibus, in ea invocántibus nomen tuum, defensiónis tuæ auxílium se nitátur. Per Dóminum \emph{\&c.}
}\switchcolumn\portugues{
\slettrine{Ó}{} Deus, que sois o autor dos dons que Vos consagramos, lançai a vossa bênção sobre esta casa de oração, a fim de que todos aqueles que aqui invocarem o vosso Nome sintam o auxílio da vossa defesa. Por nosso Senhor \emph{\&c.}
}\end{paracol}

\paragraphinfo{Comúnio}{Mt. 21, 13}
\begin{paracol}{2}\latim{
\rlettrine{D}{omus} mea domus oratiónis vocábitur, dicit Dóminus: in ea omnis, qui pétii, accipit; et qui quærit, invénit; et pulsánti aperiétur. (T. P. Allelúja.)
}\switchcolumn\portugues{
\rlettrine{A}{} minha casa será chamada casa de oração, diz o Senhor: e todo aquele que aí pede, recebe: e o que procura, acha: e ao que bate, abrir-se-lhe-á. (T. P. Aleluia.)
}\end{paracol}

\paragraph{Postcomúnio}
\begin{paracol}{2}\latim{
\rlettrine{D}{eus,} qui de vivis et electis lapídibus ætérnum majestáti tuæ prǽparas habitáculum: auxiliáre pópulo tuo supplicánti; ut, quod Ecclésiæ tuæ corporálibus próficit spátiis, spirituálibus amplificétur augméntis. Per Dóminum nostrum \emph{\&c.}
}\switchcolumn\portugues{
\slettrine{Ó}{} Deus, que preparais para a vossa majestade um Templo de pedras vivas e escolhidas para nele habitardes eternamente, auxiliai o vosso povo suplicante, a fim de que o aumento dos templos materiais faça crescer em proveito da Igreja os seus bens espirituais. Por nosso Senhor \emph{\&c.}
}\end{paracol}

\textit{No dia em que se faz a Dedicação e no seu Oitavário diz-se, em vez do precedente, o seguinte:}

\paragraph{Postcomúnio}
\begin{paracol}{2}\latim{
\qlettrine{Q}{uǽsumus,} omnípotens Deus: ut in hoc loco, quem nómini tuo indígni dedicávimus, cunctis peténtibus aures tuæ pietátis accómmodes. Per Dóminum \emph{\&c.}
}\switchcolumn\portugues{
\rlettrine{D}{ignai-Vos} conceder-nos, ó Deus omnipotente, que neste lugar, que, ainda que indignos, dedicámos ao vosso nome, ouçais benignamente todos os que Vos implorarem. Por nosso Senhor \emph{\&c.}
}\end{paracol}

\textit{Na Festa da Dedicação dum Altar celebra-se a Missa precedente, com excepção do seguinte:}

\paragraph{Oração}
\begin{paracol}{2}\latim{
\rlettrine{D}{eus,} qui ex omni coaptatióne Sanctórum ætérnum tibi condis habitáculum: da ædificatióni tuæ increménta cœléstia; ut, quorum hic relíquias pio amóre compléctimur, eórum semper méritis adjuvémur. Per Dóminum \emph{\&c.}
}\switchcolumn\portugues{
\slettrine{Ó}{} Deus, que, de um composto dos vossos Santos, fundais um templo eterno para vossa habitação, permiti que este palácio celestial tenha um aumento constante, e que os Santos, cujas relíquias honramos neste lugar com amor pio, nos sirvam com seus méritos de perpétuo auxílio. Por nosso Senhor \emph{\&c.}
}\end{paracol}

\paragraph{Secreta}
\begin{paracol}{2}\latim{
\rlettrine{D}{escéndat,} quǽsumus, Dómine, Deus noster, Spíritus tuus Sanctus super hoc altare: qui et pópuli tui dona sanctíficet, et suméntium corda dignánter emúndet. Per Dóminum \emph{\&c.}
}\switchcolumn\portugues{
\qlettrine{Q}{ue} o vosso Espírito Santo desça sobre este altar, Vos suplicamos, ó Senhor, nosso Deus; e que se digne santificar os dons do vosso povo e purificar os corações daqueles que tomarem parte nele. Por nosso Senhor \emph{\&c.}
}\end{paracol}

\paragraph{Postcomúnio}
\begin{paracol}{2}\latim{
\rlettrine{O}{mnípotens} sempitérne Deus, altare hoc, nómini tuo dedicátum, cæléstis virtútis benedictióne sanctífica: et ómnibus in te sperántibus auxílii tui munus osténde; ut et hic sacramentórum viri tus et votórum obtineátur efféctus. Per Dóminum \emph{\&c.}
}\switchcolumn\portugues{
\slettrine{Ó}{} Deus, omnipotente e sempiterno, santificai com a bênção do vosso celestial poder este Altar dedicado ao vosso nome; e a todos aqueles que esperam em Vós concedei o vosso auxílio, a fim de que neste lugar obtenham a virtude dos vossos Sacramentos e o efeito dos seus votos. Por nosso Senhor \emph{\&c.}
}\end{paracol}


\section{Missas da B. V. Maria}

\textit{Podem ser celebradas aos Sábados e Missas Votivas de acordo com os diferentes Tempos.}

\subsectioninfo{Festas da B. V. Maria}{Missa Salve, sancta Parens}\label{comumfestasmaria1}

\paragraphinfo{Intróito}{Sedulius}
\begin{paracol}{2}\latim{
\rlettrine{S}{alve,} sancta Parens, eníxa puérpera Regem: qui cœlum terrámque regit in sǽcula sæculórum. (T. P. Allelúja, allelúja.) \emph{Ps. 44, 2} Eructávit cor meum verbum bonum: dico ego ópera mea Regi.
℣. Gloria Patri \emph{\&c.}
}\switchcolumn\portugues{
\rlettrine{S}{alve,} ó Santa Maria, em cujo seio foi gerado o Rei que governa o céu e a terra em todos os séculos dos séculos (T. P. Aleluia, aleluia.) \emph{Sl. 44, 2} Meu coração exprimiu uma excelente palavra: Consagro ao Rei as minhas obras.
℣. Glória ao Pai \emph{\&c.}
}\end{paracol}

\paragraph{Oração}
\begin{paracol}{2}\latim{
\rlettrine{C}{oncéde} nos fámulos tuos, quǽsumus, Dómine Deus, perpátua mentis et córporis sanitáte gaudére: et, gloriósa beátæ Maríæ semper Vírginis intercessióne, a præsénti liberári tristítia et ætérna pérfrui lætítia. Per Dóminum \emph{\&c.}
}\switchcolumn\portugues{
\rlettrine{S}{enhor} Deus, Vos suplicamos, concedei aos vossos servos o gozo da perpétua saúde da alma e do corpo, e pela gloriosa intercessão da B. Maria, sempre Virgem, permiti que sejamos livres das tristezas do tempo presente e alcancemos o gozo da alegria eterna. Por nosso Senhor \emph{\&c.}
}\end{paracol}

\paragraphinfo{Epístola}{Ecl. 24, 14-16}
\begin{paracol}{2}\latim{
Léctio libri Sapiéntiæ.
}\switchcolumn\portugues{
Lição do Livro da Sabedoria.
}\switchcolumn*\latim{
\rlettrine{A}{b} inítio et ante sǽcula creáta sum, et usque ad futúrum sǽculum non désinam, et in habitatióne sancta coram ipso ministrávi. Et sic in Sion firmáta sum, et in civitáte sanctificáta simíliter requiévi, et in Jerúsalem potéstas mea. Et radicávi in pópulo honorificáto, et in parte Dei mei heréditas illíus, et in plenitúdine sanctórum deténtio mea.
}\switchcolumn\portugues{
\rlettrine{F}{ui} criada desde o princípio, antes de todos os séculos, e não deixarei de existir até à eternidade. Exerci perante Ele o meu ministério; e deste modo tenho habitação fixa em Sião. Ele deixa-me descansar na cidade santa e tenho poder em Jerusalém. Arraiguei-me em um povo glorioso, da parte do meu Deus e da sua herança, e permaneço na companhia dos santos.
}\end{paracol}

\paragraph{Gradual}
\begin{paracol}{2}\latim{
\rlettrine{B}{enedícta} et venerábilis es, Virgo María: quæ sine tactu pudóris invénia es Mater Salvatóris. ℣. Virgo, Dei Génetrix, quem totus non capit orbis, in tua se clausit víscera factus homo.
}\switchcolumn\portugues{
\rlettrine{B}{endita} e venerável sois vós, ó Virgem Maria, que fostes Mãe do Salvador, sem que a vossa pureza sofresse a mais leve ofensa. ℣. Ó Virgem Mãe de Deus, Aquele que nem todo o universo é capaz de conter, esteve encerrado, quando se fez homem, no vosso seio.
}\switchcolumn*\latim{
Allelúja, allelúja. ℣. Post partum, Virgo, invioláta perí mansisti: Dei Génetrix, intercéde pro nobis. Allelúja.
}\switchcolumn\portugues{
Aleluia, aleluia. Depois de haverdes dado à luz, permanecestes Virgem Imaculada: Intercedei por nós, ó Mãe de Deus. Aleluia.
}\end{paracol}

\textit{No Advento, em vez do verso precedente, diz-se:}

\begin{paracol}{2}\latim{
Allelúja, allelúja. ℣. \emph{Luc. 1, 28} Ave, María, grátia plena; Dóminus tecum: benedícta tu in muliéribus. Allelúja.
}\switchcolumn\portugues{
Aleluia, aleluia. ℣. \emph{Lc. 1, 28} Ave, Maria, cheia de graça: o Senhor é convosco: bendita sois vós entre as mulheres. Aleluia.
}\end{paracol}

\textit{Após a Septuagésima omite-se o Aleluia e o seguinte e diz-se:}

\paragraph{Trato}
\begin{paracol}{2}\latim{
\rlettrine{G}{aude,} María Virgo, cunctas hǽreses sola interemísti. ℣. Quæ Gabriélis Archángeli dictis credidísti. ℣. Dum Virgo Deum et hóminem genuísti: et post partum, Virgo, invioláta permansísti. ℣. Dei Génetrix, intercéde pro nobis.
}\switchcolumn\portugues{
\rlettrine{R}{egozijai-vos,} ó Virgem Maria, pois só vós fostes capaz de destruir todas as heresias. ℣. Acreditastes nas palavras do Arcanjo Gabriel. ℣. Sendo Virgem, gerastes o Homem-Deus; e, depois de haverdes dado à luz, permanecestes Virgem Imaculada. ℣. Intercedei por nós, ó Mãe de Deus.
}\end{paracol}

\textit{No T. Pascal omite-se o Gradual e o Trato e diz-se:}

\begin{paracol}{2}\latim{
Allelúja, allelúja. ℣. \emph{Num. 17, 8} Virga Jesse flóruit: Virgo Deum et hóminem génuit: pacem Deus réddidit, in se reconcílians ima summis. Allelúja. ℣. \emph{Luc. 1, 28} Ave, María, grátia plena; Dóminus tecum: benedícta tu in muliéribus. Allelúja.
}\switchcolumn\portugues{
Aleluia, aleluia. ℣. \emph{Nm. 17, 8} A vara de Jessé floresceu: a Virgem deu à luz do mundo o Homem-Deus: restabeleceu Deus a paz, reconciliando na sua pessoa a nossa baixeza com a suprema grandeza. Aleluia. \emph{Lc. 1, 28} Ave, Maria, cheia de graça: o Senhor é convosco: bendita sois vós entre as mulheres. Aleluia.
}\end{paracol}

\paragraphinfo{Evangelho}{Lc. 11, 27-28}
\begin{paracol}{2}\latim{
\cruz Sequéntia sancti Evangélii secúndum Lucam.
}\switchcolumn\portugues{
\cruz Continuação do santo Evangelho segundo S. Lucas.
}\switchcolumn*\latim{
\blettrine{I}{n} illo témpore: Loquénte Jesu ad turbas, extóllens vocem quædam múlier de turba, dixit illi: Beátus venter, qui te portávit, et úbera, quæ suxísti. At ille dixit: Quinímmo beáti, qui áudiunt verbum Dei, et custódiunt illud.
}\switchcolumn\portugues{
\blettrine{N}{aquele} tempo, falando Jesus às turbas, eis que uma mulher, elevando a voz no meio da multidão, Lhe disse: «Bem-aventurado o seio que Vos trouxe; bem-aventurados os peitos que Vos amamentaram!». E Jesus respondeu, dizendo: «Bem-aventurados, antes, aqueles que ouvem a palavra de Deus e a cumprem».
}\end{paracol}

\paragraphinfo{Ofertório}{Lc. 1, 28 \& 42}
\begin{paracol}{2}\latim{
\rlettrine{A}{ve,} María, grátia plena; Dóminus tecum: benedícta tu in muliéribus, et benedíctus fructus ventris tui. (T. P. Allelúja.)
}\switchcolumn\portugues{
\rlettrine{A}{ve} Maria, cheia de graça: o Senhor é convosco: bendita sois vós entre as mulheres, e bendito é o fruto do vosso ventre. (T. P. Aleluia.)
}\end{paracol}

\paragraph{Secreta}
\begin{paracol}{2}\latim{
\rlettrine{T}{ua,} Dómine, propitiatióne, et beátæ Maríæ semper Vírginis intercessióne, ad perpétuam atque præséntem hæc oblátio nobis profíciat prosperitátem et pacem. Per Dóminum \emph{\&c.}
}\switchcolumn\portugues{
\rlettrine{P}{ela} vossa misericórdia, Senhor, e pela intercessão da B. Maria, sempre Virgem, fazei que esta oblação nos assegure a prosperidade e a paz, agora e sempre. Por nosso Senhor \emph{\&c.}
}\end{paracol}

\paragraph{Comúnio}
\begin{paracol}{2}\latim{
\rlettrine{B}{eáta} viscera Maríæ Vírginis, quæ portavérunt ætérni Patris Fílium. (T. P. Allelúja.)
}\switchcolumn\portugues{
\rlettrine{B}{em-aventuradas} as entranhas da Virgem Maria, que trouxeram encerrado o Filho do Pai Eterno. (T. P. Aleluia.)
}\end{paracol}

\paragraph{Postcomúnio}
\begin{paracol}{2}\latim{
\rlettrine{S}{umptis,} Dómine, salútis nostræ subsídiis: da, quǽsumus, beátæ Maríæ semper Vírginis patrocíniis nos ubíque prótegi; in cujus veneratióne hæc tuæ obtúlimus majestáti. Per Dóminum \emph{\&c.}
}\switchcolumn\portugues{
\rlettrine{H}{avendo} nós alcançado o poderoso auxílio da vossa salvação, Senhor, fazei, Vos imploramos, que sejamos protegidos com o patrocínio da B. Maria, sempre Virgem, em cuja honra oferecemos este sacrifício à vossa majestade. Por nosso Senhor \emph{\&c.}
}\end{paracol}

\subsectioninfo{1ª Missa - Durante o Advento}{Missa Roráte cœli da Virgem Maria}\label{missamaria1}

\paragraphinfo{Intróito}{Is. 45, 8}
\begin{paracol}{2}\latim{
\rlettrine{R}{oráte,} cœli, désuper, et nubes pluant justum: aperiátur terra, et gérminet Salvatórem. \emph{Ps. 84, 2} Benedixísti, Domine, terram tuam: avertísti captivitátem Jacob.
℣. Gloria Patri \emph{\&c.}
}\switchcolumn\portugues{
\slettrine{Ó}{} céus, derramai dessas alturas o vosso orvalho: e que as nuvens chovam o Justo! Abra-se a terra e floresça o Salvador! \emph{Sl. 84, 2} Abençoastes, Senhor, a vossa terra e acabastes com a escravidão de Jacob.
℣. Glória ao Pai \emph{\&c.}
}\end{paracol}

\paragraph{Oração}
\begin{paracol}{2}\latim{
\rlettrine{D}{eus,} qui de beátæ Maríæ Vírginis útero Verbum tuum, Angelo nuntiánte, carnem suscípere voluísti: præsta supplícibus tuis; ut, qui vere eam Genetrícem Dei crédimus, ejus apud te intercessiónibus adjuvémur. Per eúndem Dóminum \emph{\&c.}
}\switchcolumn\portugues{
\slettrine{Ó}{} Deus, que, segunda a anunciação do Anjo, quisestes que o vosso Verbo assumisse a carne humana no seio da bem-aventurada Virgem Maria, concedei aos vossos suplicantes que os que crêem que Ela é verdadeira Mãe de Deus, sejam amparados na vossa presença com o auxílio das suas preces. Pelo mesmo nosso Senhor \emph{\&c.}
}\end{paracol}

\paragraphinfo{Epístola}{Is. 7, 10-15}
\begin{paracol}{2}\latim{
Léctio Isaíæ Prophétæ.
}\switchcolumn\portugues{
Lição do Profeta Isaías.
}\switchcolumn*\latim{
\rlettrine{I}{n} diébus illis: Locútus est i Dóminus ad Achaz, dicens: Pete tibi signum a Dómino, Deo tuo, in profúndum inférni, sive in excélsum supra. Et dixit Achaz: Non petam et non tentábo Dóminum. Et dixit: Audíte ergo, domus David: Numquid parum vobis est, moléstos esse homínibus, quia molesti estis et Deo meo? Propter hoc dabit Dóminus ipse vobis signum. Ecce, Virgo concípiet et páriet fílium, et vocábitur nomen ejus Emmánuel. Butýrum et mel cómedet, ut sciat reprobare malum et elígere bonum.
}\switchcolumn\portugues{
\rlettrine{N}{aqueles} dias, falou o Senhor a Acaz e disse-lhe: «Pedi ao Senhor, vosso Deus, um prodígio nas profundezas do inferno ou nas alturas do céu». Acaz respondeu: «Não pedirei tal cousa e não tentarei o Senhor». E Isaías disse: «Escutai, então, casa de David: Porventura vos não basta que fatigueis paciência dos homens, senão que queirais fatigar a do meu Deus? Eis porque o Senhor vos dará um sinal: «Uma virgem conceberá e dará à luz um filho, e o seu nome será Emanuel: Ele comerá manteiga e mel, para que saiba condenar o mal e escolher o bem».
}\end{paracol}

\paragraphinfo{Gradual}{Sl. 23, 7}
\begin{paracol}{2}\latim{
\rlettrine{T}{óllite} portas, príncipes, vestras: et elevámini, portæ æternáles: et introívit Rex glóriæ. ℣. \emph{ibid., 3-4} Quis ascéndet in montem Dómini? aut quis stabit in loco sancto ejus? Innocens mánibus et mundo corde.
}\switchcolumn\portugues{
\rlettrine{A}{bri} inteiramente as vossas frentes, ó portas; abri-vos, ó portas eternas! Então entrará o Rei da glória! ℣. \emph{ibid., 3-4} Quem subirá ao monte do Senhor? Quem permanecerá no seu santuário? Aquele que tiver as mãos inocentes e o coração limpo.
}\switchcolumn*\latim{
Allelúja, allelúja. ℣. \emph{Luc. 1, 28} Ave, María, grátia plena; Dóminus tecum: benedícta tu in muliéribus. Allelúja.
}\switchcolumn\portugues{
Aleluia, aleluia. ℣. \emph{Lc. 1, 28} Ave, Maria, cheia de graça: o Senhor é convosco: bendita sois vós entre as mulheres. Aleluia.
}\end{paracol}

\paragraphinfo{Evangelho}{Lc. 1, 26-38}
\begin{paracol}{2}\latim{
\cruz Sequéntia sancti Evangélii secúndum Lucam.
}\switchcolumn\portugues{
\cruz Continuação do santo Evangelho segundo S. Lucas.
}\switchcolumn*\latim{
\blettrine{I}{n} illo témpore: Missus est Angelus Gábriël a Deo in civitátem Galilǽæ, cui nomen Názareth, ad Vírginem desponsátam viro, cui nomen erat Joseph, de domo David, et nomen Vírginis María. Et ingréssus Angelus ad eam, dixit: Ave, grátia plena; Dóminus tecum: benedícta tu in muliéribus. Quæ cum audísset, turbáta est in sermóne ejus: et cogitábat, qualis esset ista salutátio. Et ait Angelus ei: Ne tímeas, María, invenísti enim grátiam apud Deum: ecce, concípies in útero et páries fílium, et vocábis nomen ejus Jesum. Hic erit magnus, et Fílius Altíssimi vocábitur, et dabit illi Dóminus Deus sedem David, patris ejus: et regnábit in domo Jacob in ætérnum, et regni ejus non erit finis. Dixit autem María ad Angelum: Quómodo fiet istud, quóniam virum non cognósco? Et respóndens Angelus, dixit ei: Spíritus Sanctus supervéniet in te, et virtus Altíssimi obumbrábit tibi. Ideóque et quod nascétur ex te Sanctum, vocábitur Fílius Dei. Et ecce, Elísabeth, cognáta tua, et ipsa concépit fílium in senectúte sua: et hic mensis sextus est illi, quæ vocátur stérilis: quia non erit impossíbile apud Deum omne verbum. Dixit autem María: Ecce ancílla Dómini, fiat mihi secúndum verbum tuum.
}\switchcolumn\portugues{
\blettrine{N}{aquele} tempo, foi mandado por Deus o Anjo Gabriel a uma cidade da Galileia, chamada Nazaré, a uma Virgem, desposada com um varão, cujo nome era José, da casa de David; e o nome da Virgem era Maria. Entrando o Anjo onde ela estava, disse: «Eu te saúdo, cheia de graça: o Senhor é contigo: bendita és tu entre todas as mulheres». Ouvindo ela isto, perturbou-se; e pensava na significação desta saudação. Então, disse-lhe o Anjo: «Não temas, Maria, porquanto alcançaste graça diante do Senhor: eis que conceberás no teu seio e darás à luz um Filho; e o seu nome será Jesus. Ele será grande e será chamado Filho do Altíssimo; o Senhor Deus lhe dará o trono de David, seu pai; reinará eternamente na casa de Jacob; e o seu reino não terá fim. Porém, Maria disse ao Anjo: «Como acontecerá isso, se não conheço varão?», O Anjo, respondendo, disse-lhe: «O Espírito Santo descerá sobre ti, e a virtude do Altíssimo te tocará com sua sombra. Por isso o Santo, que nascer de ti, será chamado Filho de Deus. E eis que Isabel, tua parenta, concebeu um filho na sua velhice: este é o sexto mês daquela que é chamada estéril: porque nada é impossível a Deus». Então disse Maria: «Eis aqui a escrava do Senhor, faça-se em mim segundo a tua palavra».
}\end{paracol}

\paragraphinfo{Ofertório}{Lc. 1, 28 \& 42}
\begin{paracol}{2}\latim{
\rlettrine{A}{ve,} María, grátia plena; Dóminus tecum: benedícta tu in muliéribus, et benedíctus fructus ventris tui.
}\switchcolumn\portugues{
\rlettrine{A}{ve, }Maria, cheia de graça: o Senhor é convosco: bendita sois vós entre as mulheres, e bendito é o fruto do vosso ventre.
}\end{paracol}

\paragraph{Secreta}
\begin{paracol}{2}\latim{
\rlettrine{I}{n} méntibus nostris, quǽsumus, Dómine, veræ fídei sacraménta confírma: ut, qui concéptum de Vírgine Deum verum et hóminem confitémur; per ejus salutíferæ resurrectiónis poténtiam, ad ætérnam mereámur perveníre lætítiam. Per eúndem Dóminum nostrum \emph{\&c.}
}\switchcolumn\portugues{
\rlettrine{D}{ignai-Vos} confirmar nas nossas almas, Senhor, os mystérios da verdadeira fé, a fim de que nós, que confessamos que Aquele que foi concebido pela Virgem Maria é verdadeiro Deus e Homem, mereçamos alcançar pela virtude da sua salutar ressurreição a felicidade eterna. Por nosso Senhor \emph{\&c.}
}\end{paracol}

\paragraphinfo{Comúnio}{Is. 7, 14}
\begin{paracol}{2}\latim{
\rlettrine{E}{cce,} Virgo concípiet et páriet fílium: et vocábitur nomen ejus Emmánuel.
}\switchcolumn\portugues{
\rlettrine{E}{is} que a Virgem conceberá, dará à luz um filho e o seu nome será Emanuel.
}\end{paracol}

\paragraph{Postcomúnio}
\begin{paracol}{2}\latim{
\rlettrine{G}{rátiam} tuam, quǽsumus, Dómine, méntibus nostris infúnde: ut, qui, Angelo nuntiánte, Christi, Fílii tui, incarnatiónem cognóvimus; per passiónem ejus et crucem, ad resurrectiónis glóriam perducámur. Per eúndem Dóminum \emph{\&c.}
}\switchcolumn\portugues{
\rlettrine{I}{nfundi,} Senhor, Vos suplicamos, a vossa graça em nossas almas, para que nós que pela anunciação do Anjo conhecemos a Incarnação do vosso Filho, sejamos conduzidos à glória da ressurreição pela sua Paixão e Cruz. Por nosso Senhor \emph{\&c.}
}\end{paracol}

\subsectioninfo{2.ª Missa - Desde o Natal até à Purificação}{Missa Vultum tuum da Virgem Maria}\label{missamaria2}

\paragraphinfo{Intróito}{Sl. 44, 13,15 \& 16}
\begin{paracol}{2}\latim{
\rlettrine{V}{ultum} tuum deprecabúntur omnes dívites plebis: adducántur Regi Vírgines post eam: próximæ ejus adducéntur tibi in lætítia et exsultatióne. \emph{Ps. ibid., 2} Eructávit cor meum verbum bonum: dico ego ópera mea Regi.
℣. Gloria Patri \emph{\&c.}
}\switchcolumn\portugues{
\rlettrine{T}{odos} os poderosos da terra imploram o vosso olhar; as virgens serão introduzidas perante o Rei após ela: e as suas companheiras serão apresentadas ao Rei, em transportes de alegria e de júbilo. \emph{Sl. ibid., 2} Meu coração exprimiu uma excelente palavra: Consagro ao Rei as minhas obras!
℣. Glória ao Pai \emph{\&c.}
}\end{paracol}

\paragraph{Oração}
\begin{paracol}{2}\latim{
\rlettrine{D}{eus,} qui salútis ætérnæ, beátæ Maríæ virginitáte fœcúnda, humáno generi prǽmia præstitísti: tríbue, quǽsumus; ut ipsam pro nobis intercédere sentiámus, per quam merúimus auctórem vitæ suscípere, Dóminum nostrum Jesum Christum, Fílium tuum: Qui tecum vivit \emph{\&c.}
}\switchcolumn\portugues{
\slettrine{Ó}{} Deus, que pela virgindade fecunda da B. V. Maria concedestes ao género humano o prémio da salvação eterna, fazei, Vos imploramos, que gozemos os efeitos da intercessão daquela pela qual fomos julgados dignos de receber o autor da vida, N. S. Jesus Cristo, vosso Filho: que convosco Vive e reina \emph{\&c.}
}\end{paracol}

\paragraphinfo{Epístola}{Tt. 3, 4-7.}
\begin{paracol}{2}\latim{
Léctio Epístolæ beáti Pauli Apóstoli ad Titum.
}\switchcolumn\portugues{
Lição da Ep.ª do B. Ap.º Paulo a Tito.
}\switchcolumn*\latim{
\rlettrine{C}{aríssime:} Appáruit benígnitas et humánitas Salvatóris nostri Dei: non ex opéribus justítiæ, quæ fécimus nos, sed secúndum suam misericórdiam salvos nos fecit, per lavácrum regeneratiónis et renovatiónis Spíritus Sancti, quem effúdit in nos abúnde per Jesum Christum, Salvatórem nostrum: ut, justificáti grátia ipsíus, herédes simus secúndum spem vitæ ætérnæ: in Christo Jesu, Dómino nostro.
}\switchcolumn\portugues{
\rlettrine{C}{aríssimo:} A bondade e o amor de Deus, nosso Salvador, se manifestaram. Ele salvou-nos, não por causa das obras de justiça que houvéssemos praticado, mas pela sua misericórdia, lavando-nos em um banho de regeneração e de renovação do Espírito Santo, que lançou copiosamente sobre nós por Jesus Cristo, nosso Salvador, a fim de que, justificados pela sua graça, nos tornemos herdeiros da vida eterna, segundo a esperança que depositamos em Jesus Cristo, nosso Senhor.
}\end{paracol}

\paragraphinfo{Gradual}{Sl. 44, 3 \& 2}
\begin{paracol}{2}\latim{
\rlettrine{S}{peciósus} forma præ fíliis hóminum: diffúsa est grátia in lábiis tuis. ℣. Eructávit cor meum verbum bonum: dico ego ópera mea Regi: lingua mea cálamus scribæ velóciter scribéntis.
}\switchcolumn\portugues{
\rlettrine{S}{ois} mais bela do que todos os filhos dos homens: pois a graça espalhou-se nos vossos lábios. ℣. Meu coração exprimiu uma excelente palavra: Consagro ao Rei as minhas obras. Minha língua é como a pena de um escritor perito.
}\switchcolumn*\latim{
Allelúja, allelúja. ℣. Post partum, Virgo, invioláta permansísti: Dei Génetrix, intercéde pro nobis. Allelúja.
}\switchcolumn\portugues{
Aleluia, aleluia. ℣. Depois de haverdes dado à luz, permanecestes Virgem Imaculada. Aleluia.
}\end{paracol}

\textit{Após a Septuagésima omite-se o Aleluia e o seguinte e diz-se:}

\paragraphinfo{Trato}{}
\begin{paracol}{2}\latim{
\rlettrine{G}{aude,} María Virgo, cunctas hǽreses sola interemísti. ℣. Quæ Gabriélis Archángeli dictis credidísti. ℣. Dum Virgo Deum et hóminem genuísti: et post partum, Virgo, invioláta permansísti. ℣. Dei Génetrix, intercéde pro nobis.
}\switchcolumn\portugues{
\rlettrine{R}{egozijai-vos,} ó Virgem Maria, pois só vós fostes capaz de destruir todas as heresias. ℣. Acreditastes nas palavras do Arcanjo Gabriel. ℣. Sendo Virgem, gerastes o Homem-Deus: e, depois de haverdes dado à luz, permanecestes Virgem Imaculada. ℣. Intercedei por nós, ó Mãe de Deus.
}\end{paracol}

\paragraphinfo{Evangelho}{Lc. 2, 15-20}
\begin{paracol}{2}\latim{
\cruz Sequéntia sancti Evangélii secúndum Lucam.
}\switchcolumn\portugues{
\cruz Continuação do santo Evangelho segundo S. Lucas.
}\switchcolumn*\latim{
\blettrine{I}{n} illo témpore: Pastóres loquebántur ad ínvicem: Transeámus usque Béthlehem, et videámus hoc verbum, quod factum est, quod Dóminus osténdit nobis. Et venérunt festinántes, et invenérunt Maríam, et Joseph, et Infántem pósitum in præsépio. Vidéntes autem cognovérunt de verbo, quod dictum erat illis de Púero hoc. Et omnes, qui audiérunt, miráti sunt: et de his, quæ dicta erant a pastóribus ad ipsos. María autem conservábat ómnia verba hæc, cónferens in corde suo. Et revérsi sunt pastores, glorificántes et laudántes Deum in ómnibus, quæ audíerant et víderant, sicut dictum est ad illos.
}\switchcolumn\portugues{
\blettrine{N}{aquele} tempo, disseram os pastores uns aos outros: «Vamos até Belém e vejamos o que foi isto que aconteceu, que o Senhor nos revelou». Vieram, então, a toda a pressa, e encontraram Maria, José e o Menino deitado no presépio. Vendo isto, conheceram a verdade, do que lhes havia sido revelado acerca deste Menino. E todos quantos ouviam falar os pastores ficavam admirados do que eles diziam. Ora Maria conservava todas estas cousas e meditava-as no seu íntimo. E os pastores retiraram-se, glorificando e louvando Deus pelo que tinham visto e ouvido, segundo o que lhes havia sido revelado.
}\end{paracol}

\paragraph{Ofertório}
\begin{paracol}{2}\latim{
\rlettrine{F}{elix} namque es, sacra Virgo María, et omni laude digníssima: quia ex te ortus est sol justítiæ, Christus, Deus noster.
}\switchcolumn\portugues{
\rlettrine{S}{ois} feliz e digna de todos os louvores, ó Santa Virgem Maria, pois de vós nasceu «o sol da justiça», Cristo, nosso Senhor.
}\end{paracol}

\paragraph{Secreta}
\begin{paracol}{2}\latim{
\rlettrine{D}{ómine,} propitiatióne, et beátæ Maríæ semper Vírginis intercessióne, ad perpétuam atque præséntem hæc oblátio nobis profíciat prosperitátem et pacem. Per Dóminum \emph{\&c.}
}\switchcolumn\portugues{
\rlettrine{P}{ela} vossa misericórdia, Senhor, e por intercessão da B. Maria, sempre Virgem, permiti que esta oferta nos assegure agora e sempre a prosperidade e a paz. Por nosso Senhor \emph{\&c.}
}\end{paracol}

\paragraph{Comúnio}
\begin{paracol}{2}\latim{
\rlettrine{B}{eáta} víscera Maríæ Vírginis, quæ portavérunt ætérni Patris Fílium.
}\switchcolumn\portugues{
\rlettrine{B}{em-aventuradas} as entranhas da Virgem Maria, que trouxeram encerrado o Filho do Pai Eterno.
}\end{paracol}

\paragraph{Postcomúnio}
\begin{paracol}{2}\latim{
\rlettrine{H}{æc} nos commúnio, Dómine, purget a crímine: et, intercedénte beáta Vírgine Dei Genetríce María, cœléstis remédii fáciat esse consórtes. Per eúndem Dóminum nostrum \emph{\&c.}
}\switchcolumn\portugues{
\qlettrine{Q}{ue} esta comunhão, Senhor, nos purifique de nossos crimes; e que, por intercessão da B. Virgem Maria, Mãe de Deus, nos torne participantes do remédio celestial. Por nosso Senhor \emph{\&c.}
}\end{paracol}

\subsectioninfo{3.ª Missa - Desde a Purificação até à Páscoa}{Missa Salve, sancta Parens da Virgem Maria}\label{missamaria3}

\paragraphinfo{Intróito}{Sedulius}
\begin{paracol}{2}\latim{
\rlettrine{S}{alve,} sancta Parens, eníxa puérpera Regem: qui cœlum terrámque regit in sǽcula sæculórum. \emph{Ps. 44, 2} Eructávit cor meum verbum bonum: dico ego ópera mea Regi.
℣. Gloria Patri \emph{\&c.}
}\switchcolumn\portugues{
\rlettrine{S}{alve,} ó Santa Maria, em cujo seio foi gerado o Rei que governa o céu e a terra, em todos os séculos dos séculos. \emph{Sl. 44, 2} Meu coração exprimiu uma excelente palavra: Consagro ao Rei as minhas obras!
℣. Glória ao Pai \emph{\&c.}
}\end{paracol}

\paragraph{Oração}
\begin{paracol}{2}\latim{
\rlettrine{C}{oncéde} nos fámulos tuos, quǽsumus, Dómine Deus, perpetua mentis et corporis sanitáte gaudére: et, gloriosa beátæ Maríæ semper Vírginis intercessióne, a præsénti liberári tristitia, et aeterna perfrui lætítia. Per Dóminum nostrum \emph{\&c.}
}\switchcolumn\portugues{
\rlettrine{C}{oncedei} aos vossos servos, Senhor Deus, Vos suplicamos, o gozo da perpétua saúde da alma e do corpo, e que pela gloriosa intercessão da B. Maria, sempre Virgem, sejamos livres das tristezas dos tempos presentes e alcancemos o gozo da eterna alegria. Por nosso Senhor \emph{\&c.}
}\end{paracol}

\paragraphinfo{Epístola}{Ecl. 24, 14-16}
\begin{paracol}{2}\latim{
Léctio libri Sapiéntiæ.
}\switchcolumn\portugues{
Lição do Livro da Sabedoria.
}\switchcolumn*\latim{
\rlettrine{A}{b} inítio et ante sǽcula creáta sum, et usque ad futúrum sǽculum non désinam, et in habitatióne sancta coram ipso ministrávi. Et sic in Sion firmáta sum, et in civitáte sanctificáta simíliter requiévi, et in Jerúsalem potéstas mea. Et radicávi in pópulo honorificáto, et in parte Dei mei heréditas illíus, et in plenitúdine sanctórum deténtio mea.
}\switchcolumn\portugues{
\rlettrine{F}{ui} criada desde o princípio, antes de todos os séculos, e não deixarei de existir até à eternidade. Exerci perante Ele o meu mystério; e deste modo tenho habitação fixa em Sião. Ele deixa-me descansar na cidade santa e tenho poder em Jerusalém. Arraiguei-me em um povo glorioso, da parte do meu Deus e da sua herança, e permaneço na companhia dos santos.
}\end{paracol}

\paragraph{Gradual}
\begin{paracol}{2}\latim{
\rlettrine{B}{enedícta} et venerábilis es, Virgo María: quæ sine tactu pudóris invénia es Mater Salvatóris. ℣. Virgo, Dei Génetrix, quem totus non capit orbis, in tua se clausit víscera factus homo.
}\switchcolumn\portugues{
\rlettrine{B}{endita} e venerável sois, ó Virgem Maria, que fostes Mãe do Salvador sem que a vossa pureza sofresse a mais leve ofensa. ℣. Ó Virgem, Mãe de Deus, Aquele que nem todo o universo é capaz de conter, esteve encerrado, quando se fez homem, no vosso seio.
}\switchcolumn*\latim{
Allelúja, allelúja. ℣. \emph{Num. 17, 8} Virga Jesse flóruit: Virgo Deum et hóminem génuit: pacem Deus réddidit, in se reconcílians ima summis. Allelúja.
}\switchcolumn\portugues{
Aleluia, aleluia. ℣. \emph{Num. 17, 8} A vara de Jessé floresceu: a Virgem deu à luz do mundo o Homem-Deus: restabeleceu Deus a paz, conciliando na sua pessoa a nossa baixeza com sua suprema grandeza!
}\end{paracol}

\textit{Após a Septuagésima omite-se o Aleluia e o seguinte e diz-se:}

\paragraphinfo{Trato}{}
\begin{paracol}{2}\latim{
\rlettrine{G}{aude,} María Virgo, cunctas hǽreses sola interemísti. ℣. Quæ Gabriélis Archángeli dictis credidísti. ℣. Dum Virgo Deum et hóminem genuísti: et post partum, Virgo, invioláta permansísti. ℣. Dei Génetrix, intercéde pro nobis.
}\switchcolumn\portugues{
\rlettrine{R}{egozijai-vos,} ó Virgem Maria, pois só vós fostes capaz de destruir todas as heresias. ℣. Acreditastes nas palavras do Arcanjo Gabriel. ℣. Sendo Virgem, gerastes o Homem-Deus; e, depois de haverdes dado à luz, permanecestes Virgem Imaculada. ℣. Intercedei por nós, ó Mãe de Deus!
}\end{paracol}

\paragraphinfo{Evangelho}{Lc. 11, 27-28}
\begin{paracol}{2}\latim{
\cruz Sequéntia sancti Evangélii secúndum Lucam.
}\switchcolumn\portugues{
\cruz Continuação do santo Evangelho segundo S. Lucas.
}\switchcolumn*\latim{
\blettrine{I}{n} illo témpore: Loquénte Jesu ad turbas, extóllens vocem quædam múlier de turba, dixit illi: Beátus venter, qui te portávit, et úbera, quæ suxísti. At ille dixit: Quinímmo beáti, qui áudiunt verbum Dei, et custódiunt illud.
}\switchcolumn\portugues{
\blettrine{N}{aquele} tempo, falando Jesus às turbas, eis que uma mulher, elevando a voz no meio da multidão, Lhe disse: «Bem-aventurado o seio que Vos trouxe; bem-aventurados os peitos que Vos amamentaram». E Jesus respondeu, dizendo: «Bem-aventurados, antes, aqueles que ouvem a palavra de Deus e a cumprem.»
}\end{paracol}

\paragraph{Ofertório}
\begin{paracol}{2}\latim{
\rlettrine{F}{elix} namque es, sacra Virgo María, et omni laude digníssima: quia ex te ortus est sol justítiæ, Christus, Deus noster.
}\switchcolumn\portugues{
\rlettrine{S}{ois} feliz e digna de todos os louvores, ó Santa Virgem Maria, pois de vós nasceu o «sol da justiça», Cristo, Senhor nosso.
}\end{paracol}

\paragraph{Secreta}
\begin{paracol}{2}\latim{
\rlettrine{T}{ua,} Dómine, propitiatióne, et beátæ Maríæ semper Vírginis intercessióne, ad perpétuam atque præséntem hæc oblátio nobis profíciat prosperitátem et pacem. Per Dóminum \emph{\&c.}
}\switchcolumn\portugues{
\rlettrine{P}{ela} vossa misericórdia, Senhor, e pela intercessão da B. Maria, sempre Virgem, permiti que esta oblação nos assegure agora e sempre a prosperidade e a paz. Por nosso Senhor \emph{\&c.}
}\end{paracol}

\paragraph{Comúnio}
\begin{paracol}{2}\latim{
\rlettrine{B}{eáta} viscera Maríæ Vírginis, quæ portavérunt ætérni Patris Fílium.
}\switchcolumn\portugues{
\rlettrine{B}{em-aventuradas} as entranhas da Virgem Maria, que trouxeram encerrado o Filho do Pai Eterno.
}\end{paracol}

\paragraph{Postcomúnio}
\begin{paracol}{2}\latim{
\rlettrine{S}{umptis,} Dómine, salútis nostræ subsídiis: da, quǽsumus, beátæ Maríæ semper Vírginis patrocíniis nos ubíque prótegi; in cujus veneratióne hæc tuæ obtúlimus majestáti. Per Dóminum \emph{\&c.}
}\switchcolumn\portugues{
\rlettrine{H}{avendo} nós, Senhor, alcançado o poderoso auxílio da nossa salvação, concedei-nos, Vos imploramos, que sejamos protegidos com o patrocínio da B. Maria, sempre Virgem, em cuja honra oferecemos à vossa majestade este sacrifício. Por nosso Senhor \emph{\&c.}
}\end{paracol}

\subsectioninfo{4.ª Missa - Desde a Páscoa até ao Pentecostes}{Missa Salve, sancta Parens da Virgem Maria}\label{missamaria4}

\textit{Como na Missa Precedente, excepto o seguinte:}

\textit{Depois da Epistola diz-se:}

\begin{paracol}{2}\latim{
Allelúja, allelúja. ℣. \emph{Num. 17, 8} Virga Jesse flóruit: Virgo Deum et hóminem génuit: pacem Deus réddidit, in se reconcílians ima summis. Allelúja. ℣. \emph{Luc. 1, 28} Ave, María, grátia plena; Dóminus tecum: benedícta tu in muliéribus. Allelúja.
}\switchcolumn\portugues{
Aleluia, aleluia. ℣. \emph{Nm. 17, 8} A vara de Jessé floresceu: e a Virgem deu à luz o Homem-Deus: restabeleceu Deus a paz, conciliando na sua pessoa a nossa baixeza com sua suprema grandeza. Aleluia. ℣. \emph{Lc. 1, 28} Ave, Maria, cheia de graça: o Senhor é convosco: bendita sois vós entre as mulheres. Aleluia.
}\end{paracol}

\paragraphinfo{Evangelho}{Jo. 19, 25-27}
\begin{paracol}{2}\latim{
\cruz Sequéntia sancti Evangélii secúndum Joánnem.
}\switchcolumn\portugues{
\cruz Continuação do santo Evangelho segundo S. João.
}\switchcolumn*\latim{
\blettrine{I}{n} illo témpore: Stabant juxta Crucem Jesu Mater ejus, et soror Matris ejus, María Cléophæ, et María Magdaléne. Cum vidísset ergo Jesus Matrem, et discípulum stantem, quem diligébat, dicit Matri suæ: Múlier, ecce fílius tuus. Deinde dicit discípulo: Ecce Mater tua. Et ex illa hora accépit eam discípulus in sua.
}\switchcolumn\portugues{
\blettrine{N}{aquele} tempo, estavam, junto à cruz de Jesus, sua Mãe e a irmã de sua Mãe, Maria, mulher de Cléofas, e Maria Madalena. Vendo Jesus sua Mãe e, perto dela, o discípulo Ele preferia, disse a sua Mãe: «Mulher, eis o vosso filho!». Depois disse ao discípulo: «Eis a tua Mãe!». E desde aquela hora levou-a o discípulo consigo.
}\end{paracol}

\paragraph{Ofertório}
\begin{paracol}{2}\latim{
\rlettrine{B}{eáta} es, Virgo María, quæ ómnium portásti Greatórem: genuísti qui te fecit, et in ætérnum pérmanes Virgo, allelúja.
}\switchcolumn\portugues{
\rlettrine{S}{ois} bem-aventurada, ó Virgem Maria, pois fostes digna de trazer em vosso seio o Criador do mundo. Vós gerastes Aquele que vos criou e permanecestes eternamente Virgem. Aleluia.
}\end{paracol}

\subsectioninfo{5.ª Missa - Desde o Pentecostes até ao Advento}{Missa Salve, sancta Parens da Virgem Maria}\label{missamaria5}

\textit{Como na 3.ª Missa, excepto o seguinte:}

\paragraph{Gradual}
\begin{paracol}{2}\latim{
\rlettrine{B}{enedícta} et venerábilis es, Virgo María: quæ sine tactu pudóris invénia es Mater Salvatóris. ℣. Virgo, Dei Génetrix, quem totus non capit orbis, in tua se clausit víscera factus homo.
}\switchcolumn\portugues{
\rlettrine{B}{endita} e venerável sois, ó Virgem Maria, que fostes Mãe do Salvador, sem que a vossa pureza sofresse a mais leve ofensa. ℣. O Virgem, Mãe de Deus, Aquele que nem todo o universo é capaz de conter, quando se fez homem, esteve encerrado no vosso seio.
}\switchcolumn*\latim{
Allelúja, allelúja. ℣. Post partum, Virgo, invioláta permansísti: Dei Génetrix, intercéde pro nobis. Allelúja.
}\switchcolumn\portugues{
Aleluia, aleluia. ℣. Depois de haverdes dado à luz, permanecestes Virgem imaculada: Intercedei por nós, ó Mãe de Deus. Aleluia.
}\end{paracol}

\paragraphinfo{Ofertório}{Lc. 1, 28 \& 42}
\begin{paracol}{2}\latim{
\rlettrine{A}{ve,} María, grátia plena; Dóminus tecum: benedícta tu in muliéribus, et benedíctus fructus ventris tui.
}\switchcolumn\portugues{
\rlettrine{A}{ve,} Maria, cheia de graça: o Senhor é convosco: bendita sois vós entre as mulheres, e bendito é o fruto do vosso ventre.
}\end{paracol}

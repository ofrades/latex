\section{Penitência}

\subsection{Preparação}

\textit{Quem quiser receber dignamente este Sacramento diligenciará conhecer todos e cada um dos seus Pecados. Assim, pois, interiormente, na Presença de Deus, dirá:}

\subsubsection{Oração}
Deus de bondade e de misericórdia, que estais sempre benignamente disposto a acolher os pecadores e a perdoar-lhes, dignai-Vos receber e atender a esta pobre alma que deseja lavar as suas nódoas nas águas salutares do sacramento da Penitência. Concedei-me a graça, ó meu Deus, de me aproximar de Vós com as necessárias disposições: iluminai o meu espírito, para que conheça todos meus pecados; tocai no meu coração, para que os deteste sincera e firmemente; abri a minha boca, para que os confesse integralmente; e fortificai-me a vontade, para que me emende salutarmente. Assim ó meu Deus, e só assim, poderei alcançar o vosso perdão, tão necessário e tão desejado.

\subsubsection{Exame de Consciência}
\paragraph{Oração}
Divino Espírito Santo, concedei-me a graça de conhecer todos meus pecados tão distintamente como os conhecerei quando, ao terminar os dias nesta vida, tiver de comparecer diante de Vós, Para ser julgado. Mostrai-me inteiramente, não só o mal que cometi, mas ainda o bem que omiti. Iluminai-me, fortalecei-me e não permitais que a complacência criminosa, que tenho para comigo, me seduza e cegue até ao ponto de não conhecer as minhas misérias. Não me desampareis neste momento, Senhor; antes, dizei uma só palavra, e terei encontrado o remédio da salvação eterna.

\subsection{Formulário para o Exame}
Há quanto tempo me confessei pela última vez? Recebi a Absolvição, Confessei, tive emenda e contrição de todos meus pecados? Cumpri devidamente a Satisfação Sacramental (a Penitência)? Fiz bem o Exame de Consciência para esta Confissão?

\emph{(Virtude da Fé):} Fui negligente em instruir-me nas verdades da Fé ou não acredito nalguma? Tenho escandalizado o próximo com as minhas dúvidas sobre a Fé? Tenho falado ou escrito contra Deus, contra a Religião, contra a Igreja ou contra o Clero, contra a verdade ou contra a justiça? Tenho lido, assinado, divulgado ou retido livros, jornais ou outras publicações contra a Religião? Acreditei em adivinhas ou sortes de cartas ou noutros sortilégios? Assisti a sessões de espiritismo com mesas falantes, etc.? Frequentei sociedades ímpias ou associações condenadas pela Igreja? Profanei os templos ou as imagens sagradas?

\emph{(Virtude da Esperança):} Tenho desesperado de me salvar ou de corrigir dos meus defeitos? Tenho falta de resignação nas contrariedades e sofrimentos? Murmurei contra a Divina Providência? Tenho tido presunção? Tenho abusado da misericórdia divina? Tenho contado demasiadamente com as minhas forças para alcançar o céu? Tenho confiança em Deus? Resisti às boas inspirações?

\emph{(Virtude da Caridade):} Tenho sido negligente no serviço divino? Tenho feito a Oração da manhã e da noite e recebido com frequência os Sacramentos? Tenho perdido demasiado tempo nas cousas mundanas? Tenho estado com falta de atenção nos templos e escandalizado as pessoas com as minhas atitudes ou vestuário? Comunguei sem a devida preparação? Quando fui comungar ia vestida com decência ou com vestidos mundanos, sem mangas nem meias, largos decotes, transparentes, etc., etc.? Profanei ou ridicularizei as pessoas sagradas? Tenho procurado que aqueles que estão confiados à minha guarda tenham cumprido os seus deveres para com Deus? Faltei à caridade para com o próximo nos pensamentos, palavras e acções? Não socorri os necessitados, nem perdoei aos que me ofenderam? Visitei os doentes e os abandonados? Procurei socorrer o próximo?

\emph{(Sobre os Mandamentos):} Blasfemei contra Deus ou contra os Santos e fui causa ou não evitei que outros dissessem blasfémias? Disse imprecações contra Deus ou pronunciei o seu Nome sem respeito? Jurei falso ou inutilmente, ou fui causa de que outros o fizessem? Disse mentiras, levantei calúnias ou fui causa de que outros assim procedessem? Cumpri com pontualidade os votos, promessas ou juramentos válidos que fiz? Faltei à Missa nos dias do Preceito? Cheguei tarde à Missa ou estive distraído durante ela? Desprezei a pregação do Evangelho e outras recomendações feitas pelo Sacerdote? Fui causa de que meus inferiores faltassem à Missa? Nos dias de Preceito trabalhei ou mandei trabalhar sem necessidade e durante mais de três horas?

\emph{(Deveres dos filhos):} Tive falta de amor e de confiança para com meus pais, avós e superiores? Fui desatencioso, incorrecto ou desobediente para com eles? Tive-lhes aversão, ódio ou repliquei-lhes com Palavras impróprias? Desejei-lhes algum mal, ameacei-os ou bati-lhes? Rezei por eles? Tive vergonha de meus pais? Maltratei, injuriei ou tive inveja de meus irmãos? Não tenho rezado pelas almas dos meus antepassados? Cumpri os testamentos, a que me obriguei?

\emph{(Deveres dos pais):} Faltei ao afecto e à solicitude que devia a meus filhos? Tenho-os repreendido ou castigado com muito rigor ou injustamente? Tenho tido preferência por algum deles? Retardei o seu Baptismo e fui negligente em dar-lhes educação e ensino religioso? Mandei-os à Catequese? Confiei-os a professores ímpios ou a colégios ateus? Fui negligente em procurar-lhes uma carreira, ofício ou profissão? Desviei-os da sua vocação? Não os afastei das más companhias, maus lugares, más leituras, divertimentos perigosos, danças indecentes e modas indecorosas? Tenho-lhcs dado bom exemplo e afastado dos maus exemplos? Não os tenho repreendido quando deixam de cumprir os deveres?

\emph{(Deveres dos casados):} Tenho faltado ao afecto, confiança e mútuo auxílio? Tenho estado com mau humor, ira, rancor ou impaciência? Injuriei ou maltratei o marido ou a esposa? Desobedeci ou não fui condescendente, podendo sê-lo? Faltei aos deveres do Matrimónio e abusei deles? Evitei o nascimento de filhos?

\emph{(Deveres dos inferiores):} Faltei ao respeito e obediência aos superiores? Violei os segredos da família ou dos superiores, e bem assim violei a sua correspondência ou abusei da sua confiança? Fui negligente no cumprimento dos deveres do meu cargo? Gastei dinheiro dos superiores ou dei alguma cousa deles sem a sua autorização? Disse mal deles por ira, inveja ou vingança?

\emph{(Deveres dos superiores):} Consenti na minha casa servos ímpios ou de maus costumes? Não facilitei aos meus servos o cumprimento dos deveres religiosos? Providenciei para que vivessem com moralidade? Fui rude, caprichoso, orgulhoso, injusto ou exigente para com eles? O que lhes era devido lhes não dei, nem lhes paguei os ordenados? Tenho-os alimentado convenientemente?

\emph{(Sobre outras virtudes, deveres e cargos):} Tenho ódio ou má vontade contra o próximo? Desejei-lhe algum mal? Deixei de falar a alguém por ódio ou rancor? Tenho auxiliado o próximo? Causei algum dano ao próximo? Injuriei, feri, bati ou matei alguém, ou mandei outrém maltratar o próximo? Tomei parte nalgum duelo? Promovi discórdias ou incitei alguém ao mal? Tive mau génio ou ira? Dei mau conselho ou escândalo? Comprei ou vendi, espalhei, escrevi ou auxiliei os maus jornais ou outras publicações imorais? Impedi o mal? Prejudiquei de algum modo a minha saúde? Atentei contra a minha vida ou desejei a minha morte? Maltratei os animais?

Consenti pensamentos e desejos contra a castidade? Disse ou ouvi com prazer palavras inconvenientes e entoei ou ouvi entoar canções contra a castidade? Vi gravuras, livros ou jornais imorais? Tive olhares ou familiaridades culpáveis? Estou em perigo de pecar contra a castidade, não querendo evitar a ocasião? Frequentei animatógrafos, teatros, danças e outras diversões contrárias à pureza? Usei modas indecentes ou perigosas, dando ocasião ao próximo de pecar? Tive familiaridade com pessoas doutro sexo? Afastei-me da presença dos meus superiores em companhia de pessoas doutro sexo? Pratiquei acções contrárias à castidade comigo ou com outras pessoas? Fui causa de que outros pecassem contra a castidade?

Cobicei os bens alheios? Prejudiquei o próximo? Cometi algum furto? Fiquei com o que me não pertencia? Não paguei o que devo? Paguei os salários bem como os meus votos? Enganei o próximo em meus negócios: nas medidas, pesos, géneros e preços? Auxiliei, mandei ou consenti furtos? Gastei demais no luxo, no jogo e em comidas e bebidas? Dei esmolas aos pobres? Dediquei demasiado tempo às cousas mundanas com prejuízo dos deveres religiosos e outros do meu estado e cargo? Conservei o que achei, e não procurei restituí-lo ao possuidor?

Fiz juízos temerários do próximo? Disse mal do próximo, justa ou injustamente, até ao ponto de o prejudicar? Incitei outros a caluniarem? Injuriei o próximo com palavras, gestos, maledicência ou desprezo? Disse mentiras com prejuízo doutrem? Descobri faltas ocultas doutrem ou segredos? Li cartas alheias? Murmurei e não evitei que outros murmurassem? Levantei falto testemunho, invocando o Nome de Deus? Levantei calúnias e não reparei os males que causei?

Tenho ido, pelo menos, uma vez em cada ano à Confissão? Quando fui pela última vez? Confessei-me bem?

Comunguei durante o tempo pascal e preferi a Igreja paroquial? Fiz bem a Comunhão? Não teria comungado com algum pecado ou por respeito humano? Vou comungar com frequência?

Tenho transgredido os preceitos do jejum e abstinência sem razão grave e sem necessária dispensa de quem a pode dar? Não me desculpo com falsos motivos para me dispensar do seu cumprimento? Não tenho dado escândalo, mesmo sendo dispensado do cumprimento destes preceitos?

Contribui para a côngrua sustentação do Clero? Auxiliei o culto, principalmente na Igreja paroquial? Dei esmolas para obras religiosas e de acção católica e social? No meu testamento lembrei-me de contemplar a Igreja?

Fui orgulhoso? Desprezei o próximo? Fui teimoso, caprichoso ou susceptível? Fui vaidoso e entretive-me com pensamentos de vaidade? Gastei quantias excessivas por vaidade ou luxo? Desprezei os pobres?

Sou demasiadamente apegado ao dinheiro, às riquezas e aos bens do mundo? Minhas despesas são proporcionais aos meus haveres?

Tenho tido mau humor ou impaciências? Tenho sido violento e tenho tido ódios ou rancores? Tenho sustentado questões e não me reconciliei com meus inimigos por minha culpa?

Comi ou bebi demasiadamente? Abusei das bebidas alcoólicas? Gastei demasiado nas refeições? Tomei alimentos ou bebidas que provoquem a sensualidade?

Regozijei-me com o mal alheio? Prejudiquei o próximo por ter inveja? Entristeci-me por causa do bem que goza o meu próximo? Fui invejoso?

Fui preguiçoso ao levantar-me, no trabalho, no estudo ou noutras ocupações? Tive preguiça de frequentar a Igreja e de receber os Sacramentos? Fui dado à moleza, queixei-me de frio ou de calor? Cumpri o meu regulamento de vida cristã e piedosa? Fui ocioso e perdi o tempo em cousas inúteis? Cumpri os deveres do meu cargo?

\subsection{Depois do Exame de Consciência}

\subsubsection{Acto de dor}

Que confusão a minha, ó meu Deus, ao reconhecer que caí tantas vezes e tão facilmente nos mesmos pecados, e depois de Vos haver prometido emendar-me!... Como tenho tido coragem, ó meu Deus, de cair tantas vezes no pecado, abusando dos benefícios que me concedestes, e demais, conhecendo tão bem quanto o pecado Vos desagrada!... Ó meu Deus e meu Senhor, ó Pai eterno, aplacai a vossa justa ira e não me castigueis segundo o rigor da vossa justiça que tanto mereço. Deixai-Vos aplacar pelos protestos de arrependimento que faço com o coração verdadeiramente contrito, ainda mais por conhecer que estes pecados Vos Ofendem e desagradam, do que por saber que por eles incorri nos castigos das penas eternas. Oh! sim, meu bom Jesus, tenho viva dor de Vos haver ofendido, sendo Vós amável, tão misericordioso!... Perdão, Senhor, por todo o mal que cometi; perdão Por todo o bem que omiti; perdão por todos meus pecados! Quereria reparar essas minhas ingratidões com meu sangue, com a minha própria vida. Oh! Senhor, se eu fora capaz de reparar as minhas faltas com meu arrependimento!... Supri esta minha deficiência com os merecimentos da vossa Agonia no Horto; orvalhai o meu coração com uma gota desse mar de amargura em que o vosso foi, então, abismado; e que, renovado com vossa graça, me sinta penetrado de dor e tristeza mortais!

\subsubsection{Acto de bom Propósito}

Ó meu Deus, eu não deveria viver senão para Vos amar; mas, visto que tenho tido a infelicidade de pecar, faço o propósito bem firme, auxiliado com vossa graça, de nunca mais pecar, vigiando continuamente para evitar o pecado e as suas ocasiões, e particularmente aqueles em que por fraqueza ou malícia tenho o hábito de cair. Quero sinceramente empregar todos os meios que o vosso Ministro me indicar para evitar estas quedas, cumprindo-os, como se fôsseis Vós que os indicásseis. Auxiliai-me, pois, com um pouco dessa vontade que tivestes quando Vos resolvestes a aceitar a morte na Cruz, para salvação da humanidade.

\subsubsection{Acto de Esperança}

Sei, ó Deus, até que ponto Vos tenho ofendido, e o que devia esperar da vossa indignação, se não se interpusesse a vossa misericórdia, e os méritos do meu Salvador não aplacassem a vossa justiça. Ó meu Deus, não recuseis os pedidos que o vosso amabilíssimo Filho Vos dirige por este pecador, e perdoai-me, assim como perdoastes a tantos miseráveis que a Vós recorreram. É com esta esperança que vou comparecer diante do Sagrado Tribunal para me acusar humilde, sincera e inteiramente de todos meus pecados ao vosso Ministro, para de sua boca alcançar em vosso Nome misericórdia e perdão.

\subsubsection{À B. V. Maria e Anjo da Guarda}

Ó Virgem Santíssima, Mãe de graça e de misericórdia, e refúgio seguro dos pecadores, intercedei neste momento por mim, a fim de que a Confissão, que vou fazer, não me torne ainda mais culpável, mas que me alcance perdão de todo o passado e graças necessárias para nunca mais pecar.
Meu bom Anjo, fiel e zeloso guarda da minha alma, que tendes sido testemunha das minhas quedas, consegui que este Sacramento me sirva de remédio para nunca mais cair no pecado. Amen.

\subsection{Acusação dos pecados}

\textit{Após esta Preparação, o Penitente aproximar-se-á com humildade do Confessor e dirá:}

Abençoai-me, Padre, porque pequei!

\textit{O Sacerdote abençoa, dizendo:}

Que o Senhor seja no teu coração e nos teus lábios, para que possas dignamente confessar os teus pecados. Em Nome do Pai, \cruz e do Filho, e do Espírito Santo. Amen.

Recitará, então, o «Eu Pecador me confesso a Deus...», até às Palavras «minha tão grande culpa» (veja página \pageref{confiteor}); ou, ao menos, dirá, «Eu me confesso a Deus omnipotente e a vós, meu Padre»; e logo dirá há quanto tempo se confessou pela última vez; se recebeu a Absolvição; se cumpriu a Penitencia; e se confessou todos seus Pecados. Imediatamente narrará «todos e cada um» dos seus Pecados mesmo sem o Confessor o interrogar. Compenetre-se o Penitente de que tem obrigação de declarar espontâneamente os seus Pecados, não devendo esperar que o Confessor o interrogue. O Penitente tratará o Confessor por Padre. Assim, dir-lhe-á, principiando a narração das suas faltas: Meu Padre, acuso-me de... (etc., etc.). Terminando a acusação dos Pecados, o Penitente acrescentará: Acuso-me também, de todos os pecados da minha vida passada, especialmente... (tal e tal Pecado), pedindo perdão a Deus e a Vós, meu Padre, Penitência e Absolvição, se de tal for digno. Logo, acaba o Eu me confesso... Depois, cala-se e Ouve o que lhe disser o Confessor. Depois dirá:

\subsubsection{Acto de Contrição}
Meu Deus, porque sois infinitamente bom e Vos amo de todo meu coração, pesa-me de Vos ter ofendido; e com o auxílio da vossa divina graça proponho firmemente emendar-me e nunca mais Vos tornar a ofender; peço e espero o perdão das minhas culpas pela vossa infinita misericórdia. Amen.

\textit{Atenda o Penitente bem à Penitência que lhe impuser o Confessor, e cumpra-a antes da Comunhão.}

\subsection{Depois da Confissão}

\subsubsection{Acto de Fé na Absolvição}

Ousarei persuadir-me, ó meu Deus, de que me encontro, presentemente, pela graça da Absolvição sacramental, perdoado e lavado das manchas das minhas culpas?! Sim, ó Deus de bondade, acabo de ser absolvido; e esta sentença de misericórdia restitui-me a vossa amizade, se a recebi com as devidas disposições. É ao vosso preciosíssimo Sangue, ó amável Redentor, que devo a particularíssima graça da minha reconciliação convosco, o que me dá a esperança da eterna salvação.

\subsubsection{Acção de Graças}

Ó minha alma, agradece rendidamente ao Senhor os prodígios da sua misericórdia! É preciso, ó meu Deus, que sejais cheio de infinita indulgência para usardes de tanta liberalidade para com esta pobre criatura!... Como sois bom, ó Senhor, ó Deus, ó Pai!... Mais uma vez tenho esta consoladora experiência! Porém, como poderei testemunhar-Vos o meu reconhecimento? O menos que devo fazer, ó Divino Redentor, é oferecer-Vos hoje e todos os dias da minha vida um sacrifício perene de louvores; engrandecer incessantemente a vossa infinita misericórdia. Eu o faço de todo o coração, ó meu Deus, e o farei até à morte. Sim, durante toda minha vida, louvarei, honrarei, amarei e glorificarei o Senhor, que é terno amante da minha pobre alma e quer levá-la consigo para os gozos da glória incomparável da vida eternal...

\subsubsection{Propósito de nunca pecar}

Ó meu Deus, o benefício que acabais de praticar em meu favor inspira-me tal ódio ao pecado, que me obriga a tomar sinceramente a resolução de nunca mais pecar. Vos prometo, ó Deus, que hei-de esforçar-me tanto quanto possa para mudar de vida. Fortificai, ó Senhor, esta minha resolução; tornai eficaz o propósito que faço de evitar todas as ocasiões de pecado, principalmente daqueles em que tenho caído tantas vezes e desde há tanto tempo! Empregarei todos os meios para os evitar; eu me obrigarei até por meios violentos a conseguir este meu propósito.

Ó Maria, minha boa e terna Mãe, amparo dos mortais, auxílio dos cristãos e refúgio dos pecadores, a Vós recomendo esta resolução, esperando que me auxiliareis a cumpri-la. Amen.

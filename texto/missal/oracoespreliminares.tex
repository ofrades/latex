\section{Orações Preliminares}

\subsection{Preparação para a Missa}

\emph{Antes de se Paramentar, o Sacerdote recita estas Preces:}

\begin{paracol}{2}\latim{
\emph{Ant.} Ne reminiscáris, Dómine, delícta nostra vel paréntum nostrórum, neque vindíctam sumas de peccátis nostris. (T. P. Allelúja.)
}\switchcolumn\portugues{
\emph{Ant.} Não Vos recordeis, Senhor, dos nossos delitos, nem dos delitos de nossos pais; não queirais vingar-Vos dos nossos pecados. (T. P. Aleluia.)
}\end{paracol}

\subsubsection{Salmo 83}
\begin{paracol}{2}\latim{
\qlettrine{Q}{uam} dilécta tabernácula tua, Dómine virtútum: * concupíscit, et déficit ánima mea in átria Dómini.
}\switchcolumn\portugues{
\qlettrine{Q}{uão} amáveis são os vossos tabernáculos, Senhor dos exércitos: * a minha alma suspira e desfalece pelos átrios do Senhor.
}\switchcolumn*\latim{
Cor meum, et caro mea * exsultavérunt in Deum vivum.
}\switchcolumn\portugues{
Meu coração e minha carne * regozijam-se no Deus vivo.
}\switchcolumn*\latim{
Étenim passer invénit sibi domum: * et turtur nidum sibi, ubi ponat pullos suos.
}\switchcolumn\portugues{
De facto, o passarinho acha casa para si: * e a rola um ninho para lá pôr os seus filhinhos.
}\switchcolumn*\latim{
Altária tua, Dómine virtútum: * Rex meus, et Deus meus.
}\switchcolumn\portugues{
Vossos altares, Senhor dos exércitos: * meu Rei e meu Deus.
}\switchcolumn*\latim{
Beáti, qui hábitant in domo tua, Dómine: * in sǽcula sæculórum laudábunt Te.
}\switchcolumn\portugues{
Senhor, bem-aventurados os que moram na vossa casa: * pelos séculos dos séculos Vos louvarão.
}\switchcolumn*\latim{
Beátus vir, cujus est auxílium abs Te: * ascensiónes in corde suo dispósuit, in valle lacrimárum in loco, quem pósuit.
}\switchcolumn\portugues{
Bem-aventurado o varão que de Vós espera auxílio: * preparou elevações no seu coração, neste vale de lágrimas, no lugar que destinou.
}\switchcolumn*\latim{
Étenim benedictiónem dabit legislátor, ibunt de virtúte in virtútem: * vidébitur Deus deórum in Sion.
}\switchcolumn\portugues{
De facto, o legislador dar-lhe-á a sua bênção, caminhará de virtude em virtude: * será visto o Deus dos deuses em Sião.
}\switchcolumn*\latim{
Dómine, Deus virtútum, exáudi oratiónem meam: * áuribus pércipe, Deus Jacob.
}\switchcolumn\portugues{
Senhor, ó Deus dos exércitos, ouvi a minha oração: * prestai ouvidos, ó Deus de Jacob.
}\switchcolumn*\latim{
Protéctor noster, áspice, Deus: * et réspice in fáciem Christi tui:
}\switchcolumn\portugues{
Ó Deus nosso protector, olhai para nós: * e ponde os olhos no rosto de vosso Cristo:
}\switchcolumn*\latim{
Quia mélior est dies una in átriis tuis, * super míllia.
}\switchcolumn\portugues{
Pois é melhor um só dia nos vossos átrios, * que milhares.
}\switchcolumn*\latim{
Elégi abjéctus esse in domo Dei mei: * magis quam habitáre in tabernáculis peccatórum.
}\switchcolumn\portugues{
Preferi ser o último na casa do meu Deus: * a morar nas tendas dos pecadores.
}\switchcolumn*\latim{
Quia misericórdiam, et veritátem díligit Deus: * grátiam et glóriam dabit Dóminus.
}\switchcolumn\portugues{
Pois Deus ama a misericórdia e a verdade: * o Senhor dará a graça e a glória.
}\switchcolumn*\latim{
Non privábit bonis eos, qui ámbulant in innocéntia: * Dómine virtútum, beátus homo, qui sperat in Te.
}\switchcolumn\portugues{
Não privará de seus bens aqueles que andam na inocência: * ó Senhor dos exércitos, bem-aventurado o homem que em Vós espera.
}\end{paracol}


\subsubsection{Salmo 84}
\begin{paracol}{2}\latim{
\rlettrine{B}{enedixísti,} Dómine, terram tuam: * avertísti captivitátem Jacob.
}\switchcolumn\portugues{
\rlettrine{A}{bençoastes,} ó Senhor, a vossa terra: * libertastes Jacob do cativeiro.
}\switchcolumn*\latim{
Remisísti iniquitátem plebis tuæ: * operuísti ómnia peccáta eórum.
}\switchcolumn\portugues{
Perdoastes a iniquidade de vosso povo: * cobristes todos seus pecados.
}\switchcolumn*\latim{
Mitigásti omnem iram tuam: * avertísti ab ira indignatiónis tuæ.
}\switchcolumn\portugues{
Mitigastes toda vossa ira: * suspendestes a raiva de vossa indignação.
}\switchcolumn*\latim{
Convérte nos, Deus, salutáris noster: * et avérte iram tuam a nobis.
}\switchcolumn\portugues{
Convertei-nos, ó Deus, nosso Salvador: * e afastai de nós a vossa ira.
}\switchcolumn*\latim{
Numquid in ætérnum irascéris nobis? * Aut exténdes iram tuam a generatióne in generatiónem?
}\switchcolumn\portugues{
Estareis porventura para sempre irado connosco? * Ou estendereis a vossa ira de geração em geração?
}\switchcolumn*\latim{
Deus, Tu convérsus vivificábis nos: * et plebs tua lætábitur in Te.
}\switchcolumn\portugues{
Ó Deus, voltando-Vos restituir-nos-eis a vida: * e o vosso povo alegrar-se-á em Vós.
}\switchcolumn*\latim{
Osténde nobis, Dómine, misericórdiam tuam: * et salutáre tuum da nobis.
}\switchcolumn\portugues{
Mostrai-nos, ó Senhor, a vossa misericórdia: * e dai-nos a vossa salvação.
}\switchcolumn*\latim{
Audiam quid loquátur in me Dóminus Deus: * quóniam loquétur pacem in plebem suam.
}\switchcolumn\portugues{
Ouvirei o que me disser o Senhor Deus: * porque anunciará Ele a paz ao seu povo.
}\switchcolumn*\latim{
Et super sanctos suos: * et in eos, qui convertúntur ad cor.
}\switchcolumn\portugues{
Aos seus santos: * e àqueles que se convertem de coração.
}\switchcolumn*\latim{
Verúmtamen prope timéntes eum salutáre ipsíus: * ut inhábitet glória in terra nostra.
}\switchcolumn\portugues{
Sim, a sua salvação está perto dos que O temem: * e a glória habitará na nossa terra.
}\switchcolumn*\latim{
Misericórdia, et véritas obviavérunt sibi: * justítia, et pax osculátæ sunt.
}\switchcolumn\portugues{
A misericórdia e a verdade se encontraram: * a justiça e a paz se beijaram.
}\switchcolumn*\latim{
Véritas de terra orta est: * et justítia de cælo prospéxit.
}\switchcolumn\portugues{
A verdade brotou da terra: * e a justiça olhou do céu.
}\switchcolumn*\latim{
Étenim Dóminus dabit benignitátem: * et terra nostra dabit fructum suum.
}\switchcolumn\portugues{
De facto, o Senhor dará a sua bondade: * e a nossa terra produzirá o seu fruto.
}\switchcolumn*\latim{
Justítia ante eum ambulábit: * et ponet in via gressus suos.
}\switchcolumn\portugues{
Adiante d’Ele irá a justiça : * e imprimirá os seus passos no caminho.
}\end{paracol}


\subsubsection{Salmo 85}
\begin{paracol}{2}\latim{
\rlettrine{I}{nclína,} Dómine, aurem tuam, et exáudi me: * quóniam inops, et pauper sum ego.
}\switchcolumn\portugues{
\rlettrine{I}{nclinai,} ó Senhor, o vosso ouvido e ouvi-me: * porque estou carente e pobre.
}\switchcolumn*\latim{
Custódi ánimam meam, quóniam sanctus sum: * salvum fac servum tuum, Deus meus, sperántem in Te.
}\switchcolumn\portugues{
Velai a minha alma, porque sou santo: * salvai, ó meu Deus, o vosso servo, que em Vós espera.
}\switchcolumn*\latim{
Miserére mei, Dómine, quóniam ad Te clamávi tota die: * lætífica ánimam servi tui, quóniam ad Te, Dómine, ánimam meam levávi.
}\switchcolumn\portugues{
Senhor, tende misericórdia de mim, porque a Vós clamei todo dia: * alegrai a alma de vosso servo, porque a Vós, ó Senhor, elevei a minha alma.
}\switchcolumn*\latim{
Quóniam Tu, Dómine, suávis, et mitis: * et multæ misericórdiæ ómnibus invocántibus Te.
}\switchcolumn\portugues{
Porque Vós, ó Senhor, sois suave e manso: * e de muita misericórdia para todos os que Vos invocam.
}\switchcolumn*\latim{
Áuribus pércipe, Dómine, oratiónem meam: * et inténde voci deprecatiónis meæ.
}\switchcolumn\portugues{
Prestai ouvidos, ó Senhor, à minha oração: * e atendei à voz da minha súplica.
}\switchcolumn*\latim{
In die tribulatiónis meæ clamávi ad Te: * quia exaudísti me.
}\switchcolumn\portugues{
No dia da minha tribulação clamei a Vós: * pois me tendes ouvido.
}\switchcolumn*\latim{
Non est símilis tui in diis, Dómine: * et non est secúndum ópera tua.
}\switchcolumn\portugues{
Não há semelhante a Vós nos deuses, ó Senhor: * e conforme vossas obras não há.
}\switchcolumn*\latim{
Omnes gentes quascúmque fecísti, vénient, et adorábunt coram Te, Dómine: * et glorificábunt nomen tuum.
}\switchcolumn\portugues{
Senhor, todas as gentes que criastes virão e prostradas Vos adorarão: * e glorificarão o vosso nome.
}\switchcolumn*\latim{
Quóniam magnus es Tu, et fáciens mirabília: * Tu es Deus solus.
}\switchcolumn\portugues{
Porque Vós sois grande e fazeis maravilhas: * só Vós sois Deus.
}\switchcolumn*\latim{
Deduc me, Dómine, in via tua, et ingrédiar in veritáte tua: * lætétur cor meum ut tímeat nomen tuum.
}\switchcolumn\portugues{
Guiai-me, ó Senhor, pelo vosso caminho e andarei na vossa verdade: * alegre-se o meu coração no temor do vosso nome.
}\switchcolumn*\latim{
Confitébor tibi, Dómine, Deus meus, in toto corde meo, * et glorificábo nomen tuum in ætérnum:
}\switchcolumn\portugues{
Louvar-Vos-ei, ó Senhor meu Deus, com todo meu coração, * e glorificarei eternamente o vosso nome:
}\switchcolumn*\latim{
Quia misericórdia tua magna est super me: * et eruísti ánimam meam ex inférno inferióri.
}\switchcolumn\portugues{
Pois vossa misericórdia é grande para comigo: * e livrastes a minha alma do mais profundo inferno.
}\switchcolumn*\latim{
Deus, iníqui insurrexérunt super me, et synagóga poténtium quæsiérunt ánimam meam: * et non proposuérunt Te in conspéctu suo.
}\switchcolumn\portugues{
Ó Deus, levantaram-se os maus contra mim e atentou contra a minha vida uma reunião de poderosos: * sem que Vos tivessem ante seus olhos presente.
}\switchcolumn*\latim{
Et Tu, Dómine, Deus miserátor et miséricors, * pátiens, et multæ misericórdiæ, et verax,
}\switchcolumn\portugues{
Vós sois, ó Senhor Deus, compassivo e clemente, * paciente, de muita misericórdia e veraz,
}\switchcolumn*\latim{
Réspice in me, et miserére mei, * da impérium tuum púero tuo: et salvum fac fílium ancíllæ tuæ.
}\switchcolumn\portugues{
Olhai para mim e tende piedade de mim, * dai o vosso império ao vosso servo e salvai o filho de vossa serva.
}\switchcolumn*\latim{
Fac mecum signum in bonum, ut vídeant qui odérunt me, et confundántur: * quóniam Tu, Dómine, adjuvísti me, et consolátus es me.
}\switchcolumn\portugues{
Operai em mim sinais de bondade, para que vejam os que me odeiam e sejam confundidos: * porque Vós, Senhor, me tendes socorrido e consolado.
}\end{paracol}


\subsubsection{Salmo 115}
\begin{paracol}{2}\latim{
\rlettrine{C}{rédidi,} propter quod locútus sum: * ego autem humiliátus sum nimis.
}\switchcolumn\portugues{
\rlettrine{A}{creditei,} por isso falei: * contudo, fui grandemente humilhado.
}\switchcolumn*\latim{
Ego dixi in excéssu meo: * Omnis homo mendax.
}\switchcolumn\portugues{
Disse eu no meu êxtase: * todo o homem é mentiroso.
}\switchcolumn*\latim{
Quid retríbuam Dómino, * pro ómnibus, quæ retríbuit mihi?
}\switchcolumn\portugues{
Que darei em retribuição ao Senhor, * por tudo que me deu?
}\switchcolumn*\latim{
Cálicem salutáris accípiam: * et nomen Dómini invocábo.
}\switchcolumn\portugues{
Tomarei o cálice da salvação: * e invocarei o nome do Senhor.
}\switchcolumn*\latim{
Vota mea Dómino reddam coram omni pópulo ejus: * pretiósa in conspéctu Dómini mors sanctórum ejus:
}\switchcolumn\portugues{
Cumprirei os meus votos ao Senhor, ante todo seu povo: * é preciosa aos olhos do Senhor a morte dos seus santos:
}\switchcolumn*\latim{
O Dómine, quia ego servus tuus: * ego servus tuus, et fílius ancíllæ tuæ.
}\switchcolumn\portugues{
Ó Senhor, eu sou vosso servo: * eu sou vosso servo e filho de vossa serva.
}\switchcolumn*\latim{
Dirupísti víncula mea: * tibi sacrificábo hóstiam laudis, et nomen Dómini invocábo.
}\switchcolumn\portugues{
Quebrastes as minhas cadeias: * Vos oferecerei uma hóstia de louvor e invocarei o nome do Senhor.
}\switchcolumn*\latim{
Vota mea Dómino reddam in conspéctu omnis pópuli ejus: * in átriis domus Dómini, in médio tui, Jerúsalem.
}\switchcolumn\portugues{
Cumprirei os meus votos ao Senhor ante todo seu povo: * nos átrios da casa do Senhor, no meio de Vós, ó Jerusalém.
}\end{paracol}


\subsubsection{Salmo 129}
\begin{paracol}{2}\latim{
\rlettrine{D}{e} profúndis clamávi ad Te, Dómine: * Dómine, exáudi vocem meam:
}\switchcolumn\portugues{
\rlettrine{D}{o} profundo clamei a Vós, Senhor: * ó Senhor, escutai a minha voz:
}\switchcolumn*\latim{
Fiant aures tuæ intendéntes, * in vocem deprecatiónis meæ.
}\switchcolumn\portugues{
Estejam atentos os vossos ouvidos, * à voz da minha súplica.
}\switchcolumn*\latim{
Si iniquitátes observáveris, Dómine: * Dómine, quis sustinébit?
}\switchcolumn\portugues{
Se observardes as nossas iniquidades, Senhor: * ó Senhor, quem poderá subsistir?
}\switchcolumn*\latim{
Quia apud Te propitiátio est: * et propter legem tuam sustínui Te, Dómine.
}\switchcolumn\portugues{
Pois em Vós está a clemência: * e devido à vossa lei, Senhor, sustive em Vós.
}\switchcolumn*\latim{
Sustínuit ánima mea in verbo ejus: * sperávit ánima mea in Dómino.
}\switchcolumn\portugues{
Minha alma confia na sua palavra: * esperou a minha alma no Senhor.
}\switchcolumn*\latim{
A custódia matutína usque ad noctem: * speret Israël in Dómino.
}\switchcolumn\portugues{
Desde a vigília matutina até à noite: * espere Israel no Senhor.
}\switchcolumn*\latim{
Quia apud Dóminum misericórdia: * et copiósa apud eum redémptio.
}\switchcolumn\portugues{
Pois no Senhor está a misericórdia: * e há n’Ele abundante redenção.
}\switchcolumn*\latim{
Et ipse rédimet Israël, * ex ómnibus iniquitátibus ejus.
}\switchcolumn\portugues{
Ele mesmo redimirá Israel, * de todas suas iniquidades.
}\end{paracol}


\begin{paracol}{2}\latim{
\emph{Ant.} Ne reminiscáris, Dómine, delícta nostra vel paréntum nostrórum, neque vindíctam sumas de peccátis nostris. (T. P. Allelúja.)
}\switchcolumn\portugues{
\emph{Ant.} Não Vos recordeis, Senhor, dos nossos delitos, nem dos delitos de nossos pais; não queirais vingar-Vos dos nossos pecados. (T. P. Aleluia.)
}\switchcolumn*\latim{
℣. Kyrie, eléison. Christe, eléison. Kyrie, eléison.
}\switchcolumn\portugues{
℣. Senhor, tende piedade de nós. Cristo, tende piedade de nós. Senhor, tende piedade de nós.
}\switchcolumn*\latim{
Pater noster... (secreto usque ad) ℣. Et ne nos indúcas in tentatiónem. ℟. Sed líbera nos a malo.
}\switchcolumn\portugues{
Pai-nosso... (em silêncio). ℣. E não nos deixeis cair em tentação. ℟. Mas livrai-nos do mal.
}\switchcolumn*\latim{
℣. Ego dixit: Dómine, miserére mei. ℟. Sana ánimam meam, quia peccávi tibi.
}\switchcolumn\portugues{
℣. Eu disse: Senhor, tende piedade de mim. ℟. Curai a minha alma, pois pequei contra Vós.
}\switchcolumn*\latim{
℣. Convértere, Dómine, aliquántulum. ℟. Et deprecáre super servos tuos.
}\switchcolumn\portugues{
℣. Senhor, volvei-Vos um pouco para nós. ℟. Enchei-Vos de piedade para com vossos servos.
}\switchcolumn*\latim{
℣. Fiat misericórdia tua, Dómine, super nos. ℟. Quemádmodum sperávimus in te.
}\switchcolumn\portugues{
℣. Senhor, venha a nós a vossa misericórdia. ℟. Pois esperamos em Vós.
}\switchcolumn*\latim{
℣. Sacerdótes tui induántur justítiam. ℟. Et Sancti tui exsúltent.
}\switchcolumn\portugues{
℣. Que os vossos sacerdotes se revistam de justiça. ℟. E os vossos santos exultem de alegria.
}\switchcolumn*\latim{
℣. Ab occúltis meis munda me, Dómine. ℟. Et ab aliénis parce servo tuo.
}\switchcolumn\portugues{
℣. Senhor, lavai-me das minhas faltas ocultas. ℟. E perdoai ao vosso servo as faltas alheias.
}\switchcolumn*\latim{
℣. Dómine, exáudi oratiónem meam. ℟. Et clamor meus ad te véniat.
}\switchcolumn\portugues{
℣. Senhor, ouvi a minha oração. ℟. E que meu clamor chegue até Vós.
}\switchcolumn*\latim{
℣. Dóminus vobíscum.
}\switchcolumn\portugues{
℣. O Senhor seja convosco.
}\switchcolumn*\latim{
℟. E com vosso espírito.
}\switchcolumn\portugues{
℟. E com vosso espírito.
}\switchcolumn*\latim{
\begin{nscenter} Orémus. \end{nscenter}
\switchcolumn
\begin{nscenter} Oremos. \end{nscenter}
}\switchcolumn*\latim{
\rlettrine{A}{ures} tuæ pietátis, mitíssime Deus, inclína précibus nostris, et grátia Sancti Spíritus illúmina cor nostrum: ut tuis mystériis digne ministráre, teque ætérna caritáte dilígere mereámur.
}\switchcolumn\portugues{
\slettrine{Ó}{} Deus clementíssimo, ouvi benignamente as nossas preces e iluminai o nosso coração com a graça do Espírito Santo, a fim de que mereçamos ser dignos ministros dos vossos mystérios e de Vos amarmos com caridade eterna.
}\switchcolumn*\latim{
\rlettrine{D}{eus,} cui omne cor patet et omnis volúntas lóquitur, et quem nullum latet secrétum: purífica per infusiónem Sancti Spíritus cogitatiónes cordis nostri; ut te perfécte dilígere, et digne laudáre mereámur.
}\switchcolumn\portugues{
\slettrine{Ó}{} Deus, que penetrais em todos os corações, conheceis todas as vontades e para quem nada é oculto, purificai com a efusão do vosso Espírito Santo os pensamentos do nosso coração, a fim de que possamos amar-Vos perfeitamente e louvar-Vos dignamente.
}\switchcolumn*\latim{
\rlettrine{U}{re} igne Sancti Spíritus renes nostros et cor nostrum, Dómine: ut tibi casto córpore serviámus, et mundo corde placeámus.
}\switchcolumn\portugues{
\rlettrine{S}{enhor,} queimai com o fogo do vosso Espírito Santo os nossos rins e os nossos corações, a fim de que Vos sirvamos com o corpo casto e Vos alegremos com o coração puro.
}\switchcolumn*\latim{
\rlettrine{M}{entes} nostras, quǽsumus, Dómine, Paráclitus, qui a te procédit, illúminet: et indúcat in omnem, sicut tuus promísit Fílius, veritátem.
}\switchcolumn\portugues{
\qlettrine{Q}{ue} o Divino Paracleto, que procede de Vós, ó Senhor, Vos suplicamos, ilumine os nossos espíritos, e os guie ao conhecimento da verdade, como prometeu o vosso Filho.
}\switchcolumn*\latim{
\rlettrine{A}{dsit} nobis, quǽsumus, Dómine, virtus Spíritus Sancti: quæ et corda nostra cleménter expúrget et ab ómnibus tueátur advérsis.
}\switchcolumn\portugues{
\rlettrine{S}{enhor,} dignai-Vos assistir-nos com a virtude do Espírito Santo, a fim de que Ele purifique clementemente os nossos corações e nos defenda de todas as adversidades.
}\switchcolumn*\latim{
\rlettrine{D}{eus,} qui corda fidélium Sancti Spíritus illustratióne docuísti: da nobis in eódem Spíritu recta sápere; et de ejus semper consolatióne gaudere.
}\switchcolumn\portugues{
\slettrine{Ó}{} Deus, que instruístes os corações dos fiéis com a luz do Espírito Santo, concedei-nos, pelo mesmo Espírito, que amemos o que é recto e gozemos sempre as suas consolações.
}\switchcolumn*\latim{
\rlettrine{C}{onsciéntias} nostras, quǽsumus, Dómine, visitándo purífica: ut véniens Dóminus noster Jesus Christus, Fílius tuus, parátam sibi in nobis invéniat mansiónem: Qui tecum vivit et regnat in unitáte Spíritus Sancti Deus, per ómnia sǽcula sæculórum. Amen.
}\switchcolumn\portugues{
\rlettrine{P}{urificai,} Senhor, Vos imploramos, as nossas consciências com vossa visita, para que, quando N. S. Jesus Cristo, vosso Filho, desça até nós, encontre uma digna morada preparada para Ele: Que convosco vive e reina em unidade do Espírito Santo, Deus, por todos os séculos dos séculos. Amen.
}\end{paracol}

\paragraphinfo{Oração}{Santo Ambrósio}

\begin{paracol}{2}\latim{
\rlettrine{A}{d} mensam dulcíssimi convívii tui, pie Dómine Jesu Christe, ego peccátor de própriis meis méritis nihil præsúmens, sed de tua confídens misericórdia et bonitáte, accédere véreor et contremísco.
}\switchcolumn\portugues{
\slettrine{Ó}{} clementíssimo Senhor Jesus Cristo, eu, indigno pecador, desconfiando profundamente dos meus próprios merecimentos, e confiando absolutamente na vossa misericórdia e bondade, receio e tremo ao aproximar-me da mesa do vosso suavíssimo e dulcíssimo banquete.
}\switchcolumn*\latim{
Nam cor et corpus hábeo multis crimínibus maculátum, mentem et línguam non caute custodítam.
}\switchcolumn\portugues{
É que, Senhor, apesar de Vos haver consagrado o meu coração e o meu corpo, reconheço que muitas vezes os tenho manchado com numerosos pecados, pois não tenho vigiado e guardado cuidadosamente a minha inteligência e as minhas palavras.
}\switchcolumn*\latim{
Ergo, o pia Déitas, o treménda Maiéstas, ego miser, inter angústias deprehénsus, ad te fontem misericórdiæ recúrro, ad te festíno sanándus, sub tuam protectiónem fúgio: et, quem Iúdicem sustinére néqueo, Salvatórem habére suspíro.
}\switchcolumn\portugues{
Eis por que, ó Bondade infinita, ó Majestade incomparável, encontrando-me reduzido ao último extremo da miséria, venho a Vós, que sois a fonte da misericórdia, para ser curado; e, não podendo suportar os rigores do meu Juiz, confio na vossa protecção, ó meu Salvador, e invoco ardentemente as vossas misericórdias.
}\switchcolumn*\latim{
Tibi, Dómine, plagas meas osténdo, tibi verecúndiam meam détego. Scío peccáta mea multa et magna, pro quibus tímeo. Spero in misericórdias tuas, quarum non est númerus.
}\switchcolumn\portugues{
A Vós, Senhor, revelo as minhas chagas; a Vós, Senhor, confesso toda minha vergonha. Sei que meus pecados são grandes e numerosos, o que me enche de temor; mas também confio absolutamente na vossa misericórdia, que sei ser infinita.
}\switchcolumn*\latim{
Réspice ergo in me óculis misericórdiæ tuæ, Dómine Jesu Christe, Rex ætérne, Deus et homo, crucifíxus propter hóminem. Exáudi me sperántem in te: miserére mei pleni misériis et peccátis, tu qui fontem miseratiónis numquam manáre cessábis.
}\switchcolumn\portugues{
Lançai sobre mim os vossos olhares misericordiosos, ó Senhor Jesus, Rei eterno, Deus e homem, que fostes crucificado por causa dos mesmos homens. Escutai-me, pois espero em Vós. Tende piedade de mim, que sou miserável pecador, Vós que nunca deixais de espalhar pela Terra as águas de misericórdia.
}\switchcolumn*\latim{
Salve, salutáris víctima, pro me et omni humáno génere in patíbulo Crucis obláta. Salve, nóbilis et pretióse Sanguis, de vulnéribus crucifíxi Dómini mei Jesu Christi prófluens, et peccáta totíus mundi ábluens.
}\switchcolumn\portugues{
Eu Vos saúdo, ó Vítima da salvação, oferecida no madeiro da Cruz, pelo resgate do género humano e por mim! Eu Vos saúdo, ó Sangue nobre e preciosíssimo brotando das Chagas do Crucificado, meu Senhor Jesus Cristo, e lavando os pecados do mundo inteiro!
}\switchcolumn*\latim{
Recordáre, Dómine, creatúræ tuæ, quam tuo Sánguine redemísti. Pœnitet me peccásse, cúpio emmendáre quod feci.
}\switchcolumn\portugues{
Ó Senhor, lembrai-Vos desta vossa indigníssima criatura que resgatastes com vosso Sangue. Arrependo-me de haver pecado, e desejo ardentemente emendar-me.
}\switchcolumn*\latim{
Aufer ergo a me, clementíssime Pater, omnes iniquitátes et peccáta mea, ut, purificátus mente et córpore, digne degustáre mérear Sancta sanctórum.
}\switchcolumn\portugues{
Arrancai de mim todas minhas iniquidades e pecados, a fim de que, puro de coração e de corpo, possa amar dignamente o Santo dos santos.
}\switchcolumn*\latim{
Et concéde, ut hæc sancta prælibátio Córporis et Sánguinis tui, quam ego indígnus súmere inténdo, sit peccatórum meórum remíssio, sit delictórum perfécta purgátio, sit túrpium cog­tatiónum effugátio, ac bonórum sénsuum regenerátio, operúmque tibi placéntium salúbris efficácia, ánimæ quoque et córporis contra inimicórum meórum insídias firmíssima tuítio. ℟. Amen.
}\switchcolumn\portugues{
Permiti, pela vossa graça, que a Hóstia santíssima do vosso Corpo e Sangue que, apesar de indigno, me preparo para receber, seja para remissão dos meus pecados, purifique-me inteiramente das minhas faltas, afaste de mim os maus pensamentos, desperte na minha alma bons sentimentos, obrigue-me a praticar salutares acções, segundo a vossa vontade, seja, enfim, para minha alma e para meu corpo, um abrigo seguro contra todas as ciladas dos meus inimigos. ℟. Amen.
}\end{paracol}

\paragraphinfo{Oração}{São Tomás Aquino}

\begin{paracol}{2}\latim{
\rlettrine{O}{mnípotens} sempiterne Deus, ecce, accédo ad sacraméntum unigéniti Fílii tui, Dómini nostri Jesu Christi; accédo tamquam infírmus ad médicum vitæ, immúndus ad fontem misericórdiæ, cæcus ad lumen claritátis ætérnæ, pauper et egénus ad Dóminum cæli et terræ.
}\switchcolumn\portugues{
\rlettrine{D}{eus} omnipotente e eterno, eis que me vou aproximar do Sacramento de vosso Filho Unigénito, N. S. Jesus Cristo. E eis que venho como enfermo, ao médico da vida, como manchado, à fonte de misericórdia; como um cego à luz da eterna claridade; e como pobre indigente, ao Senhor do céu e da terra.
}\switchcolumn*\latim{
Rogo ergo imménsæ largitátis tuæ abundántiam, quátenus meam curáre dignéris infirmitátem, laváre fœditátem, illumináre cæcitátem, ditáre paupertátem, vestíre nuditátem: ut panem Angelórum, Regem regum et Dóminum dominántium, tanta suscípiam reveréntia et humilítate, tanta contritióne et devotióne, tanta puritáte et fíde, tali propósito et intentióne, sicut éxpedit salúti ánimæ meæ.
}\switchcolumn\portugues{
Invoco, pois, a abundância de vossas generosidades, que são sem limites, a fim de que Vos digneis curar a minha enfermidade, lavar as minhas máculas, iluminar a minha cegueira, enriquecer a minha pobreza, e vestir a minha nudez, de forma que receba o Pão dos Anjos, o Rei dos reis e o Senhor dos senhores, com tanto respeito e humildade, com uma contrição e uma devoção tão vivas, com uma pureza e uma fé tão grandes, com um bom propósito e uma intenção tais, como o exige a salvação da
minha alma.
}\switchcolumn*\latim{
Da mihi, quǽso, Domínici Córporis et Sánguinis non solum suscípere sacraméntum, sed étiam rem et virtútem sacraménti.
}\switchcolumn\portugues{
Concedei-me, Vos suplico, a graça de receber, não somente o Sacramento do Corpo e do Sangue do Senhor, como também o efeito e a virtude deste Sacramento.
}\switchcolumn*\latim{
O mitíssime Deus, da mihi corpus unigéniti Fílii tui, Dómini nostri Jesu Christi, quod traxit de Vírgine Maria, sic suscípere, ut córpori suo mýstico mérear incorporári, et ínter ejus membra connumerári.
}\switchcolumn\portugues{
Ó Deus clementíssimo, visto que me é dado receber o Corpo de vosso Filho único, N. S. Jesus Cristo, esse Corpo que Ele assumiu no seio da Virgem Maria, fazei que O receba com disposições tão perfeitas que mereça ser incorporado no seu Corpo Místico e contado entre seus
membros.
}\switchcolumn*\latim{
O amantíssime Pater, concede mihi diléctum Fílium tuum, quem nunc velátum in via suscípere propóno, reveláta tandem fácie perpétuo contemplári.
}\switchcolumn\portugues{
Ó Pai amantíssimo, concedei-me, enfim, a graça de contemplar face a face, durante toda a eternidade, o vosso Filho amantíssimo, que me proponho receber hoje, nesta viagem terrestre, debaixo dos véus do Sacramento.
}\switchcolumn*\latim{
Qui tecum vivit et regnat in unitáte Spíritus Sancti Deus, per ómnia sǽcula sæculórum. ℟. Amen.
}\switchcolumn\portugues{
Ele que, sendo Deus, convosco vive e reina, em união com o Espírito Santo, por todos os séculos dos séculos. Amem.
}\end{paracol}

\paragraph{Oração à Santíssima Virgem}

\begin{paracol}{2}\latim{
\rlettrine{O}{} Mater pietátis et misericórdiæ, beatíssima Virgo María, ego miser et indígnus peccátor ad te confúgio toto corde et afféctu; et precor pietátem tuam, ut, sicut dulcíssimo Fílio tuo in Cruce pendénti astítisti, ita et mihi, mísero peccatóri, et sacerdótibus ómnibus, hic et in tota sancta Ecclésia hódie offeréntibus, cleménter assístere dignéris, ut, tua grátia adjúti, dignam et acceptábilem hóstiam in conspéctu summæ et indivíduæ Trinitátis offérre valeámus. Amen.
}\switchcolumn\portugues{
\slettrine{Ó}{} Mãe de bondade e de misericórdia, Santíssima Virgem Maria, eu, miserável e indigno pecador, a Vós recorro de todo o coração e com todo o amor; e Vos suplico que, assim como estivestes de pé junto ao vosso amabilíssimo Filho pendente da Cruz, me assistais também a mim, mísero pecador, e a todos os sacerdotes que hoje na Santa Igreja oferecem o Santo Sacrifício. Auxiliados pela vossa graça, possamos nós apresentar à suprema e indivisível Trindade a Vítima verdadeiramente digna de lhe ser oferecida. Amen.
}\end{paracol}

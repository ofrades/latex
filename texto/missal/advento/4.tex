\subsectioninfo{Quarto Domingo do Advento}{Missa Rorate Coeli}

\paragraphinfo{Intróito}{Is. 45, 8}

\begin{paracol}{2}\latim{
\rlettrine{R}{orate,} coeli, cœli, désuper, et nubes pluant justum: aperiátur terra, et gérminet Salvatórem.
\emph{Ps. 18, 2} Cœli enárrant glóriam Dei: et ópera mánuum ejus annúntiat firmaméntum.
℣. Gloria Patri \emph{\&c.}
}\switchcolumn\portugues{
\slettrine{Ó}{} céus, derramai dessas alturas o vosso orvalho: e que as nuvens chovam o Justo! Abra-se a terra e floresça o Salvador!
\emph{Sl. 18, 2} Os céus proclamam a glória de Deus: e o firmamento anuncia as obras das suas mãos.
℣. Glória ao Pai \emph{\&c.}
}\end{paracol}

\paragraph{Oração}

\begin{paracol}{2}\latim{
\rlettrine{E}{xcita,} quǽsumus, Dómine, poténtiam tuam, et veni: et magna nobis virtúte succúrre; ut per auxílium grátiæ tuæ, quod nostra peccáta præpédiunt, indulgéntiæ tuæ propitiatiónis accéleret: Qui vivis et regnas \emph{\&c.}
}\switchcolumn\portugues{
\rlettrine{M}{anifestai,} Senhor, o vosso poder e «vinde»; e socorrei-nos com vosso infinito poder, a fim de que, com o auxílio da vossa graça, a vossa misericordiosa indulgência se digne apressar a chegada do remédio de que os nossos pecados necessitam. Vós, que, sendo Deus, viveis \emph{\&c.}
}\end{paracol}

\paragraphinfo{Epístola}{1 Cor. 4, 1–5}

\begin{paracol}{2}\latim{
Lectio Epístolæ beati Pauli Apostoli ad Corinthios.
}\switchcolumn\portugues{
Lição da Ep.ª do B. Ap.º Paulo aos Coríntios.
}\switchcolumn*\latim{
\rlettrine{F}{ratres:} Sic nos exístimet homo ut minístros Christi, et dispensatóres mysteriórum Dei. Hic jam quǽritur inter dispensatóres, ut fidélis quis inveniátur. Mihi autem pro mínimo est, ut a vobis júdicer aut ab humano die: sed neque meípsum judico. Nihil enim mihi cónscius sum: sed non in hoc justificátus sum: qui autem júdicat me, Dóminus est. Itaque nolíte ante tempus judicáre, quoadúsque véniat Dóminus: qui et illuminábit abscóndita tenebrárum, et manifestábit consília córdium: et tunc laus erit unicuique a Deo.
}\switchcolumn\portugues{
\rlettrine{M}{eus} irmãos: Que os homens nos considerem como ministros de Cristo e distribuidores dos méritos de Deus. Ora, as qualidades que se deseja que os ministros tenham é que sejam fiéis. Quanto a mim, bem pouco me importa ser julgado por vós ou por um tribunal humano: e nem eu a mim mesmo me julgo. Na verdade, a minha consciência me não repreende de coisa alguma; contudo, nem por isso me julgo justificado, pois o meu juiz é o Senhor. Eis por que não deveis julgar antes do tempo, antes que venha o Senhor, que iluminará o que está nas trevas e manifestará os mais secretos desígnios do coração. Então cada um receberá de Deus a recompensa meritória.
}\end{paracol}

\paragraphinfo{Gradual}{Sl. 144, 18 \& 21}

\begin{paracol}{2}\latim{
\rlettrine{P}{rope} est Dóminus ómnibus invocántibus eum: ómnibus, qui ínvocant eum in veritáte.
℣. Laudem Dómini loquétur os meum: et benedícat omnis caro nomen sanctum ejus.
}\switchcolumn\portugues{
\rlettrine{O}{} Senhor está próximo de todos aqueles que O invocam; de todos aqueles que O invocam com verdade.
℣. Que minha boca publique os louvores do Senhor: e que toda minha pessoa bendiga seu santo Nome.
}\switchcolumn*\latim{
Allelúja, allelúja. ℣. Veni, Dómine, et noli tardáre: reláxa facínora plebis tuæ Israël. Allelúja.
}\switchcolumn\portugues{
Aleluia, aleluia. ℣. Vinde, Senhor, e não retardeis mais: perdoai os crimes de Israel, vosso povo. Aleluia.
}\end{paracol}

\paragraphinfo{Evangelho}{Página \pageref{evangelhosabadotemporasinverno}}

\paragraphinfo{Ofertório}{Lc. 1, 28}

\begin{paracol}{2}\latim{
\rlettrine{A}{ve} María, gratia plena; Dóminus tecum: benedícta tu in muliéribus, et benedíctus fructus ventris tui.
}\switchcolumn\portugues{
\rlettrine{A}{ve,} Maria, cheia de graça: o Senhor é convosco: bendita sois vós entre as mulheres: e bendito é o fruto do vosso ventre.
}\end{paracol}

\paragraph{Secreta}

\begin{paracol}{2}\latim{
\rlettrine{S}{acrificiis} præséntibus, quǽsumus, Dómine, placátus inténde: ut et devotióni nostræ profíciant et salúti. Per Dóminum \emph{\&c.}
}\switchcolumn\portugues{
\rlettrine{S}{enhor,} Vos suplicamos, olhai propício para estes sacrifícios que Vos apresentamos, a fim de que sirvam para aumento da nossa devoção e para conseguirmos a salvação. Por nosso Senhor \emph{\&c.}
}\end{paracol}

\paragraphinfo{Comúnio}{Is. 7, 14}

\begin{paracol}{2}\latim{
\rlettrine{E}{cce} Virgo concípiet et páriet fílium: et vocábitur nomen ejus Emmánuel.
}\switchcolumn\portugues{
\rlettrine{E}{is} que uma Virgem conceberá e dará à luz um filho, que será chamado Emanuel.
}\end{paracol}

\paragraph{Postcomúnio}

\begin{paracol}{2}\latim{
\rlettrine{S}{umptis} munéribus, quǽsumus, Dómine: ut, cum frequentatióne mystérii, crescat nostræ salútis efféctus. Per Dóminum \emph{\&c.}
}\switchcolumn\portugues{
\rlettrine{H}{avendo} nós recebido os vossos dons sacratíssimos, Senhor, Vos suplicamos, dignai-Vos aumentar em nós, pela frequente recepção deste mistério, o efeito da nossa salvação. Por nosso Senhor \emph{\&c.}
}\end{paracol}

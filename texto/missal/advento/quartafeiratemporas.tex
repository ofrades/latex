\subsectioninfo{Quarta-feira das Têmporas do Inverno}{Estação em Santa Maria Maior}

\paragraphinfo{Intróito}{Is. 45, 8}
\begin{paracol}{2}\latim{
\rlettrine{R}{oráte,} cœli, désuper, et nubes pluant justum: aperiátur terra, et gérminet Salvatórem. \emph{Ps. 18, 2} Cœli enárrant glóriam Dei: et ópera mánuum ejus annúntiat firmaméntum.
℣. Gloria Patri \emph{\&c.}
}\switchcolumn\portugues{
\slettrine{Ó}{} céus, derramai dessas alturas o vosso orvalho: e que as nuvens chovam o Justo! Abra-se a terra e floresça o Salvador! \emph{Ps. 18, 2} Os céus proclamam a glória de Deus: e o firmamento anuncia as obras das suas mãos.
℣. Glória ao Pai \emph{\&c.}
}\end{paracol}

\textit{Depois do Kyrie eleison o Padre diz:}

\begin{paracol}{2}\latim{
\begin{nscenter} Orémus. \end{nscenter}
}\switchcolumn\portugues{
\begin{nscenter} Oremos. \end{nscenter}
}\switchcolumn*\latim{
℣. Flectámus génua.
}\switchcolumn\portugues{
℣. Ajoelhemos!
}\switchcolumn*\latim{
℟. Leváte.
}\switchcolumn\portugues{
℟. Levantai-vos!
}\end{paracol}

\paragraph{Oração}
\begin{paracol}{2}\latim{
\rlettrine{P}{ræsta,} quǽsumus, omnípotens Deus: ut redemptiónis nostræ ventúra sollémnitas et præséntis nobis vitæ subsídia cónferat, et ætérnæ beatitúdinis prǽmia largiátur. Per Dóminum \emph{\&c.}
}\switchcolumn\portugues{
\rlettrine{F}{azei,} ó Deus omnipotente, Vos imploramos, que as próximas solenidades da nossa redenção nos alcancem os auxílios para a vida presente e a recompensa da bem-aventurança eterna. Por nosso Senhor \emph{\&c.}
}\end{paracol}

\paragraphinfo{Epístola}{Is. 2. 2–5}
\begin{paracol}{2}\latim{
Léctio Isaíæ Prophétæ.
}\switchcolumn\portugues{
Lição do Profeta Isaías.
}\switchcolumn*\latim{
\rlettrine{I}{n} diébus illis: Dixit Isaias Prophéta: Erit in novíssimis diébus præparátus mons domus Dómini in vértice móntium, et elevábitur super colles, et fluent ad eum omnes gentes. Et ibunt pópuli multi, et dicent: Veníte et ascendámus ad montem Dómini, et ad domum Dei Jacob, et docébit nos vias suas, et ambulábimus in sémitis ejus: quia de Sion exíbit lex, et verbum Dómini de Jerúsalem. Et judicábit gentes, et árguet pópulos multos: et conflábunt gládios suos in vómeres, et lánceas suas in falces. Non levábit gens contra gentem gládium: nec exercebúntur ultra ad prǽlium. Domus Jacob, veníte, et ambulémus in lúmine Dómini, Dei nostri.
}\switchcolumn\portugues{
\rlettrine{N}{aqueles} dias (diz o Profeta Isaías) acontecerá nos últimos tempos que a montanha da casa do Senhor será elevada sobre o pico das montanhas e acima de todas as colinas. E aí virão todos os povos. E muitos virão e dirão: vinde, subamos ao monte do Senhor, à casa de Deus de Jacob, para que nos ensine os seus caminhos e caminhemos pelas suas vias; pois de Sião sairá a lei, e a palavra de Deus sairá de Jerusalém. Ele será o árbitro das nações e arguirá muitos povos, que converterão as suas espadas em charruas, e em foices as suas lanças. Não se levantará nunca mais povo contra povo, nem se exercitarão mais em combater um contra o outro. Casa de Jacob, vinde e caminhemos à luz do Senhor, nosso Deus!
}\end{paracol}

\paragraphinfo{Gradual}{Sl. 23, 7, 3 \& 4}
\begin{paracol}{2}\latim{
\rlettrine{T}{óllite} portas, principes, vestras: et elevámini, portæ æternáles: et introíbit Rex glóriæ, ℣. Quis ascéndet in montem Dómini? aut quis stabit in loco sancto ejus? Innocens mánibus et mundo corde.
}\switchcolumn\portugues{
\slettrine{Ó}{} portas, abri-vos inteiramente! Abri-vos, ó portas eternas! Então, entrará o Rei da glória! ℣. Quem será digno de subir à montanha do Senhor? Quem será digno de permanecer no santuário? Aquele que tiver as mãos inocentes e o coração puro.
}\end{paracol}

\paragraph{Oração}
\begin{paracol}{2}\latim{
\rlettrine{F}{estína,} quǽsumus, Dómine, ne tardáveris, et auxílium nobis supérnæ virtútis impénde: ut advéntus tui consolatiónibus sublevéntur, qui in tua pietáte confídunt: Qui vivis et regnas \emph{\&c.}
}\switchcolumn\portugues{
\rlettrine{A}{pressai-Vos} em nos socorrer, ó Senhor, Vos suplicamos; não tardeis em conceder-nos o auxílio do vosso celestial poder, a fim de que as consolações do vosso advento confortem aqueles que confiam na vossa bondade: Vós, que viveis e reinais \emph{\&c.}
}\end{paracol}

\paragraphinfo{Epístola}{Is. 7, 10–15}
\begin{paracol}{2}\latim{
Lectio Isaíæ Prophétæ.
}\switchcolumn\portugues{
Lição do Profeta Isaías.
}\switchcolumn*\latim{
\rlettrine{I}{n} diébus illis: Locútus est Dóminus ad Achaz, dicens: Pete tibi signum a Dómino, Deo tuo, in profúndum inférni, sive in excélsum supra. Et dixit Achaz: Non petam et non tentábo Dóminum. Et dixit: Audíte ergo, domus David: Numquid parum vobis est, moléstos esse homínibus, quia molesti estis et Deo meo? Propter hoc dabit Dóminus ipse vobis signum. Ecce, Virgo concípiet et páriet fílium, et vocábitur nomen ejus Emmánuel. Butýrum et mel cómedet, ut sciat reprobare malum et elígere bonum.
}\switchcolumn\portugues{
\rlettrine{N}{aqueles} dias, falou o Senhor a Acaz e disse-lhe: «Pedi ao Senhor, vosso Deus, um prodígio nas profundezas do inferno ou nas alturas do céu». Acaz respondeu: «Não pedirei tal coisa e não tentarei o Senhor». E Isaías disse: «Escutai, então, casa de David: Porventura vos não basta que fatigueis a paciência dos homens, senão que queirais fatigar a do meu Deus? Eis porque o Senhor vos dará um sinal: «Uma virgem conceberá e dará à luz um filho, e o seu nome será Emanuel: Ele comerá manteiga e mel, para que saiba condenar o mal e escolher o bem».
}\end{paracol}

\paragraphinfo{Gradual}{Sl. 144, 18 \& 21}
\begin{paracol}{2}\latim{
\rlettrine{P}{rope} est Dóminus ómnibus invocántibus eum: ómnibus qui ínvocant eum in veritáte. ℣. Laudem Dómini loquétur os meum: et benedícat omnis caro nomen sanctum ejus.
}\switchcolumn\portugues{
\rlettrine{O}{} Senhor está perto de todos aqueles que o invocam: de todos que com verdade o invocam. Que minha boca publique os louvores do Senhor: e que toda a carne bendiga seu santo Nome.
}\end{paracol}

\paragraphinfo{Evangelho}{Lc. 1, 26–38}
\begin{paracol}{2}\latim{
\cruz Sequéntia sancti Evangélii secúndum Lucam.
}\switchcolumn\portugues{
\cruz Continuação do santo Evangelho segundo S. Lucas.
}\switchcolumn*\latim{
\blettrine{I}{n} illo tempore: Missus est Angelus Gábriel a Deo in civitatem Galilǽæ, cui nomen Názareth, ad Vírginem desponsátam viro, cui nomen erat Joseph, de domo David, et nomen Vírginis María. Et ingréssus Angelus ad eam, dixit: Ave, grátia plena; Dóminus tecum: benedícta tu in muliéribus. Quæ cum audísset, turbáta est in sermóne ejus: et cogitábat, qualis esset ista salutátio. Et ait Angelus ei: Ne tímeas, María, invenísti enim grátiam apud Deum: ecce, concípies in útero, et páries fílium, et vocábis nomen ejus Jesum. Hic erit magnus, et Fílius Altíssimi vocábitur, et dabit illi Dóminus Deus sedem David, patris ejus: et regnábit in domo Jacob in ætérnum, et regni ejus non erit finis. Dixit autem María ad Angelum: Quómodo fiet istud, quóniam virum non cognósco? Et respóndens Angelus, dixit ei: Spíritus Sanctus supervéniet in te, et virtus Altíssimi obumbrábit tibi. Ideóque et quod nascétur ex te Sanctum, vocábitur Fílius Dei. Et ecce, Elísabeth, cognáta tua, et ipsa concépit fílium in senectúte sua: et hic mensis sextus est illi, quæ vocátur stérilis: quia non erit impossíbile apud Deum omne verbum. Dixit autem María: Ecce ancílla Dómini, fiat mihi secúndum verbum tuum.
}\switchcolumn\portugues{
\blettrine{N}{aquele} tempo, foi mandado por Deus o Anjo Gabriel a uma cidade da Galileia, chamada Nazaré, a uma Virgem, desposada com um varão, cujo nome era José, da casa de David; e o nome da Virgem era Maria. Entrando o Anjo onde ela estava, disse: «Eu te saúdo, cheia de graça: o Senhor é contigo: bendita és tu entre todas as mulheres». Ouvindo ela isto, perturbou-se; e pensava na significação desta saudação. Então, disse-lhe o Anjo: «Não temas, Maria, porquanto alcançaste graça diante do Senhor: eis que conceberás no teu seio, e darás à luz um Filho e o seu nome será Jesus. Ele será grande e será chamado Filho do Altíssimo; o Senhor Deus lhe dará o trono de David, seu pai; reinará eternamente na casa de Jacob; e o seu reino não terá fim». Porém, Maria disse ao Anjo: «Como acontecerá isso, se não conheço varão?». O Anjo, respondendo, disse-lhe: «O Espírito Santo descerá sobre ti e a virtude do Altíssimo te tocará com sua sombra. Por isso o Santo, que nascer de ti, será chamado Filho de Deus. E eis que Isabel, tua parenta, concebeu um filho na sua velhice: este é o sexto mês daquela que é chamada estéril: porque nada é impossível a Deus». Então disse Maria: «Eis aqui a escrava do Senhor, faça-se em mim segundo a tua palavra».
}\end{paracol}

\paragraphinfo{Ofertório}{Is. 35, 4}
\begin{paracol}{2}\latim{
\rlettrine{C}{onfortámini,} et jam nolite timére: ecce enim, Deus noster retríbuet judícium: ipse véniet, et salvos nos fáciet.
}\switchcolumn\portugues{
\rlettrine{C}{onfortai-vos,} e já não receeis: eis o nosso Deus, que vem a julgar cada um de nós: Ele próprio virá e nos salvará.
}\end{paracol}

\paragraph{Secreta}
\begin{paracol}{2}\latim{
\rlettrine{A}{ccépta} tibi sint, quǽsumus, Dómine, nostra jejúnia: quæ et expiándo nos tua grátia dignos effíciant, et ad sempiter
na promíssa perdúcant. Per Dóminum \emph{\&c.}
}\switchcolumn\portugues{
\rlettrine{S}{ejam-nos} propícios os nossos jejuns, Senhor, Vos suplicamos, para que, expiando por eles as nossas faltas, nos tornemos dignos da vossa graça, e sejamos conduzidos à felicidade eterna, que nos prometestes. Por nosso Senhor \emph{\&c.}
}\end{paracol}

\paragraphinfo{Comúnio}{Is. 7, 14}
\begin{paracol}{2}\latim{
\rlettrine{E}{cce,} Virgo concípiet et páriet fílium: et vocábitur nomen ejus Emmánuel.
}\switchcolumn\portugues{
\rlettrine{E}{is} que a Virgem conceberá e dará à luz um Filho, cujo nome será Emanuel.
}\end{paracol}

\paragraph{Postcomúnio}
\begin{paracol}{2}\latim{
\rlettrine{S}{alutáris} tui, Dómine, munere satiáti, súpplices deprecámur: ut, cujus lætámur gustu, renovémur efféctu. Per Dóminum \emph{\&c.}
}\switchcolumn\portugues{
\rlettrine{S}{aciados} com vossos salutares dons, Senhor, Vos suplicamos que, assim como nos alegramos com sua recepção, assim sejamos restabelecidos com seus efeitos. Por nosso Senhor \emph{\&c.}
}\end{paracol}

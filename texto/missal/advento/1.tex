\subsectioninfo{Primeiro Domingo do Advento}{Missa Ad Te Levavi}

\paragraphinfo{Intróito}{Sl. 24, 1–3}
\begin{paracol}{2}\latim{
\rlettrine{A}{d} te levávi ánimam meam: Deus meus, in te confíde, non erubéscam: neque irrídeant me inimíci mei: étenim univérsi, qui te exspéctant, non confundéntur. \emph{Ps. ibid., 4.} Vias tuas, Dómine, demónstra mihi: et sémitas tuas édoce me.
℣. Glória Patri \emph{\&c.}
}\switchcolumn
\portugues{
\rlettrine{A}{} Vós elevo a minha alma. Ó meu Deus, confio em Vós: não permitireis que fique confundido; nem que meus inimigos zombem de mim: porquanto aqueles que em Vós esperam não serão confundidos. \emph{Sl. ibid., 4.} Mostrai-me, Senhor, os vossos caminhos: e ensinai-me a conhecer as vossas veredas.
℣. Glória ao Pai \emph{\&c.}
}\end{paracol}

\paragraph{Oração}
\begin{paracol}{2}\latim{
\rlettrine{E}{xcita,} quǽsumus, Dómine, poténtiam tuam, et veni: ut ab imminéntibus peccatórum nostrórum perículis, te mereámur protegénte éripi, te liberánte salvári: Qui vivis et regnas cum Deo Patre in unitáte Spíritus Sancti Deus: per ómnia sǽcula sæculórum. ℟. Amen.
}\switchcolumn\portugues{
\rlettrine{S}{enhor} Vos suplicamos, manifestai o vosso poder, e «vinde já ao mundo», a fim de que com vossa protecção mereçamos ser preservados dos iminentes perigos em que incorremos por causa dos nossos pecados, e sejamos livres e salvos: Vós, que viveis e reinais com Deus Pai em unidade do Espírito Santo, Deus, em todos os séculos dos séculos. ℟. Amen.
}\end{paracol}

\paragraphinfo{Epístola}{Rm. 13, 11–14}
\begin{paracol}{2}\latim{
Lectio Epístolæ beati Pauli Apostoli ad Romános.
}\switchcolumn\portugues{
Lição da Ep.ª do B. Ap.º Paulo aos Romanos.
}\switchcolumn*\latim{
\rlettrine{F}{atres:} Scientes, quia hora est jam nos de somno súrgere. Nunc enim própior est nostra salus, quam cum credídimus. Nox præcéssit, dies autem appropinquávit. Abjiciámus ergo ópera tenebrárum, et induámur arma lucis. Sicut in die honéste ambulémus: non in comessatiónibus et ebrietátibus, non in cubílibus et impudicítiis, non in contentióne et æmulatióne: sed induímini Dóminum Jesum Christum.
}\switchcolumn\portugues{
\rlettrine{M}{eus} irmãos: Sabeis que soou a hora em que devemos despertar do sono. Agora, a nossa salvação está mais próxima do que quando recebemos o dom da fé. A noite passou e o dia vem chegando. Despojemo-nos, pois, das obras das trevas, e revistamo-nos das armas da luz. Andemos honestamente, como de dia: nem em excessos de comidas e bebidas nem em luxúrias e impurezas, nem em contendas e invejas; mas revistamo-nos de Nosso Senhor Jesus Cristo.
}\end{paracol}

\paragraphinfo{Gradual}{Sl. 24, 3–4}
\begin{paracol}{2}\latim{
\rlettrine{U}{nivérsi,} qui te exspéctant, non confundéntur, Dómine.
Vias tuas, Dómine, notas fac mihi: et sémitas tuas édoce me.
}\switchcolumn\portugues{
\rlettrine{S}{enhor,} aqueles que em Vós não esperam serão confundidos.
Mostrai-me, Senhor, os vossos caminhos: e ensinai-me a conhecer as vossas veredas.
}\switchcolumn*\latim{
Allelúja, allelúja. ℣. \emph{Ps. 84, 8} Osténde nobis, Dómine, misericórdiam tuam: et salutáre tuum da nobis. Allelúja.
}\switchcolumn\portugues{
Aleluia, aleluia. ℣. \emph{Sl. 84, 8} Mostrai-nos, Senhor, a vossa misericórdia: e dai-nos a salvação. Aleluia.
}\end{paracol}

\paragraphinfo{Evangelho}{Lc. 21, 25–33}
\begin{paracol}{2}\latim{
\cruz Sequéntia sancti Evangélii secúndum Lucam.
}\switchcolumn\portugues{
\cruz Continuação do santo Evangelho segundo S. Lucas.
}\switchcolumn*\latim{
\blettrine{I}{n} illo témpore: Dixit Jesus discípulis suis: Erunt signa in sole et luna et stellis, et in terris pressúra géntium præ confusióne sónitus maris et flúctuum: arescéntibus homínibus præ timóre et exspectatióne, quæ supervénient univérso orbi: nam virtútes coelórum movebúntur. Et tunc vidébunt Fílium hóminis veniéntem in nube cum potestáte magna et maiestáte. His autem fíeri incipiéntibus, respícite et leváte cápita vestra: quóniam appropínquat redémptio vestra. Et dixit illis similitúdinem: Vidéte ficúlneam et omnes árbores: cum prodúcunt iam ex se fructum, scitis, quóniam prope est æstas. Ita et vos, cum vidéritis hæc fíeri, scitóte, quóniam prope est regnum Dei. Amen, dico vobis, quia non præteríbit generátio hæc, donec ómnia fiant. Coelum et terra transíbunt: verba autem mea non transíbunt.
}\switchcolumn\portugues{
\blettrine{N}{aquele} tempo, disse Jesus a seus discípulos: «Haverá sinais no sol, na lua e nas estrelas; e haverá angústia nos povos da terra por causa do bramido do mar e das ondas, mirrando-se os homens de susto, na expectativa daquelas coisas que irão acontecer em todo o mundo; pois os poderes do céu estremecerão. Então aparecerá o Filho do homem, que virá em uma nuvem, revestido de grande poder e majestade. Quando estas coisas começarem a acontecer, olhai para o alto e erguei as vossas cabeças, porque se avizinha a vossa redenção». Depois deu-lhes esta comparação: «Vede a figueira e as demais árvores. Quando elas começam a frutificar, por aí conheceis que está próximo o estio. Do mesmo modo, quando estas coisas acontecerem, sabereis que está próximo o reino de Deus. Em verdade vos digo: não acabará esta geração sem que isto aconteça. O céu e a terra passarão; mas as minhas palavras permanecerão para sempre».
}\end{paracol}

\paragraphinfo{Ofertório}{Sl. 24, 1–3}
\begin{paracol}{2}\latim{
\rlettrine{A}{d} te levávi ánimam meam: Deus meus, in te confído, non erubéscam: neque irrídeant me inimíci mei: étenim univérsi, qui te exspéctant, non confundéntur.
}\switchcolumn\portugues{
\rlettrine{A}{} Vós elevo a minha alma. Ó meu Deus, confio em Vós: não permitireis que fique confundido; nem que meus inimigos zombem de mim: porquanto aqueles que em Vós esperam não serão confundidos.
}\end{paracol}

\paragraph{Secreta}
\begin{paracol}{2}\latim{
\rlettrine{H}{æc} sacra nos, Dómine, poténti virtúte mundátos ad suum fáciant purióres veníre princípium. Per Dominum nostrum \emph{\&c.}
}\switchcolumn\portugues{
\rlettrine{P}{ermiti,} Senhor, que estes mistérios, depois de nos haverem purificado com vossa poderosa virtude, nos façam chegar mais puros ainda junto daquele que é o seu princípio. Por nosso Senhor \emph{\&c.}
}\end{paracol}

\paragraphinfo{Comúnio}{Sl. 84, 13}
\begin{paracol}{2}\latim{
\rlettrine{D}{óminus} dabit benignitátem: et terra nostra dabit fructum suum.
}\switchcolumn\portugues{
\rlettrine{O}{} Senhor mostrará a sua bondade: e a nossa terra dará «o seu fruto».
}\end{paracol}

\paragraph{Postcomúnio}
\begin{paracol}{2}\latim{
\rlettrine{S}{uscipiámus,} Dómine, misericórdiam tuam in médio templi tui: ut reparatiónis nostræ ventúra sollémnia cóngruis honóribus præcedámus. Per Dominum nostrum \emph{\&c.}
}\switchcolumn\portugues{
\rlettrine{S}{enhor,} permiti que alcancemos a vossa misericórdia no meio do vosso templo, para que celebremos com a conveniente preparação a próxima solenidade da nossa reparação. Por nosso Senhor \emph{\&c.}
}\end{paracol}

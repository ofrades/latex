\subsectioninfo{Sábado das Têmporas do Inverno}{Estação em São Pedro}

\paragraphinfo{Intróito}{Sl. 79, 4 \& 2}
\begin{paracol}{2}\latim{
\rlettrine{V}{eni,} et osténde nobis fáciem tuam, Dómine, qui sedes super Chérubim: et salvi érimus. \emph{Ps. ib., 2} Qui regis Israël, inténde: qui dedúcis, velut ovem, Joseph.
℣. Gloria Patri \emph{\&c.}
}\switchcolumn\portugues{
\rlettrine{V}{inde,} mostrai-nos a vossa face, ó Senhor, que Vos sentais acima dos Querubins; e seremos salvos. \emph{Sl. ib., 2} Ó Vós, que governais Israel e conduzis José, como um pastor conduz uma ovelha, dignai-Vos ouvir-nos.
℣. Glória ao Pai \emph{\&c.}
}\end{paracol}

\begin{paracol}{2}\latim{
\begin{nscenter} Orémus. \end{nscenter}
}\switchcolumn\portugues{
\begin{nscenter} Oremos. \end{nscenter}
}\switchcolumn*\latim{
℣. Flectámus génua.
}\switchcolumn\portugues{
℣. Ajoelhemos!
}\switchcolumn*\latim{
℟. Leváte.
}\switchcolumn\portugues{
℟. Levantai-vos!
}\end{paracol}

\paragraph{Oração}
\begin{paracol}{2}\latim{
\rlettrine{D}{eus,} qui cónspicis, quia ex nostra pravitáte afflígimur: concéde propítius; ut ex tua visitatióne consolémur: Qui vivis \emph{\&c.}
}\switchcolumn\portugues{
\slettrine{Ó}{} Deus, vede que estamos aflitos com o peso da nossa fraqueza; concedei-nos propício que sejamos consolados com vossa visita: Ó Vós, que, sendo Deus, viveis e \emph{\&c.}
}\end{paracol}

\paragraphinfo{1.ª Lição}{Is. 19, 20–22}
\begin{paracol}{2}\latim{
Lectio Isaíæ Prophétæ.
}\switchcolumn\portugues{
Lição do Profeta Isaías.
}\switchcolumn*\latim{
\rlettrine{I}{n} diebus illis: Clamábunt ad Dóminum a facie tribulántis, et mittet eis salvatórem et propugnatórem, qui líberet eos. Et cognoscétur Dóminus ab Ægýpto, et cognóscent Ægýptii Dóminum in die illa: et colent eum in hóstiis et in munéribus: et vota vovébunt Dómino, et solvent. Et percútiet Dóminus Ægýplum plaga, et sanábit eam: et revertántur ad Dóminum, et placábitur eis, et sanábit eos
Dóminus, Deus noster.
}\switchcolumn\portugues{
\rlettrine{N}{aqueles} dias clamarão ao Senhor diante dos opressores. E Ele lhes enviará um Salvador que pugnará por eles e os livrará. Então o Senhor será conhecido no Egipto, e os egípcios naquele dia conhecerão o Senhor e honrá-lo-ão com sacrifícios e ofertas; farão votos ao Senhor e cumpri-los-ão. O Senhor ferirá o Egipto com uma chaga e fechá-la-á; e converter-se-ão ao Senhor, que se deixará aplacar. O Senhor, nosso Deus, os curará.
}\end{paracol}

\paragraphinfo{Gradual}{Sl. 18, 7 \& 2}
\begin{paracol}{2}\latim{
\rlettrine{A}{} summo cœlo egréssio ejus: et occúrsus ejus usque ad summum ejus. ℣. Cœli enárrant glóriam Dei: et opera mánuum ejus annúntiat firmaméntum.
}\switchcolumn\portugues{
\rlettrine{S}{urge} em um extremo do céu e segue o seu curso até ao outro extremo. ℣. Os céus proclamam a glória de Deus e o firmamento anuncia as obras das suas mãos.
}\end{paracol}

\begin{paracol}{2}\latim{
\begin{nscenter} Orémus. \end{nscenter}
}\switchcolumn\portugues{
\begin{nscenter} Oremos. \end{nscenter}
}\switchcolumn*\latim{
℣. Flectámus génua.
}\switchcolumn\portugues{
℣. Ajoelhemos!
}\switchcolumn*\latim{
℟. Leváte.
}\switchcolumn\portugues{
℟. Levantai-vos!
}\end{paracol}

\paragraph{Oração}
\begin{paracol}{2}\latim{
\rlettrine{C}{oncéde,} quǽsumus, omnípotens Deus: ut, qui sub peccáti jugo et vetústa servitúte deprímimur; exspectáta unigéniti Fílii tui nova nativitáte liberémur: Qui tecum vivit \emph{\&c.}
}\switchcolumn\portugues{
\slettrine{Ó}{} Deus omnipotente, Vos suplicamos, estando nós oprimidos, desde há tanto tempo, com o jugo do pecado, como consequência da antiga escravidão, fazei que sejamos livres desse mal pelo novo nascimento, tão desejado, do vosso Filho Unigénito: Ele, que, sendo Deus, convosco vive e reina \emph{\&c.}
}\end{paracol}

\paragraphinfo{2.ª Lição}{Is. 35, 1–7}
\begin{paracol}{2}\latim{
Lectio Isaíæ Prophétæ.
}\switchcolumn\portugues{
Lição do Profeta Isaías.
}\switchcolumn*\latim{
\rlettrine{H}{æc} dicit Dóminus: Lætábitur desérta et ínvia, ei exsultábit solitúdo, et florébit quasi lílium. Gérminans germinábit, et exsultábit lætabúnda et laudans: glória Líbani data est ei: decor Carméli et Saron, ipsi vidébunt glóriam Dómini, et decórem Dei nostri. Confortáte manus dissolútas, et génua debília roboráte. Dícite pusillánimis: Confortámini, et nolíte timére: ecce, Deus vester ultiónem addúcet retributiónis: Deus ipse véniet, et salvábit vos. Tunc aperiéntur óculi cæcórum, et aures surdórum patébunt. Tunc sáliet sicut cervus claudus, et apérta erit lingua mutórum: quia scissæ sunt in desérto aquæ, et torréntes in solitúdine. Et quæ erat árida, erit in stagnum, et sítiens in fontes aquárum: ait Dóminus omnípotens.
}\switchcolumn\portugues{
\rlettrine{E}{stas} cousas diz o Senhor: Alegrar-se-á a terra deserta e sem caminhos; e a solidão alegrar-se-á e florescerá, como o lírio. Germinará e florescerá abundantemente e exultará de alegria, e de júbilo, pois foi-lhe dada a glória do Líbano e a beleza do Carmelo e de Saran; verão a glória do Senhor e a majestade do nosso Deus. Fortificai as mãos fracas e retesai os joelhos trémulos. Dizei aos pusilânimes: «Tende coragem e nada receeis; eis aí vem o nosso Deus, que traz consigo uma vingança justa». O próprio Deus virá e vos salvará. Então, os olhos dos cegos se abrirão e os ouvidos dos surdos ouvirão; o coxo saltará, como um veado, e a língua do mudo será solta; também as águas rebentarão nos desertos e as torrentes nos lugares áridos; a terra, que estava seca, será como um lago, e a que tinha sede tornar-se-á em fontes de águas: diz o Senhor omnipotente.
}\end{paracol}

\paragraphinfo{Gradual}{Sl. 18, 6 \& 7}
\begin{paracol}{2}\latim{
\rlettrine{I}{n} sole pósuit tabernáculum suum: et ipse tamquam sponsus procédens de thálamo suo. ℣. A summo cœlo egréssio ejus: et occúrsus ejus usque ad summum ejus.
}\switchcolumn\portugues{
\rlettrine{P}{ôs} o seu tabernáculo no solo: e este é como um esposo quando sai do seu tálamo. ℣. Surge em um extremo do céu e segue o seu curso até ao outro extremo.
}\end{paracol}

\begin{paracol}{2}\latim{
\begin{nscenter} Orémus. \end{nscenter}
}\switchcolumn\portugues{
\begin{nscenter} Oremos. \end{nscenter}
}\switchcolumn*\latim{
℣. Flectámus génua.
}\switchcolumn\portugues{
℣. Ajoelhemos!
}\switchcolumn*\latim{
℟. Leváte.
}\switchcolumn\portugues{
℟. Levantai-vos!
}\end{paracol}

\paragraph{Oração}
\begin{paracol}{2}\latim{
\rlettrine{I}{ndignos} nos, quǽsumus, Dómine, fámulos tuos, quos actiónis própriæ culpa contrístat, unigéniti Fílii tui advéntu lætífica: Qui tecum vivit et regnat \emph{\&c.}
}\switchcolumn\portugues{
\rlettrine{A}{os} vossos indignos servos, Senhor, a quem contrista a culpa das suas acções, Vos suplicamos, alegrai-os com o advento do vosso Filho Unigénito. Ele, que, sendo Deus \emph{\&c.}
}\end{paracol}

\paragraphinfo{3.ª Lição}{Is. 40, 9–11}
\begin{paracol}{2}\latim{
Lectio Isaíæ Prophétæ.
}\switchcolumn\portugues{
Lição do Profeta Isaías.
}\switchcolumn*\latim{
\rlettrine{H}{æc} dicit Dóminus: Super montem excélsum ascénde tu, qui evangelízas Sion: exálta in fortitúdine vocem tuam, qui evangelízas Jerúsalem: exálta, noli timére. Dic civitátibus Juda: Ecce, Deus vester: ecce, Dóminus Deus in fortitúdine véniet, et bráchium ejus dominábitur: ecce, merces ejus cum eo, et opus illíus coram illo. Sicut pastor gregem suum pascet: in bráchio suo congregábit agnos, et in sinu suo levábit, Dóminus, Deus noster.
}\switchcolumn\portugues{
\rlettrine{E}{is} o que diz o Senhor: Ó tu, que evangelizas Sião, sobe a um monte elevado. Ó tu, que evangelizas Jerusalém, eleva a tua voz sonoramente. Eleva a tua voz; não tenhas receio. Diz às cidades de Judá: Eis o nosso Deus, eis que o Senhor Deus vem revestido com o poder e dominará com seu braço. Ele traz consigo a recompensa e o salário dos trabalhos; conduzirá o seu rebanho, como um pastor; acolherá os cordeiros em seus braços e apertá-los-á ao seu seio: Ele, o Senhor, nosso Deus.
}\end{paracol}

\paragraphinfo{Gradual}{Sl. 79, 20 \& 3}
\begin{paracol}{2}\latim{
\rlettrine{D}{ómine,} Deus virtútum, convérte nos: et osténde fáciem tuam, et salvi érimus, ℣. Excita, Dómine, poténtiam tuam, et veni, ut salvos fácias nos.
}\switchcolumn\portugues{
\rlettrine{C}{onvertei-nos,} ó Senhor, Deus dos exércitos! Mostrai-nos a vossa face e seremos salvos. ℣. Mostrai, Senhor, o vosso poder e vinde, para que sejamos salvos.
}\end{paracol}

\begin{paracol}{2}\latim{
\begin{nscenter} Orémus. \end{nscenter}
}\switchcolumn\portugues{
\begin{nscenter} Oremos. \end{nscenter}
}\switchcolumn*\latim{
℣. Flectámus génua.
}\switchcolumn\portugues{
℣. Ajoelhemos!
}\switchcolumn*\latim{
℟. Leváte.
}\switchcolumn\portugues{
℟. Levantai-vos!
}\end{paracol}

\paragraph{Oração}
\begin{paracol}{2}\latim{
\rlettrine{P}{ræsta,} quǽsumus, omnípotens Deus: ut Fílii tui ventúra sollémnitas et præséntis nobis vitæ remédia cónferat, et prǽmia ætérna concédat. Per eúndem Dóminum nostrum \emph{\&c.}
}\switchcolumn\portugues{
\slettrine{Ó}{} Deus omnipotente, Vos suplicamos, concedei-nos que a próxima Solenidade do nascimento do vosso Filho nos confira os remédios para a vida presente e nos proporcione os prémios eternos. Pelo mesmo nosso Senhor \emph{\&c.}
}\end{paracol}

\paragraphinfo{4.ª Lição}{Is. 45, 1–8}
\begin{paracol}{2}\latim{
Lectio Isaíæ Prophétæ.
}\switchcolumn\portugues{
Lição do Profeta Isaías.
}\switchcolumn*\latim{
\rlettrine{H}{æc} dicit Dóminus christo meo Cyro, cujus apprehéndi déxteram, ut subjíciam ante fáciem ejus gentes, et dorsa regum vertam, et apériam coram eo jánuas, et portæ non claudéntur. Ego ante te ibo: et gloriósos terræ humiliábo: portas ǽreas cónteram, et vectes férreos confríngam. Et dabo tibi thesáuros abscónditos et arcána secretórum: ut scias, quia ego Dóminus, qui voco nomen tuum, Deus Israël. Propter servum meum Jacob, et Israël electum meum, et vocávi te nómine tuo: assimilávi te, et non cognovísti me. Ego Dóminus, et non est ámplius: extra me non est Deus: accínxi te, et non cognovísti me: ut sciant hi, qui ab ortu solis, et qui ab occidénte, quóniam absque me non est. Ego Dóminus, et non est alter, formans lucem et creans ténebras, fáciens pacem et creans malum: ego Dóminus faciens omnia hæc. Roráte, cœli, désuper, et nubes pluant justum: aperiátur terra, et gérminet Salvatórem: et justítia oriátur simul: ego Dóminus creávi eum.
}\switchcolumn\portugues{
\rlettrine{E}{stas} cousas diz o Senhor ao seu ungido Ciro, a quem conduziu pela mão direita para lhe tornar sujeitas as nações, para pôr em fuga diante dele os reis e abrir perante ele as portas, que nunca mais lhe serão fechadas: «Eu irei adiante de ti; aplanarei o que é elevado; quebrarei as portas de bronze; e despedaçarei os ferrolhos de ferro. Dar-te-ei os tesouros escondidos e as riquezas dos lugares secretos, para que saibas que sou o Senhor, o Deus de Israel, que te chamei pelo teu nome. Por amor de Jacob, meu servo, e de Israel, meu eleito, chamei-te pelo teu próprio nome e assinalei-te, e tu me não conheceste. Eu sou o Senhor, teu Deus, e outro não existe fora de mim. Eu armei-te, e não me conheceste. Saiba-se, portanto, que desde o nascente do sol até ao poente outro Deus não existe senão eu. Eu sou o Senhor, e nenhum outro existe. Eu criei a luz e formei as trevas. Sou eu que firmo a paz e desencadeio os males. Sou eu, o Senhor, criador de todas as cousas! Ó céus, derramai dessas alturas o vosso orvalho: e que as nuvens chovam o Justo! Abra-se a terra, germine o Salvador e floresça ao mesmo tempo a justiça! Eu sou o Senhor, que criou tudo quanto existe».
}\end{paracol}

\paragraphinfo{Gradual}{Sl. 79, 3, 2 \& 3}
\begin{paracol}{2}\latim{
\rlettrine{E}{xcita,} Dómine, poténtiam tuam, et veni, ut salvos fácias nos. ℣. Qui regis Israël, inténde: qui dedúcis, velut ovem, Joseph: qui sedes super Chérubim, appáre coram Ephraim, Bénjamin, et Manásse.
}\switchcolumn\portugues{
\rlettrine{M}{ostrai} o vosso poder e vinde salvar-nos. ℣. Ouvi, ó Vós, que governais Israel; que conduzis José, como um pastor conduz o rebanho; e tendes um trono acima dos Querubins, manifestai-Vos ante Efraim, Benjamim e Manassés.
}\end{paracol}

\begin{paracol}{2}\latim{
\begin{nscenter} Orémus. \end{nscenter}
}\switchcolumn\portugues{
\begin{nscenter} Oremos. \end{nscenter}
}\switchcolumn*\latim{
℣. Flectámus génua.
}\switchcolumn\portugues{
℣. Ajoelhemos!
}\switchcolumn*\latim{
℟. Leváte.
}\switchcolumn\portugues{
℟. Levantai-vos!
}\end{paracol}

\paragraph{Oração}
\begin{paracol}{2}\latim{
\rlettrine{P}{reces} pópuli tui, quǽsumus, Dómine, cleménter exáudi: ut, qui juste pro peccátis nostris afflígimur, pietátis tuæ visitatióne consolémur: Qui vivis \emph{\&c.}
}\switchcolumn\portugues{
\rlettrine{D}{ignai-Vos} ouvir com clemência as preces do vosso povo, a fim de que nós, que estamos aflitos com os nossos pecados, sejamos consolados com vossa misericordiosa visita. Ó Vós, que viveis \emph{\&c.}
}\end{paracol}

\paragraphinfo{5.ª Lição}{Dn. 3, 47–51}
\begin{paracol}{2}\latim{
Lectio Daniélis Prophétæ.
}\switchcolumn\portugues{
Lição do Profeta Daniel.
}\switchcolumn*\latim{
\rlettrine{I}{n} diebus illis: Angelus Dómini descéndit cum Azaría et sóciis ejus in fornácem: et excússit flammam ignis de fornáce, et fecit médium fornácis quasi ventum roris flantem. Flamma autem effundebátur super fornácem cúbitis quadragínta novem: et erúpit, et incéndit, quos répperit juxta fornácem de Chaldǽis, minístros regis, qui eam incendébant. Et non tétigit eos omníno ignis, neque contristavit, nec quidquam moléstia íntulit. Tunc hi tres quasi ex uno ore laudábant, et glorificábant, et benedicébant Deum in fornáce, dicéntes:
}\switchcolumn\portugues{
\rlettrine{N}{aqueles} dias, o Anjo do Senhor desceu à fornalha com Azarias e os seus companheiros e afastou da fornalha as chamas do fogo, soprando no meio delas como que um vento de orvalho. As chamas do fogo, porém, cresceram acima da fornalha quarenta e nove côvados; e, saindo fora dela, queimaram os Caldeus, ministros do rei, que estavam Perto da fornalha a atiçar o fogo, e não queimaram nenhum dos três jovens, nem os feriram, nem lhes Causaram qualquer incómodo! Então, estes três jovens louvavam, glorificavam e bendiziam Deus na fornalha, em voz uníssona, dizendo:
}\end{paracol}

\begin{paracol}{2}\latim{
\begin{nscenter} Orémus. \end{nscenter}
}\switchcolumn\portugues{
\begin{nscenter} Oremos. \end{nscenter}
}\switchcolumn*\latim{
\rlettrine{D}{eus,} qui tribus púeris mitigásti flammas ignium: concéde propítius; ut nos fámulos tuos non exúrat flamma vitiórum. Per Dóminum nostrum \emph{\&c.}
}\switchcolumn\portugues{
\slettrine{Ó}{} Deus, que mitigastes as chamas do fogo aos três jovens, concedei-nos misericordiosamente que nós, vossos servos, não sejamos queimados pelas chamas dos vícios. Por nosso Senhor \emph{\&c.}
}\end{paracol}

\paragraphinfo{Hino Benedictus Es}{Dn. 3:52}\label{benedictuses}
\gregorioscore{scores/canticos/benedictuses}

\begin{nscenter}
Bendito sois, Senhor, Deus de nossos pais: e digno de louvor e de glória em todos os séculos.
\end{nscenter}

\begin{paracol}{2}\latim{
\rlettrine{E}{t} benedíctum nomen glóriæ tuæ, quod est sanctum. Et laudábile, et gloriósum in sæcula.
}\switchcolumn\portugues{
\rlettrine{B}{endito,} santo e glorioso é o vosso nome: e digno de louvor e de glória em todos os séculos.
}\switchcolumn*\latim{
Benedíctus es in templo sancto glóriæ tuæ. Et laudábílis, et gloriósus in sæcula.
}\switchcolumn\portugues{
Bendito sois no vosso Templo santo e glorioso: e digno de louvor e de glória em todos os séculos.
}\switchcolumn*\latim{
Benedíctus es super thronum sanctum regni tui. Et laudábílis, et gloriósus in sæcula.
}\switchcolumn\portugues{
Bendito sois Vós, que estais acima do sagrado trono do vosso Reino: e digno de louvor e de glória em todos os séculos.
}\switchcolumn*\latim{
Benedíctus es super sceptrum divinitátis tuæ. Et laudábilis, et gloriósus in sæcula.
}\switchcolumn\portugues{
Bendito sois acima do ceptro da vossa divindade: e digno de louvor e de glória em todos os séculos.
}\switchcolumn*\latim{
Benedíctus es, qui sedes super Chérubim, íntuens abýssos. Et laudábilis, et gloriósus in sæcula.
}\switchcolumn\portugues{
Bendito sois Vós, que Vos sentais acima dos Querubins e vedes a profundidade dos abysmos: e digno de louvor e de glória em todos os séculos.
}\switchcolumn*\latim{
Benedictus es, qui ambulas super pennas ventórurn, et super undas maris. Et laudábilis, et gloriósus in sæcula.
}\switchcolumn\portugues{
Bendito sois Vós, que voais sobre as asas dos ventos e caminhais sobre as ondas do mar: e digno de louvor e de glória em todos os séculos.
}\switchcolumn*\latim{
Benedícant te omnes Angeli, et Sancti tui. Et laudent te, et gloríficent in sæcula.
}\switchcolumn\portugues{
Que os Anjos e os Santos Vos bendigam, louvem e glorifiquem em todos os séculos dos séculos.
}\switchcolumn*\latim{
Benedicant te cæli, terra, mare, et ómnia quæ in eis sunt. Et laudent te, et gloríficent in sæcula.
}\switchcolumn\portugues{
Que os céus, a terra e o mar e tudo quanto encerram Vos bendigam, louvem e glorifiquem por todos os séculos dos séculos.
}\switchcolumn*\latim{
Glória Patri, et Fílio, et Spirítui Sancto. Et laudábili, et glorióso in sæcula.
}\switchcolumn\portugues{
Glória ao Pai, e ao Filho, e ao Espírito Santo: a Deus que é digno de louvor e de glória em todos os séculos.
}\switchcolumn*\latim{
Sicut erat in princípio, et nunc, et semper: et in sæcula sæculórum.Amen. Et laudábili, et glorióso in sæcula.
}\switchcolumn\portugues{
Assim como era no princípio, e agora, e sempre, e por todos os séculos dos séculos. A Deus, que é digno de louvor e de glória em todos os séculos.
}\switchcolumn*\latim{
Benedíctus es, Dómine Deus patrum nostrórum. Et laudábílis, et gloriósus in saecula.
}\switchcolumn\portugues{
Bendito sois, Senhor, Deus de nossos pais: e digno de louvor e de glória em todos os séculos.
}\end{paracol}


\begin{paracol}{2}\latim{
\rlettrine{B}{enedíctus} es, Dómine, Deus patrum nostrórum. Et laudábilis et gloriósus in sǽcula.
}\switchcolumn\portugues{
\rlettrine{B}{endito} sois, Senhor, Deus de nossos pais: e digno de louvor e de glória em todos os séculos.
}\switchcolumn*\latim{
Et benedíctum nomen glóriæ tuæ, quod est sanctum. Et laudábile et gloriósum in sǽcula.
}\switchcolumn\portugues{
Bendito, santo e glorioso é o vosso nome: e digno de louvor e de glória em todos os séculos.
}\switchcolumn*\latim{
Benedíctus es in templo sancto glóriæ tuæ. Et laudábilis et gloriósus in sǽcula.
}\switchcolumn\portugues{
Bendito sois no vosso Templo santo e glorioso: e digno de louvor e de glória em todos os séculos.
}\switchcolumn*\latim{
Benedíctus es super thronum sanctum regni tui. Et laudábilis et gloriósus in sǽcula.
}\switchcolumn\portugues{
Bendito sois Vós, que estais acima do sagrado trono do vosso Reino: e digno de louvor e de glória em todos os séculos.
}\switchcolumn*\latim{
Benedíctus es super sceptrum divinitátis tuæ. Et laudábilis et gloriósus in sǽcula.
}\switchcolumn\portugues{
Bendito sois acima do ceptro da vossa divindade: e digno de louvor e de glória em todos os séculos.
}\switchcolumn*\latim{
Benedíctus es, qui sedes super Chérubim, íntuens abýssos. Et laudábilis et gloriósus in sǽcula.
}\switchcolumn\portugues{
Bendito sois Vós, que Vos sentais acima dos Querubins e vedes a profundidade dos abysmos: e digno de louvor e de glória em todos os séculos.
}\switchcolumn*\latim{
Benedíctus es, qui ámbulas super pennas ventórum et super undas maris. Et laudábilis et gloriósus in sǽcula.
}\switchcolumn\portugues{
Bendito sois Vós, que voais sobre as asas dos ventos e caminhais sobre as ondas do mar: e digno de louvor e de glória em todos os séculos.
}\switchcolumn*\latim{
Benedícant te omnes Angeli et Sancti tui. Et laudent te et gloríficent in sǽcula.
}\switchcolumn\portugues{
Que os Anjos e os Santos Vos bendigam, louvem e glorifiquem em todos os séculos dos séculos.
}\switchcolumn*\latim{
Benedícant te cœli, terra, mare, et ómnia quæ in eis sunt. Et laudent te et gloríficent in sǽcula.
}\switchcolumn\portugues{
Que os céus, a terra e o mar e tudo quanto encerram Vos bendigam, louvem e glorifiquem por todos os séculos dos séculos.
}\switchcolumn*\latim{
Glória Patri, et Fílio, et Spirítui Sancto. Et laudábili et glorióso in sǽcula.
}\switchcolumn\portugues{
Glória ao Pai, e ao Filho, e ao Espiríto Santo: a Deus que é digno de louvor e de glória em todos os séculos.
}\switchcolumn*\latim{
Sicut erat in princípio, et nunc, et semper: et in sǽcula sæculórum. Amen. Et laudábili et glorióso in sǽcula.
}\switchcolumn\portugues{
Assim como era no princípio, e agora, e sempre, e por todos os séculos dos séculos. A Deus, que é digno de louvor e de glória em todos os séculos.
}\switchcolumn*\latim{
Benedíctus es, Dómine, Deus patrum nostrórum. Et laudábilis et gloriósus in sǽcula.
}\switchcolumn\portugues{
Bendito sois, Senhor, Deus de nossos pais: e digno de louvor e de glória em todos os séculos.
}\end{paracol}

\paragraphinfo{Epístola}{2 Ts. 2, 1–8}
\begin{paracol}{2}\latim{
Lectio Epístolæ beati Pauli Apostoli ad Corinthios.
}\switchcolumn\portugues{
Lição da Ep.ª do B. Ap.º Paulo aos Tessalonicenses.
}\switchcolumn*\latim{
\rlettrine{F}{ratres:} Rogámus vos per advéntum Dómini nostri Jesu Christi, et nostræ congregatiónis in ipsum: ut non cito moveámini a vestro sensu, neque terreámini, neque per spíritum, neque per sermónem, neque per epístolam tamquam per nos missam, quasi instet dies Dómini. Ne quis vos sedúcat ullo modo: quóniam nisi
vénerit discéssio primum, et revelátus fuerit homo peccáti, fílius perditiónis, qui adversátur, et extóllitur supra omne, quod dícitur Deus aut quod cólitur, ita ut in templo Dei sédeat osténdens se, tamquam sit Deus. Non retinétis, quod, cum adhuc essem apud vos, hæc dicébam vobis? Et nunc quid detíneat, scitis, ut revelétur in suo témpore. Nam mystérium jam operátur iniquitátis: tantum ut, qui tenet nunc, téneat, donec de médio fiat. Et tunc revelábitur ille iníquus, quem Dóminus Jesus interfíciet spíritu oris sui, et déstruet illustratióne advéntus sui.
}\switchcolumn\portugues{
\rlettrine{M}{eus} irmãos: Vos rogamos, pelo advento de N. S. Jesus Cristo e pela nossa união com Ele, que não mudeis facilmente o vosso bom modo de sentir, ou vos amedronteis, nem pelo espírito, nem por discurso, nem por epístola, como se fosse por nós mandada, que faça supor que o dia do Senhor esteja próximo. Ninguém vos seduza de algum modo que seja; porque este dia não virá enquanto não venha primeiro a apostasia e se manifeste o homem do pecado, o filho da perdição, que deverá morrer miseravelmente, o qual, opondo-se a Deus, como inimigo que é d’Ele, se levantará contra tudo o que se chama Deus ou tem carácter religioso, até se sentar no templo de Deus, fazendo-se passar por Deus. Não vos lembrais de que, estando ainda convosco, vos dizia estas cousas? E, agora, bem sabeis o que o detém, para que somente em seu tempo se manifeste. Porquanto, o mystério da iniquidade já se está operando, esperando somente, para aparecer, que aquilo que o deteve até agora, desapareça. Então se manifestará aquele ímpio que o Senhor Jesus exterminará com o sopro da sua boca e ofuscará com o esplendor do seu advento.
}\end{paracol}

\paragraphinfo{Trato}{Sl. 79, 2–3}
\begin{paracol}{2}\latim{
\qlettrine{Q}{ui} regis Israël, inténde: qui dedúcis, velut ovem, Joseph. ℣. Qui sedes super Chérubim, appáre coram Ephraim, Bénjamin, et Manásse. ℣. Excita, Dómine, poténtiam tuam, et veni: ut salvos fácias nos.
}\switchcolumn\portugues{
\rlettrine{O}{uvi,} ó Vós, que governais Israel, ó Vós, que tendes um trono acima dos Querubins: manifestai-Vos ante Efraim, Benjamim e Manassés. Mostrai, Senhor, o vosso poder: e vinde, para que sejamos salvos.
}\end{paracol}

\paragraphinfo{Evangelho}{Lc. 3, 1–6}\label{evangelhosabadotemporasinverno}

\begin{paracol}{2}\latim{
\cruz Sequéntia sancti Evangélii secúndum Lucam.
}\switchcolumn\portugues{
\cruz Continuação do santo Evangelho segundo S. Lucas.
}\switchcolumn*\latim{
\blettrine{A}{nno} quintodécimo impérii Tibérii Cǽsaris, procuránte Póntio Piláto Judǽam, tetrárcha autem Galilǽæ Heróde, Philíppo autem fratre ejus tetrárcha Iturǽæ et Trachonítidis regionis, et Lysánia Abilínæ tetrárcha, sub princípibus sacerdotum Anna et Cáipha: factum est verbum Domini super Joannem, Zacharíæ filium, in deserto. Et venit in omnem regiónem Jordánis, prǽdicans baptísmum pæniténtiæ in remissiónem peccatórum, sicut scriptum est in libro sermónum Isaíæ Prophétæ: Vox clamántis in desérto: Paráte viam Dómini: rectas fácite sémitas ejus: omnis vallis implébitur: et omnis moris et collis humiliábitur: et erunt prava in dirécta, et áspera in vias planas: et vidébit omnis caro salutáre Dei.
}\switchcolumn\portugues{
\blettrine{N}{o} ano décimo quinto do império de Tibério César sendo Pôncio Pilatos governador da Judeia, Herodes tetrarca da Galileia, Filipe, seu irmão, tetrarca da Itureia e da região de Traconites, e Lisânias tetrarca da Abilínia, e sendo pontífices Anás e Caifás: fez-se ouvir a palavra do Senhor, que foi dirigida no deserto a João, filho de Zacarias, o qual percorreu toda a região vizinha do Jordão, pregando o baptismo de penitência para a remissão dos pecados, segundo o que está escrito no livro das profecias do Profeta Isaías: «A voz do que clama no deserto: Preparai o caminho do Senhor; endireitai as suas veredas: toda a planície será elevada e todas as montanhas e colinas serão arrasadas: os caminhos tortuosos serão endireitados e os ásperos aplanados; e toda a carne humana verá a salvação de Deus».
}\end{paracol}

\paragraphinfo{Ofertório}{Zc. 9, 9}
\begin{paracol}{2}\latim{
\rlettrine{E}{xsúlta} satis, fília Sion, prǽdica, fília Jerúsalem: ecce, Rex tuus venit tibi sanctus et Salvátor.
}\switchcolumn\portugues{
\rlettrine{E}{xulta} de alegria, filha de Sião; rejubila, filha de Jerusalém: eis que vem a ti o teu Rei o Santo e Salvador.
}\end{paracol}

\paragraph{Secreta}
\begin{paracol}{2}\latim{
\rlettrine{S}{acrifíciis} præséntibus, quǽsumus, Dómine, placátus inténde: ut et devotióni nostræ profíciant et salúti. Per Dóminum \emph{\&c.}
}\switchcolumn\portugues{
\rlettrine{S}{enhor,} dignai-Vos aceitar benignamente estas ofertas, a fim de que sirvam para aumentar a nossa devoção e alcançar-nos a salvação. Por nosso Senhor. \emph{\&c.}
}\end{paracol}

\paragraphinfo{Comúnio}{Sl. 18, 6–7}
\begin{paracol}{2}\latim{
\rlettrine{E}{xsultávit} ut gigas ad curréndam viam: a summo cœlo egréssio ejus, et occúrsus ejus usque ad summum ejus.
}\switchcolumn\portugues{
\rlettrine{L}{ançou-se} em seu caminho, como um gigante, partindo de uma extremidade do céu e terminando a sua carreira na outra.
}\end{paracol}

\paragraph{Postcomúnio}
\begin{paracol}{2}\latim{
\qlettrine{Q}{uǽsumus,} Dómine, Deus noster: ut sacrosáncta mystéria, quæ pro reparatiónis nostræ munímine contulísti; et præsens nobis remédium esse fácias et futúrum. Per Dóminum \emph{\&c.}
}\switchcolumn\portugues{
\rlettrine{S}{enhor,} nosso Deus, Vos rogamos que estes sacrossantos mystérios, que nos concedestes para nossa reparação, sejam o nosso remédio para o presente e para o futuro. Por nosso Senhor \emph{\&c.}
}\end{paracol}

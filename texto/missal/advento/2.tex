\subsectioninfo{Segundo Domingo do Advento} {Missa Populus Sion}

\paragraphinfo{Intróito}{Is. 30, 30.}

\begin{paracol}{2}\latim{
\rlettrine{P}{ópulus} Sion, ecce, Dóminus véniet ad salvándas gentes: et audítam fáciet Dóminus glóriam vocis suæ in lætítia cordis vestri.
\emph{Ps. 79, 2} Qui regis Israël, inténde: qui dedúcis, velut ovem, Joseph.
Glória Patri \emph{\&c.}
}\switchcolumn\portugues{
\rlettrine{P}{ovo} de Sião, eis que o Senhor vem para salvar os povos: o Senhor fará ouvir a sua gloriosa voz, e o vosso coração encher-se-á de alegria.
\emph{Sl. 79, 2} Ouvi, ó vós, que governais Israel; ó vós, que conduzis José, como um pastor conduz uma ovelha.
Glória ao Pai \emph{\&c.}
}\end{paracol}

\paragraph{Oração}

\begin{paracol}{2}\latim{
\rlettrine{E}{xcita,} Dómine, corda nostra ad præparándas Unigéniti tui vias: ut, per ejus advéntum, purificátis tibi méntibus servíre mereámur: Qui tecum \emph{\&c.}
}\switchcolumn\portugues{
\rlettrine{E}{xcitai,} Senhor, os nossos corações para que preparem os caminhos do vosso Filho Unigénito, de modo que no seu advento mereçamos servi-l’O com as almas purificadas. Ele, que, sendo Deus \emph{\&c.}
}\end{paracol}

\paragraphinfo{Epístola}{Rm. 15, 4—13}

\begin{paracol}{2}\latim{
Lectio Epístolæ beati Pauli Apostoli ad Romános.
}\switchcolumn\portugues{
Lição da Ep.ª do B. Ap.º Paulo aos Romanos.
}\switchcolumn*\latim{
\rlettrine{F}{atres:} Quæcúmque scripta sunt, ad nostram doctrínam scripta sunt: ut per patiéntiam et consolatiónem Scripturárum spem habeámus. Deus autem patiéntiæ et solácii det vobis idípsum sápere in altérutrum secúndum Jesum Christum: ut unánimes, uno ore honorificétis Deum et Patrem Dómini nostri Jesu Christi. Propter quod suscípite ínvicem, sicut et Christus suscépit vos in honórem Dei. Dico enim Christum Jesum minístrum fuísse circumcisiónis propter veritátem Dei, ad confirmándas promissiónes patrum: gentes autem super misericórdia honoráre Deum, sicut scriptum est: Proptérea confitébor tibi in géntibus, Dómine, et nómini tuo cantábo. Et íterum dicit: Lætámini, gentes, cum plebe ejus. Et iterum: Laudáte, omnes gentes, Dóminum: et magnificáte eum, omnes pópuli. Et rursus Isaías ait: Erit radix Jesse, et qui exsúrget régere gentes, in eum gentes sperábunt. Deus autem spei répleat vos omni gáudio et pace in credéndo: ut abundétis in spe et virtúte Spíritus Sancti.
}\switchcolumn\portugues{
\rlettrine{M}{eus} irmãos: Tudo aquilo que está escrito foi escrito para nossa instrução, para que pela paciência e consolação possuamos a esperança que as Escrituras nos incutem. Conceda-vos Deus, que é cheio de paciência e de consolação, a graça de manifestardes uns para com os outros estes mesmos sentimentos, segundo Jesus Cristo, a fim de que unanimemente e com uma só voz honreis Deus, Pai de N. S. Jesus Cristo. Assim, portanto, recebei-vos uns aos outros, como Cristo vos recebeu para a glória de Deus, Pois digo-vos que Jesus Cristo fez-se ministro da circuncisão, no interesse da verdade de Deus, para que se confirmassem as promessas feitas aos nossos antepassados. Por outro lado, os gentios devem dar glória a Deus, pela misericórdia que manifestou, conforme o que está escrito: «Senhor, é por isso que Vos louvarei entre os gentios e cantarei hinos em honra do vosso nome». Está ainda escrito: «Alegrai-vos com seu povo, ó gentios». E mais: «Louvai todos o Senhor, ó povos da terra; que todos os povos O glorifiquem». Também Isaías escreveu: «Sairá a raiz de Jessé, e elevar-se-á para reinar sobre os gentios, e estes têm nela toda a esperança». Que Deus, que é cheio de esperança, vos cumule de alegria e de paz na vossa fé, a fim de que a esperança abunde em vós, pela virtude do Espírito Santo.
}\end{paracol}

\paragraphinfo{Gradual}{Sl. 49, 2—3 \& 5}

\begin{paracol}{2}\latim{
\rlettrine{E}{x} Sion species decóris ejus: Deus maniféste véniet.
Congregáta illi sanctos ejus, qui ordinavérunt testaméntum ejus super sacrifícia.
}\switchcolumn\portugues{
\rlettrine{D}{e} Sião raiará a majestade no seu esplendor; pois Deus virá visivelmente. Reuniu em torno d’Ele os seus santos, que firmaram com Ele uma aliança pelo sacrifício.
}\switchcolumn*\latim{
Allelúja, allelúja. ℣. \emph{Ps. 121, 1} Lætátus sum in his, quæ dicta sunt mihi: in domum Dómini íbimus. Allelúja.
}\switchcolumn\portugues{
Aleluia, aleluia. ℣. \emph{Sl. 121, 1} Rejubilo com aqueles que me disseram: «Iremos à casa do Senhor». Aleluia.
}\end{paracol}

\paragraphinfo{Evangelho}{Mt, 11, 2–10}

\begin{paracol}{2}\latim{
\cruz Sequéntia sancti Evangélii secúndum Matthæum.
}\switchcolumn\portugues{
\cruz Continuação do santo Evangelho segundo S. Mateus.
}\switchcolumn*\latim{
\blettrine{I}{n} illo tempore: Cum audísset Joánnes in vínculis ópera Christi, mittens duos de discípulis suis, ait illi: Tu es, qui ventúrus es, an alium exspectámus? Et respóndens Jesus, ait illis: Eúntes renuntiáte Joánni, quæ audístis et vidístis. Cæci vident, claudi ámbulant, leprósi mundántur, surdi áudiunt, mórtui resúrgunt, páuperes evangelizántur: et beátus est, qui non fúerit scandalizátus in me. Illis autem abeúntibus, cœpit Jesus dícere ad turbas de Joánne: Quid exístis in desértum vidére? arúndinem vento agitátam? Sed quid exístis videre? hóminem móllibus vestitum? Ecce, qui móllibus vestiúntur, in dómibus regum sunt. Sed quid exístis vidére? Prophétam? Etiam dico vobis, et plus quam Prophétam. Hic est enim, de quo scriptum est: Ecce, ego mitto Angelum meum ante fáciem tuam, qui præparábit viam tuam ante te.
}\switchcolumn\portugues{
\blettrine{N}{aquele} tempo, tendo João ouvido encarecer, na cadeia onde estava preso, as obras de Cristo, enviou dous dos seus discípulos a perguntar-Lhe: «Sois Vós O que há-de vir, ou devemos esperar outro?» Jesus respondeu: «Ide contar a João o que ouvis e vedes: os cegos vêem e os coxos andam; os leprosos são curados e os surdos ouvem; os mortos são ressuscitados e os pobres são evangelizados. Bem-aventurado aquele que se não escandalizar por causa de mim». Então, partiram os discípulos, e logo Jesus começou a falar à multidão a respeito de João, dizendo: «Quem é aquele que fostes ver ao deserto? Uma vara agitada pelo vento? Mas quem é que fostes ver? Um homem vestido com hábitos preciosos? Vede: aqueles que vestem com luxo vivem nos palácios dos reis. Quem é, então, aquele que fostes ver? Algum Profeta? Digo-vos, também, que é mais do que Profeta, pois foi a seu respeito que se escreveram estas palavras: «Eis que envio o meu Anjo perante a vossa face, para preparar o caminho diante de Vós».
}\end{paracol}

\paragraphinfo{Ofertório}{Sl. 84, 7–8}

\begin{paracol}{2}\latim{
\rlettrine{D}{eus,} tu convérsus vivificábis nos, et plebs tua lætábitur in te: osténde nobis, Dómine, misericórdiam tuam, et salutáre tuum da nobis.
}\switchcolumn\portugues{
\rlettrine{V}{olvei} a vossa face para nós, ó Deus, e receberemos a vida; então o vosso povo alegrar-se-á convosco. Mostrai-nos, Senhor, a vossa misericórdia, e dai-nos a vossa salvação.
}\end{paracol}

\paragraph{Secreta}

\begin{paracol}{2}\latim{
\rlettrine{P}{lacáre,} quǽsumus, Dómine, humilitátis nostræ précibus et hóstiis: et, ubi nulla suppétunt suffrágia meritórum, tuis nobis succúrre præsídiis. Per Dóminum \emph{\&c.}
}\switchcolumn\portugues{
\rlettrine{S}{enhor,} Vos suplicamos, deixai-Vos aplacar com as orações da nossa humildade, juntamente com estas oblatas; e, já que os nossos sufrágios são desprovidos de quaisquer méritos, assisti-nos ao menos com vosso auxílio. Por nosso Senhor \emph{\&c.}
}\end{paracol}

\paragraphinfo{Comúnio}{Br. 5, 5 \& 4, 36}

\begin{paracol}{2}\latim{
\qlettrine{J}{erúsalem,} surge et sta in excélso, ei vide iucunditátem, quæ véniet tibi a Deo tuo.
}\switchcolumn\portugues{
\rlettrine{S}{urge,}ó Jerusalém, e ergue-te no alto da montanha. Considera a alegria que te advirá do teu Deus.
}\end{paracol}

\paragraph{Postcomúnio}

\begin{paracol}{2}\latim{
\rlettrine{R}{epléti} cibo spirituális alimóniæ, súpplices te, Dómine, deprecámur: ut, hujus participatióne mystérii, dóceas nos terréna despícere et amáre cœléstia. Per Dóminum nostrum \emph{\&c.}
}\switchcolumn\portugues{
\rlettrine{S}{aciados} já com este alimento espiritual, concedei-nos, Senhor, Vos imploramos, que pela participação deste mystério aprendamos a desprezar os bens desta vida e a amar os do céu. Por nosso Senhor \emph{\&c.}
}\end{paracol}

\subsection{Domingo da Sexagésima}

\paragraphinfo{Intróito}{Sl. 43, 23-26}
\begin{paracol}{2}\latim{
\rlettrine{E}{xsúrge,} quare obdórmis, Dómine? exsúrge, et ne repéllas in finem: quare fáciem tuam avértis, oblivísceris tribulatiónem nostram? adhǽsit in terra venter noster: exsúrge, Dómine, ádjuva nos, et líbera nos. \emph{Ps. ibid., 2} Deus, áuribus nostris audívimus: patres nostri annuntiavérunt nobis.
℣. Gloria Patri \emph{\&c.}
}\switchcolumn\portugues{
\rlettrine{L}{evantai-Vos,} Senhor, porque dormis? Levantai-Vos, e nos não desampareis para sempre. Porque afastais de nós a vossa face e esqueceis a nossa tribulação? O nosso peito está oprimido pela terra. Levantai-Vos, Senhor, socorrei-nos, salvai-nos. \emph{Sl. ibid., 2} Ó Deus, nós ouvimos com os nossos ouvidos; os nossos antepassados contaram-nos as vossas maravilhas.
℣. Glória ao Pai \emph{\&c.}
}\end{paracol}

\paragraph{Oração}
\begin{paracol}{2}\latim{
\rlettrine{D}{eus,} qui cónspicis, quia ex nulla nostra actióne confídimus: concéde propítius; ut, contra advérsa ómnia, Doctóris géntium protectióne muniámur. Per Dóminum \emph{\&c.}
}\switchcolumn\portugues{
\slettrine{Ó}{} Deus, que conheceis não podermos confiar nas nossas obras, dignai-Vos proteger-nos com a assistência do «Doutor das gentes» contra todas as adversidades. Por nosso Senhor Jesus Cristo, vosso Filho, que \emph{\&c.}
}\end{paracol}

\paragraphinfo{Epístola}{2 Cor. 11, 19-33; 12, 1-9}
\begin{paracol}{2}\latim{
Léctio Epístolæ beáti Pauli Apóstoli ad Corínthios.
}\switchcolumn\portugues{
Lição da Ep.ª do B. Ap.º Paulo aos Coríntios.
}\switchcolumn*\latim{
\rlettrine{P}{atres:} Libénter suffértis insipiéntens: cum sitis ipsi sapiéntes. Sustinétis enim, si quis vos in servitútem rédigit, si quis dévorat, si quis áccipit, si quis extóllitur, si quis in fáciem vos cædit. Secúndum ignobilitátem dico, quasi nos infírmi fuérimus in hac parte. In quo quis audet, (in insipiéntia dico) áudeo et ego: Hebrǽi sunt, et ego: Israelítæ sunt, et ego: Semen Abrahæ sunt, et ego: Minístri Christi sunt, (ut minus sápiens dico) plus ego: in labóribus plúrimis, in carcéribus abundántius, in plagis supra modum, in mórtibus frequénter. A Judǽis quínquies quadragénas, una minus, accépi. Ter virgis cæsus sum, semel lapidátus sum, ter naufrágium feci, nocte et die in profúndo maris fui: in itinéribus sæpe, perículis flúminum, perículis latrónum, perículis ex génere, perículis ex géntibus, perículis in civitáte, perículis in solitúdine, perículis in mari, perículis in falsis frátribus: in labóre et ærúmna, in vigíliis multis, in fame et siti, in jejúniis multis, in frigóre et nuditáte: præter illa, quæ extrínsecus sunt, instántia mea cotidiána, sollicitúdo ómnium Ecclesiárum. Quis infirmátur, et ego non infírmor? quis scandalizátur, et ego non uror? Si gloriári opórtet: quæ infirmitátis meæ sunt, gloriábor. Deus et Pater Dómini nostri Jesu Christi, qui est benedíctus in sǽcula, scit quod non méntior. Damásci præpósitus gentis Arétæ regis, custodiébat civitátem Damascenórum, ut me comprehénderet: et per fenéstram in sporta dimíssus sum per murum, et sic effúgi manus ejus. Si gloriári opórtet (non éxpedit quidem), véniam autem ad visiónes et revelatiónes Dómini. Scio hóminem in Christo ante annos quatuórdecim, (sive in córpore néscio, sive extra corpus néscio, Deus scit:) raptum hujúsmodi usque ad tértium cœlum. Et scio hujúsmodi hóminem, (sive in córpore, sive extra corpus néscio, Deus scit: quóniam raptus est in paradisum: et audivit arcána verba, quæ non licet homini loqui. Pro hujúsmodi gloriábor: pro me autem nihil gloriábor nisi in infirmitátibus meis. Nam, et si volúero gloriári, non ero insípiens: veritátem enim dicam: parco autem, ne quis me exístimet supra id, quod videt in me, aut áliquid audit ex me. Et ne magnitúdo revelatiónem extóllat me, datus est mihi stímulus carnis meæ ángelus sátanæ, qui me colaphízet. Propter quod ter Dóminum rogávi, ut discéderet a me: et dixit mihi: Súfficit tibi grátia mea: nam virtus in infirmitáte perfícitur. Libénter ígitur glo
riábor in infirmitátibus meis, ut inhábitet in me virtus Christi.
}\switchcolumn\portugues{
\rlettrine{M}{eus} irmãos: Como homens sensatos que sois, generosamente suportais os insensatos. E suportais, também, se vos sujeitam à escravidão, se vos devoram, se vos roubam, se vos tratam com arrogância, ou se vos esbofeteiam. Digo-o com vergonha, como se neste ponto houvéssemos sido fracos! Contudo, quem quer que ouse vangloriar-se (falo como se fora insensato), também eu me vanglorio. Eles são hebreus? Também eu. São israelitas? Também eu. São descendentes de Abraão? Também eu. São ministros de Cristo? Muito mais (falo insensatamente) sou eu do que eles: pelos meus muitos trabalhos, mais do que os deles; pelas minhas frequentes prisões, mais do que as deles; pelas pancadas sem conta que sofri, mais do que as deles; e até, frequentemente, tendo quase visto a morte. Dos judeus recebi chicotadas, em cinco quarenta vezes menos uma; três vezes fui açoitado com varas; uma vez fui apedrejado; três vezes naufraguei: passei um dia e uma noite no fundo do mar! Em minhas contínuas viagens encontrei sempre perigos: perigos nas águas, perigos nos ladrões, perigos nos meus compatriotas, perigos nos pagãos, perigos nas cidades, perigos nos desertos, perigos no mar, perigos nos irmãos falsos, nos trabalhos, nas fadigas, nas numerosas vigílias, na fome, na sede, nos muitos jejuns, no frio e na nudez! E, além destes males, que são exteriores, preocupa-me também quotidianamente a solicitude de todas as cristandades. Quem adoece, e me não vê doente? Quem é fraco, sem que eu o seja também? Quem cai no escândalo, sem que o fogo do tormento me queime? Se convém que alguém se glorie, gloriar-me-ei eu, então, pelas cousas que fez a minha fraqueza! Deus Pai de Cristo e para sempre bendito sabe que não minto, Em Damasco, o governador da província, por ordem do rei Aretas, mandou guardar a cidade para me prender; mas desceram-me pela muralha para fora em um cesto; e escapei, assim, das suas mãos. É preciso ainda (decerto não é útil) que me glorie? Recordarei, então, as visões e revelações do Senhor. Conheço um homem que há catorze anos foi arrebatado ao terceiro céu (se foi só no corpo, o não sei, Deus o sabe). Sei que este homem (se foi só no corpo, o não sei, Deus o sabe) foi levado ao paraíso, onde ouviu palavras inefáveis, que não é permitido revelar. Por este motivo poderei gloriar-me; mas, pelo que me diz respeito, só posso gloriar-me das minhas enfermidades. Se eu quisesse gloriar-me, não seria insensato, pois dizia a verdade; mas abstenho-me disso para que ninguém faça de mim uma ideia superior ao que vê em mim, ou ao que de mim ouve dizer. Para que a grandeza destas revelações me não encha de orgulho, foi-me dado um castigo na minha carne: um anjo de Satanás foi encarregado de me esbofetear. E por causa dele roguei três vezes ao Senhor que o afastasse de mim; mas o Senhor disse-me: «Basta-te a minha graça, pois a virtude aperfeiçoa-se nas tribulações». Glorio-me, pois, voluntariamente, com as minhas fraquezas, para que a virtude de Cristo resida em mim.
}\end{paracol}

\paragraphinfo{Gradual}{Sl. 82, 19 \& 14}
\begin{paracol}{2}\latim{
\rlettrine{S}{ciant} gentes, quóniam nomen tibi Deus: tu solus Altíssimus super omnem terram. ℣. Deus meus, pone illos ut rotam, et sicut stípulam ante fáciem venti.
}\switchcolumn\portugues{
\rlettrine{S}{aibam} as nações que o vosso nome é Deus; que só Vós sois o Altíssimo em todo o mundo. ℣. Ó meu Deus, tornai os meus inimigos semelhantes à roda, que gira sem cessar, ou à palha, que o vento agita e arrebata.
}\end{paracol}

\paragraphinfo{Trato}{Sl. 59, 4 \& 6}
\begin{paracol}{2}\latim{
\rlettrine{C}{ommovísti,} Dómine, terram, et conturbásti eam. ℣. Sana contritiónes ejus, quia mota est. ℣. Ut fúgiant a fácie arcus: ut liberéntur elécti tui.
}\switchcolumn\portugues{
\rlettrine{S}{enhor,} abalastes e arruinastes a terra. Reparai as suas ruínas, porque ela está abalada. Que os vossos escolhidos possam fugir diante do arco armado contra eles, e que sejam livres.
}\end{paracol}

\paragraphinfo{Evangelho}{Lc. 8, 4-15}
\begin{paracol}{2}\latim{
\cruz Sequéntia sancti Evangélii secúndum Lucam.
}\switchcolumn\portugues{
\cruz Continuação do santo Evangelho segundo S. Lucas.
}\switchcolumn*\latim{
\blettrine{I}{n} illo témpore: Cum turba plúrima convenírent, et de civitátibus properárent ad Jesum, dixit per similitúdinem: Exiit, qui séminat, semináre semen suum: et dum séminat, áliud cécidit secus viam, et conculcátum est, et vólucres cœli comedérunt illud. Et áliud cécidit supra petram: et natum áruit, quia non habébat humórem. Et áliud cécidit inter spinas, et simul exórtæ spinæ suffocavérunt illud. Et áliud cécidit in terram bonam: et ortum fecit fructum céntuplum. Hæc dicens, clamábat: Qui habet aures audiéndi, audiat. Interrogábant autem eum discípuli ejus, quæ esset hæc parábola. Quibus ipse dixit: Vobis datum est nosse mystérium regni Dei, céteris autem in parábolis: ut vidéntes non videant, et audientes non intéllegant. Est autem hæc parábola: Semen est verbum Dei. Qui autem secus viam, hi sunt qui áudiunt: déinde venit diábolus, et tollit verbum de corde eórum, ne credéntes salvi fiant. Nam qui supra petram: qui cum audierint, cum gáudio suscipiunt verbum: et hi radíces non habent: qui ad tempus credunt, et in témpore tentatiónis recédunt. Quod autem in spinas cécidit: hi sunt, qui audiérunt, et a sollicitudínibus et divítiis et voluptátibus vitæ eúntes, suffocántur, et non réferunt fructum. Quod autem in bonam terram: hi sunt, qui in corde bono et óptimo audiéntes verbum rétinent, et fructum áfferunt in patiéntia.
}\switchcolumn\portugues{
\blettrine{N}{aquele} tempo, como concorresse grande multidão de diversas cidades e viessem ter com Jesus, disse-lhes Ele esta parábola: «Saiu um semeador a semear a sua semente; e, enquanto ele semeava, caiu uma porção de semente junto ao caminho, a qual foi pisada, e, depois, comida pelos pássaros. Outra porção caiu entre as pedras e, embora tivesse nascido, secou logo, porque não tinha humidade. Ainda outra porção caiu entre espinhos, os quais cresceram com ela, afogando-a depois. Finalmente, outra porção caiu em terra boa, e, nascendo, deu fruto a cem por um». Depois de Jesus ter falado assim, acrescentou em voz alta: «Quem tem ouvidos para ouvir, ouça!». Então os seus discípulos perguntaram-Lhe a significação desta parábola. «Avós respondeu Ele, é dado conhecer o reino de Deus, porém, os outros só o conhecerão pelas parábolas; de sorte que, vendo, não vejam, e ouvindo, não compreendam. Eis o que significa esta parábola: A semente é a palavra de Deus. A semente que caiu junto ao caminho significa os que ouvem a palavra, mas logo vem o demónio, a arrebata do seu coração, com medo de que acreditem e sejam A semente que caiu em cima das pedras significa os que ouvem a palavra, e a recebem com alegria, mas não possuem a raiz: acreditam durante algum tempo, mas, quando vem a tentação, sucumbem. A semente que caiu nos espinhos designa aqueles que ouvem a palavra, mas deixam-na pouco a pouco abafar, ou pelos cuidados e inquietações do mundo, ou pelas riquezas e prazeres desta vida, e, portanto, não produzem fruto algum. Enfim, a semente que caiu na terra boa representa aqueles que ouvem a palavra com o coração recto e bom e a guardam, produzindo depois fruto pela paciência».
}\end{paracol}

\paragraphinfo{Ofertório}{Sl. 16, 5, 6-7}
\begin{paracol}{2}\latim{
\rlettrine{P}{érfice} gressus meos in sémitis tuis, ut non moveántur vestígia mea: inclína aurem tuam, et exáudi verba mea: mirífica misericórdias tuas, qui salvos facis sperántes in te, Dómine.
}\switchcolumn\portugues{
\rlettrine{F}{irmai} os meus passos nas vossas veredas, para que meus pés não tropecem: Inclinai para mim os vossos ouvidos e escutai as minhas palavras: manifestai as vossas admiráveis misericórdias, Senhor, pois salvais os que em Vós confiam!
}\end{paracol}

\paragraph{Secreta}
\begin{paracol}{2}\latim{
\rlettrine{O}{blátum} tibi, Dómine, sacrifícium, vivíficet nos semper et múniat. Per Dóminum nostrum \emph{\&c.}
}\switchcolumn\portugues{
\rlettrine{S}{enhor,} que o sacrifício que Vos oferecemos nos vivifique sempre. Por nosso Senhor Jesus Cristo, vosso Filho, que \emph{\&c.}
}\end{paracol}

\paragraphinfo{Comúnio}{Sl. 42, 4}
\begin{paracol}{2}\latim{
\rlettrine{I}{ntroíbo} ad altáre Dei, ad Deum, qui læ
tíficat juventútem meam.
}\switchcolumn\portugues{
\rlettrine{S}{ubirei} ao altar de Deus: de Deus, que é a alegria da minha juventude.
}\end{paracol}

\paragraph{Postcomúnio}
\begin{paracol}{2}\latim{
\rlettrine{S}{úpplices} te rogámus, omnípotens Deus: ut, quos tuis réficis sacraméntis, tibi étiam plácitis móribus dignánter deservíre concédas. Per Dóminum \emph{\&c.}
}\switchcolumn\portugues{
\rlettrine{D}{eus} omnipotente, humildemente Vos rogamos, concedei àqueles que alimentais com vossos sacramentos a graça de Vos servirem com uma conduta que Vos seja agradável. Por nosso Senhor \emph{\&c.}
}\end{paracol}

\subsection{Domingo da Septuagésima}

\paragraphinfo{Intróito}{Sl. 17, 5, 6 \% 7}
\begin{paracol}{2}\latim{
\rlettrine{C}{ircumdedérut} me gémitus mortis, dolóres inférni circumdedérunt me: et in tribulatióne mea invocávi Dóminum, et exaudívit de templo sancto suo vocem meam. \emph{Ps. ibd., 2-3} Díligam te, Dómine, fortitúdo mea: Dóminus firmaméntum meum, et refúgium meum, et liberátor meus.
℣. Gloria Patri \emph{\&c.}
}\switchcolumn\portugues{
\rlettrine{R}{odearam-me} os gemidos da morte e cercaram-me as dores do inferno; mas na minha tribulação invoquei o Senhor, que lá no seu santo templo ouviu a minha voz. \emph{Sl. ibd., 2-3} Eu Vos amarei, Senhor, que sois a minha fortaleza, o meu sustentáculo, o meu refúgio e o meu libertador.
℣. Glória ao Pai \emph{\&c.}
}\end{paracol}

\paragraph{Oração}
\begin{paracol}{2}\latim{
\rlettrine{P}{reces} pópuli tui, quǽsumus, Dómine, cleménter exáudi: ut, qui juste pro peccátis nostris afflígimur, pro tui nóminis glória misericórditer liberémur. Per Dóminum nostrum \emph{\&c.}
}\switchcolumn\portugues{
\rlettrine{O}{uvi} benigno, Senhor, Vos rogamos, as preces do vosso povo, a fim de que nós, que estamos justamente aflitos com o peso dos nossos pecados, sejamos misericordiosamente livres, pela glória do vosso nome. Por nosso Senhor Jesus Cristo; vosso Filho, que \emph{\&c.}
}\end{paracol}

\paragraphinfo{Epístola}{1 Cor. 9, 24-27; 10, 1-5.}
\begin{paracol}{2}\latim{
Lectio Epístolæ beati Pauli Apostoli ad Corinthios.
}\switchcolumn\portugues{
Lição da Ep.ª do B. Ap.º Paulo aos Coríntios.
}\switchcolumn*\latim{
\rlettrine{F}{ratres}: Nescítis, quod ii, qui in stádio currunt, omnes quidem currunt, sed unus áccipit bravíum? Sic cúrrite, ut comprehendátis. Omnis autem, qui in agóne conténdit, ab ómnibus
se ábstinet: et illi quidem, ut corruptíbilem corónam accípiant; nos autem incorrúptam. Ego ígitur sic curro, non quasi in incértum: sic pugno, non quasi áërem vérberans: sed castígo corpus meum, et in servitútem rédigo: ne forte, cum áliis prædicáverim, ipse réprobus effíciar. Nolo enim vos ignoráre, fratres, quóniam patres nostri omnes sub nube fuérunt, et omnes mare transiérunt, et omnes in Móyse baptizáti sunt in nube et in mari: et omnes eándem escam spiritálem manducavérunt, et omnes eúndem potum spiritálem bibérunt (bibébant autem de spiritáli, consequénte eos, petra: petra autem erat Christus): sed non in plúribus eórum beneplácitum est Deo.
}\switchcolumn\portugues{
\rlettrine{M}{eus} irmãos: Não sabeis que aqueles que correm no circo, todos correm, mas somente um ganha o prémio? Correi, pois, de tal modo que o ganheis. Aqueles que combatem na arena abstêm-se de tudo; e procedem assim para alcançar uma coroa corruptível. Nós, porém, para alcançarmos uma coroa incorruptível. Eu também assim corro, mas não ao acaso; eu pelejo, mas não como quem fere o ar; pois castigo o meu corpo e o reduzo à servidão, com medo de que, depois de ter pregado aos outros, seja condenado. Ora, meus irmãos, não quero que ignoreis que nossos pais estiveram debaixo da nuvem, passaram o mar e todos foram baptizados e guiados por Moisés na nuvem e no mar; nem mesmo que ignoreis que todos eles, também, comeram o mesmo alimento espiritual e beberam a mesma bebida. Com efeito, beberam em um rochedo espiritual, que os seguia, o qual era Cristo. Todavia, a maior parte deles não agradou a Deus.
}\end{paracol}

\paragraphinfo{Gradual}{Sl. 9, 10-11 \& 19-20}
\begin{paracol}{2}\latim{
\rlettrine{A}{djútor} in opportunitátibus, in tribulatióne: sperent in te, qui novérunt te: quóniam non derelínquis quæréntes te, Dómine. ℣. Quóniam non in finem oblívio erit páuperis: patiéntia páuperum non períbit in ætérnum: exsúrge, Dómine, non præváleat homo.
}\switchcolumn\portugues{
\rlettrine{S}{enhor,} sois o nosso auxílio nas necessidades e nas tribulações: esperem, pois, em Vós aqueles que Vos conhecem, porque nunca abandonais os que Vos procuram. O infeliz não será sempre esquecido; a paciência do pobre não será frustrada para sempre. Erguei-Vos, para que o homem mau não triunfe.
}\end{paracol}

\paragraphinfo{Trato}{Sl. 129, 1-4}
\begin{paracol}{2}\latim{
\rlettrine{D}{e} profúndis clamávi ad te. Dómine: Dómine, exáudi vocem meam. ℣. Fiant aures tuæ intendéntes in oratiónem servi tui. ℣. Si iniquitátes observáveris, Dómine: Dómine, quis sustinébit? ℣. Quia apud te propitiátio est, et propter legem tuam sustínui te, Dómine.
}\switchcolumn\portugues{
\rlettrine{D}{as} profundezas do abysmo clamei por Vós, Senhor: ouvi a minha voz. ℣. Que os vossos ouvidos estejam atentos à voz deste vosso servo. ℣. Se julgais as nossas iniquidades, Senhor, quem poderá subsistir ante Vós? ℣. Mas sois propício; por amor da vossa lei esperei em Vós.
}\end{paracol}

\paragraphinfo{Evangelho}{Mt. 20, 1-16}
\begin{paracol}{2}\latim{
\cruz Sequéntia sancti Evangélii secúndum Matthǽum.
}\switchcolumn\portugues{
\cruz Continuação do santo Evangelho segundo S. Mateus.
}\switchcolumn*\latim{
\blettrine{I}{n} illo témpore: Dixit Jesus discípulis suis parábolam hanc: Simile est regnum cœlórum hómini patrifamílias, qui éxiit primo mane condúcere operários in víneam suam. Conventióne autem facta cum operáriis ex denário diúrno, misit eos in víneam suam. Et egréssus circa horam tértiam, vidit álios stantes in foro otiósos, et dixit illis: Ite et vos in víneam meam, et quod justum fúerit, dabo vobis. Illi autem abiérunt. Iterum autem éxiit circa sextam et nonam horam: et fecit simíliter. Circa undécimam vero éxiit, et invénit álios stantes, et dicit illis: Quid hic statis tota die otiósi? Dicunt ei: Quia nemo nos condúxit. Dicit illis: Ite et vos in víneam meam. Cum sero autem factum esset, dicit dóminus víneæ procuratóri suo: Voca operários, et redde illis mercédem, incípiens a novíssimis usque ad primos. Cum veníssent ergo qui circa undécimam horam vénerant, accepérunt síngulos denários. Veniéntes autem et primi, arbitráti sunt, quod plus essent acceptúri: accepérunt autem et ipsi síngulos denários. Et accipiéntes murmurábant advérsus patremfamílias, dicéntes: Hi novíssimi una hora fecérunt et pares illos nobis fecísti, qui portávimus pondus diéi et æstus. At ille respóndens uni eórum, dixit: Amíce, non facio tibi injúriam: nonne ex denário convenísti mecum? Tolle quod tuum est, et vade: volo autem et huic novíssimo dare sicut et tibi. Aut non licet mihi, quod volo, fácere? an óculus tuus nequam est, quia ego bonus sum? Sic erunt novíssimi primi, et primi novíssimi. Multi enim sunt vocáti, pauci vero elécti.
}\switchcolumn\portugues{
\blettrine{N}{aquele} tempo, disse Jesus a seus discípulos: «O reino dos céus é semelhante a um pai de família que sai da sua casa cedo para ajustar jornaleiros, para a sua vinha. E, tendo ajustado com eles dar a cada um por dia um dinheiro, mandou-os para a vinha. Cerca da hora terceira, saiu outra vez e viu que estavam ociosos na praça outros jornaleiros. Disse-lhes: «Ide, também, trabalhar para a minha vinha, e pagar-vos-ei o que for justo»; e eles foram. Saiu, ainda, cerca da hora sexta e cerca da hora nona, e fez a mesma cousa. Enfim, havendo saído cerca da hora undécima, encontrou outros e disse-lhes: «Porque estais todo o dia ociosos?». Eles responderam: «Porque ninguém nos ajustou». E disse-lhes: «Ide vós, também, para a minha vinha». Quando era já pela tarde, disse o senhor da vinha ao seu intendente: «Chama os jornaleiros e paga-lhes os salários, começando pelos últimos e acabando nos primeiros». Quando vieram os jornaleiros da hora undécima, receberam um dinheiro cada um. Vindo, por sua vez, os primeiros, cuidavam que receberiam mais; porém receberam, também, um dinheiro cada um. Então, estes, recebendo o dinheiro, murmuraram contra o pai de família, dizendo: «Estes últimos não trabalharam senão uma hora, e dais-lhes tanto como a nós, que aguentámos no dia inteiro o peso do trabalho e do calor?». Mas o senhor, dirigindo-se a um deles, respondeu: «Meu amigo, não te faço injustiça. Não ajustaste comigo receber um dinheiro? Toma, pois, o que te pertence, e vai-te. Quanto a mim, quero dar a este último tanto como a ti. Porventura me não é lícito ser generoso para com quem quiser? Diz-me: o teu olho é mau, porque o meu é bom? Assim, os últimos serão os primeiros, e os primeiros serão os últimos; pois muitos são os chamados e poucos os escolhidos».
}\end{paracol}

\paragraphinfo{Ofertório}{Sl. 91, 2}
\begin{paracol}{2}\latim{
\rlettrine{B}{onum} est confitéri Dómino, et psállere nómini tuo, Altíssime.
}\switchcolumn\portugues{
\slettrine{É}{} bom louvar o Senhor; e cantar Salmos em honra do vosso nome, ó Altíssimo.
}\end{paracol}

\paragraph{Secreta}
\begin{paracol}{2}\latim{
\rlettrine{M}{unéribus} nostris, quǽsumus, Dómine, precibúsque suscéptis: et cœléstibus nos munda mystériis, et cleménter exáudi. Per Dóminum \emph{\&c.}
}\switchcolumn\portugues{
\rlettrine{R}{ecebendo} as nossas ofertas e orações, Senhor, dignai-Vos purificar-nos por virtude dos vossos celestiais mystérios e ouvi misericordioso os nossos rogos. Por nosso Senhor Jesus Cristo, vosso Filho, que \emph{\&c.}
}\end{paracol}

\paragraphinfo{Comúnio}{Sl. 30, 17-18}
\begin{paracol}{2}\latim{
\rlettrine{I}{llúmina} fáciem tuam super servum tuum, et salvum me fac in tua misericórdia: Dómine, non confúndar, quóniam invocávi te.
}\switchcolumn\portugues{
\rlettrine{F}{azei} resplandecer a vossa face sobre este vosso servo, e salvai-me pela vossa misericórdia, Senhor; que eu não seja confundido, pois Vos invoquei.
}\end{paracol}

\paragraph{Postcomúnio}
\begin{paracol}{2}\latim{
\rlettrine{F}{idéles} tui, Deus, per tua dona firméntur: ut eadem et percipiéndo requírant, et quæréndo sine fine percípiant. Per Dóminum \emph{\&c.}
}\switchcolumn\portugues{
\qlettrine{Q}ue{} os vossos fiéis, ó Deus, sejam fortificados com vossos dons, a fim de que, recebendo-os, continuem a procurá-los, e, havendo-os achado, sirvam para a nossa eternidade Por nosso Senhor \emph{\&c.}
}\end{paracol}

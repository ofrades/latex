\subsectioninfo{Epifania do Senhor}{6 de Janeiro}\label{epifania}

\paragraphinfo{Intróito}{Ml. 3, 1; 1 Cr. 29, 12}

\begin{paracol}{2}\latim{
\rlettrine{E}{cce,} advénit dominátor Dóminus: et regnum in manu ejus et potéstas et impérium.
\emph{Ps. 71, 1} Deus, judícium tuum Regi da: et justítiam tuam Fílio Regis.
℣. Gloria Patri \emph{\&c.}
}\switchcolumn\portugues{
\rlettrine{E}{is} que apareceu soberano Senhor: e traz empunhado o ceptro da realeza, do poder e do império.
\emph{Sl. 71, 1} Ó Deus, concedei ao Rei a graça de saber julgar: e ao filho do Rei a vossa justiça.
℣. Glória ao Pai \emph{\&c.}
}\end{paracol}

\paragraph{Oração}

\begin{paracol}{2}\latim{
\rlettrine{D}{eus,} qui hodiérna die Unigénitum tuum géntibus steila duce revelásti: concéde propítius; ut, qui jam te ex fide cognóvimus, usque ad contemplándam spéciem tuæ celsitúdinis perducámur. Per eúndem Dóminum nostrum \emph{\&c.}
}\switchcolumn\portugues{
\slettrine{Ó}{} Deus, que neste dia manifestastes o vosso Filho Unigénito aos gentios, enviando-lhes uma estrela para os guiar até onde Ele estava, concedei-nos propício que, conhecendo-Vos nós, já, pela fé, consigamos contemplar face a face o esplendor da vossa majestade. Pelo mesmo nosso Senhor \emph{\&c.}
}\end{paracol}

\paragraphinfo{Epístola}{Is. 60, 1-6}

\begin{paracol}{2}\latim{
Léctio Isaíæ Prophétæ.
}\switchcolumn\portugues{
Lição do Profeta Isaías.
}\switchcolumn*\latim{
\rlettrine{S}{urge,} illumináre, Jerúsalem: quia venit lumen tuum, et glória Dómini super te orta est. Quia ecce, ténebræ opérient terram et caligo pópulos: super te autem oriétur Dóminus, et glória ejus in te vidébitur. Et ambulábunt gentes in lúmine tuo, et reges in splendóre ortus tui. Leva in circúitu óculos tuos,
et vide: omnes isti congregáti sunt, venérunt tibi: fílii tui de longe vénient, et fíliæ tuæ de látere surgent. Tunc vidébis et áfflues, mirábitur et dilatábitur cor tuum, quando convérsa fúerit ad te multitúdo maris, fortitúdo géntium vénerit tibi. Inundátio camelórum opériet te dromedárii Mádian et Epha: omnes de Saba vénient, aurum et thus deferéntes, et laudem Dómino annuntiántes.
}\switchcolumn\portugues{
\rlettrine{E}{rgue-te} e resplandece, Jerusalém, pois a tua luz raiou, e a glória do Senhor brilhou sobre ti. Enquanto as trevas cobrem a terra, e uma noite escura envolve os povos, eis que desponta sobre ti a aurora do Senhor; e a sua glória resplandece em ti. As nações caminharão, guiadas pelo clarão da tua luz, e os reis pelo fulgor da tua aurora. Ergue os teus olhos; volve-os em torno de ti e vê: todos se congregam e vêm a ti. Os teus filhos vêm de longe e as tuas filhas surgirão ao lado. Então, tu o verás e te encontrarás na opulência. O teu coração se alegrará e dilatará, pois, as riquezas do mar virão a ti; os tesouros das nações virão à tua posse! Numerosos camelos e dromedários de Mádian e de Efa virão a ti. Todos os de Sabá virão junto de ti, trazendo ouro e incenso e louvando o Senhor.
}\end{paracol}

\paragraphinfo{Gradual}{ibid., 6 \& 1}

\begin{paracol}{2}\latim{
\rlettrine{O}{mnes} de Saba vénient, aurum et thus deferéntes, et laudem Dómino annuntiántes. ℣. Surge et illumináre, Jerúsalem: quia glória Dómini super te orta est.
}\switchcolumn\portugues{
\rlettrine{T}{odos} os de Sabá virão, trazendo ouro em incenso e louvando o Senhor. Ergue-te, Jerusalém, e resplandece, porque a glória do Senhor brilhou sobre ti.
}\switchcolumn*\latim{
Allelúja, allelúja. ℣. \emph{Matth. 2, 2} Vídimus stellam ejus in Oriénte, et vénimus cum munéribus adoráre Dóminum. Allelúja.
}\switchcolumn\portugues{
Aleluia, aleluia. ℣. \emph{Mt. 2, 2} Vimos a sua estrela no Oriente e viemos com ofertas adorar o Senhor. Aleluia.
}\end{paracol}

\paragraphinfo{Evangelho}{Mt. 2, 1-12}

\begin{paracol}{2}\latim{
\cruz Sequéntia sancti Evangélii secúndum Matthǽu
}\switchcolumn\portugues{
\cruz Continuação do santo Evangelho segundo S. Mateus.
}\switchcolumn*\latim{
\blettrine{C}{um} natus esset Jesus in Béthlehem Juda in diébus Heródis regis, ecce, Magi ab Oriénte venerunt Jerosólymam, dicéntes: Ubi est, qui natus est rex Judæórum? Vidimus enim stellam ejus in Oriénte, et vénimus adoráre eum. Audiens autem Heródes rex, turbatus est, et omnis Jerosólyma cum illo. Et cóngregans omnes principes sacerdotum et scribas pópuli, sciscitabátur ab eis, ubi Christus nasceretur. At illi dixérunt ei: In Béthlehem Judae: sic enim scriptum est per Prophétam: Et tu, Béthlehem terra Juda, nequaquam mínima es in princípibus Juda; ex te enim éxiet dux, qui regat pópulum meum Israel. Tunc Heródes, clam vocátis Magis, diligénter dídicit ab eis tempus stellæ, quæ appáruit eis: et mittens illos in Béthlehem, dixit: Ite, et interrogáte diligénter de púero: et cum invenéritis, renuntiáte mihi, ut et ego véniens adórem eum. Qui cum audíssent regem, abiérunt. Et ecce, stella, quam víderant in Oriénte, antecedébat eos, usque dum véniens staret supra, ubi erat Puer. Vidéntes autem stellam, gavísi sunt gáudio magno valde. Et intrántes domum, invenérunt Púerum cum María Matre ejus, \emph{(hic genuflectitur)} ei procidéntes adoravérunt eum. Et, apértis thesáuris suis, obtulérunt ei múnera, aurum, thus et myrrham. Et responso accépto in somnis, ne redírent ad Heródem, per aliam viam revérsi sunt in regiónem suam.
}\switchcolumn\portugues{
\blettrine{H}{avendo} Jesus nascido em Belém, de Judá, no tempo do rei Herodes, eis que vieram a Jerusalém os Magos do Oriente, dizendo: «Onde está o rei dos Judeus, que acaba de nascer? Pois vimos a sua estrela no Oriente e viemos adorá-l’O». Logo que o rei Herodes ouviu esta notícia, ficou perturbado, assim como toda a gente de Jerusalém, convocando logo todos os príncipes dos sacerdotes e os escribas do povo, para saber deles onde deveria nascer Cristo. Responderam-lhe eles: «Em Belém, de Judá, pois está escrito pelo Profeta: «E tu, Belém, terra de Judá, não serás certamente a menos importante entre as terras principais de Judá, pois em ti nascerá o Rei, que governará o meu povo de Israel». Então Herodes mandou chamar em segredo os Magos, informando-se com eles diligentemente acerca do tempo em que a estrela havia aparecido. E, enviando-os a Belém, disse-lhes: «Ide, procurai diligentemente o Menino, e, logo que O houverdes achado, avisai-me, para que eu vá, também, adorá-l’O». Os Magos, tendo ouvido estas palavras, partiram. Ora, a estrela, que tinham visto no Oriente, ia adiante deles, até que, chegando ao lugar onde estava o Menino, parou. Quando os Magos viram a estrela, alegraram-se muito. Entrando, então, na casa, encontraram o Menino com Maria, sua mãe; e, de joelhos, O adoraram. \emph{(Todos devem ajoelhar)}. E, tendo aberto os seus tesouros, ofereceram-Lhe presentes de ouro, incenso e mirra. Depois, havendo tido em sonhos aviso de que não deveriam voltar a encontrar Herodes, retiraram-se por outro caminho para o seu país.
}\end{paracol}

\paragraphinfo{Ofertório}{Sl. 71, 10-11}
\begin{paracol}{2}\latim{
\rlettrine{R}{eges} Tharsis, et ínsulæ múnera ófferent: reges Arabum et Saba dona addúcent: et adorábunt eum omnes reges terræ, omnes gentes sérvient ei.
}\switchcolumn\portugues{
\rlettrine{O}{s} reis de Társis e as ilhas oferecer-Lhe-ão tributos: Os reis de Arábia e de Sabá levar-Lhe-ão ofertas; todos os reis da terra O adorarão; e todas as nações O servirão.
}\end{paracol}

\paragraph{Secreta}

\begin{paracol}{2}\latim{
\rlettrine{E}{cclésiæ} tuæ, quǽsumus, Dómine, dona propítius intuere: quibus non jam aurum, thus et myrrha profertur; sed quod eisdem munéribus declarátur, immolátur et súmitur, Jesus Christus, fílius tuus, Dóminus noster: Qui tecum vivit \emph{\&c.}
}\switchcolumn\portugues{
\rlettrine{D}{ignai-Vos} olhar benigno, Senhor, para os dons que a vossa Igreja Vos oferece, não o ouro, o incenso e a mirra, mas o que estas dádivas representam, e que agora é imolado e dado em alimento, Jesus Cristo, vosso Filho, nosso Senhor: Que, sendo Deus \emph{\&c.}
}\end{paracol}

\paragraphinfo{Comúnio}{Mt. 2, 2}

\begin{paracol}{2}\latim{
\rlettrine{V}{ídimus} stellam ejus in Oriénte, et vénimus cum munéribus adoráre Dóminum.
}\switchcolumn\portugues{
\rlettrine{V}{imos} a sua estrela no Oriente e viemos com ofertas adorar o Senhor.
}\end{paracol}

\paragraph{Postcomúnio}

\begin{paracol}{2}\latim{
\rlettrine{P}{ræsta,} quǽsumus, omnípotens Deus: ut, quæ sollémni celebrámus officio, purificátæ mentis intellegéntia consequámur. Per Dóminum nostrum \emph{\&c.}
}\switchcolumn\portugues{
\slettrine{Ó}{} Deus omnipotente, dignai-Vos purificar inteiramente o nosso espírito, a fim de que compreenda os mystérios que celebrámos neste solene ofício. Por nosso Senhor \emph{\&c.}
}\end{paracol}

\subsection{Domingo da Quinquagésima}

\paragraphinfo{Intróito}{Sl. 30, 3-4}
\begin{paracol}{2}\latim{
\rlettrine{E}{sto} mihi in Deum protectórem, et in locum refúgii, ut salvum me fácias: quóniam firmaméntum meum et refúgium meum es tu: et propter nomen tuum dux mihi eris, et enútries me. \emph{Ps. ibid., 2} In te, Dómine, sperávi, non confúndar in ætérnum: in justítia tua líbera me et éripe me.
℣. Gloria Patri \emph{\&c.}
}\switchcolumn\portugues{
\rlettrine{S}{ede,} ó Deus, o meu protector e o lugar de refúgio, onde encontre a salvação; pois sois o meu sustentáculo e o meu refúgio, e, pela glória do vosso nome, me conduzireis e sustentareis. \emph{Sl. ibid., 2} Em Vós, Senhor, pus a minha confiança; não serei confundido para sempre. Segundo a vossa justiça, livrai-me, salvai-me.
℣. Glória ao Pai \emph{\&c.}
}\end{paracol}

\paragraph{Oração}
\begin{paracol}{2}\latim{
\rlettrine{P}{reces} nostras, quǽsumus, Dómine, cleménter exáudi: atque, a peccatórum vínculis absolútos, ab omni nos adversitáte custódi. Per Dóminum \emph{\&c.}
}\switchcolumn\portugues{
\rlettrine{S}{enhor,} dignai-Vos ouvir clemente as nossas orações; e, depois de nos livrardes dos laços dos nossos pecados, defendei-nos de todas as adversidades. Por nosso Senhor \emph{\&c.}
}\end{paracol}

\paragraphinfo{Epístola}{1 Cor. 13, 1-13}
\begin{paracol}{2}\latim{
Léctio Epístolæ beáti Pauli Apóstoli ad Corinthios.
}\switchcolumn\portugues{
Lição da Ep.ª do B. Ap.º Paulo aos Coríntios.
}\switchcolumn*\latim{
\rlettrine{F}{ratres:} Si linguis hóminum loquar et Angelórum, caritátem autem non hábeam, factus sum velut æs sonans aut cýmbalum tínniens. Et si habúero prophétiam, et nóverim mystéria ómnia et omnem sciéntiam: et si habúero omnem fidem, ita ut montes tránsferam, caritátem autem non habúero, nihil sum. Et si distribúero in cibos páuperum omnes facultátes meas, et si tradídero corpus meum, ita ut árdeam, caritátem autem non habuero, nihil mihi prodest. Cáritas patiens est, benígna est: cáritas non æmulátur, non agit pérperam, non inflátur, non est ambitiósa, non quærit quæ sua sunt, non irritátur, non cógitat malum, non gaudet super iniquitáte, congáudet autem veritáti: ómnia suifert, ómnia credit, ómnia sperat, ómnia sústinet. Cáritas numquam éxcidit: sive prophétiæ evacuabúntur, sive linguæ cessábunt, sive sciéntia destruétur. Ex parte enim cognóscimus, et ex parte prophetámus. Cum autem vénerit quod perféctum est, evacuábitur quod ex parte est. Cum essem párvulus, loquébar ut párvulus, sapiébam ut párvulus, cogitábam ut párvulus. Quando autem factus sum vir, evacuávi quæ erant párvuli. Vidémus nunc per spéculum
in ænígmate: tunc autem fácie ad fáciem. Nunc cognósco ex parte: tunc autem cognóscam, sicut et cógnitus sum. Nunc autem manent fides, spes, cáritas, tria hæc: major autem horum est cáritas.
}\switchcolumn\portugues{
\rlettrine{M}{eus} irmãos: Se eu falar as línguas dos homens e dos Anjos, mas não tiver caridade, sou como o metal, que tine, ou como o sino, que soa. E se eu tiver o dom de profecia, conhecer todos os mystérios e possuir toda a ciência; e se tiver toda a fé, até ser capaz de transportar montanhas, mas não tiver caridade, nada sou. E se eu distribuir todos meus bens, para sustento dos pobres, e se entregar o meu corpo, para ser queimado, mas não tiver caridade, de nada me aproveitará. A caridade é paciente e benigna; não é invejosa, não é leviana, não é soberba, não é ambiciosa, não procura o próprio interesse, não se irrita, não julga mal, não se alegra com a injustiça; antes regozija-se com a verdade, sofre tudo, acredita em tudo, tudo espera, tudo suporta. A caridade nunca perecerá, ainda que não houvesse mais profecias, ainda que as línguas acabassem, ainda que a ciência desaparecesse; pois estes dons da ciência e da profecia são incompletos. Ora, quando vier o que é perfeito, acabará o que é imperfeito. Quando eu era menino, falava como menino, pensava como menino e discorria como menino; mas, quando me tornei homem, desapareceu o que tinha de menino. Agora vemos em enigmas, como em um espelho; mais tarde veremos face a face. Agora conheço as cousas imperfeitamente; mais tarde conhecê-las-ei, como sou conhecido. Agora permanecem estas três cousas: a fé, a esperança e a caridade; mas a maior das três é a caridade.
}\end{paracol}

\paragraphinfo{Gradual}{Sl. 76, 15 \& 16}
\begin{paracol}{2}\latim{
\rlettrine{T}{u} es Deus qui facis mirabília solus: notam fecísti in géntibus virtútem tuam. ℣. Liberásti in bráchio tuo pópulum tuum, fílios Israel et Joseph.
}\switchcolumn\portugues{
\slettrine{Ó}{} Deus, só Vós praticais maravilhas! Fizestes conhecer aos povos o vosso poder. Com o vosso braço forte livrastes o vosso povo os filhos de Israel e de José.
}\end{paracol}

\paragraphinfo{Trato}{Sl. 99, 1-2}
\begin{paracol}{2}\latim{
\qlettrine{J}{ubiláte} Deo, omnis terra: servíte Dómino in lætítia. ℣. Intráte in conspéctu ejus in exsultatióne: scitóte, quod Dóminus ipse est Deus. ℣. Ipse fecit nos, et non ipsi nos: nos autem pópulus ejus, et oves páscuæ ejus.
}\switchcolumn\portugues{
\slettrine{Ó}{} povos de toda a terra, louvai a Deus com júbilo. Vinde à sua presença com transportes de alegria; pois sabeis que o Senhor é verdadeiro Deus. ℣. Foi Ele quem nos criou, e não nós a nós mesmos. Somos, pois, o seu povo, e as ovelhas de que Ele é o pastor.
}\end{paracol}

\paragraphinfo{Evangelho}{Lc. 18, 31-43}
\begin{paracol}{2}\latim{
\cruz Sequéntia sancti Evangélii secúndum Lucam.
}\switchcolumn\portugues{
\cruz Continuação do santo Evangelho segundo S. Lucas.
}\switchcolumn*\latim{
\blettrine{I}{n} illo témpore: Assúmpsit Jesus duódecim, et ait illis: Ecce, ascéndimus Jerosólymam, et consummabúntur ómnia, quæ scripta sunt per Prophétas de Fílio hominis. Tradátur enim Géntibus, et illudétur, et flagellábitur, et conspuétur: et postquam flagelláverint, occídent eum, et tértia die resúrget. Et ipsi nihil horum intellexérunt, et erat verbum istud abscónditum ab eis, et non intellegébant quæ dicebántur. Factum est autem, cum appropinquáret Jéricho, cæcus quidam sedébat secus viam, mendícans. Et cum audíret turbam prætereúntem, interrogábat, quid hoc esset. Dixérunt autem ei, quod Jesus Nazarénus transíret. Et clamávit, dicens: Jesu, fili David, miserére mei. Et qui præíbant, increpábant eum, ut tacéret. Ipse vero multo magis clamábat: Fili David, miserére mei. Stans autem Jesus, jussit illum addúci ad se. Et cum appropinquásset, interrogávit illum, dicens: Quid tibi vis fáciam? At ille dixit: Dómine, ut vídeam. Et Jesus dixit illi: Réspice, fides tua te salvum fecit. Et conféstim vidit, et sequebátur illum, magníficans Deum. Et omnis plebs ut vidit, dedit laudem Deo.
}\switchcolumn\portugues{
\blettrine{N}{aquele} tempo, levou Jesus consigo os Doze e disse-lhes: «Eis que subimos para Jerusalém, onde se vai cumprir o que os Profetas escreveram a respeito do Filho do homem, pois será entregue aos gentios, será escarnecido, injuriado e cuspido; e, depois de O haverem flagelado, será morto; mas ressuscitará ao terceiro dia». Porém eles não compreenderam estas palavras (pois o sentido delas era-lhes oculto) e não entendiam a sua significação. Chegou, então, Jesus perto de Jericó, onde estava um cego à beira do caminho a pedir esmola. Ouvindo este o rumor das turbas, perguntou o que era aquilo. E disseram-lhe: «É Jesus de Nazaré que passa». Logo, ele começou a gritar: «Jesus, filho de David, tende piedade de mim!». Aqueles que iam adiante repreenderam-no rudemente, dizendo-lhe que se calasse. Mas ele gritava ainda com mais força: «Jesus, filho de David, tende piedade de mim!». Então Jesus, parando, mandou que Lhe levassem o cego. Quando este já estava ao pé, interrogou-o Jesus: «Que queres que faça?». Ele respondeu: «Senhor, fazei que eu veja!». Jesus disse-lhe: «Pois vê! A tua fé salvou-te!». Logo, começou a ver. E, acompanhando Jesus, glorificava Deus. E todo o povo que presenciou isto louvou a Deus.
}\end{paracol}

\paragraphinfo{Ofertório}{Sl. 118, 12-13}
\begin{paracol}{2}\latim{
\rlettrine{B}{enedíctus} es, Dómine, doce me justificatiónes tuas: in lábiis meis pronuntiávi ómnia judícia oris tui.
}\switchcolumn\portugues{
\rlettrine{B}{endito} sois, Senhor. Ensinai-me a conhecer a vossa lei. Pronunciei com meus lábios todas as sentenças da vossa boca.
}\end{paracol}

\paragraph{Secreta}
\begin{paracol}{2}\latim{
\rlettrine{H}{æc} hóstia, Dómine, quǽsumus, emúndet nostra delícta: et, ad sacrifícium celebrándum, subditórum tibi córpora mentésque sanctíficet. Per Dóminum \emph{\&c.}
}\switchcolumn\portugues{
\rlettrine{S}{enhor,} Vos suplicamos, permiti que esta hóstia apague os nossos pecados; e, para que dignamente se celebre este sacrifício, fazei que ela santifique os corpos e as almas dos vossos fiéis. Por nosso Senhor \emph{\&c.}
}\end{paracol}

\paragraphinfo{Comúnio}{Sl. 77, 29-30}
\begin{paracol}{2}\latim{
\rlettrine{M}{anducavérunt,} et saturári sunt nimis, et desidérium eórum áttulit eis Dóminus: non sunt fraudáti a desidério suo.
}\switchcolumn\portugues{
\rlettrine{C}{omeram} e saciaram-se abundantemente. O Senhor deu-lhes segundo os seus desejos: e não ficaram frustrados os seus desejos.
}\end{paracol}

\paragraph{Postcomúnio}
\begin{paracol}{2}\latim{
\qlettrine{Q}{uǽsumus,} omnípotens Deus: ut, qui cœléstia aliménta percépimus, per hæc contra ómnia adversa muniámur. Per Dóminum \emph{\&c.}
}\switchcolumn\portugues{
\rlettrine{V}{os} pedimos, ó Deus, que este alimento celestial que recebemos nos fortifique contra todas as adversidades. Por nosso Senhor \emph{\&c.}
}\end{paracol}

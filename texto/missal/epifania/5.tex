\subsection{Quinto Domingo depois da Epifania}

\paragraphinfo{Intróito}{Sl. 96, 7-8}
\begin{paracol}{2}\latim{
\rlettrine{A}{doráte} Deum, omnes Angeli ejus: audívit, et lætáta est Sion: et exsultavérunt fíliæ Judae. \emph{Ps. ibid., 1} Dóminus regnávit, exsúltet terra: læténtur ínsulæ multæ.
℣. Gloria Patri \emph{\&c.}
}\switchcolumn\portugues{
\rlettrine{A}{dorai} a Deus, ó vós, que sois os seus Anjos. Ouviu Sião: e rejubilou: e as filhas de Judá exultaram de alegria. \emph{Sl. ibid., 1} O Senhor reinou: alegre-se a terra e as suas muitas ilhas.
℣. Glória ao Pai \emph{\&c.}
}\end{paracol}

\paragraph{Oração}
\begin{paracol}{2}\latim{
\rlettrine{F}{amíliam} tuam, quǽsumus, Dómine, contínua pietáte custódi: ut, quæ in sola spe grátiæ cœléstis innítitur, tua semper protectióne muniátur. Per Dóminum \emph{\&c.}
}\switchcolumn\portugues{
\rlettrine{S}{enhor,} dignai-Vos revestir a vossa família com vossa perpétua misericórdia, e, como ela não possui outra esperança senão a da vossa celestial graça, permiti que esteja sempre munida com vossa protecção. Por nosso Senhor \emph{\&c.}
}\end{paracol}

\paragraphinfo{Epístola}{Cl. 3, 12-17}
\begin{paracol}{2}\latim{
Léctio Epístolæ beáti Pauli Apóstoli ad Colossénses.
}\switchcolumn\portugues{
Lição da Ep.ª do B. Ap.º Paulo aos Colossenses.
}\switchcolumn*\latim{
\rlettrine{F}{ratres:} Indúite vos sicut electi Dei, sancti et dilecti, víscera misericórdiæ, benignitátem, humilitátem, modéstiam, patiéntiam: supportántes ínvicem, et donántes vobismetípsis, si quis advérsus áliquem habet querélam: sicut et Dóminus donávit vobis, ita et vos. Super ómnia autem hæc caritátem habéte, quod est vínculum perfectionis: et pax Christi exsúltet in córdibus vestris, in qua et vocáti estis in uno córpore: et grati estóte. Verbum Christi hábitet in vobis abundánter, in omni sapiéntia, docéntes et commonéntes vosmetípsos psalmis, hymnis et cánticis spirituálibus, in grátia cantántes in córdibus vestris Deo. Omne, quodcúmque fácitis in verbo aut in ópere, ómnia in nómine Dómini Jesu Christi, grátias agéntes Deo et Patri per Jesum Christum, Dóminum nostrum.
}\switchcolumn\portugues{
\rlettrine{M}{eus} irmãos: Como escolhidos de Deus, que sois, santos e amados, revesti-vos de sentimentos íntimos de misericórdia, de bondade, de humildade, de modéstia e de paciência, suportando-vos uns aos outros e perdoando-vos reciprocamente, se porventura algum tem motivos de queixa contra o outro. Assim como o Senhor nos perdoou, assim também devemos perdoar uns aos outros. Acima de tudo, tende caridade; pois esta é o vínculo da perfeição. Que a paz de Cristo, à qual também fostes chamados para formar um só corpo, reine nos vossos corações. Sede reconhecidos! Que a palavra de Cristo permaneça profundamente em vós, ensinando-vos e admoestando-vos com toda sua sabedoria por meio dos Salmos, Hinos e Cânticos espirituais, louvando a Deus nos vossos corações com cânticos. Tudo o que fizerdes, seja em palavras, seja em obras, fazei-o em nome de nosso Senhor Jesus Cristo, dando graças por Ele a Deus Pai.
}\end{paracol}

\paragraphinfo{Gradual}{Sl. 101, 16-17}
\begin{paracol}{2}\latim{
\rlettrine{T}{imébunt} gentes nomen tuum, Dómine, et omnes reges terræ glóriam tuam. ℣. Quóniam ædificávit Dóminus Sion, et vidébitur in majestáte sua.
}\switchcolumn\portugues{
\rlettrine{A}{s} nações temerão o vosso nome, Senhor; e todos os reis da terra contemplarão a vossa glória. ℣. Pois o Senhor reedificou Sião: e manifestar-se-á aí na sua majestade.
}\switchcolumn*\latim{
Allelúja, allelúja. ℣. \emph{Ps. 96,1} Dóminus regnávit, exsúltet terra: læténtur ínsulæ multæ. Allelúja.
}\switchcolumn\portugues{
Aleluia, aleluia. ℣. \emph{Sl. 96,1} O Senhor reinou: alegre-se a terra e as suas muitas ilhas. Aleluia.
}\end{paracol}

\paragraphinfo{Evangelho}{Mt. 13, 24-30}
\begin{paracol}{2}\latim{
\cruz Sequéntia sancti Evangélii secúndum Matthǽum.
}\switchcolumn\portugues{
\cruz Continuação do santo Evangelho segundo S. Mateus.
}\switchcolumn*\latim{
\blettrine{I}{n} illo témpore: Dixit Jesus turbis parábolam hanc: Símile factum est regnum cœlórum hómini, qui seminávit bonum semen in agro suo. Cum autem dormírent hómines, venit inimícus ejus, et superseminávit zizánia in médio trítici, et ábiit. Cum autem crevísset herba et fructum fecísset, tunc apparuérunt et zizánia. Accedéntes autem servi patrisfamílias, dixérunt ei: Dómine, nonne bonum semen seminásti in agro tuo? Unde ergo habet zizánia? Et ait illis: Inimícus homo hoc fecit. Servi autem dixérunt ei: Vis, imus, et collígimus ea? Et ait: Non: ne forte colligéntes zizánia eradicétis simul cum eis et tríticum. Sínite utráque créscere usque ad messem, et in témpore messis dicam messóribus: Collígite primum zizáania, et alligáte ea in fascículos ad comburéndum, tríticum autem congregáta in hórreum meum.
}\switchcolumn\portugues{
\blettrine{N}{aquele}tempo, disse Jesus às turbas: O reino dos céus é semelhante a um homem que havia semeado boa semente no seu campo; mas, enquanto os homens dormiam, veio o seu inimigo; semeou joio entre o trigo e se retirou. Havendo crescido a erva, deu fruto e apareceu também o joio. Então aproximaram-se os servos do pai de família, e disseram-lhe: «Por ventura não semeastes boa semente neste vosso campo? Donde lhe veio, pois, o joio?». Ele disse: «Foi o homem inimigo quem fez isto». Os servos retorquiram-lhe: «Quereis que vamos e colhamos o joio?». Ele disse: «Não, pois receio que, arrancando o joio, arranqueis também o trigo. Deixai crescer ambos até ao tempo da ceifa; e, então, direi aos ceifeiros: colhei primeiramente o joio e atai-o em molhos para serem queimados; e arrecadai o trigo no meu celeiro».
}\end{paracol}

\paragraphinfo{Ofertório}{Sl. 117, 16 \& 17}
\begin{paracol}{2}\latim{
\rlettrine{D}{éxtera} Dómini fecit virtutem, déxtera Dómini exaltávit me: non móriar, sed vivam, et narrábo ópera Dómini.
}\switchcolumn\portugues{
\rlettrine{A}{} dextra do Senhor mostrou o seu poder: a dextra do Senhor exaltou-me. Não morrerei: viverei e narrarei os prodígios do Senhor.
}\end{paracol}

\paragraph{Secreta}
\begin{paracol}{2}\latim{
\rlettrine{H}{óstias} tibi, Dómine, placatiónis offérimus: ut et delícta nostra miserátus absólvas, et nutántia corda tu dírigas. Per Dóminum. \emph{\&c.}
}\switchcolumn\portugues{
\rlettrine{V}{os} oferecemos, Senhor, estas hóstias de propiciação, a fim de que misericordiosamente perdoeis os nossos pecados e ampareis os nossos corações inconstantes. Por nosso Senhor \emph{\&c.}
}\end{paracol}

\paragraphinfo{Comúnio}{Lc. 4, 22}
\begin{paracol}{2}\latim{
\rlettrine{M}{irabántur} omnes de his, quæ procedébant de ore Dei.
}\switchcolumn\portugues{
\rlettrine{T}{odos} estavam admirados das palavras que saíam da boca de Deus.
}\end{paracol}

\paragraph{Postcomúnio}
\begin{paracol}{2}\latim{
\qlettrine{Q}{uǽsumus,} omnípotens Deus: ut illíus salutáris capiámus efféctum, cujus per hæc mystéria pignus accépimus. Per Dóminum \emph{\&c.}
}\switchcolumn\portugues{
\rlettrine{V}{os} suplicamos, ó Deus omnipotente, fazei que obtenhamos o efeito da salvação eterna, da qual, nestes sagrados mystérios, já recebemos o penhor. Por nosso Senhor \emph{\&c.}
}\end{paracol}

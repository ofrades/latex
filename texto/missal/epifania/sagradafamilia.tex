\subsectioninfo{Sagrada Família}{Primeiro Domingo depois da Epifania}\label{sagradafamilia}

\paragraphinfo{Intróito}{Pr. 23, 24 \& 25}
\begin{paracol}{2}\latim{
\rlettrine{E}{xsúltat} gáudio pater Justi, gáudeat Pater tuus et Mater tua, et exsúltet quæ génuit te. \emph{Ps. 83, 2-3} Quam dilécta tabernácula tua, Dómine virtútum! concupíscit et déficit ánima mea in átria Dómini.
℣. Gloria Patri \emph{\&c.}
}\switchcolumn\portugues{
\qlettrine{Q}{ue} o pai do Justo exulte de alegria, que o vosso Pai e a vossa Mãe se alegrem: que se regozije aquela que te gerou. \emph{Sl. 83, 2-3} Como são amáveis os vossos tabernáculos, ó Senhor dos exércitos! Minha alma suspira e enternece-se nos átrios do Senhor.
℣. Glória ao Pai \emph{\&c.}
}\end{paracol}

\paragraph{Oração}
\begin{paracol}{2}\latim{
\rlettrine{D}{ómine} Jesu Christe, qui, Maríæ et Joseph súbditus, domésticam vitam ineffabílibus virtútibus consecrásti: fac nos, utriúsque auxílio, Famíliæ sanctæ tuæ exémplis ínstrui; et consórtium cénsequi sempitérnum: Qui vivis \emph{\&c.}
}\switchcolumn\portugues{
\rlettrine{S}{enhor} Jesus Cristo, que, sendo submisso a Maria e a José, consagrastes a vida doméstica com a prática de inefáveis virtudes, dignai-Vos conceder-nos que com o auxílio de um e outro imitemos os exemplos da vossa Sagrada Família e alcancemos a sua eterna companhia: Vós, que, sendo Deus \emph{\&c.}
}\end{paracol}

\paragraphinfo{Epístola}{Cl. 3, 12-17}
\begin{paracol}{2}\latim{
Léctio Epístolæ beáti Pauli Apóstoli ad Colossénses.
}\switchcolumn\portugues{
Lição da Ep.ª do B. Ap.º Paulo aos Colossenses.
}\switchcolumn*\latim{
\rlettrine{F}{ratres:} Indúite vos sicut elécti Dei, sancti et dilécti, víscera misericórdiæ, benignitátem, humilitátem, modéstiam, patiéntiam: supportántes ínvicem, et donántes vobismetípsis, si quis advérsus áliquem habet querélam: sicut et Dóminus donávit vobis, ita et vos. Super ómnia autem hæc caritátem habéte, quod est vínculum perfectiónis: et pax Christi exsúltet in córdibus vestris, in qua et vocáti estis in uno córpore: et grati estóte. Verbum Christi hábitet in vobis abundánter, in omni sapiéntia, docéntes et commonéntes vosmetípsos psalmis, hymnis et cánticis spirituálibus, in grátia cantántes in córdibus vestris Deo. Omne, quodcúmque fácitis in verbo aut in ópere, ómnia in nómine Dómini Jesu Christi, grátias agéntes Deo et Patri per ipsum.
}\switchcolumn\portugues{
\rlettrine{M}{eus} irmãos: Como escolhidos de Deus, que sois, santos e amados, revesti-vos de sentimentos íntimos de misericórdia, de bondade, de humildade, de modéstia e de paciência, suportando-vos uns aos outros e perdoando-vos reciprocamente, se porventura algum tem motivos de queixa contra o outro. Assim como o Senhor nos perdoou, assim também devemos perdoar uns aos outros. Acima de tudo, tende caridade; pois esta é o vínculo da perfeição. Que a paz de Cristo, à qual também fostes chamados para formar um só corpo, reine nos vossos corações. Sede reconhecidos! Que a palavra de Cristo permaneça profundamente em vós, ensinando-vos e admoestando-vos com toda sua sabedoria por meio dos Salmos, Hinos e Cânticos espirituais, louvando a Deus nos vossos corações com cânticos. Tudo o que fizerdes, seja em palavras, seja em obras, fazei-o em nome de nosso Senhor Jesus Cristo, dando graças por Ele a Deus Pai.
}\end{paracol}

\paragraphinfo{Gradual}{Sl. 26, 4}
\begin{paracol}{2}\latim{
\rlettrine{U}{nam} pétii a Dómino, hanc requíram: ut inhábitem in domo Dómini ómnibus diébus vitæ meæ. ℣. \emph{Ps. 83, 5} Beáti, qui hábitant in domo tua, Dómine: in sǽcula sæculórum laudábunt te.
}\switchcolumn\portugues{
\rlettrine{U}{ma} só graça peço ao Senhor, a qual reclamarei: é habitar na casa do Senhor todos os dias da minha vida. ℣. \emph{Sl. 83, 5} Bem-aventurados aqueles que habitam na vossa casa, Senhor; pois louvar-Vos-ão em todos os séculos.
}\switchcolumn*\latim{
Allelúja, allelúja. ℣. \emph{Isai. 45, 15} Vere tu es Rex abscónditus, Deus Israël Salvátor. Allelúja.
}\switchcolumn\portugues{
Aleluia, aleluia. ℣. \emph{Is. 45, 15} Ó Deus de Israel e nosso Salvador, sois verdadeiramente Rei oculto. Aleluia.
}\end{paracol}

\paragraphinfo{Evangelho}{Lc. 2, 42-52}\label{evangelhosagradafamilia}
\begin{paracol}{2}\latim{
\cruz Sequéntia sancti Evangélii secúndum Lucam.
}\switchcolumn\portugues{
\cruz Continuação do santo Evangelho segundo S. Lucas.
}\switchcolumn*\latim{
\blettrine{C}{um} factus esset Jesus annórum duódecim, ascendéntibus illis Jerosólymam secúndum consuetúdinem diéi
festi, consummatísque diébus, cum redírent, remánsit puer Jesus in Jerúsalem, et non cognovérunt paréntes ejus. Existimántes autem illum esse in comitátu, venérunt iter diéi, et requirébant eum inter cognátos et notos. Et non inveniéntes, regréssi sunt in Jerúsalem, requiréntes eum. Et factum est, post tríduum invenérunt illum in templo sedéntem in médio doctórum, audiéntem illos et interrogántem eos. Stupébant autem omnes, qui eum audiébant, super prudéntia et respónsis ejus. Et vidéntes admiráti sunt. Et dixit Mater ejus ad illum: Fili, quid fecísti nobis sic? Ecce, pater tuus et ego doléntes quærebámus te. Et ait ad illos: Quid est, quod me quærebátis? Nesciebátis, quia in his, quæ Patris mei sunt, opórtet me esse? Et ipsi non intellexérunt verbum, quod locútus est ad eos. Et descéndit cum eis, et venit Názareth: et erat súbditus illis. Et Mater ejus conservábat ómnia verba hæc in corde suo. Et Jesus proficiébat sapiéntia et ætáte et grátia apud Deum et hómines.
}\switchcolumn\portugues{
\blettrine{Q}{uando} Jesus completou doze anos de idade, como seus pais tivessem ido a Jerusalém, no tempo da festa, segundo o costume, decorridos que foram os dias da mesma, voltaram para casa, tendo o Menino Jesus ficado em Jerusalém, sem que de tal os pais se apercebessem. Pensando que Ele viria com seus companheiros de jornada, fizeram um dia de viagem, procurando-O depois entre os parentes e os conhecidos. Não O encontrando, voltaram logo a Jerusalém pelo mesmo caminho. Então, aconteceu que, depois de três dias, foram achá-l’O no templo, sentado no meio dos doutores, ouvindo-os e interrogando-os. E aqueles que O ouviam estavam admirados da sua sabedoria e das suas respostas. Quando os pais O encontraram, ficaram admirados, dizendo-Lhe logo a Mãe: «Meu Filho, porque procedestes assim para connosco? Eis que vosso pai e eu Vos buscávamos aflitos!» Ele disse-lhes: «Porque me procuráveis? Não sabíeis que é preciso que me ocupe das cousas de meu Pai?», Porém eles não compreenderam o que Jesus lhes disse, Então, desceu com eles, veio para Nazaré e era-lhes obediente. E sua Mãe conservava todas estas cousas no coração. Quanto a Jesus, crescia em sabedoria, em idade e em graça, diante de Deus e dos homens.
}\end{paracol}

\paragraphinfo{Ofertório}{Lc. 2, 22}
\begin{paracol}{2}\latim{
\rlettrine{T}{ulérunt} Jesum paréntes ejus in Jerúsalem, ut sísterent eum Dómino.
}\switchcolumn\portugues{
\rlettrine{O}{s} pais de Jesus levaram-n’O a Jerusalém para O oferecerem ao Senhor.
}\end{paracol}

\paragraph{Secreta}
\begin{paracol}{2}\latim{
\rlettrine{P}{lacatiónis} hostiam offérimus tibi, Dómine, supplíciter deprecántes: ut, per intercessiónem Deíparæ Vírginis cum beáto Joseph, famílias nostras in pace et grátia tua fírmiter constítuas. Per eúndem Dóminum \emph{\&c.}
}\switchcolumn\portugues{
\rlettrine{V}{os} oferecemos, Senhor, esta hóstia de propiciação, suplicando-Vos humildemente que, pela intercessão da Virgem, Mãe de Deus, e do B. José, estabeleçais solidamente as nossas famílias na vossa paz e na vossa graça. Pelo mesmo nosso Senhor \emph{\&c.}
}\end{paracol}

\paragraphinfo{Comúnio}{Lc. 2, 51}
\begin{paracol}{2}\latim{
\rlettrine{D}{escéndit} Jesus cum eis, et venit Názareth, et erat súbditus illis.
}\switchcolumn\portugues{
\qlettrine{J}{esus} desceu com eles (os pais), veio para Nazaré e era-lhes obediente.
}\end{paracol}

\paragraph{Postcomúnio}
\begin{paracol}{2}\latim{
\qlettrine{Q}{uos} cœléstibus réficis sacraméntis, fac, Dómine Jesu, sanctæ Famíliæ tuæ exémpla júgiter imitári: ut in hora mortis nostræ, occurrénte gloriósa Vírgine Matre tua cum beáto Joseph; per te in ætérna tabernácula récipi mereámur: Qui vivis \emph{\&c.}
}\switchcolumn\portugues{
\rlettrine{S}{enhor} Jesus, concedei àqueles que se saciam com vossos celestiais sacramentos a graça de imitarem continuamente os exemplos da vossa Sagrada Família, a fim de que na hora da nossa morte a gloriosa Virgem, vossa Mãe, e o B. José venham ao nosso encontro, merecendo sermos recebidos por Vós nos tabernáculos eternos: Ó Vós, que \emph{\&c.}
}\end{paracol}

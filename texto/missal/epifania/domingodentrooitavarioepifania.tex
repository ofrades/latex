\subsection{Domingo dentro do Oitavário da Epifania}

\paragraph{Intróito}
\begin{paracol}{2}\latim{
\rlettrine{I}{n} excélso throno vidi sedére virum, quem adórat multitúdo Angelórum, psalléntes in unum: ecce, cujus impérii nomen est in ætérnum. \emph{Ps. 99, 1} Jubiláte Deo, omnis terra: servíte Dómino in lætítia.
℣. Gloria Patri \emph{\&c.}
}\switchcolumn\portugues{
\rlettrine{V}{i} um Homem, sentado em um trono muito elevado, a quem a multidão dos Anjos adorava, cantando em coro de vozes: «Eis Aquele cujo império é eterno!»
\emph{Sl. 99, 1} Aclamai jubilosamente o Senhor, ó habitantes da terra: obedecei ao Senhor com alegria.
℣. Glória ao Pai \emph{\&c.}
}\end{paracol}

\paragraph{Oração}
\begin{paracol}{2}\latim{
\rlettrine{V}{ota,} quǽsumus, Dómine, supplicántis pópuli cœlésti pietáte proséquere: ut et, quæ agénda sunt, vídeant, et ad implénda, quæ víderint, convaléscant Per Dóminum nostrum \emph{\&c.}
}\switchcolumn\portugues{
\rlettrine{A}{ceitai,} Senhor, com vossa celestial bondade os votos do vosso povo suplicante; e, Vos imploramos, permiti que os vossos servos conheçam o que devem cumprir e tenham coragem de cumprir o que conhecem. Por nosso Senhor \emph{\&c.}
}\end{paracol}

\paragraphinfo{Epístola}{Rm. 12, 1-5}
\begin{paracol}{2}\latim{
Léctio Epístolæ beáti Pauli Apóstoli ad Romános
}\switchcolumn\portugues{
Lição da Ep.ª do B. Ap. Paulo aos Romanos.
}\switchcolumn*\latim{
\rlettrine{F}{ratres:} Obsecro vos per misericórdiam Dei, ut exhibeátis córpora vestra hóstiam vivéntem, sanctam, Deo placéntem, rationábile obséquium vestrum. Et nolíte conformári huic sǽculo, sed reformámini in novitáte sensus vestri: ut probétis, quæ sit volúntas Dei bona, et benéplacens, et perfécta. Dico enim per grátiam, quæ data est mihi, ómnibus qui sunt inter vos: Non plus sápere, quam opórtet sápere, sed sápere ad sobrietátem: et unicuique sicut Deus divísit mensúram fídei. Sicut enim in uno córpore multa membra habémus, ómnia autem membra non eúndem actum habent: ita multi unum corpus sumus in Christo, sínguli autem alter alteríus membra: in Christo Jesu, Dómino nostro.
}\switchcolumn\portugues{
\rlettrine{M}{eus} irmãos: Peço-vos pela misericórdia de Deus que ofereçais os vossos corpos como hóstia viva, santa e agradável a Deus, para que o culto que Lhe prestais seja racional. Não vos conformeis com os costumes; mas reformai-vos com a graça do espírito novo, que agora possuís, para que conheçais qual é a vontade de Deus, o que é bom, agradável e perfeito. Exorto-vos, pois, a todos vós, pela graça que me foi dada, que não formeis de vós um juízo muito elevado, mas que tenhais sentimentos modestos de vós, cada um conforme a medida que Deus lhe concedeu. Porquanto, assim como em um só corpo temos muitos membros (os quais, contudo, não têm a mesma função), assim também, ainda que sejamos vários, contudo formamos um só corpo em Jesus Cristo, sendo cada um de nós em particular membros uns dos outros, em Jesus Cristo, nosso Senhor.
}\end{paracol}

\paragraphinfo{Gradual}{Sl. 71, 18 \& 3}
\begin{paracol}{2}\latim{
\rlettrine{B}{enedíctus} Dóminus, Deus Israël, qui facit mirabília magna solus a sǽculo. ℣. Suscípiant montes pacem pópulo tuo, et colles justítiam.
}\switchcolumn\portugues{
\rlettrine{B}{endito} seja o Senhor Deus de Israel, pois somente Ele opera prodígios em todos os séculos. Que os montes do vosso povo sejam bafejados pela paz: e as colinas pela justiça.
}\switchcolumn*\latim{
Allelúja, allelúja. ℣. \emph{Ps. 99, 1} Jubiláte Deo, omnis terra: servíte Dómino in lætítia. Allelúja.
}\switchcolumn\portugues{
Aleluia, aleluia. ℣. \emph{Sl. 99, 1} Aclamai jubilosamente o Senhor, ó habitantes da terra: obedecei ao Senhor com alegria. Aleluia.
}\end{paracol}

\paragraphinfo{Evangelho}{Página \pageref{evangelhosagradafamilia}}

\paragraphinfo{Ofertório}{Sl. 99, 1 \& 2}
\begin{paracol}{2}\latim{
\qlettrine{J}{ubiláte} Deo, omnis terra, servíte Dómino in lætítia: intráte in conspéctu ejus in exsultatióne: quia Dóminus ipse est Deus.
}\switchcolumn\portugues{
\rlettrine{A}{clamai} jubilosamente o Senhor, ó habitantes da terra: obedecei ao Senhor com alegria: apresentai-vos diante d’Ele com alegria; pois Ele é bom.
}\end{paracol}

\paragraph{Secreta}
\begin{paracol}{2}\latim{
\rlettrine{O}{blátum} tibi, Dómine, sacrificium vivíficet nos semper et múniat. Per Dóminum nostrum \emph{\&c.}
}\switchcolumn\portugues{
\rlettrine{F}{azei,} Senhor, que este sacrifício, que Vos é oferecido, sempre nos vivifique e conforte. Por nosso Senhor \emph{\&c.}
}\end{paracol}

\paragraphinfo{Comúnio}{Lc. 2, 48 \& 49}
\begin{paracol}{2}\latim{
\rlettrine{F}{ili,} quid fecísti nobis sic? ego et pater tuus doléntes quærebámus te. Et quid est, quod me quærebátis? nesciebátis, quia in his, quæ Patris mei sunt, opórtet me esse?
}\switchcolumn\portugues{
\rlettrine{M}{eu} filho, porque procedestes assim para connosco? Eis que o vosso pai e eu Vos buscávamos aflitos! E porque me procuráveis? Não sabíeis que é preciso que me ocupe das coisas de meu Pai?
}\end{paracol}

\paragraph{Postcomúnio}
\begin{paracol}{2}\latim{
\rlettrine{S}{úpplices} te rogámus, omnípotens Deus: ut, quos tuis réfícis sacraméntis, tibi etiam plácitis móribus dignánter deservíre concédas. Per Dóminum \emph{\&c.}
}\switchcolumn\portugues{
\rlettrine{H}{umildemente} Vos suplicamos, ó Deus omnipotente, concedei àqueles que se sustentam com vossos sacramentos a graça de Vos servirem com actos que Vos sejam agradáveis. Por nosso Senhor \emph{\&c.}
}\end{paracol}

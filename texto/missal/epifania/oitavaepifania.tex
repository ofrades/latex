\subsectioninfo{Dia da oitava da Epifania}{13 de Janeiro}

\textit{Como no dia da Festa da Epifania, na página \pageref{epifania}, excepto o seguinte:}

\paragraph{Oração}
\begin{paracol}{2}\latim{
\rlettrine{D}{eus,} cujus Unigénitus in substántia nostræ carnis appáruit: præsta, quǽsumus; ut per eum, quem símilem nobis foris agnóvimus, intus reformári mereámur: Qui tecum \emph{\&c.}
}\switchcolumn\portugues{
\slettrine{Ó}{} Deus, cujo Filho Unigénito apareceu na terra revestido com a substância da nossa carne, permiti, Vos rogamos, que mereçamos a graça de sermos reformados interiormente por Aquele que reconhecemos semelhante a nós exteriormente. Ele, que, sendo Deus \emph{\&c.}
}\end{paracol}

\paragraphinfo{Evangelho}{Jo. 1, 29-34}
\begin{paracol}{2}\latim{
\cruz Sequéntia sancti Evangélii secúndum Joánnem.
}\switchcolumn\portugues{
\cruz Continuação do santo Evangelho segundo S. João.
}\switchcolumn*\latim{
\blettrine{I}{n} illo témpore: Vidit Joánnes Jesum veniéntem ad se, et ait: Ecce Agnus Dei, ecce, qui tollit peccátum mundi. Hic est, de quo dixi: Post me venit vir, qui ante me factus est: quia prior me erat. Et ego nesciébam eum, sed ut manifestétur in Israël, proptérea veni ego in aqua baptízans. Et testimónium perhíbuit Joánnes, dicens: Quia vidi Spíritum descendéntem quasi colúmbam de cœlo, et mansit super eum. Et ego nesciébam eum: sed qui misit me baptizáre in aqua, ille mihi dixit: Super quem víderis Spíritum descendéntem, et manéntem super eum, hic est, qui baptízat in Spíritu Sancto. Et ego vidi: et testimónium perhíbui, quia hic est Fílius Dei.
}\switchcolumn\portugues{
\blettrine{N}{aquele} tempo, João viu Jesus, que caminhava para ele, e disse: «Eis o Cordeiro de Deus, eis o que tira o pecado do mundo. Este é Aquele de quem eu disse: «Depois de mim vem um homem, que me precedeu, porque já existia antes de mim. Eu O não conhecia, mas foi para que Ele fosse manifestado a Israel que vim baptizar na água». E João deu este testemunho, dizendo: «Eu vi o Espírito descer do céu, sob a forma de uma pomba, e pousar sobre Ele. Eu O não conhecia, mas Aquele que me enviou a baptizar na água disse-me: «Aquele sobre quem vires o Espírito descer e pousar é o que baptiza no Espírito Santo». Eu vi isto e afirmo que Ele é o Filho de Deus».
}\end{paracol}

\paragraph{Secreta}
\begin{paracol}{2}\latim{
\rlettrine{H}{óstias} tibi, Dómine, pro nati Fílii tui apparitióne deférimus, supplíciter exorántes: ut, sicut ipse nostrórum auctor est múnerum, ita sit ipse miséricors et suscéptor, Jesus Christus, Dóminus noster: Qui tecum \emph{\&c.}
}\switchcolumn\portugues{
\rlettrine{S}{enhor,} Vos oferecemos sacrifícios em memória da manifestação do vosso Filho, que nasceu no mundo, suplicando-Vos que, assim como Jesus Cristo, nosso Senhor, é o autor destes dons, assim também os aceites misericordiosamente. Ele, que, sendo Deus \emph{\&c.}
}\end{paracol}

\paragraph{Postcomúnio}
\begin{paracol}{2}\latim{
\rlettrine{C}{œlésti} lúmine, quǽsumus, Dómine, semper et ubíque nos prǽveni: ut mystérium, cujus nos partícipes esse voluísti, et puro cernámus intúitu, et digno percipiámus affectu. Per Dóminum nostrum \emph{\&c.}
}\switchcolumn\portugues{
\rlettrine{D}{ignai-Vos} assistir-nos sempre e em toda a parte com vossa celestial luz, Senhor, Vos pedimos, a fim de que, assim corno quisestes que participássemos deste mistério, assim também possamos contemplá-lo com olhos puros e recebê-lo com afecto digno. Por nosso Senhor \emph{\&c.}
}\end{paracol}

\subsection{Segundo Domingo depois da Epifania}

\paragraphinfo{Intróito}{Sl. 65, 4}
\begin{paracol}{2}\latim{
\rlettrine{O}{mnis} terra adóret te, Deus, et psallat tibi: psalmum dicat nómini tuo, Altíssime. \emph{Ps. ibid., 1-2} Jubiláte Deo, omnis terra, psalmum dícite nómini ejus: date glóriam laudi ejus.
℣. Gloria Patri \emph{\&c.}
}\switchcolumn\portugues{
\qlettrine{Q}{ue} todo o universo Vos adore, ó Deus, e cante hinos em vosso louvor: Que toda a terra louve com cânticos a vossa glória, ó Altíssimo! \emph{Sl. ibid., 1-2} Aclamai jubilosamente o Senhor, ó habitantes da terra: cantai hinos em honra do seu nome: glorificai-O com vossos louvores.
℣. Glória ao Pai \emph{\&c.}
}\end{paracol}

\paragraph{Oração}
\begin{paracol}{2}\latim{
\rlettrine{O}{mnípotens} sempitérne Deus, qui cœléstia simul et terréna moderáris: supplicatiónes pópuli tui cleménter exáudi; et pacem tuam nostris concéde tempóribus. Per Dóminum \emph{\&c.}
}\switchcolumn\portugues{
\slettrine{Ó}{} Deus omnipotente e eterno, que governais ao mesmo tempo o céu e a terra, ouvi misericordiosamente as súplicas do vosso povo, e concedei a vossa paz aos nossos tempos. Por nosso Senhor \emph{\&c.}
}\end{paracol}

\paragraphinfo{Epístola}{}
\begin{paracol}{2}\latim{
Léctio Epístolæ beáti Pauli Apóstoli ad Romános.
}\switchcolumn\portugues{
Lição da Ep.ª do B. Ap.º Paulo aos Romanos.
}\switchcolumn*\latim{
\rlettrine{F}{ratres:} Habéntes donatiónes secúndum grátiam, quæ data est nobis, differéntes: sive prophétiam secúndum ratiónem fídei, sive ministérium in ministrándo, sive qui docet in doctrína, qui exhortátur in exhortándo, qui tríbuit in simplicitáte, qui præest in sollicitúdine, qui miserétur in hilaritáte. Diléctio sine simulatióne. Odiéntes malum, adhæréntes bono: Caritáte fraternitátis ínvicem diligéntes: Honóore ínvicem præveniénte s: Sollicitúdine non pigri: Spíritu fervéntes: Dómino serviéntes: Spe gaudéntes: In tribulatióne patiéntes: Oratióni instántes: Necessitátibus sanctórum communicántes: Hospitalitátem sectántes. Benedícite persequéntibus vos: benedícite, et nolíte maledícere. Gaudére cum gaudéntibus, flere cum fléntibus: Idípsum ínvicem sentiéntes: Non alta sapiéntes, sed humílibus consentiéntes.
}\switchcolumn\portugues{
\rlettrine{V}{isto} que todos recebemos dons diferentes, conforme a graça que nos foi dada, assim, pois, aquele que recebeu o dom da profecia, exerça-a, segundo a regra da fé; aquele que foi chamado para o ministério do ensino, exerça-o; aquele que recebeu o dom de exortar, exorte; aquele que dá esmola, dê-a com simplicidade; aquele que governa, governe com solicitude; aquele que pratica as obras de misericórdia, pratique-as com alegria. Que a vossa caridade seja sem fingimento. Odiai o mal e amai o bem. Amai-vos reciprocamente com amor fraternal, sendo cada um solícito em honrar o outro. Com o zelo vencei a preguiça; conservai o fervor no espírito; servi o Senhor; alegrai-vos na esperança; sede pacientes nas tribulações; assíduos na oração; prontos em socorrer as necessidades dos fiéis; praticai a hospitalidade; abençoai aqueles que vos perseguem: abençoai-os e os não amaldiçoeis; alegrai-vos com os que estão alegres; chorai com os que choram; tende entre vós os mesmos sentimentos; não aspireis ao que é elevado, mas aceitai o que é humilde.
}\end{paracol}

\paragraphinfo{Gradual}{Sl. 106, 20-21}
\begin{paracol}{2}\latim{
\rlettrine{M}{isit} Dóminus verbum suum, et sanávit eos: et erípuit eos de intéritu eórum. ℣. Confiteántur Dómino misericórdiæ ejus: et mirabília ejus fíliis hóminum.
}\switchcolumn\portugues{
\rlettrine{E}{nviou} o Senhor a sua palavra; curou-os e livrou-os da morte. Louvai o Senhor pela sua misericórdia: e pelos seus prodígios para com os filhos dos homens.
}\switchcolumn*\latim{
Allelúja, allelúja. ℣. \emph{Ps. 148, 2} Laudáte Dóminum, omnes Angeli ejus: laudáte eum, omnes virtútes ejus. Allelúja.
}\switchcolumn\portugues{
Aleluia, aleluia. ℣. \emph{Sl. 148, 2} Louvai o Senhor, vós todos, que sois os seus Anjos: louvai-O, vós todos, que sois os seus exércitos. Aleluia.
}\end{paracol}

\paragraphinfo{Evangelho}{Jo. 2, 1-11}
\begin{paracol}{2}\latim{
\cruz Sequéntia sancti Evangélii secúndum Joánnem.
}\switchcolumn\portugues{
\cruz Continuação do santo Evangelho segundo S. João.
}\switchcolumn*\latim{
\blettrine{I}{n} illo témpore: Núptiæ factæ sunt in Cana Galilǽæ: et erat Mater Jesu ibi. Vocátus est autem et Jesus, et discípuli ejus ad núptias. Et deficiénte vino, dicit Mater Jesu ad eum: Vinum non habent. Et dicit ei Jesus: Quid mihi et tibi est, mulier? nondum venit hora mea. Dicit Mater ejus minístris: Quodcúmque díxerit vobis, fácite. Erant autem ibi lapídeæ hýdriæ sex pósitæ secúndum purificatiónem Judæórum, capiéntes síngulæ metrétas binas vel ternas. Dicit eis Jesus: Implete hýdrias aqua. Et implevérunt eas usque ad summum. Et dicit eis Jesus: Hauríte nunc, et ferte architriclíno. Et tulérunt. Ut autem gustávit architriclínus aquam vinum fáctam, et non sciébat unde esset, minístri autem sciébant, qui háuserant aquam: vocat sponsum architriclínus, et dicit ei: Omnis homo primum bonum vinum ponit: et cum inebriáti fúerint, tunc id, quod detérius est. Tu autem servásti bonum vinum usque adhuc. Hoc fecit inítium signórum Jesus in Cana Galilǽæ: et manifestávit glóriam suam, et credidérunt in eum discípuli ejus.
}\switchcolumn\portugues{
\blettrine{N}{aquele} tempo, celebraram-se as bodas em Caná, de Galileia, e a Mãe de Jesus estava presente. Jesus foi também convidado com seus discípulos para assistir às núpcias. Havendo, então, faltado o vinho, a Mãe de Jesus disse-Lhe: «Não têm vinho». Jesus respondeu-lhe: «Mulher, que tenho Eu e vós com isso? Ainda não chegou a minha hora». Mas sua Mãe disse aos servos: «Fazei tudo quanto Ele vos disser». Ora, estavam ali seis talhas de pedra para servirem nas purificações dos judeus, contendo cada uma delas duas ou três medidas. Jesus disse-lhes então: «Enchei as talhas com água». Eles as encheram até acima. E Jesus continuou: «Tirai, agora, e levai ao que dirige o banquete». Eles assim fizeram. Logo que o que dirigia o banquete provou o vinho (ele não sabia donde viera este vinho, mas os servos, que haviam tirado a água, sabiam) chamou o esposo e disse-lhe: «Todo o homem serve primeiramente o bom vinho, e depois que se bebe dele abundantemente é que serve o inferior; tu, porém, guardaste o melhor até este momento». Tal foi, ó Caná, de Galileia, o primeiro milagre que Jesus fez! Assim manifestou a sua glória, acreditando n’Ele os seus discípulos.
}\end{paracol}

\paragraphinfo{Ofertório}{Sl. 65, 1-2 \& 16}
\begin{paracol}{2}\latim{
\qlettrine{J}{ubiláte} Deo, univérsa terra: psalmum dícite nómini ejus: veníte et audíte, et narrábo vobis, omnes qui timétis Deum, quanta fecit Dóminus ánimæ meæ, allelúja.
}\switchcolumn\portugues{
\rlettrine{A}{clamai} jubilosamente o Senhor, ó habitantes da terra: cantai hinos em honra do seu nome! Vinde e ouvi, vós todos, que temeis Deus, e contar-vos-ei as graças que o Senhor fez à minha alma. Aleluia.
}\end{paracol}

\paragraph{Secreta}
\begin{paracol}{2}\latim{
\rlettrine{O}{blata,} Dómine, múnera sanctífica: nosque a peccatórum nostrórum máculis emúnda. Per Dóminum nostrum \emph{\&c.}
}\switchcolumn\portugues{
\rlettrine{S}{antificai,} Senhor, as oblatas que Vos oferecemos; e purificai-nos das manchas dos nossos pecados. Por nosso Senhor \emph{\&c.}
}\end{paracol}

\paragraphinfo{Comúnio}{Jo. 2, 7, 8, 9 \& 10-11}
\begin{paracol}{2}\latim{
\rlettrine{D}{icit} Dóminus: Implete hýdrias aqua et ferte architriclíno. Cum gustásset architriclínus aquam vinum factam, dicit sponso: Servásti bonum vinum usque adhuc. Hoc signum fecit Jesus primum coram discípulis suis.
}\switchcolumn\portugues{
\rlettrine{D}{isse} o Senhor: «Enchei estas talhas com água e levai-as ao que dirige o banquete». E, logo que o que dirigia o banquete provou a água, mudada em vinho, disse ao esposo: «Tu guardaste o bom vinho até este momento!». Tal foi o primeiro milagre que Jesus fez na presença de seus discípulos.
}\end{paracol}

\paragraph{Postcomúnio}
\begin{paracol}{2}\latim{
\rlettrine{A}{ugeátur} in nobis, quǽsumus, Dómine, tuæ virtútis operatio: ut divínis vegetáti sacraméntis, ad eórum promíssa capiénda, tuo múnere præparémur. Per Dóminum nostrum \emph{\&c.}
}\switchcolumn\portugues{
\rlettrine{S}{enhor,} dignai-Vos aumentar em nós os efeitos do vosso poder, a fim de que, alimentados com os divinos sacramentos, nos preparemos com vossa graça para alcançar os benefícios, de que são o penhor. Por nosso Senhor \emph{\&c.}
}\end{paracol}

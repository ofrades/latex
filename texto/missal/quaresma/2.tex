\subsection{Segundo Domingo da Quaresma}\label{2domingoquaresma}

\paragraphinfo{Intróito}{Sl. 24, 6, 3 \& 22}
\begin{paracol}{2}\latim{
\rlettrine{R}{eminíscere} miseratiónum tuarum, Dómine, et misericórdiæ tuæ, quæ a sǽculo sunt: ne umquam dominéntur nobis inimíci nostri: líbera nos, Deus Israël, ex ómnibus angústiis nostris. \emph{Ps. ibid., 1-2} Ad te, Dómine, levávi ánimam meam: Deus meus, in te confído, non erubéscam.
℣. Gloria Patri \emph{\&c.}
}\switchcolumn\portugues{
\rlettrine{L}{embrai-Vos,} Senhor, de que a vossa bondade e misericórdia são eternas! Que os nossos inimigos nunca triunfem de nós. Ó Deus de Israel, livrai-nos de todas nossas angústias. \emph{Sl. ibid., 1-2} A Vós, Senhor, elevei a minha alma: meu Deus, confio em Vós; não ficarei confundido.
℣. Glória ao Pai \emph{\&c.}
}\end{paracol}

\paragraph{Oração}
\begin{paracol}{2}\latim{
\rlettrine{D}{eus,} qui cónspicis omni nos virtúte destítui: intérius exteriúsque custódi; ut ab ómnibus adversitátibus muniámur In córpore, et a pravis cogitatiónibus mundémur in mente. Per Dóminum \emph{\&c.}
}\switchcolumn\portugues{
\slettrine{Ó}{} Deus, que conheceis como somos destituídos de toda a virtude, guardai-nos interior e exteriormente, a fim de que o nosso corpo seja preservado de todas as adversidades e a nossa alma purificada de todos os maus pensamentos. Por nosso Senhor \emph{\&c.}
}\end{paracol}

\paragraphinfo{Epístola}{1 Ts. 4, 1-7}
\begin{paracol}{2}\latim{
Léctio Epístolæ beáti Pauli Apóstoli ad Thessalonicénses.
}\switchcolumn\portugues{
Lição da Ep.ª do B. Ap.º Paulo aos Tessalonicenses.
}\switchcolumn*\latim{
\rlettrine{F}{ratres:} Rogámus vos et obsecrámus in Dómino Jesu: ut, quemádmodum accepístis a nobis, quómodo opórteat vos ambuláre et placére Deo, sic et ambulétis, ut abundétis magis. Scitis enim, quæ præcépta déderim vobis per Dóminum Jesum. Hæc est enim volúntas Dei, sanctificátio vestra: ut abstineátis vos a fornicatióne, ut sciat unusquísque vestrum vas suum possidére in sanctifícatióne et honóre; non in passióne desidérii, sicut et gentes, quæ ignórant Deum: et ne quis supergrediátur neque circumvéniat in negótio fratrem suum: quóniam vindex est Dóminus de his ómnibus, sicut prædíximus vobis et testificáti sumus. Non enim vocávit nos Deus in immundítiam, sed in sanctificatiónem: in Christo Jesu, Dómino nostro.
}\switchcolumn\portugues{
\rlettrine{M}{eus} irmãos: Vos pedimos e exortamos, em nome do Senhor Jesus, que, havendo aprendido de nós como deveis conduzir-vos para agradar a Deus, tenhais uma conduta de modo a aperfeiçoar-vos cada vez mais. Com efeito, conheceis os preceitos que vos dei da parte do Senhor Jesus. O que Ele quer é a vossa santificação: que vos abstenhais da impureza carnal, para que cada um saiba guardar o vaso do seu corpo em santidade e honestidade, e não segundo os apetites das paixões, como os pagãos, que desconhecem Deus; e que ninguém engane o seu irmão, nem o prejudique, pois, o Senhor é vingador destas coisas, como já provámos e testemunhámos. Porquanto Deus vos não chamou para os gozos da carne, mas para a santificação em nosso Senhor Jesus Cristo.
}\end{paracol}

\paragraphinfo{Gradual}{Sl. 24, 17-18}
\begin{paracol}{2}\latim{
\rlettrine{T}{ribulatiónes} cordis mei dilatátæ sunt: de necessitátibus meis éripe me, Dómine. ℣. Vide humilitátem meam et labórem meum: et dimítte ómnia peccáta mea.
}\switchcolumn\portugues{
\rlettrine{A}{s} tribulações do meu coração cresceram. Ó Senhor, livrai-me das minhas misérias. ℣. Vede a minha humilhação e fadiga e perdoai os meus pecados.
}\end{paracol}

\paragraphinfo{Trato}{Sl. 105, 1-1}
\begin{paracol}{2}\latim{
\rlettrine{C}{onfitémini} Dómino, quóniam bonus: quóniam in sǽculum misericórdia ejus. ℣. Quis loquétur poténtias Dómini: audítas fáciet omnes laudes ejus? ℣. Beáti, qui custódiunt judícium et fáciunt justítiam in omni témpore. ℣. Meménto nostri, Dómine, in beneplácito pópuli tui: vísita nos in salutári tuo.
}\switchcolumn\portugues{
\rlettrine{L}{ouvai} o Senhor, pois Ele é bom: a sua misericórdia é eterna. ℣. Quem será capaz de narrar as maravilhas da omnipotência do Senhor e apregoar os seus louvores? ℣. Bem-aventurados aqueles que procedem com equidade e justiça em todas as ocasiões. ℣. Pela vossa bondade para com vosso povo, Senhor, lembrai-Vos de nós: visitai-nos para alcançarmos a salvação.
}\end{paracol}

\paragraphinfo{Evangelho}{Mt, 17, 1-9}
\begin{paracol}{2}\latim{
\cruz Sequéntia sancti Evangélii secúndum Matthǽum.
}\switchcolumn\portugues{
\cruz Continuação do santo Evangelho segundo S. Mateus.
}\switchcolumn*\latim{
\blettrine{I}{n} illo témpore: Assúmpsit Jesus Petrum, et Jacóbum, et Joánnem fratrem eius, et duxit illos in montem excélsum seórsum: et transfigurátus est ante eos. Et resplénduit fácies ejus sicut sol: vestiménta autem ejus facta sunt alba sicut nix. Et ecce, apparuérunt illis Móyses et Elías cum eo loquéntes. Respóndens autem Petrus, dixit ad Jesum: Dómine, bonum est nos hic esse: si vis, faciámus hic tria tabernácula, tibi unum, Móysi unum et Elíæ unum. Adhuc eo loquénte, ecce, nubes lúcida obumbrávit eos. Et ecce vox de nube, dicens: Hic est Fílius meus diléctus, in quo mihi bene complácui: ipsum audíte. Et audiéntes discípuli, cecidérunt in fáciem suam, et timuérunt valde. Et accéssit Jesus, et tétigit eos, dixítque eis: Súrgite, et nolíte timére. Levántes autem óculos suos, néminem vidérunt nisi solum Jesum. Et descendéntibus illis de monte, præcépit eis Jesus, dicens: Némini dixéritis visiónem, donec Fílius hóminis a mórtuis resúrgat.
}\switchcolumn\portugues{
\blettrine{N}{aquele} tempo, Jesus levou consigo Pedro, Tiago e João e conduziu-os a um monte alto e separado, transfigurando-se ante eles: seu rosto resplandecia, como o sol, e os seus vestidos tornaram-se brancos, como a neve! E Moisés e Elias apareceram, conversando com Jesus. Então, Pedro disse a Jesus: «Senhor, é tão bom estar aqui!... Se quereis, façamos aqui três tendas: uma para Vós, outra para Moisés e outra para Elias!». Ainda ele falava, eis que uma nuvem brilhante os envolveu, saindo do seio dela uma voz, que dizia: «Este é o meu Filho muito amado, em quem pus as minhas complacências; ouvi-O». Havendo escutado a voz, os discípulos caíram com o rosto no chão e ficaram atemorizados. Mas Jesus tocou-os e disse-lhes: «Levantai-vos; não vos amedronteis». Então, erguendo os olhos, já nada viram senão só Jesus. Desceram do monte, dando-lhes Jesus esta ordem: «Não conteis a ninguém esta visão até que o Filho do homem ressuscite dos mortos».
}\end{paracol}

\paragraphinfo{Ofertório}{Sl. 118,47 \& 48}
\begin{paracol}{2}\latim{
\rlettrine{M}{editábor} in mandátis tuis, quæ diléxi valde: et levábo manus meas ad mandáta tua, quæ diléxi.
}\switchcolumn\portugues{
\rlettrine{M}{editarei} nos vossos Mandamentos, que muito amo: e levantarei as minhas mãos para cumprir esses Mandamentos, que, repito, muito amo.
}\end{paracol}

\paragraph{Secreta}
\begin{paracol}{2}\latim{
\rlettrine{S}{acrifíciis} præséntibus, Dómine, quǽsumus, inténde placátus: ut et devotióni nostræ profíciant et salúti. Per Dóminum \emph{\&c.}
}\switchcolumn\portugues{
\rlettrine{S}{enhor,} Vos suplicamos, dignai-Vos olhar benigno para o presente sacrifício, a fim de que sirva de proveito à nossa piedade e à nossa salvação. Por nosso Senhor \emph{\&c.}
}\end{paracol}

\paragraphinfo{Comúnio}{Sl. 5, 2-4}
\begin{paracol}{2}\latim{
\rlettrine{I}{ntéllege} clamórem meum: inténde voci oratiónis meæ, Rex meus et Deus meus: quóniam ad te orábo, Dómine.
}\switchcolumn\portugues{
\rlettrine{O}{uvi} o meu clamor; ouvi a minha oração, ó meu Rei, ó meu Deus; pois orei a Vós.
}\end{paracol}

\paragraph{Postcomúnio}
\begin{paracol}{2}\latim{
\rlettrine{S}{úpplices} te rogámus, omnípotens Deus: ut quos tuis réficis sacraméntis, tibi etiam plácitis móribus dignánter deservíre concédas. Per Dóminum \emph{\&c.}
}\switchcolumn\portugues{
\slettrine{Ó}{} Deus omnipotente, Vos suplicamos instantemente que aqueles a quem sustentais com vossos sacramentos tenham uma conduta de vida que Vos seja agradável. Por nosso Senhor \emph{\&c.}
}\end{paracol}
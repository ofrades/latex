\subsectioninfo{Segunda-feira da 2.ª Semana da Quaresma}{Estação em S. Clemente}

\paragraphinfo{Intróito}{Sl. 25, 11-12}
\begin{paracol}{2}\latim{
\rlettrine{R}{édime} me, Dómine, et miserére mei: pes enim meus stetit in via recta: in ecclésiis benedícam Dóminum. \emph{Ps. ibid., 1} Júdica me, Dómine, quóniam ego in innocéntia mea ingréssus sum: et in Dómino sperans, non infirmábor.
℣. Gloria Patri \emph{\&c.}
}\switchcolumn\portugues{
\rlettrine{R}{esgatai-me,} Senhor, e tende piedade de mim; pois os meus pés estão no caminho direito: bendirei o Senhor nas assembleias. \emph{Sl. ibid., 1} Fazei-me justiça, Senhor, porque procedi com inocência; e confiei no Senhor, sem nunca vacilar.
℣. Glória ao Pai \emph{\&c.}
}\end{paracol}

\paragraph{Oração}
\begin{paracol}{2}\latim{
\rlettrine{P}{ræsta,} quǽsumus, omnípotens Deus: ut fámilia tua, quæ se, affligéndo carnem, ab aliméntis ábstinet: sectándo justítiam, a culpa jejúnet. Per Dóminum \emph{\&c.}
}\switchcolumn\portugues{
\slettrine{Ó}{} Deus omnipotente, Vos suplicamos, fazei que os vossos fiéis, que para mortificação da carne se privam dos alimentos, se abstenham também do pecado, praticando a justiça. Por nosso Senhor \emph{\&c.}
}\end{paracol}

\paragraphinfo{Epístola}{Dn. 9, 15-19}
\begin{paracol}{2}\latim{
Léctio Daniélis Prophétæ.
}\switchcolumn\portugues{
Lição do Profeta Daniel.
}\switchcolumn*\latim{
\rlettrine{I}{n} diébus illis: Orávit Dániel Dóminum, dicens: Dómine, Deus noster, qui eduxísti pópulum tuum de terra Ægýpti in manu forti, et fecísti tibi nomen secúndum diem hanc; peccávimus, iniquitátem fécimus, Dómine, in omnem justítiam tuam: avertátur, óbsecro, ira tua et furor tuus a civitáte tua Jerúsalem et monte sancto tuo. Propter peccáta enim nostra et iniquitátes patrum nostrórum. Jerúsalem et pópulus tuus in oppróbrium sunt ómnibus per circúitum nostrum. Nunc ergo exáudi, Deus noster, oratiónem servi tui et preces ejus: et osténde fáciem tuam super sanctuárium tuum, quod desértum est, propter temetípsum. Inclína, Deus meus, aurem tuam, et audi: áperi óculos tuos, et vide desolatiónem nostram et civitátem, super quam invocátum est nomen tuum: neque enim in justificatiónibus nostris prostérnimus preces ante fáciem tuam, sed in miseratiónibus tuis multis. Exáudi, Dómine, placáre, Dómine: atténde et fac: ne moréris propter temetípsum, Deus meus: quia nomen tuum invocátum est super civitátem et super pópulum tuum, Dómine, Deus noster.
}\switchcolumn\portugues{
\rlettrine{N}{aqueles} dias, Daniel dirigiu ao Senhor esta oração: «Senhor, nosso Deus, que tirastes o vosso povo da terra do Egipto com vossa mão omnipotente, pelo que adquiristes uma glória que permaneceu até hoje: nós pecámos, Senhor; cometemos iniquidades contra a vossa lei! Afastai, pois, Senhor, a vossa ira e indignação da vossa cidade de Jerusalém, do vosso monte sagrado; pois é por causa dos nossos pecados e das iniquidades de nossos pais que Jerusalém e o vosso povo vivem no opróbrio, diante de todas as nações que nos rodeiam. Então, agora, ó nosso Deus, ouvi a oração e as preces do vosso servo; mostrai a vossa face no santuário, que está abandonado; fazei isto pela vossa própria glória! Inclinai, meu Deus, os vossos ouvidos e escutai-nos. Abri os olhos e vede a nossa desolação e a desolação da cidade, que possui o vosso nome. Não é confiados na nossa justiça que depomos a vossos pés as nossas súplicas, mas sim com o pensamento na vossa profunda misericórdia. Ouvi-nos, Senhor! Deixai-Vos aplacar, Senhor! Atendei-nos, Senhor! Perdoai-nos, Senhor! Fazei o que Vos pedimos! Não tardeis, por causa da honra do vosso nome, ó meu Deus; pois esta cidade e este povo, que Vos pertencem, têm a honra de usar o vosso nome, ó Senhor, nosso Deus»!
}\end{paracol}

\paragraphinfo{Gradual}{Sl. 69, 6 \& 3}
\begin{paracol}{2}\latim{
\rlettrine{A}{djútor} meus et liberátor meus esto: Dómine, ne tardáveris. ℣. Confundántur et revereántur inimíci mei, qui quærunt ánimam meam.
}\switchcolumn\portugues{
\rlettrine{S}{ede} o meu auxiliar e o meu libertador! Senhor, não tardeis! ℣. Que sejam confundidos e envergonhados os meus inimigos, que procuram tirar-me a vida.
}\end{paracol}

\paragraphinfo{Trato}{Página \pageref{tratoquartacinzas}}

\paragraphinfo{Evangelho}{Jo. 8, 21-29}
\begin{paracol}{2}\latim{
\cruz Sequéntia sancti Evangélii secúndum Joánnem.
}\switchcolumn\portugues{
\cruz Continuação do santo Evangelho segundo S. João.
}\switchcolumn*\latim{
\blettrine{I}{n} illo témpore: Dixit Jesus turbis Judæórum: Ego vado, et quærétis me, et in peccáto vestro moriémini. Quo ego vado, vos non potéstis veníre. Dicébant ergo Judǽi: Numquid interfíciet semetípsum, quia dixit: Quo ego vado, vos non potéstis veníre? Et dicébat eis: Vos de deórsum estis, ego de supérnis sum. Vos de mundo hoc estis, ego non sum de hoc mundo. Dixi ergo vobis, quia moriémini in peccátis vestris: si enim non credidéritis, quia ego sum, moriémini in peccáto vestro. Dicébant ergo ei: Tu quis es? Dixit eis Jesus: Princípium, qui et loquor vobis. Multa habeo de vobis loqui et judicáre. Sed qui me misit, verax est: et ego quæ audívi ab eo, hæc loquor in mundo. Et non cognovérunt, quia Patrem ejus dicébat Deum. Dixit ergo eis Jesus: Cum exaltavéritis Fílium hóminis, tunc cognoscétis quia ego sum, et a meípso fácio nihil: sed, sicut dócuit me Pater, hæc loquor: et qui me misit, mecum est, et non relíquit me solum: quia ego, quæ plácita sunt ei, fácio semper.
}\switchcolumn\portugues{
\blettrine{N}{aquele} tempo, Jesus disse às turbas dos judeus: «Eu vou; procurar-me-eis e morrereis no vosso pecado. Aonde eu vou não podereis vós ir». Diziam, então, os judeus: «Porventura matar-se-á Ele a si mesmo? Pois diz: «Aonde eu vou não podereis vós ir?». Ele disse-lhes: «Vós sois debaixo e Eu sou do alto. Vós sois do mundo e Eu não sou deste mundo. Eu vos digo, pois: morrereis no vosso pecado porque, se não acreditardes quem Eu sou, morrereis no vosso pecado». Diziam-lhe, então: «Quem sois Vós?». Jesus respondeu-lhes: «Eu, que vos falo, sou o princípio. Muitas cousas tenho de que deva censurar-vos e condenar-vos; mas Aquele que me enviou é verdadeiro, e o que Eu digo ao mundo com Ele o aprendi». Porém, eles não compreenderam que Ele com estas palavras queria dizer que Deus era seu Pai. Disse-lhes, portanto, Jesus: «Quando levantardes da terra o Filho do homem, então conhecereis quem Eu sou; assim como que não faço nada de mim próprio e que o que digo foi meu Pai quem me ensinou. Aquele que me enviou está comigo, e não me deixa só, porquanto faço sempre o que Lhe agrada».
}\end{paracol}

\paragraphinfo{Ofertório}{Sl. 15, 7 \& 8}
\begin{paracol}{2}\latim{
\rlettrine{B}{enedícam} Dóminum, qui tríbuit mihi intelléctum: providébam Dóminum in conspéctu meo semper: quóniam a dextris est mihi, ne commóvear.
}\switchcolumn\portugues{
\rlettrine{B}{endirei} o Senhor, que me deu inteligência; tenho os meus olhos voltados continuamente para o Senhor; não vacilarei, porque Ele está à minha direita.
}\end{paracol}

\paragraph{Secreta}
\begin{paracol}{2}\latim{
\rlettrine{H}{æc} hóstia, Dómine, placatiónis et laudis, tua nos protectióne dignos effíciat. Per Dóminum \emph{\&c.}
}\switchcolumn\portugues{
\qlettrine{Q}{ue} esta hóstia de propiciação e de louvor, Senhor, nos torne dignos da vossa protecção. Por nosso Senhor \emph{\&c.}
}\end{paracol}

\paragraphinfo{Comúnio}{Sl. 8, 2}
\begin{paracol}{2}\latim{
\rlettrine{D}{ómine,} Dóminus noster, quam admirábile est nomen tuum in univérsa terra!
}\switchcolumn\portugues{
\rlettrine{S}{enhor,} nosso Deus, quão admirável é o vosso nome em toda a terra!
}\end{paracol}

\paragraph{Postcomúnio}
\begin{paracol}{2}\latim{
\rlettrine{H}{æc} nos commúnio, Dómine, purget a crímine: et cœléstis remédii fáciat esse consórtes. Per Dóminum \emph{\&c.}
}\switchcolumn\portugues{
\qlettrine{Q}{ue} esta comunhão, Senhor, nos purifique de nossos crimes e nos faça participantes dos remédios celestiais. Por nosso Senhor \emph{\&c.}
}\end{paracol}

\paragraph{Oração sobre o povo}
\begin{paracol}{2}\latim{
\begin{nscenter} Orémus. \end{nscenter}
}\switchcolumn\portugues{
\begin{nscenter} Oremos. \end{nscenter}
}\switchcolumn*\latim{
Humiliáte cápita vestra Deo.
}\switchcolumn\portugues{
Inclinai as vossas cabeças diante de Deus.
}\switchcolumn*\latim{
Adésto supplicatiónibus nostris, omnípotens Deus: et, quibus fidúciam sperándæ pietátis indúlges; consuétæ misericórdiæ tríbue benígnus efféctum. Per Dóminum \emph{\&c.}
}\switchcolumn\portugues{
Sede atento às nossas súplicas, ó Deus omnipotente; e permiti benignamente que aqueles que de Vós mereceram a doce confiança de lhes revelardes que podiam ter esperança na vossa bondade alcancem os efeitos da vossa habitual misericórdia. Por nosso Senhor \emph{\&c.}
}\end{paracol}

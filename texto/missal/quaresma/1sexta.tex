\subsectioninfo{Sexta-feira da l.ª Semana da Quaresma - Têmporas da Primavera}{Estação na Igreja dos Doze Apóstolos}

\paragraphinfo{Intróito}{Sl. 24, 17 \& 18}
\begin{paracol}{2}\latim{
\rlettrine{D}{e} necessitátibus meis éripe me, Dómine: vide humilitátem meam et labórem meum, et dimítte ómnia peccáta mea. \emph{Ps. ibid., 1-2} Ad te, Dómine, levávi ánimam meam: Deus meus, in te confído, non erubéscam.
℣. Gloria Patri \emph{\&c.}
}\switchcolumn\portugues{
\rlettrine{L}{ivrai-me,} Senhor, das minhas tribulações; vede a minha humilhação e as minhas penas e perdoai-me todos meus pecados. \emph{Sl. ibid., 1-2} A Vós, Senhor, elevo a minha alma: meu Deus, em Vós confio: não esperarei em vão.
℣. Glória ao Pai \emph{\&c.}
}\end{paracol}

\paragraph{Oração}
\begin{paracol}{2}\latim{
\rlettrine{E}{sto,} Dómine, propítius plebi tuæ: et, quam tibi facis esse devótam, benígno réfove miserátus auxílio. Per Dóminum \emph{\&c.}
}\switchcolumn\portugues{
\rlettrine{S}{ede} propício ao vosso povo, Senhor; e, visto que lhe concedestes a graça da devoção para convosco, auxiliai-o agora benignamente com vossa misericórdia. Por nosso Senhor \emph{\&c.}
}\end{paracol}

\paragraphinfo{Epístola}{Ez. 18, 20-28}
\begin{paracol}{2}\latim{
Léctio Ezechiélis Prophétæ.
}\switchcolumn\portugues{
Lição do Profeta Ezequiel.
}\switchcolumn*\latim{
\rlettrine{H}{æc} dicit Dóminus Deus: Anima, quae peccáverit, ipsa moriétur: fílius non portábit iniquitátem patris, et pater non portábit iniquitátem fílii: justítia justi super eum erit, et impíetas ímpii erit super eum. Si autem ímpius égerit pæniténtiam ab ómnibus peccátis suis, quæ operátus est, et custodíerit ómnia præcépta mea, et fécerit judícium et justítiam: vita vivet, et non moriétur. Omnium iniquitátum ejus, quas operátus est, non recordábor: in justítia sua, quam operátus est, vivet. Numquid voluntátis meæ est mors ímpii, dicit Dóminus Deus, et non ut convertátur a viis suis, et vivat? Si autem avértent se justus a justítia sua, et fécerit iniquitátem secúndum omnes abominatiónes, quas operári solet ímpius, numquid vivet? omnes justítiæ ejus, quas fécerat, non recordabúntur: in prævaricatióne, qua prævaricátus est, et in peccáto suo, quod peccávit, in ipsis moriétur. Et dixístis: Non est æqua via Dómini. Audíte ergo, domus Israël: Numquid via mea non est æqua, et non magis viæ vestræ pravæ sunt? Cum enim avértent se justus a justítia sua, et fecerit iniquitátem, moriétur in eis: in injustítia, quam operátus est, moriétur. Et cum avértent se ímpius ab impietáte sua, quam operátus est, et fécerit judícium et justítiam: ipse ánimam suam vivificábit. Consíderans enim, et avértens se ab ómnibus iniquitátibus suis, quas operátus est, vita vivet, et non moriétur, ait Dóminus omnípotens.
}\switchcolumn\portugues{
\rlettrine{I}{sto} diz o Senhor Deus: «A alma que pecar morrerá. O filho não levará consigo a maldade do pai, e o pai não ficará com a maldade do filho. A justiça do justo irá com ele, e a impiedade do iníquo recairá sobre si. Mas, se o iníquo fizer penitência de todos os pecados que tiver cometido; se observar os meus preceitos; se proceder segundo o direito e a justiça: permanecerá na vida e não morrerá. Não me recordarei mais das suas iniquidades; e, por causa da justiça, que praticou, viverá. Porventura quero a morte do iníquo? Diz o Senhor Deus. Porventura não quero antes que ele se afaste dos maus caminhos, e viva? Se o justo se afasta da justiça e comete as iniquidades e as abominações que o ímpio costuma praticar, porventura viverá? Todas as boas obras de justiça que praticara não serão recordadas, e, por causa da prevaricação em que caiu e do pecado, que praticou, morrerá. Entretanto, vós dizeis: «O caminho do Senhor não é direito». Ouvi, pois, ó casa de Israel: porventura não é direito o meu caminho e não são antes perversos os vossos caminhos? Porquanto, logo que o justo se afasta da justiça, cometa iniquidades e morra neste estado, será por ter praticado estas iniquidades que morrerá. Mas, logo que o ímpio se afasta da iniquidade, que praticava, e proceda segundo o direito e a justiça, ele próprio proporcionará a vida à sua alma. Se ele, pois, medita no seu estado e se afasta de todas as iniquidades que tinha cometido, certamente viverá na vida eterna e não morrerá: diz o Senhor omnipotente».
}\end{paracol}

\paragraphinfo{Gradual}{Sl. 85, 2 \& 6}
\begin{paracol}{2}\latim{
\rlettrine{S}{alvum} fac servum tuum. Deus meus, sperántem in te. ℣. Auribus pércipe, Dómine, oratiónem meam.
}\switchcolumn\portugues{
\rlettrine{S}{alvai,} ó meu Deus, o vosso servo, pois em Vós pôs a sua esperança. Escutai a minha oração, Senhor!
}\end{paracol}

\paragraphinfo{Trato}{Página \pageref{tratoquartacinzas}}

\paragraphinfo{Evangelho}{Jo. 5, 1-15}
\begin{paracol}{2}\latim{
\cruz Sequéntia sancti Evangélii secúndum Joánnem.
}\switchcolumn\portugues{
\cruz Continuação do santo Evangelho segundo S. João.
}\switchcolumn*\latim{
\blettrine{I}{n} illo témpore: Erat dies festus Judæórum, et ascéndit Jesus Jerosólymam. Est autem Jerosólymis Probática piscína, quæ cognominátur hebráice Bethsáida, quinque pórticus habens. In his jacébat multitúdo magna languéntium, cæcórum, claudórum, aridórum exspectántium aquæ motum. Angelus autem Dómini descendébat secúndum tempus in piscínam, et movebátur aqua. Et, qui prior descendísset in piscínam post motiónem aquæ, sanus fiébat, a quacúmque detinebátur infirmitáte. Erat autem quidam homo ibi, trigínta et octo annos habens in infirmitáte sua. Hunc cum vidísset Jesus jacéntem, et cognovisset, quia jam multum tempus habéret, dicit ei: Vis sanus fíeri? Respóndit ei lánguidus: Dómine, hóminem non hábeo, ut, cum turbáta fúerit aqua, mittat me in piscínam: dum vénio enim ego, álius ante me descéndit. Dicit ei Jesus: Surge, tolle grabátum tuum, et ámbula. Et statim sanus factus est homo ille: et sústulit grabátum suum, et ambulábat. Erat autem sábbatum in die illo. Dicébant ergo Judǽi illi, qui sanátus fúerat: Sábbatum est, non licet tibi tóllere grabátum tuum. Respóndit eis: Qui me sanum fecit, ille mihi dixit: Tolle grabátum tuum, et ámbula. Interrogavérunt ergo eum: Quis est ille homo, qui dixit tibi: Tolle grabátum tuum et ámbula? Is autem, qui sanus fúerat efféctus, nesciébat, quis esset. Jesus enim declinávit a turba constitúta in loco. Póstea invénit eum Jesus in templo, et dixit illi: Ecce, sanus factus es: jam noli peccáre, ne detérius tibi áliquid contíngat. Abiit ille homo, et nuntiávit Judǽis, quia Jesus esset, qui fecit eum sanum.
}\switchcolumn\portugues{
\blettrine{N}{aquele} tempo, sendo o dia da festa dos judeus, Jesus subiu até Jerusalém. Ora há perto de Jerusalém uma piscina que se chama em hebreu Betsaida e tem cinco alpendres, debaixo dos quais costumava estar deitada grande multidão de enfermos: cegos, coxos e paralíticos, à espera de que a água se movesse; pois um Anjo do Senhor descia de tempos a tempos à piscina, revolvia a água e o primeiro que descia à piscina, depois do movimento da água, ficava curado de qualquer enfermidade que tivesse. Havia um certo homem doente que estava ali havia trinta e oito anos. Vendo-o Jesus deitado e sabendo que estava doente havia tanto tempo, disse-lhe: «Queres ser curado?». Respondeu-lhe o doente: «Senhor, não tenho ninguém que me meta na piscina, quando a água é movida; pois, enquanto vou, um outro chega primeiro». Jesus disse-lhe: «Levanta-te, toma tua cama e anda». E naquele instante o homem ficou curado; e, tomando a cama, andava. Era, porém, sábado aquele dia. Disseram, pois, os judeus ao que tinha sido curado: «É sábado, não te é lícito levar a cama». Ele respondeu-lhes: «Aquele que me curou disse-me: «Toma a tua cama e anda». Perguntaram-lhe, pois: «Quem é esse homem que te disse toma a tua cama e anda?». Mas o que tinha sido curado não sabia quem era, pois Jesus retirara-se da multidão que estava ali. Mais tarde Jesus encontrou-o no templo e disse-lhe: «Agora, que já estás são, não tornes a pecar, para que te não suceda cousa pior». Partiu este homem e foi anunciar aos judeus que tinha sido Jesus quem o curara.
}\end{paracol}

\paragraphinfo{Ofertório}{Sl. 102, 2 \& 5}
\begin{paracol}{2}\latim{
\rlettrine{B}{énedic,} anima mea, Dómino, et noli oblivísci omnes retributiónes ejus: et renovábitur, sicut áquilæ, juvéntus tua.
}\switchcolumn\portugues{
\rlettrine{B}{endizei} o Senhor, ó minha alma, e não esqueçais nunca os seus benefícios; e a vossa juventude se renovará, como a da águia.
}\end{paracol}

\paragraph{Secreta}
\begin{paracol}{2}\latim{
\rlettrine{S}{úscipe,} quǽsumus, Dómine, múnera nostris obláta servítiis: et tua propítius dona sanctífica. Per Dóminum \emph{\&c.}
}\switchcolumn\portugues{
\rlettrine{A}{ceitai,} Senhor, Vos suplicamos, as oblatas que a nossa escravidão Vos apresenta, e, propício, dignai-Vos santificar estes dons que de Vós recebemos. Por nosso Senhor \emph{\&c.}
}\end{paracol}

\paragraphinfo{Comúnio}{Sl. 6, 11}
\begin{paracol}{2}\latim{
\rlettrine{E}{rubéscant} et conturbéntur omnes inimíci mei: avertántur retrórsum, et erubéscant valde velóciter.
}\switchcolumn\portugues{
\rlettrine{E}{nvergonhem-se} e perturbem-se todos meus inimigos; apressem-se em fugir, cheios de vergonha.
}\end{paracol}

\paragraph{Postcomúnio}
\begin{paracol}{2}\latim{
\rlettrine{P}{er} hujus, Dómine, operatiónem mystérii, et vítia nostra purgéntur, et justa desidéria compleántur. Per Dóminum nostrum \emph{\&c.}
}\switchcolumn\portugues{
\qlettrine{Q}{ue} por efeito deste mystério, Senhor, os nossos vícios desapareçam e sejam realizados os nossos justos desejos. Por nosso Senhor \emph{\&c.}
}\end{paracol}

\paragraph{Oração sobre o povo}
\begin{paracol}{2}\latim{
\begin{nscenter} Orémus. \end{nscenter}
}\switchcolumn\portugues{
\begin{nscenter} Oremos. \end{nscenter}
}\switchcolumn*\latim{
Humiliáte cápita vestra Deo.
}\switchcolumn\portugues{
Inclinai as vossas cabeças diante de Deus.
}\switchcolumn*\latim{
Exáudi nos, miséricors Deus: et méntibus nostris grátiæ tuæ lumen osténde. Per Dóminum \emph{\&c.}
}\switchcolumn\portugues{
Ó Deus de misericórdia, ouvi-nos; e mostrai às nossas almas a luz da vossa graça. Por nosso Senhor \emph{\&c.}
}\end{paracol}

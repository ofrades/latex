\subsectioninfo{Quinta-Feira Santa}{Estação em S. João de Latrão}

\paragraphinfo{Intróito}{Gl. 6, 14}
\begin{paracol}{2}\latim{
\rlettrine{N}{os} autem gloriári opórtet in Cruce Dómini nostri Jesu Christi: in quo est salus, vita et resurréctio nostra: per quem salváti et liberáti sumus. \emph{Ps. 66, 2} Deus misereátur nostri, et benedícat nobis: illúminet vultum suum super nos, et misereátur nostri.
}\switchcolumn\portugues{
\rlettrine{N}{ós,} porém, devemos ufanar-nos na Cruz de N. S. Jesus Cristo, pois Ele é a nossa salvação, vida e ressurreição: e por Ele fomos salvos e livres. \emph{Sl. 66, 2} Que Deus tenha misericórdia de nós e nos abençoe; e se compadeça de nós!
}\end{paracol}

\paragraph{Oração}
\begin{paracol}{2}\latim{
\rlettrine{D}{eus,} a quo et Judas reatus sui pœnam, et confessiónis suæ latro prǽmium sumpsit, concéde nobis tuæ propitiatiónis efféctum: ut, sicut in passióne sua Jesus Christus, Dóminus noster, diversa utrísque íntulit stipéndia meritórum; ita nobis, abláto vetustátis erróre, resurrectiónis suæ grátiam largiátur: Qui tecum \emph{\&c.}
}\switchcolumn\portugues{
\slettrine{Ó}{} Deus, de quem Judas recebeu o castigo da sua perfídia e o ladrão a recompensa da sua confissão, concedei-nos o efeito da vossa misericórdia, a fim de que, assim como N. S. Jesus Cristo durante a sua Paixão tratou a um e ao outro segundo os seus méritos, assim também, havendo desaparecido a nossa malícia do «homem velho», nos tornemos participantes da sua ressurreição. Ele, que, sendo Deus \emph{\&c.}
}\end{paracol}

\paragraphinfo{Epístola}{1 Cor. 11, 20-32}
\begin{paracol}{2}\latim{
Léctio Epístolæ beáti Pauli Apóstoli ad Corínthios.
}\switchcolumn\portugues{
Lição da Ep.ª do B. Ap.º Paulo aos Coríntios.
}\switchcolumn*\latim{
\rlettrine{F}{ratres:} Conveniéntibus vobis m unum, jam non est Domínicam cœnam manducáre. Unusquísque enim suam cenam præsúmit ad manducándum. Et alius quidem ésurit: álius autem ébrius est. Numquid domos non habétis ad manducándum et bibéndum? aut ecclésiam Dei contémnitis, et confúnditis eos, qui non habent? Quid dicam vobis? Laudo vos? In hoc non laudo. Ego enim accépi a Dómino quod et trádidi vobis, quóniam Dóminus Jesus, in qua nocte tradebátur, accépit panem, et grátias agens tregit, et dixit: Accípite, et manducáte: hoc est corpus meum, quod pro vobis tradétur: hoc fácite in meam commemoratiónem. Simíliter et cálicem, postquam cœnávit, dicens: Hic calix novum Testaméntum est in meo sánguine: hoc fácite, quotiescúmque bibétis, in meam commemoratiónem. Quotiescúmque enim manducábitis panem hunc et cálicem bibétis: mortem Dómini annuntiábitis, donec véniat. Itaque quicúmque manducáverit panem hunc vel bíberit cálicem Dómini indígne, reus erit córporis et sánguinis Dómini. Probet autem seípsum homo: et sic de pane illo edat et de cálice bibat. Qui enim mandúcat et bibit indígne, judícium sibi mandúcat et bibit: non dijúdicans corpus Dómini. Ideo inter vos multi infirmi et imbecílles, et dórmiunt multi. Quod si nosmetípsos dijudicarémus, non útique judicarémur. Dum judicámur autem, a Dómino corrípimur, ut non cum hoc mundo damnémur.
}\switchcolumn\portugues{
\rlettrine{M}{eus} irmãos: Quando vos reunis em assembleia, já não é para comer a Ceia do Senhor que o fazeis; pois cada um de vós se antecipa em comer a sua própria ceia à parte, de modo que um fica com fome e o outro fica ébrio. Porventura não tendes as vossas casas para aí comer e beber? Ou, desprezando a assembleia de Deus, quereis humilhar aqueles que nada possuem? Que vos direi? Louvar-vos? Decerto que não posso louvar-vos por isto; pois foi o Senhor que me ensinou o que a este respeito vos transmiti, isto é: que o Senhor Jesus, na noite em que foi traído, tomou o pão e, depois de haver dado graças, partiu-o e disse: «Aceitai e comei. Isto é o meu corpo, que será entregue por vós. Fazei isto em minha memória». Do mesmo modo, depois de haver ceado, tomou o cálice e disse: «Este cálice é o Novo Testamento no meu sangue. Fazei isto mesmo, em minha memória, sempre que o beberdes». Todas as vezes que comerdes este pão e beberdes este cálice anunciareis a morte do Senhor, até que Ele venha. É por isso que todo aquele que indignamente comer este pão ou beber o cálice do Senhor será réu do Corpo e do Sangue do Senhor. Examine-se, pois, o homem a si mesmo: e, assim, coma daquele pão e beba do cálice; pois o que comer e beber indignamente, não discernindo o Corpo do Senhor, come e bebe a sua condenação. É por isto que no meio de vós há muitos enfermos e fracos e muitos outros dormem. Portanto, se nos julgarmos a nós mesmos, não seremos condenados; pois, enquanto nos julgamos, o Senhor nos corrigirá, a fim de não sermos condenados com este mundo.
}\end{paracol}

\paragraphinfo{Gradual}{Fl. 2, 8-9}
\begin{paracol}{2}\latim{
\rlettrine{C}{hristus} factus est pro nobis obœdiens usque ad mortem, mortem autem crucis ℣. Propter quod et Deus exaltávit illum: et dedit illi nomen, quod est super omne nomen.
}\switchcolumn\portugues{
\rlettrine{C}{risto} tornou-se obediente por nós até à morte, e morte de cruz. ℣. Pelo que Deus O exaltou e lhe deu um nome que é superior a todos os nomes.
}\end{paracol}

\paragraphinfo{Evangelho}{Jo. 13, 1-15}
\begin{paracol}{2}\latim{
\cruz Sequéntia sancti Evangélii secúndum Joánnem.
}\switchcolumn\portugues{
\cruz Continuação do santo Evangelho segundo S. João.
}\switchcolumn*\latim{
\blettrine{A}{nte} diem festum Paschae, sciens Jesus, quia venit hora ejus, ut tránseat ex hoc mundo ad Patrem: cum dilexísset suos, qui erant in mundo, in finem diléxit eos. Et cena facta, cum diábolus jam misísset in cor, ut tráderet eum Judas Simónis Iscariótæ: sciens, quia ómnia dedit ei Pater in manus, et quia a Deo exivit, et ad Deum vadit: surgit a cena et ponit vestiménta sua: et cum accepísset línteum, præcínxit se. Deinde mittit aquam in pelvim, et cœpit laváre pedes discipulórum, et extérgere línteo, quo erat præcínctus. Venit ergo ad Simónem Petrum. Et dicit ei Petrus: Dómine, tu mihi lavas pedes? Respóndit Jesus et dixit ei: Quod ego fácio, tu nescis modo, scies autem póstea. Dicit ei Petrus: Non lavábis mihi pedes in ætérnum.
}\switchcolumn\portugues{
\blettrine{A}{ntes} do dia da festa da Páscoa, sabendo Jesus que chegara a sua hora em que devia passar deste mundo para seu Pai, havendo amado os seus, que estavam no mundo, amou-os até ao fim. E, depois da ceia, quando já o demónio havia posto o desígnio de O atraiçoar no coração de Judas Iscariotes, filho de Simão, sabendo Jesus que o Pai havia deixado todas as cousas nas suas mãos e que, havendo Ele saído de Deus, para Deus voltava, levantou-se da mesa, tirou o seu manto e cingiu-se com uma toalha. Em seguida, deitou água em uma bacia, começou a lavar os pés dos discípulos e enxugou-lhos com a toalha com que se cingira. Chegou, enfim, a Simão-Pedro, o qual lhe disse: «Senhor, quereis lavar-me os pés?». Jesus respondeu: «O que Eu faço o não compreendes agora; mais tarde compreendê-lo-ás». Pedro disse-Lhe: «Não; jamais me lavareis os pés!».
}\switchcolumn*\latim{
Respóndit ei Jesus: Si non lávero te, non habébis partem mecum. Dicit ei Simon Petrus: Dómine, non tantum pedes meos, sed et manus et caput. Dicit ei Jesus: Qui lotus est, non índiget nisi ut pedes lavet, sed est mundus totus. Et vos mundi estis, sed non omnes. Sciébat enim, quisnam esset, qui tráderet eum: proptérea dixit: Non estis mundi omnes. Postquam ergo lavit pedes eórum et accépit vestiménta sua: cum recubuísset íterum, dixit eis: Scitis, quid fécerim vobis? Vos vocátis me Magíster et Dómine: et bene dícitis: sum étenim. Si ergo ego lavi pedes vestros, Dóminus et Magíster: et vos debétis alter altérius laváre pedes. Exémplum enim dedi vobis, ut, quemádmodum ego feci vobis, ita et vos faciátis.
}\switchcolumn\portugues{
Jesus respondeu-lhe: «Se te não lavar os pés, não terás parte comigo». Simão-Pedro disse, então: «Senhor, não só os pés, mas ainda as mãos e a cabeça!». E Jesus disse-lhe: «Quem está lavado só precisa de lavar os pés; pois está todo limpo. Vós também estais limpos, mas não todos». Pois Ele sabia quem havia de entregá-l’O; por isso disse: «Não estais todos limpos». Depois de lhes lavar os pés, tomou os vestidos, assentou-se à mesa e disse: «Sabeis o que vos fiz? Chamais-me Senhor e Mestre e dizeis bem, porque, na verdade, o sou. Se Eu, pois, sendo vosso Mestre e Senhor, vos lavei os pés, também deveis lavá-los uns aos outros. Dei-Vos o exemplo, para que, assim como Eu vos fiz, assim façais também».
}\end{paracol}

\paragraphinfo{Ofertório}{Sl. 117, 16 \& 17}
\begin{paracol}{2}\latim{
\rlettrine{D}{éxtera} Dómini fecit virtútem, déxtera Dómini exaltávit me: non móriar, sed vivam, et narrábo ópera Dómini.
}\switchcolumn\portugues{
\rlettrine{A}{} dextra do Senhor mostrou o seu poder; a dextra do Senhor exaltou-me! Não morrerei, mas viverei e publicarei as maravilhas do Senhor.
}\end{paracol}

\paragraph{Secreta}
\begin{paracol}{2}\latim{
\rlettrine{I}{pse} tibi, quǽsumus, Dómine sancte, Pater omnípotens, ætérne Deus, sacrifícium nostrum reddat accéptum, qui discípulis suis in sui commemoratiónem hoc fíeri hodiérna traditióne monstrávit, Jesus Christus, Fílius tuus, Dóminus noster: Qui tecum vivit et regnat \emph{\&c.}
}\switchcolumn\portugues{
\rlettrine{S}{enhor} santo, Pai omnipotente e Deus eterno, permiti que este nosso sacrifício Vos seja agradável por Jesus Cristo, vosso Filho, que, instituindo-o neste dia, prescreveu aos discípulos que o celebrassem em sua memória. Ele, que, sendo Deus, convosco vive e \emph{\&c.}
}\end{paracol}

\paragraphinfo{Comúnio}{Jo. 13, 12, 13 \& 15}
\begin{paracol}{2}\latim{
\rlettrine{D}{óminus} Jesus, postquam cœnávit cum discípulis suis, lavit pedes eórum, et ait illis: Scitis, quid fécerim vobis ego, Dóminus et Magíster? Exemplum dedi vobis, ut et vos ita faciátis.
}\switchcolumn\portugues{
\rlettrine{O}{} Senhor Jesus, depois de haver ceado com seus discípulos, lavou-lhes os pés e disse-lhes: «Sabeis o que vos fiz, sendo vosso Senhor e Mestre? Dei-vos o exemplo, para que façais também o mesmo».
}\end{paracol}

\paragraph{Postcomúnio}
\begin{paracol}{2}\latim{
\rlettrine{R}{efécti} vitálibus aliméntis, quǽsumus, Dómine, Deus
noster: ut, quod témpore nostræ mortalitátis exséquimur, immortalitátis tuæ múnere consequámur. Per Dóminum nostrum \emph{\&c.}
}\switchcolumn\portugues{
\slettrine{Ó}{} Senhor, nosso Deus, havendo nós sido saciados com este alimento de vida, concedei-nos a graça de, com vosso socorro, alcançarmos no seio da imortalidade o que procurámos durante a vida mortal. Por nosso Senhor \emph{\&c.}
}\end{paracol}

\subsubsection{Procissão do SS. Sacramento}

\paragraph{Pange Lingua}
\gregorioscore{scores/pangelingua1verse}

\begin{nscenter}
Canta, ó minha língua, o mistério do Corpo e do Sangue precioso que foi derramado para resgate do mundo, fruto dum seio fecundo, o Rei dos povos.
\end{nscenter}

\begin{paracol}{2}\latim{
\rlettrine{N}{obis} datus, nobis natus
Ex intácta Vírgine,
Et in mundo conversátus,
Sparso verbi sémine,
Sui moras incolátus
Miro clausit órdine.
}\switchcolumn\portugues{
\rlettrine{F}{oi-nos} dado; para nós nasceu da Virgem Imaculada; viveu no mundo, e, depois de haver espalhado a semente da palavra, terminou a sua passagem neste mundo com uma admirável instituição.
}\switchcolumn*\latim{
In suprémæ nocte coenæ
Recúmbens cum frátribus
Observáta lege plene
Cibis in legálibus,
Cibum turbæ duodénæ
Se dat suis mánibus.
}\switchcolumn\portugues{
Na noite da última ceia, estando à mesa com seus irmãos depois de haver observado os ritos legais, Ele próprio se deu com suas mãos em alimento aos Doze.
}\switchcolumn*\latim{
Verbum caro, panem verum
Verbo carnem éfficit:
Fitque sanguis Christi merum,
Et si sensus déficit,
Ad firmándum cor sincérum
Sola fides súfficit.
}\switchcolumn\portugues{
O Verbo feito carne mudou pela sua palavra um pão verdadeiro na própria Carne, e o vinho no Sangue de Cristo; e se a razão desfalece, não podendo compreender isto, a fé basta para corroborar esta crença nos corações sinceros.
}\end{paracol}

\gregorioscore{scores/tantum}

\begin{nscenter}
Adoremos, pois, prostrados este tão grande Sacramento: cedam os ritos antigos o lugar ao novo mistério e que a fé supra a fraqueza dos nossos sentidos.
\end{nscenter}

\gregorioscore{scores/genitori}

\begin{nscenter}
Glória, honra, louvor, poder, acção de graças e bênçãos sejam dadas ao Pai e ao Filho: e dêem-se iguais louvores ao que procede de um e do outro. Amen.
\end{nscenter}


\subsubsection{Desnudação dos Altares}

\paragraphinfo{Antífona}{Sl. 21, 19}
\begin{paracol}{2}\latim{
\rlettrine{D}{iviserunt} sibi vestimenta mea: et super vestem meam misérunt sortem.
}\switchcolumn\portugues{
\rlettrine{D}{ividiram} entre si os meus vestidos e sobre a minha túnica jogaram sortes.
}\end{paracol}

\paragraph{Salmo 21}
\begin{paracol}{2}\latim{
\rlettrine{D}{eus,} Deus meus, réspice in me: quare me dereliquísti? * longe a salúte mea verba delictórum meórum.
}\switchcolumn\portugues{
\rlettrine{D}{eus,} ó meu Deus, olhai para mim, porque me abandonastes? * Os clamores dos meus pecados afastam de mim a salvação.
}\switchcolumn*\latim{
Deus meus, clamábo per diem, et non exáudies: * et nocte, et non ad insipiéntiam mihi.
}\switchcolumn\portugues{
Meu Deus, clamarei durante o dia e me não ouvireis: * clamarei de noite e não por minha culpa.
}\switchcolumn*\latim{
Tu autem in sancto hábitas, * laus Israël.
}\switchcolumn\portugues{
Mas Vós morais no lugar santo, * ó glória de Israel.
}\switchcolumn*\latim{
In te speravérunt patres nostri: * speravérunt, et liberásti eos.
}\switchcolumn\portugues{
Em Vós esperaram nossos pais: * esperaram e os libertastes.
}\switchcolumn*\latim{
Ad te clamavérunt, et salvi facti sunt: * in te speravérunt, et non sunt confúsi.
}\switchcolumn\portugues{
A Vós clamaram e foram salvos: * em Vós esperaram e não foram confundidos.
}\switchcolumn*\latim{
Ego autem sum vermis, et non homo: * oppróbrium hóminum, et abjéctio plebis.
}\switchcolumn\portugues{
Eu, porém, sou um verme e não um homem: * opróbio dos homens e abjecção da plebe.
}\switchcolumn*\latim{
Omnes vidéntes me, derisérunt me: * locúti sunt lábiis, et movérunt caput.
}\switchcolumn\portugues{
Todos os que me viram escarneceram de mim: * falaram com os lábios e menearam a cabeça.
}\switchcolumn*\latim{
Sperávit in Dómino, erípiat eum: * salvum fáciat eum, quóniam vult eum.
}\switchcolumn\portugues{
Esperou no Senhor, livre-o: * salve-o, se é que o ama.
}\switchcolumn*\latim{
Quóniam tu es, qui extraxísti me de ventre: * spes mea ab ubéribus matris meæ. In te projéctus sum ex útero:
}\switchcolumn\portugues{
Pois Vós sois quem do ventre me tirou: * minha esperança desde o seio de minha mãe. Fui do útero lançado para Vós:
}\switchcolumn*\latim{
De ventre matris meæ Deus meus es tu, * ne discésseris a me:
}\switchcolumn\portugues{
Vós sois o meu Deus desde o ventre materno, * de mim Vos não retireis:
}\switchcolumn*\latim{
Quóniam tribulátio próxima est: * quóniam non est qui ádjuvet.
}\switchcolumn\portugues{
Porque a tribulação está próxima: * porque não há quem me ajude.
}\switchcolumn*\latim{
Circumdedérunt me vítuli multi: * tauri pingues obsedérunt me.
}\switchcolumn\portugues{
Um grande número de vitelos me cercara: * vi-me sitiado de gordos touros.
}\switchcolumn*\latim{
Aperuérunt super me os suum, * sicut leo rápiens et rúgiens.
}\switchcolumn\portugues{
Abriram sobre mim sua boca, * como um leão arrebatador e que ruge.
}\switchcolumn*\latim{
Sicut aqua effúsus sum: * et dispérsa sunt ómnia ossa mea.
}\switchcolumn\portugues{
Derramei-me como água: * e todos meus ossos se desconjuntaram.
}\switchcolumn*\latim{
Factum est cor meum tamquam cera liquéscens * in médio ventris mei.
}\switchcolumn\portugues{
Meu coração tornou-se como cera derretida * no meio das minhas entranhas.
}\switchcolumn*\latim{
Aruit tamquam testa virtus mea, et lingua mea adhǽsit fáucibus meis: * et in púlverem mortis deduxísti me.
}\switchcolumn\portugues{
Meu vigor secou-se como barro queimado e minha língua pegou-se ao paladar: * e conduzistes-me até ao pó da sepultura.
}\switchcolumn*\latim{
Quóniam circumdedérunt me canes multi: * concílium malignántium obsédit me.
}\switchcolumn\portugues{
Porquanto me rodearam muitos cães raivosos: * uma turba de malignos me assaltou.
}\switchcolumn*\latim{
Fodérunt manus meas et pedes meos: * dinumeravérunt ómnia ossa mea.
}\switchcolumn\portugues{
Traspassaram as minhas mãos e os meus pés: * contaram todos meus ossos.
}\switchcolumn*\latim{
Ipsi vero consideravérunt et inspexérunt me: * divisérunt sibi vestiménta mea, et super vestem meam misérunt sortem.
}\switchcolumn\portugues{
Estiveram-me veramente considerando e olhando: * repartiram entre si as minhas vestes e lançaram sortes sobre a minha túnica.
}\switchcolumn*\latim{
Tu autem, Dómine, ne elongáveris auxílium tuum a me: * ad defensiónem meam cónspice.
}\switchcolumn\portugues{
Mas Vós, ó Senhor, não afasteis de mim o vosso auxílio: * atendei à minha defesa.
}\switchcolumn*\latim{
Erue a frámea, Deus, ánimam meam: * et de manu canis únicam meam:
}\switchcolumn\portugues{
Livrai a minha alma da espada, ó Deus: * e minha única das garras dos cães:
}\switchcolumn*\latim{
Salva me ex ore leónis: * et a córnibus unicórnium humilitátem meam.
}\switchcolumn\portugues{
Salvai-me da boca do leão: * e a minha humildade das hastes dos unicórnios.
}\switchcolumn*\latim{
Narrábo nomen tuum frátribus meis: * in médio ecclésiæ laudábo te.
}\switchcolumn\portugues{
Narrarei o vosso nome aos meus irmãos: * no meio da igreja Vos louvarei.
}\switchcolumn*\latim{
Qui timétis Dóminum, laudáte eum: * univérsum semen Jacob, glorificáte eum.
}\switchcolumn\portugues{
Vós que temeis o Senhor, louvai-O: * vós todos, descendência de Jacób, glorificai-O.
}\switchcolumn*\latim{
Tímeat eum omne semen Israël: * quóniam non sprevit, neque despéxit deprecatiónem páuperis:
}\switchcolumn\portugues{
Tema-O toda a posteridade de Israel: * porque Ele não desprezou nem desatendeu a súplica do pobre:
}\switchcolumn*\latim{
Nec avértit fáciem suam a me: * et cum clamárem ad eum, exaudívit me.
}\switchcolumn\portugues{
Nem escondeu de mim a sua face: * mas me ouviu quando O chamava.
}\switchcolumn*\latim{
Apud te laus mea in ecclésia magna: * vota mea reddam in conspéctu timéntium eum.
}\switchcolumn\portugues{
A Vós dirigir-se-á o meu louvor numa grande igreja: * cumprirei os meus votos em presença dos que O temem.
}\switchcolumn*\latim{
Edent páuperes, et saturabúntur: et laudábunt Dóminum qui requírunt eum: * vivent corda eórum in sǽculum sǽculi.
}\switchcolumn\portugues{
Os pobres comerão e serão saciados: e os que buscam o Senhor louvá-l’O-ão: * os seus corações viverão pelos séculos dos séculos.
}\switchcolumn*\latim{
Reminiscéntur et converténtur ad Dóminum * univérsi fines terræ:
}\switchcolumn\portugues{
Lembrar-se-ão e converter-se-ão ao Senhor * todos os limites da terra:
}\switchcolumn*\latim{
Et adorábunt in conspéctu ejus * univérsæ famíliæ géntium.
}\switchcolumn\portugues{
E adorá-l'O-ão na sua presença * todas as famílias das gentes.
}\switchcolumn*\latim{
Quóniam Dómini est regnum: * et ipse dominábitur géntium.
}\switchcolumn\portugues{
Porque o reino pertence ao Senhor: * e Ele reinará sobre as gentes.
}\switchcolumn*\latim{
Manducavérunt et adoravérunt omnes pingues terræ: * in conspéctu ejus cadent omnes qui descéndunt in terram.
}\switchcolumn\portugues{
Comeram e adoraram todos os ricos da terra: * diante d’Ele se prostraram todos os mortais.
}\switchcolumn*\latim{
Et ánima mea illi vivet: * et semen meum sérviet ipsi.
}\switchcolumn\portugues{
E a minha alma viverá para Ele: * e a minha descendência servi-l'O-á.
}\switchcolumn*\latim{
Annuntiábitur Dómino generátio ventúra: * et annuntiábunt cæli justítiam ejus pópulo qui nascétur, quem fecit Dóminus.
}\switchcolumn\portugues{
A geração vindoura será anunciada ao Senhor: * e o que fez o Senhor, os céus anunciarão a sua justiça ao povo que há-de nascer.
}\end{paracol}


\emph{Depois do Salmo 21 repetir Antífona anterior.}

\subsubsection{Lava-Pés}

\emph{Evangelho igual ao anterior.}

\paragraphinfo{Antífona}{Jo. 13, 34}
\begin{paracol}{2}\latim{
\rlettrine{M}{andátum} novum do vobis: ut diligátis ínvicem, sicut diléxi vos, dicit Dóminus. \emph{Ps. 118, 1} Beáti immaculáti in via: qui ámbulant in lege Dómini.
}\switchcolumn\portugues{
\rlettrine{D}{ou-vos} um novo mandamento: «Amai-vos uns aos outros, como vos amei», diz o Senhor. \emph{Sl. 118, 1} Bem-aventurados os que são puros na sua vida: e que seguem a lei do Senhor.
}\end{paracol}

\paragraphinfo{Antífona}{Jo. 13, 4, 5 \& 15}
\begin{paracol}{2}\latim{
\rlettrine{P}{ostquam} surréxit Dóminus a cœna, misit aquam in pelvim, et cœpit laváre pedes discipulórum suórum: hoc exémplum réliquit eis. \emph{Ps. 47, 2} Magnus Dóminus, et laudábilis nimis: in civitáte Dei nostri, in monte sancto ejus. Postquam surréxit Dóminus.
}\switchcolumn\portugues{
\rlettrine{D}{epois} que o Senhor se levantou da ceia, deitou água em uma bacia e começou a lavar os pés aos discípulos, deixando este exemplo. \emph{Sl. 47, 2} O Senhor é grande e digno de todo o louvor na cidade de nosso Deus, na sua montanha sagrada.
}\end{paracol}

\paragraphinfo{Antífona}{Jo. 13, 12, 13 \& 15}
\begin{paracol}{2}\latim{
\rlettrine{D}{óminus} Jesus, postquam cœnávit cum discípulis suis, lavit pedes eórum, et ait illis: Scitis, quid fécerim vobis ego, Dóminus et Magíster? Exémplum dedi vobis, ut et vos ita faciátis. \emph{Ps. 84, 2} Benedixísti, Dómine, terram tuam: avertísti captivitátem Jacob.
}\switchcolumn\portugues{
\rlettrine{O}{} Senhor Jesus, depois de haver ceado com os discípulos, lavou-lhes os pés e disse-lhes: «Sabeis o que acabo de vos fazer, posto que seja vosso Senhor e Mestre? Dei-vos o exemplo, a fim de que façais o que acabo de fazer». \emph{Sl. 84, 2} Abençoastes, Senhor, a vossa terra; livrastes Jacob do cativeiro.
}\end{paracol}

\paragraphinfo{Antífona}{ Jo. 13, 6-7 \& 8.}
\begin{paracol}{2}\latim{
\rlettrine{D}{ómine,} tu mihi lavas pedes? Respóndit Jesus et dixit ei: Si non lávero tibi pedes, non habébis partem mecum. ℣. Venit ergo ad Simónem Petrum, et dixit ei Petrus.
}\switchcolumn\portugues{
\rlettrine{S}{enhor,} quereis lavar-me os pés? Jesus respondeu-lhe e disse: «Se te não lavar os pés, não terás parte comigo». ℣. Porém, quando Jesus chegou junto de Simão-Pedro, este disse-Lhe:
}\switchcolumn\latim{
Dómine, tu mihi lavas pedes? Respóndit Jesus et dixit ei: Si non lávero tibi pedes, non habébis partem mecum. ℣. Quod ego fácio, tu nescis modo: scies autem póstea.
}\switchcolumn\portugues{
Senhor, quereis lavar-me os pés? Jesus respondeu-lhe e disse: «Se te não lavar os pés, não terás parte comigo». ℣. «O que faço presentemente tu o ignoras; mas sabê-lo-ás depois».
}\switchcolumn\latim{
Dómine, tu mihi lavas pedes? Respóndit Jesus et dixit ei: Si non lávero tibi pedes, non habébis partem mecum.
}\switchcolumn\portugues{
Senhor, quereis lavar-me os pés? Jesus respondeu-lhe e disse: «Se te não lavar os pés, não terás parte comigo».
}\end{paracol}

\paragraph{Antífona}
\begin{paracol}{2}\latim{
\rlettrine{S}{i} ego, Dóminus et Magíster vester, lavi vobis pedes: quanto magis debétis alter altérius laváre pedes? \emph{Ps. 48, 2} Audíte hæc, omnes gentes: áuribus percípite, qui habitátis orbem.
}\switchcolumn\portugues{
\rlettrine{S}{e} Eu, vosso Senhor e Mestre, vos lavei os pés, quanto mais deveis lavar os pés uns aos outros. \emph{Sl. 48, 2} Ó povos, escutai todos esta palavra. Ouvi-a, ó habitantes da terra.
}\end{paracol}

\paragraphinfo{Antífona}{Jo. 13, 35}
\begin{paracol}{2}\latim{
\rlettrine{I}{n} hoc cognóscent omnes, quia discípuli mei estis, si dilectiónem habuéritis ad ínvicem. ℣. Dixit Jesus discípulis suis.
}\switchcolumn\portugues{
\rlettrine{T}{odos} conhecerão que sois meus discípulos, se vos amardes uns aos outros. ℣. Disse Jesus a seus discípulos.
}\end{paracol}

\paragraphinfo{Antífona}{1 Cor. 13, 13}
\begin{paracol}{2}\latim{
\rlettrine{M}{áneant} in vobis fides, spes, cáritas, tria hæc: major autem horum est cáritas. ℣. Nunc autem manent fides, spes, cáritas, tria hæc: major horum est cáritas.
}\switchcolumn\portugues{
\qlettrine{Q}{ue} a fé, a esperança e a caridade permaneçam em vós; mas a caridade é a maior destas três virtudes. ℣. Presentemente existem três virtudes, mas a maior das três é a caridade.
}\end{paracol}

\paragraph{Antífona}
\begin{paracol}{2}\latim{
\rlettrine{B}{enedícta} sit sancta Trínitas atque indivísa Unitas: confitébimur ei, quia fecit nobíscum misericórdiam suam. ℣. Benedicámus Patrem, et Fílium, cum Sancto Spíritu. \emph{Ps. 83, 23} Quam dilécta tabernácula tua, Dómine virtútum! concupíscit, et déficit ánima mea in átria Dómini.
}\switchcolumn\portugues{
\rlettrine{B}{endita} seja a Santíssima Trindade e a unidade indivisível! Cantaremos os seus louvores, porque Deus espalhou sobre nós a sua misericórdia. ℣. Bendigamos o Pai, e o Filho, e o Espírito Santo. \emph{Sl. 83, 23} Como são amáveis os vossos tabernáculos, ó Deus dos exércitos. Minha alma voa em transportes de amor ao pensar nos átrios do Senhor.
}\end{paracol}

\paragraphinfo{Antífona}{1 Jo. 2; 3; 4}
\begin{paracol}{2}\latim{
\rlettrine{U}{bi} cáritas et amor, Deus ibi est. ℣. Congregávit nos in unum Christi amor. ℣. Exsultémus et in ipso jucundémur. ℣. Timeámus et amémus Deum vivum. ℣. Et ex corde diligámus nos sincéro.
}\switchcolumn\portugues{
\rlettrine{D}{eus} está onde estiverem a caridade e o amor. ℣. Foi o amor ele Cristo que nos reuniu. ℣. Alegremo-nos e encontremos n’Ele as delícias. ℣. Temamos e amemos Deus vivo. ℣. Amemo-nos uns aos outros, sinceramente.
}\switchcolumn*\latim{
Ubi cáritas et amor, Deus ibi est. ℣. Simul ergo cum in unum congregámur: ℣. Ne nos mente dividámur, caveámus. ℣. Cessent júrgia malígna, cessent lites. ℣. Et in médio nostri sit Christus Deus.
}\switchcolumn\portugues{
Deus esta onde estiverem a caridade e o amor. ℣. Estamos reunidos em urna única assembleia. ℣. Evitemos tudo o que possa dividir os nossos corações. ℣. Longe de nós as rixas e as dissensões. ℣. Que Cristo, nosso Deus, esteja no meio de nós.
}\switchcolumn*\latim{
Ubi cáritas et amor, Deus ibi est. ℣. Simul quoque cum Beátis videámus ℣. Gloriánter vultum tuum, Christe Deus: ℣. Gáudium, quod est imménsum atque probum. ℣. Sǽcula per infiníta sæculórum. ℟. Amen.
}\switchcolumn\portugues{
Deus está onde estiverem a caridade e o amor. ℣. Fazei-nos ver com os Bem-aventurados. ℣. Vosso rosto está na glória, Cristo nosso Deus: ℣. Alegria imensa e pura. ℣. Em todos os séculos, pelos infinitos séculos. ℟. Amen.
}\end{paracol}

\begin{paracol}{2}\latim{
 Pater noster, \emph{secréto}. ℣. Et ne nos indúcas in tentatiónem.
}\switchcolumn\portugues{
Pai-nosso, \emph{em silêncio}. ℣. E não nos deixeis cair em tentação.
}\switchcolumn*\latim{
℟. Sed líbera nos a malo.
}\switchcolumn\portugues{
℟. Mas livrai-nos do mal.
}\switchcolumn*\latim{
℣. Tu mandásti mandáta tua, Dómine.
}\switchcolumn\portugues{
℣. Ordenastes, Senhor, que os vossos mandamentos:
}\switchcolumn*\latim{
℟. Custodíri nimis.
}\switchcolumn\portugues{
℟. Fossem bem observados.
}\switchcolumn*\latim{
℣. Tu lavásti pedes discipulórum tuórum.
}\switchcolumn\portugues{
℣. Lavastes os pés aos vossos discípulos.
}\switchcolumn*\latim{
℟. Opera mánuum tuárum ne despícias.
}\switchcolumn\portugues{
℟. Não desprezeis as obras das vossas mãos.
}\switchcolumn*\latim{
℣. Dómine, exáudi oratiónem meam.
}\switchcolumn\portugues{
℣. Senhor, ouvi a minha oração.
}\switchcolumn*\latim{
℟. Et clamor meus ad te véniat.
}\switchcolumn\portugues{
℟. E que meu clamor chegue até Vós.
}\switchcolumn*\latim{
℣. Dóminus vobíscum.
}\switchcolumn\portugues{
℣. O Senhor seja convosco.
}\switchcolumn*\latim{
℟. Et cum spíritu tuo.
}\switchcolumn\portugues{
℟. E com vosso espírito.
}\end{paracol}

\paragraph{Oração}
\begin{paracol}{2}\latim{
\rlettrine{A}{désto,} Dómine, quǽsumus, officio servitútis nostræ: et quia tu discípulis tuis pedes laváre dignátus es, ne despícias ópera mánuum tuárum, quæ nobis retinénda mandásti: ut, sicut hic nobis et a nobis exterióra abluúntur inquinaménta; sic a te ómnium nostrum interióra lavéntur peccáta. Quod ipse præstáre dignéris, qui vivis et regnas Deus: per ómnia sǽcula sæculórum. ℟. Amen.
}\switchcolumn\portugues{
\rlettrine{S}{enhor,} Vos imploramos, aceitai benignamente estas homenagens da nossa humildade, e, já que não hesitastes em lavar os pés aos vossos discípulos, não desprezeis o que acabámos de fazer, segundo o que nos mandastes, a fim de que, havendo sido purificados das manchas exteriores do corpo, sejamos também lavados por Vós das manchas interiores dos nossos pecados. Concedei-nos esta graça. Vós que, sendo Deus, viveis e reinais por todos os séculos dos séculos. ℟. Amen.
}\end{paracol}

\subsectioninfo{Sábado Santo}{Estação em S. João de Latrão}

\paragraph{Oração}
\begin{paracol}{2}\latim{
\rlettrine{D}{eus,} qui per Fílium tuum, angulárem scílicet lápidem, claritátis tuæ ignem fidélibus contulísti: prodúctum e sílice, nostris profutúrum úsibus, novum hunc ignem sanctí \cruz fica: et concéde nobis, ita per hæc festa paschália cœléstibus desidériis inflammári; ut ad perpétuæ claritátis, puris méntibus, valeámus festa pertíngere. Per eúndem Christum, Dóminum nostrum. Amen.
}\switchcolumn\portugues{
\slettrine{Ó}{} Deus, que pelo vosso Filho, que é a pedra angular da Igreja, fizestes resplandecer diante dos fiéis as chamas do fogo da vossa caridade, \cruz santificai este lume novo, que fizemos sair da pederneira, a fim de servir para nosso uso; e concedei-nos durante estas festas pascais que sejamos inflamados em santos desejos dos bens celestiais, de, tal sorte que com os corações purificados possamos chegar às festividades, onde se goza a luz perpétua. Pelo mesmo Cristo, nosso Senhor. Amen.
}\end{paracol}

\paragraph{Oração}
\begin{paracol}{2}\latim{
\rlettrine{D}{ómine} Deus, Pater omnípotens, lumen indefíciens, qui es cónditor ómnium lúminum: béne \cruz dic hoc lumen, quod a te sanctificátum atque benedíctum est, qui illuminásti omnem mundum: ut ab eo lúmine accendámur, atque illuminémur igne claritátis tuæ: et sicut illuminásti Móysen exeúntem de Ægýpto, ita illúmines corda, et sensus nostros; ut ad vitam et lucem ætérnam perveníre mereámur. Per Christum, Dóminum nostrum. ℟. Amen.
}\switchcolumn\portugues{
\rlettrine{S}{enhor} Deus, Pai omnipotente, Luz eterna e Criador de todas as luzes, \cruz abençoai este lume, que por Vós foi santificado e abençoado quando iluminastes com ele o mundo inteiro; fazei sair dele uma luz divina, que nos ilumine e abrase no fogo da vossa caridade; e, assim como iluminastes Moisés, quando saiu do Egipto, assim também iluminai agora os nossos corações e os nossos espíritos, para que mereçamos alcançar a vida e a luz eternas. Por Cristo, nosso Senhor. ℟. Amen.
}\end{paracol}

\paragraph{Oração}
\begin{paracol}{2}\latim{
\rlettrine{D}{ómine} sancte, Pater omnípotens, ætérne Deus: benedicéntibus nobis hunc ignem in nómine tuo, et unigéniti Fílii tui, Dei ac Dómini nostri Jesu Christi, et Spíritus Sancti, cooperári dignéris; et ádjuva nos contra igníta tela inimíci, et illústra grátia cœlésti: Qui vivis et regnas \emph{\&c.} ℟. Amen.
}\switchcolumn\portugues{
\rlettrine{S}{enhor} santo, Pai omnipotente, Deus eterno, dignai-Vos cooperar connosco, benzendo hoje este fogo no vosso nome e no do vosso Filho, Jesus Cristo, Deus e Senhor nosso, e no do Espírito Santo; auxiliai-nos a repelir as lanças inflamadas do inimigo e iluminai-nos com a graça celestial. Ó Vós, que viveis e reinais \emph{\&c.} ℟. Amen.
}\end{paracol}

\paragraph{Benção do Incenso}
\begin{paracol}{2}\latim{
\rlettrine{V}{éniat,} quǽsumus, omnípotens Deus, super hoc incénsum larga tuæ bene \cruz dictiónis infúsio: et hunc noctúrnum splendórem invisíbilis regenerátor accénde; ut non solum sacrifícium, quod hac nocte litátum est, arcána lúminis tui admixtióne refúlgeat; sed in quocúmque loco ex hujus sanctificatiónis mystério aliquid fúerit deportátum, expúlsa diabólicæ fraudis nequítia, virtus tuæ majestátis assístat. Per Christum, Dóminum nostrum. ℟. Amen.
}\switchcolumn\portugues{
\rlettrine{V}{os} suplicamos, ó Deus omnipotente, lançai sobre este incenso uma abundante efusão das vossas \cruz bênçãos; acendei, ó regenerador invisível, esta luz que deve iluminar-nos durante esta noite, a fim de que não seja somente o sacrifício que esta noite Vos é oferecido que projecte os clarões da vossa luz misteriosa, mas também, onde quer que seja levada qualquer porção do que, hoje, aqui benzemos, sejam aniquilados pelo poder da vossa majestade os artifícios da malícia do demónio. Por Cristo, nosso Senhor. ℟. Amen.
}\end{paracol}

\textit{A Procissão dirige-se para o Altar-mor. O Diácono leva a serpentina e canta:}

\begin{paracol}{2}\latim{
℣. Lumen Christi.
}\switchcolumn\portugues{
℣. A luz de Cristo.
}\switchcolumn*\latim{
 ℟. Deo grátias.
}\switchcolumn\portugues{
℟. Dêmos graças a Deus.
}\end{paracol}

\textit{Prossegue-se até ao meio do templo, todos ajoelham e o Diácono repete:}

\begin{paracol}{2}\latim{
℣. Lumen Christi.
}\switchcolumn\portugues{
℣. A luz de Cristo.
}\switchcolumn*\latim{
 ℟. Deo grátias.
}\switchcolumn\portugues{
℟. Dêmos graças a Deus.
}\end{paracol}

\textit{Á entrada do Altar novamente, todos ajoelham e o Diácono repete:}

\begin{paracol}{2}\latim{
℣. Lumen Christi.
}\switchcolumn\portugues{
℣. A luz de Cristo.
}\switchcolumn*\latim{
℟. Deo grátias.
}\switchcolumn\portugues{
℟. Dêmos graças a Deus.
}\end{paracol}

\paragraph{Bênção do Círio Pascal}

\textit{O Diácono pede a Bênção ao Celebrante e o Sacerdote:}

\begin{paracol}{2}\latim{
Dóminus sit in corde tuo et in labiis tuis: ut digne et competénter annúnties suum paschále præcóniurn: In nómine Patris, et Fílii, \cruz et Spíritus Sancti. ℟. Amen.
}\switchcolumn\portugues{
O Senhor seja no teu coração e nos teus lábios, para que dignamente possas anunciar, como convém, os louvores da Páscoa. Em nome do Pai, e do Filho, \cruz e do Espírito Santo. ℟. Amen.
}\end{paracol}

\begin{paracol}{2}\latim{
Exsúltet iam angélica turba cælórum: exsúltent divína mystéria: et pro tanti Regis victória tuba ínsonet salutáris. Gáudeat et tellus, tantis irradiáta fulgóribus: et ætérni Regis splendóre illustráta, tótius orbis se séntiat amisísse calíginem. Lætétur et mater Ecclésia, tanti lúminis adornáta fulgóribus: et magnis populórum vócibus hæc aula resúltet. Quaprópter astántes vos, fratres caríssimi, ad tam miram huius sancti lúminis claritátem, una mecum, quæso, Dei omnipoténtis misericórdiam invocáte. Ut, qui me non meis méritis intra Levitárum númerum dignátus est aggregáre, lúminis sui claritátem infúndens, cérei huius laudem implére perfíciat. ℟. Amen.
}\switchcolumn\portugues{
Exulte de alegria desde já no céu a multidão dos coros angélicos; celebrem-se com alegria os divinos mystérios; anuncie a tuba sagrada a vitória do soberano Rei! Que a terra seja iluminada e se alegre com os clarões duma tal glória; que o esplendor do Rei eterno, irradiando sobre ela, faça sentir ao universo que as trevas foram dissipadas! Alegre-se também a Igreja, nossa Mãe, adornada com os fulgores duma tal luz, e que ressoem neste templo as vozes festivas deste povo! Por isso, caríssimos irmãos, que aqui vos reunistes para participar dos esplendores desta tão santa luz, peço-vos que invoqueis comigo a misericórdia de Deus omnipotente, a fim de que eu, agregado ao número dos Levitas, ainda que sem merecimentos, receba um raio da sua luz e possa pela sua graça louvar dignamente este Círio. Por N. S. Jesus Cristo, seu Filho, que, sendo Deus, vive e reina com Ele na unidade do Espírito Santo, em todos os séculos dos séculos. ℟. Amen.
}\switchcolumn*\latim{
℣. Dóminus vobíscum.
}\switchcolumn\portugues{
℣. O Senhor seja convosco-
}\switchcolumn*\latim{
℟. Et cum spíritu tuo.
}\switchcolumn\portugues{
℟. E com vosso espírito.
}\switchcolumn*\latim{
℣. Sursum corda.
}\switchcolumn\portugues{
℣. Levantai os corações ao alto!
}\switchcolumn*\latim{
℟. Habémus ad Dóminum.
}\switchcolumn\portugues{
℟. Assim os temos para o Senhor.
}\switchcolumn*\latim{
℣. Grátias agámus Dómino Deo nostro.
}\switchcolumn\portugues{
℣. Dêmos graças ao Senhor, nosso Deus.
}\switchcolumn*\latim{
℟. Dignum et iustum est.
}\switchcolumn\portugues{
℟. Assim é digno e justo.
}\switchcolumn*\latim{
Vere dignum et iustum est, invisíbilem Deum Patrem omnipoténtem Filiúmque eius unigénitum, Dóminum nostrum Iesum Christum, toto cordis ac mentis afféctu et vocis ministério personáre. Qui pro nobis ætérno Patri Adæ débitum solvit, et véteris piáculi cautiónem pio cruóre detérsit. Hæc sunt enim festa paschália, in quibus verus ille Agnus occíditur, cuius sánguine postes fidélium consecrántur. Hæc nox est, in qua primum patres nostros, fílios Israel edúctos de Ægypto, Mare Rubrum sicco vestígio transíre fecísti. Hæc ígitur nox est, quæ peccatórum ténebras colúmnæ illuminatióne purgávit. Hæc nox est, quæ hódie per univérsum mundum in Christo credéntes, a vítiis sæculi et calígine peccatórum segregátos, reddit grátiæ, sóciat sanctitáti. Hæc nox est, in qua, destrúctis vínculis mortis, Christus ab ínferis victor ascéndit. Nihil enim nobis nasci prófuit, nisi rédimi profuísset. O mira circa nos tuæ pietátis dignátio! O inæstimábilis diléctio caritátis: ut servum redímeres, Fílium tradidísti! O certe necessárium Adæ peccátum, quod Christi morte delétum est! O felix culpa, quæ talem ac tantum méruit habére Redemptórem! O vere beáta nox, quæ sola méruit scire tempus et horam, in qua Christus ab ínferis resurréxit! Hæc nox est, de qua scriptum est: Et nox sicut dies illuminábitur: et nox illuminátio mea in delíciis meis. Huius ígitur sanctificátio noctis fugat scélera, culpas lavat: et reddit innocéntiam lapsis et mæstis lætítiam. Fugat ódia, concórdiam parat et curvat impéria.
}\switchcolumn\portugues{
Verdadeiramente é digno e justo louvar do íntimo do nosso coração e da nossa alma com os nossos cânticos o Deus invisível, o Pai omnipotente e o seu Filho Unigénito, Jesus Cristo, nosso Senhor, o qual pagou por nós ao Pai Eterno a dívida de Adão e apagou com seu precioso sangue o reato da antiga culpa. Porquanto são estas as festas pascais em que é imolado o verdadeiro Cordeiro, cujo sangue consagra as portas dos fiéis. Eis a noite em que tirastes do Egipto os nossos pais os filhos de Israel e os fizestes passar pelo mar Vermelho a pé enxuto. É esta a noite em que todo o universo foi arrancado aos vícios do mundo e às trevas do pecado, e os que crêem em Cristo restituídos à graça e unidos à sociedade dos santos. É esta a noite em que Cristo saiu triunfante dos infernos, depois de haver quebrado as cadeias da morte. Pois de nada nos aproveitaria havermos nascido, se não tivéssemos sido resgatados. Como é admirável a vossa bondade para connosco! Ó excesso incomparável da vossa caridade! Para resgatar o escravo, entregastes o vosso Filho! Ó pecado cie, Adão, sem dúvida necessário para ser apagado pela morte de Cristo! Ó feliz culpa, que nos alcançou um tal e tão grande Redentor! Ó noite deveras ditosa, que só tu conheceste o tempo e a hora em que Cristo ressuscitou dos infernos! Esta é a noite de que está escrito: «A noite será clara, como o dia; a noite será clarão para me iluminar nas minhas delícias». A santidade desta noite afugenta os crimes, lava as culpas, restitui aos culpados a inocência e dá alegria aos aflitos: dissipa os ódios, atrai a concórdia e subjuga os impérios.
}\switchcolumn*\latim{
\emph{Hic Diaconus infigit quinque grana incensi benedicti in Cereo m modum crucis, hoc ordine:}
}\switchcolumn\portugues{
\emph{O Diácono coloca os cinco grãos do Incenso no Cirio:}
}\switchcolumn*\latim{
O vere beáta nox, in qua terrénis cæléstia, humánis divína iungúntur! In huius ígitur noctis grátia, súscipe, sancte Pater, laudis huius sacrifícium vespertínum, quod tibi in hac cérei oblatióne solémni, per ministrórum manus de opéribus apum, sacrosáncta reddit Ecclésia. Sed iam colúmnæ huius præcónia nóvimus, quam in honórem Dei rútilans ignis accéndit.
}\switchcolumn\portugues{
Recebei, pois, ó Pai omnipotente, nesta noite sagrada, o sacrifício vespertino deste incenso, que Vos oferece a santa Igreja pelas mãos dos seus ministros com a oferta deste Círio, que é o fruto do trabalho das abelhas. Conhecemos já o que significa esta coluna de cera, que uma chama de fogo vai acender em honra de Deus.
}\switchcolumn*\latim{
\emph{Hic Diaconus accendit Cereum cum una ex tribus candelis in arundine positis. }
}\switchcolumn\portugues{
\emph{O Diácono acende o Círio:}
}\switchcolumn*\latim{
Qui, lícet sit divísus in partes, mutuáti tamen lúminis detrimenta non novit. Alitur enim liquántibus ceris, quas in substántiam pretiósæ huius lámpadis apis mater edúxit.
}\switchcolumn\portugues{
Esta chama, ainda que dividida, não sofre, contudo, diminuição alguma, comunicando a sua luz, pois tem como alimento a cera, que a abelha produziu para formar este precioso facho.
}\switchcolumn*\latim{
\emph{Hic accenduntur lámpades.}
}\switchcolumn\portugues{
\emph{O Acólito tira luz da Serpentina e acende as lâmpadas do Templo.}
}\switchcolumn*\latim{
O vere beáta nox, quæ exspoliávit Ægýptios, ditávit Hebrǽos! Nox, in qua terrénis cæléstia, humánis divína jungúntur. Orámus ergo te, Dómine: ut Céreus iste in honórem tui nóminis consecrátus, ad noctis hujus calíginem destruéndam, indefíciens persevéret. Et in odórem suavitátis accéptus, supérnis lumináribus misceátur. Flammas ejus lúcifer matutínus invéniat. Ille, inquam, lúcifer, qui nescit occásum. Ille, qui regréssus ab ínferis, humáno géneri serénus illúxit. Precámur ergo te, Dómine: ut nos fámulos tuos, omnémque clerum, et devotíssimum pópulum: una cum beatíssimo Papa nostro {\redx N.} et Antístite nostro {\redx N.} quiéte témporum concéssa, in his paschálibus gáudiis, assídua protectióne régere, gubernáre et conserváre dignéris. Réspice étiam ad devotíssimum Imperatórem\footnote[2]{Si non est coronatus, dicatur: eléctum.} nostrum {\redx N.} cujus tu, Deus, desidérii vota prænóscens, ineffábili pietátis et misericórdiæ tuæ múnere, tranquíllum perpétuæ pacis accómmoda: et cæléstem victóriam cum omni pópulo suo. Per eúndem Dóminum nostrum Jesum Christum, Fílium tuum: Qui tecum vivit et regnat in unitáte Spíritus Sancti Deus: per ómnia sǽcula sæculórum. ℟. Amen.
}\switchcolumn\portugues{
Ó noite verdadeiramente feliz, que despojou os egípcios para enriquecer os hebreus! Noite em que o céu se uniu à terra; e as cousas divinas às humanas! Vos suplicamos, pois, ó Senhor, permitais que este Círio, consagrado em honra do vosso nome, arda incessantemente para dissipar as trevas desta noite; que sua luz, evolando-se, como perfume suave, se misture com as luzes celestiais; que a Estrela da manhã, aquela Estrela que não conhece ocaso e que, surgindo dos lugares sombrios, espalhou a sua luz serena sobre o género humano, o encontre ainda aceso. Vos suplicamos, ainda, ó Senhor, que Vos digneis conceder-nos a paz e a tranquilidade nestas alegrias pascais; que a vossa constante protecção governe, conserve e dirija a todos nós, vossos servos, a todo o clero e ao povo fiel, com o nosso beatíssimo Santo Padre {\redx N.} e com o nosso Prelado {\redx N.}. Observai também o nosso devotíssimo Imperador\footnote[2]{Se não é coroado, diz-se: eleito.} {\redx N.}, e já que Vós sabeis, ó Senhor, os desejos do seu coração, concedei por vossa inefável graça, bondade e misericórdia, que ele desfrute, com todo seu povo, a tranquilidade da paz perpétua e da vitória celeste. Que convosco, e com o Espírito Santo, vive e reina por todos os séculos. ℟. Amen.
}\end{paracol}

\paragraphinfo{Primeira Profecia}{Gn. 1, 1-31; 2, 1-2}
\begin{paracol}{2}\latim{
\rlettrine{I}{n} princípio creavit Deus cœlum et terram. Terra autem erat inánis et vácua, et ténebræ erant super fáciem abýssi: et Spíritus Dei ferebátur super aquas. Dixítque Deus: Fiat lux. Et facta est lux. Et vidit Deus lucem, quod esset bona: et divísit lucem a ténebris. Appellavítque lucem Diem, et ténebras Noctem: factúmque est véspere et mane, dies unus. Dixit quoque Deus: Fiat firmaméntum in médio aquárum: et dívidat aquas ab aquis. Et fecit Deus firmaméntum, divisítque aquas, quæ erant sub firmaménto, ab his, quæ erant super firmaméntum. Et factum est ita. Vocavítque Deus firmaméntum, Cœlum: et factum est véspere et mane, dies secúndus. Dixit vero Deus: Congregéntur aquæ, quæ sub cœlo sunt, in locum unum: et appáreat árida. Et factum est ita. Et vocávit Deus áridam, Terram: congregationésque aquárum appellávit Maria. Et vidit Deus, quod esset bonum. Et ait: Gérminet terra herbam viréntem et faciéntem semen, et lignum pomíferum fáciens fructum juxta genus suum, cujus semen in semetípso sit super terram. Et factum est ita. Et prótulit terra herbam viréntem et faciéntem semen juxta genus suum, lignúmque fáciens fructum, et habens unumquódque seméntem secúndum spéciem suam. Et vidit Deus, quod esset bonum. Et factum est véspere et mane, dies tértius. Dixit autem Deus: Fiant luminária in firmaménto cœli, et dívidant diem ac noctem, et sint in signa et témpora et dies et annos: ut lúceant in firmaménto cœli, et illúminent terram. Et factum est ita. Fecítque Deus duo luminária magna: lumináre majus, ut præésset diéi: et lumináre minus, ut præésset nocti: et stellas. Et pósuit eas in firmaménto cœli, ut lucérent super terram, et præéssent diéi ac nocti, et divíderent lucem ac ténebras. Et vidit Deus, quod esset bonum. Et factum est véspere et mane, dies quartus. Dixit etiam Deus: Prodúcant aquæ réptile ánimæ vivéntis, et volátile super terram sub firmaménto caeli. Creavítque Deus cete grándia, et omnem ánimam vivéntem atque motábilem, quam prodúxerant aquæ in spécies suas, et omne volátile secúndum genus suum. Et vidit Deus, quod esset bonum. Benedixítque eis, dicens: Créscite et multiplicámini, et repléte aquas maris: avésque multiplicéntur super terram. Et factum est véspere et mane, dies quintus. Dixit quoque Deus: Prodúcat terra ánimam vivéntem in génere suo: juménta et reptília, et béstias terræ secúndum spécies suas. Factúmque est ita. Et fecit Deus béstias terræ juxta spécies suas,
et juménta, et omne réptile terræ in génere suo. Et vidit Deus, quod esset bonum, et ait: Faciámus hóminem ad imáginem et similitúdinem nostram: et præsit píscibus maris et volatílibus cœli, et béstiis universǽque terræ, omníque réptíli, quod movétur in terra. Et creávit Deus hóminem ad imáginem suam: ad imáginem Dei creávit illum, másculum et féminam creévit eos. Benedixítque illis Deus, et ait: Créscite et multiplicámini, et repléte terram, et subjícite eam, et dominámini píscibus maris et volatílibus cœli, et univérsis animántibus, quæ movéntur super terram. Dixítque Deus: Ecce, dedi vobis omnem herbam afferéntem semen super terram, et univérsa ligna, quæ habent in semetípsis seméntem géneris sui, ut sint vobis in escam: et cunctis animántibus terrae, omníque vólucri cœli, et univérsis, quæ movéntur in terra, et in quibus est ánima vivens, ut hábeant ad vescéndum. Et factum est ita. Vidítque Deus cuncta, quæ fécerat: et erant valde bona. Et factum est véspere et mane, dies sextus. Igitur perfécti sunt cœli et terra, et omnis ornátus eórum. Complevítque Deus die séptimo opus suum, quod fécerat: et requiévit die séptimo ab univérso ópere, quod patrárat.
}\switchcolumn\portugues{
\rlettrine{N}{o} princípio criou Deus o céu e a terra. A terra, porém, era informe e vazia; as trevas cobriam o abysmo; e o Espírito de Deus movia-se sobre as águas. Disse, então, Deus: «Faça-se a luz!». E a luz foi feita. E Deus viu que a luz era boa, separando a luz das trevas. E à luz chamou dia e às trevas chamou noite. E da tarde e da manhã se fez o primeiro dia. Disse, também, Deus: «Faça-se o firmamento, no meio das águas, para separar umas das outras». Fez-se, pois, o firmamento, que dividiu as águas que estavam debaixo do firmamento das águas que estavam acima do firmamento. Assim aconteceu. E Deus chamou céu ao firmamento. E da tarde e da manhã se fez o segundo dia. Disse ainda Deus: «Que as águas, que estão debaixo do céu, se reúnam em um só lugar e apareça o elemento árido». Assim aconteceu, chamando terra ao elemento árido, e chamando mares ao conjunto das águas. E viu Deus que era bom tudo quanto havia feito. Depois, disse Deus: «Que a terra produza erva verde; que dê semente; que as árvores produzam frutos, segundo a sua espécie, e contenham em si a sua semente própria». Assim aconteceu: a terra produziu erva verde, que dá semente, segundo a sua espécie, e as árvores produziram frutos, segundo a sua espécie, contendo cada uma delas a sua semente própria, segundo a sua espécie. E Deus viu que tudo era bom. E da tarde e da manhã se fez o terceiro dia. E disse Deus: «Que haja luminares no firmamento do céu, para distinguir o dia da noite; que eles sirvam de sinais para assinalar os tempos, as estações, os dias e os anos; e que brilhem no firmamento do céu e iluminem a terra». E assim aconteceu. Formou, então, Deus dous grandes luminares, sendo o maior para presidir ao dia e o menor para presidir à noite; e fez também as estrelas, que colocou no firmamento do céu para resplandecerem sobre a terra, presidindo umas ao dia e outras à noite e separando a luz das trevas. E Deus viu que isto era bom. E da tarde e da manhã se fez o quarto dia. Disse mais Deus: «Que as águas produzam animais, Vivendo nas águas, e que as aves voem sobre a terra, debaixo do firmamento do céu». Criou, então, Deus peixes grandes e todos os seres viventes que se movem, produzidos pelas águas, cada um segundo a sua espécie; e criou do mesmo modo todas as aves, segundo a sua espécie. E Deus viu que tudo isto era bom. Então, abençoou tudo, dizendo: «Crescei e multiplicai-vos e enchei as águas do mar; e que as aves se multipliquem na terra». E da tarde e da manhã se fez o quinto dia. E Deus continuou: «Que a terra produza seres animados, cada um segundo a sua espécie: animais domésticos, répteis e animais selvagens, segundo a sua espécie». E assim aconteceu. Deus criou, pois, os animais selvagens da terra, segundo a sua espécie, e os animais domésticos e os répteis, cada um segundo a sua espécie. E viu Deus que tudo isto era bom. Em seguida Deus disse: «Façamos o homem à nossa imagem e semelhança; e que ele mande nos peixes do mar, nas aves do céu, nos animais selvagens, em toda a terra e nos répteis, que se movem na terra». E Deus criou o homem à sua imagem. Ele o criou à imagem de Deus; e criou-os masculino e feminino. Então abençoou-os Deus e disse-lhes: «Crescei e multiplicai-vos; enchei a terra e governai-a; dominai os peixes do mar, as aves do céu e todos os animais que se movem na terra». Acrescentou Deus: «Eis que vos dou todas as ervas, que produzem sementes na terra, e todas as árvores, que dão sementes da sua espécie, para que vos sirvam de alimento, bem como a todos os animais da terra, a todas as aves do céu e a todos os animais vivos que se movem na terra e tenham sopro de vida, a fim de que possam alimentar-se». Assim aconteceu. E Deus viu todas as cousas que tinha feito; e viu que todas eram boas. E da tarde e da manhã se fez o sexto dia. Ficaram, pois, assim criados o céu, a terra e todos seus ornamentos. E concluiu Deus no sétimo dia todas as obras que havia feito; e no sétimo dia descansou de todas suas obras.
}\switchcolumn*\latim{
\begin{nscenter} Orémus. \end{nscenter}
}\switchcolumn\portugues{
\begin{nscenter} Oremos. \end{nscenter}
}\switchcolumn*\latim{
℣. Flectámus génua.
}\switchcolumn\portugues{
℣. Ajoelhemos!
}\switchcolumn*\latim{
℟. Leváte.
}\switchcolumn\portugues{
℟. Levantai-vos!
}\switchcolumn*\latim{
Deus, qui mirabíliter creásti hóminem et mirabílius redemísti: da nobis, quǽsumus, contra oblectaménta peccáti, mentis ratióne persístere; ut mereámur ad ætérna gáudia perveníre. Per Dóminum \emph{\&c.}
}\switchcolumn\portugues{
Ó Deus, que criastes o homem duma forma admirável e o resgatastes duma forma ainda mais admirável, permiti, Vos suplicamos, que, vigiando nós continuamente o nosso espírito, resistamos aos atractivos do pecado, a fim de merecermos a posse dos gozos eternos. Por nosso Senhor \emph{\&c.}
}\end{paracol}

\paragraphinfo{Segunda Profecia}{Gn. 5; 6; 7 \& 8}
\begin{paracol}{2}\latim{
\rlettrine{N}{oë} vero cum quingentórum esset annórum, génuit Sem, Cham et Japheth. Cumque cœpíssent hómines multiplicári super terram et fílias procreássent, vidéntes fílii Dei fílias hóminum, quod essent pulchræ, accepérunt sibi uxóres ex ómnibus, quas elégerant. Dixítque Deus: Non permanébit spíritus meus in hómine in ætérnum, quia caro est: erúntque dies illíus centum vigínti annórum. Gigántes autem erant super terram in diébus illis. Postquam enim ingréssi sunt fílii Dei ad fílias hóminum illǽque genuérunt, isti sunt poténtes a sǽculo viri famósi. Videns autem Deus, quod multa malítia hóminum esset in terra, et cuncta cogitátio cordis inténta esset ad malum omni témpore, pænítuit eum, quod hóminem fecísset in terra. Et tactus dolóre cordis intrínsecus: Delébo, inquit, hóminem, quem creávi, a fácie terræ, ab hómine usque ad animántia, a réptili usque ad vólucres cœli; pǽnitet enim me fecísse eos. Noë vero invénit grátiam coram Dómino. Hæ sunt generatiónes Noë: Noë vir justus atque perféctus fuit in generatiónibus suis, cum Deo ambulávit. Et génuit tres fílios, Sem, Cham et Japheth. Corrúpta est autem terra coram Deo et repléta est iniquitáte. Cumque vidísset Deus terram esse corrúptam (omnis quippe caro corrúperat viam suam super terram), dixit ad Noë: Finis univérsæ carnis venit coram me: repléta est terra iniquitáte a fácie eórum, et ego dispérdam eos cum terra. Fac tibi arcam de lignis lævigátis: mansiúnculas in arca fácies, et bitúmine línies intrínsecus et extrínsecus. Et sic fácies eam: Trecentórum cubitórum erit longitúdo arcæ, quinquagínta cubitórum latitúdo, et trigínta cubilórum altitúdo illíus. Fenéstram in arca fácies, et in cúbito consummábis summitátem ejus: óstium autem arcæ pones ex látere: deórsum cenácula et trístega fácies in ea. Ecce, ego addúcam aquas dilúvii super terram, ut interfíciam omnem carnem, in qua spíritus vitæ est subter cœlum. Univérsa, quæ in terra sunt, consuméntur. Ponámque fœdus meum tecum: et ingrédiens arcam tu et fílii tui, uxor tua et uxóres filiórum tuórum tecum. Et ex cunctis animántibus univérsæ carnis bina indúces in arcam, ut vivant tecum: masculíni sexus et feminíni. De volúcribus juxta genus suum, et de juméntis in génere suo, et ex omni réptili terræ secúndum genus suum: bina de ómnibus ingrediántur tecum, ut possint vívere. Tolles ígitur tecum ex ómnibus escis, quæ mandi possunt, et comportábis apud te: et erunt tam tibi quam illis in cibum. Fecit ígitur Noë ómnia, quæ præcéperat illi Deus. Erátque sexcentórum annórum, quando dilúvii aquæ inundavérunt super terram. Rupti sunt omnes fontes abýssi magnæ, et cataráctæ cœli apértæ sunt: et facta est plúvia super terram quadragínta diébus et quadragínta nóctibus. In artículo diei illíus ingréssus est Noë, et Sem et Cham et Japheth, fílii ejus, uxor illíus et tres uxóres filiórum ejus cum eis in arcam: ipsi, et omne ánimal secúndum genus suum, univérsaque juménta in génere suo, et omne, quod movétur super terram in génere suo, cunctúmque volátile secúndum genus suum. Porro arca ferebátur super aquas. Et aquæ prævaluérunt nimis super terram: opertíque sunt omnes montes excélsi sub univérso cœlo. Quíndecim cúbitis áltior fuit aqua super montes, quos operúerat. Consúmptaque est omnis caro, quæ movebátur super terram, vólucrum, animántium, bestiárum, omniúmque reptílium, quæ reptant super terram. Remánsit autem solus Noë, et qui cum eo erant in arca. Obtinuerúntque aquæ terram centum quinquagínta diébus. Recordátus autem Deus Noë, cunctorúmque animántium et ómnium jumentórum, quæ erant cum eo in arca, addúxit spíritum super terram, et imminútæ sunt aquæ. Et clausi sunt fontes abýssi et cataráctæ cœli: et prohíbitæ sunt plúviæ de cœlo. Reversǽque sunt aquæ de terra eúntes et redeúntes: et cœpérunt mínui post centum quinquagínta dies. Cumque transíssent quadragínta dies, apériens Noe fenéstram arcæ, quam fécerat, dimísit corvum, qui egrediebátur, et non revertebátur, donec siccaréntur aquæ super terram. Emísit quoque colúmbam post eum, ut vidéret, si jam cessássent aquæ super fáciem terræ. Quæ cum non invenísset, ubi requiésceret pes ejus, revérsa est ad eum in arcam: aquæ enim erant super univérsam terram: extendítque manum et apprehénsam íntulit in arcam. Exspectátis autem ultra septem diébus áliis, rursum dimisit colúmbam ex arca. At illa venit ad eum ad vésperam, portans ramum olívæ viréntibus fóliis in ore suo. Intelléxit ergo Noë, quod cessássent aquæ super terram. Exspectavítque nihilminus septem álios dies: et emísit colúmbam, quæ non est revérsa ultra ad eum. Locútus est autem Deus ad Noë, dicens: Egrédere de arca, tu et uxor tua, fílii tui et uxóres filiórum tuórum tecum. Cuncta animántia, quæ sunt apud te, ex omni carne, tam in volatílibus quam in béstiis et univérsis reptílibus, quæ reptant super terram, educ tecum, et ingredímini super terram: créscite et multiplicámini super eam. Egréssus est ergo Noë et fílii ejus, uxor illíus et uxóres filiórum ejus cum eo. Sed et ómnia animántia, juménta et reptília, quæ reptant super terram, secúndum genus suum, egréssa sunt de arca. Ædificávit autem Noë altáre Dómino: et tollens de cunctis pecóribus et volúcribus mundis, óbtulit holocáusta super altáre. Odoratúsque est Dóminus odórem suavitátis.
}\switchcolumn\portugues{
\qlettrine{Q}{uando,} pois, Noé contava a idade de quinhentos anos, gerou Sem, Cam e Jafet. E, tendo os homens começado a multiplicar-se sobre a terra e tendo gerado filhas, viram os filhos de Deus que as filhas dos homens eram formosas; e, então, escolheram para suas mulheres as que lhes agradaram mais. Disse, pois, Deus: «Meu espírito não permanecerá sempre no homem, porquanto este não é senão carnal. Seus dias serão somente cento e vinte anos!». Ora, naquele tempo, havia gigantes na terra, porquanto, depois que os filhos de Deus se reuniram às filhas dos homens, nasceram delas aqueles homens robustos e afamados em toda a antiguidade. Vendo, pois, Deus que a malícia daqueles homens era grande e que todos os pensamentos do seu coração se concentravam continuamente no mal, arrependeu-se de haver criado o homem no mundo. Então, cheio de dor, até ao íntimo do seu coração, disse: «Exterminarei da face da terra o homem, que criei, e bem assim os animais, os répteis e as aves do céu; pois estou arrependido de os haver criado». Mas Noé encontrou graça diante do Senhor. Eis a posteridade de Noé: Noé foi varão justo, perfeito e obediente a Deus, havendo gerado três filhos: Sem, Cam e Jafet. Entretanto, a terra estava corrompida diante de Deus e repleta de iniquidades. Vendo, pois, Deus que a terra estava corrompida (pois, segundo o modo de vida dos homens na terra, toda a carne o estava também), disse o seguinte a Noé: «O fim de toda a carne está chegado diante de mim. Destruirei todos os homens da face da terra, assim como esta, pois os homens a encheram, com seus crimes! Mas tu construirás uma arca de madeiras bem aparelhadas; farás nela divisões pequenas e taparás todos seus buracos com betume, tanto por dentro, como por fora. Eis como procederás: terá a arca trezentos côvados de comprimento, cinquenta de largura e trinta de altura. Farás na arca uma janela, que terá um côvado de altura; a porta da arca será ao lado; e dentro construirás aposentos com três andares. Vou inundar a terra com um dilúvio de águas para destruir tudo o que seja vivente e que se encontra debaixo do céu e acima da terra; mas contigo farei uma aliança. Entrarás na arca com tua mulher, teus filhos e suas mulheres. Também farás entrar na arca, para conservares contigo, dous animais de cada espécie: um macho, outro fêmea. As aves, segundo a sua espécie, os animais domésticos das diversas espécies e todos os répteis que rastejam na terra (dous de cada espécie) entrarão contigo na arca, para que se possam conservar. Farás provisão abundante de comidas e as acumularás contigo, para servirem de alimento, tanto a ti, como a eles». Noé fez, então, tudo como o Senhor lhe ordenara. Contava ele seiscentos anos quando as águas do dilúvio inundaram a terra: romperam-se as fontes e depósitos do grande abysmo e abriram-se as cataratas do céu, caindo a chuva sobre a terra durante quarenta dias e quarenta noites! Tendo chegado o dia designado, Noé entrou na arca com seus filhos Sem, Cam e Jafet, sua mulher e as três mulheres de seus filhos, e bem assim todos os animais selvagens, segundo a sua espécie, e também todos os répteis, segundo a sua espécie, e todas as aves, que voam nos ares, segundo a sua espécie. E a arca flutuava sobre as águas, as quais, engrossando cada vez mais, excederam multo a terra e cobriram as mais altas montanhas que havia debaixo do céu! As águas elevaram-se quinze côvados sobre as montanhas que ela cobria. E assim pereceu todo o animal que se movia na terra: aves, animais domésticos e selvagens, répteis e tudo o que se movia na terra, sobrevivendo somente Noé e os que estavam com ele na arca. As águas ficaram cobrindo a terra por espaço de cento e cinquenta dias! Deus recordou-se, então, de Noé e de todos os animais selvagens e domésticos que estavam com ele na arca, e fez soprar sobre a terra um vento forte, diminuindo logo as águas. As fontes dos abysmos e as cataratas do céu fecharam-se, cessando a chuva. As águas, tendo sido agitadas, fortemente, pelo vento, retiraram-se pouco a pouco da terra e diminuíram depois de cento e cinquenta dias. Passados quarenta dias, Noé abriu a janela, que havia feito na arca, e soltou um corvo, o qual saiu e não voltou até que as águas secaram sobre a terra. Depois, soltou de ao pé de si uma pomba, para conhecer se as águas já haviam diminuído da face da terra; mas a pomba, não havendo encontrado onde pousar o pé (pois a terra ainda estava coberta de águas), voltou para a arca. Noé estendeu a mão e recolheu-a dentro da arca. Esperou ainda Noé outros sete dias, após os quais novamente soltou da arca uma pomba, que pela tarde desse dia voltou, trazendo no bico um ramo de oliveira com as folhas verdes; pelo que conheceu Noé que as águas se haviam retirado da superfície da terra. Mais sete dias esperou ainda Noé, e outra vez tornou a soltar uma pomba, a qual não tornou a voltar à arca. E Deus falou a Noé, dizendo-lhe: «Sai da arca, tu, tua mulher, teus filhos e suas mulheres; e bem assim faz sair todos os animais que entraram contigo, de todas as espécies, tanto de aves, como de animais e de répteis, que rastejam na terra. Espalhai-vos de novo pela terra e crescei e multiplicai-vos por ela». Noé saiu, pois, da arca e, consigo, sua mulher, seus filhos e as mulheres de seus filhos, e todos os animais, tanto os selvagens, como os domésticos e os répteis, que rastejam pela terra, cada um segundo a sua espécie. Construiu então Noé um altar em honra do Senhor, e, tomando animais puros e aves limpas, ofereceu-os em holocausto sobre o altar. E o Senhor recebeu este sacrifício como uma oferta de agradável odor.
}\switchcolumn*\latim{
\begin{nscenter} Orémus. \end{nscenter}
}\switchcolumn\portugues{
\begin{nscenter} Oremos. \end{nscenter}
}\switchcolumn*\latim{
℣. Flectámus génua.
}\switchcolumn\portugues{
℣. Ajoelhemos!
}\switchcolumn*\latim{
℟. Leváte.
}\switchcolumn\portugues{
℟. Levantai-vos!
}\switchcolumn*\latim{
D eus, incommutábilis virtus et lumen ætérnum: réspice propítius ad totíus Ecclésiæ tuæ mirábile sacraméntum, et opus salútis humánæ, perpétuæ dispositiónis efféctu, tranquíllius operáre; totúsque mundus experiátur et vídeat, dejécta erigi, inveteráta renovári, et per ipsum redire ómnia in intégrum, a quo sumpsére princípium:
Dóminum nostrum Jesum Christum, Fílium tuum: Qui tecum \emph{\&c.}
}\switchcolumn\portugues{
Ó Deus, poder imutável e luz eterna, dignai-Vos olhar propício para as maravilhas da vossa Igreja; e, por efeito dos vossos eternos decretos, dignai-Vos operar a salvação humana, a fim de que o mundo inteiro experimente e veja que está erguido o que jazia por terra; que está renovado o que estava envelhecido; e que tudo foi restabelecido na sua primitiva integridade por Aquele que é o princípio de tudo: nosso Senhor Jesus Cristo, vosso Filho, que convosco vive \emph{\&c.}
}\end{paracol}

\paragraphinfo{Terceira Profecia}{Gn. 22, 1-19}\label{terceiraprofecia}
\begin{paracol}{2}\latim{
\rlettrine{I}{n} diébus illis: Tentávit Deus Abraham, et dixit ad eum: Abraham, Abraham. At ille respóndit: Adsum. Ait illi: Tolle fílium tuum unigénitum, quem diligis, Isaac, et vade in terram visiónis: atque ibi ófferes eum in holocáustum super unum móntium, quem monstrávero tibi. Igitur Abraham de nocte consúrgens, stravit ásinum suum: ducens secum duos júvenes et Isaac, fílium suum. Cumque concidísset ligna in holocáustum, ábiit ad locum, quem præcéperat ei Deus. Die autem tértio, elevátis óculis, vidit locum procul: dixítque ad púeros suos: Exspectáte hic cum ásino: ego et puer illuc usque properántes, postquam adoravérimus, revertémur ad vos. Tulit quoque ligna holocáusti, et impósuit super Isaac, fílium suum: ipse vero portábat in mánibus ignem et gládium. Cumque duo pérgerent simul, dixit Isaac patri suo: Pater mi. At ille respóndit: Quid vis, fili? Ecce, inquit, ignis et ligna: ubi est víctima holocáusti? Dixit autem Abraham: Deus providébit sibi víctimam holocáusti, fili mi. Pergébant ergo páriter: et venérunt ad locum, quem osténderat ei Deus, in quo ædificávit altáre et désuper ligna compósuit: cumque alligásset Isaac, fílium suum, pósuit eum in altare super struem lignórum. Extendítque manum et arrípuit gládium, ut immoláret fílium suum. Et ecce, Angelus Dómini de cœlo clamávit, dicens: Abraham, Abraham. Qui respóndit: Adsum. Dixítque ei: Non exténdas manum tuam super púerum neque fácias illi quidquam: nunc cognóvi, quod times Deum, et non pepercísti unigénito fílio tuo propter me. Levávit Abraham óculos suos, vidítque post tergum aríetem inter vepres hæréntem córnibus, quem assúmens óbtulit holocáustum pro fílio. Appellavítque nomen loci illíus, Dóminus videt. Unde usque hódie dícitur: In monte Dóminus vidébit. Vocávit autem Angelus Dómini Abraham secúndo de cœlo, dicens: Per memetípsum jurávi, dicit Dóminus: quia fecísti hanc rem, et non pepercísti fílio tuo unigénito propter me: benedícam tibi, et multiplicábo semen tuum sicut stellas cœli et velut arénam, quæ est in lítore maris: possidébit semen tuum portas inimicórum suórum, et benedicéntur in sémine tuo omnes gentes terræ, quia obœdísti voci meæ. Revérsus est Abraham ad púeros suos, abierúntque Bersabée simul, et habitávit ibi.
}\switchcolumn\portugues{
\rlettrine{N}{aqueles} dias, provou Deus a Abraão, dizendo-lhe: «Abraão, Abraão!». Este respondeu: «Eis-me aqui». E Deus disse: «Toma teu filho único, Isaque, a quem amas, vai à terra da visão e oferece-mo em holocausto, sobre um dos montes que Eu te indicar». Levantou-se, então, Abraão, de madrugada, aparelhou o jumento, levou consigo dous criados e seu filho Isaque. E, havendo cortado a lenha para o holocausto, encaminhou-se para o lugar que Deus lhe indicara. Ao terceiro dia, Abraão, erguendo os olhos, viu ao longe este lugar. Disse, pois, a seus servos: «Ficai aqui com o jumento, enquanto eu e Isaque vamos até lá; e, depois de havermos adorado, voltaremos para junto de vós». Tomou a lenha para o holocausto e entregou-a a Isaque, para este a conduzir, levando ele na mão o fogo e o cutelo. Caminhando assim, disse Isaque a seu pai Abraão: «Meu Pai!». Este respondeu: «Que queres, meu filho?». Isaque continuou: «Eis aqui o fogo e a lenha; mas onde está o cordeiro para o holocausto?». Abraão respondeu: «O próprio Deus cuidará de nos dar a vítima para o holocausto, meu filho!». E continuou a caminhar, até que chegaram ao lugar que Deus havia designado. Abraão levantou aí o altar, sobre o qual preparou a lenha. Depois amarrou seu filho Isaque e deitou-o em cima da lenha. Logo, estendeu a mão e empunhou o cutelo para degolar o filho. Então o Anjo do Senhor gritou do céu: «Abraão! Abraão!». Ele respondeu: «Eis-me aqui!». O Anjo continuou: «Não estendas a tua mão para teu filho e lhe não faças mal; pois agora sei que temes o Senhor e que, para me obedeceres, nem poupavas o teu filho único!». Abraão, tendo erguido os olhos ao céu, viu atrás de si um carneiro preso no mato pelas hastes. Tomou, pois, o carneiro e ofereceu-o em holocausto, em lugar do filho. Abraão chamou depois a este lugar: «O Senhor vê»; o qual ainda hoje conserva esse nome. O Anjo do Senhor novamente chamou do céu Abraão, dizendo: «Juro-o por mim mesmo, diz o Senhor, pois que procedeste assim e não poupavas o teu filho único por amor de mim, Eu te abençoarei; e multiplicarei a tua descendência, como as estrelas do céu e como a areia da praia do mar; a tua posteridade possuirá as cidades de seus inimigos e todas as gerações da terra serão abençoadas naquele que sairá de ti; pois obedeceste à minha voz». Então, Abraão voltou ao lugar onde estavam os seus servos e tornaram juntos para Bersabeia, onde habitou.
}\switchcolumn*\latim{
\begin{nscenter} Orémus. \end{nscenter}
}\switchcolumn\portugues{
\begin{nscenter} Oremos. \end{nscenter}
}\switchcolumn*\latim{
℣. Flectámus génua.
}\switchcolumn\portugues{
℣. Ajoelhemos!
}\switchcolumn*\latim{
℟. Leváte.
}\switchcolumn\portugues{
℟. Levantai-vos!
}\switchcolumn*\latim{
D eus, fidélium Pater summe, qui in toto orbe terrárum, promissiónis tuæ fílios diffúsa adoptiónis grátia multíplicas: et per paschále sacraméntum, Abraham púerum tuum universárum, sicut jurásti, géntium éfficis patrem; da pópulis tuis digne ad grátiam tuæ vocatiónis introíre. Per Dóminum \emph{\&c.}
}\switchcolumn\portugues{
Ó Deus, Pai soberano dos fiéis, que, espalhando por toda a terra a graça da adopção, multiplicais nela os filhos da promessa, e que, segundo a vossa promessa, pelo mystério pascal estabelecestes o vosso servo Abraão como pai de todas as nações, concedei aos vossos povos a graça de corresponderem dignamente à vossa vocação. Por nosso Senhor \emph{\&c.}
}\end{paracol}

\paragraphinfo{Quarta Profecia}{Ex. 14, 24-31; 15, 1}\label{quartaprofecia}
\begin{paracol}{2}\latim{
\rlettrine{I}{n} diébus illis: Factum est in vigília matutina, et ecce, respíciens Dóminus super castra Ægyptiórum per colúmnam ignis et nubis, interfécit exércitum eórum: et subvértit rotas cúrruum, ferebantúrque in profúndum. Dixérunt ergo Ægýptii: Fugiámus Isrǽlem: Dóminus enim pugnat pro eis contra nos. Et ait Dóminus ad Móysen: Exténde manum tuam super mare, ut revertántur aquæ ad Ægýptios super currus et équites eórum. Cumque extendísset Moyses manum contra mare, revérsum est primo dilúculo ad priórem locum: fugientibúsque Ægýptiis occurrérunt aquæ, et invólvit eos Dóminus in médiis flúctibus. Reversǽque sunt aquæ, et operuérunt currus, et équites cuncti exércitus Pharaónis, qui sequéntes ingréssi fúerant mare: nec unus quidem supérfuit ex eis. Fílii autem Israël perrexérunt per médium sicci maris, et aquæ eis erant quasi pro muro a dextris et a sinístris: liberavítque Dóminus in die illa Israël de manu Ægyptiórum. Et vidérunt Ægýptios mórtuos super litus maris, et manum magnam, quam exercúerat Dóminus contra eos: timuítque pópulus Dóminum, et credidérunt Dómino et Moysi, servo ejus. Tunc cécinit Moyses et fílii Israël carmen hoc Dómino, et dixérunt:
}\switchcolumn\portugues{
\rlettrine{N}{aqueles} dias, chegada a vigília da manhã, olhando o Senhor, através da coluna de fogo da nuvem para o arraial dos egípcios, destroçou o seu exército e despedaçou as rodas dos carros, que foram lançados nos abysmos do mar. Disseram, então, os egípcios: «Fujamos diante de Israel, pois o Senhor combate em seu favor contra nós». O Senhor disse a Moisés: «Estende a tua mão por cima do mar, para que as águas recuem sobre os egípcios, seus carros e seus cavaleiros». Moisés, quando amanheceu, estendeu a mão por cima do mar, o qual voltou ao seu curso habitual; e, querendo os egípcios fugir, vieram as águas ao seu encontro, e o Senhor os envolveu no meio das ondas do mar. Tornaram-se a unir as águas e cobriram os carros e os cavaleiros de Faraó, que haviam entrado no mar, em sua perseguição. Porém, os filhos de Israel caminharam em seco no meio do mar, formando as águas como que uma muralha à direita e à esquerda. Assim salvou o Senhor, naquele dia, Israel das mãos dos egípcios; e Israel viu os egípcios mortos na praia do mar e os efeitos da mão poderosa do Senhor, levantada contra eles. Então o povo temeu Deus e acreditou em Deus e em Moisés, seu servo. E Moisés e os filhos de Israel cantaram a Deus este hino:
}\end{paracol}

\paragraphinfo{Trato}{Ex. 15, 1 \& 2}\label{tratoquartaprofecia}
\begin{paracol}{2}\latim{
\rlettrine{C}{antémus} Dómino: glorióse enim honorificátus est: equum et ascensórem projécit in mare: adjútor et protéctor factus est mihi in salútem. ℣. Hic Deus meus, et honorificábo eum: Deus patris mei, et exaltábo eum. ℣. Dóminus cónterens bella: Dóminus nomen est illi.
}\switchcolumn\portugues{
\rlettrine{C}{antemos} ao Senhor, porque gloriosamente manifestou o seu poder, precipitando no mar o cavalo e o cavaleiro. Ele foi o meu auxílio e protecção; foi o meu salvador. ℣. Ele é o meu Deus. Eu o glorificarei. Este é o Deus de meu pai. Eu o exaltarei. ℣. É o Senhor quem vence as guerras: o seu nome é Jeová.
}\switchcolumn*\latim{
\begin{nscenter} Orémus. \end{nscenter}
}\switchcolumn\portugues{
\begin{nscenter} Oremos. \end{nscenter}
}\switchcolumn*\latim{
℣. Flectámus génua.
}\switchcolumn\portugues{
℣. Ajoelhemos!
}\switchcolumn*\latim{
℟. Leváte.
}\switchcolumn\portugues{
℟. Levantai-vos!
}\switchcolumn*\latim{
D eus, cujus antíqua mirácula etiam nostris sǽculis coruscáre sentímus: dum, quod uni pópulo, a persecutióne Ægyptíaca liberándo, déxteræ tuæ poténtia contulísti, id in salútem géntium per aquam regeneratiónis operáris: præsta; ut in Abrahæ fílios et in ísraëlíticam dignitátem, totíus mundi tránseat plenitúdo. Per Dóminum \emph{\&c.}
}\switchcolumn\portugues{
Ó Deus, que em nossos dias renovais ainda as vossas antigas maravilhas, operando, para a salvação das nações, pela água da regeneração, o que o poder da vossa dextra praticou para a salvação de um povo, livrando-o da perseguição dos egípcios, determinai que todos os homens da terra se tornem filhos de Abraão e participem das honras concedidas ao povo de Israel. Por nosso Senhor \emph{\&c.}
}\end{paracol}

\paragraphinfo{Quinta Profecia}{Is. 54, 17; 55, 1-11}
\begin{paracol}{2}\latim{
\rlettrine{H}{æc} est heréditas servórum Dómini: et justítia eórum apud me, dicit Dóminus. Omnes sitiéntes, veníte ad aquas: et qui non habétis argéntum, properáte, émite et comédite: veníte, émite absque argénto et absque ulla commutatióne vinum et lac. Quare appénditis argéntum non in pánibus, et labórem vestrum non in saturitáte? Audíte audiéntes me, et comédite bonum, et delectábitur in crassitúdine ánima vestra. Inclináte aurem vestram, et veníte ad me: audíte, et vivet ánima vestra, et fériam vobíscum pactum sempitérnum, misericórdias David fidéles. Ecce, testem pópulis dedi eum, ducem ac præceptórem géntibus. Ecce, gentem, quam nesciébas, vocábis: et gentes, quæ te non cognovérunt, ad te current propter Dóminum, Deum tuum, et sanctum Israël, quia glorificávit te. Quǽrite Dóminum, dum inveníri potest: invocáte eum, dum prope est. Derelínquat ímpius viam suam et vir iníquus cogitatiónes suas, et revertátur ad Dóminum, et miserébitur ejus, et ad Deum nostrum: quóniam multus est ad ignoscéndum. Non enim cogitatiónes meæ cogitatiónes vestræ: neque viæ vestræ viæ meæ, dicit Dóminus. Quia sicut exaltántur cœli a terra, sic exaltátæ sunt viæ meæ a viis vestris, et cogitatiónes meæ a cogitatiónibus vestris. Et quómodo descéndit imber et nix de cœlo, et illuc ultra non revértitur, sed inébriat terram, et infúndit eam, et germináre eam facit, et dat semen serénti et panem comedénti: sic erit verbum meum, quod egrediátur de ore meo: non revertátur ad me vácuum, sed fáciet, quæcúmque volui, et prosperábitur in his, ad quæ misi illud: dicit Dóminus omnípotens.
}\switchcolumn\portugues{
\rlettrine{E}{sta} é a herança dos servos do Senhor; esta é a justiça que devem esperar de mim, diz o Senhor. «Ó vós, que tendes sede, vinde às águas; ó vós, que não tendes dinheiro, vinde depressa, comprai e comei; vinde comprar o vinho e o leite sem dinheiro e sem nada dar em troca. Porque gastais o vosso dinheiro no que vos não pode alimentar? Porque empregais o vosso trabalho no que não pode saciar-vos? Ouvi-me, pois, com atenção: comei o que é bom, e a vossa alma se deleitará com os manjares mais substanciosos. Escutai-me e vinde a mim; escutai-me e viverá a vossa alma; e farei convosco um pacto eterno, concedendo-vos as graças que prometi a David. Eis Aquele que enviei aos povos, como testemunho, e às nações, como príncipe, como governador e mestre. Chamareis um povo, que não conheceis; e as nações, que vos não conheciam, correrão para vós, por amor do Senhor, vosso Deus, e do santo de Israel, que vos glorificou. Procurai Senhor, enquanto podeis encontrá-lo; invocai-o, enquanto está próximo. Que o ímpio abandone o mau caminho; que o homem iníquo afugente os maus pensamentos; que se converta ao Senhor, que será misericordioso; que se volte para o nosso Deus, que lhe perdoará generosamente. Porquanto, disse o Senhor, os meus pensamentos não são os vossos pensamentos, nem os meus caminhos são os vossos caminhos. Assim como o céu é mais elevado do que a terra, assim os meus caminhos são mais elevados do que os vossos e os meus pensamentos mais nobres do que os vossos. E assim como a chuva e a neve caem do céu e para lá não tornam sem que saciem a terra e a fecundem e nela façam produzir pão para alimento e para a semente, assim também a palavra, que há-de sair de mim, não voltará a mim sem haver produzido fruto. Ela fará tudo aquilo que Eu quero e produzirá aquele efeito para que a enviei», diz o Senhor omnipotente.
}\switchcolumn*\latim{
\begin{nscenter} Orémus. \end{nscenter}
}\switchcolumn\portugues{
\begin{nscenter} Oremos. \end{nscenter}
}\switchcolumn*\latim{
℣. Flectámus génua.
}\switchcolumn\portugues{
℣. Ajoelhemos!
}\switchcolumn*\latim{
℟. Leváte.
}\switchcolumn\portugues{
℟. Levantai-vos!
}\switchcolumn*\latim{
O mnípotens sempitérne Deus, multíplica in honórem nóminis tui, quod patrum fídei spopondísti: et promissiónis fílios sacra adoptióne diláta; ut, quod prióres Sancti non dubitavérunt futúrum, Ecclésia tua magna jam ex parte cognóscat implétum. Per Dóminum \emph{\&c.}
}\switchcolumn\portugues{
Deus omnipotente e eterno, para glória do vosso nome, multiplicai a posteridade prometida à fé de nossos pais, e, pela santa adopção, aumentai o número dos filhos da promessa, a fim de que a vossa Igreja conheça que no seu seio tiveram já realização, em grande parte, aquelas promessas que os primeiros santos acreditaram firmemente que haviam de se cumprir. Por nosso Senhor \emph{\&c.}
}\end{paracol}

\paragraphinfo{Sexta Profecia}{Br. 3, 9-38}\label{6profecia}
\begin{paracol}{2}\latim{
\rlettrine{A}{udi,} Israël, mandata vitæ: áuribus pércipe, ut scias prudéntiam. Quid est, Israël, quod in terra inimicórum es? Inveterásti in terra aliéna, coinquinátus es cum mórtuis: deputátus es cum descendéntibus in inférnum. Dereliquísti
fontem sapiéntiæ. Nam si in via Dei ambulásses, habitásses útique in pace sempitérna. Disce, ubi sit prudéntia, ubi sit virtus, ubi sit intelléctus: ut scias simul, ubi sit longitúrnitas vitæ et victus, ubi sit lumen oculórum et pax. Quis invénit locum ejus? et quis intrávit in thesáuros ejus? Ubi sunt príncipes géntium, et qui dominántur super béstias, quæ sunt super terram? qui in ávibus cœli ludunt, qui argéntum thesaurízant et aurum, in quo confídunt hómines, et non est finis acquisitiónis eórum? qui argéntum fábricant, et sollíciti sunt, nec est invéntio óperum illórum? Extermináti sunt, et ad ínferos descendérunt, et álii loco eórum surrexérunt. Júvenes vidérunt lumen, et habitavérunt super terram: viam autem disciplínæ ignoravérunt, neque intellexérunt sémitas ejus, neque fílii eórum suscepérunt eam, a fácie ipsórum longe facta est: non est audíta in terra Chánaan, neque visa est in Theman. Fílii quoque Agar, qui exquírunt prudéntiam, quæ de terra est, negotiatóres Merrhæ et Theman, et fabulatóres, et exquisitóres prudéntiæ et intellegéntias: viam autem sapiéntiæ nesciérunt, neque commemoráti sunt sémitas ejus. O Israël, quam magna est domus Dei et ingens locus possessiónis ejus! Magnus est et non habet finem: excélsus et imménsus. Ibi fuérunt gigántes nomináti illi, qui ab inítio fuérunt, statúra magna, sciéntes bellum. Non hos elegit Dóminus, neque viam disciplínæ invenérunt: proptérea periérunt. Et quóniam non habuérunt sapiéntiam, interiérunt propter suam insipiéntiam. Quis ascéndit in cœlum, et accépit eam et edúxit eam de núbibus? Quis transfretávit mare, et invénit illam? et áttulit illam super aurum eléctum? Non est, qui possit scire vias ejus neque qui exquírat sémitas ejus: sed qui scit univérsa, novit eam et adinvénit eam prudéntia sua: qui præparávit terram in ætérno témpore, et replévit eam pecúdibus et quadrupédibus: qui emíttit lumen, et vadit: et vocávit illud, et obǽdit illi in tremore. Stellæ autem dedérunt lumen in custódiis suis, et lætátæ sunt: vocátæ sunt, et dixérunt: Adsumus: et luxérunt ei cum jucunditáte, qui fecit illas. Hic est Deus noster, et non æstimábitur álius advérsus eum. Hic adinvénit omnem viam disciplínæ, et trádidit illam Jacob púero suo et Israël dilécto suo. Post hæc in terris visus est, et cum homínibus conversátus est.
}\switchcolumn\portugues{
\rlettrine{O}{uve,} ó Israel, os preceitos da vida; aplica bem os ouvidos, para ficares conhecendo as regras da prudência. Porque, ó Israel, estás na terra dos teus inimigos? Tu envelheceste em terra estrangeira! Tu contaminaste-te com os mortos! Tu és contada entre os que desceram ao lugar do castigo! Foi porque abandonaste a fonte da sabedoria. Ah! Se tu tivesses transitado sempre pelos caminhos de Deus, permanecerias eternamente na paz! Aprende, pois, onde estão a prudência, a virtude e a inteligência, para que ao mesmo tempo saibas onde se goza a estabilidade da vida e a sua conservação, a luz dos olhos e a paz. Quem achou a morada da sabedoria? Quem entrou nos seus tesouros? Onde estão, pois, os príncipes das nações que dominaram os animais da terra e que se recrearam, caçando as aves do céu? Onde estão os que entesouraram a prata e o ouro, em que os homens confiam, que se não esforcem incessantemente em adquiri-la? Onde estão aqueles que põem solicitamente o dinheiro em circulação em empresas raras? Foram exterminados e desceram à habitação dos mortos. Nos lugares deles surgiram outros. Eram jovens e cercados de esplendor; eram senhores da terra. Contudo ignoraram o caminho da verdadeira sabedoria e não conheceram as suas veredas! Seus filhos a não receberam; e afastaram-se até para bem longe dela. Nunca ouviram falar nela na terra de Canaan, nem a viram em Téman. Também os filhos de Agar, que procuraram uma prudência terrena, os negociantes de Merra e de Téman, os narradores de fábulas e tantos outros inventores da prudência e da inteligência ignoraram, outro tanto, o caminho da verdadeira sabedoria, e nem conheceram as suas veredas. Ó Israel, como é grande a casa do Senhor! Como é vasto o território que está sob a sua posse?! Sim! Ele é grande, ilimitado, elevado, imenso! Lá existiam aqueles afamados gigantes de elevada estatura e destros na guerra, que viveram no princípio. Mas não foi a esses que o Senhor escolheu; e nem eles acharam também o caminho da sabedoria. Sua loucura precipitou-os na morte! Quem subiu ao céu, e, encontrando aí a sabedoria, a trouxe dos astros? Quem atravessou o mar, e, tendo-a encontrado, a trouxe, de preferência ao ouro escolhido? Não há quem possa conhecer os seus caminhos e seguir as suas veredas! Só Aquele, que tudo sabe, a conhece; pois esse encontra-a em si mesmo e pela sua própria ciência: Ele, que igualmente criou a terra para sempre e a povoou com animais de todas as espécies; Ele, que manda na luz, e a luz vai; Ele, que chama a luz, e a luz obedece-Lhe, trémula; Ele, por cuja ordem as estrelas, cada uma na sua posição, espalham alegremente a luz pela terra e, chamadas por Ele, logo respondem «Eis-nos aqui», iluminando festivamente Aquele que as criou; Ele, que é o nosso Deus e outro não existe que com Ele se compare; Ele, que encontrou todos os caminhos da verdadeira ciência e que a deu a seu servo Jacob e ao seu amado Israel. Depois disto apareceu na terra e conversou com os homens.
}\switchcolumn*\latim{
\begin{nscenter} Orémus. \end{nscenter}
}\switchcolumn\portugues{
\begin{nscenter} Oremos. \end{nscenter}
}\switchcolumn*\latim{
℣. Flectámus génua.
}\switchcolumn\portugues{
℣. Ajoelhemos!
}\switchcolumn*\latim{
℟. Leváte.
}\switchcolumn\portugues{
℟. Levantai-vos!
}\switchcolumn*\latim{
Deus, qui Ecclésiam tuam semper géntium vocatióne multíplicas: concéde propítius;
ui, quos aqua baptísmatis ábluis, contínua protectióne tueáris. Per Dóminum \emph{\&c.}
}\switchcolumn\portugues{
Ó Deus, que, incessantemente, pela vocação dos gentios, dais à vossa Igreja novos filhos, dignai-Vos propício conceder a vossa contínua assistência àqueles a quem ides purificar com a água do Baptismo. Por nosso Senhor \emph{\&c.}
}\end{paracol}

\paragraphinfo{Sétima Profecia}{Ez. 37, 1-14}\label{7profecia}
\begin{paracol}{2}\latim{
\rlettrine{I}{n} diébus illis: Facta est super me manus Dómini, et edúxit me in spíritu Dómini: et dimísit me in médio campi, qui erat plenus óssibus: et circumdúxit me per ea in gyro: erant autem multa valde super fáciem campi síccaque veheménter. Et dixit ad me: Fili hóminis, putásne vivent ossa ista? Et dixi: Dómine Deus, tu nosti. Et dixit ad me: Vaticináre de óssibus istis: et dices eis: Ossa árida, audíte verbum Dómini. Hæc dicit Dóminus Deus óssibus his: Ecce, ego intromíttam in vos spíritum, et vivétis. Et dabo super vos nervos, et succréscere fáciam super vos carnes, et superexténdam in vobis cutem: et dabo vobis spíritum, et vivétis, et sciétis, quia ego Dóminus. Et prophetávi, sicut præcéperat mihi: factus est autem sónitus prophetánte me, et ecce commótio: et accessérunt ossa ad ossa, unumquódque ad junctúram suam. Et vidi, et ecce, super ea nervi et carnes ascendérunt: et exténta est in eis cutis désuper, et spíritum non habébant. Et dixit ad me: Vaticináre ad spíritum, vaticináre, fili hóminis, et dices ad spíritum: Hæc dicit Dóminus Deus: A quátuor ventis veni, spíritus, et insúffla super interféctos istos, et revivíscant. Et prophetávi, sicut præcéperat mihi: et ingréssus est in ea spíritus, et vixérunt: steterúntque super pedes suos exércitus grandis nimis valde. Et dixit ad me: Fili hóminis, ossa hæc univérsa, domus Israël est: ipsi dicunt: Aruérunt ossa nostra, et périit spes nostra, et abscíssi sumus. Proptérea vaticináre, et dices ad eos: Hæc dicit Dóminus Deus: Ecce, ego apériam túmulos vestros, et edúcam vos de sepúlcris vestris, pópulus meus: et indúcam vos in terram Israël. Et sciétis, quia ego Dóminus, cum aperúero sepúlcra vestra et edúxero vos de túmulis vestris, pópule meus: et dédero spíritum meum in vobis, et vixéritis, et requiéscere vos fáciam super humum vestram: dicit Dóminus omnípotens.
}\switchcolumn\portugues{
\rlettrine{N}{aqueles} dias, a mão do Senhor segurou-me e conduziu-me em espírito ao meio duma planície, coberta de ossos. Então, fez-me passar em torno deles, vendo eu que eram muitos e que estavam mirrados. Disse-me, pois, o Senhor: «Filho do homem, porventura poderão reviver estes ossos?». Eu respondi: «Senhor e Deus, bem o sabeis». Disse-me Ele ainda: «Profetiza a respeito desses ossos e diz-lhes: Ossos secos, ouvi a palavra do Senhor! Assim fala o Senhor e Deus: Eis que vos insuflarei o espírito e vivereis; dar-vos-ei nervos; cobrir-vos-ei de carne e de pele; dar-vos-ei o espírito. Então vivereis e sabereis que sou o Senhor». E profetizei, como me havia sido ordenado. Logo que acabei de profetizar, eis que se ouviu um grande ruído e comoção, após o que os ossos se aproximaram uns dos outros, cada um nas suas articulações. Depois olhei e vi que se revestiam de músculos, de carne e de pele, mas não possuíam ainda espírito. E o Senhor disse-me: «Fala ao espírito: Profetiza, filho do homem, e fala ao espírito: Isto diz o Senhor e Deus: Vem dos quatro ventos, ó espírito, sopra sobre estes mortos para que revivam». Eu profetizei, como o Senhor me mandara, entrando logo o espírito neles e comunicando-lhes a vida. E puseram-se de pé, como um grande exército! Continuou o Senhor a dizer-me: «Filho do homem, todos estes ossos são a casa de Israel. Eles dizem: Secaram-se os nossos ossos; acabou a nossa esperança; estamos perdidos! Profetiza-lhes, pois, e diz-lhes: Assim fala o Senhor: Eis que abrirei vossos túmulos, ó meu povo, vos tirarei deles e vos conduzirei à terra de Israel. E conhecereis, ó meu povo, que sou o Senhor, depois de ter aberto vossas sepulturas, de vos haver tirado delas e dado o meu espírito. Então vivereis e repousareis na vossa terra», diz o Senhor omnipotente.
}\switchcolumn*\latim{
\begin{nscenter} Orémus. \end{nscenter}
}\switchcolumn\portugues{
\begin{nscenter} Oremos. \end{nscenter}
}\switchcolumn*\latim{
℣. Flectámus génua.
}\switchcolumn\portugues{
℣. Ajoelhemos!
}\switchcolumn*\latim{
℟. Leváte.
}\switchcolumn\portugues{
℟. Levantai-vos!
}\switchcolumn*\latim{
Deus, qui nos ad celebrándum paschále sacraméntum utriúsque Testaménti páginis ínstruis: da nobis intellégere misericórdiam tuam; ut ex perceptióne præséntium múnerum firma sit exspectátio futurórum. Per Dóminum \emph{\&c.}
}\switchcolumn\portugues{
Ó Deus, que nas páginas dos dous Testamentos nos ensinais a celebrar dignamente o mystério pascal, concedei-nos o dom do conhecimento da vossa misericórdia, a fim de que as dádivas, que alcançamos nesta vida, nos façam ter esperança firme nos bens futuros. Por nosso Senhor \emph{\&c.}
}\end{paracol}

\paragraphinfo{Oitava Profecia}{Is. 4, 1-6}\label{8profecia}
\begin{paracol}{2}\latim{
\rlettrine{A}{pprehéndent} septem mulíeres virum unum in die illa, dicéntes: Panem nostrum comedémus et vestiméntis nostris operiémur: tantúmmodo invocétur nomen tuum super nos, aufer oppróbrium nostrum. In die illa erit germen Dómini in magnificéntia et glória, et fructus terræ súblimis, et exsultátio his, qui salváti fúerint de Israël. Et erit: Omnis, qui relíctus fúerit in Sion et resíduus in Jerúsalem, sanctus vocábitur, omnis, qui scriptus est in vita in Jerúsalem. Si ablúerit Dóminus sordes filiárum Sion, et sánguinem Jerúsalem láverit de médio ejus, in spíritu judícii et spíritu ardóris. Et creábit Dóminus super omnem locum montis Sion, et ubi invocátus est, nubem per diem, et fumum et splendórem ignis flammántis in nocte: super omnem enim glóriam protéctio. Et tabernáculum erit in umbráculum diéi ab æstu, et in securitátem et absconsiónem a túrbine et a plúvia.
}\switchcolumn\portugues{
\rlettrine{N}{aqueles} tempos, sete mulheres prenderam um só homem, dizendo-lhe: «Comeremos o nosso pão e usaremos os nossos vestidos; somente te pedimos que nos permitas usar o teu nome, a fim de sairmos do opróbrio». Naquele dia, o germe do Senhor manifestar-se-á com magnificência e glória; o fruto da terra será exaltado com honra; e aqueles que houverem sido salvos da ruína de Israel ficarão cheios de júbilo. Então, aqueles que ficaram em Sião e se espalharam por Jerusalém serão chamados santos; bem como aqueles que estão inscritos no livro da vida em Jerusalém, quando o Senhor tiver apagado as manchas das filhas de Sião e purificado Jerusalém das suas nódoas de sangue impuro, enviando o espírito de justiça e o espírito do ardor. Então o Senhor, em toda a extensão da montanha de Sião e onde seja invocado, criará uma nuvem, durante o dia, e uma chama de fogo resplandecente, durante a noite; pois protegerá de todos os lados o lugar da sua glória. E o seu tabernáculo servirá de sombra, durante o calor do dia, e de refúgio e abrigo, durante a tempestade e a chuva.
}\end{paracol}

\paragraphinfo{Trato}{Is. 5, 1 \& 2}
\begin{paracol}{2}\latim{
\rlettrine{V}{ínea} facta est dilécto in cornu, in loco úberi. ℣. Et macériam circúmdedit, et circumfódit: et plantávit víneam Sorec, et ædificávit turrim in médio ejus. ℣. Et tórcular fodit in ea: vínea enim Dómini Sábaoth domus Israël est.
}\switchcolumn\portugues{
\rlettrine{O}{} meu amado possui uma vinha em um outeiro fértil. ℣. E cercou-a com sebes e fossos, plantando nela bacelos de Soreque e construindo uma torre no meio. ℣. E construiu também aí um lagar. Ora a vinha do Senhor dos exércitos é a casa de Israel.
}\switchcolumn*\latim{
\begin{nscenter} Orémus. \end{nscenter}
}\switchcolumn\portugues{
\begin{nscenter} Oremos. \end{nscenter}
}\switchcolumn*\latim{
℣. Flectámus génua.
}\switchcolumn\portugues{
℣. Ajoelhemos!
}\switchcolumn*\latim{
℟. Leváte.
}\switchcolumn\portugues{
℟. Levantai-vos!
}\switchcolumn*\latim{
Deus, qui in ómnibus Ecclésiæ tuæ fíliis, sanctórum Prophetárum voce manifestásti, in omni loco dominatiónis tuæ, satórem te bonórum séminum, et electórum pálmitum esse cultórem: tríbue pópulis tuis, qui et vineárum apud te nómine censéntur et ségetum; ut, spinárum et tribulórum squalóre resecáto, digna efficiántur fruge fecúndi. Per Dóminum \emph{\&c.}
}\switchcolumn\portugues{
Ó Deus, que pela palavra dos vossos santos Profetas revelastes a todos os fiéis da vossa Igreja que sois Vós quem na grandeza do vosso império semeais a boa semente e cultivais as plantas escolhidas, concedei aos vossos povos (que são designados por Vós com os nomes de vinha e messe) que, depois de haverdes arrancado deles os espinhos e as silvas, que lhes envolvem o coração, se tornem capazes de produzir abundantes frutos. Por nosso Senhor \emph{\&c.}
}\end{paracol}

\paragraphinfo{Nona Profecia}{Ex. 12, 1-11}
\begin{paracol}{2}\latim{
\rlettrine{I}{n} diébus illis: Dixit Dóminus ad Móysen et Aaron in terra Ægýpti: Mensis iste vobis princípium ménsium: primus erit in ménsibus anni. Loquímini ad univérsum cœtum filiórum Israël, et dícite eis: Décima die mensis hujus tollat unusquísque agnum per famílias et domos suas. Sin autem minor est númerus, ut suffícere possit ad vescéndum agnum, assúmet vicínum suum, qui junctus est dómui suæ, juxta númerum animárum, quæ suffícere possunt ad esum agni. Erit autem agnus absque mácula, másculus, annículus: juxta quem ritum tollétis et hædum. Et servábitis eum usque ad quartam décimam diem mensis hujus: immolabítque eum univérsa multitúdo filiórum Israël ad vésperam. Et sument de sánguine ejus, ac ponent super utrúmque postem et in superlimináribus domórum, in quibus cómedent illum. Et edent carnes nocte illa assas igni, et ázymos panes cum lactúcis agréstibus. Non comedétis ex eo crudum quid nec coctum aqua, sed tantum assum igni: caput cum pédibus ejus et intestínis vorábitis. Nec remanébit quidquam ex eo usque mane. Si quid resíduum fúerit, igne comburétis. Sic autem comedétis illum: Renes vestros accingétis, et calceaménta habébitis in pédibus, tenéntes báculos in mánibus, et comedétis festinánter: est enim Phase (id est tránsitus) Dómini.
}\switchcolumn\portugues{
\rlettrine{N}{aqueles} dias, disse o Senhor, na terra do Egipto, a Moisés e a Aarão: «Que este mês seja para vós o princípio dos meses: o primeiro dos meses do ano. Falai a toda a assembleia dos filhos de Israel, dizendo: No décimo dia deste mês cada um tome um cordeiro para cada família e para cada casa. Se na casa houver poucas pessoas para comer o cordeiro, chamar-se-ão em casa do vizinho que estiver mais perto tantas pessoas quantas sejam necessárias para come: o cordeiro totalmente. Esse cordeiro será sem mancha, masculino e com um ano de idade; se porventura faltar o cordeiro, podereis tomar um cabrito com iguais condições. Guardareis esse cordeiro até ao dia décimo quarto desse mês, imolando-o, então, pela tarde, toda a multidão dos filhos de Israel. Tomar-se-á o seu sangue, com o qual pintarão as ombreiras e alizares das portas das casas em que o cordeiro for comido. Nessa mesma noite comerão com pão sem fermento e leitugas silvestres a carne, a qual será assada no lume. Não comereis desse cordeiro nada que seja cru ou cozido em água; mas todo será assado no lume. Comereis a cabeça, os pés e os intestinos, e nada deverá ficar para o dia seguinte; porém, se alguma cousa ficar, tereis o cuidado de consumi-la no fogo. Haveis de comê-lo desta maneira: rins cingidos, pés calçados e bordão na mão. Comê-lo-eis com pressa, pois é a ocasião da Páscoa, isto é, a passagem do Senhor».
}\switchcolumn*\latim{
\begin{nscenter} Orémus. \end{nscenter}
}\switchcolumn\portugues{
\begin{nscenter} Oremos. \end{nscenter}
}\switchcolumn*\latim{
℣. Flectámus génua.
}\switchcolumn\portugues{
℣. Ajoelhemos!
}\switchcolumn*\latim{
℟. Leváte.
}\switchcolumn\portugues{
℟. Levantai-vos!
}\switchcolumn*\latim{
Omnípotens sempitérne Deus, qui in ómnium óperum tuórum dispensatióne mirábilis es: intéllegant redémpti tui, non fuísse excelléntius, quod inítio factus est mundus, quam quod in fine sæculórum Pascha nostrum immolátus est Christus: Qui tecum \emph{\&c.}
}\switchcolumn\portugues{
Omnipotente e eterno Deus, que sois admirável na economia das vossas obras, concedei às criaturas, que remistes, o dom de compreenderem que a criação do mundo, no princípio dos tempos, não ultrapassa o prodígio da imolação de Cristo, nossa Páscoa, que se realizou na plenitude dos tempos. O qual, sendo Deus \emph{\&c.}
}\end{paracol}

\paragraphinfo{Décima Profecia}{Jn. 3, 1-10}
\begin{paracol}{2}\latim{
\rlettrine{I}{n} diébus illis: Factum est verbum Dómini ad Jonam Prophétam secúndo, dicens: Surge, et vade in Níniven civitátem magnam: et prǽdica in ea prædicatiónem, quam ego loquor ad te. Et surréxit Jonas, et ábiit in Níniven juxta verbum Dómini. Et Nínive erat cívitas magna itínere trium diérum. Et cœpit Jonas introíre in civitátem itínere diéi uníus: et clamávit et dixit: Adhuc quadragínta dies, et Nínive subvertétur. Et credidérunt viri Ninivítæ in Deum: et prædicavérunt jejúnium, et vestíti sunt saccis a majóre usque ad minórem. Et pervénit verbum ad regem Nínive: et surréxit de sólio suo, et abjécit vestiméntum suum a se, et indútus est sacco, et sedit in cínere. Et clamávit et dixit in Nínive ex ore regis et príncipum ejus, dicens: Hómines et juménta et boves et pécora non gustent quidquam: nec pascántur, et aquam non bibant. Et operiántur saccis hómines et juménta, et clament ad Dóminum in fortitúdine, et convertatur vir a via sua mala, et ab iniquitáte, quæ est in mánibus eórum. Quis scit, si convertátur et ignóscat Deus: et revertátur a furóre iræ suæ, et non períbimus? Et vidit Deus ópera eórum, quia convérsi sunt de via sua mala: et misértus est pópulo suo Dóminus, Deus noster.
}\switchcolumn\portugues{
\rlettrine{N}{aqueles} dias, falou o Senhor segunda vez ao Profeta Jonas, dizendo: «Ergue-te, vai à grande cidade de Nínive e prega lá o que Eu te inspirar». Jonas ergueu-se e foi a Nínive, segundo a palavra do Senhor. Ora Nínive era uma grande cidade, a três dias de caminho. Jonas entrou na cidade, caminhou durante um dia e começou a pregar, dizendo: «Ainda quarenta dias e Nínive será destruída». Então os ninivitas acreditaram em Deus, proclamaram um jejum público e vestiram-se com sacos, desde o maior ao mais pequeno dos seus habitantes. Chegando isto ao conhecimento do rei de Nínive, ergueu-se ele do trono, despiu a túnica real, vestiu um saco e sentou-se na cinza. Em seguida fez publicar em Nínive pela sua boca e pelos grandes da cidade: Que nem homens, nem animais (ou os bois ou as ovelhas) comessem, pastassem ou bebessem água; que os homens e animais se cobrissem com sacos; que os homens clamassem ao Senhor fortemente; e que toda a criatura humana abandonasse o mau caminho e a iniquidade com que suas mãos estavam manchadas. Quem sabe se Deus se não arrependerá de nos perdoar e não voltará ao furor da sua ira, de modo que todos pereçamos? E Deus viu as suas obras; viu que se convertiam e afastavam dos maus caminhos; e teve piedade do seu povo: o Senhor, nosso Deus».
}\switchcolumn*\latim{
\begin{nscenter} Orémus. \end{nscenter}
}\switchcolumn\portugues{
\begin{nscenter} Oremos. \end{nscenter}
}\switchcolumn*\latim{
℣. Flectámus génua.
}\switchcolumn\portugues{
℣. Ajoelhemos!
}\switchcolumn*\latim{
℟. Leváte.
}\switchcolumn\portugues{
℟. Levantai-vos!
}\switchcolumn*\latim{
Deus, qui diversitátem géntium in confessióne tui nóminis adunásti: da nobis et velle et posse, quæ prǽcipis; ut, pópulo ad æternitátem vocáto, una sit fides méntium et píetas actiónum. Per Dóminum \emph{\&c.}
}\switchcolumn\portugues{
Ó Deus, que reunistes na confissão do vosso nome povos tão diferentes, concedei-nos a graça de podermos e querermos cumprir tudo o que mandais, a fim de que o vosso povo, que é chamado a gozar a glória eterna, tenha a mesma fé no espírito e a mesma santidade nas acções. Por nosso Senhor \emph{\&c.}
}\end{paracol}


\paragraphinfo{Décima Primeira Profecia}{Dt. 31, 22-30}\label{11profecia}
\begin{paracol}{2}\latim{
\rlettrine{I}{n} diébus illis: Scripsit Móyses canticum, et dócuit fílios Israël. Præcepítque Dóminus Josue, fílio Nun, et ait: Confortáre, et esto robústus: tu enim introdúces fílios Israël in terram, quam pollícitus sum, et ego ero tecum. Postquam ergo scripsit Móyses verba legis hujus in volúmine, atque complévit: præcépit Levítis, qui portábant arcam fœderis Dómini, dicens: Tóllite librum istum, et pónite eum in látere arcæ fœderis Dómini, Dei vestri: ut sit ibi contra te in testimónium. Ego enim scio contentiónem tuam et cérvicem tuam duríssimam. Adhuc vivénte me et ingrediénte vobíscum, semper contentióse egístis contra Dóminum: quanto magis, cum mórtuus fúero? Congregáte ad me omnes majóres natu per tribus vestras, atque doctóres, et loquar audiéntibus eis sermónes istos, et invocábo contra eos cœlum et terram. Novi enim, quod post mortem meam iníque agétis et declinábitis cito de via, quam præcépi vobis: et occúrrent vobis mala in extrémo témpore, quando fecéritis malum in conspéctu Dómini, ut irritétis eum per ópera mánuum vestrárum. Locútus est ergo Móyses, audiénte univérso cœtu Israël, verba cárminis hujus, et ad finem usque complévit.
}\switchcolumn\portugues{
\rlettrine{N}{aqueles} dias, Moisés escreveu um cântico e ensinou-o aos filhos de Israel. E o Senhor ordenou a Josué, filho de Num, dizendo-lhe: «Sê forte e tem coragem, pois conduzirás os filhos de Israel ao país que lhes prometi com juramento. Eu serei contigo». Logo que Moisés acabou de escrever em um livro as palavras desta lei, ordenou aos Levitas, que levavam a Arca da Aliança do Senhor, o seguinte: «Tomai este livro da Lei e colocai-o ao lado da Arca da Aliança do Senhor, vosso Deus, para que seja ali testemunho contra vós, pois sei que vosso espírito é rebelde e vossa cabeça dura! Se, enquanto estou vivo e no meio de vós, sempre tendes sido rebeldes contra o Senhor, quanto mais quando tiver morrido! Reuni junto de mim todos os anciãos e doutores das vossas tribos, e pronunciarei na sua presença este cântico e invocarei contra eles o testemunho do céu e da terra, pois sei que, depois da minha morte, procedereis iniquamente e vos afastareis do caminho que vos tracei. Mas a infelicidade vos assaltará, no decorrer dos tempos, por haverdes pecado contra o Senhor, irritando-O com vossas obras». Pronunciou, então, Moisés, diante de toda a assembleia de Israel, as palavras deste cântico até ao fim:
}\end{paracol}

\paragraphinfo{Trato}{Dt. 32, 1-4}
\begin{paracol}{2}\latim{
\rlettrine{A}{tténde,} cœlum, et loquar: et áudiat terra verba ex ore meo. ℣. Exspectétur sicut plúvia elóquium meum: et descéndant sicut ros verba mea. ℣. Sicut imber super gramen et sicut nix super fænum: quia nomen Dómini invocábo. ℣. Date magnitúdinem Deo nostro: Deus, vera ópera ejus, et omnes viæ ejus judícia. ℣. Deus fidélis, in quo non est iníquitas: justus et sanctus Dóminus.
}\switchcolumn\portugues{
\rlettrine{O}{uvi,} ó céus, pois falarei; e que a terra ouça as palavras da minha bocal Que minhas palavras sejam esperadas com ansiedade, como a chuva para os campos sequiosos! Que minhas palavras caiam na terra, como o orvalho! Como as chuvas na relva e como a neve no feno, pois invocarei o nome do Senhor. Aclamai o nosso Deus, porque as suas obras são verdadeiras e as suas leis são justas. Deus é a verdade; n’Ele não há injustiça: o Senhor é justo e santo!
}\switchcolumn*\latim{
\begin{nscenter} Orémus. \end{nscenter}
}\switchcolumn\portugues{
\begin{nscenter} Oremos. \end{nscenter}
}\switchcolumn*\latim{
℣. Flectámus génua.
}\switchcolumn\portugues{
℣. Ajoelhemos!
}\switchcolumn*\latim{
℟. Leváte.
}\switchcolumn\portugues{
℟. Levantai-vos!
}\switchcolumn*\latim{
Deus, celsitúdo humílium et fortitúdo rectórum, qui per sanctum Móysen, púerum tuum, ita erudíre pópulum tuum sacri cárminis tui decantatióne voluísti, ut illa legis iterátio fíeret étiam nostra diréctio: éxcita in omnem justificatárum géntium plenitúdinem poténtiam tuam, et da lætítiam, mitigándo terrórem; ut, ómnium peccátis tua remissióne delétis, quod denuntiátum est in ultiónem, tránseat in salútem. Per Dóminum \emph{\&c.}
}\switchcolumn\portugues{
Ó Deus, exaltação dos humildes e fortaleza dos justos, que quisestes instruir-nos com o sagrado cântico do vosso servo Moisés, o qual é ao mesmo tempo uma repetição da vossa lei e uma regra de conduta, dignai-Vos mostrar o vosso poder a todas as nações, e, dissipando os seus terrores, espalhai nelas a alegria, a fim de que, sendo apagadas as culpas de todas elas pela vossa misericórdia, o castigo anunciado se torne em salvação. Por nosso Senhor \emph{\&c.}
}\end{paracol}

\paragraphinfo{Décima Segunda Profecia}{Dn. 3, 1-24}
\begin{paracol}{2}\latim{
\rlettrine{I}{n} diébus illis: Nabuchodónosor rex fecit státuam áuream, altitúdine cubitórum sexagínta, latitúdine cubitórum sex, et státuit eam in campo Dura provínciæ Babylónis. Itaque Nabuchodónosor rex misit ad congregándos sátrapas, magistrátus, et júdices, duces, et tyránnos, et præféctos, omnésque príncipes regiónum, ut convenírent ad dedicatiónem státuæ, quam eréxerat Nabuchodónosor rex. Tunc congregáti sunt sátrapæ, magistrátus, et júdices, duces, et tyránni, et optimátes, qui erant in potestátibus constitúti, et univérsi príncipes regiónum, ut convenírent ad dedicatiónem státuæ, quam eréxerat Nabuchodónosor rex. Stabant autem in conspéctu státuæ, quam posúerat Nabuchodónosor rex, et præco clamábat valénter: Vobis dícitur populis, tríbubus et linguis: In hora, qua audiéritis sónitum tubæ, et fístulæ, et cítharæ, sambúcæ, et psaltérii, et symphóniæ, et univérsi géneris musicórum, cadéntes adoráte státuam áuream, quam constítuit Nabuchodónosor rex. Si quis autem non prostrátus adoráverit, eádem hora mittétur in fornácem ignis ardéntis. Post hæc ígitur statim ut audiérunt omnes pópuli sónitum tubæ, fístulæ, et cítharæ, sambúcæ, et psaltérii, et symphóniæ, et omnis géneris musicórum, cadéntes omnes pópuli, tribus et linguæ adoravérunt státuam auream, quam constitúerat Nabuchodónosor rex. Statímque in ipso témpore accedéntes viri Chaldǽi accusavérunt Judǽos, dixerúntque Nabuchodónosor regi: Rex, in ætérnum vive: tu, rex, posuísti decrétum, ut omnis homo, qui audiérit sónitum tubæ, fístulæ, et cítharæ, sambúcæ, et psaltérii, et symphóniæ, et univérsi géneris musicórum, prostérnat se et adóret státuam áuream: si quis autem non prócidens adoráverit, mittátur in fornácem ignis ardéntis. Sunt ergo viri Judǽi, quos constituísti super ópera regiónis Babylónis, Sidrach, Misach et Abdénago: viri isti contempsérunt, rex, decrétum tuum: deos tuos non colunt, et státuam áuream, quam erexísti, non adórant. Tunc Nabuchodónosor in furóre et in ira præcépit, ut adduceréntur Sidrach, Misach et Abdénago: qui conféstim addúcti sunt in conspéctu regis. Pronuntiánsque Nabuchodónosor rex, ait eis: Veréne, Sidrach, Misach et Abdénago, deos meos non cólitis, et státuam áuream, quam constítui, non adorátis? Nunc ergo si estis parati, quacúmque hora audieritis sonitum tubæ, fístulæ, cítharæ, sambúcæ, et psaltérii, et symphóniæ, omnísque géneris musicórum, prostérnite vos et adoráte státuam, quam feci: quod si non adoravéritis, eadem hora mittémini in fornácem ignis ardéntis; et quis est Deus, qui erípiet vos de manu mea? Respondéntes Sidrach, Misach et Abdénago, dixérunt regi Nabuchodónosor: Non opórtet nos de hac re respóndere tibi. Ecce enim, Deus noster, quem cólimus, potest erípere nos de camíno ignis ardéntis, et de mánibus tuis, o rex, liberáre. Quod si nolúerit, notum sit tibi; rex, quia deos tuos non cólimus et státuam áuream, quam erexísti, non adorámus. Tunc Nabuchodónosor replétus est furóre, et aspéctus faciéi illíus immutátus est super Sidrach, Misach et Abdénago, et præcépit, ut succenderétur fornax séptuplum, quam succéndi consuéverat. Et viris fortíssimis de exércitu suo jussit, ut, ligátis pédibus Sidrach, Misach et Abdénago, mítterent eos in fornácem ignis ardéntis. Et conféstim viri illi vincti, cum braccis suis et tiáris et calceaméntis et véstibus, missi sunt in médium fornácis ignis ardéntis: nam jússio regis urgébat: fornax autem succénsa erat nimis. Porro viros illos, qui míserant Sidrach, Misach et Abdénago, interfécit flamma ignis. Viri autem hi tres, id est, Sidrach, Misach et Abdénago, cecidérunt in médio camíno ignis ardéntis colligáti. Et ambulábant in médio flammæ laudántes Deum, et benedicéntes Dómino.
}\switchcolumn\portugues{
\rlettrine{N}{aqueles} dias, o rei Nabucodonosor mandou fabricar uma estátua de ouro de sessenta côvados de altura e seis de largura, erigindo-a na planície de Dura, na província da Babilónia. Então, o rei Nabucodonosor convocou os sátrapas, os magistrados e os juízes, os capitães, os governadores, os presidentes e os príncipes das províncias, para assistirem à dedicação da estátua, que o rei erigira. Reuniram-se, pois, os sátrapas, os magistrados e os juízes, os capitães, os governadores, os presidentes e os grandes, revestidos de poder, e os príncipes das províncias, para assistirem à dedicação da estátua que Nabucodonosor levantara. Estando, então, todos de pé, em redor da estátua, publicava o pregoeiro com voz forte: «Faz-se saber a vós todos, povos, tribos e pessoas de todas as línguas, que, desde o momento em que ouvirdes o som da trombeta, da flauta, da cítara, da sambuca, do saltério, da sanfonina e de toda a espécie de instrumentos, vos prostrareis, adorando a estátua de ouro que Nabucodonosor mandou erigir; e todo aquele que se não prostrar e não adorar a estátua será lançado imediatamente em uma fornalha de fogo ardente!». Portanto, logo que os povos ouviram o som da trombeta, da flauta, da cítara, da sambuca, do saltério, da sanfonina e de todo o género de instrumentos músicos, prostrados todos os povos, tribos e nações de todas as línguas, adoraram a estátua de ouro. Mas naquele mesmo momento aproximaram-se do rei Nabucodonosor os Caldeus, acusando os judeus e dizendo: «Para sempre vivas, ó rei! Publicaste um decreto, ordenando que todo o homem que ouvisse o som da trombeta, da flauta, da cítara, da sambuca, do saltério, da sanfonina e de toda a espécie de instrumentos músicos se prostrasse e adorasse a estátua de ouro; e quem o não fizesse fosse lançado na fornalha de fogo ardente. Ora há três judeus, a quem nomeaste intendentes da província da Babilónia, quais são Sidrac, Misac e Abdénago, que desprezaram; ó rei, o teu decreto, não prestando culto aos deuses, nem adorando a estátua de ouro que mandaste erigir!». Então Nabucodonosor, irritado e furioso, mandou vir à sua presença Sidrac, Misac e Abdénago, os quais, efectivamente, compareceram. E disse-lhes o rei: «Porventura é verdade que vós, Sidrac, Misac e Abdénago, não prestastes culto aos deuses, nem adorastes a estátua de ouro que mandei levantar? Assim, pois, se estais dispostos a obedecer-me, logo que ouçais o som da trombeta, da flauta, da cítara, da sambuca, da sanfonina e de toda a espécie de instrumentos, prostrai-vos e adorai a estátua que erigi; e, se a não adorardes, sereis precipitados nesse mesmo instante em uma fornalha de fogo ardente! Qual o Deus que poderá livrar-vos das minhas mãos?”. Responderam então Sidrac, Misac e Abdénago ao rei Nabucodonosor: «A esse respeito não é necessário, ó rei, responder-vos, pois o Rei a quem adoramos pode arrebatar-nos da fornalha de fogo ardente e livrar-nos, ó rei, das tuas mãos. E, mesmo que o não queira fazer, saberás, ó rei, que não renderemos culto aos deuses, nem adoraremos a estátua de ouro que erigiste!». Nabucodonosor enfureceu-se, e, fitando Sidrac, Misac e Abdénago com o rosto alterado e com os olhos chispando ira, mandou acender a fogueira de fogo sete vezes mais forte do que o costume, ordenando aos soldados mais fortes da sua guarda que amarrassem de pés e mãos Sidrac, Misac e Abdénago e os lançassem nas chamas da fornalha. Logo estes três homens foram amarrados e lançados no meio das chamas mesmo com suas roupas, turbantes, calçado e outras vestes, pois a ordem do rei era instante. A fornalha estava extremamente chamejante! Ora aqueles homens que lançaram no fogo Sidrac, Misac e Abdénago foram logo abrasados, enquanto que os três, isto é, Sidrac, Misac e Abdénago, caíram amarrados no meio das chamas; mas logo se ergueram, e passeavam, louvavam Deus e bendiziam o Senhor no meio das chamas!
}\switchcolumn*\latim{
\begin{nscenter} Orémus. \end{nscenter}
}\switchcolumn\portugues{
\begin{nscenter} Oremos. \end{nscenter}
}\switchcolumn*\latim{
℣. Flectámus génua.
}\switchcolumn\portugues{
℣. Ajoelhemos!
}\switchcolumn*\latim{
℟. Leváte.
}\switchcolumn\portugues{
℟. Levantai-vos!
}\switchcolumn*\latim{
Omnípotens sempitérne Deus, spes única mundi, qui Prophetárum tuorum præcónio præséntium témporum declarásti mystéria: auge pópuli tui vota placátus; quia in nullo fidélium, nisi ex tua inspiratióne, provéniunt quarúmlibet increménta virtútum. Per Dóminum \emph{\&c.}
}\switchcolumn\portugues{
Ó Deus omnipotente e eterno, única esperança do mundo, que pela boca dos vossos Profetas anunciastes os mystérios destes tempos, dignai-Vos propício aumentar o xervor dos votos do vosso povo, pois nenhum dos vossos fiéis poderá progredir nas virtudes sem a vossa inspiração. Por nosso Senhor \emph{\&c.}
}\end{paracol}

\paragraph{Bênção da Pia Baptismal}\label{piabaptismal}

\textit{Dirige-se agora a Procissão para a Pia Baptismal. Durante o percurso canta-se:}

\paragraphinfo{Trato}{Sl. 41, 2-4}
\begin{paracol}{2}\latim{
\rlettrine{S}{icut} cervus desíderat ad fontes aquárum: iía desíderat ánima mea ad te, Deus. ℣. Sitívit ánima mea ad Deum vivum: quando véniam, et apparébo ante fáciem Dei? ℣. Fuérunt mihi lácrimæ meæ panes die ac nocte, dum dícitur mihi per síngulos dies: Ubi est Deus tuus?
}\switchcolumn\portugues{
\rlettrine{A}{ssim} como o veado sequioso procura as fontes das águas, assim a minha alma suspira por Vós, ó meu Deus! ℣. Minha alma tem sede de Deus vivo! Quando irei e aparecerei perante Deus! ℣. Minhas lágrimas têm sido dia e noite o meu alimento, quando continuamente me perguntam: onde está o teu Deus?
}\end{paracol}

\paragraph{Oração}
\begin{paracol}{2}\latim{
\rlettrine{O}{mnípotens} sempitérne Deus, réspice propítius ad devotiónem pópuli renascéntis, qui, sicut cervus, aquárum tuárum éxpetit fontem: et concéde propítius; ut fídei ipsíus sitis, baptísmatis mystério, ánimam corpúsque sanctíficet. Per Dóminum \emph{\&c.} ℟. Amen.
}\switchcolumn\portugues{
\slettrine{Ó}{} Deus omnipotente e sempiterno, dignai-Vos olhar benignamente para a piedade do vosso povo, que quer renascer e que, sequioso como o veado, procura a fonte das vossas águas; dignai-Vos permitir que esta sede do dom da Fé lhe santifique a alma e o corpo pelo mystério do Baptismo. Por nosso Senhor \emph{\&c.} ℟. Amen.
}\end{paracol}

\paragraph{Oração}
\begin{paracol}{2}\latim{
\rlettrine{O}{mnípotens} sempitérne Deus, adésto magnæ pietátis tuæ mystériis, adésto sacraméntis: et ad recreándos novos pópulos, quos tibi fons baptísmatis párturit, spíritum adoptiónis emítte; ut, quod nostræ humilitátis geréndum est ministério, virtútis tuæ impleátur efféctu. Per Dóminum nostrum Jesum Christum, Fílium tuum: Qui tecum vivit et regnat \emph{\&c.}
}\switchcolumn\portugues{
\rlettrine{D}{eus} omnipotente e eterno, sede atento a estes profundos mystérios da vossa bondade e a estes augustos Sacramentos; e, para regenerar os novos povos, que a Fonte Baptismal vai dar à luz, enviai o Espírito de adopção, a fim de que aquilo que nós praticamos por meio do nosso humilde ministério seja eficazmente realizado por efeito do vosso poder. Por nosso Senhor \emph{\&c.}
}\end{paracol}

\paragraph{Prefácio}
\begin{paracol}{2}\latim{
\rlettrine{V}{ere} dignum et justum est, æquum et salutáre, nos tibi semper et ubíque grátias ágere: Dómine sancte, Pater omnípotens, ætérne Deus. Qui invisíbili poténta sacramentórum tuórum mirabíliter operáris efféctum: Et licet nos tantis mystériis exsequéndis simus indígni, tu tamen grátiæ tuæ dona non déserens, etiam ad nostras preces aures tuæ pietátis inclínas. Deus, cujus Spíritus super aquas in-ter ipsa mundi primórdia ferebátur: ut jam tunc virtútem sanctificatiónis aquárum natúra concíperet. Deus, qui, nocéntis mundi crímina per aquas ábluens, regeneratiónis spéciem in ipsa dilúvii effusióne signásti: ut, uníus ejusdémque eleménti mystério, et finis esset vítiis et orígo virtútibus. Réspice, Dómine, in fáciem Ecclésiæ tuæ, et multíplica in ea regeneratiónes tuas, qui grátiæ tuæ affluéntis ímpetu lætíficas civitátem tuam: fontémque baptísmatis áperis toto orbe terrárum géntibus innovándis: ut, tuæ majestátis império, sumat Unigéniti tui grátiam de Spíritu Sancto.
}\switchcolumn\portugues{
\slettrine{É}{} realmente digno e justo, racional e salutar dar-Vos graças sempre e em todos os lugares, Senhor santo, Pai omnipotente, Deus eterno, que com poder invisível operais os admiráveis efeitos dos vossos Sacramentos; e, ainda que sejamos indignos de desempenhar tão elevados mystérios, contudo, como os dons da vossa graça são inesgotáveis, dignai-Vos ouvir propiciamente as nossas orações. Ó Deus, cujo Espírito no princípio do mundo passava sobre as águas, a fim de que, desde então, este elemento possuísse a virtude de santificar as almas; ó Deus, que, lavando com as águas os pecados do mundo criminoso, fizestes ver no dilúvio uma imagem de regeneração, de modo que um só e o mesmo elemento, por um mystério admirável, exterminava os vícios e despertava as virtudes: lançai, Senhor, os vossos olhares benignos sobre a vossa Igreja; multiplicai nela os vossos novos filhos: ó Vós, que encheis de alegria a vossa cidade santa com o ímpeto da vossa graça; e abri neste dia para toda a terra a Fonte Baptismal para regenerar todos os povos, a fim de que, segundo a vontade da vossa divina majestade, esta Igreja receba a graça de vosso Filho Unigénito pelo Espírito Santo.
}\switchcolumn*\latim{
\emph{Hic Sacerdos in modum crucis aquam dividit manu extensa, quam statim linteo extergit, dicens:}
}\switchcolumn\portugues{
\emph{O Sacerdote divide a água em forma de Cruz:}
}\switchcolumn*\latim{
Qui hanc aquam, regenerándis homínibus præparátam, arcána sui núminis admixtióne fœcúndet: ut, sanctificatióne concépta, ab immaculáto divíni fontis útero, in novam renáta creatúram, progénies cæléstis emérgat: Et quos aut sexus in córpore aut ætas discérnit in témpore, omnes in unam páriat grátia mater infántiam. Procul ergo hinc, jubénte te, Dómine, omnis spíritus immundus abscédat: procul tota nequítia diabólicæ fraudis absístat. Nihil hic loci hábeat contráriæ virtútis admíxtio: non insidiándo circúmvolet: non laténdo subrépat: non inficiéndo corrúmpat.
}\switchcolumn\portugues{
Que este Espírito se digne fecundar, pela acção misteriosa da sua divindade, esta água, preparada para a regeneração humana, a fim de que, por uma conceição santificante, renasça no seio imaculado da divina fonte uma nova criatura, uma raça celestial; e que a graça, como uma mãe, fecunde para a mesma vida aqueles filhos que, agora, se distinguem no corpo, pelo sexo, e no tempo, pela idade. Ordenai, pois, Senhor, que todo o espírito de impureza saia desta água, bem como toda a malícia diabólica: que o poder do inimigo não tenha parte alguma nestas águas, nem gire em torno delas, nera nelas se introduza, pretendendo corrompê-las.
}\switchcolumn*\latim{
\emph{Aquam manu tangit.}
}\switchcolumn\portugues{
\emph{Toca com a mão na água.}
}\switchcolumn*\latim{
Sit hæc sancta et ínnocens creatúra líbera ab omni impugnatóris incúrsu, et totíus nequítiæ purgáta discéssu. Sit fons vivus, aqua regénerans, unda puríficans: ut omnes hoc lavácro salutífero diluéndi, operánte in eis Spíritu Sancto, perféctæ purgatiónis indulgéntiam consequántur.
}\switchcolumn\portugues{
Que esta criatura santa e inocente seja livre de qualquer incursão do inimigo e purificada, sendo dela expulsa toda a malícia; que seja fonte da vida água regeneradora e fonte purificadora a fim de que todos aqueles que sejam lavados neste banho salutar alcancem, por obra do Espírito Santo, a graça duma pureza perfeita.
}\switchcolumn*\latim{
\emph{Facit tres cruces super Fontem, dicens:}
}\switchcolumn\portugues{
\emph{Faz três vezes o sinal da Cruz:}
}\switchcolumn*\latim{
Unde benedíco te, creatúra aquæ, per Deum \cruz vivum, per Deum \cruz verum, per Deum \cruz sanctum: per Deum, qui in princípio verbo separávit ab árida: cujus Spíritus super te ferebátur.
}\switchcolumn\portugues{
Eu te abençoo, criatura de água, em nome de Deus \cruz vivo, em nome de Deus \cruz verdadeiro, em nome de Deus \cruz santo: em nome de Deus, que, no princípio do mundo, com uma só palavra te separou da terra, e cujo Espírito passava sobre ti.
}\switchcolumn*\latim{
\emph{Hic manu aquam dividit et effundit eam versus quatuor mundi partes, dicens:}
}\switchcolumn\portugues{
\emph{Divide a água com a mão, deitando quatro Porções dela para fora da Pia:}
}\switchcolumn*\latim{
Qui te de paradísi fonte manáre fecit, et in quátuor flumínibus totam terram rigáre præcépit. Qui te in desérto amáram, suavitáte índita, fecit esse potábilem, et sitiénti pópulo de petra prodúxit. Be \cruz nedíco te et per Jesum Christum, Fílium ejus únicum, Dominum nostrum: qui te in Cana Galilǽæ signo admirábili, sua poténtia convértit in vinum. Qui pédibus super te ambulávit: et a Joánne in Jordáne in te baptizátus est. Qui te una cum sánguine de látere suo prodúxit: et discípulis suis jussit, ut credéntes baptizaréntur in te, dicens: Ite, docéte omnes gentes, baptizántes eos in nómine Patris, et Fílii, et Spíritus Sancti.
}\switchcolumn\portugues{
Em nome de Deus, que te fez brotar da fonte do paraíso, e, dividindo-te em quatro rios, mandou que regasses toda a terra; em nome de Deus, que no deserto, quando eras amarga, te tornou potável e mais tarde te fez sair do rochedo para saciar um Povo sequioso. Eu te \cruz abençoo, também, em nome de Jesus Cristo, Filho Unigénito de Deus, nosso Senhor, que milagrosamente, em Caná, na Galileia, por meio dum admirável prodígio do seu poder, te mudou em vinho; que caminhou a pé enxuto sobre ti; que em ti foi baptizado no Jordão por João; que te fez sair juntamente com seu sangue do seu lado; que mandou aos discípulos que em ti fossem baptizados aqueles que acreditassem, dizendo-lhes: «Ide, ensinai todos os povos, baptizando-os em nome do Pai, e do Filho, e do Espírito Santo».
}\switchcolumn*\latim{
\emph{Halat ter in aquam in modum crucis, dicens:}
}\switchcolumn\portugues{
\emph{O Celebrante sopra três vezes sobre a água:}
}\switchcolumn*\latim{
Tu has símplices aquas tuo ore benedícito: ut præter naturálem emundatiónem, quam lavándis possunt adhibére corpóribus, sint etiam purificándis méntibus efficáces.
}\switchcolumn\portugues{
Abençoai, Vós, ó Deus, com vossa boca, estas águas puras, a fim de que, além da virtude que possuem de lavar os corpos, recebam também a graça de purificar as almas,
}\switchcolumn*\latim{
\emph{Hic Sacerdos paululum demittit Cereum in aquam: et resumens tonum Præfationis, dicit:}
}\switchcolumn\portugues{
\emph{O Sacerdote põe três vezes o Círio Pascal na água:}
}\switchcolumn*\latim{
Descéndat in hanc plenitúdinem fontis, virtus Spíritus Sancti.
}\switchcolumn\portugues{
Que a virtude do Espírito Santo desça sobre toda a água desta fonte.
}\switchcolumn*\latim{
\emph{Et deinde sufflans ter in aquam}
}\switchcolumn\portugues{
\emph{O Celebrante sopra três vezes a água:}
}\switchcolumn*\latim{
Totamque hujus aquæ substántiam regenerándi fecúndet efféctu.
}\switchcolumn\portugues{
Que ela (a virtude do Espírito Santo) torne esta água fecunda e capaz de regenerar.
}\switchcolumn*\latim{
\emph{Hic tollitur Cereus de aqua, et prosequitur:}
}\switchcolumn\portugues{
\emph{O Celebrante retira o Círio da água:}
}\switchcolumn*\latim{
Hic ómnium peccatórum máculæ deleántur: hic natúra ad imáginem tuam cóndita, et ad honórem sui reformáta princípii, cunctis vetustátis squalóribus emundétur: ut omnis homo, sacraméntum hoc regeneratiónis ingréssus, in veræ innocéntiæ novam infántiam renascátur. Per Dóminum nostrum Jesum Christum, Fílium tuum: Qui ventúrus est judicáre vivos et mórtuos, et sǽculum per ignem.
}\switchcolumn\portugues{
Que aqui se apaguem todas as nódoas dos pecados; que aqui a nossa natureza, criada à vossa imagem e restituída à dignidade da sua origem, seja purificada de todas as máculas do «homem velho», a fim de que todo o homem que receber este Sacramento de regeneração renasça para a verdadeira inocência duma nova infância. Por nosso Senhor Jesus Cristo, vosso Filho, que há-de vir a julgar os vivos e os mortos e destruir este mundo pelo fogo.
}\switchcolumn*\latim{
℟. Amen.
}\switchcolumn\portugues{
℟. Amen.
}\switchcolumn*\latim{
\emph{Deinde per assistentes Sacerdotes spargitur de ipsa aqua benedícta super pópulum. Et interim unus ex ministris ecclesiæ accipit in vase aliquo de eadem aqua ad aspergendum in domibus, et aliis locis. His peractis, Sacerdos, qui benedicit Fontem, infundit de Oleo Catechumenorum in aquam in modum crucis, intellegibili voce dicens:}
}\switchcolumn\portugues{
\emph{Faz-se, então, a Aspersão do Clero e dos fiéis. Depois o Celebrante deita na água os Santos óleos, dizendo:}
}\switchcolumn*\latim{
Sanctificétur et fœcundétur fons iste Oleo salútis renascéntibus ex eo, in vitam ætérnam.
}\switchcolumn\portugues{
Que esta Fonte seja santificada e se torne fecunda com a infusão deste Óleo de salvação, para dar a vida eterna àqueles que renascerem do seu seio.
}\switchcolumn*\latim{
℟. Amen.
}\switchcolumn\portugues{
℟. Amen.
}\switchcolumn*\latim{
Infúsio Chrísmatis Dómini nostri Jesu Christi, et Spíritus Sancti Parácliti, fiat in nómine sanctæ Trinitátis.
}\switchcolumn\portugues{
Que a infusão do Crisma de nosso Senhor Jesus Cristo e do Espírito Santo se opere em nome da Santíssima Trindade.
}\switchcolumn*\latim{
℟. Amen.
}\switchcolumn\portugues{
℟. Amen.
}\switchcolumn*\latim{
Commíxtio Chrísmatis sanctificatiónis, et Olei unctiónis, et Aquæ baptísmatis, páriter fiat in nómine Pa \cruz tris, et Fí \cruz lii, et Spíritus \cruz Sancti.
}\switchcolumn\portugues{
Que a mistura do Crisma da santificação e do Óleo da unção com a Água Baptismal se opere em nome do \cruz Pai, e do \cruz Filho, e do Espírito \cruz Santo.
}\switchcolumn*\latim{
℟. Amen.
}\switchcolumn\portugues{
℟. Amen.
}\end{paracol}

\paragraphinfo{Ladainha dos Santos}{Página \pageref{ladainhasantos}}

\subsection{Missa de Sábado Santo}

\paragraph{Oração}
\begin{paracol}{2}\latim{
\rlettrine{D}{eus,} qui hanc sacratíssimam noctem glória Domínicæ Resurrectiónis illústras: consérva in nova famíliæ tuæ progénie adoptiónis spíritum, quem dedísti; ut, córpore et mente renováti, puram tibi exhíbeant servitútem. Per eúndem Dóminum nostrum \emph{\&c.}
}\switchcolumn\portugues{
\slettrine{Ó}{} Deus, que iluminais esta santíssima noite com os esplendores da Ressurreição do Senhor, conservai nos novos filhos da vossa família o Espírito de adopção, que lhes concedestes, a fim de que, renovados de corpo e de espírito, Vos sirvam cheios de pureza. Pelo mesmo nosso Senhor \emph{\&c.}
}\end{paracol}

\paragraphinfo{Epístola}{Cl. 3, 1-4}
\begin{paracol}{2}\latim{
Léctio Epístolæ beáti Pauli Apóstoli ad Colossénses.
}\switchcolumn\portugues{
Lição da Ep.ª do B. Ap.º Paulo aos Colossenses.
}\switchcolumn*\latim{
\rlettrine{F}{ratres:} Si consurrexístis cum Christo, quæ sursum sunt quǽrite, ubi Christus est in déxtera Dei sedens: quæ sursum sunt sápite, non quæ super terram. Mórtui enim estis, et vita vestra est abscóndita cum Christo in Deo. Cum Christus appáruerit, vita vestra: tunc et vos apparébitis cum ipso in glória.
}\switchcolumn\portugues{
\rlettrine{M}{eus} irmãos: Se ressuscitastes com Cristo, procurai as cousas que são do céu, onde Cristo está assentado à direita de Deus. Aspirai às cousas do céu e não às da terra, pois estais mortos e a vossa vida está oculta em Deus com Cristo. Quando Cristo, que é a vossa vida, aparecer, então também aparecereis com Ele na glória.
}\switchcolumn*\latim{
Allelúja. Allelúja. Allelúja. ℣. \emph{Ps. 117, 1} Confitémini Dómino, quóniam bonus: quóniam in sǽculum misericordia ejus.
}\switchcolumn\portugues{
Aleluia! Aleluia! Aleluia! ℣. \emph{Sl. 117, 1} Glorificai o Senhor, pois a sua misericórdia é eterna!
}\end{paracol}

\paragraphinfo{Trato}{Sl. 116, 1-2}
\begin{paracol}{2}\latim{
\rlettrine{L}{audáte} Dóminum, omnes gentes: et collaudáte eum, omnes pópuli. ℣. Quóniam confirmáta est super nos misericórdia ejus: et véritas Dómini manet in ætérnum.
}\switchcolumn\portugues{
\slettrine{Ó}{} nações, louvai todas o Senhor! Anunciai todos o Senhor, ó povos! ℣. Sua misericórdia para connosco confirmou-se e a fidelidade do Senhor permanecerá eternamente.
}\end{paracol}

\paragraphinfo{Evangelho}{Mt. 28, 1-7}
\begin{paracol}{2}\latim{
\cruz Sequéntia sancti Evangélii secúndum Matthǽum.
}\switchcolumn\portugues{
\cruz Continuação do santo Evangelho segundo S. Mateus.
}\switchcolumn*\latim{
\blettrine{V}{éspere} autem sábbati, quæ luce scit in prima sábbati, venit María Magdaléne et áltera María vidére sepúlcrum. Et ecce, terræmótus factus est magnus. Angelus enim Dómini descéndit de cœlo: et accédens revólvit lápidem, et sedébat super eum: erat autem aspéctus ejus sicut fulgur: et vestiméntum ejus sicut nix. Præ timóre autem ejus extérriti sunt custódes, et facti sunt velut mórtui. Respóndens autem Angelus, dixit muliéribus: Nolíte timére vos: scio enim, quod Jesum, qui crucifíxus est, quǽritis: non est hic: surréxit enim, sicut dixit. Veníte, et vidéte locum, ubi pósitus erat Dóminus. Et cito eúntes, dícite discípulis ejus, quia surréxit: et ecce, præcédit vos in Galilǽam: ibi eum vidébitis. Ecce, prædíxi vobis.
}\switchcolumn\portugues{
\blettrine{A}{pós} as vésperas de sábado, ao romper da aurora do primeiro dia depois de sábado, Maria Madalena e a outra Maria foram visitar o sepulcro. Houve então um grande tremor de terra: e um Anjo do Senhor desceu do céu, aproximou-se do túmulo, revolveu a pedra e assentou-se sobre ela. Seu rosto tinha o brilho de um relâmpago e os seus vestidos eram brancos, como a neve. Os guardas, logo que o viram, encheram-se de tal pavor, que ficaram como mortos! E o Anjo, começando a falar, disse às mulheres: «Não tenhais medo! Sei que procurais Jesus, que foi crucificado. Ele não está aqui, porque ressuscitou, como dissera! Vinde e vede o lugar onde o Senhor havia sido colocado! Ide depressa dizer aos seus discípulos que Ele ressuscitou e que vos precederá na Galileia, onde o vereis. Eis o que antecipadamente vos anuncio».
}\end{paracol}

\paragraph{Secreta}
\begin{paracol}{2}\latim{
\rlettrine{S}{uscipe,} quǽsumus, Dómine, preces pópuli tui, cum oblatiónibus hostiárum: ut paschálibus initiá tam ystériis, ad æternitátis nobis medélam, te operánte, profíciant. Per Dóminum \emph{\&c.}
}\switchcolumn\portugues{
\rlettrine{A}{ceitai,} Senhor, Vos imploramos, as preces do vosso povo, unidas à oblação destas hóstias, a fim de que, santificadas pelo mystério pascal, nos sirvam, por efeito da vossa graça, de remédio para a eternidade. Por nosso Senhor \emph{\&c.}
}\end{paracol}

\paragraphinfo{Comunicantes}{diz-se até ao Sábado seguinte}
\begin{paracol}{2}\latim{
\rlettrine{C}{ommunicántes,} et noctem sacratíssimam celebrántes Resurrectiónis Dómini nostri Jesu Christi secúndum carnem: sed et memóriam venerántes, in primis gloriósæ semper Vírginis Maríæ, Genetrícis ejusdem Dei et Dómini nostri Jesu Christi: \emph{\&c.}
}\switchcolumn\portugues{
\rlettrine{U}{nidos} em uma mesma comunhão e celebrando a noite (ou o dia) sacratíssima da Ressurreição, segundo a carne, de Nosso Senhor Jesus Cristo, veneramos em primeiro lugar a memória da gloriosa sempre Virgem Maria, Mãe do mesmo Deus e Nosso Senhor Jesus Cristo \emph{\&c.} (tudo o mais como ordinariamente).
}\end{paracol}

\paragraphinfo{Hanc ígitur}{diz-se até ao Sábado seguinte}
\begin{paracol}{2}\latim{
\rlettrine{H}{anc} ígitur oblatiónem servitútis nostræ, sed et cunctæ famíliæ tuæ, quam tibi offérimus pro his quoque, quos regeneráre dignátus es ex aqua et Spíritu Sancto, tríbuens eis remissiónem ómnium peccatórum, quǽsumus, Dómine, ut placátus accípias: diésque nostros in tua pace dispónas, atque ab ætérna damnatióne nos éripi, et in electórum tuórum júbeas grege numerári. Jungit manus. Per Christum, Dóminum nostrum \emph{\&c.}
}\switchcolumn\portugues{
\rlettrine{P}{or} este motivo, Senhor, Vos rogamos, dignai-Vos receber favoravelmente este sacrifício, que eu, vosso indigno servo, e toda vossa família, Vos oferecemos hoje, especialmente por aqueles que Vos dignastes regenerar pela água e pelo Espírito Santo, concedendo-lhes a remissão de todos os pecados: dai-nos o gozo da vossa paz nos nossos dias desta vida, livrai-nos da condenação eterna e admiti-nos ao número dos vossos escolhidos. Por nosso Senhor \emph{\&c.}
}\end{paracol}

\textit{E o restante como no Ordinário da Missa, menos o Agnus Dei. Após a Comunhão, cantam-se as:}

\subsubsection{Vésperas}

\paragraph{Antífona}
\begin{paracol}{2}\latim{
Allelúja, allelúja, allelúja!
}\switchcolumn\portugues{
Aleluia, aleluia, aleluia!
}\end{paracol}

\paragraphinfo{Salmo 116}{Sl. 116, 1-2}
\begin{paracol}{2}\latim{
\rlettrine{L}{audáte} Dóminum, omnes gentes: laudáte eum, omnes pópuli. Quóniam confirmáta est super nos misericórdia ejus: et véritas Dómini manet in ætérnum. Glória Patri \emph{\&c.}
}\switchcolumn\portugues{
\slettrine{Ó}{} nações, louvai todas o Senhor! Anunciai todos o Senhor, ó povos! Sua misericórdia para connosco confirmou-se e a fidelidade do Senhor permanecerá eternamente. Glória ao Pai \emph{\&c.}
}\end{paracol}

\textit{Repete-se a Antífona anterior.}

\paragraph{Antífona}
\begin{paracol}{2}\latim{
\rlettrine{V}{éspere} autem sábbati, quæ lucéscit in prima sábbati, venit María Magdaléne, et áltera María, vidére sepúlchrum, allelúja.
}\switchcolumn\portugues{
\rlettrine{A}{pós} as vésperas de sábado, ao romper da aurora do primeiro dia depois de sábado, Maria Madalena e a outra Maria foram visitar o sepulcro. Aleluia.
}\end{paracol}

\paragraphinfo{Magnificat}{Lc. 1, 46-55}
\begin{paracol}{2}\latim{
\rlettrine{M}{agníficat} anima mea Dóminum: Et exsultávit spíritus meus in Deo, salutári meo. Quia respéxit humilitátem ancíllæ suæ: ecce enim, ex hoc beátam me dicent omnes generatiónes. Quia fecit mihi magna qui potens est: et sanctum nomen ejus. Et misericórdia ejus a progénie in progénie timéntibus eum. Fecit poténtiam in bráchio suo: dispérsit supérbos mente cordis sui. Depósuit poténtes de sede, et exaltávit húmiles. Esuriéntes implévit bonis: et dívites dimísit inánes. Suscépit Israël, púerum suum, recordátus misericórdiæ suæ. Sicut locútus est ad patres nostros, Abraham, et sémini ejus in sǽcula. Glória Patri \emph{\&c.}
}\switchcolumn\portugues{
\rlettrine{A}{} minha alma glorifica o Senhor. E o meu espírito exultou em Deus, meu salvador. Visto que Ele olhou para a humildade da sua serva, desde agora todas as gerações me chamarão bem-aventurada. Pois o Omnipotente operou em mim grandes maravilhas: e o seu nome é santo. Sua misericórdia espalha-se de geração em geração sobre os que o temem. Manifestou-se o poder do seu braço: dispersou os soberbos, cujo coração é cheio de orgulho. Depôs os poderosos dos seus tronos e ergueu os humildes. Saciou de bens os que tinham fome e deixou as mãos vazias aos ricos. Recebeu Israel como seu servo, lembrando-se da sua misericórdia: Tal como anunciara a nossos pais: a Abraão e à sua descendência para sempre.
Glória ao Pai \emph{\&c.}
}\end{paracol}

\textit{Repete-se a Antífona anterior.}

\paragraph{Oração}
\begin{paracol}{2}\latim{
\rlettrine{S}{píritum} nobis, Dómine, tuæ caritátis infúnde: ut, quos sa
craméntis paschálibus satiásti, tua fácias pietáte concórdes. Per Dóminum \emph{\&c.}
}\switchcolumn\portugues{
\rlettrine{I}{nfundi} em nós, Senhor, o espírito da vossa caridade, a fim de que aqueles que foram alimentados com o sacramento pascal permaneçam sempre, pela vossa bondade, em perfeita concórdia. Por nosso Senhor \emph{\&c.}
}\end{paracol}

\subsectioninfo{Terça-feira da l.ª Semana da Quaresma}{Estação em Santa Anastácia}

\paragraphinfo{Intróito}{Sl. 89, 1 \& 2}
\begin{paracol}{2}\latim{
\rlettrine{D}{ómine,} refúgium factus es nobis a generatióne et progénie: a sǽculo et in sǽculum tu es. \emph{Ps. ibid., 2} Priúsquam montes fíerent, aut formarétur terra et orbis: a sǽculo et usque in sǽculum tu es Deus.
℣. Gloria Patri \emph{\&c.}
}\switchcolumn\portugues{
\rlettrine{S}{enhor,} tendes sido o nosso refúgio de geração em geração. Vós existis desde toda a eternidade e existireis sempre! \emph{Sl. ibid., 2} Antes que os montes fossem criados e que criásseis a terra e o mundo, Vós éreis já o nosso Deus, desde a eternidade até à eternidade.
℣. Glória ao Pai \emph{\&c.}
}\end{paracol}

\paragraph{Oração}
\begin{paracol}{2}\latim{
\rlettrine{R}{éspice,} Dómine, famíliam tuam: et præsta; ut apud te mens nostra tuo desidério fúlgeat, quæ se carnis maceratióne castígat. Per Dóminum nostrum \emph{\&c.}
}\switchcolumn\portugues{
\rlettrine{O}{lhai} benigno para a vossa família, Senhor, e fazei que a nossa alma, que se castiga com a maceração da carne, suspire ardentemente em desejos de Vos possuir. Por nosso Senhor \emph{\&c.}
}\end{paracol}

\paragraphinfo{Epístola}{Is. 55, 6-11}
\begin{paracol}{2}\latim{
Léctio Isaíæ Prophétæ.
}\switchcolumn\portugues{
Lição do Profeta Isaías.
}\switchcolumn*\latim{
\rlettrine{I}{n} diébus illis: Locútus est Isaías Prophéta, dicens: Quǽrite Dóminum, dum inveníri potest: invocáte eum, dum prope est. Derelínquat ímpius viam suam, et vir iníquus cogitatiónes suas, et revertátur ad Dóminum: et miserébitur ejus, et ad Deum nostrum: quóniam multus est ad ignoscéndum. Non enim cogitationes meæ cogitatiónes vestræ: neque viæ vestræ viæ meæ, dicit Dóminus. Quia sicut exaltántur cœli a terra, sic exaltátæ sunt viæ meæ a viis vestris, et cogitatiónes meæ a cogitatiónibus vestris. Et quómodo descéndit imber et nix de cœlo, et illuc ultra non revértitur, sed inébriat terram, et infúndit eam, et germináre eam facit, et dat semen serénti, et panem comedénti: sic erit verbum meum, quod egrediétur de ore meo: non revertétur ad me vácuum, sed fáciet quæcúmque volui, et prosperábitur in his, ad quæ misi illud: ait Dóminus omnípotens.
}\switchcolumn\portugues{
\rlettrine{N}{aqueles} dias, o Profeta Isaías falou assim: «Procurai o Senhor, enquanto pode ser encontrado; invocai-O, enquanto está perto. Que o ímpio abandone o seu caminho e o homem iníquo os seus pensamentos. Que se convertam ao Senhor, nosso Deus, o qual se apiedará deles e lhes perdoará generosamente. Na Verdade, os meus pensamentos não são os vossos; Meus caminhos não são os vossos, diz o Senhor. Porquanto, assim como os céus são elevados acima da terra, assim os meus caminhos são elevados acima dos vossos e os meus pensamentos acima dos vossos. Assim como a chuva e a neve caem do céu, e lá não voltam mais sem regar a terra e fazê-la produzir, dando a semente ao lavrador e o pão ao que come, assim também será a palavra, que sai da minha boca: não tornará a mim sem produzir efeito, e fará tudo o que eu quiser, tornando prósperas as coisas para que a destinei: diz o Senhor omnipotente».
}\end{paracol}

\paragraphinfo{Gradual}{Sl. 140, 2}
\begin{paracol}{2}\latim{
\rlettrine{D}{irigátur} orátio mea sicut incénsum in conspéctu tuo, Dómine, ℣. Elevátio mánuum meárum sacrifícium vespertínum.
}\switchcolumn\portugues{
\rlettrine{E}{leve-se} a minha oração, Senhor, até à vossa presença, como o incenso. ℣. Que a elevação das minhas mãos Vos seja agradável, como o sacrifício vespertino.
}\end{paracol}

\paragraphinfo{Evangelho}{Mt. 21, 10-17}
\begin{paracol}{2}\latim{
\cruz Sequéntia sancti Evangélii secúndum Matthǽum.
}\switchcolumn\portugues{
\cruz Continuação do santo Evangelho segundo S. Mateus.
}\switchcolumn*\latim{
\blettrine{I}{n} illo témpore: Cum intrásset Jesus Jerosólymam, commóta est univérsa cívitas, dicens: Quis est hic? Pópuli autem dicébant: Hic est Jesus Prophéta a Názareth Galilǽæ. Et intrávit Jesus in templum Dei, et ejiciébat omnes vendéntes, et eméntes in templo; et mensas nummulariórum et cáthedras vendéntium colúmbas evértit: et dicit eis: Scriptum est: Domus mea domus oratiónis vocábitur: vos autem fecístis illam spelúncam latrónum. Et accessérunt ad eum cæci et claudi in templo: et sanávit eos. Vidéntes autem príncipes sacerdótum et scribæ mirabília, quæ fecit, et púeros clamantes in templo, et dicéntes: Hosánna fílio David: indignáti sunt, et dixérunt ei: Audis, quid isti dicunt? Jesus autem dixit eis: Utique. Numquam legístis: Quia ex ore infántium et lacténtium perfecísti laudem? Et relíctis illis, ábiit foras extra civitátem in Bethániam: ibíque mansit.
}\switchcolumn\portugues{
\blettrine{N}{aquele} tempo, entrando Jesus em Jerusalém, toda a cidade se alvoroçou, dizendo: «Quem é este?». E o povo respondia: «É Jesus, o profeta de Nazaré, de Galileia». E Jesus entrou no templo de Deus, expulsando todos os que lá vendiam e compravam e deitando por terra as mesas dos que negociavam e as cadeiras dos que vendiam os pombos, dizendo: «Minha casa é chamada casa de oração; e vós tornaste-la em um covil de ladrões!...». Então vieram ao templo ter com Ele cegos e coxos, curando-os. Porém, os príncipes dos sacerdotes e os escribas, ouvindo narrar os milagres que Ele fazia e os meninos a cantar louvores, dizendo: «Hosana ao Filho de David», indignaram-se e disseram-Lhe: «Ouvis o que dizem?». Jesus respondeu-lhes: «Sim; pois não lestes: «Da boca dos meninos e das crianças de peito sairão louvores perfeitos?» E, tendo-os deixado, saiu da cidade e foi para Betânia, onde permaneceu.
}\end{paracol}

\paragraphinfo{Ofertório}{Sl. 30, 15-16}
\begin{paracol}{2}\latim{
\rlettrine{I}{n} te sperávi, Dómine; dixi: Tu es Deus meus, in mánibus tuis témpora mea.
}\switchcolumn\portugues{
\rlettrine{E}{m} Vós, Senhor, pus a minha confiança. Eu disse: Vós sois o meu Deus; nas vossas mãos ponho o meu futuro.
}\end{paracol}

\paragraph{Secreta}
\begin{paracol}{2}\latim{
\rlettrine{O}{blátis,} quǽsumus, Dómine, placáre munéribus: et a cunctis nos defénde perículis. Per Dóminum \emph{\&c.}
}\switchcolumn\portugues{
\rlettrine{S}{enhor,} Vos suplicamos, deixai-Vos aplacar com estas ofertas e livrai-nos de todos os perigos. Por nosso Senhor \emph{\&c.}
}\end{paracol}

\paragraphinfo{Comúnio}{Sl. 4, 2}
\begin{paracol}{2}\latim{
\rlettrine{C}{um} invocárem te, exaudísti me, Deus justítiæ meæ: in tribulatióne dilatásti me: miserére mihi, Dómine, et exáudi oratiónem meam.
}\switchcolumn\portugues{
\qlettrine{Q}{uando} Vos invoquei, ó Deus da minha justiça, atendestes-me: e consolastes-me na angústia. Tende piedade de mim, Senhor, e ouvi a minha oração.
}\end{paracol}

\paragraph{Postcomúnio}
\begin{paracol}{2}\latim{
\qlettrine{Q}{uǽsumus,} omnípotens Deus: ut illíus salutáris capiámus efféctum, cujus per hæc mystéria pignus accépimus. Per Dóminum \emph{\&c.}
}\switchcolumn\portugues{
\rlettrine{D}{eus} omnipotente, Vos suplicamos, concedei-nos a salvação, de que já alcançámos o penhor nestes mistérios. Por nosso Senhor \emph{\&c.}
}\end{paracol}

\paragraph{Oração sobre o povo}
\begin{paracol}{2}\latim{
\begin{nscenter} Orémus. \end{nscenter}
}\switchcolumn\portugues{
\begin{nscenter} Oremos. \end{nscenter}
}\switchcolumn*\latim{
Humiliáte cápita vestra Deo.
}\switchcolumn\portugues{
Inclinai as vossas cabeças diante de Deus.
}\switchcolumn*\latim{
Acéndant ad te, Dómine, preces nostræ: et ab Ecclésia tua cunctam repélle nequítiam. Per Dóminum nostrum \emph{\&c.}
}\switchcolumn\portugues{
Que as nossas orações, Senhor, subam até Vós e que toda a espécie do mal seja afastada da vossa Igreja. Por nosso Senhor \emph{\&c.}
}\end{paracol}

\subsectioninfo{Quinta-feira da Semana da Paixão}{Estação em Santo Apolinário}

\paragraphinfo{Intróito}{Dn. 3, 31}
\begin{paracol}{2}\latim{
\rlettrine{O}{mnia,} quæ fecísti nobis, Dómine, in vero judício fecísti: quia peccávimus tibi, et mandátis tuis non obœdívimus: sed da glóriam nómini tuo, et fac nobíscum secúndum multitúdinem misericórdiæ tuæ. \emph{Ps. 118, 1} Beáti immaculáti in via: qui ámbulant in lege Dómini.
}\switchcolumn\portugues{
\qlettrine{Q}{uantos} males caíram sobre nós, Senhor, foi com verdadeira justiça que o fizestes, pois pecamos e não obedecemos aos vossos mandamentos; mas agora glorificai o vosso nome e usai para connosco de misericórdia. \emph{Ps. 118, 1} Bem-aventurados aqueles cuja vida é perfeita e que praticam a Lei do Senhor.
}\end{paracol}

\paragraph{Oração}
\begin{paracol}{2}\latim{
\rlettrine{P}{ræsta,} quǽsumus, omnípotens Deus: ut dígnitas condiciónis humánæ, per immoderántiam sauciáta, medicinális parsimóniæ stúdio reformétur. Per Dóminum \emph{\&c.}
}\switchcolumn\portugues{
\rlettrine{P}{ermiti,} Vos suplicamos, ó Deus omnipotente, que a dignidade da natureza humana, corrompida pela intemperança, seja restabelecida pela prática da salutar abstinência. Por nosso Senhor \emph{\&c.}
}\end{paracol}

\paragraphinfo{Epístola}{Dn. 3, 25 \& 34-45}
\begin{paracol}{2}\latim{
Léctio Daniélis Prophétæ.
}\switchcolumn\portugues{
Lição do Profeta Daniel.
}\switchcolumn*\latim{
\rlettrine{I}{n} diébus illis: Orávit Azarías Dóminum, dicens: Dómine, Deus noster: ne, quǽsumus, tradas nos in perpétuum propter nomen tuum, et ne díssipes testaméntum tuum: neque áuferas misericórdiam tuam a nobis propter Abraham diléctum tuum, et Isaac servum tuum, et Israël sanctum tuum: quibus locútus es, póllicens, quod multiplicáres semen eórum sicut stellas cœli et sicut arénam, quæ est in lítore maris: quia, Dómine, imminúti sumus plus quam omnes gentes, sumúsque húmiles in univérsa terra hódie propter peccáta nostra. Et non est in témpore hoc princeps, et dux, et prophéta, neque holocáustum, neque sacrifícium, neque oblátio, neque incénsum, neque locus primitiárum coram te, ut póssimus inveníre misericórdiam tuam: sed in ánimo contríto et spíritu humilitátis suscipiámur. Sicut in holocáusto aríetum et taurórum, et sicut in mílibus agnórum pínguium: sic fiat sacrifícium nostrum in conspéctu tuo hódie, ut pláceat tibi: quóniam non est confúsio confidéntibus in te. Et nunc séquimur te in toto corde, et timémus te, et quǽrimus fáciem tuam. Ne confúndas nos: sed fac nobíscum juxta mansuetúdinem tuam et secúndum multitúdinem misericórdiæ tuæ. Et érue nos in mirabílibus tuis, et da glóriam nómini tuo, Dómine: et confundántur omnes, qui osténdunt servis tuis mala, confundántur in omni poténtia tua: et robur eórum conterátur: et sciant, quia tu es Dóminus, Deus solus, et gloriósus super orbem terrárum, Dómine, Deus noster.
}\switchcolumn\portugues{
\rlettrine{N}{aqueles} dias, Azarias orou ao Senhor e disse: «Senhor, nosso Deus, Vos pedimos, nos não abandoneis perpetuamente, por causa do vosso nome, e não quebreis a vossa aliança, nem nos deixeis sem a vossa misericórdia, por causa de Abraão, vosso muito amado, de Isaque, vosso servo, e de Israel, vosso santo, aos quais prometestes multiplicar-lhes as suas descendências, como as estrelas do céu e como os grãos de areia da praia. Porquanto, Senhor, estamos reduzidos a um número menor que todos os outros povos e estamos hoje humilhados perante toda a terra, por causa dos nossos pecados. Presentemente, não temos nem príncipe, nem chefe, nem profeta, nem holocausto, nem sacrifício, nem oblação, nem incenso, nem lugar onde possamos oferecer as nossas primícias, a fim de podermos alcançar a vossa misericórdia. Dignai-Vos, pois, ouvir-nos; atendei ao nosso coração contrito e ao nosso espírito humilhado. Que este sacrifício, oferecido hoje em vossa presença, Vos seja agradável, como um holocausto de carneiros e de touros, como a oferta de mil cordeiros gordos; porquanto aqueles que em Vós confiam não serão iludidos. Agora Vos seguimos do íntimo do coração; agora Vos tememos; agora procuramos a vossa presença. Não nos confundais, mas tratai-nos conforme a vossa mansidão e segundo a grandeza da vossa misericórdia! Livrai-nos, Senhor, com vossos prodígios e dai glória ao vosso nome! Que sejam confundidos aqueles que maltratam os vossos servos! Sim! Que eles sejam confundidos pela vossa omnipotência! Que sua força seja aniquilada; e que conheçam que sois o Senhor, único Deus e o glorioso Senhor em toda a terra. Sim, Senhor, nosso Deus».
}\end{paracol}

\paragraphinfo{Gradual}{Sl. 95, 8-9}
\begin{paracol}{2}\latim{
\rlettrine{T}{óllite} hóstias, et introíte in átria ejus: adoráte Dóminum in aula sancta ejus. ℣. \emph{Ps. 28, 9} Revelávit Dóminus condénsa: et in templo ejus omnes dicent glóriam.
}\switchcolumn\portugues{
\rlettrine{L}{evai} convosco as vossas ofertas e entrai nos átrios: adorai o Senhor no seu templo santo. ℣. \emph{Sl. 28, 9} O Senhor descobrirá o que está oculto, e todos O glorificarão no seu templo.
}\end{paracol}

\paragraphinfo{Evangelho}{Lc. 7, 36-50}\label{evangelhopaixaoquinta}
\begin{paracol}{2}\latim{
\cruz Sequéntia sancti Evangélii secúndum Lucam.
}\switchcolumn\portugues{
\cruz Continuação do santo Evangelho segundo S. Lucas.
}\switchcolumn*\latim{
\blettrine{I}{n} illo témpore: Rogábat Jesum quidam de pharisǽis, ut manducáret cum illo. Et ingréssus domum pharisǽi, discúbuit. Et ecce múlier, quæ erat in civitáte peccátrix, ut cognóvit, quod accubuísset in domo pharisǽi, áttulit alabástrum unguénti: et stans retro secus pedes ejus, lácrimis coepit rigáre pedes ejus, et capíllis cápitis sui tergébat, et osculabátur pedes ejus, et unguénto ungébat. Videns autem pharisǽus, qui vocáverat eum, ait intra se, dicens: Hic si esset Prophéta, sciret útique, quæ et qualis est múlier, quæ tangit eum: quia peccátrix est. Et respóndens Jesus, dixit ad illum: Simon, hábeo tibi áliquid dícere. At ille ait: Magíster, dic. Duo debitóres erant cuidam fæneratóri: unus debébat denários quingéntos, et álius quinquagínta. Non habéntibus illis, unde rédderent, donávit utrísque. Quis ergo eum plus díligit? Respóndens Simon, dixit: Æstimo, quia is, cui plus donávit. At ille dixit ei: Recte judicásti. Et convérsus ad mulíerem, dixit Simóni: Vides hanc mulíierem? Intrávi in domum tuam, aquam pédibus meis non dedísti: hæc autem lácrimis rigávit pedes meos et capíllis suis tersit. Osculum mihi non dedísti: hæc autem, ex quo intrávit, non cessávit osculári pedes meos. Oleo caput meum non unxísti: hæc autem unguénto unxit pedes meos. Propter quod dico tibi: Remittúntur ei peccáta multa, quóniam diléxit multum. Cui autem minus dimíttitur, minus díligit. Dixit autem ad illam: Remittúntur tibi peccáta. Et cœpérunt, qui simul accumbébant, dícere intra se: Quis est hic, qui étiam peccáta dimíttit? Dixit autem ad mulíerem: Fides tua te salvam fecit: vade in pace.
}\switchcolumn\portugues{
\blettrine{N}{aquele} tempo, um fariseu pediu a Jesus que comesse com ele à sua mesa. Havendo, pois, Jesus entrado em sua casa, sentou-se à mesa. E eis que uma mulher pecadora que havia na cidade, sabendo que Jesus estava à mesa, em casa do fariseu, trouxe um vaso de alabastro, cheio de perfumes, e, ficando por detrás d’Ele, a seus pés, começou a regar-Lhe os pés com suas lágrimas e a enxugar-lhos com os cabelos da cabeça. E beijava os pés de Jesus, ungindo-os com os perfumes! Vendo isto o fariseu que havia convidado Jesus, pensava intimamente o seguinte: «Se este homem fosse profeta, certamente saberia quem e que espécie de mulher é esta que o toca, pois é uma pecadora». Então Jesus, tomando a palavra, disse: «Simão, tenho certa cousa a dizer-te». Este respondeu: «Falai, Mestre!». «Um credor — disse Jesus — tinha dous devedores: um devia quinhentos dinheiros e o outro cinquenta. Não tendo eles com que pagar, a ambos o credor quitou as dívidas. Qual deles, pois, deverá amá-lo mais?». Simão respondeu: «Aquele, penso eu, a quem o credor perdoou mais». Jesus disse-lhe: «Falaste bem». Voltando-se, então, para a mulher, disse a Simão: «Vês esta mulher? Entrei em tua casa, e não deitaste água sobre os meus pés; mas ela tem-os lavado com suas lágrimas e enxugado com seus cabelos. Tu não me deste o ósculo; porém, esta, desde que entrou, não cessou de beijar os meus pés. Tu não ungiste a minha cabeça; contudo ela ungiu com perfumes os meus pés. Pelo que te digo: os seus numerosos pecados são-lhe perdoados, porque ela amou muito; mas a quem pouco se perdoa é porque pouco ama!». Depois disse à mulher: «Os teus pecados são-te perdoados». Então aqueles que estavam à mesa começaram a dizer uns aos outros: «Quem é este que também perdoa pecados?». E Jesus disse à mulher: «A tua fé salvou-te, vai em paz».
}\end{paracol}

\paragraphinfo{Ofertório}{Sl. 136, 1}
\begin{paracol}{2}\latim{
\rlettrine{S}{uper} flúmina Babylónis illic sédimus et flévimus: dum recordarémur tui, Sion.
}\switchcolumn\portugues{
\rlettrine{A}{o} pé das margens dos rios da Babilónia nos sentámos e chorámos, recordando-nos com saudade de Sião!
}\end{paracol}

\paragraph{Secreta}
\begin{paracol}{2}\latim{
\rlettrine{D}{ómine,} Deus noster, qui in his pótius creatúris, quas ad fragilitátis nostræ subsídium condidísti, tuo quoque nómini múnera jussísti dicánda constítui: tríbue, quǽsumus; ut et vitæ nobis præséntis auxílium et æternitátis effíciant sacraméntum. Per Dóminum \emph{\&c.}
}\switchcolumn\portugues{
\rlettrine{S}{enhor,} nosso Deus, que quisestes que as criaturas destinadas por Vós para alimento da nossa fraqueza fossem também imoladas em honra do vosso nome, dignai-Vos permitir, Vos suplicamos, que elas nos sirvam de auxílio na vida presente e ao mesmo tempo de sacramento de salvação eterna. Por nosso Senhor \emph{\&c.}
}\end{paracol}

\paragraphinfo{Comúnio}{Sl. 118, 49-50}
\begin{paracol}{2}\latim{
\rlettrine{M}{eménto} verbi tui servo tuo, Dómine, in quo mihi spem dedísti: hæc me consoláta est in humilitáte mea.
}\switchcolumn\portugues{
\rlettrine{L}{embrai-Vos,} Senhor, da promessa que fizestes ao vosso servo e com a qual me enchestes de esperança: ela me tem consolado na humilhação.
}\end{paracol}

\paragraph{Postcomúnio}
\begin{paracol}{2}\latim{
\qlettrine{Q}{uod} ore súmpsimus, Dómine, pura mente capiámus: et de munere temporáli, fiat nobis remédium sempitérnum. Per Dóminum \emph{\&c.}
}\switchcolumn\portugues{
\rlettrine{F}{azei,} Senhor, que guardemos com o coração puro aquilo que a nossa boca recebeu; e que este dom temporal se torne para nós em remédio eterno. Por nosso Senhor \emph{\&c.}
}\end{paracol}

\paragraph{Oração sobre o povo}
\begin{paracol}{2}\latim{
\begin{nscenter} Orémus. \end{nscenter}
}\switchcolumn\portugues{
\begin{nscenter} Oremos. \end{nscenter}
}\switchcolumn*\latim{
Humiliáte cápita vestra Deo.
}\switchcolumn\portugues{
Inclinai as vossas cabeças diante de Deus.
}\switchcolumn*\latim{
Esto, quǽsumus, Dómine, propítius plebi tuæ: ut, quæ tibi non placent, respuéntes; tuórum pótius repleántur delectatiónibus mandatórum. Per Dóminum \emph{\&c.}
}\switchcolumn\portugues{
Senhor, Vos imploramos, sede favorável ao vosso povo, a fim de que, repelindo tudo o que Vos desagrada, lhe advenham as delícias que resultam da prática dos vossos mandamentos. Por nosso Senhor \emph{\&c.}
}\end{paracol}

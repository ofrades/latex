\subsectioninfo{Sábado da 4.ª Semana da Quaresma}{Estação em S. Nicolau no Cárcere}

\paragraphinfo{Intróito}{Is. 55, 1}
\begin{paracol}{2}\latim{
\rlettrine{S}{itiéntes,} venite ad aquas, dicit Dóminus: et qui non habétis prétium, veníte et bíbite cum lætítia. \emph{Ps. 77, 1} Atténdite, pópule meus, legem meam: inclináte aurem vestram in verba oris mei.
℣. Gloria Patri \emph{\&c.}
}\switchcolumn\portugues{
\slettrine{Ó}{} vós, que estais sedentos, diz o Senhor, vinde às águas: e, ainda que não tenhais dinheiro, vinde e bebei com alegria. Escutai a minha lei, ó meu povo! \emph{Sl. 77, 1} Inclinai os vossos ouvidos às palavras que saem da minha boca.
℣. Glória ao Pai \emph{\&c.}
}\end{paracol}

\paragraph{Oração}
\begin{paracol}{2}\latim{
\rlettrine{F}{iat,} Dómine, quǽsumus, per grátiam tuam fructuósus nostræ devotiónis afféctus: quia tunc nobis próderunt suscépta jejúnia, si tuæ sint plácita pietáti. Per Dóminum nostrum \emph{\&c.}
}\switchcolumn\portugues{
\rlettrine{S}{enhor,} permiti pela vossa graça que o afecto da nossa devoção se torne frutuoso, porquanto só nos serão proveitosos os nossos jejuns se forem agradáveis à vossa bondade. Por nosso Senhor \emph{\&c.}
}\end{paracol}

\paragraphinfo{Epístola}{Is. 49, 8-15}
\begin{paracol}{2}\latim{
Léctio Isaíæ Prophétæ.
}\switchcolumn\portugues{
Lição do Profeta Isaías.
}\switchcolumn*\latim{
\rlettrine{H}{æc} dicit Dóminus: In témpore plácito exaudívi te, et in die salútis auxiliátus sum tui: et servávi te, et dedi te in fœdus pópuli, ut suscitáres terram, et possidéres hereditátes dissipátas: ut díceres his, qui vincti sunt: Exíte: et his, qui in ténebris: Revelámini. Super vias pascéntur, et in ómnibus planis páscua eórum. Non esúrient neque sítient, et non percútiet eos æstus et sol: quia miserátor eórum reget eos, et ad fontes aquárum potábit eos. Et ponam omnes montes meos in viam, et sémitæ meæ exaltabúntur. Ecce, isti de longe vénient, ei ecce illi ab aquilóne et mari, et isti de terra austráli. Laudáte, cœli, et exsúlta, terra, jubiláte, montes, laudem: quia consolátus est Dóminus pópulum suum, et páuperum suórum miserébitur. Et dixit Sion: Derelíquit me Dóminus, et Dóminus oblítus est mei. Numquid oblivísci potest múlier infántem suum, ut non misereátur fílio uteri sui? et si illa oblíta fúerit, ego tamen non oblivíscar tui, dicit Dóminus omnípotens.
}\switchcolumn\portugues{
\rlettrine{A}{ssim} fala o Senhor: «No tempo favorável eu te ouvi; e no dia da salvação eu te assisti. Guardei-te e destinei-te para seres a aliança do meu povo, restaurares a terra, possuíres as heranças dissipadas, dizeres aos cativos, que estão a ferros: «Sois livres», e aos que estão nas trevas: «Vinde à luz». Apascentar-se-ão livremente em todos os caminhos e todas as planícies lhes servirão de pastagens. Não terão fome, nem sede; os não incomodará o calor, nem os queimará o sol; pois Aquele, que é misericordioso, os guiará e conduzirá às águas frescas para beberem. Então aplanarei os montes e torná-los-ei em caminhos e as veredas serão alteadas. Eis que chegam de longe: uns virão do norte e do mar e outros das terras do meio-dia. Ó céus, rejubilai! Ó terra, exultai! Ó montanhas, cantai hinos de alegria, pois o Senhor consolou o seu povo e teve compaixão dos seus pobres. Entretanto disse Sião: O Senhor abandonou-me; o Senhor esqueceu-se de mim! Porventura pode uma mãe esquecer o seu filho, de modo que não tenha piedade do filho das suas entranhas? Pois? ainda que esta o esqueça, Eu te não esquecerei»: diz o Senhor omnipotente.
}\end{paracol}

\paragraphinfo{Gradual}{Sl. 9, 14 \& 1-2}
\begin{paracol}{2}\latim{
\rlettrine{T}{ibi,} Dómine, derelíctus est pauper: pupíllo tu eris adjútor. ℣. Ut quid, Dómine, recessísti longe, déspicis in opportunitátibus, in tribulatióne? dum supérbit ímpius, incénditur pauper.
}\switchcolumn\portugues{
\rlettrine{A}{} Vós, Senhor, se abandona o pobre. Sois o protector do órfão. ℣. Porque, Senhor, Vos afastais para longe e Vos escondeis no tempo da tribulação, quando o ímpio se orgulha e persegue o pobre com ardor?
}\end{paracol}

\paragraphinfo{Evangelho}{Jo. 8, 12-20}
\begin{paracol}{2}\latim{
\cruz Sequéntia sancti Evangélii secúndum Joánnem.
}\switchcolumn\portugues{
\cruz Continuação do santo Evangelho segundo S. Lucas.
}\switchcolumn*\latim{
\blettrine{I}{n} illo témpore: Locútus est Jesus turbis Judæórum, dicens: Ego sum lux mundi: qui séquitur me, non ámbulat in ténebris, sed habébit lumen vitæ. Dixérunt ergo ei pharisǽi: Tu de te ipso testimónium pérhibes: testimónium tuum non est verum. Respóndit Jesus et dixit eis: Et si ego testimónium perhíbeo de meípso, verum est testimónium meum: quia scio, unde veni et quo vado: vos autem nescítis, unde vénio aut quo vado. Vos secúndum carnem judicátis: ego non júdico quemquam: et si júdico ego, judícium meum verum est, quia solus non sum: sed ego et, qui misit me, Pater. Et in lege vestra scriptum est, quia duórum hóminum testimónium verum est. Ego sum, qui testimónium perhíbeo de meípso: et testimónium pérhibet de me, qui misit me, Pater. Dicébant ergo ei: Ubi est Pater tuus? Respóndit Jesus: Neque me scitis neque Patrem meum: si me sciretis, fórsitan et Patrem meum scirétis. Hæc verba locútus est Jesus in gazophylácio, docens in templo: et nemo apprehéndit eum, quia necdum vénerat hora ejus.
}\switchcolumn\portugues{
\blettrine{N}{aquele} tempo, Jesus falou à turba dos judeus, dizendo: «Eu sou a luz do mundo. Aquele que me seguir não andará nas trevas, mas terá a luz da vida». Disseram-Lhe então os fariseus: «Tu dás testemunho de ti mesmo; o teu testemunho não é verdadeiro». Jesus respondeu-lhes: «Ainda que dê testemunho de mim mesmo, o meu testemunho é verdadeiro, porque sei donde venho e para onde vou; porém, vós ignorais donde venho e para onde vou. Vós julgais segundo a carne, enquanto que Eu não julgo ninguém. E, se julgo, o meu juízo é verdadeiro; pois não sou só, mas estou com o Pai, que me mandou. Na vossa Lei está escrito: «Que o testemunho de dous homens é verdadeiro». Ora, Eu dou testemunho de mim; e o Pai, que me mandou, dá também testemunho de mim». Perguntaram-Lhe então: «Onde está o vosso Pai?». Jesus respondeu: «Vós me não conheceis a mim, nem ao meu Pai. Se me conhecêsseis, conheceríeis também o meu Pai». Estas cousas ensinou Jesus no templo, junto do mealheiro, e ninguém o prendeu, porque não era ainda chegada a sua hora.
}\end{paracol}

\paragraphinfo{Ofertório}{Sl. 17, 3}
\begin{paracol}{2}\latim{
\rlettrine{F}{actus} est Dóminus firmaméntum meum, et refúgium meum, et liberátor meus: et sperábo in eum.
}\switchcolumn\portugues{
\rlettrine{O}{} Senhor constituiu-se o meu sustentáculo, o meu refúgio e o meu salvador. Eu tenho esperança n’Ele!
}\end{paracol}

\paragraph{Secreta}
\begin{paracol}{2}\latim{
\rlettrine{O}{blatiónibus} nostris, quǽsumus, Dómine, placáre suscéptis: et ad te nostras etiam rebélles compélle propítius voluntátes. Per Dóminum \emph{\&c.}
}\switchcolumn\portugues{
\rlettrine{D}{eixai-Vos} aplacar, Senhor, Vos suplicamos, aceitando as nossas oblações; e pela vossa misericórdia obrigai as nossas vontades rebeldes a procurarem-Vos. Por nosso Senhor \emph{\&c.}
}\end{paracol}

\paragraphinfo{Comúnio}{Sl. 22, 1-2}
\begin{paracol}{2}\latim{
\rlettrine{D}{óminus} regit me, et nihil mihi déerit: in loco páscuæ ibi me collocávit: super aquam refectiónis educávit me.
}\switchcolumn\portugues{
\rlettrine{O}{} Senhor é quem me governa; nada me faltará. Ele conduziu-me aos lugares de bom pasto; sentou-me ao pé da água, que me refrigera.
}\end{paracol}

\paragraph{Postcomúnio}
\begin{paracol}{2}\latim{
\rlettrine{T}{ua} nos, quǽsumus, Dómine, sancta puríficent: et operatióne sua tibi plácitos esse perfíciant. Per Dóminum \emph{\&c.}
}\switchcolumn\portugues{
\qlettrine{Q}{ue} os vossos sagrados mystérios, Senhor, Vos suplicamos, nos purifiquem; e que pela sua virtude nos tornemos sempre agradáveis a vossos olhos. Por nosso Senhor \emph{\&c.}
}\end{paracol}

\paragraph{Oração sobre o povo}
\begin{paracol}{2}\latim{
\begin{nscenter} Orémus. \end{nscenter}
}\switchcolumn\portugues{
\begin{nscenter} Oremos. \end{nscenter}
}\switchcolumn*\latim{
Humiliáte cápita vestra Deo.
}\switchcolumn\portugues{
Inclinai as vossas cabeças diante de Deus.
}\switchcolumn*\latim{
Deus, qui sperántibus in te miseréri pótius éligis quam irasci: da nobis digne flere mala, quæ fécimus; ut tuæ consolatiónis grátiam inveníre mereámur. Per Dóminum \emph{\&c.}
}\switchcolumn\portugues{
Ó Deus, que àqueles que em Vós esperam antes desejais mostrar o esplendor da vossa glória do que o peso da vossa ira, concedei-nos que saibamos devidamente chorar os males que temos praticado, a fim de que mereçamos alcançar a graça da vossa consolação. Por nosso Senhor \emph{\&c.}
}\end{paracol}

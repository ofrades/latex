\subsection{Quarto Domingo da Quaresma}

\paragraphinfo{Intróito}{Is. 66, 10 \& 11}
\begin{paracol}{2}\latim{
\rlettrine{L}{ætáre,} Jerúsalem: et convéntum fácite, omnes qui dilígitis eam: gaudéte cum lætítia, qui in tristítia fuístis: ut exsultétis, et satiémini ab ubéribus consolatiónis vestræ. \emph{Ps. 121, 1} Lætátus sum in his, quæ dicta sunt mihi: in domum Dómini íbimus.
℣. Gloria Patri \emph{\&c.}
}\switchcolumn\portugues{
\rlettrine{R}{ejubilai,} ó Jerusalém; ó vós todos, que a amais, uni-vos com júbilo; exultai de alegria, ó vós, que tendes estado tristes: e ficareis alegres e saciados com o leite das suas consolações. \emph{Sl. 121, 1} Alegrei-me com estas palavras que me disseram: Iremos à Casa do Senhor.
℣. Glória ao Pai \emph{\&c.}
}\end{paracol}

\paragraph{Oração}
\begin{paracol}{2}\latim{
\rlettrine{C}{oncéde,} quǽsumus, omnípotens Deus: ut, qui ex merito nostræ actiónis afflígimur, tuæ grátiæ consolatióne respirémus. Per Dóminum nostrum \emph{\&c.}
}\switchcolumn\portugues{
\slettrine{Ó}{} Deus omnipotente, Vos suplicamos, visto que estamos justamente aflitos com o peso dos nossos pecados, concedei-nos que sejamos aliviados com a consolação da vossa graça. Por nosso Senhor \emph{\&c.}
}\end{paracol}

\paragraphinfo{Epístola}{Gl. 4, 22-31}
\begin{paracol}{2}\latim{
Léctio Epístolæ beáti Pauli Apóstoli ad Gálatas.
}\switchcolumn\portugues{
Lição da Ep.ª do B. Ap.º Paulo aos Gálatas.
}\switchcolumn*\latim{
\rlettrine{F}{ratres:} Scriptum est: Quóniam Abraham duos fílios habuit: unum de ancílla, et unum de líbera. Sed qui de ancílla, secúndum carnem natus est: qui autem de líbera, per repromissiónem: quæ sunt per allegóriam dicta. Hæc enim sunt duo testaménta. Unum quidem in monte Sina, in servitútem génerans: quæ est Agar: Sina enim mons est in Arábia, qui conjúnctus est ei, quæ nunc est Jerúsalem, et servit cum fíliis suis. Illa autem, quæ sursum est Jerúsalem, líbera est, quæ est mater nostra. Scriptum est enim: Lætáre, stérilis, quæ non paris: erúmpe, et clama, quæ non párturis: quia multi fílii desértæ, magis quam ejus, quæ habet virum. Nos autem, fratres, secúndum Isaac promissiónis fílii sumus. Sed quómodo tunc is, qui secúndum carnem natus fúerat, persequebátur eum, qui secúndum spíritum: ita et nunc. Sed quid dicit Scriptura? Ejice ancillam et fílium ejus: non enim heres erit fílius ancíllæ cum fílio líberæ. Itaque, fratres, non sumus ancíllæ fílii, sed líberæ: qua libertáte Christus nos liberávit.
}\switchcolumn\portugues{
\rlettrine{M}{eus} irmãos: Está escrito que Abraão teve dous filhos: um da escrava e outro da mulher livre. O da escrava nasceu segundo a carne, enquanto que o da mulher livre nasceu segundo a promessa. Digo-vos estas cousas em sentido alegórico: estas mulheres são as duas alianças. A primeira (a do Sinai) gera para a escravidão: assim Agar. Com efeito, o Sinai é um monte situado na Arábia, o qual corresponde à actual Jerusalém, que está sob a escravidão, assim como seus filhos. Porém a outra (a Jerusalém do alto) é livre. Esta é a nossa mãe; pis está escrito: «Alegra-te, ó estéril, que não geras! Regozija-te; canta sonoramente, ó tu, que não geras; pois os filhos da abandonada serão mais numerosos do que os da mulher casada». Nós, meus irmãos, somos os filhos da promessa, figurados em Isaque; e, como então, aquele que nascia segundo a carne perseguia aquele que nascia segundo o espírito, assim acontece agora. Que diz a Escritura? «Expulsa a escrava e o seu filho, porque o filho da escrava não será herdeiro como o filho da mulher livre». Assim, meus irmãos, nós não somos filhos da escrava, mas da mulher livre, em cuja liberdade Cristo nos libertou.
}\end{paracol}

\paragraphinfo{Gradual}{Sl. 121, 1 \& 7}
\begin{paracol}{2}\latim{
\rlettrine{L}{ætátus} sum in his, quæ dicta sunt mihi: in domum Dómini íbimus. ℣. Fiat pax in virtúte tua: et abundántia in túrribus tuis.
}\switchcolumn\portugues{
\rlettrine{A}{legrei-me} com estas palavras que me disseram: Iremos à Casa do Senhor. ℣. Que a paz reine dentro dos teus muros: e a abundância nos teus palácios.
}\end{paracol}

\paragraphinfo{Trato}{Sl. 124, 1-2}
\begin{paracol}{2}\latim{
\qlettrine{Q}{ui} confídunt in Dómino, sicut mons Sion: non commovébitur in ætérnum, qui hábitat in Jerúsalem. ℣. Montes in circúitu ejus: et Dóminus in circúitu pópuli sui, ex hoc nunc et usque in sǽculum.
}\switchcolumn\portugues{
\rlettrine{A}{queles} que confiam no Senhor estão firmes, como o monte Sião. Aquele que habita em Jerusalém nunca será abalado. ℣. O Senhor cerca com montanhas o seu povo, e fica em torno dele agora e em todos os séculos.
}\end{paracol}

\paragraphinfo{Evangelho}{Jo. 6, 1-15}
\begin{paracol}{2}\latim{
\cruz Sequéntia sancti Evangélii secúndum Joánnem.
}\switchcolumn\portugues{
\cruz Continuação do santo Evangelho segundo S. João.
}\switchcolumn*\latim{
\blettrine{I}{n} illo témpore: Abiit Jesus trans mare Galilǽæ, quod est Tiberíadis: et sequebátur eum multitúdo magna, quia vidébant signa, quæ faciébat super his, qui infírmabántur. Súbiit ergo in montem Jesus: et ibi sedébat cum discípulis suis. Erat autem próximum Pascha, dies festus Judæórum. Cum sublevásset ergo óculos Jesus et vidísset, quia multitúdo máxima venit ad eum, dixit ad Philíppum: Unde emémus panes, ut mandúcent hi? Hoc autem dicebat tentans eum: ipse enim sciébat, quid esset factúrus. Respóndit ei Philíppus: Ducentórum denariórum panes non suffíciunt eis, ut unusquísque módicum quid accípiat. Dicit ei unus ex discípulis ejus, Andréas, frater Simónis Petri: Est puer unus hic, qui habet quinque panes hordeáceos et duos pisces: sed hæc quid sunt inter tantos? Dixit ergo Jesus: Fácite hómines discúmbere. Erat autem fænum multum in loco. Discubuérunt ergo viri, número quasi quinque mília. Accépit ergo Jesus panes, et cum grátias egísset, distríbuit discumbéntibus: simíliter et ex píscibus, quantum volébant. Ut autem impléti sunt, dixit discípulis suis: Collígite quæ superavérunt fragménta, ne péreant. Collegérunt ergo, et implevérunt duódecim cóphinos fragmentórum ex quinque pánibus hordeáceis, quæ superfuérunt his, qui manducáverant. Illi ergo hómines cum vidíssent, quod Jesus fécerat signum, dicébant: Quia hic est vere Prophéta, qui ventúrus est in mundum. Jesus ergo cum cognovísset, quia ventúri essent, ut ráperent eum et fácerent eum regem, fugit íterum in montem ipse solus.
}\switchcolumn\portugues{
\blettrine{N}{aquele} tempo, foi Jesus para a outra margem do mar da Galileia ou de Tiberíades, acompanhando-O grande multidão; pois viam os milagres que operava, curando os doentes. Jesus subiu a um monte, e aí se assentou com seus discípulos. Ora a Páscoa, que era a principal festa dos judeus, estava. Próxima. E, levantando Jesus os olhos, viu que grande multidão de povo estava com Ele. Então, disse a Filipe: «Onde compraremos pão para tanta gente?». Ele dizia isto para experimentar Filipe, pois bem sabia o que havia de fazer. Filipe respondeu-Lhe: «Duzentos dinheiros de pão não bastarão para que cada um receba um bocado!». Mas um dos discípulos, André, irmão de Simão-Pedro, disse-Lhe: «Está aí um homem que tem cinco pães de aveia e dous peixes; porém, que é isto para tanta gente?!...» Jesus disse: «Mandai-os assentar todos». Assentaram-se eles, sendo cerca de cinco mil! Então Jesus tomou os pães, e, tendo dado graças, distribuiu-os; do mesmo modo distribuiu os peixes. E comeu cada um quanto quis! Quando já estavam fartos, disse Jesus a seus discípulos: «Recolhei os sobejos, para que se não percam». Recolheram-nos eles, enchendo doze cestos com os bocados, que haviam sobejado, dos cinco pães de aveia! Então, estes homens, vendo o milagre que Jesus acabava de fazer, diziam: «Verdadeiramente este é o Profeta que deve vir ao mundo!». Porém Jesus, sabendo que eles queriam aclamá-l’O Rei, fugiu só para o monte.
}\end{paracol}

\paragraphinfo{Ofertório}{Sl. 134, 3 \& 6}
\begin{paracol}{2}\latim{
\rlettrine{L}{audáte} Dóminum, quia benígnus est: psállite nómini ejus, quóniam suávis est: ómnia, quæcúmque vóluit, fecit in cœlo et in terra.
}\switchcolumn\portugues{
\rlettrine{L}{ouvai} o Senhor, porque Ele é bom: Cantai hinos em louvor do seu nome, porque Ele é suave. O Senhor criou no céu e na terra tudo quanto quis.
}\end{paracol}

\paragraph{Secreta}
\begin{paracol}{2}\latim{
\rlettrine{S}{acrifíciis} præséntibus, Dómine, quǽsumus, inténde placátus: ut et devotióni nostræ profíciant et salúti. Per Dóminum \emph{\&c.}
}\switchcolumn\portugues{
\rlettrine{S}{enhor,} Vos suplicamos, olhai aplacado para este sacrifício; e que ele alente a nossa piedade e nos alcance a salvação. Por nosso Senhor \emph{\&c.}
}\end{paracol}

\paragraphinfo{Comúnio}{Sl. 121,3-4}
\begin{paracol}{2}\latim{
\qlettrine{J}{erúsalem,} quæ ædificátur ut cívitas, cujus participátio ejus in idípsum: illuc enim ascendérunt tribus, tribus Dómini, ad confiténdum nómini tuo. Dómine.
}\switchcolumn\portugues{
\qlettrine{J}{erusalém} é uma cidade tão bem edificada, que está agrupada em um conjunto. Foi lá que subiram as tribos (as tribos do Senhor) para louvar o vosso nome, ó Senhor.
}\end{paracol}

\paragraph{Postcomúnio}
\begin{paracol}{2}\latim{
\rlettrine{D}{a} nobis, quǽsumus, miséricors Deus: ut sancta tua, quibus incessánter explémur, sincéris tractémus obséquiis, et fidéli semper mente sumámus. Per Dóminum \emph{\&c.}
}\switchcolumn\portugues{
\slettrine{Ó}{} Deus de misericórdia, concedei-nos a graça de nos aproximarmos com respeito sincero dos vossos sagrados mystérios, de que fomos saciados, e de os recebermos sempre com espírito de fé. Por nosso Senhor \emph{\&c.}
}\end{paracol}

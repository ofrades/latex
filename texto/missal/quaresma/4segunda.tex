\subsectioninfo{Segunda-feira da 4.ª Semana da Quaresma}{Estação nos Santos Coroados}

\paragraphinfo{Intróito}{Sl. 53, 3-4}
\begin{paracol}{2}\latim{
\rlettrine{D}{eus,} in nómine tuo salvum me fac, et in virtúte tua líbera me: Deus, exáudi oratiónem meam: áuribus pércipe verba oris mei. \emph{Ps. ibid., 5} Quóniam aliéni insurrexérunt in me: et fortes Quæsiérunt ánimam meam.
℣. Gloria Patri \emph{\&c.}
}\switchcolumn\portugues{
\slettrine{Ó}{} Deus, pelo vosso nome, salvai-me: e livrai-me com vosso poder. Ó Deus, ouvi a minha oração; abri os vossos ouvidos às orações dos meus lábios. \emph{Sl. ibid., 5} Porquanto os estrangeiros levantam-se contra mim: e homens violentos querem tirar-me a vida.
℣. Glória ao Pai \emph{\&c.}
}\end{paracol}

\paragraph{Oração}
\begin{paracol}{2}\latim{
\rlettrine{P}{ræsta,} quǽsumus, omnípotens Deus: ut, observatiónes sacras ánnua devotióne recoléntes, et córpore tibi placeámus et mente. Per Dóminum \emph{\&c.}
}\switchcolumn\portugues{
\rlettrine{C}{oncedei-nos,} ó Deus omnipotente, Vos suplicamos, a graça de observarmos anualmente com devoção estas santas práticas quaresmais e de Vos agradarmos corporal e espiritualmente. Por nosso Senhor \emph{\&c.}
}\end{paracol}

\paragraphinfo{Epístola}{3 Rs. 3, 16-28}
\begin{paracol}{2}\latim{
Léctio libri Regum.
}\switchcolumn\portugues{
Lição do Livro dos Reis.
}\switchcolumn*\latim{
\rlettrine{I}{n} diébus illis: Venérunt duæ mulíeres meretríces ad regem Salomónem, steterúntque coram eo, quarum una ait: Obsecro, mi dómine: ego et múlier hæc habitabámus in domo una, et péperi apud eam in cubículo. Tértia autem die postquam ego péperi, péperit et hæc: et erámus simul, nullúsque álius nobíscum in domo, excéptis nobis duábus. Mórtuus est autem fílius mulíeris hujus nocte: dórmiens quippe oppréssit eum. Et consúrgens intempéstæ noctis siléntio, tulit fílium meum de látere meo ancíllæ tuæ dormiéntis, et collocávit in sinu suo: suum autem fílium, qui erat mórtuus, pósuit in sinu meo. Cumque surrexíssem mane, ut darem lac fílio meo, appáruit mórtuus: quem diligéntius íntuens clara luce, deprehéndi non esse meum, quem genúeram. Respondítque áltera múlier Non est ita, ut dicis, sed fílius tuus mórtuus est, meus autem vivit. E contrário illa dicébat: Mentiris: fílius quippe meus vivit, et fílius tuus mórtuus est. Atque in hunc modum contendébant coram rege. Tunc rex ait: Haec dicit Fílius meus vivit, et fílius tuus mórtuus est. Et ista respóndit: Non, sed fílius tuus mórtuus est, meus autem vivit. Dixit ergo rex: Affért mihi gládium. Cumque attulíssent gládium coram rege: Divídite, inquit, infántem vivum in duas partes, et dat dimídiam partem uni, et dimídiam partem alteri. Dixit autem múlier, cujus fílius erat vivus, ad regem (commóta sunt quippe víscera ejus super fílio suo): Obsecro, dómine, date illi infántem virum, et nolíte interfícere dum. E contrário illa dicebat: Nec mihi nec tibi sit, sed diridátur. Respóndit rex et ait: Date huic infántem vivum, et non occidátur: hæc est enim mater ejus. Audívit itaque omnis Israel judícium, quod judicásset rex, et timuérunt regem, vidéntes sapiéntiam Dei esse in eo ad faciéndum judícium.
}\switchcolumn\portugues{
\rlettrine{N}{aqueles} dias, vieram duas mulheres pecadoras e apresentaram-se diante do rei Salomão. Uma delas disse-lhe: «Meu senhor, atendei-me: Eu e esta mulher morávamos na mesma casa. Eu dei à luz uma criança, estando com ela no mesmo aposento. Passados três dias, esta mulher deu também à luz uma outra criança. Nós habitávamos juntas; nenhum estranho havia em casa. Ora o filho desta mulher morreu de noite; pois ela, quando dormia, sufocou-o. Então ela, no meio da noite, levantando-se sem ruído, tirou o meu filho do meu lado, enquanto esta vossa serva dormia, e colocou-o ao lado dela, pondo ao meu lado o seu filho, que estava morto. Quando, de manhã, me levantei para dar leite ao meu filho, encontrei-o morto. Porém, olhando-o com atenção e sendo mais de dia já, reconheci que não era aquele o que eu gerara». Depois disto, disse a outra mulher: «Não é assim como dizes. O teu filho é que morreu; este, que está vivo, é o meu». E a outra replicava: «Mentis, pois o meu filho vive: o teu é que morreu». E assim disputavam ante o rei. Então disse este: «Uma diz: o meu filho vive e o teu morreu. E a outra responde: não; o teu filho é que morreu; o meu Vive». E o rei acrescentou: «Trazei-me uma espada». Logo que lhe levaram a espada, disse o rei: «Cortai a criança viva em duas partes e entregai metade a cada uma das mulheres!». Então aquela cujo filho estava vivo, disse ao rei (pois ela sentiu o seu íntimo mover-se de dor pelo filho): «Senhor, peço-te que dês a esta o menino Vivo, para que o não matem». Pelo contrário, a outra dizia: «Não seja a criança para mim, nem para ti; mas seja dividida». Então, o rei disse: «Entregai à primeira o menino vivo, e não seja morto; porque esta é a sua mãe». E todo o povo de Israel admirou a sentença que o rei proferiu, e o temeram, vendo que a sabedoria do Senhor o inspirava para fazer justiça.
}\end{paracol}

\paragraphinfo{Gradual}{Sl. 30, 3}
\begin{paracol}{2}\latim{
\rlettrine{E}{sto} mihi in Deum protectórem et in locum refúgii, ut salvum me fácias. ℣. \emph{Ps. 70, 1} Deus, in te sperávi: Dómine, non confúndar in ætérnum.
}\switchcolumn\portugues{
\rlettrine{S}{ede,} ó Deus, o meu protector: sede o refúgio onde eu encontre a salvação. ℣. \emph{Sl. 70, 1} Ó Deus, esperei em Vós! Que eu não seja, pois, para sempre confundido, Senhor!
}\end{paracol}

\paragraphinfo{Trato}{Página \pageref{tratoquartacinzas}}

\paragraphinfo{Evangelho}{Jo. 2, 13-25}
\begin{paracol}{2}\latim{
\cruz Sequéntia sancti Evangélii secúndum Joánnem.
}\switchcolumn\portugues{
\cruz Continuação do santo Evangelho segundo S. João.
}\switchcolumn*\latim{
\blettrine{I}{n} illo témpore: Prope erat Pascha Judæórum, et ascéndit Jesus Jerosólymam: et invénit in templo vendéntes boves et oves et colúmbas, et nummulários sedéntes. Et cum fecísset quasi flagéllum de funículis, omnes ejécit de templo, oves quoque et boves, et nummulariórum effúdit æs et mensas subvértit. Et his, qui colúmbas vendébant, dixit: Auférte ista hinc, et nolíte fácere domum Patris mei domum negotiationis. Recordáti sunt vero discipuli ejus, quia scriptum est: Zelus domus tuæ comédit me. Respondérunt ergo Judǽi, et dixérunt ei: Quod signum osténdis nobis, quia hæc facis? Respóndit Jesus et dixit eis: Sólvite templum hoc, et in tribus diébus excitábo illud. Dixérunt ergo Judǽi: Quadragínta et sex annis ædificátum est templum hoc, et tu in tribus diébus excitábis illud? Ille autem dicébat de templo córporis sui. Cum ergo resurrexísset a mórtuis, recordáti sunt discípuli ejus, quia hoc dicébat, et credidérunt Scriptúrae, et sermóni, quem dixit Jesus. Cum autem esset Jerosólymis in Pascha in die festo, multi credidérunt in nómine ejus, vidéntes signa ejus, quæ faciébat. Ipse autem Jesus non credébat semetípsum eis, eo quod ipse nosset omnes, et quia opus ei non erat, ut quis testimónium perhibéret de hómine: ipse enim sciébat, quid esset in hómine.
}\switchcolumn\portugues{
\blettrine{N}{aquele} Naquele tempo, estando próxima a Páscoa dos judeus, Jesus subiu a Jerusalém e entrou no templo, onde encontrou negociantes de bois, de ovelhas e de pombas, e ainda os cambistas de dinheiro, que estavam abancados. Logo, fez um chicote com cordas, expulsando do templo, tanto os negociantes e cambistas de dinheiro como as ovelhas e bois, atirando ao chão com o dinheiro e as mesas que lá estavam, e dizendo aos que vendiam as pombas: «Tirai isto daqui e não torneis a casa de meu Pai em casa de negócio». Então, lembraram-se os seus discípulos de que está escrito: «O zelo da vossa casa me devora». E os judeus disseram-Lhe: «Que sinal nos mostras para procederes assim?». Jesus disse-lhes: «Destruí este templo, que em três dias o reedificarei». Os judeus replicaram: «Em quarenta e seis anos foi este templo edificado, e tu em três dias queres levantá-lo?». Porém, Ele falava do templo do seu corpo. E, quando ressuscitou dos mortos, recordaram os discípulos o que Ele havia dito e acreditaram na Escritura e nas palavras que lhes dissera. Estando Jesus em Jerusalém, no dia da festa da Páscoa, muitos, vendo os milagres que fazia, acreditaram no seu nome, porém, Jesus não se fiava neles, porque os conhecia a todos e não necessitava de que ninguém Lhe desse testemunho de qualquer homem; pois sabia bem o que havia no íntimo de cada um.
}\end{paracol}

\paragraphinfo{Ofertório}{Sl. 99, 1-2}
\begin{paracol}{2}\latim{
\qlettrine{J}{ubiláte} Deo, omnis terra, servíte Dómino in lætítia: intráte in conspéctu ejus in exsultatióne: quia Dóminus ipse est Deus.
}\switchcolumn\portugues{
\rlettrine{A}{clamai} Deus, ó habitantes de toda a terra; obedecei ao Senhor com alegria, vinde diante d’Ele com júbilo: pois o Senhor é Deus.
}\end{paracol}

\paragraph{Secreta}
\begin{paracol}{2}\latim{
\rlettrine{O}{blátum} tibi, Dómine, sacrifícium vivíficet nos semper et múniat. Per Dóminum \emph{\&c.}
}\switchcolumn\portugues{
\rlettrine{P}{ermiti,} Senhor, que o sacrifício, que Vos oferecemos, sempre nos vivifique e nos fortaleça. Por nosso Senhor \emph{\&c.}
}\end{paracol}

\paragraphinfo{Comúnio}{Sl. 18, 13 \& 14}
\begin{paracol}{2}\latim{
\rlettrine{A}{b} occúltis meis munda me, Dómine: et ab aliénis parce servo tuo.
}\switchcolumn\portugues{
\rlettrine{S}{enhor,} purificai-me dos meus delitos ocultos e livrai-me dos delitos alheios.
}\end{paracol}

\paragraph{Postcomúnio}
\begin{paracol}{2}\latim{
\rlettrine{S}{umptis,} Dómine, salutáribus sacraméntis: ad redemptiónis ætérnæ, quǽsumus, proficiámus augméntum. Per Dóminum \emph{\&c.}
}\switchcolumn\portugues{
\rlettrine{T}{endo} nós sido saciados com os sacramentos da salvação, Senhor, Vos imploramos, permiti que possamos alcançar aumento da redenção eterna. Por nosso Senhor \emph{\&c.}
}\end{paracol}

\paragraph{Oração sobre o povo}
\begin{paracol}{2}\latim{
\begin{nscenter} Orémus. \end{nscenter}
}\switchcolumn\portugues{
\begin{nscenter} Oremos. \end{nscenter}
}\switchcolumn*\latim{
Humiliáte cápita vestra Deo.
}\switchcolumn\portugues{
Inclinai as vossas cabeças diante de Deus.
}\switchcolumn*\latim{
Deprecatiónem nostram, quǽsumus. Dómine, benígnus exáudi: et, quibus supplicándi præstas afféctum, tríbue defensiónis auxílium. Per Dóminum \emph{\&c.}
}\switchcolumn\portugues{
Senhor, Vos suplicamos, ouvi benigno a nossa prece e socorrei com vossa protecção aqueles a quem inspirais o desejo da oração. Por nosso Senhor \emph{\&c.}
}\end{paracol}

\subsectioninfo{Quinta-feira da 4.ª Semana da Quaresma}{Estação em S. Silvestre e S. Martinho}

\paragraphinfo{Intróito}{Sl. 104, 3-4}
\begin{paracol}{2}\latim{
\rlettrine{L}{ætétur} cor quæréntium Dóminum: quǽrite Dóminum, et confirmámini: quǽrite fáciem ejus semper. \emph{Ps. ibid., 1} Confitémini Dómino, et invocáte nomen ejus: annuntiáte inter gentes ópera ejus.
℣. Gloria Patri \emph{\&c.}
}\switchcolumn\portugues{
\rlettrine{A}{legre-se} o coração daqueles que procuram o Senhor: procurai o Senhor e ficareis cheios de fortaleza: procurai incessantemente a sua presença. \emph{Sl. ibid., 1} Louvai o Senhor e aclamai o seu nome: anunciai as suas obras no meio dos povos.
℣. Glória ao Pai \emph{\&c.}
}\end{paracol}

\paragraph{Oração}
\begin{paracol}{2}\latim{
\rlettrine{P}{ræsta,} quǽsumus, omnípotens Deus: ut, quos jejúnia votíva castígant, ipsa quoque devótio sancta lætíficet; ut, terrénis afféctibus mitigátis, facílius cœléstia capiámus. Per Dóminum \emph{\&c.}
}\switchcolumn\portugues{
\rlettrine{P}{ermiti,} ó Deus omnipotente, Vos suplicamos, que aqueles que voluntariamente se castigam com jejuns sejam consolados com a alegria duma piedade santa, a fim de que, mitigado o ardor dos afectos terrenos, gozemos mais largamente os bens celestiais. Por nosso Senhor \emph{\&c.}
}\end{paracol}

\paragraphinfo{Epístola}{4 Rs. 4, 25-38}
\begin{paracol}{2}\latim{
Léctio libri Regum.
}\switchcolumn\portugues{
Lição do Livro dos Reis.
}\switchcolumn*\latim{
\rlettrine{I}{n} diébus illis: Venit múlier Sunamítis ad Eliséum in montem Carméli: cumque vidísset eam vir Dei e contra, ait ad Giézi púerum suum: Ecce Sunamítis illa. Vade ergo in occúrsum ejus, et dic ei: Recte ne ágitur circa te, et circa virum tuum, et circa fílium tuum? Quæ respóndit: Recte. Cumque venísset ad virum Dei in montem, apprehéndit pedes ejus: et accéssit Giézi, ut amovéret eam. Et ait homo Dei: Dimítte illam: ánima enim ejus in amaritúdine est, et Dóminus celávit a me, et non indicávit mihi. Quæ dixit illi: Numquid petívi fílium a dómino meo? Numquid non dixi tibi: Ne illúdas me? Et ille ait ad Giézi: Accínge lumbos tuos, et tolle báculum meum in manu tua, et vade. Si occurrérit tibi homo, non salútes eum: et si salutáverit te quíspiam, non respóndeas illi: et pones báculum meum super fáciem púeri. Porro mater pueri ait: Vivit Dóminus et vivit ánima tua, non dimíttam te. Surréxit ergo, et secútus est eam. Giézi autem præcésserat ante eos, et posúerat báculum super fáciem púeri, et non erat vox neque sensus: reversúsque est in occúrsum ejus, et nuntiávit ei, dicens: Non surréxit puer. Ingréssus est ergo Eliséus domum, et ecce, puer mórtuus jacébat in léctulo ejus: ingressúsque clausit óstium super se et super púerum: et orávit ad Dóminum. Et ascéndit, et incúbuit super púerum: posuítque os suum super os ejus, et óculos suos super óculos ejus, et manus suas super manus ejus: et incurvávit se super eum: et calefácta est caro púeri. At ille revérsus, de ambulávit in domo, semel huc atque illuc: et ascéndit et incúbuit super eum: et oscitávit puer sépties, aperuítque oculos. At ille vocávit Giézi, et dixit ei: Voca Sunamítidem hanc. Quæ vocáta ingréssa est ad eum. Qui ait: Tolle fílium tuum. Venit illa, et córruit ad pedes ejus, et adorávit super terram: tulítque fílium suum, et egréssa est, et Eliséus revérsus est in Gálgala.
}\switchcolumn\portugues{
\rlettrine{N}{aqueles} dias, uma mulher Sunamite foi ter com o Profeta Eliseu ao monte Carmelo. Como o varão de Deus a visse aproximar, disse a Giézi, seu servo: «Eis aí vem a Sunamite. Vai, pois, ao seu encontro e diz-lhe: «Porventura corre tudo com exactidão em tua casa? O teu marido e o teu filho passam bem?». Ela respondeu: «Sim!». Logo que ela chegou ao monte, ao pé do homem de Deus, segurou-se-lhe aos pés. Giézi aproximou-se dela para a afastar. Mas o homem de Deus disse: «Deixa-a, pois a sua alma está amargurada, ainda que o Senhor me não deu a conhecer o motivo». Ela disse-lhe, então: «Porventura havia eu pedido um filho ao meu Senhor? Não te disse eu: não me iludas?». Logo Eliseu disse a Giézi: «Cinge os teus rins, toma o meu bordão na tua mão e parte. Se encontrares alguém, o não saúdes: se alguém te saudar, não correspondas; e colocarás o meu bordão em cima do rosto do menino». Mas a mãe do menino disse ao Profeta: «Viva o Senhor e viva a tua alma; eu, porém, te não deixarei!». Levantou-se, então, Eliseu e seguiu-a. Entretanto, Giézi, tendo partido antes deles, logo que lá chegou, pôs o bordão em cima do rosto do menino, que já não tinha voz, nem sentidos; e, voltando ao encontro de Eliseu, contou-lhe o que se passara, dizendo: «O menino não ressuscitou». Entrou, então, Eliseu em casa e encontrou o menino deitado no leito e já morto. Eliseu fechou a porta, ficando só com o menino, e orou ao Senhor. Depois subiu para o leito, estendeu-se em cima do menino, pôs a sua boca sobre a boca do menino, os seus olhos sobre os olhos dele, as suas mãos sobre as mãos dele e inclinou-se sobre ele. E logo a carne do menino adquiriu calor. Eliseu, tendo descido do leito, passeou pela casa duas vezes; depois tornou a subir e a inclinar-se sobre o menino. Então este bocejou sete vezes, abrindo os olhos. Eliseu chamou logo Giézi e disse-lhe: «Chama a Sunamite». Giézi chamou-a. Veio ela junto de Eliseu, que lhe disse: «Toma o teu filho». E ela, aproximando-se, lançou-se aos pés de Eliseu e prostrou-se por terra. Depois levou o filho e se retirou. Eliseu voltou para Gálgala.
}\end{paracol}

\paragraphinfo{Gradual}{Sl. 73, 20, 19 \& 22}
\begin{paracol}{2}\latim{
\rlettrine{R}{éspice,} Dómine, in testaméntum tuum: et ánimas páuperum tuórum ne obliviscáris in finem. ℣. Exsúrge, Dómine, júdica causam tuam: memor esto oppróbrii servórum tuórum.
}\switchcolumn\portugues{
\rlettrine{R}{ecordai-Vos,} Senhor, da vossa aliança; não esqueçais perpetuamente as almas dos vossos pobres servos. Erguei-Vos, Senhor, e julgai esta vossa causa: lembrai-Vos dos opróbrios que sofrem os vossos servos.
}\end{paracol}

\paragraphinfo{Evangelho}{Lc. 7, 11-16}
\begin{paracol}{2}\latim{
\cruz Sequéntia sancti Evangélii secúndum Lucam.
}\switchcolumn\portugues{
\cruz Continuação do santo Evangelho segundo S. Lucas.
}\switchcolumn*\latim{
\blettrine{I}{n} illo témpore: Ibat Jesus in civitátem, quæ vocátur Naim: et ibant cum eo discípuli ejus et turba copiósa. Cum autem appropinquáret portæ civitátis, ecce, defúnctus efferebátur fílius únicus matris suæ: et hæc vidua erat, et turba civitátis multa cum illa. Quam cum vidísset Dóminus, misericórdia motus super eam, dixit illi: Noli flere. Et accéssit et tétigit lóculum. (Hi autem, qui portábant, steterunt.) Et ait: Adoléscens, tibi dico, surge. Et resédit, qui erat mórtuus, et cœpit loqui. Et dedit illum matri suæ. Accepit autem omnes timor: et magnificábant Deum, dicéntes: Quia Prophéta magnus surréxit in nobis: et quia Deus visitávit plebem suam.
}\switchcolumn\portugues{
\blettrine{N}{aquele} tempo, caminhava Jesus para uma cidade chamada Naim, acompanhado por seus discípulos e muito povo. Chegando à porta da cidade, encontrou um cadáver, filho único duma viúva, o qual ia acompanhado por muitas pessoas da cidade. Vendo o Senhor a viúva, compadeceu-se dela e disse-lhe: «Não chores». E, aproximando-se, tocou no esquife (pois aqueles que o levavam tinham parado), dizendo: «Adolescente, eu te mando: levanta-te!». E no mesmo instante levantou-se o morto e começou a falar. Então Jesus entregou-o a sua mãe. Pelo que todos os assistentes ficaram atemorizados e glorificavam o Senhor, dizendo: «Um grande Profeta se levantou no meio de nós; Deus visitou o seu povo».
}\end{paracol}

\paragraphinfo{Ofertório}{Sl. 69, 2,3 \& 4}
\begin{paracol}{2}\latim{
\rlettrine{D}{ómine,} ad adjuvándum me festína: confundántur omnes, qui cógitant servis tuis mala.
}\switchcolumn\portugues{
\rlettrine{S}{enhor,} apressai-Vos em socorrer-me: que fiquem cheios de confusão aqueles que querem fazer mal aos vossos servos.
}\end{paracol}

\paragraph{Secreta}
\begin{paracol}{2}\latim{
\rlettrine{P}{urífica} nos, misericors Deus: ut Ecclésiæ tuæ preces, quæ tibi gratæ sunt, pia múnera deferéntes, fiant expiátis méntibus gratióres. Per Dóminum \emph{\&c.}
}\switchcolumn\portugues{
\rlettrine{P}{urificai-nos,} ó Deus de misericórdia, a fim de que as preces da vossa Igreja, que Vos são agradáveis, a Vós se tornem mais agradáveis ainda pela expiação daqueles que Vos oferecem estes pios dons. Por nosso Senhor \emph{\&c.}
}\end{paracol}

\paragraphinfo{Comúnio}{Sl. 70, 16-17 \& 18}
\begin{paracol}{2}\latim{
\rlettrine{D}{ómine,} memorábor justítiæ tuæ solíus: Deus, docuísti me a juventúte mea: et usque in senéctam et sénium, Deus, ne derelínquas me.
}\switchcolumn\portugues{
\rlettrine{S}{enhor,} empregar-me-ei somente nas obras da vossa justiça! Ó Deus, instruístes-me desde a minha infância. Não me abandonareis, pois, ó Deus, até à velhice, até aos cabelos brancos!
}\end{paracol}

\paragraph{Postcomúnio}
\begin{paracol}{2}\latim{
\rlettrine{C}{œléstia} dona capiéntibus, quǽsumus, Dómine: non ad judícium proveníre patiáris, quæ fidélibus tuis ad remédium providísti. Per Dóminum nostrum \emph{\&c.}
}\switchcolumn\portugues{
\rlettrine{N}{ão} permitais, Senhor, Vos suplicamos, que estes dons celestiais, que preparastes para remédio dos fiéis, sirvam de condenação aos que os recebem. Por nosso Senhor \emph{\&c.}
}\end{paracol}

\paragraph{Oração sobre o povo}
\begin{paracol}{2}\latim{
\begin{nscenter} Orémus. \end{nscenter}
}\switchcolumn\portugues{
\begin{nscenter} Oremos. \end{nscenter}
}\switchcolumn*\latim{
Humiliáte cápita vestra Deo.
}\switchcolumn\portugues{
Inclinai as vossas cabeças diante de Deus.
}\switchcolumn*\latim{
Pópuli tui, Deus, institútor et rector, peccáta, quibus impugnátur, expélle: ut semper tibi plácitus, et tuo munímine sit secúrus. Per Dóminum \emph{\&c.}
}\switchcolumn\portugues{
Ó Deus, instituidor e guia do vosso povo, afastai dele os pecados que o assaltam, a fim de que, agradando-Vos sempre, esteja certo da vossa protecção. Por nosso Senhor \emph{\&c.}
}\end{paracol}

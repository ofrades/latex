\subsectioninfo{Terça-feira da 2.ª Semana da Quaresma}{Estação em Santa Balbina}

\paragraphinfo{Intróito}{Sl. 26, 8 \& 9}
\begin{paracol}{2}\latim{
\rlettrine{T}{ibi} dixit cor meum, quæsívi vultum tuum, vultum tuum, Dómine, requíram: ne avértas fáciem tuam a me. \emph{Ps. ibid., 1} Dóminus illuminátio mea, et salus mea: quem timébo?
℣. Gloria Patri \emph{\&c.}
}\switchcolumn\portugues{
\rlettrine{D}{a} vossa parte, Senhor, me diz meu coração: «Procura a minha presença». Sim, não deixarei, Senhor, de procurar a vossa presença. Não Vos afasteis, pois, de mim. O Senhor é a minha luz e a minha salvação. A quem, pois, temerei?
℣. Glória ao Pai \emph{\&c.}
}\end{paracol}

\paragraph{Oração}
\begin{paracol}{2}\latim{
\rlettrine{P}{érfice,} quǽsumus, Dómine, benignus in nobis observántiæ sanctæ subsídium: ut, quæ te auctóre faciénda cognóvimus, te operánte impleámus. Per Dóminum \emph{\&c.}
}\switchcolumn\portugues{
\rlettrine{S}{enhor,} Vos imploramos, continuai a assistir-nos com vossa bondade durante a observância deste santo jejum, a fim de que com vosso auxílio pratiquemos esta boa obra que nos ensinastes com vosso exemplo. Por nosso Senhor \emph{\&c.}
}\end{paracol}

\paragraphinfo{Epístola}{3 Rs. 17, 8-16}
\begin{paracol}{2}\latim{
Léctio libri Regum.
}\switchcolumn\portugues{
Lição do Livro dos Reis.
}\switchcolumn*\latim{
\rlettrine{I}{n} diébus illis: Factus est sermo Dómini ad Elíam Thesbíten, dicens: Surge et vade in Saréphta Sidoniórum, et manébis ibi: præcépi enim ibi mulíeri víduæ, ut pascat te. Surréxit et ábiit in Saréphta. Cumque venísset ad portam civitátis, appáruit ei múlier vídua cólligens ligna, et vocávit eam, dixítque ei: Da mihi páululum aquæ in vase, ut bibam. Cumque illa pérgeret, ut afférret, clamávit post tergum ejus, dicens: Affer mihi, óbsecro, et buccéllam panis in manu tua. Quæ respóndit: Vivit Dóminus, Deus tuus, quia non habeo panem, nisi quantum pugíllus cápere potest farínæ in hýdria, et páululum ólei in lécytho: en, collige duo ligna, ut ingrédiar, et fáciam illum mihi et fílio meo, ut comedámus et moriámur. Ad quam Elías ait: Noli timére, sed vade, et fac, sicut dixísti: verúmtamen mihi primum fac de ipsa farínula subcinerícium panem párvulum, et affer ad me: tibi autem et fílio tuo fácies póstea. Hæc autem dicit Dóminus, Deus Israël: Hýdria farínæ non defíciet, nec lécythus ólei minuétur, usque ad diem, in qua Dóminus datúrus est plúviam super fáciem terræ. Quæ ábiit, et fecit juxta verbum Elíæ: et comédit ipse et illa et domus ejus: et ex illa die hýdria farínæ non defécit, et lécythus ólei non est imminútus, juxta verbum Dómini, quod locútus fúerat in manu Elíæ.
}\switchcolumn\portugues{
\rlettrine{N}{aqueles} dias, falou o Senhor a Elias Tesbiteu, dizendo: «Ergue-te, vai a Sarepta de Sídon e fica lá, pois mandei a uma viúva daquela terra que te sustente». Elias obedeceu à voz de Deus, caminhando para Sarepta. Quando chegou às portas da cidade, encontrou uma viúva, que andava à procura de lenha. Chamou-a logo, dizendo-lhe: «Dá-me um pouco de água num vaso, para eu beber». A viúva dirigiu-se a casa, para trazer a água; mas Elias, que ficou atrás dela, disse-lhe: «Traze-me também nas tuas mãos um bocado de pão, para comer». Ela respondeu: «Viva o Senhor, teu Deus, como não tenho pão algum! Apenas tenho em uma vasilha uma mão de farinha e um pouco de azeite em uma almotolia. Eis que procuro dous bocados de lenha para preparar, para mim e meu filho, quando voltar para casa, esse pouco que possuo. Comeremos isso e morreremos depois». Elias disse-lhe: «Não temas; vai e faz como disseste; mas antes faz para mim um bolo dessa farinha, coze-o no borralho e traze-mo. Para ti e teu filho prepararás depois; porque (diz o Senhor, Deus de Israel) a farinha que está na talha se não acabará e o azeite que está na almotolia se não extinguirá, até ao dia em que o Senhor fizer cair a chuva na terra». Foi ela e procedeu segundo a palavra de Elias. E comeram ele, ela e a família de sua casa! E desde aquele dia não mais faltou a farinha na talha, nem o azeite na almotolia, segundo a palavra que o Senhor proferira pela boca de Elias.
}\end{paracol}

\paragraphinfo{Gradual}{Sl. 54, 23, 17, 18 \& 19}
\begin{paracol}{2}\latim{
\qlettrine{J}{acta} cogitátum tuum in Dómino, et ipse te enútriet. ℣. Dum clamárem ad Dóminum, exaudívit vocem meam ab his, qui appropínquant mihi.
}\switchcolumn\portugues{
\rlettrine{D}{eixai} as vossas preocupações nas mãos do Senhor, que Ele cuidará de vós! ℣. Quando eu rezava ao Senhor, ouviu Ele a minha voz e salvou-me daqueles que me cercavam.
}\end{paracol}

\paragraphinfo{Evangelho}{Mt. 23, 1-12}
\begin{paracol}{2}\latim{
\cruz Sequéntia sancti Evangélii secúndum Matthǽum.
}\switchcolumn\portugues{
\cruz Continuação do santo Evangelho segundo S. Mateus.
}\switchcolumn*\latim{
\blettrine{I}{n} illo témpore: Locútus est Jesus ad turbas et ad discípulos suos, dicens: Super cáthedram Moysi sedérunt scribæ et pharisǽi. Omnia ergo, quæcúmque díxerint vobis, serváte et fácite: secúndum ópera vero eórum nolíte fácere: dicunt enim, et non fáciunt. Alligant enim ónera grávia et importabília, et impónunt in húmeros hóminum: dígito autem suo nolunt ea movére. Omnia vero ópera sua fáciunt, ut videántur ab homínibus: dilátant enim phylactéria sua, et magníficant fímbrias. Amant autem primos recúbitus in cenis, et primas cáthedras in synagógis, et salutatiónes in foro, et vocári ab homínibus Rabbi. Vos autem nolíte vocári Rabbi: unus est enim Magíster vester, omnes autem vos fratres estis. Et patrem nolíte vocáre vobis super terram, unus est enim Pater vester, qui in cœlis est. Nec vocémini magístri: quia Magíster vester unus est, Christus. Qui major est vestrum, erit miníster vester. Qui autem se exaltáverit, humiliábitur: et qui se humiliáverit, exaltábitur.
}\switchcolumn\portugues{
\blettrine{N}{aquele} tempo, Jesus falou às turbas e aos discípulos, dizendo: «Na cadeira de Moisés sentaram-se os escribas e os fariseus. Observai, pois, e fazei tudo o que vos disserem, mas não procedais segundo o exemplo das suas obras; pois eles ensinam, mas não praticam. Preparam fardos pesados e difíceis de levar e põem-nos às costas dos homens; mas eles nem com o dedo os querem mover. Praticam todas as obras, para serem vistos pelos homens; usam filactérios mais amplos e franjas mais compridas; querem tomar os primeiros lugares nos banquetes e as primeiras cadeiras nas sinagogas; procuram saudações na praça pública e querem que lhes chamem Rabi (Mestre). Vós, porém, não queirais que vos chamem Rabi, porque um só é o vosso Mestre e todos vós sois irmãos. Não chameis a ninguém na terra vosso pai, porque um só é o vosso Pai: Aquele que está nos céus. Não queirais ser chamados mestres, porque um só é o vosso Mestre: Cristo. O maior entre vós será vosso servo. ’Quem se exaltar a si mesmo será humilhado; e quem se humilhar será exaltado».
}\end{paracol}

\paragraphinfo{Ofertório}{Sl. 50, 3}
\begin{paracol}{2}\latim{
\rlettrine{M}{iserére} mei, Dómine, secúndum magnam misericórdiam tuam: dele, Dómine, iniquitátem meam.
}\switchcolumn\portugues{
\rlettrine{T}{ende} piedade de mim, Senhor, segundo a grandeza da vossa misericórdia; dignai-Vos, Senhor, esquecer a minha iniquidade.
}\end{paracol}

\paragraph{Secreta}
\begin{paracol}{2}\latim{
\rlettrine{S}{anctificatiónem} tuam nobis, Dómine, his mystériis operáre placátus: quæ nos et a terrénis purget vítiis, et ad cœléstia dona perdúcat. Per Dóminum \emph{\&c.}
}\switchcolumn\portugues{
\rlettrine{S}{enhor,} dignai-Vos benignamente santificar-nos pela virtude destes mystérios, a fim de que, estando purificados dos nossos vícios, alcancemos as recompensas celestiais. Por nosso Senhor \emph{\&c.}
}\end{paracol}

\paragraphinfo{Comúnio}{Sl. 9, 2-3}
\begin{paracol}{2}\latim{
\rlettrine{N}{arrábo} ómnia mirabília tua: lætábor, et exsultábo in te: psallam nómini tuo, Altíssime.
}\switchcolumn\portugues{
\rlettrine{P}{ublicarei} todas vossas maravilhas: em Vós me alegrarei e regozijarei: e cantarei hinos em louvor do vosso nome, ó Altíssimo.
}\end{paracol}

\paragraph{Postcomúnio}
\begin{paracol}{2}\latim{
\rlettrine{U}{t} sacris, Dómine, reddámur digni munéribus: fac nos tuis, quǽsumus, semper obœdíre mandátis. Per Dóminum nostrum \emph{\&c.}
}\switchcolumn\portugues{
\rlettrine{P}{ara} que sejamos dignos dos vossos sacrossantos dons, fazei, Senhor, Vos suplicamos, que obedeçamos sempre aos vossos mandamentos. Por nosso Senhor \emph{\&c.}
}\end{paracol}

\paragraph{Oração sobre o povo}
\begin{paracol}{2}\latim{
\begin{nscenter} Orémus. \end{nscenter}
}\switchcolumn\portugues{
\begin{nscenter} Oremos. \end{nscenter}
}\switchcolumn*\latim{
Humiliáte cápita vestra Deo.
}\switchcolumn\portugues{
Inclinai as vossas cabeças diante de Deus.
}\switchcolumn*\latim{
Propitiáre, Dómine, supplicatiónibus nostris, et animárum nostrárum medére languóribus: ut, remissióne percépta, in tua semper benedictióne lætámur. Per Dóminum \emph{\&c.}
}\switchcolumn\portugues{
Sede propício às nossas súplicas, Senhor, e curai as doenças das nossas almas, a fim de que, havendo alcançado o perdão, nos alegremos sempre com vossa bênção. Por nosso Senhor \emph{\&c.}
}\end{paracol}

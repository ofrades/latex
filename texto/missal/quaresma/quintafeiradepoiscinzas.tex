\subsectioninfo{Quinta-feira depois das Cinzas}{Estação em S. Jorge}

\paragraphinfo{Intróito}{Sl. 54, 17, 19, 20 \& 23}
\begin{paracol}{2}\latim{
\rlettrine{D}{um} clamárem ad Dóminum, exaudívit vocem meam ab his, qui appropínquant mihi, et humiliávit eos, qui est ante sǽcula et manet in ætérnum: jacta cogitátum tuum in Dómino, et ipse te enútriet. \emph{Ps. ibid., 2-3} Exáudi, Deus, oratiónem meam, et ne despéxeris deprecatiónem meam: inténde mihi et exáudi me.
℣. Gloria Patri \emph{\&c.}
}\switchcolumn\portugues{
\qlettrine{Q}{uando} invoquei o Senhor, ouviu Ele a minha voz e livrou-me daqueles que vinham contra mim: e Aquele que existe antes de todos os séculos e existirá eternamente os humilhará. Abandonai ao Senhor os vossos cuidados, pois Ele providenciará. \emph{Sl. ibid., 2-3} Ouvi, ó Deus, a minha oração e não desprezeis a minha humilde súplica.
℣. Glória ao Pai \emph{\&c.}
}\end{paracol}

\paragraph{Oração}
\begin{paracol}{2}\latim{
\rlettrine{D}{eus,} qui culpa offénderis, pæniténtia placáris: preces pópuli tui supplicántis propítius réspice; et flagélla tuæ iracúndiæ, quæ pro peccátis nostris merémur, avérte. Per Dóminum \emph{\&c.}
}\switchcolumn\portugues{
\slettrine{Ó}{} Deus, que Vos ofendeis com o pecado e aplacais com a penitência, ouvi propício as preces do vosso povo suplicante e afastai os flagelos da vossa ira, que merecemos pelos nossos pecados. Por nosso Senhor \emph{\&c.}
}\end{paracol}

\paragraphinfo{Epístola}{Is. 38, 1-6}
\begin{paracol}{2}\latim{
Léctio Isaíæ Prophétæ.
}\switchcolumn\portugues{
Lição do Profeta Isaías.
}\switchcolumn*\latim{
\rlettrine{I}{n} diébus illis: Ægrotávit Ezechías usque ad mortem: et introívit ad eum Isaías fílius Amos Prophéta, et dixit ei: Hæc dicit Dóminus: Dispóne dómui tuæ, quia moriéris tu, et non vives. Et convértit Ezechías fáciem suam ad paríetem, et orávit ad Dóminum, et dixit: Obsecro, Dómine, meménto, quæso, quómodo ambuláverim coram te in veritáte et in corde perfécto, et, quod bonum est in óculis tuis, fécerim. Et flevit Ezechías fletu magno. Et factum est verbum Dómini ad Isaíam, dicens: Vade, et dic Ezechíæ: Hæc dicit Dóminus, Deus David patris tui: Audívi oratiónem tuam, et vidi lácrimas tuas: ecce, ego adjíciam super dies tuos quíndecim annos: et de manu regis Assyriórum éruam te et civitátem istam, et prótegam eam, ait Dóminus omnípotens.
}\switchcolumn\portugues{
\rlettrine{N}{aqueles} dias, Ezequias foi atacado de doença mortal. Então, o Profeta Isaías, filho de Amós, veio ter com ele e disse-lhe: «Isto diz o Senhor: «Dispõe os negócios da tua casa, porque morrerás e não viverás». Ezequias voltou o rosto para a parede e dirigiu esta oração ao Senhor: «Lembrai-Vos, Senhor, Vos suplico, de que tenho caminhado na vossa presença com a verdade e com o coração íntegro e tenho praticado o que é agradável a vossos olhos». Depois, Ezequias chorou amargamente. Então o Senhor falou a Isaías nestes termos: «Ide e dizei a Ezequias: Isto diz o Senhor, Deus de David, vosso Pai: Ouvi a vossa oração e vi as vossas lágrimas; eis que acrescentarei ainda a vossa vida com quinze anos e vos livrarei, e a esta terra, do jugo do rei da Assíria e a protegerei»: diz o Senhor omnipotente.
}\end{paracol}

\paragraphinfo{Gradual}{Sl. 54, 23, 17, 18 \& 19}
\begin{paracol}{2}\latim{
\qlettrine{J}{acta} cogitátum tuum in Dómino, et ipse te enútriet. ℣. Dum clamárem ad Dóminum, exaudívit vocem meam ab his, qui appropínquant mihi.
}\switchcolumn\portugues{
\rlettrine{A}{bandonai} ao Senhor os vossos cuidados, pois Ele providenciará. ℣. Quando invoquei o Senhor, ouviu Ele a minha voz e livrou-me daqueles que vinham contra mim.
}\end{paracol}

\paragraphinfo{Evangelho}{Mt. 8, 5-13}
\begin{paracol}{2}\latim{
\cruz Sequéntia sancti Evangélii secúndum Matthǽum.
}\switchcolumn\portugues{
\cruz Continuação do santo Evangelho segundo S. Lucas.
}\switchcolumn*\latim{
\blettrine{I}{n} illo témpore: Cum introísset Jesus Caphárnaum, accéssit ad eum centúrio, rogans eum, et dicens: Dómine, puer meus jacet in domo paralýticus, et male torquétur. Et ait illi Jesus: Ego véniam et curábo eum. Et respóndens centúrio, ait: Dómine, non sum dignus, ut intres sub tectum meum: sed tantum dic verbo, et sanábitur puer meus. Nam et ego homo sum sub potestáte constitútus, habens sub me mílites, et dico huic: Vade, et vadit; et alii: Veni, et venit; et servo meo: Fac hoc, et facit. Audiens autem Jesus, mirátus est, et sequéntibus se dixit: Amen, dico vobis, non invéni tantam fidem in Israël. Dico autem vobis, quod multi ab Oriénte et Occidénte vénient, et recúmbent cum Abraham et Isaac et Jacob in regno cœlórum: fílii autem regni ejiciéntur in ténebras exterióres: ibi erit fletus et stridor déntium. Et dixit Jesus centurióni: Vade, et, sicut credidísti, fiat tibi. Et sanátus est puer in illa hora.
}\switchcolumn\portugues{
\blettrine{N}{aquele} tempo, entrando Jesus em Cafarnaum, aproximou-se dele um centurião, pedindo-Lhe e dizendo: «Senhor, o meu servo jaz em casa paralítico e sofre gravemente». Jesus disse-lhe: «Eu irei e o curarei». Mas o centurião respondeu: «Senhor, não sou digno de que entreis em minha casa; dizei somente uma palavra e o meu servo será curado. Pois eu, posto que seja um homem sujeito a outros superiores, tenho soldados debaixo das minhas ordens. E digo a um: vai; e ele vai. E digo a outro: vem; e ele vem. E digo ao meu servo: faz isto; e ele faz». Ouvindo Jesus isto, ficou admirado e disse aos que O seguiam: «Em verdade vos digo que nunca encontrei tão grande fé em Israel! Declaro-vos que muitos virão do Oriente e do Ocidente e tomarão lugar no banquete com Abraão, Isaque e Jacob, no reino dos céus; mas os filhos do reino serão lançados nas trevas exteriores, onde só haverá pranto e ranger de dentes». Então Jesus disse ao centurião: «Vai; e, assim como acreditaste, assim acontecerá». E naquela hora o servo foi curado.
}\end{paracol}

\paragraphinfo{Ofertório}{Sl. 24, 1-3}
\begin{paracol}{2}\latim{
\rlettrine{A}{d} te, Dómine, levávi ánimam meam: Deus meus, in te confído, non erubéscam: neque irrídeant me inimíci mei: étenim univérsi, qui te exspéctant, non confundéntur.
}\switchcolumn\portugues{
\rlettrine{E}{levei} a minha alma até Vós, Senhor, meu Deus: tenho confiança em Vós: não serei confundido; não permitireis que meus inimigos zombem de mim, porque todos aqueles que esperam em Vós, não serão confundidos.
}\end{paracol}

\paragraph{Secreta}
\begin{paracol}{2}\latim{
\rlettrine{S}{acrifíciis} præséntibus, Dómine, quǽsumus, inténde placátus: ut et devotióni nostræ profíciant et salúti. Per Dóminum \emph{\&c.}
}\switchcolumn\portugues{
\rlettrine{S}{enhor,} Vos suplicamos, olhai benigno para o presente sacrifício, a fim de que sirva para aumento da nossa devoção e para a nossa salvação. Por nosso Senhor \emph{\&c.}
}\end{paracol}

\paragraphinfo{Comúnio}{Sl. 50, 21}
\begin{paracol}{2}\latim{
\rlettrine{A}{cceptábis} sacrifícium justítiæ, oblatiónes et holocáusta, super altáre tuum, Dómine.
}\switchcolumn\portugues{
\rlettrine{R}{ecebeis} sobre o vosso altar, Senhor, o sacrifício de justiça, as ofertas e os holocaustos.
}\end{paracol}

\paragraph{Postcomúnio}
\begin{paracol}{2}\latim{
\rlettrine{C}{œléstis} doni benedictióne percépta: súpplices te, Deus omnípotens, deprecámur; ut hoc idem nobis et sacraménti causa sit et salútis. Per Dóminum \emph{\&c.}
}\switchcolumn\portugues{
\rlettrine{H}{avendo} recebido a bênção do dom celestial, humildemente Vos suplicamos, Deus omnipotente, que este mesmo dom seja para nós motivo de satisfação e de salvação. Por nosso Senhor \emph{\&c.}
}\end{paracol}

\paragraph{Oração sobre o povo}
\begin{paracol}{2}\latim{
\begin{nscenter} Orémus. \end{nscenter}
}\switchcolumn\portugues{
\begin{nscenter} Oremos. \end{nscenter}
}\switchcolumn*\latim{
Humiliáte cápita vestra Deo.
}\switchcolumn\portugues{
Inclinai as vossas cabeças diante de Deus.
}\switchcolumn*\latim{
Parce, Dómine, parce populo tuo: ut, dignis flagellatiónibus castigátus, in tua miseratióne respíret. Per Dóminum \emph{\&c.}
}\switchcolumn\portugues{
Perdoai, Senhor, perdoai ao vosso povo, a fim de que, castigado merecidamente com os flagelos, possa, enfim, respirar, bafejado pela vossa misericórdia. Por nosso Senhor \emph{\&c.}
}\end{paracol}

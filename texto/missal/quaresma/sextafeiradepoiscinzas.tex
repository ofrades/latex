\subsectioninfo{Sexta-feira depois das Cinzas}{Estação na Igreja dos SS. João e Paulo, Mártires}\label{sextafeiradepoiscinzas}

\paragraphinfo{Intróito}{Sl. 29, 11}
\begin{paracol}{2}\latim{
\rlettrine{A}{udívit} Dóminus, et misértus est mihi: Dóminus factus est adjútor meus. \emph{Ps. ibid., 2} Exaltábo te, Dómine, quóniam suscepísti me: nec delectásti inimícos meos super me.
℣. Gloria Patri \emph{\&c.}
}\switchcolumn\portugues{
\rlettrine{O}{} Senhor ouviu-me e teve piedade de mim. O Senhor veio em meu auxílio. \emph{Sl. ibid., 2} Senhor, louvar-Vos-ei porque me socorrestes e não permitistes que meus inimigos mofassem de mim.
℣. Glória ao Pai \emph{\&c.}
}\end{paracol}

\paragraph{Oração}
\begin{paracol}{2}\latim{
\rlettrine{I}{nchoáta} jejúnia, quǽsumus, Dómine, benígno favore proséquere: ut observántiam, quam corporáliter exhibémus, méntibus etiam sincéris exercére valeámus. Per Dóminum \emph{\&c.}
}\switchcolumn\portugues{
\rlettrine{S}{enhor,} Vos suplicamos, acolhei com benigno favor os jejuns agora começados, a fim de que, exercendo corporalmente esta observância, possamos praticá-la com o coração sincero. Por nosso Senhor \emph{\&c.}
}\end{paracol}

\paragraphinfo{Epístola}{Is. 58, 1-0}
\begin{paracol}{2}\latim{
Léctio Isaíæ Prophétæ.
}\switchcolumn\portugues{
Lição do Profeta Isaías.
}\switchcolumn*\latim{
\rlettrine{H}{æc} dicit Dóminus Deus: Clama, ne cesses: quasi tuba exálta vocem tuam: et annúntia pópulo meo scélera eórum, et dómui Jacob peccáta eórum. Me étenim de die in diem quærunt, et scire vias meas volunt: quasi gens, quæ justítiam fécerit, et judícium Dei sui non derelíquerit: rogant me judícia justítiæ: appropinquáre Deo volunt. Quare jejunávimus, et non aspexísti: humiliávimus ánimas nostras, et nescísti? Ecce, in die jejúnii vestri invénitur volúntas vestra, et omnes debitóres vestros repétitis. Ecce, ad lites et contentiónes jejunátis, et percútitis pugno ímpie. Nolíte jejunáre sicut usque ad hanc diem, ut audiátur in excélso clamor vester. Numquid tale est jejúnium, quod elégi, per diem afflígere hóminem ánimam suam? numquid contorquére quasi círculum caput suum, et saccum et cínerem stérnere? numquid istud vocábis jejúnium, et diem acceptábilem Dómino? Nonne hoc est magis jejúnium quod elégi? dissólve colligatiónes impietátis, solve fascículos depriméntes: dimítte eos, qui confrácti sunt, líberos, et omne onus dirúmpe. Frange esuriénti panem tuum, et egénos vagósque induc in domum tuam: cum víderis nudum, operi eum, et carnem tuam ne despéxeris. Tunc erúmpet quasi mane lumen tuum, et sánitas tua cítius oriétur, et anteíbit fáciem tuam justítia tua, et glória Dómini cóllige t te. Tunc invocábis, et Dóminus exáudiet: clamábis, et dicet: Ecce, adsum. Quia miséricors sum, Dóminus, Deus tuus.
}\switchcolumn\portugues{
\rlettrine{I}{sto} diz o Senhor Deus: «Clama incessantemente; faz ouvir a tua voz, como uma trombeta, denunciando ao meu povo os seus crimes e à casa de Jacob os seus pecados. Eles procuram-me quotidianamente e desejam conhecer os meus caminhos, como uma nação que tenha procedido sempre com justiça e não haja abandonado a lei do seu Deus. Perguntam-me os direitos da justiça; querem aproximar-se de Deus. De que nos aproveita havermos jejuado, dizem eles, se não atendeis a isso? De que serve que nos humilhemos, se mostrais que ignorais tal coisa? No dia do jejum a vossa má vontade persiste; demandais os vossos devedores; disputais e questionais reciprocamente até vos agredirdes impiedosamente uns aos outros com pancadas. Não jejueis mais assim, pois com um tal jejum a vossa voz não será ouvida no alto. Porventura o jejum que me agrada consiste em o homem humilhar em um só dia a sua alma, ou em curvar a sua cabeça, como que formando um círculo, ou em usar o «saco e a cinza»? É a isto que chamais jejum e dia agradável ao Senhor? O jejum que me agrada é aquele em que quebrais as cadeias da impiedade; em que desatais os nós do jugo, que vos oprime; em que aliviais os oprimidos; e em que terminais as violências. O jejum que me agrada é antes aquele em que repartis o pão pelos que têm fome; em que recolheis em vossa casa os infelizes sem asilo; e em que, encontrando um homem em nudez, lhe dais de vestir, e não desprezais a vossa carne! Então, sim, a vossa luz brilhará, como a aurora; a vossa saúde cedo voltará; a vossa justiça caminhará ante a vossa face; e a glória do Senhor vos inundará. Então, sim, invocareis o Senhor, e vos escutará. Clamareis por Ele, e vos responderá: «Eis-me aqui, pois sou misericordioso: Eu, o Senhor, vosso Deus».
}\end{paracol}

\paragraphinfo{Gradual}{Sl. 26, 4}
\begin{paracol}{2}\latim{
\rlettrine{U}{nam} pétii a Dómino, hanc requíram, ut inhábitem in domo Dómini. ℣. Ut vídeam voluptátem Dómini, et prótegar a templo sancto ejus.
}\switchcolumn\portugues{
\rlettrine{U}{ma} só coisa peço ao Senhor e continuarei a pedi-la: que possa habitar na sua Casa: ℣. Para que possa gozar os seus esplendores e ser protegido no seu santo templo.
}\end{paracol}

\paragraphinfo{Trato}{Página \pageref{tratoquartacinzas}}

\paragraphinfo{Evangelho}{Mt. 5, 43-48 \& 6, 1-4}
\begin{paracol}{2}\latim{
\cruz Sequéntia sancti Evangélii secúndum Matthǽum.
}\switchcolumn\portugues{
\cruz Continuação do santo Evangelho segundo S. Mateus.
}\switchcolumn*\latim{
\blettrine{I}{n} illo témpore: Dixit Jesus discípulis suis: Audístis, quia dictum est: Diliges próximum tuum, et odio habébis inimícum tuum. Ego autem dico vobis: Dilígite inimícos vestros, benefácite his, qui odérunt vos, et oráte pro persequéntibus et calumniántibus vos, ut sitisfílii Patris vestri, qui in cœlis est: qui solem suum oriri facit super bonos et malos, et pluit super justos et injústos. Si enim dilígitis eos, qui vos díligunt, quam mercédem habébitis? nonne et publicáni hoc fáciunt? Et si salutavéritis fratres vestros tantum, quid ámplius fácitis? nonne et éthnici hoc fáciunt? Estóte ergo vos perfécti, sicut et Pater vester cœléstis perféctus est. Atténdite, ne justítiam vestram faciátis coram homínibus, ut videámini ab eis: alióquin mercédem non habébitis apud Patrem vestrum, qui in cœlis est. Cum ergo facis eleemósynam, noli tuba cánere ante te, sicut hypócritæ fáciunt in synagógis et in vicis, ut honorificéntur ab homínibus. Amen, dico vobis, recepérunt mercédem suam. Te autem faciénte eleemósynam, nésciat sinístra tua, quid fáciat déxtera tua, ut sit eleemósyna tua in abscóndito, et Pater tuus, qui videt in abscóndito, reddet tibi.
}\switchcolumn\portugues{
\blettrine{N}{aquele} tempo, disse Jesus aos seus discípulos: «Aprendestes o que vos ensinaram: «Amareis o vosso próximo e aborrecereis o vosso inimigo» ? Eu, porém, digo-vos: amai os vossos inimigos; fazei bem àqueles que vos odeiam; rezai pelos que vos perseguem e caluniam, a fim de que sejais filhos do vosso Pai, que está nos céus, o qual faz nascer o sol para os maus e bons e chover para os justos e injustos. Se amais os que vos amam, que recompensa mereceis? Não procedem do mesmo modo os publicanos? Se saudais sàmente os vossos irmãos, que fazeis mais do que os outros? Não procedem assim os pagãos? Sede, portanto, perfeitos como o vosso Pai celestial é perfeito. Tende cuidado, não pratiqueis obras justas na presença dos homens, com intenção de que sejam vistas por eles, pois não alcançareis recompensa junto de vosso Pai, que está nos céus. Por isso, quando derdes esmola, não toqueis a trombeta para serdes elogiados, como fazem os hipócritas nas sinagogas e nas ruas. Na verdade vos digo: eles já receberam a recompensa. Mas, quando derdes esmola, procedei de modo que a mão esquerda não saiba o que faz a direita, a fim de que a vossa esmola seja oculta, e vosso Pai, que conhece todos os segredos, vos recompensará».
}\end{paracol}

\paragraphinfo{Ofertório}{Sl. 118, 154 \& 125}
\begin{paracol}{2}\latim{
\rlettrine{D}{ómine,} vivífica me secúndum elóquium tuum: ut sciam testimónia tua.
}\switchcolumn\portugues{
\rlettrine{V}{ivificai-me,} Senhor, segundo a vossa palavra, para que conheça os vossos preceitos.
}\end{paracol}

\paragraph{Secreta}
\begin{paracol}{2}\latim{
\rlettrine{S}{acrifícium,} Dómine, observántiæ quadragesimális, quod offérimus, præsta, quǽsumus: ut tibi et mentes nostras reddat accéptas, et continéntiæ promptióris nobis tríbuat facultátem. Per Dóminum \emph{\&c.}
}\switchcolumn\portugues{
\rlettrine{C}{oncedei-nos,} Senhor, Vos suplicamos, que o sacrifício do preceito quaresmal, que oferecemos, Vos torne agradáveis as nossas almas, e nos alcance a graça de praticarmos mais facilmente a continência. Por nosso Senhor \emph{\&c.}
}\end{paracol}

\paragraphinfo{Comúnio}{Sl. 2, 11-12}
\begin{paracol}{2}\latim{
\rlettrine{S}{ervi} te Dómino in timóre, et exsultáte ei cum tremóre: apprehéndite disciplínam, ne pereátis de via justa.
}\switchcolumn\portugues{
\rlettrine{O}{bedecei} ao Senhor com temor e alegrai-vos n’Ele com tremor: abraçai a sua lei para que vos não afasteis dos caminhos direitos.
}\end{paracol}

\paragraph{Postcomúnio}
\begin{paracol}{2}\latim{
\rlettrine{S}{píritum} nobis, Dómine, tuæ cantátis infúnde: ut, quos uno pane cœlésti satiásti, tua fácias pietáte concórdes. Per Dóminum \emph{\&c.}
}\switchcolumn\portugues{
\rlettrine{I}{nfundi} em nossas almas, Senhor, o espírito da caridade, a fim de que aqueles que saciastes com o mesmo pão celestial permaneçam em concórdia pela vossa bondade. Por nosso Senhor \emph{\&c.}
}\end{paracol}

\paragraph{Oração sobre o povo}
\begin{paracol}{2}\latim{
\begin{nscenter} Orémus. \end{nscenter}
}\switchcolumn\portugues{
\begin{nscenter} Oremos. \end{nscenter}
}\switchcolumn*\latim{
Humiliáte cápita vestra Deo.
}\switchcolumn\portugues{
Inclinai as vossas cabeças diante de Deus.
}\switchcolumn*\latim{
Tuére, Dómine, pópulum tuum et ab ómnibus peccátis cleménter emúnda: quia nulla ei nocébit advérsitas, si nulla ei dominétur iníquitas. Per Dóminum \emph{\&c.}
}\switchcolumn\portugues{
Senhor, defendei o vosso povo; e, para que nenhuma adversidade o possa prejudicar nem nenhuma iniquidade o possa dominar, purificai-o pela vossa clemência. Por nosso Senhor \emph{\&c.}
}\end{paracol}

\subsection{Terceiro Domingo da Quaresma}

\paragraphinfo{Intróito}{Sl. 24, 15-16}
\begin{paracol}{2}\latim{
\rlettrine{O}{culi} mei semper ad Dóminum, quia ipse evéllet de láqueo pedes meos: réspice in me, et miserére mei, quóniam únicus et pauper sum ego. \emph{Ps. ibid., 1-2} Ad te, Dómine, levávi ánimam meam: Deus meus, in te confído, non erubéscam.
℣. Gloria Patri \emph{\&c.}
}\switchcolumn\portugues{
\rlettrine{O}{s} meus olhos estão sempre fitos no Senhor, pois Ele livrará os meus pés do laço. Volvei vossos olhos para mim e tende misericórdia de mim, meu Deus, pois sou só e pobre. \emph{Sl. ibid., 1-2} A Vós, Senhor, elevei a minha alma: meu Deus, confio em Vós, não ficarei envergonhado.
℣. Glória ao Pai \emph{\&c.}
}\end{paracol}

\paragraph{Oração}
\begin{paracol}{2}\latim{
\qlettrine{Q}{uǽsumus,} omnípotens Deus, vota humílium réspice: atque, ad defensiónem nostram, déxteram tuæ majestátis exténde. Per Dóminum nostrum \emph{\&c.}
}\switchcolumn\portugues{
\slettrine{Ó}{} Deus omnipotente, Vos suplicamos, atendei aos humildes votos da nossa humildade; e que a dextra da vossa majestade nos conceda a sua protecção. Por nosso Senhor \emph{\&c.}
}\end{paracol}

\paragraphinfo{Epístola}{Ef. 5, 1 9}
\begin{paracol}{2}\latim{
Léctio Epístolæ beáti Pauli Apóstoli ad Ephésios.
}\switchcolumn\portugues{
Lição da Ep.ª do B. Ap.º Paulo aos Efésios.
}\switchcolumn*\latim{
\rlettrine{F}{ratres:} Estote imitatores Dei, sicut fílii caríssimi: et ambuláte in dilectióne, sicut et Christus dilexit nos, et tradidit semetipsum pro nobis oblatiónem, et hostiam Deo in odorem suavitátis. Fornicatio autem et omnis immunditia aut avaritia nec nominetur in vobis, sicut decet sanctos: aut turpitudo aut stultiloquium aut scurrilitas, quæ ad rem non pertinet: sed magis gratiárum actio. Hoc enim scitóte intelligentes, quod omnis fornicator aut immundus aut avarus, quod est idolorum servitus, non habet hereditátem in regno Christi et Dei. Nemo vos sedúcat inanibus verbis: propter hæc enim venit ira Dei in filios diffidéntiæ. Nolíte ergo effici participes eórum. Erátis enim aliquando tenebrae: nunc autem lux in Dómino. Ut fílii lucis ambuláte: fructus enim lucis est in omni bonitate et justítia et veritáte.
}\switchcolumn\portugues{
\rlettrine{M}{eus} irmãos: Sede imitadores de Deus, como filhos caríssimos, e vivei na caridade, como Cristo, que nos amou e se entregou por nós a Deus, como uma oblação e uma hóstia de agradável odor. Que a incontinência, toda a impureza e a avareza, nem sequer sejam nomeadas entre vós, como é próprio dos santos. O mesmo vos digo quanto a palavras desonestas, torpes, insensatas, ou ainda jocosas, pois todas são impróprias. Proferi antes palavras de acções de graças; pois nenhum incontinente, impuro ou avarento o que é idolatra terá parte no reino de Cristo e de Deus. Que ninguém vos seduza com discursos vãos; porque é por causa destas coisas que a ira de Deus cai sobre os filhos da rebeldia. Não tenhais nada de comum com eles; pois, outrora, estáveis nas trevas, mas, agora, possuís a luz do Senhor. Procedei como filhos da luz, cujo fruto consiste em toda a espécie de bondade, justiça e verdade.
}\end{paracol}

\paragraphinfo{Gradual}{Sl. 9, 20 \& 4}
\begin{paracol}{2}\latim{
\rlettrine{E}{xsúrge,} Dómine, non præváleat homo: judicéntur gentes in conspéctu tuo. ℣. In converténdo inimícum meum retrórsum, infirmabúntur, et períbunt a facie tua.
}\switchcolumn\portugues{
\rlettrine{E}{rguei-Vos,} Senhor, para que o homem não triunfe: que os povos sejam julgados na vossa presença! ℣. Quando o meu inimigo tiver fugido, eles tremerão e morrerão diante de Vós.
}\end{paracol}

\paragraphinfo{Trato}{Sl. 122, 1-3}
\begin{paracol}{2}\latim{
\rlettrine{A}{d} te levávi óculos meos, qui hábitas in cœlis. ℣. Ecce, sicut óculi servórum in mánibus dominórum suórum. ℣. Et sicut óculi ancíllæ in mánibus dóminæ suæ: ita óculi nostri ad Dóminum, Deum nostrum, donec misereátur nostri. ℣. Miserére nobis, Dómine, miserére nobis.
}\switchcolumn\portugues{
\rlettrine{A}{} Vós, que habitais nos céus, ergui os meus olhos. Como os olhos dos servos estão fixos na mão do seu senhor. ℣. E os olhos da escrava nas mãos da sua senhora; assim os nossos olhos se volvem e fixam em o Senhor, nosso Deus, até que tenha piedade de nós. ℣. Tende piedade de nós, Senhor, tende piedade de nós.
}\end{paracol}

\paragraphinfo{Evangelho}{Lc. 11, 14-28}
\begin{paracol}{2}\latim{
\cruz Sequéntia sancti Evangélii secúndum Lucam.
}\switchcolumn\portugues{
\cruz Continuação do santo Evangelho segundo S. Lucas.
}\switchcolumn*\latim{
\blettrine{I}{n} illo témpore: Erat Jesus ejíciens dæmónium, et illud erat mutum. Et cum ejecísset dæmónium, locútus est mutus, et admirátæ sunt turbæ. Quidam autem ex eis dixérunt: In Beélzebub, príncipe dæmoniórum, éjicit dæmónia. Et alii tentántes, signum de cœlo quærébant ab eo. Ipse autem ut vidit cogitatiónes eórum, dixit eis: Omne regnum in seípsum divísum desolábitur, et domus supra domum cadet. Si autem et sátanas in seípsum divísus est, quómodo stabit regnum ejus? quia dícitis, in Beélzebub me ejícere dæmónia. Si autem ego in Beélzebub ejício dæmónia: fílii vestri in quo ejíciunt? Ideo ipsi júdices vestri erunt. Porro si in dígito Dei ejício dæmónia: profécto pervénit in vos regnum Dei. Cum fortis armátus custódit átrium suum, in pace sunt ea, quæ póssidet. Si autem fórtior eo supervéniens vícerit eum, univérsa arma ejus áuferet, in quibus confidébat, et spólia ejus distríbuet. Qui non est mecum, contra me est: et qui non cólligit mecum, dispérgit. Cum immúndus spíritus exíerit de hómine, ámbulat per loca inaquósa, quærens réquiem: et non invéniens, dicit: Revértar in domum meam, unde exivi. Et cum vénerit, invénit eam scopis mundátam, et ornátam. Tunc vadit, et assúmit septem alios spíritus secum nequióres se, et ingréssi hábitant ibi. Et fiunt novíssima hóminis illíus pejóra prióribus. Factum est autem, cum hæc díceret: extóllens vocem quædam múlier de turba, dixit illi: Beátus venter, qui te portávit, et úbera, quæ suxísti. At ille dixit: Quinímmo beáti, qui áudiunt verbum Dei, et custódiunt illud.
}\switchcolumn\portugues{
\blettrine{N}aquele{} tempo, estava Jesus a expulsar um demónio, o qual era mudo. Logo que o expulsou, o mudo falou, ficando as turbas admiradas. Alguns dos presentes disseram: «É pelo poder de Belzebu, príncipe dos demónios, que Ele expulsa os demónios». Outros, para O tentarem, pediam que lhes desse um sinal do céu. Jesus, conhecendo os pensamentos de todos, disse-lhes: «Todo o reino dividido contra si próprio tornar-se-á em um reino devastado; pois as casas cairão umas sobre as outras. Se, portanto, Satanás está dividido contra si próprio, como subsistirá o seu reino? Vós dizeis que é pelo poder de Belzebu que expulso os demónios? Ora, se expulso os demónios pelo poder de Belzebu, com que poder os expulsarão os vossos filhos? Por isso eles serão os vossos juízes. Porém, se é pelo poder de Deus que expulso os demónios, certamente o reino de Deus já chegou a vós. Quando um homem forte e bem armado guarda a entrada de sua casa, está em paz o que lhe pertence; mas, se sobrevier um outro mais forte do que ele, vence-o, tira-lhe as armas, em que confiava, e repartirá os seus despojos. Quem não está comigo, está contra mim; e quem não economiza, dissipa. Quando o espírito impuro sai dum homem, vai pelos lugares desertos em procura de repouso. Não o encontrando, diz: volto para casa donde saí. E quando lá chega encontra casa limpa e adornada. Então, retira-se, vai buscar e traz consigo outros sete espíritos piores do que ele, e, entrando na casa, estabelecem-se nela. Este último estado deste homem tornou-se pior do que o primeiro». Ora aconteceu que, enquanto Jesus falava, uma mulher, que estava a ouvi-l’O, elevando a voz no meio da multidão, disse-Lhe: «Bem-aventurado o seio que te encerrou e os peitos que te aleitaram!». Ele respondeu: «Bem-aventurados antes aqueles que ouvem a palavra de Deus e a observam».
}\end{paracol}

\paragraphinfo{Ofertório}{Sl. 18, 9, 10, 11 \& 12}
\begin{paracol}{2}\latim{
\qlettrine{J}{ustítiæ} Dómini rectæ, lætificántes corda, et judícia ejus dulci ora super mel et favum: nam et servus tuus custódit ea.
}\switchcolumn\portugues{
\rlettrine{O}{s} preceitos do Senhor são rectos, alegrando o coração: e os seus juízos são mais doces que o favo de mel; assim, pois, ó Deus, o vosso servo guarda-os fielmente.
}\end{paracol}

\paragraph{Secreta}
\begin{paracol}{2}\latim{
\rlettrine{H}{æc} hóstia, Dómine, quǽsumus, emúndet nostra
delícta: et, ad sacrifícium celebrándum, subditórum tibi córpora mentésque sanctíficet. Per Dóminum \emph{\&c.}
}\switchcolumn\portugues{
\qlettrine{Q}{ue} esta hóstia, Senhor, Vos suplicamos, nos purifique dos nossos pecados e santifique os corpos e as almas dos vossos servos, a fim de que celebrem dignamente este sacrifício. Por nosso Senhor \emph{\&c.}
}\end{paracol}

\paragraphinfo{Comúnio}{Sl. 83, 4-5}
\begin{paracol}{2}\latim{
\rlettrine{P}{asser} invénit sibi domum, et turtur nidum, ubi repónat pullos suos: altária tua, Dómine virtútum, Rex meus, et Deus meus: beáti, qui hábitant in domo tua, in sǽculum sǽculi laudábunt te.
}\switchcolumn\portugues{
\rlettrine{O}{} pássaro encontra um abrigo e a rola um ninho, para aí criarem os filhos. Assim eu encontre os vossos altares, Senhor dos exércitos, meu Rei e meu Deus! Felizes os que habitam na vossa casa e Vos louvam em todos os séculos dos séculos.
}\end{paracol}

\paragraph{Postcomúnio}
\begin{paracol}{2}\latim{
\rlettrine{A}{} cunctis nos, quǽsumus, Dómine, reátibus et perículis propitiátus absólve: quos tanti mystérii tríbuis esse partícipes. Per Dóminum \emph{\&c.}
}\switchcolumn\portugues{
\rlettrine{S}{enhor,} assim como nos tornastes participantes de tão grande mistério, assim também, pela vossa misericórdia, livrai-nos de todos os pecados e perigos. Por nosso Senhor \emph{\&c.}
}\end{paracol}

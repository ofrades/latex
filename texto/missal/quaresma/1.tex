\subsection{Primeiro Domingo da Quaresma}

\paragraphinfo{Intróito}{Sl. 90, 15 \& 16}
\begin{paracol}{2}\latim{
\rlettrine{I}{nvocábit} me, et ego exáudiam eum: erípiam eum, et glorificábo eum: longitúdine diérum adimplébo eum. \emph{Ps. ibid., 1} Qui hábitat in adjutório Altíssimi, in protectióne Dei cœli commorábitur.
℣. Gloria Patri \emph{\&c.}
}\switchcolumn\portugues{
\rlettrine{I}{nvocar-me-á,} e ouvi-lo-ei: livrá-lo-ei, e glorificá-lo-ei: conceder-lhe-ei longa vida. \emph{Sl. ibid., 1} Aquele que habita à sombra do Altíssimo descansará sob a protecção de Deus do céu.
℣. Glória ao Pai \emph{\&c.}
}\end{paracol}

\paragraph{Oração}
\begin{paracol}{2}\latim{
\rlettrine{D}{eus,} qui Ecclésiam tuam ánnua quadragesimáli observatióne puríficas: præsta famíliæ tuæ; ut, quod a te obtinére abstinéndo nítitur, hoc bonis opéribus exsequátur. Per Dóminum \emph{\&c.}
}\switchcolumn\portugues{
\slettrine{Ó}{} Deus, que purificais a vossa Igreja com a observância anual da quaresma, concedei à vossa família que o que ela deseja alcançar pela abstinência o pratique com suas boas obras. Por nosso Senhor \emph{\&c.}
}\end{paracol}

\paragraphinfo{Epístola}{2 Cor. 6, 1-10}
\begin{paracol}{2}\latim{
Léctio Epístolæ beáti Pauli Apóstoli ad Corínthios.
}\switchcolumn\portugues{
Lição da Ep.ª do B. Ap.º Paulo aos Coríntios.
}\switchcolumn*\latim{
\rlettrine{F}{ratres:} Exhortámur vos, ne in vácuum grátiam Dei recipiátis. Ait enim: Témpore accépto exaudívi te, et in die salútis adjúvi te. Ecce, nunc tempus acceptábile, ecce, nunc dies salútis. Némini dantes ullam offensiónem, ut non vituperétur ministérium nostrum: sed in ómnibus exhibeámus nosmetípsos sicut Dei minístros, in multa patiéntia, in tribulatiónibus, in necessitátibus, in angústiis, in plagis, in carcéribus, in seditiónibus, in labóribus, in vigíliis, in jejúniis, in castitáte, in sciéntia, in longanimitáte, in suavitáte, in Spíritu Sancto, in caritáte non ficta, in verbo veritátis, in virtúte Dei, per arma justítiæ a dextris et a sinístris: per glóriam et ignobilitátem: per infámiam et bonam famam: ut seductóres et veráces: sicut qui ignóti et cógniti: quasi moriéntes et ecce, vívimus: ut castigáti et non mortificáti: quasi tristes, semper autem gaudéntes: sicut egéntes, multos autem locupletántes: tamquam nihil habéntes et ómnia possidéntes.
}\switchcolumn\portugues{
\rlettrine{M}{eus} irmãos: Vos exortamos a que não recebais em vão a graça de Deus, pois Ele diz: «No tempo propício ouço-te, e no dia da salvação socorro-te». Eis, pois, agora o tempo propício; eis agora os dias da salvação. Não dêmos a ninguém escândalo, para que o nosso ministério não seja censurado; mas portemo-nos em tudo como ministros de Deus, com muita paciência nas tribulações, nas necessidades, nas angústias, nas ofensas corporais, nas prisões, nas sedições, nos trabalhos, nas vigílias, nos jejuns, pela castidade, pela ciência, pela longanimidade, pela mansidão, pelo Espírito Santo, pela caridade sincera, pela palavra da verdade, pelo poder de Deus, pelas armas da justiça com que devemos combater, à direita ou à esquerda, na honra ou na desonra, na boa reputação ou na infâmia; sendo julgados sedutores, ainda que sejamos sinceros e verdadeiros; sendo reputados desconhecidos, e, contudo, sendo bem conhecidos; pensando eles que éramos moribundos, mas estando bem vivos; como castigados, mas escapando à morte; como tristes, mas sempre alegres; como pobres, mas enriquecendo os outros; como não tendo nada, mas possuindo tudo.
}\end{paracol}

\paragraphinfo{Gradual}{Sl. 90,11-1 2}
\begin{paracol}{2}\latim{
\rlettrine{A}{ngelis} suis Deus mandávit de te, ut custódiant te in ómnibus viis tuis. ℣. In mánibus portábunt te, ne umquam offéndas ad lápidem pedem tuum.
}\switchcolumn\portugues{
\rlettrine{D}{eus} ordenou aos seus Anjos que te guardassem em todos teus caminhos. ℣. Levar-te-ão nas suas mãos, para que não tropeces nas pedras.
}\end{paracol}

\paragraphinfo{Trato}{ibid., 1-7 \& 11-16}
\begin{paracol}{2}\latim{
\qlettrine{Q}{ui} hábitat in adjutório Altíssimi, in protectióne Dei cœli commorántur. ℣. Dicet Dómino: Suscéptor meus es tu et refúgium meum: Deus meus, sperábo in eum. ℣. Quóniam ipse liberávit me de láqueo venántium et a verbo áspero. ℣. Scápulis suis obumbrábit tibi, et sub pennis ejus sperábis. ℣. Scuto circúmdabit te véritas ejus: non timébis a timóre noctúrno. ℣. A sagítta volánte per diem, a negótio perambulánte in ténebris, a ruína et dæmónio meridiáno. ℣. Cadent a látere tuo mille, et decem mília a dextris tuis: tibi autem non appropinquábit. ℣. Quóniam Angelis suis mandávit de te, ut custódiant te in ómnibus viis tuis. ℣. In mánibus portábunt te, ne umquam offéndas ad lápidem pedem tuum. ℣. Super áspidem et basilíscum ambulábis, et conculcábis leónem et dracónem. ℣. Quóniam in me sperávit, liberábo eum: prótegam eum, quóniam cognóvit nomen meum. ℣. Invocábit me, et ego exáudiam eum: cum ipso sum in tribulatióne. ℣. Erípiam eum et glorificábo eum: longitúdine diérum adimplébo eum, et osténdam illi salutáre meum.
}\switchcolumn\portugues{
\rlettrine{A}{quele} que habita à sombra do Altíssimo descansará sob a protecção de Deus do céu. ℣. Ele dirá ao Senhor: sois o meu protector e o meu refúgio; sois o meu Deus, em quem confio! ℣. Pois livrastes-me do laço do caçador e das palavras funestas! ℣. O Senhor te acolherá sob a sua protecção: agasalhar-te-á sob as suas asas! ℣. Sua fidelidade é como um escudo: livrar-te-á dos terrores da noite; da seta, que voa de dia; das traições, que se tramam durante a noite; e dos assaltos do demónio, em pleno meio-dia. ℣. Cairão mil à tua esquerda e dez mil à tua direita; mas tu não serás atingido! ℣. Pois Deus ordenou aos seus Anjos que te guardassem em todos teus caminhos. ℣. Levar-te-ão nas suas mãos para que não tropeces nas pedras. ℣. Caminharás sobre a víbora e o basilisco; pisarás o leão e o dragão. ℣. Visto que esperou em mim, livrá-lo-ei; e protegê-lo-ei, pois, conhece e invoca o meu nome. ℣. Invocar-me-á e ouvi-lo-ei: estarei com ele nos dias da tribulação. ℣. Livrá-lo-ei e glorificá-lo-ei, conceder-lhe-ei longa vida e mostrar-lhe-ei a minha salvação.
}\end{paracol}

\paragraphinfo{Evangelho}{Mt. 4, 1-11}
\begin{paracol}{2}\latim{
\cruz Sequéntia sancti Evangélii secúndum Matthǽum.
}\switchcolumn\portugues{
\cruz Continuação do santo Evangelho segundo S. Mateus.
}\switchcolumn*\latim{
\blettrine{I}{n} illo témpore: Ductus est Jesus in desértum a Spíritu, ut tentarétur a diábolo. Et cum jejunásset quadragínta diébus et quadragínta nóctibus, postea esúriit. Et accédens tentátor, dixit ei: Si Fílius Dei es, dic, ut lápides isti panes fiant. Qui respóndens, dixit: Scriptum est: Non in solo pane vivit homo, sed in omni verbo, quod procédit de ore Dei. Tunc assúmpsit eum diábolus in sanctam civitátem, et státuit eum super pinnáculum templi, et dixit ei: Si Fílius Dei es, mitte te deór
sum. Scriptum est enim: Quia Angelis suis mandávit de te, et in mánibus tollent te, ne forte offéndas ad lápidem pedem tuum. Ait illi Jesus: Rursum scriptum est: Non tentábis Dóminum, Deum tuum. Iterum assúmpsit eum diábolus in montem excélsum valde: et ostendit ei ómnia regna mundi et glóriam eórum, et dixit ei: Hæc ómnia tibi dabo, si cadens adoráveris me. Tunc dicit ei Jesus: Vade, Sátana; scriptum est enim: Dóminum, Deum tuum, adorábis, et illi soli sérvies. Tunc relíquit eum diábolus: et ecce, Angeli accessérunt et ministrábant ei.
}\switchcolumn\portugues{
\blettrine{N}{aquele} tempo, foi Jesus conduzido ao deserto pelo espírito, para ser tentado pelo demónio. E, havendo jejuado quarenta dias e quarenta noites, teve fome. Então o tentador aproximou-se de Jesus e disse-Lhe: «Se sois o filho de Deus, mandai que estas pedras se tornem em pães». Jesus respondeu: «Está escrito: «não só de pão vive o homem, mas de toda a palavra que procede da boca de Deus». Então o demónio conduziu Jesus à cidade santa; e, levando-O até ao pináculo do templo, disse-Lhe: «Se sois o Filho de Deus, lançai-Vos daqui para baixo, pois está escrito: «ordenou aos seus Anjos que Vos levassem nas suas mãos para que os vossos pés não tropeçassem nas pedras». Jesus disse-lhe: «Também está escrito «não tentareis ao Senhor, teu Deus». Ainda o demónio conduziu Jesus a um monte muito elevado; e, mostrando-Lhe todos os reinos do mundo, revestidos das suas glórias, disse-Lhe: «Dar-Vos-ei tudo isto, se, de joelhos, me adorardes». Então disse-lhe Jesus: «Retira-te, Satanás, pois está escrito: «Adorarás ao Senhor, teu Deus, e só a Ele servirás», Logo o demónio deixou Jesus, aproximando-se d’Ele os Anjos, que O serviram.
}\end{paracol}

\paragraphinfo{Ofertório}{Sl. 90, 4-5}
\begin{paracol}{2}\latim{
\rlettrine{S}{cápulis} suis obumbrábit tibi Dóminus, et sub pennis ejus sperábis: scuto circúmdabit te véritas ejus.
}\switchcolumn\portugues{
\rlettrine{O}{} Senhor vos acolherá à sua sombra: e sob as suas asas vos esperará: a sua fidelidade proteger-vos-á, como um escudo.
}\end{paracol}

\paragraph{Secreta}
\begin{paracol}{2}\latim{
\rlettrine{S}{acrifícium} quadragesimális inítii sollémniter immolámus, te, Dómine, deprecántes: ut, cum epulárum restrictióne carnálium, a noxiis quoque voluptátibus lemperémus. Per Dóminum \emph{\&c.}
}\switchcolumn\portugues{
\rlettrine{S}{enhor,} imolamos solenemente este sacrifício no princípio da Quaresma, suplicando-Vos que, fazendo-nos restringir o uso das carnes, nos abstenhamos também dos prazeres funestos. Por nosso Senhor \emph{\&c.}
}\end{paracol}

\paragraphinfo{Comúnio}{Sl. 90,4-5}
\begin{paracol}{2}\latim{
\rlettrine{S}{cápulis} suis obumbrábit tibi Dóminus, et sub pennis ejus sperábis: scuto circúmdabit te véritas ejus.
}\switchcolumn\portugues{
\rlettrine{O}{} Senhor vos acolherá à sua sombra: e sob as suas asas vos esperará: a sua verdade proteger-vos-á, como um escudo.
}\end{paracol}

\paragraph{Postcomúnio}
\begin{paracol}{2}\latim{
\qlettrine{Q}{ui} nos, Dómine, sacraménti libátio sancta restáuret: et a vetustáte purgátos, in mystérii salutáris fáciat transíre consórtium. Per Dóminum \emph{\&c.}
}\switchcolumn\portugues{
\qlettrine{Q}{ue} a participação, que tomámos, no vosso sacramento, Senhor, nos restaure; e que, despojando-nos do «homem velho», nos faça alcançar o mistério da salvação. Por nosso Senhor \emph{\&c.}
}\end{paracol}

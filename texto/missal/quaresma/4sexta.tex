\subsectioninfo{Sexta-feira da 4.ª Semana da Quaresma}{Estação em Santo Eusébio}

\paragraphinfo{Intróito}{Sl. 18, 15}
\begin{paracol}{2}\latim{
\rlettrine{D}{editátio} cordis mei in conspéctu tuo semper: Dómine, adjútor meus, et redémptor meus. \emph{Ps. ibid., 2} Cœli enárrant glóriam Dei: et ópera mánuum ejus annúntiat firmaméntum.
℣. Gloria Patri \emph{\&c.}
}\switchcolumn\portugues{
\rlettrine{O}{s} pensamentos do meu coração estarão sempre na vossa presença. Senhor, sois o meu sustentáculo e o meu Redentor. \emph{Sl. ibid., 2} Os céus publicam a glória do Senhor e o firmamento anuncia as obras das suas mãos.
℣. Glória ao Pai \emph{\&c.}
}\end{paracol}

\paragraph{Oração}
\begin{paracol}{2}\latim{
\rlettrine{D}{eus,} qui ineffabílibus mundum rénovas sacraméntis: præsta, quǽsumus; ut Ecclésia tua et ætérnis profíciat institútis, et temporálibus non destituátur auxíliis. Per Dóminum \emph{\&c.}
}\switchcolumn\portugues{
\slettrine{Ó}{} Deus, que restaurais o mundo com inefáveis mistérios, permiti, Vos imploramos, que a vossa Igreja prossiga o seu fim pelos meios eternos que lhe conferistes, e não seja desprovida do vosso socorro nas necessidades temporais. Por nosso Senhor \emph{\&c.}
}\end{paracol}

\paragraphinfo{Epístola}{3 Rs. 17, 17-24}
\begin{paracol}{2}\latim{
Léctio libri Regum.
}\switchcolumn\portugues{
Lição do Livro dos Reis.
}\switchcolumn*\latim{
\rlettrine{I}{n} diébus illis: Ægrotávit fílius mulíeris matrisfamílias, et erat lánguor fortíssimus, ita ut non remanéret in eo hálitus. Dixit ergo ad Elíam: Quid mihi et tibi, vir Dei? Ingréssus es ad me, ut rememoraréntur iniquitátes meæ, et interfíceres fílium meum? Et ait ad eam Elías: Da mihi fílium tuum. Tulítque eum de sinu ejus, et portávit in cenáculum, ubi ipse manébat, et pósuit super léctulum suum, et clamávit ad Dóminum, et dixit: Dómine, Deus meus, étiam ne víduam, apud quam ego utcúmque susténtor, afflixísti, ut interfíceres fílium ejus? Et expándit se, atque mensus est super púerum tribus vícibus, et clamávit ad Dóminum, et ait: Dómine, Deus meus, revertátur, óbsecro, ánima púeri hujus in víscera ejus. Et exaudívit Dóminus vocem Elíæ: et revérsa est ánima púeri intra eum, et revíxit. Tulítque Elías púerum, et depósuit eum de cenáculo in inferiórem domum, et trádidit matri suæ, et ait illi: En, vivit fílius tuus. Dixítque múlier ad Elíam: Nunc in isto cognóvi, quóniam vir Dei es tu, et verbum Dómini in ore tuo verum est.
}\switchcolumn\portugues{
\rlettrine{N}{aqueles} dias, adoeceu tão gravemente o filho de uma mãe de família, que já não havia nele sopro de vida. Esta mãe disse, então, a Elias: «Que há entre ti e mim, ó homem de Deus? Porventura vieste à minha casa para que sejam recordadas as minhas iniquidades e morto o meu filho?». Respondeu-lhe Elias: «Dá-me o teu filho». E Elias tirou-o do seio da mãe, levou-o para o quarto onde estava hospedado, deitou-o no leito e, invocando em voz alta o Senhor, disse: «Senhor, meu Deus, até afligis a pobre viúva que me alberga, matando-lhe o filho? E, estendendo-se três vezes sobre o menino e colocando-se sobre ele, clamou ao Senhor e disse: «Senhor e meu Deus, fazei, Vos suplico, que a alma deste menino volte ao corpo!». O Senhor ouviu a voz de Elias. E logo a alma do menino reentrou nele, tornando a viver. Elias tomou depois o menino, desceu com ele até aos baixos da casa e entregou-o à mãe, dizendo-lhe: «Eis o teu filho. Agora está vivo». Então a mulher disse a Elias: «Agora reconheço que sois um homem de Deus e que a palavra do Senhor, que anunciais, é verdadeira».
}\end{paracol}

\paragraphinfo{Gradual}{Sl. 117, 8-9}
\begin{paracol}{2}\latim{
\rlettrine{B}{onum} est confídere in Dómino, quam confídere in hómine. ℣. Bonum est speráre in Dómino, quam speráre in princípibus.
}\switchcolumn\portugues{
\slettrine{É}{} melhor confiar no Senhor do que no homem. É melhor ter esperança no Senhor do que nos príncipes.
}\end{paracol}

\paragraphinfo{Trato}{Página \pageref{tratoquartacinzas}}

\paragraphinfo{Evangelho}{Jo. 11, 1-45}
\begin{paracol}{2}\latim{
\cruz Sequéntia sancti Evangélii secúndum Joánnem.
}\switchcolumn\portugues{
\cruz Continuação do santo Evangelho segundo S. João.
}\switchcolumn*\latim{
\blettrine{I}{n} illo témpore: Erat quidam languens Lázarus a Bethánia, de castéllo Maríæ et Marthæ, soróris ejus. (María autem erat, quæ unxit Dóminum unguento, et extérsit pedes ejus capíllis suis: cujus frater Lázarus infirmabátur.) Misérunt ergo soróres ejus ad eum, dicéntes: Dómine, ecce, quem amas infirmátur. Audiens autem Jesus, dixit eis: Infírmitas hæc non est ad mortem, sed pro glória Dei, ut glorificétur Fílius Dei per eam. Diligébat autem Jesus Martham et sorórem ejus, Maríam, et Lázarum. Ut ergo audívit, quia infirmabátur, tunc quidem mansit in eódem loco duóbus diébus. Déinde post hæc dixit discípulis suis: Eámus in Judǽam íterum. Dicunt ei discípuli: Rabbi, nunc quærébant te Judǽi lapidáre, et íterum vadis illuc? Respóndit Jesus: Nonne duódecim sunt horæ diéi? Si quis ambuláverit in die, non offéndit, quia lucem hujus mundi videt: si autem ambuláverit in nocte, offéndit, quia lux non est in eo. Hæc ait, et post hæc dixit eis: Lázarus, amícus noster, dormit: sed vado, ut a somno éxcitem eum. Dixérunt ergo discípuli ejus: Dómine, si dormit, salvus erit. Díxerat autem Jesus de morte ejus: illi autem putavérunt, quia de dormitióne somni díceret. Tunc ergo Jesus dixit eis maniféste: Lazarus mórtuus est: et gáudeo propter vos, ut credátis, quóniam non eram ibi: sed eámus ad eum. Dixit ergo Thomas, qui dícitur Dídymus, ad condiscípulos: Eámus et nos, ut moriámur cum eo. Venit itaque Jesus, et invénit eum quátuor dies jam in monuménto habéntem. (Erat autem Bethánia juxta Jerosólymam quasi stádiis quíndecim.) Multi autem ex Judǽis vénerant ad Martham et Maríam, ut consolaréntur eas de fratre suo. Martha ergo, ut audívit quia Jesus venit, occúrrit illi: María autem domi sedébat. Dixit ergo Martha ad Jesum: Dómine, si fuísses hic, frater meus non fuísset mórtuus: sed et nunc scio, quia, quæcúmque popósceris a Deo, dabit tibi Deus. Dicit illi Jesus: Resúrget frater tuus. Dicit ei Martha: Scio, quia resúrget in resurrectióne in novíssimo die. Dixit ei Jesus: Ego sum resurréctio et vita: qui credit in me, etiam si mórtuus fúerit, vivet: et omnis, qui vivit et credit in me, non moriétur in ætérnum. Credis hoc? Ait illi: Utique, Dómine, ego crédidi, quia tu es Christus, Fílius Dei vivi, qui in hunc mundum venísti. Et cum hæc dixísset, ábiit et vocávit Maríam, sorórem suam, siléntio, dicens: Magíster adest, et vocat te. Illa ut audívit, surgit cito, et venit ad eum: nondum enim vénerat Jesus in castéllum; sed erat adhuc in illo loco, ubi occúrrerat ei Martha. Judǽi ergo, qui erant cum ea in domo et consolabántur eam, cum vidíssent Maríam, quia cito surréxit et éxiit, secúti sunt eam, dicéntes: Quia vadit ad monuméntum, ut ploret ibi. María ergo, cum venísset, ubi erat Jesus, videns eum, cécidit ad pedes ejus, et dicit ei: Dómine, si fuísses hic, non esset mórtuus frater meus. Jesus ergo, ut vidit eam plorántem, et Judǽos, qui vénerant cum ea, plorántes, infrémuit spíritu, et turbávit seípsum, et dixit: Ubi posuístis eum? Dicunt ei: Dómine, veni et vide. Et lacrimátus est Jesus. Dixérunt ergo Judǽi: Ecce, quómodo amábat eum. Quidam autem ex ipsis dixérunt: Non póterat hic, qui apéruit óculos cæci nati, facere, ut hic non morerétur? Jesus ergo rursum fremens in semetípso, venit, ad monuméntum. Erat autem spelúnca, et lapis superpósitus erat ei. Ait Jesus: Tóllite lápidem. Dicit ei Martha, soror ejus, qui mórtuus fuerat: Dómine, jam fetet, quatriduánus est enim. Dicit ei Jesus: Nonne dixi tibi, quóniam, si credíderis, vidébis glóriam Dei? Tulérunt ergo lápidem: Jesus autem, elevátis sursum óculis, dixit: Pater, grátias ago tibi, quóniam audísti me. Ego autem sciébam, quia semper me audis, sed propter pópulum, qui circúmstat, dixi: ut credant, quia tu me misísti. Hæc cum dixísset, voce magna clamávit: Lázare, veni foras. Et statim pródiit, qui fúerat mórtuus, ligátus pedes et manus ínstitis, et fácies illíus sudário erat ligáta. Dixit eis Jesus: Sólvite eum, et sínite abíre. Multi ergo ex Judǽis, qui vénerant ad Maríam et Martham, et víderant quæ fecit Jesus, credidérunt in eum.
}\switchcolumn\portugues{
\blettrine{N}{aquele} tempo, estava doente um certo homem chamado Lázaro, de Betânia, aldeia de Maria e de Marta, suas irmãs. (Maria era aquela que ungira o Senhor com perfumes e Lhe enxugara os pés com os cabelos; e o que estava doente era seu irmão). As irmãs mandaram, então, dizer a Jesus: «Senhor, aquele a quem amais está enfermo». Jesus, ouvindo isto, disse: «Esta doença não é para produzir a morte, mas para a glória de Deus, a fim de que o filho de Deus seja glorificado por ela». Ora Jesus era amigo de Marta, de sua irmã Maria e de Lázaro; contudo, ainda que tivesse sabido que ele estava doente, deixou-se ficar mais dois dias no mesmo lugar. Depois disso, disse aos discípulos: «Voltemos para a Judeia». Os discípulos disseram-Lhe: «Ainda há pouco os judeus quiseram apedrejar-Vos, e já quereis voltar para lá?». Jesus respondeu-lhes: «Porventura o dia não tem doze horas? Se alguém anda de dia, não tropeça, porque vê a luz deste mundo; porém, se anda de noite, tropeça, porque não tem luz». Depois de haver falado assim, acrescentou: «O nosso amigo Lázaro dorme, mas vou acordá-lo». Disseram-Lhe então os discípulos: «Senhor, se ele dorme, será salvo». Mas Jesus havia falado da sua morte, cuidando os discípulos que se referia ao repouso do sono. Então Jesus disse-lhes claramente: «Lázaro morreu; mas por causa de vós, para que acrediteis, alegro-me de lá não estar. Vamos ter com ele». Disse Tomé, o Dídimo, aos companheiros: «Vamos e morramos com Ele». Vindo, pois, Jesus chegou quando havia quatro dias que Lázaro estava sepultado. Ora, Betânia era próximo de Jerusalém cerca de quinze estádios, tendo vindo muitos judeus consolar Marta e Maria, por causa da morte do irmão. Logo que Marta soube que Jesus vinha, saiu-lhe ao encontro, ficando, porém, Maria sentada em casa. Marta disse então a Jesus: «Senhor, se estivésseis aqui, o meu irmão não teria morrido; mas também sei que tudo o que pedirdes a Deus ser-Vos-á concedido». Disse-lhe Jesus: «O teu irmão ressuscitará». Disse Marta: «Sei que ressuscitará na ressurreição do último dia». Jesus continuou: «Eu sou a ressurreição e a vida; quem acreditar em mim viverá, ainda que esteja morto; e quem vive e crê em mim não morrerá para sempre. Acreditas nisto?». Respondeu ela: «Sim, Senhor, creio que sois Cristo, Filho de Deus vivo, que viestes a este mundo!». E depois que disse isto, foi chamar em segredo sua irmã Maria, dizendo: «Está ali o Mestre, que te chama». Assim que Maria ouviu isto, levantou-se e foi ao seu encontro; pois Jesus não havia ainda chegado à aldeia, mas estava no lugar onde Marta o encontrara. Vendo os judeus (que estavam com Maria em sua casa para a consolar) que ela se levantava e saía, seguiram-na, dizendo: «Vai ao sepulcro para chorar». Porém, Maria, logo que chegou ao lugar onde estava Jesus e O viu, prostrou-se de joelhos e disse-Lhe: «Se estivésseis aqui, o meu irmão não teria morrido!». Jesus, vendo que ela e os judeus que a acompanhavam choravam, suspirou, comoveu-se e perturbou-se também, dizendo: «Onde o pusestes?». Disseram-Lhe: «Senhor, vinde e vede». E Jesus chorou também! Disseram então os judeus: «Vede como o amava!». E alguns acrescentaram: «Não podia Ele; que abriu os olhos ao cego de nascença, evitar que este tivesse morrido?». Porém, Jesus, comovendo-se outra vez, veio à sepultura, que era em uma caverna sobre a qual tinham colocado uma pedra. Disse Jesus: «Tirai a pedra». Marta, a irmã do defunto, respondeu-Lhe: «Senhor, já cheira mal; pois há quatro dias que está aí». E Jesus disse-lhe: «Não te afirmei que, se acreditasses, verias a glória de Deus?». Então, tiraram a pedra. E Jesus, elevando os olhos ao céu, disse: «Pai, dou-Vos graças pelas vezes que me tendes já ouvido. Bem sei que Vós sempre me ouvis, mas digo isto por causa do povo que me rodeia, para que creia que me enviastes». Havendo dito isto, chamou em voz alta: «Lázaro, sai para fora!». Logo saiu o defunto, tendo os pés e as mãos ligados com faixas e o rosto envolvido no sudário! E Jesus continuou: «Desatai-o e deixai-o ir!». Então muitos judeus, que tinham acompanhado Maria e Marta, vendo isto, acreditaram em Jesus.
}\end{paracol}

\paragraphinfo{Ofertório}{Sl. 17, 28 \& 32}
\begin{paracol}{2}\latim{
\rlettrine{P}{ópulum} húmilem salvum fácies, Dómine, et óculos superbórum humiliábis: quóniam quis Deus præter te, Dómine?
}\switchcolumn\portugues{
\rlettrine{V}{ós,} Senhor, salvais o povo humilde e abateis os olhos dos soberbos. Quem é, pois, Deus senão Vós, Senhor?
}\end{paracol}

\paragraph{Secreta}
\begin{paracol}{2}\latim{
\rlettrine{M}{únera} nos, Dómine, quǽsumus, obláta puríficent: et te nobis jugiter fáciant esse placátum. Per Dóminum nostrum \emph{\&c.}
}\switchcolumn\portugues{
\rlettrine{S}{enhor,} Vos imploramos, permiti que estes dons, que Vos oferecemos, nos purifiquem e aplaquem incessantemente a vossa ira contra nós. Por nosso Senhor \emph{\&c.}
}\end{paracol}

\paragraphinfo{Comúnio}{Jo. 11, 33, 35, 43, 44 \& 39}
\begin{paracol}{2}\latim{
\rlettrine{V}{idens} Dóminus flentes soróres Lázari ad monuméntum, lacrimátus est coram Judǽis, et exclamávit: Lázare, veni foras: et pródiit ligátis mánibus et pédibus, qui fúerat quatriduánus mórtuus.
}\switchcolumn\portugues{
\rlettrine{O}{} Senhor, vendo chorar as irmãs de Lázaro, perto do sepulcro, chorou também, na presença dos judeus, e clamou: «Lázaro, sai para fora». E aquele que estava morto havia quatro dias apareceu com os pés e as mãos ligados!
}\end{paracol}

\paragraph{Postcomúnio}
\begin{paracol}{2}\latim{
\rlettrine{H}{æc} nos, quǽsumus, Dómine, participátio sacraménti: et a propriis reátibus indesinénter expédiat, et ab ómnibus tueátur advérsis. Per Dóminum \emph{\&c.}
}\switchcolumn\portugues{
\qlettrine{Q}{ue} esta nossa participação nos celestiais mistérios, Senhor, Vos suplicamos, nos livre sempre das nossas culpas e nos defenda de todas as adversidades. Por nosso Senhor \emph{\&c.}
}\end{paracol}

\paragraph{Oração sobre o povo}
\begin{paracol}{2}\latim{
\begin{nscenter} Orémus. \end{nscenter}
}\switchcolumn\portugues{
\begin{nscenter} Oremos. \end{nscenter}
}\switchcolumn*\latim{
Humiliáte cápita vestra Deo.
}\switchcolumn\portugues{
Inclinai as vossas cabeças diante de Deus.
}\switchcolumn*\latim{
Da nobis, quǽsumus, omnípotens Deus: ut, qui infirmitátis nostræ cónscii, de tua virtúte confídimus, sub tua semper pietáte gaudeámus. Per Dóminum \emph{\&c.}
}\switchcolumn\portugues{
Ó Deus omnipotente, conhecendo nós a nossa fraqueza e tendo confiança no vosso poder, concedei-nos, Vos suplicamos, que gozemos sempre os efeitos da vossa bondade. Por nosso Senhor \emph{\&c.}
}\end{paracol}

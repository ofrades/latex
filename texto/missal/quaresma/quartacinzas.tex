\subsection{Quarta-feira de Cinzas}

\paragraphinfo{Antífona}{Sl. 68, 17}
\begin{paracol}{2}\latim{
\rlettrine{E}{xáudi} nos, Dómine, quóniam benígna est misericórdia tua: secúndum multitúdinem miseratiónum tuárum réspice nos, Dómine. \emph{Ps. ibid., 2} Salvum me fac, Deus: quóniam intravérunt aquæ usque ad ánimam meam. ℣. Glória Patri \emph{\&c.}
}\switchcolumn\portugues{
\rlettrine{O}{uvi-nos,} Senhor, pois a vossa misericórdia é compassiva. Lançai para nós os vossos olhares, Senhor, segundo a grandeza da vossa misericórdia. \emph{Sl. ibid., 2} Salvai-me, ó Deus, porque as águas penetraram até ao íntimo da minha alma. ℣. Glória ao Pai \emph{\&c.}
}\end{paracol}

\paragraph{Oração}
\begin{paracol}{2}\latim{
\rlettrine{O}{mnípotens} sempitérne Deus, parce pæniténtibus, propitiáre supplicántibus: et míttere dignéris sanctum Angelum tuum de cœlis, qui bene \cruz dícat et sanctí \cruz ficet hos cíneres, ut sint remédium salúbre ómnibus nomen sanctum tuum humilíter implorántibus, ac semetípsos pro consciéntia delictórum suórum accusántibus, ante conspéctum divínæ cleméntiæ tuæ facínora sua deplorántibus, vel sereníssimam pietátem tuam supplíciter obnixéque flagitántibus: et præsta per invocatiónem sanctíssimi nóminis tui; ut, quicúmque per eos aspérsi fúerint, pro redemptióne peccatórum suórum, córporis sanitátem et ánimæ tutélam percípiant. Per Christum, Dóminum nostrum. ℟. Amen.
}\switchcolumn\portugues{
\rlettrine{D}{eus,} omnipotente e eterno, perdoai aos penitentes, sede propício para com os suplicantes e dignai-Vos enviar do céu o vosso Anjo para abençoar \cruz e santificar \cruz estas Cinzas, a fim de que sejam remédio salutar para todos os que imploram humildemente o vosso santo nome, e, reconhecendo os seus delitos, a si mesmos se acusam, deplorando esses crimes, diante da vossa divina clemência, e suplicando instante e humildemente a vossa compassiva misericórdia. Dignai-Vos permitir que pela invocação do vosso santíssimo nome todos os que com estas Cinzas forem aspergidos alcancem, além da remissão dos pecados, saúde para o corpo e protecção para a alma. Por Cristo, nosso Senhor. ℟. Amen.
}\end{paracol}

\paragraph{Oração}
\begin{paracol}{2}\latim{
\rlettrine{D}{eus,} qui non mortem, sed pæniténtiam desíderas peccatórum: fragilitátem condiciónis humánæ benigníssime réspice; et hos cíneres, quos, causa proferéndæ humilitátis atque promeréndæ véniæ, capítibus nostris impóni decérnimus, bene \cruz dícere pro tua pietáte dignáre: ut, qui nos cínerem esse, et ob pravitátis nostræ deméritum in púlverem reversúros cognóscimus; peccatórum ómnium véniam, et prǽmia pæniténtibus repromíssa, misericórditer cónsequi mereámur.
Per Christum, Dóminum nostrum. ℟. Amen.
}\switchcolumn\portugues{
\slettrine{Ó}{} Deus, que não quereis a morte dos pecadores, mas que façam penitência, olhai benigníssimo para a fragilidade da natureza humana, e dignai-Vos misericordiosamente abençoar \cruz estas Cinzas, que desejamos sejam impostas nas nossas cabeças em sinal de humildade e para merecermos o perdão, a fim de que, reconhecendo que não somos senão cinza e que nos tornaremos em pó em punição da nossa malícia, mereçamos da vossa bondade alcançar o perdão dos nossos pecados e as recompensas prometidas aos que fazem penitência. Por Cristo, nosso Senhor. ℟. Amen.
}\end{paracol}

\paragraph{Oração}
\begin{paracol}{2}\latim{
\rlettrine{D}{eus,} qui humiliatióne flécteris, et satisfactióne placáris: aurem tuæ pietátis inclína précibus nostris; et capítibus servórum tuórum, horum cínerum aspersióne contáctis, effúnde propítius grátiam tuæ benedictiónis: ut eos et spíritu compunctiónis répleas et, quæ juste postuláverint, efficáciter tríbuas; et concéssa perpétuo stabilíta et intácta manére decérnas. Per Christum, Dóminum nostrum. ℟. Amen.
}\switchcolumn\portugues{
\slettrine{Ó}{} Deus, que pela humilhação Vos inclinais a perdoar e pela satisfação Vos aplacais, dignai-Vos benignamente escutar as nossas preces, infundindo nos vossos servos, cujas cabeças recebem estas Cinzas, a graça da vossa bênção; e, enchendo-os com o espírito de compunção, concedei-lhes o que com justiça Vos suplicarem, a fim de que perpetuamente conservem, firme e intacto, o que de Vós houverem alcançado. Por Cristo, nosso Senhor. ℟. Amen.
}\end{paracol}

\paragraph{Oração}
\begin{paracol}{2}\latim{
\rlettrine{O}{mnípotens} sempitérne Deus, qui Ninivítis, in cínere et cilício pæniténtibus, indulgéntiæ tuæ remédia præstitísti: concéde propítius; ut sic eos imitémur hábitu, quaténus véniæ prosequámur obténtu. Per Dóminum.
}\switchcolumn\portugues{
\rlettrine{D}{eus} omnipotente e eterno, que propiciastes o remédio e o perdão aos Ninivitas, que fizeram penitência por meio das cinzas e do cilício, concedei-nos indulgentemente que de tal sorte os imitemos que, como eles, alcancemos o vosso perdão. Por nosso Senhor.
}\end{paracol}

\paragraphinfo{Antífona}{Jl. 2, 13}
\begin{paracol}{2}\latim{
\rlettrine{I}{mmutémur} hábitu, in cínere et cilício: jejunémus, et plorémus ante Dóminum: quia multum miséricors est dimíttere peccáta nostra Deus noster.
}\switchcolumn\portugues{
\rlettrine{M}{udemos} os vestidos e cubramo-nos com a cinza e com o cilício; jejuemos e choremos diante do Senhor, pois Ele é misericordioso e está pronto a perdoar os nossos pecados.
}\switchcolumn\latim{
{\redx Alia Antiph.} \emph{ibid., 17} Inter vestíbulum et altáre plorábunt sacerdótes minístri Dómini, et dicent: Parce, Dómine, parce pópulo tuo: et ne claudas ora canéntium te, Dómine.
}\switchcolumn\portugues{
{\redx Outra Antífona} \emph{ibid., 17} Entre o vestíbulo e o altar chorarão os sacerdotes e os Ministros do Senhor, que dirão: «Perdoai, Senhor, perdoai ao vosso povo; e não fecheis a boca àqueles que cantam os vossos louvores, ó Senhor».
}\end{paracol}

\paragraphinfo{Responsório}{Est 13; Jl. 2}
\begin{paracol}{2}\latim{
\rlettrine{E}{mendémus} in mélius, quæ ignoránter peccávimus: ne, subito præoccupáti die mortis, quærámus spátium pæniténtiæ, et inveníre non póssimus. Atténde, Dómine, et miserére: quia peccávimus tibi, \emph{℣. Ps. 78,9} Adjuva nos, Deus, salutáris noster: et propter honórem nóminis tui, Dómine, líbera nos. Atténde, Dómine. ℣. Glória Patri \emph{\&c.}
}\switchcolumn\portugues{
\rlettrine{R}{eparemos} o mal que praticámos por ignorância, para que não aconteça que, surpreendidos pelo dia da morte, queiramos fazer penitência, mas já não tenhamos tempo! Ouvi-nos, Senhor, e tende misericórdia de nós, pois pecámos contra Vós. \emph{℣. Sl. 78,9} Auxiliai-nos, ó Deus, nosso Salvador, e, pela honra do vosso nomes, salvai-nos, Senhor. Ouvi-nos, Senhor, e tende misericórdia de nós, pois pecámos contra Vós. Glória ao Pai \emph{\&c.}
}\end{paracol}

\paragraphinfo{Imposição das Cinzas}{Gn 3, 19}
\begin{paracol}{2}\latim{
\rlettrine{M}{emento,} homo, quia pulvis es, et in púlverem revertéris.
}\switchcolumn\portugues{
\rlettrine{L}{embra-te,} homem, que és pó e que em pó te hás-de tornar.
}\end{paracol}

\paragraph{Oração}
\begin{paracol}{2}\latim{
\rlettrine{C}{oncéde} nobis, Dómine, præsídia milítiæ christiánæ sanctis inchoáre jejúniis: ut, contra spiritáles nequítias pugnatúri, continéntiæ muniámur auxíliis. Per Christum, Dóminum nostrum. ℟. Amen.
}\switchcolumn\portugues{
\rlettrine{P}{ermiti,} Senhor, que iniciemos com estes salutares jejuns esta estação da milícia cristã, a fim de que, havendo nós de combater contra os espíritos do mal, estejamos munidos contra os seus esforços com os auxílios da abstinência. Por Cristo, nosso Senhor. ℟. Amen.
}\end{paracol}

\paragraphinfo{Intróito}{Sb, 11, 24, 25 \& 27}
\begin{paracol}{2}\latim{
\rlettrine{M}{iseréris} ómnium, Dómine, et nihil odísti eórum quæ fecísti, dissímulans peccáta hóminum propter pæniténtiam et parcens illis: quia tu es Dóminus, Deus noster. \emph{Ps. 56, 2} Miserére mei, Deus, miserére mei: quóniam in te confídit ánima mea.
℣. Gloria Patri \emph{\&c.}
}\switchcolumn\portugues{
\rlettrine{S}{enhor,} tendes piedade de todos e não odiais aqueles que criastes; atenuais e perdoais os pecados dos homens, para que façam penitência, porquanto sois o Senhor, nosso Deus. \emph{Sl. 56, 2} Tende piedade de mim, pois em Vós procura refúgio a minha alma. Glória ao Pai
℣. Glória ao Pai \emph{\&c.}
}\end{paracol}

\paragraph{Oração}
\begin{paracol}{2}\latim{
\rlettrine{P}{ræsta,} Dómine, fidélibus tuis: ut jejuniórum veneránda sollémnia, et cóngrua pietáte suscípiant, et secúra devotióne percúrrant. Per Dóminum \emph{\&c.}
}\switchcolumn\portugues{
\rlettrine{P}{ermiti,} Senhor, aos vossos fiéis que iniciem com sincera piedade as veneráveis solenidades destes jejuns e que possam continuá-las até ao fim sempre devotamente. Por nosso Senhor \emph{\&c.}
}\end{paracol}

\paragraphinfo{Epístola}{Jl. 2, 12-19}
\begin{paracol}{2}\latim{
Léctio Joélis Prophétæ.
}\switchcolumn\portugues{
Lição do Profeta Joel.
}\switchcolumn*\latim{
\rlettrine{H}{æc} dicit Dóminus: Convertímini ad me in toto corde vestro, in jejúnio, et in fletu, et in planctu. Et scíndite corda vestra, et non vestiménta vestra, et convertímini ad Dóminum, Deum vestrum: quia benígnus et miséricors est, pátiens, et multæ misericórdiæ, et præstábilis super malítia. Quis scit, si convertátur, et ignóscat, et relínquat post se benedictiónem, sacrifícium et libámen Dómino, Deo vestro? Cánite tuba in Sion, sanctificáte jejúnium, vocáte cœtum, congregáte pópulum, sanctificáte ecclésiam, coadunáte senes, congregáte parvulos et sugéntes úbera: egrediátur sponsus de cubíli suo, et sponsa de thálamo suo. Inter vestíbulum et altare plorábunt sacerdótes minístri Dómini, et dicent: Parce, Dómine, parce pópulo tuo: et ne des hereditátem tuam in oppróbrium, ut dominéntur eis natiónes. Quare dicunt in pópulis: Ubi est Deus eórum? Zelátus est Dóminus terram suam, et pepércit pópulo suo. Et respóndit Dóminus, et dixit populo suo: Ecce, ego mittam vobis fruméntum et vinum et óleum, et replebímini eis: et non dabo vos ultra oppróbrium in géntibus: dicit Dóminus omnípotens.
}\switchcolumn\portugues{
\rlettrine{E}{is} o que diz o Senhor: Convertei-vos a mim, de todo o coração, com jejuns, lágrimas e gemidos. Rasgai os vossos corações e não os vossos vestidos; e convertei-vos ao Senhor, vosso Deus, que é benigno, clemente, paciente, compassivo, cheio de misericórdia e cuja bondade excede a malícia humana. Quem sabe se Ele não retrocederá e não se arrependerá, não deixando após si uma bênção para apresentardes como oferta de sacrifício ao Senhor, nosso Deus?! Tocai a trombeta em Sião; ordenai que jejuem; convocai uma assembleia; chamai o povo; publicai uma reunião santa; reuni os velhos; congregai os meninos, ainda mesmo os que bebem o leite maternal; o marido saia do seu quarto e a esposa do seu leito nupcial. Que os sacerdotes, ministros do Senhor, chorem entre o vestíbulo e o altar e digam: «Perdoai, Senhor, perdoai ao vosso povo; não deixeis a vossa herança cair no opróbrio para servir de troça às nações estrangeiras»! Pois dir-se-á entre os povos: «Onde está o Deus deste povo? O Senhor compadeceu-se da sua terra e perdoou ao povo». Respondeu, então, o Senhor e disse ao seu povo: «Eis que vos mandarei trigo, vinho e azeite, e ficareis saciados; nunca mais vos abandonarei ao opróbrio das nações»: isto disse o Senhor omnipotente.
}\end{paracol}

\paragraphinfo{Gradual}{Sl. 56, 2 et 4}
\begin{paracol}{2}\latim{
\rlettrine{M}{iserére} mei, Deus, miserére mei: quóniam in te confídit ánima mea. ℣. Misit de cœlo, et liberávit me, dedit in oppróbrium conculcántes me.
}\switchcolumn\portugues{
\rlettrine{T}{ende} misericórdia de mim, ó Deus; tende misericórdia de mim; pois a minha alma confia em Vós. ℣. Mandou auxílio do céu, livrando-me, e condenou ao opróbrio aqueles que me espezinhavam.
}\end{paracol}

\paragraphinfo{Trato}{Sl. 102, 10}\label{tratoquartacinzas}
\begin{paracol}{2}\latim{
\rlettrine{D}{ómine,} non secúndum peccáta nostra, quæ fécimus nos: neque secúndum iniquitátes nostras retríbuas nobis. ℣. Ps. 78, 8-9. Dómine, ne memíneris iniquitátum nostrarum antiquarum: cito antícipent nos misericórdiæ tuæ, quia páuperes facti sumus nimis. \emph{(Hic genuflectitur)} ℣. Adjuva nos, Deus, salutáris noster: et propter glóriam nóminis tui, Dómine, libera nos: et propítius esto peccátis nostris, propter nomen tuum.
}\switchcolumn\portugues{
\rlettrine{S}{enhor,} nos não castigueis, consoante merecemos, pelos pecados que praticámos: nem nos julgueis, segundo as nossas iniquidades. Esquecei-Vos, Senhor, das nossas iniquidades passa- das, apressai-Vos em revestir-nos com vossas misericórdias, pois grande é a nossa miséria. \emph{(Aqui genuflectir)}. ℣. Auxiliai-nos, ó Deus, nosso Salvador, e, pela glória do vosso nome, livrai-nos, Senhor, e perdoai os nossos pecados por causa do vosso nome.
}\end{paracol}

\paragraphinfo{Evangelho}{Mt. 6, 16-21}
\begin{paracol}{2}\latim{
\cruz Sequéntia sancti Evangélii secúndum Matthǽum.
}\switchcolumn\portugues{
\cruz Continuação do santo Evangelho segundo S. Mateus.
}\switchcolumn*\latim{
\blettrine{I}{n} illo témpore: Dixit Jesus discípulis suis: Cum jejunátis, nolíte fíeri, sicut hypócritæ, tristes. Extérminant enim fácies suas, ut appáreant homínibus jejunántes. Amen, dico vobis,
quia recepérunt mercédem suam. Tu autem, cum jejúnas, unge caput tuum, et fáciem tuam lava, ne videáris homínibus jejúnans, sed Patri tuo, qui est in abscóndito: et Pater tuus, qui videt in abscóndito, reddet tibi. Nolíte thesaurizáre vobis thesáuros in terra: ubi ærúgo et tínea demólitur: et ubi fures effódiunt et furántur. Thesaurizáte autem vobis thesáuros in cœlo: ubi neque ærúgo neque tínea demólitur; et ubi fures non effódiunt nec furántur. Ubi enim est thesáurus tuus, ibi est et cor tuum.
}\switchcolumn\portugues{
\blettrine{N}{aquele} tempo, disse Jesus aos seus discípulos: «Quando jejuardes, não vos mostreis tristonhos, como os hipócritas; pois estes costumam desfigurar o rosto para mostrarem aos homens que jejuam. Na verdade vos digo: já receberam a sua recompensa. Porém, quando jejuardes, perfumai a cabeça e lavai o rosto, para que não pareça aos homens que jejuais, mas sim a vosso Pai, que conhece todas as coisas ocultas; e, conhecendo esse segredo, vos dará a recompensa. Não guardeis tesouros na terra, onde a traça e a ferrugem os gastam e os ladrões esburacam a parede e os roubam. Guardai antes vossos tesouros no céu, onde nem a ferrugem nem a traça os destroem, nem os ladrões os desenterram ou furtam. Pois, onde estiver o vosso tesouro, aí estará o vosso coração.
}\end{paracol}

\paragraphinfo{Ofertório}{Sl. 29,2-3}
\begin{paracol}{2}\latim{
\rlettrine{E}{xaltábo} te, Dómine, quóniam suscepísti me, nec delectásti inimícos meos super me: Dómine, clamávi ad te, et sanásti me.
}\switchcolumn\portugues{
\rlettrine{E}{xaltar-Vos-ei,} Senhor, porque me escolhestes e não permitistes que meus inimigos abusassem de mim. Clamei por Vós, Senhor, e curastes-me.
}\end{paracol}

\paragraph{Secreta}
\begin{paracol}{2}\latim{
\rlettrine{F}{ac} nos, quǽsumus, Dómine, his munéribus offeréndis conveniénter aptári: quibus ipsíus venerábilis sacraménti celebrámus exórdium. Per Dóminum \emph{\&c.}
}\switchcolumn\portugues{
\rlettrine{V}{os} suplicamos, Senhor, tornai-nos dignos de Vos oferecermos, como devemos, estes dons sacratíssimos com os quais iniciamos a celebração deste venerando tempo, cheio de mistério. Por nosso Senhor \emph{\&c.}
}\end{paracol}

\paragraphinfo{Comúnio}{Sl. 1, 2 \& 3}
\begin{paracol}{2}\latim{
\qlettrine{Q}{ui} meditábitur in lege Dómini die ac nocte, dabit fructum suum in témpore suo.
}\switchcolumn\portugues{
\rlettrine{A}{quele} que meditar durante o dia e a noite na Lei do Senhor dará fruto no tempo próprio.
}\end{paracol}

\paragraph{Postcomúnio}
\begin{paracol}{2}\latim{
\rlettrine{P}{ercépta} nobis, Dómine, prǽbeant sacraménta subsídium: ut tibi grata sint nostra jejúnia, et nobis profíciant ad medélam. Per Dóminum nostrum \emph{\&c.}
}\switchcolumn\portugues{
\rlettrine{S}{enhor,} permiti que os sacramentos, que recebemos, nos propiciem o auxílio de que carecemos, a fim de que os nossos jejuns Vos sejam agradáveis e se tornem em remédio para os nossos males. Por nosso Senhor \emph{\&c.}
}\end{paracol}

\subsectioninfo{Sábado da 2.ª Semana da Quaresma}{Estação em S. Marcelino e S. Pedro}

\paragraphinfo{Intróito}{Sl. 18, 8}
\begin{paracol}{2}\latim{
\rlettrine{L}{ex} Dómini irreprehensíbilis, convértens ánimas: testimónium Dómini fidéle, sapiéntiam præstans párvulis. \emph{Ps. ibid., 2} Cœli enárrant glóriam Dei: et ópera mánuum ejus annúntiat firmaméntum.
℣. Gloria Patri \emph{\&c.}
}\switchcolumn\portugues{
\rlettrine{A}{} lei do Senhor é perfeita e converte as almas; o testemunho do Senhor é digno de fé e ensina os ignorantes. \emph{Sl. ibid., 2} Os céus publicam a glória do Senhor e o firmamento anuncia as obras das suas mãos.
℣. Glória ao Pai \emph{\&c.}
}\end{paracol}

\paragraph{Oração}
\begin{paracol}{2}\latim{
\rlettrine{D}{a,} quǽsumus, Dómine, nostris efféctum jejúniis salutárem: ut castigátio carnis assúmpta, ad nostrárum vegetatiónem tránseat animárum. Per Dóminum \emph{\&c.}
}\switchcolumn\portugues{
\rlettrine{D}{ai,} Senhor, Vos suplicamos, aos nossos jejuns efeitos salutares, para que a mortificação que fazemos na carne seja proveitosa à saúde das nossas almas. Por nosso Senhor \emph{\&c.}
}\end{paracol}

\paragraphinfo{Epístola}{Gn. 27, 6-40}
\begin{paracol}{2}\latim{
Léctio libri Genesis.
}\switchcolumn\portugues{
Lição do Livro do Génesis.
}\switchcolumn*\latim{
\rlettrine{I}{n} diébus illis: Dixit Rebécca fílio suo Jacob: Audívi patrem tuum loquéntem cum Esau fratre tuo, et dicéntem ei: Affer mihi de venatióne tua, et fac cibos, ut comédam et benedícam tibi coram Dómino, ántequam móriar. Nunc ergo, fili mi, acquiésce consíliis meis: et pergens ad gregem, affer mihi duos hædos óptimos, ut fáciam ex eis escas patri tuo, quibus libénter véscitur: quas cum intúleris et coméderit, benedícat tibi, priúsquam moriátur. Cui ille respóndit: Nosti, quod Esau, frater meus, homo pilósus sit, et ego lenis: si attrectáverit me pater meus et sénserit, tímeo, ne putet me sibi voluísse illúdere, et indúcam super me maledictiónem pro benedictióne. Ad quem mater: In me sit, ait, ista male díctio, fili mi: tantum audi vocem meam, et pergens affer quæ dixi. Abiit, et áttulit, dedítque matri. Parávit illa cibos, sicut velle nóverat patrem illíus. Et véstibus Esau valde bonis, quas apud se habébat domi, índuit eum: pelliculásque hædórum circúmdedit mánibus, et colli nuda protéxit. Dedítque pulméntum, et panes, quos cóxerat, trádidit. Quibus illátis, dixit: Pater mi! At ille respóndit: Audio. Quis es tu, fili mi? Dixítque Jacob: Ego sum primogénitus tuus Esau: feci, sicut præcepísti mihi: surge, sede, et cómede de venatióne mea, ut benedícat mihi ánima tua. Rursúmque Isaac ad fílium suum: Quómodo, inquit, tam cito inveníre potuísti, fili mi? Qui respóndit: Volúntas Dei fuit, ut cito occúrreret mihi quod volébam. Dixítque Isaac: Accéde huc, ut tangam te, fili mi, et probem, utrum tu sis fílius meus Esau, an non. Accéssit ille ad patrem, et palpáto eo, dixit Isaac: Vox quidem vox Jacob est, sed manus manus sunt Esau. Et non cognóvit eum, quia pilósæ manus similitúdinem majóris exprésserant. Benedícens ergo illi, ait: Tu es fílius meus Esau? Respóndit: Ego sum. At ille: Affer mihi, inquit, cibos de venatióne tua, fili mi, ut benedícat tibi ánima mea. Quos cum oblátos comedísset, óbtulit ei étiam vinum. Quo hausto, dixit ad eum: Accéde ad me, et da mihi ósculum, fili mi. Accéssit, et osculátus est eum. Statímque ut sensit vestimentórum illíus fragrántiam, benedícens illi, ait: Ecce, odor fílii mei sicut odor agri pleni, cui benedíxit Dóminus. Det tibi Deus de rore cœli, et de pinguédine terræ abundántiam fruménti et vini. Et sérviant tibi pópuli, et ad orent te tribus: esto dóminus fratrum tuórum, et incurvéntur ante te fílii matris tuæ. Qui male díxerit tibi, sit ille maledíctus: et qui benedíxerit tibi, benedictiónibus repleátur. Vix Isaac sermónem impléverat, et egrésso Jacob foras, venit Esau, coctósque de venatióne cibos íntulit patri, dicens: Surge, pater mi, et cómede de venatióne fílii tui, ut benedícat mihi ánima tua. Dixítque illi Isaac: Quis enim es tu? Qui respóndit: Ego sum fílius tuus primogénitus Esau. Expávit Isaac stupóre veheménti, et ultra quam credi potest, admírans, ait: Quis ígitur ille est, qui dudum captam venatiónem áttulit mihi, et comédi ex ómnibus, priúsquam tu veníres? Benedixíque ei, et erit benedíctus. Audítis Esau sermónibus patris, irrúgiit clamóre magno, et consternátus, ait: Bénedic etiam et mihi, pater mi. Qui ait: Venit germánus tuus fraudulénter, et accépit benedictiónem tuam. At ille subjunxit: Juste vocátum est nomen ejus Jacob: supplantávit enim me en áltera vice: primogénita mea ante tulit, et nunc secúndo surrípuit benedictiónem meam. Rursúmque ad patrem: Numquid non reservásti, ait, et mihi benedictiónem? Respóndit Isaac: Dóminum tuum illum constítui, et omnes fratres ejus servitúti illíus subjugávi: fruménto et vino stabilívi eum, et tibi post hæc, fili mi, ultra quid fáciam? Cui Esau: Num unam, inquit, tantum benedictiónem habes, pater? mihi quoque óbsecro ut benedícas. Cumque ejulátu magno fleret, motus Isaac, dixit ad eum: In pinguédine terræ, et in rore cœli désuper erit benedíctio tua.
}\switchcolumn\portugues{
\rlettrine{N}{aqueles} dias, disse Rebeca a seu filho Jacob: «Ouvi teu pai, que falava com teu irmão Esaú e lhe dizia: «Dá-me alguma cousa da tua caça e prepara-a, para que eu a coma. Depois abençoar-te-ei, diante do Senhor, antes da minha morte». Agora, pois, meu filho, segue os meus conselhos: vai ao rebanho e traze-me dous cabritos gordos, para que eu faça com eles os manjares de que teu pai gosta mais. E tu lhos apresentarás, ele comerá e te dará a bênção antes de morrer». Jacob respondeu-lhe: «Bem sabeis que meu irmão Esaú é um homem cabeludo e eu não tenho pêlo. Se, pois, meu pai me apalpar e conhecer, receio que julgue querer iludi-lo; então me dará a maldição, em vez da bênção». Ao que a mãe respondeu: «Chamarei para mim essa maldição, meu filho. Ouve somente a minha voz e vai buscar o que te disse». Então, ele foi buscar os cabritos, trouxe-os e deu-os a sua mãe, que preparou manjares como o pai gostava. Depois, vestiu Jacob com os melhores vestidos de Esaú, que tinha guardados em casa, cobriu-lhe as mãos e o pescoço, até onde estava descoberto, com as peles dos cabritos, e deu-lhe os manjares e os pães que cozera. Levando Jacob isto a Isaque, disse-lhe: «Meu pai», Ele respondeu: «Bem ouço. Quem és tu, meu filho? ». Jacob disse: «Sou o teu filho primogénito Esaú. Fiz como me mandaste. Ergue-te, senta-te e come da minha caça, para que a tua alma me abençoe». Disse ainda Isaque a seu filho: «Como pudeste encontrá-la tão depressa, meu filho?». Este respondeu: «Foi vontade de Deus que depressa encontrasse o que queria». Disse Isaque: «Chega-te cá, meu filho, para que te apalpe e conheça se és ou não o meu filho Esaú». Chegou-se ele a seu pai. Isaque, tendo-o apalpado, disse: «Na verdade, a voz é de Jacob, mas as mãos são de Esaú». E o não conheceu, porque as mãos estavam cobertas de pêlo, à semelhança das do filho primogénito. Isaque, abençoando-o, disse: «És tu o meu filho Esaú?». «Sou eu» , respondeu. E Isaque continuou: «Traze-me os manjares da tua caça, meu filho, para que minha alma te abençoe». Jacob apresentou-lhos. E depois que Isaque comeu, levou-lhe também vinho. Então Isaque, havendo-o bebido, disse: «Aproxima-te de mim e beija-me, meu filho». Ele aproximou-se e beijou-o. Logo que Isaque sentiu a fragrância dos vestidos que Jacob levava, disse, abençoando-o: «Eis que o cheiro de meu filho é como o cheiro dum campo fértil que o Senhor abençoou! Deus te conceda abundância de trigo e de vinho, pelo orvalho do céu e gordura da terra. Sirvam-te os povos; e que as tribos se inclinem diante de ti, Sê senhor dos teus irmãos; e que os filhos da tua mãe se prostrem diante de ti. Maldito seja quem te amaldiçoar; e abençoado seja quem te abençoar». Apenas Isaque acabara de falar e havendo já saído Jacob, veio Esaú e levou a seu pai manjares da sua caça, dizendo-lhe: «Levanta-te, meu pai, e come da caça de teu filho, para que a tua alma me abençoe». Isaque disse-lhe: «Pois quem és tu?». Ele respondeu: «Sou Esaú, teu filho primogénito». Isaque ficou profundamente surpreendido (até mais do que se pode acreditar) e, admirado, disse: «Quem foi, pois, aquele que há pouco me trouxe caça, e da qual comi, antes que viesses? Eu o abençoei, e abençoado será». Ouvindo Esaú estas palavras do pai, bradou com grande clamor e cheio de consternação: «Abençoa-me também a mim, meu pai». Isaque respondeu: «O teu irmão veio fraudulentamente e recebeu a tua bênção». Então Esaú acrescentou: «É, pois, com razão que ele foi chamado Jacob, pois já me suplantou duas vezes: primeiramente, tirou-me o direito de primogenitura; e agora, também com nova fraude, arrebatou-me a bênção!». E acrescentou, dirigindo-se a seu pai: «Porventura não reservaste uma bênção para mim?». Respondeu Isaque: «Eu instituí-o teu senhor e sujeitei ao seu poder todos teus irmãos; dei-lhe a posse do trigo e do vinho. Depois disto, meu filho, que resta para ti?». Esaú disse: «Acaso tens só uma bênção, meu pai? Suplico-te que me abençoes também». E Esaú chorou com fortes soluços e gritos! Então Isaque, comovido, disse-lhe: «A tua bênção estará na gordura da terra e no orvalho do céu».
}\end{paracol}

\paragraphinfo{Gradual}{Sl. 91, 2-3}
\begin{paracol}{2}\latim{
\rlettrine{B}{onum} est confitéri Dómino: et psállere nómini tuo, Altíssime. ℣. Ad annuntiándum mane misericórdiam tuam, et veritátem tuam per noctem.
}\switchcolumn\portugues{
\slettrine{É}{} bom louvar o Senhor e cantar o vosso nome, ó Altíssimo. ℣. Para anunciar de manhã a vossa misericórdia e de noite a vossa verdade.
}\end{paracol}

\paragraphinfo{Evangelho}{Lc. 15, 11-32}
\begin{paracol}{2}\latim{
\cruz Sequéntia sancti Evangélii secúndum Lucam.
}\switchcolumn\portugues{
\cruz Continuação do santo Evangelho segundo S. Lucas.
}\switchcolumn*\latim{
\blettrine{I}{n} illo témpore: Dixit Jesus pharisǽis et scribis parábolam istam: Homo quidam hábuit duos fílios, et dixit adolescéntior ex illis patri: Pater, da mihi portiónem substántiæ, quæ me cóntingit. Et divísit illis substántiam. Et non post multos dies, congregátis ómnibus, adolescéntior fílius péregre proféctus est in regiónem longínquam, et ibi dissipávit substántiam suam vivéndo luxurióse. Et postquam ómnia consummásset, facta est fames válida in regióne illa, et ipse cœpit egére. Et ábiit, et adhǽsit uni cívium regiónis illíus. Et misit illum in villam suam, ut pásceret porcos. Et cupiébat implére ventrem suum de síliquis, quas porci manducábant: et nemo illi dabat. In se autem revérsus, dixit: Quanti mercennárii in domo patris mei abúndant pánibus, ego autem hic fame péreo? Surgam, et ibo ad patrem meum, et dicam ei: Pater, peccávi in cœlum et coram te: jam non sum dignus vocari fílius tuus: fac me sicut unum de mercennáriis tuis. Et surgens venit ad patrem suum. Cum autem adhuc longe esset, vidit illum pater ipsíus, et misericórdia motus est, et accúrrens cécidit super collum ejus, et osculátus est eum. Dixítque ei fílius: Pater, peccávi in cœlum et coram te, jam non sum dignus vocari fílius tuus. Dixit autem pater ad servos suos: Cito proférte stolam primam, et indúite illum, et date ánulum in manum ejus, et calceaménta in pedes ejus: et addúcite vítulum saginátum et occídite, et manducémus et epulémur, quia hic fílius meus mórtuus erat, et revíxit: períerat, et invéntus est. Et cœpérunt epulári. Erat autem fílius ejus senior in agro: et cum veníret, et appropinquáret dómui, audívit symphóniam et chorum: et vocávit unum de servis, et interrogávit, quid hæc essent. Isque dixit illi: Frater tuus venit, et occídit pater tuus vítulum saginátum, quia salvum illum recépit. Indignátus est autem, et nolébat introíre. Pater ergo illíus egréssus, cœpit rogáre illum. At ille respóndens, dixit patri suo: Ecce, tot annis sérvio tibi, et numquam mandátum tuum præterívi, et numquam dedísti mihi hædum, ut cum amícis meis epulárer: sed postquam fílius tuus hic, qui devorávit substántiam suam cum meretrícibus, venit, occidísti illi vítulum saginátum. At ipse dixit illi: Fili, tu semper mecum es, et ómnia mea tua sunt: epulári autem et gaudére oportébat, quia frater tuus hic mórtuus erat, et revíxit: períerat, et invéntus est.
}\switchcolumn\portugues{
\blettrine{N}{aquele} tempo, disse Jesus esta parábola aos fariseus e escribas: «Um homem tinha dous filhos, dos quais o mais novo disse ao pai: «Dá-me a parte da herança que me pertence». O pai dividiu, então, entre eles os seus bens. Ainda não eram passados muitos dias, o filho mais novo, tendo ajuntado tudo quanto lhe pertencia, partiu para um país estrangeiro e longínquo, onde dissipou todo seu património, vivendo dissolutamente. Depois de ter gasto tudo, houve uma grande fome naquela região, começando ele a sentir necessidades. Então, pôs-se ao serviço dum habitante daquele país, que o mandou para a sua quinta apascentar porcos. Ele bem desejava encher o estômago das landes de que comiam os porcos, mas ninguém lhas dava! Então, meditando consigo mesmo, disse: «Quantos jornaleiros em casa de meu pai têm pão com abundância, e eu aqui morro de fome! Levantar-me-ei, irei ao meu pai e dir-lhe-ei: «Pai, pequei contra o céu e contra ti; já não sou digno de ser tratado como teu filho, mas trata-me como um dos teus jornaleiros». E, levantando-se, foi para seu pai. Estando ainda longe de casa, viu-o seu pai e encheu-se de compaixão. Logo o filho correu para o pai, lançou-se-lhe ao pescoço, beijou-o e disse-lhe: «Pai, pequei contra o céu e contra ti; já não sou digno de ser chamado teu filho». Mas o pai disse a seus servos: «Trazei-me depressa o melhor vestido, e vesti-lho; colocai-lhe um anel na mão e uns sapatos nos pés; trazei também o vitelo gordo e matai-o. E comamos e alegremo-nos, porque este meu filho estava morto e reviveu; estava perdido e foi achado!». E começaram a se banquetear. Ora, o filho mais velho estava no campo; e, quando vinha e se aproximava de casa, ouviu música e danças. Chamando, pois, um dos servos, perguntou o que era aquilo. Este disse-lhe: «O teu irmão chegou; e teu pai matou o vitelo gordo, porque ele regressou são e salvo». Então, indignou-se o irmão, não querendo entrar. Mas o pai foi ao seu encontro e rogou-lhe que entrasse. Respondendo, este disse ao pai: «Eis que há tantos anos te sirvo, sem nunca ter transgredido as tuas ordens, e nunca me deste um cabrito para comer com meus amigos; porém, agora, chega este teu filho, que malbaratou a tua fazenda com as meretrizes, e mataste para ele o vitelo gordo!». O pai disse-lhe: «Filho, tu estás sempre comigo e as minhas cousas são tuas; porém, convinha fazer um banquete e alegrarmo-nos, porque este teu irmão estava morto, e reviveu; perdera-se, e foi achado».
}\end{paracol}

\paragraphinfo{Ofertório}{Sl. 12, 4-5}
\begin{paracol}{2}\latim{
\rlettrine{I}{llúmina} óculos meos, ne umquam obdórmiam in morte: ne quando dicat inimícus meus: Præválui advérsus eum.
}\switchcolumn\portugues{
\rlettrine{I}{luminai} os meus olhos para que eu nunca adormeça na morte. Que nunca o meu inimigo diga: prevaleci contra ele.
}\end{paracol}

\paragraph{Secreta}
\begin{paracol}{2}\latim{
\rlettrine{H}{is} sacrifíciis, Dómine, concéde placátus: ut, qui própriis orámus absólvi delíc
tis, non gravémur extérnis. Per Dóminum \emph{\&c.}
}\switchcolumn\portugues{
\rlettrine{D}{eixai-Vos} aplacar, Senhor, com estes sacrifícios, a fim de que nós, que Vos pedimos perdão das nossas faltas, nos não vejamos sobrecarregados com as dos outros. Por nosso Senhor \emph{\&c.}
}\end{paracol}

\paragraphinfo{Comúnio}{Lc. 15, 32}
\begin{paracol}{2}\latim{
\rlettrine{O}{pórtet} te, fili, gaudére, quia frater tuus mórtuus fúerat, et revíxit: períerat, et invéntus est.
}\switchcolumn\portugues{
\rlettrine{C}{onvém} que te alegres, filho, porque o teu irmão estava morto, e voltou à vida; estava perdido, e foi encontrado.
}\end{paracol}

\paragraph{Postcomúnio}
\begin{paracol}{2}\latim{
\rlettrine{S}{acraménti} tui, Dómine, divína libátio, penetrália nostri cordis infúndat: et sui nos partícipes poténter effíciat. Per Dóminum \emph{\&c.}
}\switchcolumn\portugues{
\qlettrine{Q}{ue} a divina libação do vosso Sacramento, Senhor, penetre até ao íntimo do nosso coração e nos torne eficazmente participantes da sua graça. Por nosso Senhor \emph{\&c.}
}\end{paracol}

\paragraph{Oração sobre o povo}
\begin{paracol}{2}\latim{
\begin{nscenter} Orémus. \end{nscenter}
}\switchcolumn\portugues{
\begin{nscenter} Oremos. \end{nscenter}
}\switchcolumn*\latim{
Humiliáte cápita vestra Deo.
}\switchcolumn\portugues{
Inclinai as vossas cabeças diante de Deus.
}\switchcolumn*\latim{
Famíliam tuam, quǽsumus, Dómine, contínua pietáte custódi: ut, quæ in sola spe grátiæ cœléstis innítitur, cœlésti étiam protectióne muniátur. Per Dóminum \emph{\&c.}
}\switchcolumn\portugues{
Guardai, Senhor, a vossa família, Vos suplicamos, com vossa contínua bondade, a fim de que, confiando no auxílio único da graça celestial, seja sempre munida com vossa celestial protecção. Por nosso Senhor \emph{\&c.}
}\end{paracol}

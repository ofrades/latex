\subsectioninfo{Segunda-feira da 3.ª Semana da Quaresma}{Estação em S. Marcos}

\paragraphinfo{Intróito}{Sl. 55, 5}
\begin{paracol}{2}\latim{
\rlettrine{I}{n} Deo laudábo verbum, in Dómino laudábo sermónem: in Deo sperábo, non timebo, quid fáciat mihi homo. \emph{Ps. ibid., 2} Miserére mei, Deus, quóniam conculcávit me homo: tota die bellans tribulávit me.
℣. Gloria Patri \emph{\&c.}
}\switchcolumn\portugues{
\rlettrine{A}{uxiliado} com a graça de Deus, louvarei a sua palavra: auxiliado com a graça do Senhor, louvarei a sua doutrina; tenho esperança em Deus e não temo: o que poderá fazer o homem contra mim? \emph{Sl. ibid., 2} Tende compaixão de mim, meu Deus, pois o homem espezinha-me, persegue-me continuamente e enche-me de tribulações.
℣. Glória ao Pai \emph{\&c.}
}\end{paracol}

\paragraph{Oração}
\begin{paracol}{2}\latim{
\rlettrine{C}{órdibus} nostris, quǽsumus, Dómine, grátiam tuam benígnus infúnde: ut, sicut ab escis carnálibus abstinémus; ita sensus quoque nostros a nóxiis retrahámus excéssibus. Per Dóminum \emph{\&c.}
}\switchcolumn\portugues{
\rlettrine{S}{enhor,} Vos suplicamos, infundi benigno nos nossos corações a vossa graça, para que, assim como nos abstemos dos alimentos carnais, assim também guardemos os nossos sentidos dos perigosos excessos. Por nosso Senhor \emph{\&c.}
}\end{paracol}

\paragraphinfo{Epístola}{4 Rs. 5, 1-15}
\begin{paracol}{2}\latim{
Léctio libri Regum.
}\switchcolumn\portugues{
Lição do Livro dos Reis.
}\switchcolumn*\latim{
\rlettrine{I}{n} diébus illis: Náaman, princeps milítiæ regis Sýriæ, erat vir magnus apud dóminum suum, et honorátus: per illum enim dedit Dóminus salútem Sýriæ: erat autem vir fortis et dives, sed leprósus. Porro de Sýria egressi fúerant latrúnculi, et captivam dúxerant de terra Israel puéllam párvulam, quæ erat in obséquio uxoris Náaman, quæ ait ad dóminam suam: Utinam fuísset dóminus meus ad Prophétam, qui est in Samaría: profécto curásset eum a lepra, quam habet. Ingréssus est itaque Náaman ad dóminum suum, et nuntiávit ei, dicens: Sic et sic locúta est puélla de terra Israël. Dixítque ei rex Sýriæ: Vade, et mittam lítteras ad regem Israël. Qui cum proféctus esset, et tulísset secum decem talénta argénti, et sex mília áureos, et decemmutatória vestimentórum, détulit lítteras ad regem Israël in hæc verba: Cum accéperis epístolam hanc, scito, quod míserim ad te Náaman servum meum, ut cures eum a lepra sua. Cumque legísset rex Israël lítteras, scidit vestiménta sua, et ait: Numquid Deus ego sum, ut occídere possim et vivificáre, quia iste misit ad me, ut curem hóminem a lepra sua? animadvértite et vidéte, quod occasiónes quærat advérsum me. Quod cum audísset Eliséus vir Dei, scidísse videlícet regem Israël vestiménta sua, misit ad eum, dicens: Quare scidísti vestiménta tua? véniat ad me, et sciat esse prophétam in Israël. Venit ergo Náaman cum equis et cúrribus, et stetit ad óstium domus Eliséi: misítque ad eum Eliséus núntium, dicens: Vade, et laváre sépties in Jordáne, et recípiet sanitátem caro tua, atque mundáberis. Iratus Náaman recedébat, dicens: Putábam, quod egrederétur ad me, et stans invocáret nomen Dómini, Dei sui, et tángeret manu sua locum lepræ, et curáret me. Numquid non melióres sunt Abana et Pharphar, flúvii Damásci, ómnibus aquis Israël, ut laver in eis, et munder? Cum ergo vertísset se, ci abíret indígnans, accessérunt ad eum servi sui, et locúti sunt ei: Pater, et si rem grandem dixísset tibi Prophéta, certe fácere debúeras: quanto magis quia nunc dixit tibi: Laváre, et mundáberis? Descéndit, et lavit in Jordáne sépties, juxta sermónem viri Dei, et restitúta est caro ejus, sicut caro pueri párvuli, et mundátus est. Reversúsque ad virum Dei cum univérso comitátu suo, venit, et stetit coram eo, et ait: Vere scio, quod non sit álius Deus in univérsa terra, nisi tantum in Israël.
}\switchcolumn\portugues{
\rlettrine{N}{aqueles} dias, Náaman, general das tropas do rei da Síria, tinha grande valimento junto do seu senhor e era muito considerado, pois foi por ele que o Senhor salvou a Síria. Era forte e rico; porém, estava atacado de lepra. Ora, aconteceu que uns ladrões saíram da Síria e levaram cativa uma menina do país de Israel, que foi para o serviço da mulher de Náaman. Esta menina disse à sua senhora: «Ah! se o meu senhor fosse ter com o Profeta, que está na Samaria, decerto o curaria da lepra». Então Náaman foi ter com seu rei, anunciando-lhe e dizendo: «A filha de Israel falou desta e desta maneira». Disse-lhe o rei da Síria: «Vai; eu mandarei cartas ao rei de Israel». Náaman partiu, levando consigo dez talentos de prata, seis mil de ouro e dez vestidos para mudar. Levava também a carta do rei da Síria para o rei de Israel, a qual continha estas palavras: «Quando receberes esta carta, saberás que te envio o meu servo Náaman para o curares da sua lepra». Logo que o rei de Israel recebeu esta carta, rasgou os vestidos e disse: «Acaso sou Deus, para que possa dar a vida ou a morte a alguém? Pois este envia-me um homem para o curar da sua lepra! Meditai e vede, se este rei não procura um pretexto contra mim!». Então Eliseu, homem de Deus, constando-lhe que o rei de Israel rasgara os vestidos, mandou-lhe dizer: «Porque rasgaste os vestidos? Que o leproso venha ter comigo e saberá que há um Profeta em Israel». Veio, pois, Náaman com cavalos e coches, e parou à porta de Eliseu. Este mandou-lhe um mensageiro, dizendo: «Vai e lava-te sete vezes no Jordão. Então a tua carne receberá a saúde e ficarás limpo». Náaman, indignado, retirou-se, dizendo: «Eu pensava que ele viesse a mim, e que, estando presente, invocasse o nome do Senhor, seu Deus, tocasse com sua mão no lugar da lepra e me curasse. Não são, porventura, melhores as águas de Abana e de Pharphar e as dos rios de Damasco do que todas as águas de Israel, para me lavar nelas e ficar curado?». E, voltando para sua casa, ia indignado. Aproximaram-se dele os seus servos e disseram-lhe: «Pai, ainda que o Profeta tivesse pedido alguma coisa grande, certamente deverias fazê-la, quanto mais que só te disse: lava-te e ficarás limpo». Ouvindo isto, desceu ele e lavou-se sete vezes no Jordão, segundo a palavra do homem de Deus. Logo a sua carne recobrou saúde, como se fora a carne dum menino tenro, ficando limpo! Voltou, então, à presença do homem de Deus com seu séquito e disse: «Verdadeiramente, conheço que não existe outro Deus em toda a terra senão aquele que existe em Israel!».
}\end{paracol}

\paragraphinfo{Gradual}{Sl. 55, 9 \& 2}
\begin{paracol}{2}\latim{
\rlettrine{D}{eus,} vitam meam annuntiávi tibi: posuísti lácrimas meas in conspéctu tuo. ℣. Miserére mei, Dómine, quóniam conculcávit me homo: tota die bellans tribulávit me.
}\switchcolumn\portugues{
\slettrine{Ó}{} Deus, narrei-Vos toda minha vida: e vistes as minhas lágrimas. ℣. Tende compaixão de mim, meu Deus, pois o homem espezinhou-me: continuamente me persegue e enche de tribulações.
}\end{paracol}

\paragraphinfo{Trato}{Página \pageref{tratoquartacinzas}}

\paragraphinfo{Evangelho}{Lc. 4, 23-30}
\begin{paracol}{2}\latim{
\cruz Sequéntia sancti Evangélii secúndum Lucam.
}\switchcolumn\portugues{
\cruz Continuação do santo Evangelho segundo S. Lucas.
}\switchcolumn*\latim{
\blettrine{I}{n} illo témpore: Dixit Jesus pharisǽis: Utique dicétis mihi hanc similitúdinem: Médice, cura teípsum: quanta audívimus facta in Caphárnaum, fac et hic in pátria tua. Ait autem: Amen, dico vobis, quia nemo prophéta accéptus est in pátria sua. In veritáte dico vobis, multæ víduæ erant in diébus Elíæ in Israël, quando clausum est cœlum annis tribus et ménsibus sex, cum facta esset fames magna in omni terra: et ad nullam illarum missus est Elías, nisi in Sarépta Sidóniæ ad mulíerem viduam. Et multi leprósi erant in Israël sub Eliséo Prophéta: et nemo eórum mundátus est nisi Náaman Syrus. Et repléti sunt omnes in synagóga ira, hæc audiéntes. Et surrexérunt, et ejecérunt illum extra civitátem: et duxérunt illum usque ad supercílium montis, super quem cívitas illórum erat ædificáta, ut præcipitárent eum. Ipse autem tránsiens per médium illórum, ibat.
}\switchcolumn\portugues{
\blettrine{N}{aquele} tempo, disse Jesus aos fariseus: «Sem dúvida me aplicareis este provérbio: «Médico, cura-te a ti mesmo». Quantas coisas ouvimos dizer que fizeste em Cafarnaum?! Pratica-as, pois, na tua pátria». E continuou: «Na verdade vos digo: nenhum Profeta é bem recebido na sua pátria. Na verdade havia muitas viúvas em Israel, nos dias de Elias, quando o céu esteve fechado durante três anos e seis meses, e havia grande fome em toda a terra; contudo, Elias não foi enviado a nenhuma delas, mas a uma mulher viúva de Sarepta, no país de Sidon. E muitos leprosos havia em Israel no tempo do Profeta Eliseu; todavia, nenhum deles foi curado senão o siro Náaman». Ouvindo estas coisas, todos os que estavam na sinagoga se encheram de ira. E levantaram-se, expulsando-O da cidade e conduzindo-O ao cimo da montanha em que a cidade estava edificada, para daí O precipitarem! Porém, Jesus, passando pelo meio deles, se retirou.
}\end{paracol}

\paragraphinfo{Ofertório}{Sl. 54, 2-3}
\begin{paracol}{2}\latim{
\rlettrine{E}{xáudi,} Deus, oratiónem meam, et ne despéxeris deprecatiónem meam: inténde in me, et exáudi me.
}\switchcolumn\portugues{
\rlettrine{O}{uvi} a minha oração, ó Deus, e não desprezeis a minha súplica: atendei-me, escutai-me.
}\end{paracol}

\paragraph{Secreta}
\begin{paracol}{2}\latim{
\rlettrine{M}{unus,} quod tibi, Dómine, nostræ servitútis offérimus, tu salutáre nobis pérfice sacraméntum. Per Dóminum nostrum \emph{\&c.}
}\switchcolumn\portugues{
\rlettrine{P}{ermiti,} Senhor, que este dom, que Vos oferecemos em sinal da nossa sujeição, se torne para nós em um sacramento salutar. Por nosso Senhor \emph{\&c.}
}\end{paracol}

\paragraphinfo{Comúnio}{Sl. 13, 7}
\begin{paracol}{2}\latim{
\qlettrine{Q}{uis} dabit ex Sion salutáre Israël? cum avértent Dóminus captivitátem plebis suæ, exsultábit Jacob, et lætábitur Israël.
}\switchcolumn\portugues{
\qlettrine{Q}{uem} de Sião salvará Israel? Quando o Senhor extinguir o cativeiro do seu povo, Jacob exultará e Israel alegrar-se-á.
}\end{paracol}

\paragraph{Postcomúnio}
\begin{paracol}{2}\latim{
\rlettrine{P}{ræsta,} quǽsumus, omnípotens et miséricors Deus: ut, quod ore contíngimus, pura mente capiámus. Per Dóminum \emph{\&c.}
}\switchcolumn\portugues{
\rlettrine{C}{oncedei-nos,} ó Deus omnipotente e misericordioso, Vos suplicamos, que guardemos com o coração puro o Sacramento que recebemos na nossa boca. Por nosso Senhor \emph{\&c.}
}\end{paracol}

\paragraph{Oração sobre o povo}
\begin{paracol}{2}\latim{
\begin{nscenter} Orémus. \end{nscenter}
}\switchcolumn\portugues{
\begin{nscenter} Oremos. \end{nscenter}
}\switchcolumn*\latim{
Humiliáte cápita vestra Deo.
}\switchcolumn\portugues{
Inclinai as vossas cabeças diante de Deus.
}\switchcolumn*\latim{
Subvéniat nobis, Dómine, misericórdia tua: ut ab imminéntibus peccatórum nostrórum perículis, te mereámur protegénte éripi, te liberánte salvári. Per Dóminum nostrum \emph{\&c.}
}\switchcolumn\portugues{
Que a vossa misericórdia, Senhor, venha em nosso auxílio, a fim de que pela vossa protecção mereçamos ser livres dos iminentes perigos, em que incorremos pelos nossos pecados, e ser salvos com vosso socorro. Por nosso Senhor \emph{\&c.}
}\end{paracol}

\subsectioninfo{Terça-feira da 4.ª Semana da Quaresma}{Estação em S. Lourenço em Damaso}

\paragraphinfo{Intróito}{Sl. 54, 2-3}
\begin{paracol}{2}\latim{
\rlettrine{E}{xáudi,} Deus, oratiónem meam, et ne despéxeris deprecatiónem meam: inténde in me et exáudi me. \emph{Ps. ib., 3-4} Contristátus sum in exercitatióne mea: et conturbátus sum avoce inimíci et a tribulatióne peccatóris.
℣. Gloria Patri \emph{\&c.}
}\switchcolumn\portugues{
\rlettrine{O}{uvi,} ó Deus, a minha oração e não desprezeis a minha súplica! Atendei-me e escutai-me. Estou amargurado, por causa do sofrimento que me aflige! \emph{Sl. ib., 3-4} Estou perturbado, porque ouço a voz do meu inimigo e sinto a perseguição do homem pecador!
℣. Glória ao Pai \emph{\&c.}
}\end{paracol}

\paragraph{Oração}
\begin{paracol}{2}\latim{
\rlettrine{S}{acræ} nobis, quæsumus, Dómine, observatiónis jejúnia: et piæ conversationis augméntum, et tuæ propitiatiónis contínuum præstent auxílium. Per Dóminum \emph{\&c.}
}\switchcolumn\portugues{
\qlettrine{Q}{ue} esta nossa santa observância dos jejuns, Senhor, Vos rogamos, nos alcance o progresso na vida piedosa e e contínuo auxílio da vossa misericórdia. Por nosso Senhor \emph{\&c.}
}\end{paracol}

\paragraphinfo{Epístola}{Ex. 32, 7-14}
\begin{paracol}{2}\latim{
Léctio libri Exodi.
}\switchcolumn\portugues{
Lição do Livro do Êxodo.
}\switchcolumn*\latim{
\rlettrine{I}{n} diébus illis: Locútus est Dóminus ad Móysen, dicens: Descénde de monte: peccávit pópulus tuus, quem eduxísti de terra Ægýpti. Recessérunt cito de via, quam ostendísti eis: fecerúntque sibi vítulum conflátilem, et adoravérunt, atque immolántes ei hóstias, dixérunt: Isti sunt dii tui, Israël, qui te e duxérunt de terra Ægýpti. Rursúmque ait Dóminus ad Móysen: Cerno, quod pópulus iste duræ cervícis sit: dimítte me, ut irascátur furor meus contra eos, et déleam eos, faciámque te in gentem magnam. Móyses autem orábat Dóminum, Deum suum, dicens: Cur, Dómine, iráscitur furor tuus contra pópulum tuum, quem eduxísti de terra Ægýpti in fortitúdine magna et in manu robústa? Ne quæro dicant Ægýptii: Cállide edúxit eos, ut interfíceret in móntibus et deléret e terra: quiéscat ira tua, et esto placábilis super nequítia pópuli tui. Recordáre Abraham, Isaac et Israël, servórum tuórum, quibus jurásti per temetípsum, dicens: Multiplicábo semen vestrum sicut stellas cœli: et univérsam terram hanc, de qua locútus sum, dabo sémini vestro, et possidébitis eam semper. Placatúsque est Dóminus, ne fáceret malum, quod locútus fúerat advérsus pópulum suum.
}\switchcolumn\portugues{
\rlettrine{N}{aqueles} dias, falou o Senhor a Moisés, dizendo: «Desce do monte; pois o teu povo, que tiraste da terra do Egipto, pecou. Depressa eles se afastaram do caminho que lhes mostraste; pois fabricaram em metal a imagem dum bezerro e a adoraram, oferecendo-lhe sacrifícios e dizendo: «São estes os deuses, ó Israel, que te livraram da terra do Egipto». Ainda o Senhor disse a Moisés: «Conheço que este povo tem a cabeça dura. Deixa-me, pois, para que minha ira se inflame contra ele e o extermine; depois far-te-ei senhor dum grande povo». Porém, Moisés orava ao Senhor, seu Deus, dizendo: «Porque, Senhor, porque se inflama a vossa ira contra o vosso povo, que tirastes do Egipto com vossa mão forte e poderosa? Que assim não seja, Senhor, para que os egípcios não digam: «Fê-los sair com malícia para os matar nos montes e exterminá-los da terra». Aquiete-se, pois, a vossa ira. Aplacai-Vos contra a maldade do vosso povo. Lembrai-Vos de Abraão, de Isaque e de Israel, vossos servos, aos quais, sob juramento, dissestes: «Multiplicarei a vossa descendência, como as estrelas do céu, e darei à vossa geração toda a terra de que vos tenho falado, para que a possuam para sempre». Então o Senhor aplacou-se, não castigando aquele povo, como o ameaçara.
}\end{paracol}

\paragraphinfo{Gradual}{Sl. 43, 26 \& 2}
\begin{paracol}{2}\latim{
\rlettrine{E}{xsúrge,} Dómine, fer opem nobis: et líbera nos propter nomen tuum. ℣. Deus, áuribus nostris audívimus: et patres nostri annuntiavérunt nobis opus, quod operátus es in diébus eórum et in diébus antíquis.
}\switchcolumn\portugues{
\rlettrine{L}{evantai-Vos,} Senhor, acudi-nos e livrai-nos por causa do vosso nome. ℣. Ó Deus, ouvimos com os nossos ouvidos; os nossos pais anunciaram-nos quantas maravilhas praticastes nos seus dias e nos tempos antigos.
}\end{paracol}

\paragraphinfo{Evangelho}{Jo. 7, 14-31}
\begin{paracol}{2}\latim{
\cruz Sequéntia sancti Evangélii secúndum Joánnem.
}\switchcolumn\portugues{
\cruz Continuação do santo Evangelho segundo S. João.
}\switchcolumn*\latim{
\blettrine{I}{n} illo témpore: Jam die festo mediánte, ascendit Jesus in templum, et docébat. Et mirabántur Judǽi, dicéntes: Quómodo hic lítteras scit, cum non didícerit? Respóndit eis Jesus et dixit: Mea doctrína non est mea, sed ejus, qui misit me. Si quis volúerit voluntátem ejus fácere, cognóscet de doctrína, utrum ex Deo sit, an ego a meípso loquar. Qui a semetípso lóquitur, glóriam própriam quærit. Qui autem quærit glóriam ejus, qui misit eum, hic verax est, et injustítia in illo non est. Nonne Móyses dedit vobis legem: et nemo ex vobis facit legem? quid me quǽritis interfícere ? Respóndit turba, et dixit: Dæmónium habes: quis te quærit interfícere ? Respóndit Jesus et dixit eis: Unum opus feci, et omnes mirámini. Proptérea Móyses dedit vobis circumcisiónem (non quia ex Móyse est, sed ex pátribus): et in sábbato circumcíditis hóminem. Si circumcisiónem accipit homo in sábbato, ut non solvátur lex Móysi: mihi indignámini, quia totum hóminem sanum feci in sábbato? Nolíte judicáre secúndum fáciem, sed justum judícium judicáte. Dicébant ergo quidam ex Jerosólymis: Nonne hic est, quem quærunt interfícere ? Et ecce, palam lóquitur, et nihil ei dicunt. Numquid vere cognovérunt príncipes, quia hic est Christus? Sed hunc scimus, unde sit: Christus autem, cum vénerit, nemo scit, unde sit. Clamábat ergo Jesus in templo docens, et dicens: Et me scitis et, unde sim, scitis, et a meípso non veni, sed est verus, qui misit me, quem vos nescítis. Ego scio eum, quia ab ipso sum, et ipse me misit. Quærébant ergo eum apprehéndere: et nemo misit in illum manus, quia nondum vénerat hora ejus. De turba autem multi credidérunt in eum.
}\switchcolumn\portugues{
\blettrine{N}{aquele} tempo, estando já em meio os dias da festa, Jesus subiu ao templo e aí ensinava. E os judeus admiravam-se, dizendo: «Como conhece Ele as Escrituras, se as não estudou?». Respondeu-lhes Jesus: «Esta doutrina não é minha, mas d’Aquele que me mandou. Se alguém quiser fazer a vontade de Deus, conhecerá se a minha doutrina é de Deus, ou se falo de mim mesmo. Quem fala de si mesmo procura a sua própria glória; mas quem procura a glória de quem o mandou é verdadeiro; nele não há impostura. Porventura Moisés vos não deu a Lei? Contudo, nenhum de vós a cumpre! Porque procurais matar-me?». A multidão respondeu e disse: «Tu estás possesso do demónio! Quem é que quer matar-te?». Respondeu Jesus, dizendo: «Uma só obra pratiquei e todos vos maravilhais! Pois Moisés ordenou a circuncisão (não que ela venha de Moisés, mas dos Patriarcas) e vós ao sábado circuncidais o homem. Se, pois, o homem recebe a circuncisão ao sábado para não violar a Lei de Moisés, porque vos indignais comigo por Eu haver curado completamente um homem ao sábado? Não julgueis segundo a aparência, mas julgai segundo a justiça». Diziam, então, alguns de Jerusalém: «Não é Este a quem intentam matar? Eis que fala publicamente e lhe não dizem nada! Porventura os príncipes do povo terão reconhecido que Este é verdadeiramente o Cristo? Este bem sabemos donde é; porém, quando vier o Cristo, ninguém saberá donde Ele é». Clamava, então, Jesus no templo, ensinando e dizendo: «Conheceis-me e sabeis donde sou; contudo, não vim de mim mesmo; mas Aquele que me enviou é verdadeiro, ainda que vós O não conheçais. Eu conheço-O, porque sou d’Ele, e foi Ele quem me mandou». Procuravam, então, prendê-l’O; mas ninguém pôs mão sobre Ele, Porque ainda não chegara a sua hora. E muitos da multidão, ouvindo-O, acreditaram.
}\end{paracol}

\paragraphinfo{Ofertório}{Sl. 39, 2, 3 \& 4}
\begin{paracol}{2}\latim{
\rlettrine{E}{xspéctans} exspectávi Dóminum, et respéxit me: et exaudívit deprecatiónem meam: et immísit in os meum cánticum novum, hymnum Deo nostro.
}\switchcolumn\portugues{
\rlettrine{C}{om} toda a confiança esperei no Senhor; e Ele inclinou-se para mim, ouviu a minha voz e pôs na minha boca um cântico novo: um hino de louvor ao nosso Deus.
}\end{paracol}

\paragraph{Secreta}
\begin{paracol}{2}\latim{
\rlettrine{H}{æc} hóstia, Dómine, quǽsumus, emúndet nostra delicta: et, ad sacrifícium celebrándum, subditórum tibi córpora mentésque sanctíficet. Per Dóminum \emph{\&c.}
}\switchcolumn\portugues{
\qlettrine{Q}{ue} estas hóstias, Senhor, Vos suplicamos, apaguem os nossos pecados e santifiquem os corpos e as almas dos vossos servos, para celebrarem dignamente este sacrifício. Por nosso Senhor \emph{\&c.}
}\end{paracol}

\paragraphinfo{Comúnio}{Sl. 19, 6}
\begin{paracol}{2}\latim{
\rlettrine{L}{ætábimur} in salutári tuo: et in nómine Dómini, Dei nostri, magnificábimur.
}\switchcolumn\portugues{
\rlettrine{R}{egozijar-nos-emos} com vossa salvação; e seremos glorificados no nome do Senhor, nosso Deus.
}\end{paracol}

\paragraph{Postcomúnio}
\begin{paracol}{2}\latim{
\rlettrine{H}{ujus} nos, Dómine, percéptio sacraménti mundet a crímine: et ad cœléstia regna perdúcat. Per Dóminum \emph{\&c.}
}\switchcolumn\portugues{
\rlettrine{S}{enhor,} que a comunhão deste sacramento nos limpe de todo o pecado e nos conduza ao reino celestial. Por nosso Senhor \emph{\&c.}
}\end{paracol}

\paragraph{Oração sobre o povo}
\begin{paracol}{2}\latim{
\begin{nscenter} Orémus. \end{nscenter}
}\switchcolumn\portugues{
\begin{nscenter} Oremos. \end{nscenter}
}\switchcolumn*\latim{
Humiliáte cápita vestra Deo.
}\switchcolumn\portugues{
Inclinai as vossas cabeças diante de Deus.
}\switchcolumn*\latim{
Miserére, Dómine, pópulo tuo: et contínuis tribulatiónibus laborántem, propítius respiráre concéde. Per Dóminum \emph{\&c.}
}\switchcolumn\portugues{
Tende piedade, Senhor, do vosso povo e aliviai-o propiciamente das contínuas tribulações que o afligem. Por nosso Senhor \emph{\&c.}
}\end{paracol}

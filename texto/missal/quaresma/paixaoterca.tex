\subsectioninfo{Terça-feira da Semana da Paixão}{Estação em S. Ciríaco}

\paragraphinfo{Intróito}{Sl. 26, 14}
\begin{paracol}{2}\latim{
\rlettrine{E}{xspécta} Dóminum, viríliter age: et confortétur cor tuum, et sústine Dóminum. \emph{Ps. ibid., 1} Dóminus illuminátio mea et salus mea: quem timebo?
}\switchcolumn\portugues{
\rlettrine{E}{sperai} no Senhor; procedei com firmeza e o vosso coração será confortado. Esperai, pois, no Senhor. \emph{Sl. ibid., 1} O Senhor é a minha luz e salvação. A quem hei-de temer?
}\end{paracol}

\paragraph{Oração}
\begin{paracol}{2}\latim{
\rlettrine{N}{ostra} tibi, Dómine, quǽsumus, sint accepta jejúnia: quæ nos et expiándo grátia tua dignos effíciant; et ad remédia perdúcant ætérna. Per Dóminum \emph{\&c.}
}\switchcolumn\portugues{
\rlettrine{V}{os} rogamos, Senhor, permiti que os nossos jejuns Vos sejam agradáveis, a fim de que, servindo de expiação das nossas faltas, nos tornem dignos da vossa graça e nos sirvam de remédio para alcançarmos a salvação eterna. Por nosso Senhor \emph{\&c.}
}\end{paracol}

\paragraphinfo{Epístola}{Dn. 14, 27 \& 28-42}
\begin{paracol}{2}\latim{
Léctio Daniélis Prophétæ.
}\switchcolumn\portugues{
Lição do Profeta Daniel.
}\switchcolumn*\latim{
\rlettrine{I}{n} diébus illis: Congregáti sunt Babylónii ad regem, et dixérunt ei: Trade nobis Daniélem, qui Bel destrúxit et dracónem interfecit, alioquin interficiémus te et domum tuam. Vidit ergo rex, quod irrúerent in eum veheménter: et necessitáte compúlsus trádidit eis Daniélem. Qui misérunt eum in lacum leónum, et erat ibi diébus sex. Porro in lacu erant leónes septem, et dabántur eis duo córpora cotídie et duæ oves: et tunc non data sunt eis, ut devorárent Daniélem. Erat autem Hábacuc prophéta in Judǽa, et ipse cóxerat pulméntum, et intríverat panes in alvéolo: et ibat in campum, ut ferret messóribus. Dixítque Angelus Dómini ad Hábacuc: Fer prándium, quod habes, in Babylónem Daniéli, qui est in lacu leónum. Et dixit Hábacuc: Dómine, Babylónem non vidi, et lacum néscio. Et apprehéndit eum Angelus Dómini in vértice ejus, et portávit eum capíllo cápitis sui, posuítque eum in Babylóne supra lacum in ímpetu spíritus sui. Et clamávit Hábacuc, dicens: Dániel, serve Dei, tolle prándium, quod misit tibi Deus. Et ait Dániel: Recordátus es mei, Deus, et non dereliquísti diligéntes te. Surgénsque Daniel comédit. Porro Angelus Dómini restítuit Hábacuc conféstim in loco suo. Venit ergo rex die séptimo, ut lugéret Daniélem: et venit ad lacum et introspéxit, et ecce Dániel sedens in médio leónum. Et exclamávit voce magna rex, dicens: Magnus es, Dómine, Deus Daniélis. Et extráxit eum de lacu leónum. Porro illos, qui perditiónis ejus causa fúerant, intromísit in lacum, et devoráti sunt in moménto coram eo. Tunc rex ait: Páveant omnes habitántes in univérsa terra Deum Daniélis: quia ipse est salvátor, fáciens signa et mirabília in terra: qui liberávit Daniélem de lacu leónum.
}\switchcolumn\portugues{
\rlettrine{N}{aqueles} dias, reuniram-se os babilónios, foram ao rei e disseram-lhe: «Entrega-nos Daniel, que destruiu Bel e matou o dragão; senão matar-te-emos a ti e à tua família». Vendo o rei que eles instavam e o Obrigavam violentamente, entregou-lhes Daniel, levado pela necessidade. Então, lançaram-no na cova dos leões, onde esteve seis dias. Ora, havia na cova sete leões, aos quais costumavam dar de comer, todos os dias, dois cadáveres e duas oves lhas. Durante aqueles dias lhes não deram nada, para que devorassem Daniel. Ao mesmo tempo, havia na Judeia um Profeta chamado Habacuc. Estando este a preparar a sua comida, acabara de misturá-la em uma vasilha com pão e ia levá-la ao campo aos ceifeiros, quando o Anjo do Senhor lhe disse: «Leva o jantar que aí tens a Daniel, que está em Babilónia, na cova dos leões». Habacuc respondeu: «Senhor, eu nunca vi Babilónia, nem sei onde é a cova!». Então o Anjo do Senhor segurou-o pelo alto da cabeça e, levando-o pelos cabelos, conduziu-o com a rapidez do seu espírito a Babilónia, mesmo por cima da cova. E Habacuc gritou: «Daniel, servo do Senhor, toma o jantar, que Deus te enviou». Daniel disse: «Meu Deus, recordastes-Vos de mim; pois não abandonais aqueles que Vos amam». Levantando-se Daniel, comeu. Logo o Anjo do Senhor levou Habacuc para sua casa. No sétimo dia foi o rei perto da cova para prantear Daniel; e, tendo olhado para dentro, viu Daniel, sentado no meio dos leões. Logo o rei, estupefacto, exclamou em voz alta: «Grande sois Vós, Senhor, Deus de Daniel!». E tirou-o da cova dos leões. Imediatamente mandou meter na mesma cova aqueles que intentaram matar Daniel, os quais à vista de todos, num momento, foram devorados pelos leões. Então o rei disse: «Que todos os habitantes da terra temam Aquele que é o Deus de Daniel, pois Ele é o Salvador, que opera milagres e maravilhas na terra e que livrou Daniel da cova dos leões!».
}\end{paracol}

\paragraphinfo{Gradual}{Sl. 42, 1 \& 3}
\begin{paracol}{2}\latim{
\rlettrine{D}{iscérne} causam meam, Dómine: ab homine iníquo et dolóso éripe me. ℣. Emítte lucem tuam et veritátem tuam: ipsa me deduxérunt, et adduxérunt in montem sanctum tuum.
}\switchcolumn\portugues{
\rlettrine{D}{efendei} a minha causa, Senhor: livrai-me do homem iníquo e fraudulento. ℣. Que resplandeça em mim a vossa luz e a vossa verdade: e elas me conduzirão e guiarão até ao monte sagrado.
}\end{paracol}

\paragraphinfo{Evangelho}{Jo. 7, 1-13}
\begin{paracol}{2}\latim{
\cruz Sequéntia sancti Evangélii secúndum Joánnem.
}\switchcolumn\portugues{
\cruz Continuação do santo Evangelho segundo S. João.
}\switchcolumn*\latim{
\blettrine{I}{n} illo témpore: Ambulábat Jesus in Galilǽam, non enim volébat in Judǽam ambuláre, quia quærébant eum Judǽi interfícere. Erat autem in próximo dies festus Judæórum, Scenopégia. Dixérunt autem ad eum fratres ejus: Transi hinc, et vade in Judǽam, ut et discípuli tui vídeant ópera tua, quæ facis. Nemo quippe in occúlto quid facit, et quærit ipse in palam esse: si hæc facis, manifesta teipsum mundo. Neque enim fratres ejus credébant in eum. Dixit ergo eis Jesus: Tempus meum nondum advénit: tempus autem vestrum semper est parátum. Non potest mundus odísse vos: me autem odit: quia ego testimónium perhíbeo de illo, quod ópera ejus mala sunt. Vos ascéndite ad diem festum hunc, ego autem non ascénde ad diem festum istum: quia meum tempus nondum implétum est. Hæc cum dixísset, ipse mansit in Galilǽa. Ut autem ascendérunt fratres ejus, tunc et ipse ascéndit ad diem festum non maniféste, sed quasi in occúlto. Judǽi ergo quærébant eum in die festo, et dicébant: Ubi est ille? Et murmur multum erat in turba de eo. Quidam enim dicébant: Quia bonus est. Alii autem dicébant: Non, sed sedúcit turbas. Nemo tamen palam loquebátur de illo, propter metum Judæórum.
}\switchcolumn\portugues{
\blettrine{N}{aquele} tempo, andava Jesus pela Galileia e já não queria ir à Judeia, porque os judeus intentavam matá-lo. Estava já próximo o dia da festa dos tabernáculos. Disseram-Lhe, então, os seus irmãos: «Saí daqui e ide para a Judeia, para que os vossos discípulos vejam as obras que praticais, pois ninguém faz em segredo uma coisa, quando procura que ela seja conhecida. Se praticais estas maravilhas, mostrai-Vos ao mundo. Porquanto, nem mesmo os seus irmãos acreditavam n’Ele!». Jesus disse-lhes: «Meu tempo ainda não chegou, mas o vosso tempo está sempre presente. O mundo não pode aborrecer-se de vós, mas aborrece-se de mim, porque dou testemunho de que suas obras são más. Ide vós a essa festa; Eu não vou, porque o meu tempo ainda não chegou». Havendo dito isto, ficou na Galileia. Quando seus irmãos haviam já partido, resolveu então ir à festa, não manifesta, mas ocultamente. Ora, no dia da festa, os judeus procuravam-no e diziam: «Onde está Ele?». Havia, então, grande rumor na multidão a seu respeito; porquanto, uns afirmavam: é um homem de bem; outros diziam: não, pois seduz as turbas. Todavia, ninguém falava d’Ele abertamente» porque receavam os judeus.
}\end{paracol}

\paragraphinfo{Ofertório}{Sl. 9, 11-12 \& 13}
\begin{paracol}{2}\latim{
\rlettrine{S}{perent} in te omnes, qui novérunt nomen tuum, Dómine: quóniam non derelínquis quæréntes te: psállite Dómino, qui habitat in Sion: quóniam non est oblítus oratiónes páuperum.
}\switchcolumn\portugues{
\qlettrine{Q}{ue} todos aqueles que conhecem o vosso nome, Senhor, confiem em Vós, porque não abandonais os que Vos procuram. Cantai hinos ao Senhor que habita em Sião: pois não esquecerá as orações dos pobres.
}\end{paracol}

\paragraph{Secreta}
\begin{paracol}{2}\latim{
\rlettrine{H}{óstias} tibi, Dómine, deférimus immolándas: quæ temporálem consolatiónem signíficent; ut promíssa non desperémus ætérna. Per Dóminum \emph{\&c.}
}\switchcolumn\portugues{
\rlettrine{S}{enhor,} que estas hóstias, que imolamos e oferecemos em vosso louvor, nos façam sentir consolação temporal, a fim de não perdermos a esperança das promessas eternas. Por nosso Senhor \emph{\&c.}
}\end{paracol}

\paragraphinfo{Comúnio}{Sl. 24, 22}
\begin{paracol}{2}\latim{
\rlettrine{R}{édime} me, Deus Israël, ex ómnibus angústiis meis.
}\switchcolumn\portugues{
\rlettrine{L}{ivrai-me,} ó Deus de Israel, de todas minhas angústias.
}\end{paracol}

\paragraph{Postcomúnio}
\begin{paracol}{2}\latim{
\rlettrine{D}{a,} quǽsumus, omnípotens Deus: ut, quæ divína sunt, jugiter exsequéntes, donis mereámur cœléstibus propinquáre. Per Dóminum nostrum \emph{\&c.}
}\switchcolumn\portugues{
\slettrine{Ó}{} Deus omnipotente, Vos pedimos, fazei que, recebendo nós incessantemente os divinos mistérios, mereçamos alcançar os bens celestiais. Por nosso Senhor \emph{\&c.}
}\end{paracol}

\paragraph{Oração sobre o povo}
\begin{paracol}{2}\latim{
\begin{nscenter} Orémus. \end{nscenter}
}\switchcolumn\portugues{
\begin{nscenter} Oremos. \end{nscenter}
}\switchcolumn*\latim{
Humiliáte cápita vestra Deo.
}\switchcolumn\portugues{
Inclinai as vossas cabeças diante de Deus.
}\switchcolumn*\latim{
Da nobis, quǽsumus, Dómine: perseverántem in tua voluntáte famulátum; ut in diébus nostris, et mérito et número, pópulus tibi sérviens augeátur. Per Dóminum nostrum \emph{\&c.}
}\switchcolumn\portugues{
Vos suplicamos, Senhor, concedei-nos a graça da perseverança no cumprimento da vossa vontade, para que em nossos dias o povo, que Vos serve, cresça em número e em merecimentos. Por nosso Senhor \emph{\&c.}
}\end{paracol}
\subsectioninfo{Sábado da Semana da Paixão}{Estação em S. João em frente à Porta Latina}

\paragraphinfo{Intróito, Gradual, Ofertório e Comúnio}{Página \pageref{paixaosexta}}

\paragraph{Oração}
\begin{paracol}{2}\latim{
\rlettrine{P}{rofíciat,} quǽsumus, Dómine, plebs tibi dicáta piæ devotiónis afféctu: ut sacris actiónibus erudíta, quanto majestáti tuæ fit grátior, tanto donis potióribus augeátur. Per Dóminum \emph{\&c.}
}\switchcolumn\portugues{
\rlettrine{P}{ermiti,} Senhor, Vos imploramos, que o povo, que Vos é consagrado, aumente com fervor a sua piedade, a fim de que, instruindo-se com estes actos da religião, alcance tanto mais dons celestiais quanto mais se tornar agradável à vossa divina majestade. Por nosso Senhor \emph{\&c.}
}\end{paracol}

\paragraphinfo{Epístola}{Jr. 18, 18-23}
\begin{paracol}{2}\latim{
Léctio Jeremíæ Prophétæ.
}\switchcolumn\portugues{
Lição do Profeta Jeremias.
}\switchcolumn*\latim{
\rlettrine{I}{n} diébus illis: Dixérunt ímpii Judǽi ad ínvicem: Veníte, et cogitémus contra justum cogitatiónes: non enim períbit lex a sacerdóte, neque consílium a sapiénte, nec sermo a prophéta: veníte, et percutiámus eum lingua, et non attendámus ad univérsos sermónes ejus. Atténde, Dómine, ad me, et audi vocem adversariórum meórum. Numquid rédditur pro bono malum, quia fodérunt fóveam ánimæ meæ? Recordáre, quod stéterim in conspéctu tuo, ut lóquerer pro eis bonum, et avérterem indignatiónem tuam ab eis. Proptérea da fílios eórum in famem, et deduc eos in manus gládii: fiant uxóres eórum absque líberis, et víduæ: et viri eárum interficiántur morte: júvenes eórum confodiántur gládio in prǽlio. Audiátur clamor de dómibus eórum: addúces enim super eos latrónem repénte: quia fodérunt foveam, ut cáperent me, et láqueos abscondérunt pédibus meis. Tu autem, Dómine, scis omne consílium eórum advérsum me in mortem: ne propitiéris iniquitáti eórum, et peccátum eórum a fácie tua non deleátur. Fiant corruéntes in conspéctu tuo, in témpore furóris tui ab útere eis, Dómine, Deus noster.
}\switchcolumn\portugues{
\rlettrine{N}{aqueles} dias, disseram os ímpios judeus uns aos outros: «Vinde e formemos um plano contra o justo, pois não faltarão sacerdotes, que nos ensinem a lei, nem sábios, que nos aconselhem, nem profetas, que nos anunciem a palavra. Vinde, maltratemo-lo com a língua e não atendamos a nenhum dos seus ensinos». Senhor, lançai os vossos olhos para mim e ouvi a voz dos meus adversários. Porventura pagarão o bem com o mal e abrirão uma cova para me enterrarem? Lembrai-Vos de que me apresentei diante de Vós para pedir em seu favor e afastar deles a vossa indignação. Mandai, pois, a fome aos seus filhos e abandonai-os ao furor das espadas! Que suas esposas percam os filhos e fiquem viúvas! Que seus maridos morram com a peste! Que seus mancebos morram feridos pela espada em combate! Ouça-se o clamor dos seus gritos, saídos de suas casas; pois fareis cair imprevistamente sobre eles um inimigo terrível. Porquanto abriram uma cova para me lançarem nela e ocultamente armaram laços aos meus pés. Senhor, conheceis todos seus planos para me levarem à morte ; não perdoeis a sua iniquidade. Que seu pecado se não apague aos vossos olhos; que fiquem esmagados na vossa presença; e que no dia da vossa indignação sejam tratados sem clemência, ó Senhor, nosso Deus».
}\end{paracol}

\paragraphinfo{Evangelho}{Jo. 12, 10-36}
\begin{paracol}{2}\latim{
\cruz Sequéntia sancti Evangélii secúndum Joánnem.
}\switchcolumn\portugues{
\cruz Continuação do santo Evangelho segundo S. Lucas.
}\switchcolumn*\latim{
\blettrine{I}{n} illo témpore: Cogitavérunt príncipes sacerdótum, ut et Lázarum interfícerent: quia multi propter illum abíbant ex Judǽis, et credébant in Jesum. In crastínum autem turba multa, quæ vénerat ad diem festum, cum audíssent, quia venit Jesus Jerosólymam, accepérunt ramos palmárum, et processérunt óbviam ei, et clamábant: Hosánna, benedíctus, qui venit in nómine Dómini, Rex Israël. Et invénit Jesus aséllum, et sedit super eum, sicut scriptum est: Noli timére, fília Sion: ecce, Rex tuus venit sedens super pullum ásinæ. Hæc non cognovérunt discípuli ejus primum: sed quando glorificátus est Jesus, tunc recordáti sunt, quia hæc erant scripta de eo: et hæc fecérunt ei. Testimónium ergo perhibébat turba, quæ erat cum eo, quando Lázarum vocávit de monuménto, et suscitávit eum a mórtuis. Proptérea et óbviam venit ei turba: quia audiérunt eum fecísse hoc signum. Pharisǽi ergo dixérunt ad semetípsos: Vidétis, quia nihil profícimus? Ecce, mundus totus post eum ábiit. Erant autem quidam gentíles ex his, qui ascénderant, ut adorárent in die festo. Hi ergo accessérunt ad Philíppum, qui erat a Bethsáida Galilǽæ: et rogábant eum, dicéntes: Dómine, vólumus Jesum vidére. Venit Philíppus, et dicit Andréæ: Andréas rursum et Philíppus dixérunt Jesu. Jesus autem respóndit eis, dicens: Venit hora, ut clarificétur Fílius hóminis. Amen, amen, dico vobis, nisi granum fruménti cadens in terram mórtuum fúerit, ipsum solum manet: si autem mórtuum fúerit, multum fructum affert. Qui amat ánimam suam, perdet eam: et qui odit ánimam suam in hoc mundo, in vitam ætérnam custódit eam. Si quis mihi minístrat, me sequátur: et ubi sum ego, illic et miníster meus erit. Si quis mihi ministráverit, honorificábit eum Pater meus. Nunc anima mea turbáta est. Et quid dicam? Pater, salvífica me ex hac hora. Sed proptérea veni in horam hanc. Pater, clarífica nomen tuum. Venit ergo vox de cœlo: Et clarificávi, et íterum clarificábo. Turba ergo, quæ stabat et audíerat, dicebat tonítruum esse factum. Alii dicébant: Angelus ei locútus est. Respóndit Jesus et dixit: Non propter me hæc vox venit, sed propter vos. Nunc judícium est mundi, nunc princeps hujus mundi ejiciétur foras. Et ego si exaltátus fúero a terra, ómnia traham ad meípsum. (Hoc autem dicebat, signíficans, qua morte esset moritúrus.) Respóndit ei turba: Nos audívimus ex lege, quia Christus manet in ætérnum, et quómodo tu dicis: Oportet exaltári Fílium hominis? Quis est iste Fílius hominis? Dixit ergo eis Jesus: Adhuc módicum lumen in vobis est. Ambuláte, dum lucem habétis, ut non vos ténebræ comprehéndant: et qui ámbulat in ténebris, nescit, quo vadat. Dum lucem habétis, crédite in lucem: ut fílii lucis sitis. Hæc locútus est Jesus: et ábiit, et abscóndit se ab eis.
}\switchcolumn\portugues{
\blettrine{N}{aquele} tempo, os príncipes dos sacerdotes pensaram mandar matar também Lázaro, porque muitos judeus, por causa da ressurreição de Lázaro, acreditavam em Jesus, afastando-se deles. No dia seguinte, numerosas pessoas, que tinham vindo à festa, sabendo que Jesus se encontrava em Jerusalém, empunharam ramos de palmeira e iam adiante dele, gritando: «Hosana! Bendito seja o que vem em nome do Senhor, Rei de Israel!». E Jesus encontrou um jumentinho e sentou-se em cima dele, segundo o que está escrito: «Não temas, ó filha de Sião, eis que o teu Rei vem montado em um filho duma jumenta». Seus discípulos ao princípio não compreenderam estas cousas; mas, quando Jesus entrou na glória, lembraram-se de que elas haviam sido escritas a seu respeito, e que se cumpriram. Ora, a multidão que estava com Jesus, quando ressuscitou Lázaro, testemunhou este acontecimento; e por isso muitos vieram ao seu encontro, pois tinham ouvido dizer que praticara uma tal maravilha. Os fariseus disseram então uns aos outros: «Vede bem que não temos nenhum proveito. Eis que toda a gente vai atrás d’Ele!». Havia ali alguns gentios, que tinham subido para adorar durante a festa. Então, aproximaram-se de Filipe, que era de Betsaida, na Galileia, fazendo-lhe este pedido: «Senhor, queremos ver Jesus». Filipe veio e disse a André; e depois André e Filipe foram dizer a Jesus. Respondeu-lhes Jesus: «É chegada a hora em que o Filho do homem deve ser glorificado. Em verdade, em verdade vos digo: se o grão de trigo, caído na terra, não morre, ficará estéril; porém, se morre, dará muito fruto. Quem ama a sua vida perdê-la-á; e quem despreza a sua vida neste mundo conservá-la-á para a vida eterna. Se alguém quer ser meu servo, que me siga; e onde Eu estiver aí estará também o meu servo. Se alguém me servir, meu Pai o honrará. Agora a minha alma está perturbada. O que direi Eu? Pai, livrai-me desta hora! Mas foi para isto que cheguei até esta hora! Pai, glorificai o vosso nome». De repente, uma voz veio do céu, que dizia: «Eu o glorifiquei e o glorificarei ainda». A multidão que estava presente, ouvindo isto, dizia: «É um trovão». Outros diziam: «Foi um Anjo que Lhe falou». Jesus disse: «Não foi para mim que esta voz falou, mas para vós. É agora o julgamento do mundo; é agora que o príncipe deste mundo vai ser lançado fora! E Eu, quando tiver sido elevado na terra, atrairei tudo a mim». Jesus dizia isto para dar a entender de que morte havia de morrer. A multidão respondeu-Lhe: «Aprendemos na Lei que Cristo apareceria eternamente. Como dizeis, pois: «É preciso que o Filho do homem seja elevado na terra»? Quem é este Filho do homem?». Jesus disse-lhes: «Ainda por algum tempo a luz estará no meio de vós. Caminhai, pois, enquanto tendes luz, para que as trevas vos não surpreendam. Aqueles que andam nas trevas não sabem para onde vão. Enquanto tendes luz, acreditai na luz, a fim de que sejais filhos da luz». Ditas estas cousas, Jesus retirou-se e escondeu-se deles.
}\end{paracol}

\paragraph{Secreta}
\begin{paracol}{2}\latim{
\rlettrine{A}{} cunctis nos, quǽsumus, Dómine, reátibus et perículis propitiátus absólve: quos tanti mystérii tríbuis esse consórtes. Per Dóminum \emph{\&c.}
}\switchcolumn\portugues{
\rlettrine{A}{} nós, Senhor, que Vos dignastes tornar participantes deste tão solene mystério, concedei-nos o perdão de todas nossas culpas, Vos rogamos, e livrai-nos de todos os perigos. Por \emph{\&c.}
}\end{paracol}

\paragraph{Postcomúnio}
\begin{paracol}{2}\latim{
\rlettrine{D}{ivíni} múneris largitáte satiáti, quǽsumus, Dómine, Deus noster: ut hujus semper participatióne vivámus. Per Dóminum \emph{\&c.}
}\switchcolumn\portugues{
\rlettrine{S}{aciados} com a liberdade deste divino dom, Vos imploramos, Senhor, nosso Deus, permiti que sejamos aviventados sempre que dele comparticiparmos. Por nosso Senhor \emph{\&c.}
}\end{paracol}

\paragraph{Oração sobre o povo}
\begin{paracol}{2}\latim{
\begin{nscenter} Orémus. \end{nscenter}
}\switchcolumn\portugues{
\begin{nscenter} Oremos. \end{nscenter}
}\switchcolumn*\latim{
Humiliáte cápita vestra Deo.
}\switchcolumn\portugues{
Inclinai as vossas cabeças diante de Deus.
}\switchcolumn*\latim{
Tueátur, quǽsumus, Dómine, déxtera tua pópulum deprecántem: et purificátum dignánter erúdiat; ut, consolatióne præsénti, ad futúra bona profíciat. Per Dóminum \emph{\&c.}
}\switchcolumn\portugues{
Que a vossa dextra, Senhor, proteja o vosso povo suplicante, Vos pedimos, e que, purificando-se dignamente, o instrua, a fim de que a consolação, que acaba de alcançar, lhe sirva para conseguir os bens futuros. Por nosso Senhor \emph{\&c.}
}\end{paracol}

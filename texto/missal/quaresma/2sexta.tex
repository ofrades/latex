\subsectioninfo{Sexta-feira da 2.ª Semana da Quaresma}{Estação em S. Vital}

\paragraphinfo{Intróito}{Sl. 16, 15}
\begin{paracol}{2}\latim{
\rlettrine{E}{go} autem cum justítia apparébo in conspéctu tuo: satiábor, dum manifestábitur glória tua. \emph{Ps. ibid., 1} Exáudi, Dómine, justitiam meam: inténde deprecatióni meæ.
℣. Gloria Patri \emph{\&c.}
}\switchcolumn\portugues{
\rlettrine{E}{u,} porém, cheio de justiça, Senhor, aparecerei diante de Vós: e, quando me for manifestada a vossa glória, serei saciado. \emph{Sl. ibid., 1} Senhor, ouvi a minha prece justa; escutai a minha súplica.
℣. Glória ao Pai \emph{\&c.}
}\end{paracol}

\paragraph{Oração}
\begin{paracol}{2}\latim{
\rlettrine{D}{a,} quǽsumus, omnípotens Deus: ut, sacro nos purificánte jejúnio, sincéris méntibus ad sancta ventúra fácias perveníre. Per Dóminum nostrum \emph{\&c.}
}\switchcolumn\portugues{
\rlettrine{C}{oncedei-nos,} Deus omnipotente, Vos suplicamos, que, purificados com estes sagrados jejuns, cheguemos com o coração sincero às próximas solenidades. Por nosso Senhor \emph{\&c.}
}\end{paracol}

\paragraphinfo{Epístola}{Gn. 37, 6-22}
\begin{paracol}{2}\latim{
Léctio libri Genesis.
}\switchcolumn\portugues{
Lição do Livro do Génesis.
}\switchcolumn*\latim{
\rlettrine{I}{n} diébus illis: Dixit Joseph frátribus suis: Audíte sómnium meum, quod vidi: Putábam nos ligáre manípulos in agro: et quasi consúrgere manípulum meum et stare, vestrósque manípulos circumstántes adoráre manípulum meum. Respondérunt fratres ejus: Numquid rex noster eris? aut subjiciémur dicióni tuæ? Hæc ergo causa somniórum atque sermónum, invídiæ et ódii fómitem ministrávit. Aliud quoque vidit sómnium, quod narrans frátribus, ait: Vidi per sómnium, quasi solem et lunam et stellas úndecim adoráre me. Quod cum patri suo et frátribus rettulísset, increpávit eum pater suus, et dixit: Quid sibi vult hoc sómnium, quod vidísti? Num ego et mater tua et fratres tui adorábimus te super terram? Invidébant ei igitur fratres sui: pater vero rem tácitus considerábat. Cumque fratres illíus in pascéndis grégibus patris moraréntur in Sichem, dixit ad eum Israël: Fratres tui pascunt oves in Síchimis: veni, mittam te ad eos. Quo respondénte: Præsto sum, ait ei: Vade et vide, si cuncta próspera sint erga fratres tuos et pécora: et renúntia mihi, quid agatur. Missus de valle Hebron, venit in Sichem: invenítque eum vir errántem in agro, et interrogávit, quid quǽreret. At ille respóndit: Fratres meos quæro: índica mihi, ubi pascant greges. Dixítque ei vir: Recessérunt de loco isto: audívi autem eos dicéntes: Eámus in Dóthain. Perréxit ergo Joseph post fratres suos, et invénit eos in Dóthain. Qui cum vidíssent eum procul, ántequam accéderet ad eos, cogitavérunt illum occídere, et mútuo loquebántur: Ecce, somniátor venit; veníte, occidámus eum, et mittámus in cistérnam véterem, dicemúsque: Fera péssima devorávit eum: et tunc apparébit, quid illi prosint sómnia sua. Audiens autem hoc Ruben, nitebátur liberáre eum de mánibus eórum, et dicébat: Non interficiátis ánimam ejus, nec effundátis sánguinem: sed projícite eum in cistérnam hanc, quæ est in solitúdine, manúsque vestras serváte innóxias: hoc autem dicébat, volens erípere eum de mánibus eórum, et réddere patri suo.
}\switchcolumn\portugues{
\rlettrine{N}{aqueles} dias, José disse a seus irmãos: «Escutai o sonho que tive: parecia-me que estávamos a atar molhos em um campo e que meu molho se erguia e ficava de pé, enquanto que os outros o rodeavam e como que o adoravam». Responderam-lhe os irmãos: «Porventura serás nosso rei ou ficaremos sob o teu domínio?». E estes sonhos e estas palavras fomentaram cada vez mais o ódio e a inveja dos irmãos contra ele. Teve José ainda outro sonho, que contou também aos irmãos, dizendo-lhes: «Vi em sonho que o sol, a lua e onze estrelas se prostravam diante de mim». Tendo ele contado isto ao pai e aos irmãos, o pai repreendeu-o e disse-lhe: «Que significa o sonho que tiveste? Porventura eu, tua mãe e teus irmãos te adoraremos na terra?». E seus irmãos ficaram com inveja; enquanto que o pai meditava neste caso em segredo. Ora, um dia, estando os irmãos de José a apascentar os rebanhos do pai em Siquém, disse Israel (Jacob) a José: «Os teus irmãos apascentam as ovelhas nas terras de Siquém; vem, pois, para que te mande onde eles estão». José respondeu: «Estou pronto». Jacob disse-lhe: «Vai, vê se teus irmãos estão bem; se os gados se encontram em bom estado; e volta a dizer-me o que acontece por lá». Partiu José do vale de Hébron e chegou a Siquém. Um homem encontrou-o errante no caminho e interrogou-o para saber o que ele procurava. «Procuro meus irmãos; indicai-me onde apascentam os rebanhos». O homem respondeu: «Saíram já deste lugar; porém, ouvi-lhes dizer: vamos até Dótain». Partiu, pois, José após os irmãos e encontrou-os em Dótain. Estes, tendo-o visto ao longe, combinaram matá-lo, antes que se aproximasse. E diziam uns aos outros: «Aí vem o sonhador; vamos; matemo-lo; e lancemo-lo numa cisterna velha. Depois diremos que um animal feroz o devorou. Então se verá de que lhe serviram os sonhos». Mas Ruben esforçava-se para o livrar, dizendo: «Não o mateis nem derrameis o seu sangue. Lançai-o antes numa cisterna, que há no deserto, e conservai as vossas mãos puras». Isto dizia, querendo arrancá-lo das mãos dos irmãos e entregá-lo ao pai.
}\end{paracol}

\paragraphinfo{Gradual}{Sl. 119, 1-2}
\begin{paracol}{2}\latim{
\rlettrine{A}{d} Dóminum, cum tribulárer, clamávi, et exaudívit me. ℣. Dómine, líbera ánimam meam a lábiis iníquis et a lingua dolósa.
}\switchcolumn\portugues{
\rlettrine{E}{nquanto} estava na tribulação, chamei pelo Senhor, que me atendeu. ℣. Senhor, livrai a minha alma dos lábios iníquos e da. língua enganadora.
}\end{paracol}

\paragraphinfo{Trato}{Página \pageref{tratoquartacinzas}}

\paragraphinfo{Evangelho}{Mt, 21, 33-46}
\begin{paracol}{2}\latim{
\cruz Sequéntia sancti Evangélii secúndum Matthǽum.
}\switchcolumn\portugues{
\cruz Continuação do santo Evangelho segundo S. Mateus.
}\switchcolumn*\latim{
\blettrine{I}{n} illo témpore: Dixit Jesus turbis Judæórum et princípibus sacerdótum parábolam hanc: Homo erat paterfamílias, qui plantávit víneam, et sepem circúmdedit ei, et fodit in ea tórcular, et ædificávit turrim, et locávit eam agrícolis, et péregre proféctus est. Cum autem tempus frúctuum appropinquásset, misit servos suos ad agrícolas, ut accíperent fructus ejus. Et agrícolæ, apprehénsis servis ejus, alium cecidérunt, alium occidérunt, álium vero lapidavérunt. Iterum misit álios servos plures prióribus, et fecérunt illis simíliter. Novíssime autem misit ad eos fílium suum, dicens: Verebúntur fílium meum. Agrícolæ autem vidéntes fílium, dixérunt intra se: Hic est heres, veníte, occidámus eum, et habébimus hereditátem ejus. Et apprehénsum eum ejecérunt extra víneam, et occidérunt. Cum ergo vénerit dóminus víneæ, quid fáciet agrícolis illis? Ajunt illi: Malos male perdet: et víneam suam locábit áliis agrícolis, qui reddant ei fructum tempóribus suis. Dicit illis Jesus: Numquam legístis in Scriptúris: Lápidem, quem reprobavérunt ædificántes, hic factus est in caput ánguli? A Dómino factum est istud, et est mirábile in óculis nostris. Ideo dico vobis, quia auferétur a vobis regnum Dei, et dábitur genti faciénti fructus ejus. Et qui cecíderit super lápidem istum, confringétur: super quem vero cecíderit, cónteret eum. Et cum audíssent príncipes sacerdótum et pharisǽi parábolas ejus, cognovérunt, quod de ipsis díceret. Et quæréntes eum tenére, timuérunt turbas: quóniam sicut Prophétam eum habébant.
}\switchcolumn\portugues{
\blettrine{N}{aquele} tempo, disse Jesus à turba dos judeus e aos príncipes dos sacerdotes esta parábola: «Havia um homem, pai de família, que plantou uma vinha, cercou-a com uma sebe, construiu nela um lagar, edificou uma torre, arrendou-a a uns lavradores e partiu para longe. Aproximando-se o tempo da colheita, mandou o pai de família os seus servos aos lavradores, para deles receberem os frutos. Mas os lavradores, prendendo os servos, espancaram um, mataram, outro e apedrejaram outro! Tornou o pai de família a mandar outros servos, em maior número do que os primeiros, e fizeram-lhes o mesmo. Por último, mandou-lhes o seu próprio filho, dizendo: «A meu filho, ao menos, guardarão respeito». Mas os lavradores, vendo o filho, disseram entre si: «Este é o herdeiro; vamos, matemo-lo e ficaremos com sua herança». E, segurando-o, lançaram-no fora da vinha e mataram-no». «Ora — continuou Jesus — quando vier o dono da vinha, que fará a estes lavradores?». Eles responderam: «Castigará os malvados e arrendará a sua vinha a outros lavradores, que paguem o fruto em seu tempo». Disse-lhes Jesus: «Nunca lestes nas escrituras: «A pedra que os edificadores rejeitaram tornou-se em pedra angular? Isto é uma obra do Senhor e é maravilha aos nossos olhos?!». Por isso vos digo: o reino de Deus ser-vos-á tirado e dado a outro povo, que produza os seus frutos. Quem cair sobre esta pedra quebrar-se-á; e aquele sobre quem ela cair ficará esmagado». Ouvindo isto, perceberam os príncipes dos sacerdotes e os fariseus que falava deles. Então quiseram prendê-l’O, mas tiveram medo das turbas, pois estas tinham-n’O na conta de Profeta.
}\end{paracol}

\paragraphinfo{Ofertório}{Sl. 39, 14 \& 15}
\begin{paracol}{2}\latim{
\rlettrine{D}{ómine,} in auxílium meum réspice: confundántur et revereántur, qui quærunt ánímam meam, ut áuferant eam: Dómine, in auxílium meum réspice.
}\switchcolumn\portugues{
\rlettrine{S}{enhor,} volvei para mim um olhar de protecção. Caiam na confusão e no opróbrio aqueles que procuram tirar-me a vida! Senhor, volvei um olhar protector para mim.
}\end{paracol}

\paragraph{Secreta}
\begin{paracol}{2}\latim{
\rlettrine{H}{æc} in nobis sacrifícia, Deus, et actióne permáneant, et operatióne firméntur. Per Dóminum \emph{\&c.}
}\switchcolumn\portugues{
\qlettrine{Q}{ue} estes sacrifícios, ó Deus, exerçam em nós uma acção permanente e uma influência fortificadora. Por nosso Senhor \emph{\&c.}
}\end{paracol}

\paragraphinfo{Comúnio}{Sl. 11, 8}
\begin{paracol}{2}\latim{
\rlettrine{T}{u,} Dómine, servábis nos, et custódies nos a generatióne hac in ætérnum.
}\switchcolumn\portugues{
\rlettrine{S}{enhor,} guardar-nos-eis e defender-nos-eis sempre desta geração.
}\end{paracol}

\paragraph{Postcomúnio}
\begin{paracol}{2}\latim{
\rlettrine{F}{ac} nos, quǽsumus, Dómine: accépto pígnore salútis ætérnæ, sic téndere congruénter; ut ad eam perveníre póssimus. Per Dóminum \emph{\&c.}
}\switchcolumn\portugues{
\rlettrine{F}{azei,} Senhor, Vos suplicamos, que, tendo nós recebido o penhor da salvação eterna, de tal sorte procuremos merecê-la que possamos possuí-la. Por nosso Senhor \emph{\&c.}
}\end{paracol}

\paragraph{Oração sobre o povo}
\begin{paracol}{2}\latim{
\begin{nscenter} Orémus. \end{nscenter}
}\switchcolumn\portugues{
\begin{nscenter} Oremos. \end{nscenter}
}\switchcolumn*\latim{
Humiliáte cápita vestra Deo.
}\switchcolumn\portugues{
Inclinai as vossas cabeças diante de Deus.
}\switchcolumn*\latim{
Da, quǽsumus, Dómine, pópulo tuo salútem mentis et córporis: ut, bonis opéribus inhæréndo, tuæ semper virtútis mereátur protectióne deféndi. Per Dóminum \emph{\&c.}
}\switchcolumn\portugues{
Senhor, concedei ao vosso povo, Vos suplicamos, a saúde da alma e do corpo, a fim de que, ocupando-se em boas obras, mereça ser sempre assistido com a protecção da vossa majestade. Por nosso Senhor \emph{\&c.}
}\end{paracol}

\subsectioninfo{Quarta-feira da 4.ª Semana da Quaresma}{Estação em S. Paulo fora dos Muros}

\paragraphinfo{Intróito}{Ez. 36, 23-26}
\begin{paracol}{2}\latim{
\rlettrine{D}{um} sanctificátus fúero in vobis, congregábo vos de univérsis terris: et effúndam super vos aquam mundam, et mundabímini ab ómnibus inquinaméntis vestris: et dabo vobis spíritum novum. \emph{Ps. 33, 2} Benedícam Dóminum in omni témpore: semper laus ejus in ore meo.
℣. Gloria Patri \emph{\&c.}
}\switchcolumn\portugues{
\qlettrine{Q}{uando} eu for santificado no meio de vós, reunir-vos-ei de todos os países. Então vos aspergirei com água pura, ficareis limpos de todas as iniquidades e vos darei um espírito novo. \emph{Sl. 33, 2} Bendirei o Senhor continuamente; sempre a minha boca o louvará.
℣. Glória ao Pai \emph{\&c.}
}\end{paracol}

\begin{paracol}{2}\latim{
\begin{nscenter} Orémus. Flectámus génua. \end{nscenter}
}\switchcolumn\portugues{
\begin{nscenter} Oremos. Ajoelhemos! \end{nscenter}
}\switchcolumn*\latim{
℟. Leváte.
}\switchcolumn\portugues{
℟. Levantai-vos!
}\end{paracol}

\paragraph{Oração}
\begin{paracol}{2}\latim{
\rlettrine{D}{eus,} qui et justis prǽmia meritórum et peccatóribus per jejúnium véniam præbes: miserére supplícibus tuis; ut reátus nostri conféssio indulgéntiam valeat percípere delictórum. Per Dóminum \emph{\&c.}
}\switchcolumn\portugues{
\slettrine{Ó}{} Deus, que por meio do jejum concedeis aos justos a recompensa de seus méritos e aos pecadores o perdão de seus pecados, tende misericórdia dos vossos fiéis suplicantes, para que, por meio da confissão das nossas culpas, possamos alcançar o perdão das mesmas. Por nosso Senhor \emph{\&c.}
}\end{paracol}

\paragraphinfo{Lição}{Ez. 36, 23-28}
\begin{paracol}{2}\latim{
Léctio Ezechiélis Prophétæ.
}\switchcolumn\portugues{
Lição da Ep.ª do B. Ap.º Paulo aos Coríntios.
}\switchcolumn*\latim{
\rlettrine{H}{æc} dicit Dóminus Deus: Sanctificábo nomen meum magnum, quod pollútum est inter gentes, quod polluístis in médio eárum: ut sciant gentes, quia ego Dóminus, cum sanctificátus Mero in vobis coram eis. Tollam quippe vos de géntibus, et congregábo vos de univérsis terris, et addúcam vos in terram vestram. Et effúndam super vos aquam mundam, et mundabímini ab ómnibus inquinaméntis vestris, et ab univérsis ídolis vestris mundábo vos. Et dabo vobis cor novum, et spíritum novum ponam in médio vestri: et áuferam cor lapídeum de carne vestra, et dabo vobis cor cárneum. Et spíritum meum ponam in médio vestri: et fáciam, ut in præcéptis meis ambulétis, et judicia mea custodiátis et operémini. Et habitábitis in terra, quam dedi pátribus vestris: et éritis mihi in pópulum, et ego ero vobis in Deum: dicit Dóminus omnípotens.
}\switchcolumn\portugues{
\rlettrine{A}{ssim} fala o Senhor Deus: «Santificarei o meu admirável nome, que foi profanado pelos povos (entre os quais vós também o desonrastes), para que as nações conheçam que sou o Senhor. Quando Eu for santificado no meio de vós, diante dos meus olhos, vos tirarei do meio dos povos, vos reunirei de todos os países e vos conduzirei à vossa terra. Então vos aspergirei Com água pura; vos limparei de todas as nódoas; e vos purificarei dos vossos ídolos. E dar-vos-ei um coração novo; insuflarei no vosso íntimo um espírito novo; tirarei da vossa carne o coração de pedra que tendes, dando-vos um coração de carne; o meu espírito ficará dentro de vós; e farei que caminheis orientados pelos meus preceitos e que guardeis e pratiqueis os meus mandamentos. Então habitareis na terra que dei a vossos pais; sereis para mim o meu povo; e serei para vós o vosso Deus»: diz o Senhor omnipotente.
}\end{paracol}

\paragraphinfo{Gradual}{Sl. 38, 12 \& 6}
\begin{paracol}{2}\latim{
\rlettrine{V}{eníte,} fílii, audíte me: timórem Dómini docébo vos. ℣. Accédite ad eum, et illuminámini: et fácies vestræ non confundéntur.
}\switchcolumn\portugues{
\rlettrine{V}{inde,} filhos, escutai-me: Eu vos ensinarei a temer o Senhor. ℣. Aproximai-vos d’Ele e ficareis cheios de luz; a confusão não mais cobrirá o vosso rosto.
}\end{paracol}

\paragraph{Oração}
\begin{paracol}{2}\latim{
\rlettrine{P}{ræsta,} quǽsumus, omnípotens Deus: ut, quos jejúnia votiva castígant, ipsa quoque devótio sancta lætíficet; ut, terrénis afféctibus mitigátis, facílius cœléstia capiámus. Per Dóminum \emph{\&c.}
}\switchcolumn\portugues{
\slettrine{Ó}{} Deus omnipotente, castigando nós voluntariamente os nossos corpos com estes solenes jejuns, concedei-nos, Vos suplicamos, que sejamos consolados com a alegria duma piedade santa, a fim de que, sendo mitigado o ardor dos nossos afectos terrenos, alcancemos mais facilmente os bens celestiais. Por nosso Senhor \emph{\&c.}
}\end{paracol}

\paragraphinfo{Epístola}{Is. 1, 16-19}
\begin{paracol}{2}\latim{
Léctio Isaíæ Prophétæ.
}\switchcolumn\portugues{
Lição do Profeta Isaías.
}\switchcolumn*\latim{
\rlettrine{H}{æc} dicit Dóminus Deus: Lavámini, mundi estóte, auférte malum cogitatiónum vestrárum ab óculis meis: quiéscite ágere pervérse, díscite benefácere: quǽrite judícium, subveníte opprésso, judicáte pupíllo, deféndite víduam. Et veníte et argúite me, dicit Dóminus: si fúerint peccáta vestra ut cóccinum, quasi nix dealbabúntur: et si fúerint rubra quasi vermículus, velut lana alba erunt. Si voluéritis et audiéritis me, bona terræ comedétis: dicit Dóminus omnípotens.
}\switchcolumn\portugues{
\rlettrine{I}{sto} diz o Senhor Deus: «Levantai-vos, purificai-vos, afastai diante dos meus olhos a malícia dos vossos pensamentos, deixai de praticar o mal, aprendei a praticar o bem, procurai a justiça, amparai o oprimido, respeitai os direitos do órfão e defendei a viúva. E, depois de assim procederdes, se for Eu não misericordioso, vinde e acusai-me»: diz o Senhor. «Ainda que os vossos pecados fossem como o escarlate, tornar-se-iam brancos como a neve; e, quando fossem encarnados como o carmesim, tornar-se-iam brancos como a lã. Se quiserdes ouvir a minha voz, comereis as coisas boas desta terra»: diz o Senhor omnipotente.
}\end{paracol}

\paragraphinfo{Gradual}{Sl. 32, 12 \& 6}
\begin{paracol}{2}\latim{
\rlettrine{B}{eáta} gens, cujus est Dóminus Deus eórum: pópulus, quem elégit Dóminus in hereditátem sibi. ℣. Verbo Dómini cœli firmáti sunt: et spíritu oris ejus omnis virtus eórum.
}\switchcolumn\portugues{
\rlettrine{B}{em-aventurada} a nação cujo Deus é o Senhor; bem-aventurado o povo que Ele escolheu para sua herança. Pela palavra do Senhor foram criados os céus; e do sopro da sua boca veio toda sua virtude.
}\end{paracol}

\paragraphinfo{Trato}{Página \pageref{tratoquartacinzas}}

\paragraphinfo{Evangelho}{Jo. 9, 1-38}
\begin{paracol}{2}\latim{
\cruz Sequéntia sancti Evangélii secúndum Joánnem.
}\switchcolumn\portugues{
\cruz Continuação do santo Evangelho segundo S. João.
}\switchcolumn*\latim{
\blettrine{I}{n} illo témpore: Prætériens Jesus vidit hóminem cæcum a nativitáte: et interrogavérunt eum discípuli ejus: Rabbi, quis peccávit, hic aut paréntes ejus, ut cæcus nascerétur? Respóndit Jesus: Neque hic peccávit neque paréntes ejus: sed ut manifesténtur ópera Dei in illo. Me opórtet operári ópera ejus, qui misit me, donec dies est: venit nox, quando nemo potest operári. Quámdiu sum in mundo, lux sum mundi. Hæc cum dixísset, éxspuit in terram, et fecit lutum ex sputo, et linívit lutum super óculos ejus, et dixit ei: Vade, lava in natatória Síloe (quod interpretátur Missus). Abiit ergo, et lavit, et venit videns. Itaque vicíni, et qui víderant eum prius, quia mendícus erat, dicébant: Nonne hic est, qui sedébat et mendicábat? Alii dicébant: Quia hic est. Alii autem: Nequáquam, sed símilis est ei. Ille vero dicébat: Quia ego sum. Dicébant ergo ei: Quómodo apérti sunt tibi óculi? Respóndit: Ille homo, qui dícitur Jesus, lutum fecit, et unxit oculos meos, et dixit mihi: Vade ad natatória Síloe, et lava. Et ábii, et lavi, et vídeo. Et dixérunt ei: Ubi est ille? Ait: Néscio. Addúcunt eum ad pharisǽos, qui cæcus fúerat. Erat autem sábbatum, quando lutum fecit Jesus, et apéruit óculos ejus. Iterum ergo interrogábant eum pharisǽi, quómodo vidísset. Ille autem dixit eis: Lutum mihi posuit super oculos, et lavi, et video. Dicébant ergo ex pharisæis quidam: Non est hic homo a Deo, qui sábbatum non custódit. Alii autem dicébant: Quómodo potest homo peccator hæc signa facere ? Et schisma erat inter eos. Dicunt ergo cæco íterum: Tu quid dicis de illo, qui apéruit óculos tuos? Ille autem dixit: Quia Prophéta est. Non credidérunt ergo Judǽi de illo, quia cæcus fuísset et vidísset, donec vocavérunt paréntes ejus, qui víderat: et interrogavérunt eos, dicéntes: Hic est fílius vester, quem vos dícitis, quia cæcus natus est? Quómodo ergo nunc videt? Respondérunt eis paréntes ejus, et dixérunt: Scimus, quia hic est fílius noster, et quia cæcus natus est: quómodo autem nunc vídeat, nescímus: aut quis ejus aperuit oculos, nos nescímus: ipsum interrogáte, aetatem habet, ipse de se loquatur. Hæc dixérunt paréntes ejus, quóniam timébant Judǽos: jam enim conspiráverant Judǽi, ut, si quis eum confiterétur esse Christum, extra synagógam fíeret. Proptérea paréntes ejus dixérunt: Quia ætatem habet, ipsum interrogáte. Vocavérunt ergo rursum hóminem, qui fúerat cæcus, et dixérunt ei: Da glóriam Deo. Nos scimus, quia hic homo peccátor est. Dixit ergo eis ille: Si peccátor est, néscio: unum scio, quia, cæcus cum essem, modo vídeo. Dixérunt ergo illi: Quid fecit tibi? quómodo apéruit tibi óculos? Respóndit eis: Dixi vobis jam, et audístis: quid íterum vultis audíre? Numquid et vos vultis discípuli ejus fíeri? Male dixérunt ergo ei, et dixérunt: Tu discípulus illíus sis: nos autem Móysi discípuli sumus. Nos scimus, quia Moysi locútus est Deus: hunc autem nescímus, unde sit. Respóndit ille homo et dixit eis: In hoc enim mirábile est, quia vos néscitis, unde sit, et apéruit meos óculos: scimus autem, quia peccatóres Deus non audit: sed, si quis Dei cultor est et voluntátem ejus facit, hunc exáudit. A sǽculo non est audítum, quia quis apéruit óculos cæci nati. Nisi esset hic a Deo, non póterat fácere quidquam. Respondérunt et dixérunt ei: In peccátis natus es totus, et tu doces nos? Et ejecérunt eum foras. Audívit Jesus, quia ejecérunt eum foras, et cum invenísset eum, dixit ei: Tu credis in Fílium Dei? Respóndit ille et dixit: Quis est, Dómine, ut credam in eum? Et dixit ei Jesus: Et vidísti eum, et qui lóquitur tecum, ipse est. At ille ait: Credo, Dómine. \emph{(Hic genuflectitur)} Et prócidens adorávit eum.
}\switchcolumn\portugues{
\blettrine{N}{aquele} tempo, indo Jesus a passar, viu um cego de nascença. Então os seus discípulos interrogaram-n’O: «Mestre, quem foi que pecou, para que ele tivesse nascido cego? Ele ou os seus pais?». Respondeu Jesus: «Nem ele, nem os pais; mas aconteceu isto para que as obras de Deus se manifestassem nele. É necessário que Eu faça as obras d’Aquele que me mandou, enquanto é dia; porque, quando vem a noite, não se pode trabalhar. Enquanto estou no mundo, sou a luz do mundo». Tendo assim falado, cuspiu no chão, fez uma espécie de lodo com a saliva, e, pondo esse lodo nos olhos do cego, disse-lhe: «Vai e lava-te na piscina de Siloé» (que quer dizer: enviado). Foi ele, lavou-se e voltou, vendo já; de sorte que os vizinhos e aqueles que antes o tinham visto pedir esmola diziam: «Porventura não é este o que antes estava sentado a mendigar?». Uns diziam: «É este mesmo». Mas outros diziam: «Não é este; mas outro que se parece com ele». Porém, ele dizia: «Sou eu mesmo». Então diziam-lhe: «Como, pois, se abriram os teus olhos?». Ele respondeu: «Aquele homem que se chama Jesus fez lodo, untou-me os olhos com o lodo e disse-me: «Vai à piscina de Siloé e lava-te»• E eu fui, lavei-me e agora vejo!». E disseram-lhe: «Onde está Ele?». «Não sei» , disse o curado. Levaram, então, aos fariseus aquele que havia sido cego • Ora, era sábado quando Jesus fizera o lodo e lhe abrira os olhos. De novo os fariseus lhe perguntaram como fora curado. Ele disse-lhes: «Pôs-me lodo nos olhos, lavei-me e vejo». Alguns fariseus diziam: «Esse homem não é de Deus, pois não observa o sábado». Outros diziam: «Mas como pode um homem pecador fazer tais maravilhas?». E havia dissenção entre eles. Então, chamaram mais uma vez o cego e perguntaram-lhe: «Que dizes daquele que te abriu os olhos?». Ele respondeu: «É um Profeta». Mas os judeus não acreditaram que ele tivesse sido cego e que agora houvesse recobrado a vista. Chamaram, pois, à sua presença os pais do que recobrara a vista, dizendo-lhes: «É este o vosso filho, que dizeis ter nascido cego? Como, pois, vê ele agora?». Os pais responderam: «Sabemos que este é o nosso filho e que nasceu cego; mas ignoramos como agora vê e quem foi que lhe abriu os olhos. Interrogai-o a ele próprio, pois já tem idade para falar de si». Os pais falaram assim, porque temiam os judeus; pois estes haviam resolvido que todo aquele que reconhecesse Jesus como Cristo seria expulso da sinagoga. Por isso os pais disseram: «Já tem idade, interrogai-o a ele próprio». Novamente os fariseus chamaram o que havia sido cego e disseram-lhe: «Dá glória a Deus! Sabemos que esse homem é um pecador». Replicou o que fora cego: «Se Ele é pecador, não sei. Uma coisa sei: é que eu era cego e agora vejo». E disseram-lhe ainda: «Que te fez Ele? Como te abriu os olhos?». Respondeu-lhes: «Já vo-lo disse e ouvistes; para que quereis ouvir ainda? Porventura quereis também fazer-vos seus discípulos?». Então os fariseus amaldiçoaram-no e disseram-lhe: «Sejas tu seu discípulo; nós somos discípulos de Moisés. Sabemos que Deus falou a Moisés; porém, este não sabemos donde vem». Respondeu aquele homem: «Na verdade, é para admirar que não saibais donde vem Aquele que me abriu os olhos, pois sabemos que Deus não atende aos pecadores; mas, se alguém é temente a Deus e faz a sua vontade, Deus atende-o. Nunca se ouviu dizer que uma qualquer pessoa tenha aberto os olhos a um cego de nascença! Se Ele não fosse Deus, nada poderia fazer!». Responderam-lhe eles: «Nasceste inteiramente em pecado e queres ensinar-nos?». E expulsaram-no!... Logo que Jesus ouviu dizer que haviam expulsado o cego, foi ao seu encontro e disse-lhe: «Tu crês no Filho de Deus?». Respondeu o cego, dizendo: «Quem é ele, Senhor, para que eu creia?». Jesus disse-lhe: «Já o tens visto. É Aquele que fala contigo!». E o cego disse: «Creio, Senhor!». \emph{(Todos devem ajoelhar)} E de joelhos o adorou!
}\end{paracol}

\paragraphinfo{Ofertório}{Sl. 65, 8-9 \& 20}
\begin{paracol}{2}\latim{
\rlettrine{B}{enedícite,} gentes, Dóminum, Deum nostrum, et obaudíte vocem laudis ejus: qui pósuit ánimam meam ad vitam, et non dedit commovéri pedes meos: benedíctus Dóminus, qui non amóvit deprecatiónem mam, et misericórdiam suam a me.
}\switchcolumn\portugues{
\slettrine{Ó}{} povos, bendizei o Senhor, nosso Deus, e fazei ressoar os seus louvores: foi Ele quem conservou a vida à minha alma e não deixou que meus pés tropeçassem. Bendito seja o Senhor, que não desprezou a minha oração nem afastou de mim a sua misericórdia.
}\end{paracol}

\paragraph{Secreta}
\begin{paracol}{2}\latim{
\rlettrine{S}{úpplices} te rogámus, omnípotens Deus: ut his sacrifíciis peccáta nostra mundéntur; quia tunc veram nobis tríbuis et mentis et córporis sanitátem. Per Dóminum \emph{\&c.}
}\switchcolumn\portugues{
\rlettrine{O}{rdenai,} ó Deus omnipotente, humildemente Vos rogamos, que estes sacrifícios nos purifiquem e nos concedam a verdadeira saúde da alma e do corpo. Por nosso Senhor \emph{\&c.}
}\end{paracol}

\paragraphinfo{Comúnio}{Jo. 9, 11}
\begin{paracol}{2}\latim{
\rlettrine{L}{utum} fecit ex sputo Dóminus, et linívit óculos meos: et ábii, et lavi, et vidi, et crédidi Deo.
}\switchcolumn\portugues{
\rlettrine{O}{} Senhor fez lodo com sua saliva e untou os meus olhos. Então, lavei-me, vi e acreditei em Deus.
}\end{paracol}

\paragraph{Postcomúnio}
\begin{paracol}{2}\latim{
\rlettrine{S}{acraménta,} quæ súmpsimus, Dómine, Deus noster: et spirituálibus nos répleant aliméntis, et corporálibus tueántur auxíliis. Per Dóminum \emph{\&c.}
}\switchcolumn\portugues{
\qlettrine{Q}{ue} estes sacramentos, que recebemos, Senhor, nosso Deus, saciem nossas almas com o alimento espiritual e protejam nossos corpos com os auxílios temporais. Por nosso Senhor \emph{\&c.}
}\end{paracol}

\paragraph{Oração sobre o povo}
\begin{paracol}{2}\latim{
\begin{nscenter} Orémus. \end{nscenter}
}\switchcolumn\portugues{
\begin{nscenter} Oremos. \end{nscenter}
}\switchcolumn*\latim{
Humiliáte cápita vestra Deo.
}\switchcolumn\portugues{
Inclinai as vossas cabeças diante de Deus.
}\switchcolumn*\latim{
Páteant aures misericórdiæ tuæ. Dómine, précibus supplicántium: et, ut peténtibus desideráta concédas; fac eos, quæ tibi sunt plácita, postuláre. Per Dóminum \emph{\&c.}
}\switchcolumn\portugues{
Senhor, que os ouvidos da vossa misericórdia escutem atentos as orações daqueles que a imploram; e, a fim de que alcancem o que desejam, fazei que Vos peçam o que Vos é agradável. Por nosso Senhor \emph{\&c.}
}\end{paracol}

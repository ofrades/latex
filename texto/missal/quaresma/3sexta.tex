\subsectioninfo{Sexta-feira da 3.ª Semana da Quaresma}{Estação em S. Lourenço, em Lucina}

\paragraphinfo{Intróito}{Sl. 85, 17}
\begin{paracol}{2}\latim{
\rlettrine{F}{ac} mecum, Dómine, signum in bonum: ut vídeant, qui me oderunt, et confundántur: quóniam tu, Dómine, adjuvísti me et consolátus es me. \emph{Ps. ibid., 1} Inclína, Dómine, aurem tuam, et exáudi me: quóniam inops et pauper sum ego.
℣. Gloria Patri \emph{\&c.}
}\switchcolumn\portugues{
\rlettrine{P}{raticai,} Senhor, um dos vossos prodígios em meu favor, para que meus inimigos contemplem o vosso poder e sejam confundidos; Pois Vós, Senhor, sois o meu auxílio e conforto. \emph{Sl. ibid., 1} Inclinai, Senhor, os vossos ouvidos para mim e escutai-me; porque sou pobre e infeliz.
℣. Glória ao Pai \emph{\&c.}
}\end{paracol}

\paragraph{Oração}
\begin{paracol}{2}\latim{
\qlettrine{J}{ejúnia} nostra, quǽsumus, Dómine, benígno favóre proséquere: ut, sicut ab aliméntis abstinémus in córpore; ita a vítiis jejunémus in mente. Per Dóminum \emph{\&c.}
}\switchcolumn\portugues{
\rlettrine{A}{companhai} os nossos jejuns com vossa benigna graça, Senhor, a fim de que, assim como o nosso corpo se abstém das carnes, assim a nossa alma se abstenha dos nossos vícios. Por nosso Senhor \emph{\&c.}
}\end{paracol}

\paragraphinfo{Epístola}{Nm. 20, 1, 3 et 6-13}
\begin{paracol}{2}\latim{
Léctio libri Numeri.
}\switchcolumn\portugues{
Lição do Livro dos Números.
}\switchcolumn*\latim{
\rlettrine{I}{n} diébus illis: Convenérunt fílii Israël adversum Móysen et Aaron: et versi in seditiónem, dixérunt: Date nobis aquam, ut bibámus. Ingressúsque Móyses et Aaron, dimíssa multitúdine, tabernáculum fǿderis, corruérunt proni in terram, clamaverúntque ad Dóminum, atque dixérunt: Dómine Deus, audi clamórem hujus pópuli, et áperi eis thesáurum tuum, fontem aquæ vivæ, ut, satiáti, cesset murmurátio eórum. Et appáruit glória Dómini super eos. Locutúsque est Dóminus ad Móysen, dicens: Tolle virgam, et cóngrega pópulum, tu et Aaron frater tuus, et loquímini ad petram coram eis, et illa dabit aquas. Cumque edúxeris aquam de petra, bibet omnis multitúdo et juménta ejus. Tulit ígitur Móyses virgam, quæ erat in conspéctu Dómini, sicut præcéperat ei, congregáta multitúdine ante petram, dixítque eis: Audíte, rebélles et incréduli: Num de petra hac vobis aquam potérimus ejícere? Cumque elevásset Móyses manum, percútiens virga bis sílicem, egréssæ sunt aquæ largíssimæ, ita ut pópulus bíberet, et juménta. Dixítque Dóminus ad Móysen et Aaron: Quia non credidístis mihi, ut sanctificarétis me coram fíliis Israël, non introducétis hos pópulos in terram, quam dabo eis. Hæc est aqua contradictiónis, ubi jurgáti sunt fílii Israël contra Dóminum, et sanctificátus est in eis.
}\switchcolumn\portugues{
\rlettrine{N}{aqueles} dias, os filhos de Israel reuniram-se contra Moisés e Aarão, e, havendo formado uma sedição, disseram: «Dai-nos água para bebermos». Então Moisés e Aarão, deixando a multidão, entraram no tabernáculo da Aliança, e, prostrados com o rosto em terra, clamaram ao Senhor e disseram: «Senhor, Deus, ouvi o clamor deste povo e abri-lhe o vosso tesouro, concedendo-lhe uma fonte de água viva, para que, sendo saciado, cesse a murmuração». Então a glória do Senhor apareceu sobre eles. E falou o Senhor a Moisés, dizendo: «Toma a vara, reúne o povo, tu e teu irmão Aarão, impõe o teu poder à rocha diante deles e dela brotará água. Quando isto acontecer, beberá toda a multidão e os seus animais». Pegou, pois, Moisés na vara, que estava diante do Senhor, como lhe havia sido ordenado, reuniu a multidão diante da rocha e disse-lhes: «Ouvi, rebeldes e incrédulos: poderemos nós fazer sair água deste rochedo, para vos dar a beber?». E Moisés levantou a mão e bateu duas vezes com a vara no rochedo, saindo dele água abundante; de tal sorte que bebeu todo o povo, assim como os seus animais. E disse o Senhor a Moisés e a Aarão: «Porque me não acreditastes, para manifestardes a minha santidade diante dos filhos de Israel, não tereis a dita de fazer entrar este povo na terra que lhes darei. Esta é a água da contradição: quando os filhos de Israel murmuraram contra o Senhor, e quando o Senhor manifestou a sua santidade diante deles».
}\end{paracol}

\paragraphinfo{Gradual}{Sl. 27, 7 \& 1}
\begin{paracol}{2}\latim{
\rlettrine{I}{n} Deo sperávit cor meum, et adjútus sum: et reflóruit caro mea, et ex voluntáte mea confitébor illi. ℣. Ad te, Dómine, clamávi: Deus meus, ne síleas, ne discédas a me.
}\switchcolumn\portugues{
\rlettrine{O}{} meu coração esperou em Deus e foi socorrido. Então a minha carne rejuvenesceu. Por isso hei-de celebrar com alegria os louvores do Senhor. ℣. A Vós, Senhor, clamei: meu Deus, não fecheis os ouvidos, nem Vos afasteis de mim.
}\end{paracol}

\paragraphinfo{Trato}{Página \pageref{tratoquartacinzas}}

\paragraphinfo{Evangelho}{Jo. 4, 5-42}
\begin{paracol}{2}\latim{
\cruz Sequéntia sancti Evangélii secúndum Joánnem.
}\switchcolumn\portugues{
\cruz Continuação do santo Evangelho segundo S. João.
}\switchcolumn*\latim{
\blettrine{I}{n} illo témpore: Venit Jesus in civitátem Samaríæ, quæ dícitur Sichar: juxta prǽdium, quod dedit Jacob Joseph, fílio suo. Erat autem ibi fons Jacob. Jesus ergo fatigátus ex itínere, sedébat sic supra fontem. Hora erat quasi sexta. Venit múlier de Samaría hauríre aquam. Dicit ei Jesus: Da mihi bíbere. (Discípuli enim ejus abíerant in civitátem, ut cibos émerent.) Dicit ergo ei múlier illa Samaritána: Quómodo tu, Judǽus cum sis, bíbere a me poscis, quæ sum múlier Samaritána? non enim coutúntur Judǽi Samaritánis. Respóndit Jesus et dixit ei: Si scires donum Dei, et quis est, qui dicit tibi: Da mihi bibere: tu fórsitan petísses ab eo, et dedísset tibi aquam vivam. Dicit ei múlier: Dómine, neque in quo háurias habes, et púteus alius est: unde ergo habes aquam vivam? Numquid tu major es patre nostro Jacob, qui dedit nobis púteum, et ipse ex eo bibit et fílii ejus et pécora ejus? Respóndit Jesus et dixit ei: Omnis, qui bibit ex aqua hac, sítiet íterum: qui autem bíberit ex aqua, quam ego dabo ei, non sítiet in ætérnum: sed aqua, quam ego dabo ei, fiet in eo fons aquæ saliéntis in vitam ætérnam. Dicit ad eum mulier: Dómine, da mihi hanc aquam, ut non sítiam neque véniam huc hauríre. Dicit ei Jesus: Vade, voca virum tuum, et veni huc. Respóndit múlier, et dixit: Non hábeo virum. Dicit ei Jesus: Bene dixísti, quia non hábeo virum: quinque enim viros habuísti, et nunc, quem habes, non est tuus vir: hoc vere dixísti. Dicit ei múlier: Dómine, vídeo, quia Prophéta es tu. Patres nostri in monte hoc adoravérunt, et vos dícitis, quia Jerosólymis est locus, ubi adoráre opórtet. Dicit ei Jesus: Múlier, crede mihi, quia venit hora, quando neque in monte hoc, neque in Jerosólymis adorábitis Patrem. Vos adorátis, quod nescítis: nos adorámus, quod scimus, quia salus ex Judǽis est. Sed venit hora, et nunc est, quando veri adoratóres adorábunt Patrem in spíritu et veritáte. Nam et Pater tales quærit, qui adórent eum. Spíritus est Deus: et eos, qui adórant eum, in spíritu et veritáte opórtet adoráre. Dicit ei mulier: Scio, quia Messías venit (qui dícitur Christus). Cum ergo vénerit ille, nobis annuntiábit ómnia. Dicit ei Jesus: Ego sum, qui loquor tecum. Et contínuo venérunt discípuli ejus: et mirabántur, quia cum mulíere loquebátur. Nemo tamen dixit: Quid quæris, aut quid loquéris cum ea? Reliquit ergo hýdriam suam múlier, et ábiit in civitátem, et dicit illis homínibus: Veníte, et vidéte hóminem, qui dixit mihi ómnia, quæcúmque feci: numquid ipse est Christus? Exiérunt ergo de civitáte, et veniébant ad eum. Intérea rogábant eum discípuli, dicéntes: Rabbi, mandúca. Ille autem dicit eis: Ego cibum habeo manducáre, quem vos nescítis. Dicébant ergo discípuli ad ínvicem: Numquid áliquis áttulit ci manducáre? Dicit eis Jesus: Meus cibus est, ut fáciam voluntátem ejus, qui misit me, ut perfíciam opus ejus. Nonne vos dícitis, quod adhuc quátuor menses sunt, et messis venit? Ecce, dico vobis: Leváte óculos vestros, et vidéte regiónes, quia albæ sunt jam ad messem. Et qui metit, mercédem áccipit, et cóngregat fructum in vitam ætérnam: ut, et qui séminat, simul gáudeat, et qui metit. In hoc enim est verbum verum: quia álius est qui séminat, et álius est qui metit. Ego misi vos métere quod vos non laborástis: alii laboravérunt, et vos in labóres eórum introístis. Ex civitáte autem illa multi credidérunt in eum Samaritanórum, propter verbum mulíeris testimónium perhibéntis: Quia dixit mihi ómnia, quæcúmque feci. Cum veníssent ergo ad illum Samaritáni, rogavérunt eum, ut ibi manéret. Et mansit ibi duos dies. Et multo plures credidérunt in eum propter sermónem ejus. Et mulíeri dicébant: Quia jam non propter tuam loquélam crédimus: ipsi enim audívimus, et scimus, quia hic est vere Salvátor mundi.
}\switchcolumn\portugues{
\blettrine{N}{aquele} tempo, veio Jesus a uma cidade de Samaria, chamada Sichar, próximo do campo que Jacob havia dado a seu filho José. Era ah a fonte de Jacob. Jesus, estando fatigado da viagem, sentou-se à beira do poço. Era cerca da hora sexta do dia. Veio, então, uma mulher de Samaria buscar água. Jesus disse-lhe: «Dá-me de beber», pois os seus discípulos tinham ido à cidade comprar víveres. A mulher samaritana disse-Lhe: «Como é que Vós, que sois judeu, me pedis de beber, sendo eu samaritana?» (Porquanto os judeus não se correspondiam com os samaritanos). Respondeu-lhe Jesus: «Se tu conhecesses o dom de Deus e quem é Aquele que te diz: dá-me de beber, certamente tu lhe pedirias de beber, e ele te daria a água viva!». Disse-Lhe a mulher: «Senhor, não tendes com que tirar a água e o poço é fundo; onde, pois, tendes a água viva? Sois porventura maior que o nosso pai Jacob, que nos deu este poço e bebeu dele, assim como seus filhos e gados?». Jesus respondeu: «Todo aquele que beber desta água tornará a ter sede; porém, aquele que beber da água, que eu lhe der, nunca mais terá sede; pois a água, que eu der, tornar-se-á em uma fonte de água impetuosa para a vida eterna». Disse-Lhe a mulher: «Senhor, dai-me dessa água, para que nunca mais tenha sede, nem volte aqui a buscá-la». Disse-lhe Jesus: «Vai, chama teu marido e volta». Respondeu-Lhe a mulher: «Não tenho marido». Disse-lhe Jesus: «Bem disseste que não tens marido; porque cinco maridos tiveste, e o que agora tens não é teu. Falando assim disseste a verdade». Respondeu-Lhe a mulher: «Senhor, vejo que sois Profeta. Nossos pais adoraram neste monte e Vós dizeis que Jerusalém é o lugar em que se deve adorar?». Disse-lhe Jesus: «Mulher, acredita-me: vai chegar a hora em que nem neste monte, nem em Jerusalém adorareis o Pai. Vós adorais o que não conheceis; nós adoramos o que conhecemos, pois a salvação vem dos judeus. Mas chega a hora, que é esta, em que os verdadeiros adoradores adorarão o Pai, em espírito e em verdade, pois é a tais adoradores que o Pai procura. Deus é espírito; aqueles, pois, que O adoram devem adorá-l’O em espírito e em verdade». Disse-Lhe a mulher: «Sei que o Messias virá (o qual se chama Cristo). Quando Ele vier, nos anunciará todas as cousas». Disse-lhe Jesus: «Eu, que contigo falo, sou o Messias!». Nisto vieram os seus discípulos, admirando-se de que estivesse a falar com uma mulher. Nenhum, contudo, lhe disse: «Que quereis Vós? Ou que falais com ela?». Deixou, então, a mulher o seu cântaro, foi à cidade e disse aos homens: «Vinde ver um homem que me disse todas as cousas que tenho feito. Não será ele o Cristo?». Saíram os homens da cidade e vieram ter com Ele. Entretanto, os seus discípulos suplicavam-Lhe, dizendo: «Mestre, comei». Mas Ele disse-lhes: «Tenho uma comida para tomar que não conheceis». Os discípulos diziam uns aos outros: «Acaso alguém Lhe trouxe comida?». Jesus disse: «Minha comida é fazer a vontade d’Aquele que me enviou, para que se cumpra a sua vontade. Não dizeis que ainda faltam quatro meses até que venha a ceifa? Pois eu vos digo: levantai os olhos e vede que os campos já estão brancos para a ceifa! Aquele que ceifa recebe uma recompensa e guarda fruto para a vida eterna, para que aquele que semeia se regozije também com aquele que ceifa. Aqui se confirma o ditado: um semeia e o outro ceifa. Eu vos enviei a ceifar o que não cultivastes; outros trabalharam e vós entrastes nos seus trabalhos». Ora, muitos samaritanos daquela cidade acreditaram em Jesus por causa da mulher que lhes dera este testemunho: «Ele disse-me tudo quanto tenho feito». Os samaritanos vieram, pois, ter com Ele e pediram-Lhe que ficasse com eles; tendo permanecido ali dous dias. E houve um grande número que acreditou, por haver escutado a sua palavra. E diziam à mulher: «Agora, já não acreditamos por causa do que nos dissestes, porquanto nós próprios ouvimos; e sabemos que este é verdadeiramente o Salvador do mundo».
}\end{paracol}

\paragraphinfo{Ofertório}{Sl. 5, 3-4}
\begin{paracol}{2}\latim{
\rlettrine{I}{nténde} voci oratiónis meæ, Rex meus, et Deus meus: quóniam ad te orábo, Dómine.
}\switchcolumn\portugues{
\rlettrine{S}{ede} atento à voz da minha oração, ó meu Rei e meu Deus; pois a Vós, Senhor, orarei.
}\end{paracol}

\paragraph{Secreta}
\begin{paracol}{2}\latim{
\rlettrine{R}{éspice,} quǽsumus, Dómine, propítius ad múnera, quæ sacrámus: ut tibi grata sint, et nobis salutária semper exsístant. Per Dóminum nostrum \emph{\&c.}
}\switchcolumn\portugues{
\rlettrine{O}{lhai} propício, Senhor, Vos suplicamos, para estes dons que Vos consagramos, a fim de que Vos sejam agradáveis, e a nós sejam sempre salutares. Por nosso Senhor \emph{\&c.}
}\end{paracol}

\paragraphinfo{Comúnio}{Jo. 4, 13 \& 14}
\begin{paracol}{2}\latim{
\qlettrine{Q}{ui} bíberit aquam, quam ego dabo ei, dicit Dóminus, fiet in eo fons aquæ saliéntis in vitam ætérnam.
}\switchcolumn\portugues{
\rlettrine{A}{quele} que beber da água que Eu lhe der, diz o Senhor, terá em si uma fonte de água impetuosa para a vida eterna.
}\end{paracol}

\paragraph{Postcomúnio}
\begin{paracol}{2}\latim{
\rlettrine{H}{ujus} nos, Dómine, percéptio sacraménti mundet a crimine: et ad cœléstia regna perdúcat. Per Dóminum \emph{\&c.}
}\switchcolumn\portugues{
\qlettrine{Q}{ue} a recepção deste Sacramento, Senhor, nos limpe dos nossos crimes e nos guie até ao reino celestial. Por nosso Senhor \emph{\&c.}
}\end{paracol}

\paragraph{Oração sobre o povo}
\begin{paracol}{2}\latim{
\begin{nscenter} Orémus. \end{nscenter}
}\switchcolumn\portugues{
\begin{nscenter} Oremos. \end{nscenter}
}\switchcolumn*\latim{
Humiliáte cápita vestra Deo.
}\switchcolumn\portugues{
Inclinai as vossas cabeças diante de Deus.
}\switchcolumn*\latim{
Præsta, quǽsumus, omnípotens Deus: ut, qui in tua protectióne confídimus, cuncta nobis adversántia, te adjuvánte, vincámus. Per Dóminum nostrum \emph{\&c.}
}\switchcolumn\portugues{
Ó Deus omnipotente, Vos imploramos, dignai-Vos corroborar a confiança que em Vós depositamos, a fim de que com vosso socorro possamos vencer todas as adversidades. Por nosso Senhor \emph{\&c.}
}\end{paracol}

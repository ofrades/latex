{\redx C.}\subsectioninfo{Sexta-Feira Santa}{Estação em Santa Cruz de Jerusalém}

\paragraphinfo{1.ª Lição}{Os. 6, 1-6}
\begin{paracol}{2}\latim{
\rlettrine{H}{æc} dicit Dóminus: In tribulatione sua mane consúrgent ad me: Veníte, et revertámur ad Dóminum: quia ipse cepit, et sanábit nos: percútiet, et curábit nos. Vivificábit nos post duos dies: in die tértia suscitábit nos, et vivémus in conspéctu ejus. Sciémus, sequemúrque, ut cognoscámus Dóminum: quasi dilúculum præparátus est egréssus ejus, et véniet quasi imber nobis temporáneus et serótinus terræ. Quid fáciam tibi, Ephraim? Quid fáciam tibi, Juda? misericórdia vestra quasi nubes matutína: et quasi ros mane pertránsiens. Propter hoc dolávi in prophétis, occídi eos in verbis oris mei: et judícia tua quasi lux egrediéntur. Quia misericórdiam vólui, et non sacrifícium, et sciéntiam Dei, plus quam holocáusta.
}\switchcolumn\portugues{
\rlettrine{E}{is} o que disse o Senhor: «No meio da sua tribulação terão pressa de recorrer a mim. Vinde, dirão eles, convertamo-nos ao Senhor, pois Ele mesmo nos feriu e nos curará. Ele nos castigará e nos sarará. Em dois dias nos dará a vida; ao terceiro dia nos ressuscitará e viveremos na sua presença. Então conheceremos o Senhor e segui-l’O-emos, a fim de O conhecermos melhor. Seu despertar será como o da aurora; virá como a chuva do Outono, que rega a terra. Que posso eu fazer-te, ó Efraim? Que posso eu fazer-te, ó Judá? vossa misericórdia é como a nuvem da manhã; é como o orvalho, que se evapora. Por isso te fiz sofrer pelos Profetas; matei-os com palavras, saídas da minha boca; e o teu julgamento brilhará, como a luz. É a misericórdia que eu quero. Prefiro o conhecimento de Deus a todos os holocaustos que me ofereçais».
}\end{paracol}

\paragraphinfo{Trato}{Hab. 3}
\begin{paracol}{2}\latim{
\rlettrine{D}{ómine,} audívi audítum tuum, et tímui: considerávi ópera tua, et expávi. ℣. In médio duórum animálium innotescéris: dum appropinquáverint anni, sognoscéris: dum advénerit tempus, osténdens. ℣. In eo, dum conturbáta fúerit ánima mea: in ira, misericórdiæ memor eris. ℣. Deus a Líbano véniet, et Sanctus de monte umbróso et condénso. ℣. Opéruit cœlos majéstas ejus: et laudis ejus plena est terra.
}\switchcolumn\portugues{
\rlettrine{S}{enhor,} ouvi a vossa voz e fiquei cheio de temor; meditei nas vossas maravilhas e fiquei extasiado. ℣. Manifestar-Vos-eis entre dois animais: quando os anos tiverem passado e quando os tempos tiverem vindo, manifestar-Vos-eis outra vez. ℣. Então a minha alma perturbar-se-á; mas Vos recordareis da vossa misericórdia para com ela, no dia da vossa ira. ℣. Deus virá do Líbano: e Aquele que é Santo descerá da montanha sombria e arborizada. ℣. Sua majestade cobrirá os céus; e a sua glória e o seu louvor encherão a terra.
}\end{paracol}

\paragraph{Oremos}
\begin{paracol}{2}\latim{
℣. Flectámus génua!
}\switchcolumn\portugues{
℣. Ajoelhemos!
}\switchcolumn*\latim{
℟. Leváte.
}\switchcolumn\portugues{
℟. Levantai-vos!
}\switchcolumn*\latim{
\rlettrine{D}{eus,} a quo et Judas reátus sui pœnam, et confessiónis suæ latro prǽmium sumpsit, concéde nobis tuæ propitiatiónis efféctum: ut, sicut in passióne sua Jesus Christus, Dóminus noster, divérsa utrísque íntulit stipéndia meritórum; ita nobis, abláto vetustátis erróre, resurrectiónis suæ grátiam largiátur: Qui tecum.\emph{\&c.}
}\switchcolumn\portugues{
\slettrine{Ó} Deus, de quem Judas recebeu o castigo da sua perfídia e o ladrão a recompensa da sua confissão, concedei-nos o efeito da vossa misericórdia, a fim de que, assim como N. S. Jesus Cristo durante a sua Paixão tratou a um e ao outro segundo os seus méritos, assim também, havendo desaparecido a nossa malícia do «homem velho», nos tornemos participantes da sua ressurreição. Ele, que, sendo Deus \emph{\&c.}
}\end{paracol}

\paragraphinfo{2.ª Lição}{Ex. 12, 1-11}
\begin{paracol}{2}\latim{
\rlettrine{I}{n} diébus illis: Dixit Dóminus ad Móysen et Aaron in terra Ægýpti: Mensis iste vobis princípium ménsium primus erit in ménsibus anni Loquímini ad univérsum cœtum filiórum Israël, et dícite eis: Décima die mensis hujus tollat unusquísque agnum per famílias et domos suas. Sin autem minor est númerus, ut suffícere possit ad vescéndum agnum, assúmet vicínum suum, qui junctus est dómui suæ, juxta númerum animárum, quæ suffícere possunt ad esum agni. Erit autem agnus absque mácula, másculus, annículus: juxta quem ritum tollétis et hædum. Et servábitis eum usque ad quartam décimam diem mensis hujus: immolabítque eum univérsa multitúdo filiórum Israël ad vésperam. Et sument de sánguine ejus, ac ponent super utrúmque postem et in superlimináribus domórum, in quibus cómedent illum. Et edent carnes nocte illa assas igni, et ázymos panes cum lactúcis agréstibus. Non comedétis ex eo crudum quid nec coctum aqua, sed tantum assum igni: caput cum pédibus ejus et intestínis vorábitis. Nec remanébit quidquam ex eo usque mane. Si quid resíduum fúerit, igne comburétis. Sic autem comedétis illum: Renes vestros accingétis, et calceaménta habébitis in pédibus, tenéntes báculos in mánibus, et comedétis festinánter: est enim Phase (id est tránsitus) Dómini.
}\switchcolumn\portugues{
\rlettrine{N}{aqueles} dias, disse o Senhor, na terra do Egipto, a Moisés e a Aarão: «Que este mês seja para vós o princípio dos meses: o primeiro dos meses do ano. Falai a toda a assembleia dos filhos de Israel, dizendo: «No décimo dia deste mês cada um tome um cordeiro para cada família e para cada casa. Se na casa houverem poucas pessoas para comer o cordeiro, chamar-se-ão em casa do vizinho, que estiver mais perto, tantas pessoas quantas sejam necessárias para comer o cordeiro inteiramente. Esse cordeiro será sem mancha, masculino e com um ano de idade; se porventura faltar o cordeiro, podereis tomar um cabrito com iguais condições. Guardareis esse cordeiro até ao dia décimo quarto desse mês, imolando-o, então, pela tarde, toda a multidão dos filhos de Israel. Tomar-se-á o seu sangue, com o qual pintarão as ombreiras e alizares das portas das casas em que o cordeiro for comido. Nessa mesma noite comerão com pão sem fermento e leitugas silvestres a carne, a qual será assada no lume. Não comereis desse cordeiro nada que seja cru ou cozido em água; mas todo será assado no lume. Comereis a cabeça, os pés e os intestinos, e nada deverá ficar para o dia seguinte; porém, se alguma coisa ficar, tereis o cuidado de consumi-la no fogo. Haveis de comê-lo desta maneira: rins cingidos, pés calçados e bordão na mão. Comê-lo-eis com pressa, pois é a ocasião da páscoa, isto é, a passagem do Senhor».
}\end{paracol}

\paragraphinfo{Trato}{Sl. 139, 2-10 \& 14}
\begin{paracol}{2}\latim{
\rlettrine{E}{ripe} me, Dómine, ab homine malo: a viro iníquo líbera me. ℣. Qui cogitavérunt malítias in corde: tota die constituébant prœlia. ℣. Acuérunt linguas suas sicut serpéntis: venénum áspidum sub labiis eórum. ℣. Custódi me, Dómine, de manu peccatóris: et ab homínibus iníquis libera me. ℣. Qui cogitavérunt supplantáre gressus meos: abscondérunt supérbi láqueum mihi. ℣. Et funes extendérunt in láqueum pédibus meis: juxta iter scándalum posuérunt mihi. ℣. Dixi Dómino: Deus meus es tu: exáudi, Dómine, vocem oratiónis meæ. ℣. Dómine, Dómine, virtus salútis meæ: obúmbra caput meum in die belli. ℣. Ne tradas me a desidério meo peccatóri: cogitavérunt advérsus me: ne derelínquas me, ne umquam exalténtur. ℣. Caput circúitus eórum: labor labiórum ipsórum opériet eos. ℣. Verúmtamen justi confitebúntur nómini tuo: et habitábunt recti cum vultu tuo.
}\switchcolumn\portugues{
\rlettrine{L}{ivrai-me,} Senhor, do homem iníquo; livrai-me do homem injusto. ℣. No seu coração intentam desígnios iníquos; continuamente estão prontos para me combater. ℣. Afiaram as suas línguas, como as das serpentes; nos seus lábios está a peçonha das víboras. ℣. Defendei-me, Senhor, dos ataques das mãos do pecador; livrai-me dos homens injustos. ℣. Porquanto procuram o meio de lançar-me por terra; estes orgulhosos armaram-me ciladas. ℣. Armaram laços para me prender; prepararam ciladas e embustes no meu caminho. ℣. Eu disse ao Senhor: Sois o meu Deus; ouvi, Senhor, a voz da minha súplica. ℣. Senhor, Senhor, sois o meu sustentáculo e a minha salvação: no dia do combate abrigai a minha cabeça. ℣. Não me entregueis, Senhor, à fúria dos pecadores; não me deixeis à mercê dos seus desejos, para que não triunfem contra mim. ℣. Seus embustes cairão sobre si; as suas calúnias contra si se voltarão. ℣. Os justos, Senhor, louvarão o vosso nome: e os que possuem o coração recto contemplar-Vos-ão face a face.
}\end{paracol}

\paragraphinfo{Narração da Paixão}{Jo. 18, 1-40; 19, 1-42}
\begin{paracol}{2}\latim{
\cruz Pássio Dómini nostri Jesu Christi secúndum Joánnem
}\switchcolumn\portugues{
\cruz Paixão de N. S. Jesus Cristo, segundo S. João.
}\switchcolumn*\latim{
\blettrine{I}{n} illo témpore: Egréssus est Jesus cum discípulis suis trans torréntem Cedron, ubi erat hortus, in quem introívit ipse et discípuli ejus. Sciébat autem et Judas, qui tradébat eum, locum: quia frequénter Jesus convénerat illuc cum discípulis suis. Judas ergo cum accepísset cohórtem, et a pontifícibus et pharisǽis minístros, venit illuc cum latérnis et fácibus et armis. Jesus ítaque sciens ómnia, quæ ventúra erant super eum, procéssit, et dixit eis: \cruz Quem quǽritis? {\redx C.} Respondérunt ei: {\redx S.} Jesum Nazarénum. {\redx C.} Dicit eis Jesus: \cruz Ego sum. {\redx C.} Stabat autem et Judas, qui tradébat eum, cum ipsis. Ut ergo dixit eis: Ego sum: abiérunt retrorsum, et cecidérunt in terram. Iterum ergo interrogávit eos: \cruz Quem quǽritis? {\redx C.} Illi autem dixérunt: {\redx S.} Jesum Nazarénum. {\redx C.} Respóndit Jesus: \cruz Dixi vobis, quia ego sum: si ergo me quǽritis, sinite hos abíre. {\redx C.} Ut implerétur sermo, quem dixit: Quia quos dedísti mihi, non pérdidi ex eis quemquam. Simon ergo Petrus habens gládium edúxit eum: et percússit pontíficis servum: et abscídit aurículam ejus déxteram. Erat autem nomen servo Malchus. Dixit ergo Jesus Petro: \cruz Mitte gládium tuum in vagínam. Cálicem, quem dedit mihi Pater, non bibam illum?
}\switchcolumn\portugues{
\blettrine{N}{aquele} tempo, passou Jesus com os discípulos para o outro lado da corrente do Cédron, onde havia ’um jardim, e ali entrou com eles. Judas, que o traía, conhecia também este lugar, pois Jesus vinha ali frequentemente com os discípulos. Então Judas, pondo-se à frente da coorte e dos servos, que os pontífices e os fariseus lhe haviam fornecido, veio ali com lanternas, archotes e armas. Ora, sabendo Jesus o que ia acontecer, foi ao seu encontro e disse: \cruz «A quem procurais?». {\redx C.} E responderam-Lhe: {\redx S.} «A Jesus Nazareno!». {\redx C.} Disse-lhes Jesus: \cruz «Sou Eu!». {\redx C.} Judas, que o traía, estava também com eles. Apenas, pois, Jesus lhes disse «Sou eu», retrocederam e caíram por terra. Perguntou-lhes então Jesus pela segunda vez: \cruz «A quem procurais?». {\redx C.} Eles responderam: {\redx S.} «A Jesus Nazareno!». {\redx C.} Respondeu-lhes Jesus: \cruz «Já vos disse que sou Eu; se, pois, me buscais só a mim, deixai ir estes». {\redx C.} Disse isto para ser cumprida a palavra que havia proferido: «Não perdi nenhum dos que me destes». Então Simão-Pedro, que tinha uma espada, desembainhou-a, acutilou um servo do pontífice e cortou-lhe a orelha direita. Chamava-se Malco. Mas Jesus disse a Pedro: \cruz «Mete a espada na bainha. Porventura não hei-de beber o cálice que deu meu Pai?».
}\switchcolumn*\latim{
{\redx C.} Cohors ergo et tribúnus et minístri Judæórum comprehendérunt Jesum, et ligavérunt eum: et adduxérunt eum ad Annam primum, erat enim socer Cáiphæ, qui erat póntifex anni illíus. Erat autem Cáiphas, qui consílium déderat Judǽis: Quia expédit, unum hóminem mori pro pópulo. Sequebátur autem Jesum Simon Petrus et álius discípulus. Discípulus autem ille erat notus pontífici, et introívit cum Jesu in átrium pontíficis. Petrus autem stabat ad óstium foris. Exívit ergo discípulus álius, qui erat notus pontífici, et dixit ostiáriæ: et introdúxit Petrum. Dicit ergo Petro ancílla ostiária: {\redx S.} Numquid et tu ex discípulis es hóminis istíus? {\redx C.} Dicit ille: {\redx S.} Non sum. {\redx C.} Stabant autem servi et minístri ad prunas, quia frigus erat, et calefaciébant se: erat autem cum eis et Petrus stans et calefáciens se. Póntifex ergo interrogávit Jesum de discípulis suis et de doctrína ejus. Respóndit ei Jesus: \cruz Ego palam locútus sum mundo: ego semper dócui in synagóga et in templo, quo omnes Judǽi convéniunt: et in occúlto locútus sum nihil. Quid me intérrogas? intérroga eos, qui audiérunt, quid locútus sim ipsis: ecce, hi sciunt, quæ díxerim ego. {\redx C.} Hæc autem cum dixísset, unus assístens ministrórum dedit álapam Jesu, dicens: {\redx S.} Sic respóndes pontífici? {\redx C.} Respóndit ei Jesus: \cruz Si male locútus sum, testimónium pérhibe de malo: si autem bene, quid me cædis?
}\switchcolumn\portugues{
{\redx C.} Então a coorte, o tribuno e os satélites dos judeus prenderam e amarraram Jesus. Depois conduziram-n’O à presença de Anás, que era sogro de Caifás e pontífice naquele ano. Fora Caifás quem dera este conselho aos judeus: «Convém mais que morra um só homem, do que todo o povo!». Entretanto, Simão-Pedro seguia Jesus com outro discípulo, o qual, sendo conhecido do pontífice, saiu, falou à porteira e fez entrar Pedro. Ao vê-lo, disse-lhe a porteira: {\redx S.} «Não és tu, também, pertencente aos discípulos deste homem?». {\redx C.} Respondeu Pedro: {\redx S.} «Não sou». {\redx C.} Os servos e os satélites estavam em torno do lume a aquecer-se, pois estava frio. Pedro estava também com eles, de pé, e se aquecia. Entretanto fez o pontífice perguntas a Jesus sobre os seus discípulos e sobre a sua doutrina. Respondeu-lhe Jesus: \cruz «Eu falei sempre ao mundo às claras! Ensinei na sinagoga e no templo, onde se reuniam todos os judeus, e nada ensinei ocultamente. Porque me interrogas, pois? Pergunta antes àqueles que ouviram o que ensinei. Eles aí estão, e muito bem sabem o que lhes disse». {\redx C.} Enquanto Jesus dizia isto, um dos guardas que lá estava deu-Lhe uma bofetada, dizendo: {\redx S.} «Assim respondeis ao pontífice?». {\redx C.} Jesus disse-lhe: \cruz «Se Eu falei mal, aponta-me o mal que disse. Se, porém, falei bem, porque me bates?».
}\switchcolumn*\latim{
{\redx C.} Et misit eum Annas ligátum ad Cáipham pontíficem. Erat autem Simon Petrus stans et calefáciens se. Dixérunt ergo ei: {\redx S.} Numquid et tu ex discípulis ejus es? {\redx C.} Negávit ille et dixit: {\redx S.} Non sum. {\redx C.} Dicit ei unus ex servis pontíficis, cognátus ejus, cujus abscídit Petrus aurículam: {\redx S.} Nonne ego te vidi in horto cum illo? {\redx C.} Iterum ergo negávit Petrus: et statim gallus cantávit.
}\switchcolumn\portugues{
{\redx C.} Anás enviou-O, amarrado, a Caifás, que era o pontífice. Simão-Pedro continuava no mesmo lugar, se aquecendo. Disseram-lhe então: {\redx S.} «Porventura não és tu discípulo d’Ele?». {\redx C.} Pedro negou, dizendo: {\redx S.} «Não sou». {\redx C.} Um dos servos do pontífice, parente daquele a quem Pedro cortara a orelha, disse ainda a este: {\redx S.} «Acaso te não vi eu no horto com Ele?». {\redx C.} Outra vez Pedro negou; e, logo, o galo cantou!
}\switchcolumn*\latim{
Addúcunt ergo Jesum a Cáipha in prætórium. Erat autem mane: et ipsi non introiérunt in prætórium, ut non contaminaréntur, sed ut manducárent pascha. Exívit ergo Pilátus ad eos foras et dixit: {\redx S.} Quam accusatiónem affértis advérsus hóminem hunc? {\redx C.} Respondérunt et dixérunt ei: {\redx S.} Si non esset hic malefáctor, non tibi tradidissémus eum. {\redx C.} Dixit ergo eis Pilátus: {\redx S.} Accípite eum vos, et secúndum legem vestram judicáte eum. {\redx C.} Dixérunt ergo ei Judǽi: {\redx S.} Nobis non licet interfícere quemquam. {\redx C.} Ut sermo Jesu implerétur, quem dixit, signíficans, qua morte esset moritúrus. Introívit ergo íterum in prætórium Pilátus, et vocávit Jesum et dixit ei: S, Tu es Rex Judæórum? {\redx C.} Respóndit Jesus: \cruz A temetípso hoc dicis, an álii dixérunt tibi de me? {\redx C.} Respóndit Pilátus: {\redx S.} Numquid ego Judǽus sum? Gens tua et pontífices tradidérunt te mihi: quid fecísti? {\redx C.} Respóndit Jesus: \cruz Regnum meum non est de hoc mundo. Si ex hoc mundo esset regnum meum, minístri mei útique decertárent, ut non tráderer Judǽis: nunc autem regnum meum non est hinc. {\redx C.} Dixit itaque ei Pilátus: {\redx S.} Ergo Rex es tu? {\redx C.} Respóndit Jesus: \cruz Tu dicis, quia Rex sum ego. Ego in hoc natus sum et ad hoc veni in mundum, ut testimónium perhíbeam veritáti: omnis, qui est ex veritáte, audit vocem meam. {\redx C.} Dicit ei Pilátus: {\redx S.} Quid est véritas? {\redx C.} Et cum hoc dixísset, íterum exívit ad Judǽos, et dicit eis: {\redx S.} Ego nullam invénio in eo causam. Est autem consuetúdo vobis, ut unum dimíttam vobis in Pascha: vultis ergo dimíttam vobis Regem Judæórum? {\redx C.} Clamavérunt ergo rursum omnes, dicéntes: {\redx S.} Non hunc, sed Barábbam. {\redx C.} Erat autem Barábbas latro. Tunc ergo apprehéndit Pilátus Jesum et flagellávit. Et mílites plecténtes corónam de spinis, imposuérunt cápiti ejus: et veste purpúrea circumdedérunt eum. Et veniébant ad eum, et dicébant: {\redx S.} Ave, Rex Judæórum. {\redx C.} Et dabant ei álapas. Exívit ergo íterum Pilátus foras et dicit eis: {\redx S.} Ecce, addúco vobis eum foras, ut cognoscátis, quia nullam invénio in eo causam. {\redx C.} (Exívit ergo Jesus portans corónam spíneam et purpúreum vestiméntum.) Et dicit eis: {\redx S.} Ecce homo. {\redx C.} Cum ergo vidíssent eum pontífices et minístri, clamábant, dicéntes: {\redx S.} Crucifíge, crucifíge eum. {\redx C.} Dicit eis Pilátus: {\redx S.} Accípite eum vos et crucifígite: ego enim non invénio in eo causam. {\redx C.} Respondérunt ei Judǽi: {\redx S.} Nos legem habémus, et secúndum legem debet mori, quia Fílium Dei se fecit. {\redx C.} Cum ergo audísset Pilátus hunc sermónem, magis tímuit. Et ingréssus est prætórium íterum: et dixit ad Jesum: {\redx S.} Unde es tu? {\redx C.} Jesus autem respónsum non dedit ei. Dicit ergo ei Pilátus: {\redx S.} Mihi non lóqueris? nescis, quia potestátem hábeo crucifígere te, et potestátem hábeo dimíttere te? {\redx C.} Respóndit Jesus: \cruz Non habéres potestátem advérsum me ullam, nisi tibi datum esset désuper. Proptérea, qui me trádidit tibi, majus peccátum habet. {\redx C.} Et exínde quærébat Pilátus dimíttere eum. Judǽi autem clamábant dicéntes: {\redx S.} Si hunc dimíttis, non es amícus Cǽsaris. Omnis enim, qui se regem facit, contradícit Cǽsari. {\redx C.} Pilátus autem cum audísset hos sermónes, addúxit foras Jesum, et sedit pro tribunáli, in loco, qui dícitur Lithóstrotos, hebráice autem Gábbatha. Erat autem Parascéve Paschæ, hora quasi sexta, et dicit Judǽis: {\redx S.} Ecce Rex vester. {\redx C.} Illi autem clamábant: {\redx S.} Tolle, tolle, crucifíge eum. {\redx C.} Dicit eis Pilátus: {\redx S.} Regem vestrum crucifígam? {\redx C.} Respondérunt pontífices: {\redx S.} Non habémus regem nisi Cǽsarem. {\redx C.} Tunc ergo trádidit eis illum, ut crucifigerétur. Suscepérunt autem Jesum et eduxérunt.
}\switchcolumn\portugues{
Depois disto conduziram Jesus de casa de Caifás para o Pretório. Era já de manhã; e por isso não entraram, a fim de que se não contaminassem e pudessem comer a Páscoa. Saiu, pois, Pilatos fora, a ouvi-los, e disse: {\redx S.} «Que acusação fazeis a este homem?». {\redx C.} Responderam e disseram: {\redx S.} «Se Ele não fosse um malfeitor não to teríamos entregue». {\redx C.} E Pilatos disse-lhes: {\redx S.} «Tomai-O vós e julgai-O, segundo a vossa lei». {\redx C.} Ao que os judeus retorquiram: {\redx S.} «Não nos é permitido condenar ninguém à morte». {\redx C.} Estas palavras foram ditas para que se cumprisse o que Jesus anunciara, indicando de que morte havia de morrer. Entrou, então, Pilatos no Pretório, chamou Jesus e disse-Lhe: {\redx S.} «Sois o Rei dos judeus?». {\redx C.} Jesus respondeu-lhe: \cruz «Dizes isso de ti mesmo, ou foram outros que te disseram isso de mim?». {\redx C.} Pilatos respondeu-Lhe: {\redx S.} «Acaso sou eu judeu? vosso povo e os pontífices entregaram-Vos às minhas mãos. Que mal fizestes?». {\redx C.} Jesus disse: \cruz «Meu reino não é deste mundo. Se o meu reino fosse deste mundo, os meus ministros certamente teriam combatido para que Eu não fosse entregue aos judeus; mas o meu reino não é deste mundo». {\redx C.} Disse-Lhe Pilatos: {\redx S.} «Então sois Rei». {\redx C.} Respondeu Jesus: \cruz «Tu o dizes: Eu sou Rei! Eu nasci e vim a este mundo para dar testemunho da verdade. Todo aquele que procura a verdade escuta a minha voz». {\redx C.} Disse-Lhe, pois, Pilatos: {\redx S.} «Que é a verdade?». {\redx C.} E, dizendo isto, foi novamente falar com os judeus, dizendo-lhes: {\redx S.} «Não acho n’Ele crime algum digno de condenação. Ora, como é costume entre vós dar liberdade a um preso na Páscoa, quereis que solte o Rei dos judeus?». {\redx C.} Então clamaram, novamente, todos: {\redx S.} «Esse, não; mas sim Barrabás». {\redx C.} Barrabás era, porém, um ladrão. Então Pilatos mandou açoitar Jesus, Os soldados, tecendo uma coroa de espinhos, puseram-Lha na cabeça; e vestiram-n’O com um manto de púrpura. Vinham ter com Ele e diziam-Lhe: {\redx S.} «Salve, ó Rei dos judeus!». {\redx C.} Davam-Lhe também bofetadas. Pilatos saiu outra vez para fora e disse-lhes: {\redx S.} «Eis que vo-l’O apresento novamente, para que saibais que não há n’Ele causa de condenação». {\redx C.} Apareceu então Jesus, trazendo a coroa de espinhos e um manto de púrpura. E Pilatos disse: {\redx S.} «Eis aqui o homem!». {\redx C.} Apenas os príncipes dos sacerdotes e os satélites viram Jesus, gritavam e diziam: {\redx S.} «Crucifica-O; crucifica-O!». {\redx C.} E Pilatos respondeu: {\redx S.} «Tomai-O vós e crucificai-O; pois não encontro n’Ele crime algum digno de condenação». {\redx C.} Retorquiram-lhe os judeus: {\redx S.} «Nós temos uma lei e, segundo ela, Jesus deve morrer, porque se diz Filho de Deus». {\redx C.} Quando Pilatos ouviu estas palavras, temeu ainda mais. E, entrando outra vez no Pretório, perguntou a Jesus: {\redx S.} «Donde sois Vós?». {\redx C.} Jesus lhe não respondeu. Pilatos continuou então: {\redx S.} «Não me respondeis? Ignorais que tenho poder para Vos mandar crucificar ou dar liberdade?». {\redx C.} Respondeu-lhe Jesus: \cruz «Nenhum poder teríeis em mim, se vos não fora dado pelo alto; por isso, aquele que me entregou a ti é culpado de maior pecado». {\redx C.} E Pilatos procurava algum meio com que salvasse Jesus; contudo, os judeus clamavam, dizendo: {\redx S.} «Se O soltas não és amigo de César; porquanto, quem se faz rei declara-se contra César», Ouvindo estas palavras, Pilatos conduziu Jesus para fora e sentou-se no tribunal, em um lugar chamado Litóstrotos (que em hebreu significa Gabbata). Era então o dia de Parasceve (Preparação) da Páscoa, e quase a hora sexta. Pilatos disse aos judeus: {\redx S.} «Eis o vosso rei!». {\redx C.} Mas eles clamavam: {\redx S.} «Tira-O; tira-O; crucifica-O!». {\redx C.} E disse-lhes Pilatos: {\redx S.} «Pois hei-de crucificar o vosso rei?». {\redx C.} Os pontífices responderam: {\redx S.} «Não temos outro rei senão César». {\redx C.} Entregou-lhes, pois, finalmente, Jesus, para que fosse crucificado. Então seguraram-n’O e levaram-n’O.
}\switchcolumn*\latim{
Et bájulans sibi Crucem, exívit in eum, qui dícitur Calváriæ, locum, hebráice autem Gólgotha: ubi crucifixérunt eum, et cum eo alios duos, hinc et hinc, médium autem Jesum. Scripsit autem et títulum Pilátus: et pósuit super crucem. Erat autem scriptum: Jesus Nazarénus, Rex Judæórum. Hunc ergo títulum multi Judæórum legérunt, quia prope civitátem erat locus, ubi crucifíxus est Jesus. Et erat scriptum hebráice, græce et latíne. Dicébant ergo Piláto pontífices Judæórum: {\redx S.} Noli scríbere Rex Judæórum, sed quia ipse dixit: Rex sum Judæórum. {\redx C.} Respóndit Pilátus: {\redx S.} Quod scripsi, scripsi. {\redx C.} Mílites ergo cum crucifixíssent eum, acceperunt vestimenta ejus (et fecérunt quátuor partes: unicuique míliti partem), et túnicam. Erat autem túnica inconsútilis, désuper contéxta per totum. Dixérunt ergo ad ínvicem: {\redx S.} Non scindámus eam, sed sortiámur de illa, cujus sit. {\redx C.} Ut Scriptúra implerétur, dicens: Partíti sunt vestiménta mea sibi: et in vestem meam misérunt sortem. Et mílites quidem hæc fecérunt.
}\switchcolumn\portugues{
Puseram-Lhe, pois, uma cruz aos ombros e conduziram-n’O para um lugar, fora da cidade, chamado Calvário (que em hebreu significa Gólgota), onde O crucificaram, e com Ele dois outros, um de cada lado, e no meio Jesus. Pilatos escreveu também uma inscrição, que mandou colocar na parte superior da cruz, a qual dizia: «Jesus Nazareno, Rei dos Judeus». Muitos judeus leram este título, pois o lugar onde Jesus fora crucificado era perto da cidade. O título estava escrito em hebreu, grego e latim. Os pontífices diziam a Pilatos: {\redx S.} «Não escrevas Rei dos judeus; mas sim que Ele dizia: Sou o Rei dos judeus». {\redx C.} Respondeu-lhes Pilatos: {\redx S.} «O que eu escrevi fica escrito». {\redx C.} Entretanto, havendo sido crucificado, tomaram-Lhe os vestidos e dividiram-nos em quatro partes, sendo uma para cada soldado. Quanto à túnica, como era sem costura, toda tecida de alto a baixo, combinaram entre si, dizendo uns aos outros: {\redx S.} «Não a rasguemos, mas deitemos sortes para ver a quem ficará». {\redx C.} Isto aconteceu para que se cumprisse a Escritura, que dizia: «Repartiram entre si os meus vestidos e sobre a minha túnica deitaram sortes». Isto mesmo fizeram os soldados.
}\switchcolumn*\latim{
Stabant autem juxta Crucem Jesu Mater ejus et soror Matris ejus, María Cléophæ, e María Magdaléne. Cum vidísset ergo Jesus Matrem et discípulum stantem, quem diligébat, dicit Matri suæ: \cruz Múlier, ecce fílius tuus. {\redx C.} Deinde dicit discípulo: \cruz Ecce mater tua. {\redx C.} Et ex illa hora accépit eam discípulus in sua. Póstea sciens Jesus, quia ómnia consummáta sunt, ut consummarétur Scriptúra, dixit: \cruz Sítio. {\redx C.} Vas ergo erat pósitum acéto plenum. Illi autem spóngiam plenam acéto, hyssópo circumponéntes, obtulérunt ori ejus. Cum ergo accepísset Jesus acétum, dixit: \cruz Consummátum est. {\redx C.} Et inclináte cápite trádidit spíritum.
}\switchcolumn\portugues{
Estavam, então, de pé, junto à cruz de Jesus, sua Mãe e a irmã de sua Mãe, Maria de Cléofas, e Maria Madalena. Vendo Jesus sua Mãe e perto dela o discípulo que Ele preferia, disse à Mãe: \cruz «Mulher, eis aí o teu Filho!». {\redx C.} Depois disse ao discípulo: \cruz «Eis a tua Mãe!». {\redx C.} Desde aquela hora, o discípulo a tomou a seu cuidado. Depois, sabendo Jesus que tudo estava consumado para se cumprir o que a Escritura anunciara, disse: \cruz «Tenho sede». {\redx C.} Havia ali perto um vaso cheio de vinagre. Foram, pois, os soldados buscá-lo e, embebendo nele uma esponja, ataram-na a um ramo de hissopo e chegaram-Lho à boca. Havendo Jesus tomado o vinagre, disse: \cruz «Tudo está consumado!». {\redx C.} E, inclinando a cabeça, entregou o espírito!
}\switchcolumn*\latim{
\emph{(Hic genuflectitur, et pausatur aliquantulum)}
}\switchcolumn\portugues{
\emph{(Ajoelha-se durante algum tempo, meditando-se no que se leu.)}
}\switchcolumn*\latim{
Judǽi ergo (quóniam Parascéve erat), ut non remanérent in cruce córpora sábbato (erat enim magnus dies ille sábbati), rogavérunt Pilátum, ut frangeréntur eórum crura et tolleréntur. Venérunt ergo mílites: et primi quidem fregérunt crura et alteríus, qui crucifíxus est cum eo. Ad Jesum autem cum veníssent, ut vidérunt eum jam mórtuum, non fregérunt ejus crura, sed unus mílitum láncea latus ejus apéruit, et contínuo exívit sanguis et aqua. Et qui vidit, testimónium perhíbuit: et verum est testimónium ejus. Et ille scit, quia vera dicit: ut et vos credátis. Facta sunt enim hæc, ut Scriptúra implerétur: Os non comminuétis ex eo. Et íterum ália Scriptúra dicit: Vidébunt in quem transfixérunt.
}\switchcolumn\portugues{
Os judeus (porque era o dia da Preparação da Páscoa), não desejando que os corpos ficassem na cruz para o sábado (pois o sábado era solene), pediram a Pilatos consentisse que partissem as pernas aos crucificados e os descessem da cruz. Os soldados vieram e quebraram as pernas dos que haviam sido crucificados com Ele. Mas, tendo vindo a Jesus, como O vissem já morto, Lhe não quebraram as pernas, mas um dos soldados abriu-lhe com a lança o lado, do qual saiu sangue e água. E quem isto viu dá testemunho disso, e o seu testemunho é verdadeiro, pois sabe que diz a verdade, para que lhe deis crédito. Aconteceram estas coisas para se cumprir o que dizia a Escritura: «Não quebrareis nenhum dos meus ossos». Ainda a Escritura diz noutro lugar: «Contemplarão Aquele que traspassaram».
}\end{paracol}

\textit{O Celebrante vai ao meio do Altar e diz o MUNDA COR MEUM... (página 33). Depois prossegue:}

\begin{paracol}{2}\latim{
Post hæc autem rogávit Pilátum Joseph ab Arimathǽa (eo quod esset discípulus Jesu, occúltus autem propter metum Judæórum), ut tólleret corpus Jesu. Et permísit Pilátus. Venit ergo et tulit corpus Jesu. Venit autem et Nicodémus, qui vénerat ad Jesum nocte primum, ferens mixtúram myrrhæ et áloes, quasi libras centum. Accepérunt ergo corpus Jesu, et ligavérunt illud línteis cum aromátibus, sicut mos est Judǽis sepelíre. Erat autem in loco, ubi crucifíxus est, hortus: et in horto monuméntum novum, in quo nondum quisquam pósitus erat. Ibi ergo propter Parascéven Judæórum, quia juxta erat monuméntum, posuérunt Jesum.
}\switchcolumn\portugues{
Em seguida, José de Arimateia (que fora discípulo de Jesus, ocultamente, com medo dos judeus), pediu a Pilatos o corpo de Jesus, o que Pilatos permitiu. Veio, pois, e tirou o corpo de Jesus. Acompanhou-o Nicodemos (que no princípio da noite viera procurar Jesus) com uma mistura de mirra e de aloés, de quase cem libras de preço. Tomaram, então, o corpo de Jesus e envolveram-no em lençóis com aromas, segundo o costume dos judeus. Havia no lugar em que Jesus foi crucificado um jardim, e nele uma sepultura nova, onde ninguém fora ainda depositado. Foi aí (por ser o dia da Preparação da Páscoa dos judeus) que depositaram Jesus, pois este túmulo estava próximo.
}\end{paracol}

\subsection{Missa dos Penitentes}

\paragraph{Pela Santa Igreja}
\begin{paracol}{2}\latim{
\rlettrine{O}{rémus,} dilectíssimi nobis, pro Ecclésia sancta Dei, ut eam Deus et Dóminus noster pacificáre, adunáre et custodíre dignétur
toto orbe terrárum, detque nobis, quiétam et tranquíllam vitam degéntibus, glorificáre Deum Patrem omnipoténtem.
}\switchcolumn\portugues{
\rlettrine{O}{remos,} irmãos caríssimos, pela Santa Igreja de Deus, a fim de que o Senhor, nosso Deus, se digne conceder-lhe paz e união e a guarde em toda a terra, sujeitando-lhe espiritualmente todos os principados e potestades; e que nos conceda uma vida calma e tranquila para glorificarmos Deus Pai omnipotente.
}\switchcolumn*\latim{
\begin{nscenter} Orémus. \end{nscenter}
}\switchcolumn\portugues{
\begin{nscenter} Oremos. \end{nscenter}
}\switchcolumn*\latim{
℣. Flectámus génua.
}\switchcolumn\portugues{
℣. Ajoelhemos!
}\switchcolumn*\latim{
℟. Leváte.
}\switchcolumn\portugues{
℟. Levantai-vos!
}\switchcolumn*\latim{
Omnípotens sempitérne Deus, qui glóriam tuam ómnibus in Christo géntibus revelásti: custódi ópera misericórdiæ tuæ; ut Ecclésia tua, toto orbe
diffúsa, stábili fide in confessióne tui nóminis persevéret. Per eúndem Dóminum nostrum \emph{\&c.} ℟. Amen.
}\switchcolumn\portugues{
Omnipotente e eterno Deus, que revelastes a vossa glória a todas as nações por meio de Cristo, conservai a obra da vossa misericórdia, para que a vossa Igreja, espalhada por todo o mundo, persevere com fé firme na confissão do vosso Nome. Pelo mesmo nosso Senhor \emph{\&c.} ℟. Amen.
}\end{paracol}

\paragraph{Pelo Santíssimo Padre}
\begin{paracol}{2}\latim{
\rlettrine{O}{rémus,} et pro beatíssimo Papa nostro {\redx N.}, ut Deus et Dóminus noster, qui elégit eum in órdine episcopátus, salvum atque incólumem custódiat Ecclésiæ suæ sanctæ, ad regéndum pópulum sanctum Dei.
}\switchcolumn\portugues{
\rlettrine{O}{remos,} pelo Santíssimo Padre, o Papa {\redx N.} para que o Senhor, nosso Deus, que o elevou à ordem do Episcopado, o conserve incólume e livre, para utilidade da sua Igreja e para governar o santo povo de Deus.
}\switchcolumn*\latim{
\begin{nscenter} Orémus. \end{nscenter}
}\switchcolumn\portugues{
\begin{nscenter} Oremos. \end{nscenter}
}\switchcolumn*\latim{
℣. Flectámus génua.
}\switchcolumn\portugues{
℣. Ajoelhemos!
}\switchcolumn*\latim{
℟. Leváte.
}\switchcolumn\portugues{
℟. Levantai-vos!
}\switchcolumn*\latim{
Omnípotens sempitérne Deus, cujus judício univérsa fundántur: réspice propítius ad preces nostras, et electum nobis Antístitem tua pietáte
consérva; ut christiána plebs, quæ te gubernátur auctóre, sub tanto Pontífice, credulitátis suæ méritis augeátur. Per Dóminum nostrum \emph{\&c.} ℟. Amen.
}\switchcolumn\portugues{
Omnipotente e eterno Deus, que pela vossa sabedoria fazeis subsistir todas as coisas, acolhei benigno as nossas súplicas, e pela vossa bondade conservai o Pontífice escolhido, para que o povo cristão, que a vossa autoridade governa, aumente nos méritos da sua fé, debaixo da direcção de tão grande Pontífice. Por nosso Senhor \emph{\&c.} ℟. Amen.
}\end{paracol}

\paragraph{Por todo o Corpo da Igreja}
\begin{paracol}{2}\latim{
\rlettrine{O}{rémus,} et pro ómnibus Epíscopis nostro, Presbýteris, Diacónibus,Subdiacónis, Acólythis, Exorcístis, Lectóribus, Ostiáriis, Confessóribus, Virgínibus, Víduis: et pro omni pópulo sancto Dei.
}\switchcolumn\portugues{
\rlettrine{O}{remos,} também por todos os Bispos, Presbíteros, Diáconos, Subdiáconos, Acólitos, Exorcistas, Leitores, Ostiários, Confessores, Virgens, Viúvas e ainda por todo o santo povo de Deus.
}\switchcolumn*\latim{
\begin{nscenter} Orémus. \end{nscenter}
}\switchcolumn\portugues{
\begin{nscenter} Oremos. \end{nscenter}
}\switchcolumn*\latim{
℣. Flectámus génua.
}\switchcolumn\portugues{
℣. Ajoelhemos!
}\switchcolumn*\latim{
℟. Leváte.
}\switchcolumn\portugues{
℟. Levantai-vos!
}\switchcolumn*\latim{
Deus, cujus Spíritu totum corpus Ecclésiæ sanctificátur et régitur: exáudi nos pro univérsis ordínibus supplicán
tes; ut, grátiæ tuæ múnere, ab ómnibus tibi grádibus fidéliter serviátur. Per Dóminum \emph{\&c.} ℟. Amen.
}\switchcolumn\portugues{
Omnipotente e eterno Deus, cujo Espírito santifica e governa todo o corpo da Igreja, ouvi as nossas súplicas por todas as Ordens, a fim de que pelo dom da vossa graça cada uma dessas jerarquias Vos sirva fielmente. Por nosso Senhor \emph{\&c.} ℟. Amen.
}\end{paracol}

\paragraph{Pelo Imperador Romano}
\begin{paracol}{2}\latim{
\rlettrine{O}{rémus,} et pro Christianíssimo\footnote[2]{Si non est coronatus, dicatur: elécto Imperatóre.} Imperatóre nostro {\redx N.} ut Deus et Dóminus noster súbditas fáciat omnes barbaras natiónes ad nostram perpétuam pacem.
}\switchcolumn\portugues{
\rlettrine{O}{remos,} também pelo nosso Cristianíssimo\footnote[2]{Se não é coroado, diz-se: Imperador eleito.} Imperador {\redx N.} para que o Senhor, nosso Deus, lhe submeta todas as nações bárbaras, para nossa perpétua paz.
}\switchcolumn*\latim{
\begin{nscenter} Orémus. \end{nscenter}
}\switchcolumn\portugues{
\begin{nscenter} Oremos. \end{nscenter}
}\switchcolumn*\latim{
℣. Flectámus génua.
}\switchcolumn\portugues{
℣. Ajoelhemos!
}\switchcolumn*\latim{
℟. Leváte.
}\switchcolumn\portugues{
℟. Levantai-vos!
}\switchcolumn*\latim{
Omnípotens sempitérne Deus, in cujus manu sunt ómnium potestátes et ómnium jura regnórum: réspice ad Románum benígnus Impérium; ut gentes, quæ in sua feritáte confídunt, poténtiæ tuæ déxtera comprimántur. Per Dóminum nostrum \emph{\&c.} ℟. Amen.
}\switchcolumn\portugues{
Omnipotente e eterno Deus, em cujas mãos estão todas as potestades e todas as leis do reino: olhai benignamente para o Império Romano; de modo que as nações que confiam em sua própria força, fiquem sujeitas à sua dextra. Por nosso Senhor \emph{\&c.} ℟. Amen.
}\end{paracol}

\paragraph{Pelos Catecúmenos}
\begin{paracol}{2}\latim{
\rlettrine{O}{rémus,} et pro catechúmenis nostris: ut Deus et Dóminus noster adapériat aures præcordiórum ipsórum januámque misericordiæ; ut, per lavácrum regeneratiónis accépta remissióne ómnium peccatórum, et ipsi inveniántur in Christo Jesu, Dómino nostro.
}\switchcolumn\portugues{
\rlettrine{O}{remos,} também pelos nossos Catecúmenos, para que o Senhor, nosso Deus, lhes abra os ouvidos do coração e as portas da misericórdia, e, assim, havendo alcançado a remissão dos pecados pelo banho da regeneração, sejam connosco incorporados em Jesus Cristo, nosso Senhor.
}\switchcolumn*\latim{
\begin{nscenter} Orémus. \end{nscenter}
}\switchcolumn\portugues{
\begin{nscenter} Oremos. \end{nscenter}
}\switchcolumn*\latim{
℣. Flectámus génua.
}\switchcolumn\portugues{
℣. Ajoelhemos!
}\switchcolumn*\latim{
℟. Leváte.
}\switchcolumn\portugues{
℟. Levantai-vos!
}\switchcolumn*\latim{
Omnípotens sempitérne Deus, qui Ecclésiam tuam nova semper prole fecúndas: auge fidem et intellectum catechúmenis nostris; ut, renáti fonte baptismátis, adoptiónis tuæ fíliis aggregéntur. Per Dóminum \emph{\&c.} ℟. Amen.
}\switchcolumn\portugues{
Omnipotente e eterno Deus, que dais continuamente novos filhos à vossa Igreja, aumentai a fé e a inteligência dos nossos Catecúmenos, a fim de que, renascidos na fonte baptismal, sejam agregados aos vossos filhos de adopção. Por nosso Senhor \emph{\&c.} ℟. Amen.
}\end{paracol}

\paragraph{Pelas Necessidade dos Fiéis}
\begin{paracol}{2}\latim{
\rlettrine{O}{rémus,} dilectíssimi nobis, Deum Patrem omnipoténtem, ut cunctis mundum purget erróribus: morbos áuferat: famem depellat: apériat cárceres: víncula dissólvat: peregrinántibus réditum: infirmántibus sanitátem: navigántibus portum salútis indúlgeat.
}\switchcolumn\portugues{
\rlettrine{O}{remos,} caríssimos irmãos, a Deus Pai omnipotente, pedindo-Lhe que purifique o mundo de todos os erros; afaste as doenças; desterre a fome; abra as prisões; quebre as cadeias; conceda aos viajantes feliz viagem; dê aos enfermos a saúde; e conduza os navegantes a porto de salvamento.
}\switchcolumn*\latim{
\begin{nscenter} Orémus. \end{nscenter}
}\switchcolumn\portugues{
\begin{nscenter} Oremos. \end{nscenter}
}\switchcolumn*\latim{
℣. Flectámus génua.
}\switchcolumn\portugues{
℣. Ajoelhemos!
}\switchcolumn*\latim{
℟. Leváte.
}\switchcolumn\portugues{
℟. Levantai-vos!
}\switchcolumn*\latim{
Omnípotens sempitérne Deus, mæstórum consolátio, laborántium fortitúdo: pervéniant ad te preces de quacúmque tribulatióne clamántium; ut omnes sibi in necessitátibus suis misericórdiam tuam gáudeant affuísse. Per Dóminum nostrum \emph{\&c.} ℟. Amen.
}\switchcolumn\portugues{
Omnipotente e eterno Deus, consolação dos tristes e força dos que trabalham, permiti que cheguem até Vós as súplicas dos que em qualquer tribulação a Vós recorrem, para que nas suas necessidades todos sintam com alegria o auxílio da vossa misericórdia. Por nosso Senhor \emph{\&c.} ℟. Amen.
}\end{paracol}

\paragraph{Pelos Hereges e Cismáticos}
\begin{paracol}{2}\latim{
\rlettrine{O}{rémus,} et pro hæréticis et schismáticis: ut Deus et Dóminus noster éruat eos ab erróribus univérsis; et ad sanctam matrem Ecclésiam Cathólicam atque Apostólicam revocáre dignétur.
}\switchcolumn\portugues{
\rlettrine{O}{remos,} também pelos hereges e cismáticos: para que o Senhor, nosso Deus, os livre de todos os erros e se digne reconduzi-los ao seio da santa mãe Igreja Católica e Apostólica.
}\switchcolumn*\latim{
\begin{nscenter} Orémus. \end{nscenter}
}\switchcolumn\portugues{
\begin{nscenter} Oremos. \end{nscenter}
}\switchcolumn*\latim{
℣. Flectámus génua.
}\switchcolumn\portugues{
℣. Ajoelhemos!
}\switchcolumn*\latim{
℟. Leváte.
}\switchcolumn\portugues{
℟. Levantai-vos!
}\switchcolumn*\latim{
Omnípotens sempitérne Deus, qui salvas omnes, et néminem vis períre: réspice ad ánimas diabólica fraude decéptas; ut, omni hærética pravitáte
depósita, errántium corda resipíscant, et ad veritátis tuæ rédeant unitátem. Per Dóminum nostrum \emph{\&c.} ℟. Amen.
}\switchcolumn\portugues{
Omnipotente e eterno Deus, que quereis salvar todos os homens e não quereis que nenhum pereça, lançai vossos olhares de compaixão para as almas seduzidas pelos artifícios do demónio, a fim de que, abandonando elas toda a maldade, se arrependam dos erros e regressem à unidade da vossa doutrina. Por nosso Senhor \emph{\&c.} ℟. Amen.
}\end{paracol}

\paragraph{Pelos Judeus}
\begin{paracol}{2}\latim{
\rlettrine{O}{rémus,} et pro Judǽis: ut Deus et Dóminus noster illúminet corda eórum; ut agnóscant Jesum Christum sálvatorem Dóminum hóminum.
}\switchcolumn\portugues{
\rlettrine{O}{remos,} também pelos Judeus: para que o Senhor, nosso Deus, ilumine os seus corações; para que eles reconheçam Jesus Cristo, o salvador dos homens.
}\switchcolumn*\latim{
\begin{nscenter} Orémus. \end{nscenter}
}\switchcolumn\portugues{
\begin{nscenter} Oremos. \end{nscenter}
}\switchcolumn*\latim{
℣. Flectámus génua.
}\switchcolumn\portugues{
℣. Ajoelhemos!
}\switchcolumn*\latim{
℟. Leváte.
}\switchcolumn\portugues{
℟. Levantai-vos!
}\switchcolumn*\latim{
Omnípotens sempitérne Deus, qui vis ut omnes hómines sálvi fíant et ad ágnitionem veritátis véniant, concéde propítius, ut plenitúdine géntium in Ecclésiam Tuam íntrante omnis Ísrael sálvus fíat. Per Dóminum nostrum \emph{\&c.} ℟. Amen.
}\switchcolumn\portugues{
Omnipotente e eterno Deus, que desejais que todos os homens se salvem e alcancem o conhecimento da verdade, concedei que, entrando a plenitude dos povos na vossa Igreja, todo Israel seja salvo. Pelo mesmo nosso Senhor \emph{\&c.} ℟. Amen.
}\end{paracol}

\paragraph{Pelos Pagãos}
\begin{paracol}{2}\latim{
\rlettrine{O}{rémus,} et pro pagánis: ut Deus omnípotens áuferat iniquitátem a córdibus eórum; ut, relíctis idólis suis, convertántur ad Deum vivum et verum, et únicum Fílium ejus Jesum Christum, Deum et Dóminum nostrum.
}\switchcolumn\portugues{
\rlettrine{O}{remos,} ainda pelos pagãos, a fim de que Deus omnipotente lhes arranque dos corações a iniquidade, e, abandonando os seus ídolos, se convertam a Deus vivo e verdadeiro e a seu Filho Unigénito Jesus Cristo, nosso Deus e Senhor.
}\switchcolumn*\latim{
\begin{nscenter} Orémus. \end{nscenter}
}\switchcolumn\portugues{
\begin{nscenter} Oremos. \end{nscenter}
}\switchcolumn*\latim{
℣. Flectámus génua.
}\switchcolumn\portugues{
℣. Ajoelhemos!
}\switchcolumn*\latim{
℟. Leváte.
}\switchcolumn\portugues{
℟. Levantai-vos!
}\switchcolumn*\latim{
Omnípotens sempitérne Deus, qui non mortem peccatórum, sed vitam semper inquíris: súscipe propítius oratiónem nostram, et líbera eos ab
idolórum cultúra; et ággrega Ecclésiæ tuæ sanctæ, ad laudem et glóriam nóminis tui. Per Dóminum \emph{\&c.} ℟. Amen.
}\switchcolumn\portugues{
Omnipotente e eterno Deus, que procurais sempre não a morte dos pecadores mas a sua vida, ouvi benigno a nossa oração, livrai os pagãos do culto aos ídolos e agregai-os à vossa santa Igreja, para honra e glória do vosso nome. Por nosso Senhor \emph{\&c.} ℟. Amen.
}\end{paracol}

\subsubsection{Adoração da Cruz}

\paragraph{Veníte, adorémus}
\begin{paracol}{2}\latim{
\rlettrine{E}{cce} lignum Crucis, in quo salus mundi pependit.
}\switchcolumn\portugues{
\rlettrine{E}{is} o Lenho da Cruz, do qual pendeu a salvação do mundo!
}\switchcolumn*\latim{
℟. Veníte, adoremus.
}\switchcolumn\portugues{
℟. Vinde, adoremo-lo!
}\switchcolumn*\latim{
℣. Pópule meus, quid feci tibi? aut in quo contristávi te? respónde mihi. ℣. Quia edúxi te de terra Ægýpti: parásti Crucem Salvatóri tuo
}\switchcolumn\portugues{
℣. Ó meu povo, que mal te fiz ou em que te contristei? Responde-me! ℣. Foi por te haver tirado da terra do Egipto que preparaste a Cruz para o teu Salvador?
}\switchcolumn*\latim{
℟. Agios o Theós. ℟. Sanctus Deus. ℟. Agios ischyrós. ℟. Sanctus fortis. ℟. Agios athánatos, eléison imas. ℟. Sanctus immortális, miserére nobis.
}\switchcolumn\portugues{
℟. Ó Deus santo! ℟. Ó Deus santo! ℟. Ó santo forte! ℟. Ó santo forte! ℟. Ó santo imortal, compadecei-Vos de nós! ℟. Ó santo imortal, compadecei-Vos de nós!
}\switchcolumn*\latim{
℣. Quia edúxi te per desértum quadragínta annis, et manna cibávi te, et introdúxi te in terram satis bonam: parásti Crucem Salvatóri tuo.
}\switchcolumn\portugues{
℣. Foi porque durante quarenta anos te conduzi no deserto, te alimentei com o maná e te introduzi numa terra excelente que preparaste a Cruz para o teu Salvador?
}\switchcolumn*\latim{
℟. Agios o Theós. ℟. Sanctus Deus. ℟. Agios ischyrós. ℟. Sanctus fortis. ℟. Agios athánatos, eléison imas. ℟. Sanctus immortális, miserére nobis.
}\switchcolumn\portugues{
℟. Ó Deus santo! ℟. Ó Deus santo! ℟. Ó santo forte! ℟. Ó santo forte! ℟. Ó santo imortal, compadecei-Vos de nós! ℟. Ó santo imortal, compadecei-Vos de nós!
}\switchcolumn*\latim{
℣. Quid ultra débui fácere tibi, et non feci? Ego quidem plantávi te víneam meam speciosíssimam: et tu facta es mihi nimis amára: acéto namque sitim meam potásti: et láncea perforásti latus Salvatóri tuo.
}\switchcolumn\portugues{
℣. Que mais por ti pudera fazer, que não tivesse feito? Plantei-te, como vinha especiosíssima! E tu converteste-te para mim na maior amargura: pois com vinagre atravessaste quiseste o mitigar-me lado do teu a sede Salvador!
}\switchcolumn*\latim{
℟. Agios o Theós. ℟. Sanctus Deus. ℟. Agios ischyrós. ℟. Sanctus fortis. ℟. Agios athánatos, eléison imas. ℟. Sanctus immortális, miserére nobis.
}\switchcolumn\portugues{
℟. Ó Deus santo! ℟. Ó Deus santo! ℟. Ó santo forte! ℟. Ó santo forte! ℟. Ó santo imortal, compadecei-Vos de nós! ℟. Ó santo imortal, compadecei-Vos de nós!
}\switchcolumn*\latim{
℣. Ego propter te flagellávi Ægýptum cum primogénitis suis: et tu me flagellátum tradidísti.
}\switchcolumn\portugues{
℣. Por tua causa flagelei o Egipto em seus primogénitos! E tu entregaste-me para ser flagelado!
}\switchcolumn*\latim{
℟. Pópule meus, quid feci tibi? aut in quo contristávi te? respónde mihi.
}\switchcolumn\portugues{
℟. Ó meu povo, que mal te fiz ou em que te contristei? Responde-me!
}\switchcolumn*\latim{
℣. Ego edúxi te de Ægýpto, demérso Pharaóne in Mare Rubrum: et tu me tradidísti princípibus sacerdótum.
}\switchcolumn\portugues{
℣. Tirei-te do Egipto e submergi Faraó nas águas do mar Vermelho! E tu entregaste-me aos príncipes dos sacerdotes!
}\switchcolumn*\latim{
℟. Pópule meus, quid feci tibi? aut in quo contristávi te? respónde mihi.
}\switchcolumn\portugues{
℟. Ó meu povo, que mal te fiz ou em que te contristei? Responde-me!
}\switchcolumn*\latim{
℣. Ego ante te apérui mare: et tu aperuísti láncea latus meum.
}\switchcolumn\portugues{
℣. Abri o mar à tua passagem! E tu abriste-me o lado com uma lança!
}\switchcolumn*\latim{
℟. Pópule meus, quid feci tibi? aut in quo contristávi te? respónde mihi.
}\switchcolumn\portugues{
℟. Ó meu povo, que mal te fiz ou em que te contristei? Responde-me!
}\switchcolumn*\latim{
℣. Ego ante te præívi in colúmna nubis: et tu me duxísti ad prætórium Piláti.
}\switchcolumn\portugues{
℣. Caminhei diante de ti, como nuvem luminosa! E tu levaste-me ao pretório de Pilatos!
}\switchcolumn*\latim{
℟. Pópule meus, quid feci tibi? aut in quo contristávi te? respónde mihi
}\switchcolumn\portugues{
℟. Ó meu povo, que mal te fiz ou em que te contristei? Responde-me!
}\switchcolumn*\latim{
℣. Ego te pavi manna per desértum: et tu me cecidísti álapis et flagéllis.
}\switchcolumn\portugues{
℣. Com o maná te alimentei no deserto! E tu encheste-me de bofetadas e açoites!
}\switchcolumn*\latim{
℟. Pópule meus, quid feci tibi? aut in quo contristávi te? respónde mihi.
}\switchcolumn\portugues{
℟. Ó meu povo, que mal te fiz ou em que te contristei? Responde-me!
}\switchcolumn*\latim{
℣. Ego te potávi aqua salútis de petra: et tu me potásti felle et acéto.
}\switchcolumn\portugues{
℣. Fiz brotar água do rochedo para te saciar! E tu deste-me a beber fel e vinagre!
}\switchcolumn*\latim{
℟. Pópule meus, quid feci tibi? aut in quo contristávi te? respónde mihi.
}\switchcolumn\portugues{
℟. Ó meu povo, que mal te fiz ou em que te contristei? Responde-me!
}\switchcolumn*\latim{
℣. Ego propter te Chananæórum reges percússi: et tu percussísti arúndine caput meum.
}\switchcolumn\portugues{
℣. Por tua causa feri os reis dos Cananeus! E tu feriste-me a cabeça com uma cana!
}\switchcolumn*\latim{
℟. Pópule meus, quid feci tibi? aut in quo contristávi te? respónde mihi.
}\switchcolumn\portugues{
℟. Ó meu povo, que mal te fiz ou em que te contristei? Responde-me!
}\switchcolumn*\latim{
℣. Ego dedi tibi sceptrum regale: et tu dedísti capiti meo spíneam coronam.
}\switchcolumn\portugues{
℣. Dei-te o ceptro da realeza! E tu colocaste na minha cabeça uma coroa de espinhos!
}\switchcolumn*\latim{
℟. Pópule meus, quid feci tibi? aut in quo contristávi te? respónde mihi.
}\switchcolumn\portugues{
℟. Ó meu povo, que mal te fiz ou em que te contristei? Responde-me!
}\switchcolumn*\latim{
℣. Ego te exaltávi magna virtúte: et tu me suspendísti in patíbulo Crucis.
}\switchcolumn\portugues{
℣. Elevei-te, revestindo-te com grande poder! E tu suspendeste-me no patíbulo da Cruz!
}\switchcolumn*\latim{
℟. Pópule meus, quid feci tibi? aut in quo contristávi te? respónde mihi.
}\switchcolumn\portugues{
℟. Ó meu povo, que mal te fiz ou em que te contristei? Responde-me!
}\switchcolumn*\latim{
℣. Crucem tuam adorámus, Dómine: et sanctam resurrectiónem tuam laudámus et glorificámus: ecce enim, propter lignum venit gaudium in univérso mundo.
}\switchcolumn\portugues{
℣. Senhor, adoramos a vossa Cruz; louvamos e glorificamos a vossa santa Ressurreição; pois foi por este Lenho que a alegria apareceu em todo o mundo.
}\switchcolumn*\latim{
\emph{Ps. 66, 2} Deus misereátur nostri et benedícat nobis:
}\switchcolumn\portugues{
\emph{Sl. 66, 2} Que Deus tenha piedade de nós e nos abençoe.
}\switchcolumn*\latim{
℟. Illúminet vultum suum super nos et misereátur nostri.
}\switchcolumn\portugues{
℟. Que nos ilumine com o brilho da sua face e seja misericordioso para connosco.
}\switchcolumn*\latim{
℣. Crucem tuam adorámus, Dómine: et sanctam resurrectiónem tuam laudámus et glorificámus: ecce enim, propter lignum venit gáudium in univérso mundo.
}\switchcolumn\portugues{
℣. Senhor, adoramos a vossa Cruz; louvamos e glorificamos a vossa santa Ressurreição; pois foi por este Lenho que a alegria apareceu em todo o mundo.
}\switchcolumn*\latim{
℟. Crux fidélis, inter omnes arbor una nóbilis: nulla silva talem profert fronde, flore, gérmine. Dulce lignum dulces clavos, dulce pondus sústinet.
}\switchcolumn\portugues{
℟. Ó Cruz, em que tenho fé, árvore única, a mais nobre entre todas! Nenhuma floresta produz outra igual, nem nas folhas, nem nas flores, nem nos frutos. Ó amável Lenho, ó cravos sagrados, que segurais um fardo tão precioso!
}\switchcolumn*\latim{
{\redx Hymnus} ℣. Pange, lingua, gloriósi láuream certáminis, et super Crucis trophǽo dic triúmphum nóbilem: quáliter Redémptor orbis immolátus vícerit.
}\switchcolumn\portugues{
{\redx Hino} ℣. Canta, ó língua, os louros do glorioso combate; celebra o nobre triunfo de que a Cruz é o troféu! Canta a vitória que o Redentor do mundo alcançou, se imolando.
}\switchcolumn*\latim{
℟. Crux fidélis, inter omnes arbor una nóbilis: nulla silva talem profert fronde, flore, gérmine.
}\switchcolumn\portugues{
℟. Ó Cruz, em que tenho fé, árvore única, a mais nobre entre todas! Nenhuma floresta produz outra igual, nem nas folhas, nem nas flores, nem nos frutos.
}\switchcolumn*\latim{
℣. De paréntis protoplásti fraude Factor cóndolens, quando pomi noxiális in necem morsu ruit: ipse lignum tunc notávit, damna ligni ut sólveret.
}\switchcolumn\portugues{
℣. Condoído da infelicidade que a sedução trouxe ao nosso primeiro pai, precipitado na morte por haver comido o fruto funesto, o Criador, desde então, designou outra árvore para reparar os males da primeira.
}\switchcolumn*\latim{
℟. Dulce lignum dulces clavos, dulce pondus sústinet.
}\switchcolumn\portugues{
℟. Ó amável Lenho, ó cravos sagrados, que segurais um fardo tão precioso!
}\switchcolumn*\latim{
℣. Hoc opus nostræ salútis ordo depopóscerat: multifórmis proditóris ars ut artem fálleret: et medélam ferret inde, hostis unde lǽserat.
}\switchcolumn\portugues{
℣. Tal obra era necessária para a nossa salvação. A sabedoria divina frustrou deste modo o astuto traidor, vindo-nos o remédio pelo instrumento de que se servira o inimigo para nos ferir.
}\switchcolumn*\latim{
℟. Crux fidélis, inter omnes arbor una nóbilis: nulla silva talem profert fronde, flore, gérmine.
}\switchcolumn\portugues{
℟. Ó Cruz, em que tenho fé, árvore única, a mais nobre entre todas! Nenhuma floresta produz outra igual, nem nas folhas, nem nas flores, nem nos frutos.
}\switchcolumn*\latim{
℣. Quando venit ergo sacri plenitúdo témporis, missus est ab arce Patris Natus, orbis Cónditor: atque ventre virgináli carne amíctus pródiit.
}\switchcolumn\portugues{
℣. Quando veio a plenitude do tempo assinalado, Aquele por quem o mundo foi criado foi mandado do trono do Pai; e, fazendo-se carne em um seio virginal, apareceu neste mundo.
}\switchcolumn*\latim{
℟. Dulce lignum dulces clavos, dulce pondus sústinet.
}\switchcolumn\portugues{
℟. Ó amável Lenho, ó cravos sagrados, que segurais um fardo tão precioso!
}\switchcolumn*\latim{
℣. Vagit Infans inter arcta cónditus præsépia: membra pannis involúta Virgo Mater álligat: et Dei manus pedésque stricta cingit fáscia.
}\switchcolumn\portugues{
℣. Deu os primeiros vagidos deitado em pobre presépio, e a Virgem Mãe cobriu-lhe com panos os delicados membros, ficando cativas com faixas de pano as mãos e os pés de um Deus!
}\switchcolumn*\latim{
℟. Crux fidélis, inter omnes arbor una nóbilis: nulla silva talem profert fronde, flore, gérmine.
}\switchcolumn\portugues{
℟. Ó Cruz, em que tenho fé, árvore única, a mais nobre entre todas! Nenhuma floresta produz outra igual, nem nas folhas, nem nas flores, nem nos frutos.
}\switchcolumn*\latim{
℣. Lustra sex qui jam perégit, tempus implens córporis, sponte líbera Redémptor passióni déditus, Agnus in Crucis levátur immolándus stípite.
}\switchcolumn\portugues{
℣. Depois de haver vivido seis lustros, estando completo o tempo da sua vida mortal, o Redentor entregou-se livremente ao sofrimento. O Cordeiro foi elevado na Cruz para nela ser imolado.
}\switchcolumn*\latim{
℟. Dulce lignum dulces clavos, dulce pondus sústinet.
}\switchcolumn\portugues{
℟. Ó amável Lenho, ó cravos sagrados, que segurais um fardo tão precioso!
}\switchcolumn*\latim{
℣. Felle potus ecce languet: spina, clavi, láncea mite corpus perforárunt, unda manat et cruor: terra, pontus, astra, mundus, quo lavántur flúmine!
}\switchcolumn\portugues{
℣. Eis que na agonia dão-Lhe a beber fel; e os espinhos, os cravos e a lança ferem o seu delicado corpo, donde manam água e sangue. E este digno rio lava a terra, o mar, os astros e o mundo inteiro.
}\switchcolumn*\latim{
℟. Crux fidélis, inter omnes arbor una nóbilis: nulla silva talem profert fronde, flore, gérmine.
}\switchcolumn\portugues{
℟. Ó Cruz, em que tenho fé, árvore única, a mais nobre entre todas! Nenhuma floresta produz outra igual, nem nas folhas, nem nas flores, nem nos frutos.
}\switchcolumn*\latim{
℣. Flecte ramos, arbor alta, tensa laxa víscera, et rigor lentéscat ille, quem dedit natívitas: et supérni membra Regis tende miti stípite.
}\switchcolumn\portugues{
℣. Ó árvore augusta, verga os teus ramos, afrouxa as fibras, quebra a rigidez que te deu a natureza, e torna-te em leito macio para os membros do Rei supremo!
}\switchcolumn*\latim{
℟. Dulce lignum dulces clavos, dulce pondus sústinet.
}\switchcolumn\portugues{
℟. Ó amável Lenho, ó cravos sagrados, que segurais um fardo tão precioso!
}\switchcolumn*\latim{
℣. Sola digna tu fuísti ferre mundi víctimam: atque portum præparáre arca mundo náufrago: quam sacer cruor perúnxit, fusus Agni córpore.
}\switchcolumn\portugues{
℣. Só tu foste julgada digna de sustentar em teus braços a Vítima do mundo. Para este mundo naufragado, tu, banhada pelo sangue do divino Cordeiro, foste o primeiro piloto que o conduziu ao porto.
}\switchcolumn*\latim{
℟. Crux fidélis, inter omnes arbor una nóbilis: nulla silva talem profert fronde, flore, gérmine.
}\switchcolumn\portugues{
℟. Ó Cruz, em que tenho fé, árvore única, a mais nobre entre todas! Nenhuma floresta produz outra igual, nem nas folhas, nem nas flores, nem nos frutos.
}\switchcolumn*\latim{
℣. Sempitérna sit beátæ Trinitáti glória: æqua Patri Filióque; par decus Paráclito: Uníus Triníque nomen laudet univérsitas. ℟. Amen.
}\switchcolumn\portugues{
℣. Glória eterna à bem-aventurada Trindade; igual homenagem ao Pai, e ao Filho, e ao Paráclito. Que o nome de Deus uno e trino seja louvado em todo o orbe. ℟. Amen.
}\switchcolumn*\latim{
℟. Dulce lignum dulces clavos, dulce pondus sústinet.
}\switchcolumn\portugues{
℟. Ó amável Lenho, ó cravos sagrados, que segurais um fardo tão precioso!
}\end{paracol}

\subsection{Missa dos Pré-Santificados}

\emph{Conduz-se Processionalmente a Divina Hóstia para o Altar onde se celebra o Ofício. Entretanto, canta-se o Hino:}

\paragraph{Vexilla Regis}
\gregorioscore{scores/paixao/vexillaregisprodeunt}

\begin{nscenter}
Ó nobre estandarte do Rei dos reis, ó misteriosa Cruz, aparece agora, pois a vida sofreu a morte, e pela sua morte nos deu a vida!
\end{nscenter}

\begin{paracol}{2}\latim{
\qlettrine{Q}{uæ,} vulneráta lánceæ
Mucróne diro, críminum
Ut nos laváret sórdibus,
Manávit unda et sánguine.
}\switchcolumn\portugues{
\rlettrine{D}{o} seu lado, ferido pela cruel lança, correm a água e o sangue, destinados a lavrar a nódoa dos nossos crimes.
}\switchcolumn*\latim{
Impléta sunt quæ cóncinit
David fidéli cármine,
Dicéndo natiónibus :
Regnávit a ligno Deus.
}\switchcolumn\portugues{
Cumpriu-se o oráculo de David, que nos seus cânticos inspirados havia anunciado às nações: «Deus reinará pelo madeiro».
}\switchcolumn*\latim{
Arbor decóra et fúlgida,
Ornáta Regis púrpura,
Elécta digno stípite
Tam sancta membra tángere.
}\switchcolumn\portugues{
Sois bela e brilhante de gloória, ó árvore enaltecida com a púrpura do Rei: tronco escolhido e julgado digno de tocar nos membros dos santos.
}\switchcolumn*\latim{
Beáta, cuius bráchiis
Prétium pepéndit sǽculi,
Statéra facta córporis,
Tulítque prædam tártari.
}\switchcolumn\portugues{
Ó feliz Cruz, de cujos braços pendeu o penhor do mundo! Fostes a balança que pesou o Corpo, cujo peso arrancou ao inferno a sua presa!
}\switchcolumn*\latim{
O Crux, ave, spes única,
Hoc Passiónis témpore
Piis adáuge grátiam,
Reísque dele crímina.
}\switchcolumn\portugues{
Salve, ó Cruz, nossa única esperança, nestes dias consagrados a honrar a Paixão do Salvador concedei aos justos aumento da graça e aos pecadores apagai seus crimes.
}\switchcolumn*\latim{
Te, fons salútis, Trínitas,
Colláudet omnis spíritus :
Quibus Crucis victóriam
Largíris, adde prǽmium.
Amen.
}\switchcolumn\portugues{
Que todos os espíritos cantem vossos louvores, ó Trindade, fonte da nossa salvação. Vós, que nos dais a vitória pela Cruz, dignai-Vos aumentá-la com a recompensa. Amen.
}\end{paracol}


\begin{paracol}{2}\latim{
\rlettrine{I}{n} spiritu humilitátis et in ánimo contríto suscipiámur a te, Dómine: et sic fiat sacrifícium nostrum in conspéctu tuo hódie, ut pláceat tibi, Dómine Deus.
}\switchcolumn\portugues{
\rlettrine{C}{om} o espírito humilhado e com o coração contrito, Senhor, Vos pedimos, dignai-Vos receber-nos, para que este sacrifício seja feito hoje na vossa presença e de modo que Vos seja agradável.
}\switchcolumn*\latim{
Oráte, fratres, ut meum ac vestrum sacrifícium acceptábile fiat apud Deum Patrem omnipoténtem.
}\switchcolumn\portugues{
Orai, meus irmãos, a fim de que este meu sacrifício, que é também vosso, seja agradável a Deus, Pai omnipotente.
}\switchcolumn*\latim{
\begin{nscenter} Orémus. \end{nscenter}
}\switchcolumn\portugues{
\begin{nscenter} Oremos. \end{nscenter}
}\switchcolumn*\latim{
Percéptio Córporis tui, Dómine Jesu Christe, quod ego indígnus súmere præsúmo, non mihi provéniat in judícium et condemnatiónem: sed pro tua pietáte prosit mihi ad tutaméntum mentis et córporis, et ad
medélam percipiéndam: Qui vivis et regnas \emph{\&c.}
}\switchcolumn\portugues{
Senhor Jesus Cristo, que este vosso Corpo, que eu, ainda que indignamente, me proponho receber, não seja para meu juízo e condenação; mas que, pela vossa misericórdia, sirva à minha alma e ao meu corpo de defesa e de remédio salutar. Vós, que, sendo Deus \emph{\&c.}
}\switchcolumn*\latim{
Panem cœléstem accípiam, et nomen Dómini invocábo.
}\switchcolumn\portugues{
Tomarei o Pão Celestial e invocarei o nome do Senhor.
}\switchcolumn*\latim{
Dómine, non sum dignus, ut intres sub tectum meum: sed tantum dic verbo, et sanábitur ánima mea.
}\switchcolumn\portugues{
Senhor, não sou digno de que entreis em minha morada, mas dizei uma só palavra e minha alma será salva.
}\switchcolumn*\latim{
Corpus Dómini nostri Jesu Christi custódiat ánimam meam in vitam ætérnam. Amen.
}\switchcolumn\portugues{
Que o corpo de N. S. Jesus Cristo guarde a minha alma para a vida eterna. Amen.
}\switchcolumn*\latim{
Quod ore súmpsimus, Dómine, pura mente capiámus: et de múnere temporáli fiat nobis remédium sempitérnum.
}\switchcolumn\portugues{
Concedei-nos, Senhor, que conservemos com pureza de coração o que acaba de receber a nossa boca; e que esta dádiva temporal, que nos fizestes, se torne para nós em um remédio eterno.
}\end{paracol}

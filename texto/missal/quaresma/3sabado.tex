\subsectioninfo{Sábado da 3.ª Semana da Quaresma}{Estação em Santa Susana}

\paragraphinfo{Intróito}{Sl. 5, 2-3}
\begin{paracol}{2}\latim{
\rlettrine{V}{erba} mea áuribus pércipe, Dómine, intéllege clamórem meum: inténde voci oratiónis meæ, Rex meus et Deus meus. \emph{Ps. ibid., 4} Quóniam ad te orábo, Dómine: mane exáudies vocem meam.
℣. Gloria Patri \emph{\&c.}
}\switchcolumn\portugues{
\rlettrine{O}{uvi} as minhas palavras, Senhor: atendei ao meu clamor e escutai a voz da minha oração, ó meu Rei e meu Deus! \emph{Sl. ibid., 4} Porquanto a Vós Orarei, Senhor; desde manhã ouvireis a minha voz.
℣. Glória ao Pai \emph{\&c.}
}\end{paracol}

\paragraph{Oração}
\begin{paracol}{2}\latim{
\rlettrine{P}{ræsta,} quǽsumus, omnípotens Deus: ut, qui se, affligéndo carnem, ab aliméntis ábstinent; sectándo justítiam, a culpa jejúnent. Per Dóminum \emph{\&c.}
}\switchcolumn\portugues{
\slettrine{Ó}{} Deus omnipotente, àqueles que, para mortificação da sua carne, se abstêm dos alimentos concedei a graça, Vos suplicamos, de se absterem também das culpas, praticando sempre a virtude. Por nosso Senhor \emph{\&c.}
}\end{paracol}

\paragraphinfo{Epístola}{Dn. 13., 1-9, 15-17, 19-30 \& 33-62}
\begin{paracol}{2}\latim{
Léctio Daniélis Prophétæ.
}\switchcolumn\portugues{
Lição do Profeta Daniel.
}\switchcolumn*\latim{
\rlettrine{I}{n} In diébus illis: Erat vir hábitans in Babylóne, et nomen ejus Jóakim: et accépit uxorem nómine Susánnam, fíliam Helcíæ, pulchram nimis, et timéntem Deum: paréntes enim illíus, cum essent justi, erudiérunt fíliam suam secúndum legem Móysi. Erat autem Jóakim dives valde, et erat ei pomárium vicínum dómui suæ: et ad ipsum confluébant Judǽi, eo quod esset honorabílior ómnium. Et constítuti sunt de pópulo duo senes júdices in illo anno: de quibus locútus est Dóminus: Quia egréssa est iníquitas de Babylóne a senióribus judícibus, qui videbántur régere pópulum. Isti frequentábant domum Jóakim, et veniébant ad eos omnes, qui habébant judícia. Cum autem pópulus revertísset per merídiem, ingrediebátur Susánna, et deambulábat in pomário viri sui. Et vidébant eam senes cotídie ingrediéntem et deambulántem: et exarsérunt in concupiscéntiam ejus: etevertérunt sensum suum, et declinavérunt óculos suos, ut non vidérent cœlum, neque recordaréntur judiciórum justórum. Factum est autem, cum observárent diem aptum, ingréssa est aliquándo sicut heri et núdius tértius, cum duábus solis puéllis, voluítque lavári in pomário: æstus quippe erat, et non erat ibi quisquam, præter duos senes abscónditos et contemplántes eam. Dixit ergo puéllis: Afférte mihi óleum et smígmata, et óstia pomárii cláudite, ut laver. Cum autem egréssæ essent puéllæ, surrexérunt duo senes, et accurrérunt ad eam, et dixérunt: Ecce, óstia pomárii clausa sunt, et nemo nos videt, et nos in concupiscéntia tui sumus: quam ob rem assentíre nobis, et commiscére nobiscum. Quod si nolúeris, dicémus contra te testimónium, quod fúerit tecum júvenis, et ob hanc causam emíseris puéllas a te. Ingémuit Susánna, et ait: Angústiæ sunt mihi úndique: si enim hoc égero, mors mihi est: si autem non egero, non effúgiam manus vestras. Sed mélius est mihi absque ópere incídere in manus vestras, quam peccáre in conspéctu Dómini. Et exclamávit voce magna Susánna: exclamavérunt autem et senes adversus eam. Et cucúrrit unus ad óstia pomárii, et aperuit. Cum ergo audíssent clamórem fámuli domus in pomário, irruérunt per postícum, ut vidérent, quidnam esset. Postquam autem senes locúti sunt, erubuérunt servi veheménter: quia numquam dictus fúerat sermo hujuscémodi de Susánna. Et facta est dies crástina. Cumque venísset pópulus ad Jóakim virum ejus, venérunt et duo senióres, pleni iníqua cogitatióne advérsus Susánnam, ut interfícerent eam. Et dixérunt coram pópulo: Míttite ad Susánnam fíliam Helcíæ, uxórem Jóakim. Et statim misérunt. Et venit cum paréntibus et fíliis et univérsis cognátis suis. Fiébant ígitur sui, et omnes qui nóverant eam. Consurgéntes autem duo senióres in médio pópuli, posuérunt manus suas super caput ejus. Quæ flens suspéxit ad cœlum: erat enim cor ejus fidúciam habens in Dómino. Et dixérunt senióres: Cum deambularémus in pomário soli, ingréssa est hæc cum duábus puéllis: et clausit óstia pomárii, et dimísit a se puéllas. Venítque ad eam adoléscens, qui erat abscónditus, et concúbuit cum ea. Porro nos, cum essémus in ángulo pomárii, vidéntes iniquitátem, cucúrrimus ad eos, et vídimus eos pariter commiscéri. Et illum quidem non quívimus comprehéndere, quia fórtior nobis erat, et apértis óstiis exsilívit: hanc autem cum apprehendissémus, interrogávimus, quisnam esset adoléscens, et nóluit indicáre nobis: hujus rei testes sumus. Crédidit eis multitúdo quasi sénibus et judícibus pópuli, et condemnavérunt eam ad mortem. Exclamávit autem voce magna Susánna, et dixit: Deus ætérne, qui absconditórum es cógnitor, qui nosti ómnia, ántequam fiant, tu scis, quóniam falsum testimónium tulérunt contra me: et ecce, mórior, cum nihil horum fécerim, quæ isti malitióse composuérunt advérsum me. Exaudívit autem Dóminus vocem ejus. Cumque ducerétur ad mortem, suscitávit Dóminus spíritum sanctum pueri junióris, cujus nomen Dániel. Et exclamávit voce magna: Mundus ego sum a sánguine hujus. Et convérsus omnis pópulus ad eum, dixit: Quis est iste sermo, quem tu locútus es? Qui cum staret in médio eórum, ait: Sic fátui, fílii Israël, non judicántes, neque quod verum est cognoscéntes, condemnástis fíliam Israël? Revertímini ad judícium, quia falsum testimónium locúti sunt advérsus eam. Revérsus est ergo pópulus cum festinatióne. Et dixit ad eos Dániel: Separáte illos ab ínvicem procul, et dijudicábo eos. Cum ergo divísi essent alter ab áltero, vocávit unum de eis, et dixit ad eum: Inveteráte diérum malórum, nunc venérunt peccáta tua, quæ operabáris prius: júdicans judícia injústa, innocéntes ópprimens, et dimíttens nóxios, dicénte Dómino: Innocéntem et justum non interfícies. Nunc ergo, si vidisti eam, dic, sub qua arbóre vidéris eos colloquéntes sibi. Qui ait: Sub schino. Dixit autem Dániel: Recte mentítus es in caput tuum. Ecce enim, Angelus Dei, accépta senténtia ab eo, scindet te médium. Et, amóto eo, jussit veníre álium, et dixit ei: Semen Chánaan, et non Juda, spécies decépit te, et concupiscéntia subvértit cor tuum: sic faciebátis filiábus Israël, et illæ timéntes loquebántur vobis: sed fília Juda non sustínuit iniquitátem vestram. Nunc ergo dic mihi, sub qua arbóre comprehénderis eos loquéntes sibi. Qui ait: Sub prino. Dixit autem ei Dániel: Recte mentítus es et tu in caput tuum: manet enim Angelus Dómini, gládium habens, ut secet te médium, et interfíciat vos. Exclamávit itaque omnis cœtus voce magna, et benedixérunt Deum, qui salvat sperántes in se. Et consurrexérunt advérsus duos senióres (convícerat enim eos Dániel ex ore suo falsum dixísse testimónium), fecerúntque eis, sicut male égerant advérsus próximum: et interfecérunt eos, et salvátus est sanguis innóxius in die illa.
}\switchcolumn\portugues{
\rlettrine{N}{aqueles} dias, habitava em Babilónia um varão cujo nome era Joaquim, que desposara uma mulher, chamada Susana, filha de Helcias, a qual era formosíssima e temente a Deus, pois seus pais, que eram justos, haviam-na instruído na lei de Moisés. Joaquim era muito rico e possuía um jardim próximo de sua casa. Os judeus procuravam-no sempre, porque ele era o mais honrado de todos. Naquele ano foram constituídos juízes do povo dous velhos, dos quais disse o Senhor: «Que a iniquidade saiu da Babilónia pelos velhos, que eram juízes e pareciam conduzir o povo». Estes frequentavam a casa de Joaquim, onde vinham também todos quantos tinham questões a dirimir. Quando, porém, cerca do meio-dia, se retirava o povo, entrava Susana e passeava no jardim de seu marido. Viam-na os velhos, todos os dias, entrar e passear. Então conceberam uma ardente paixão por ela; e, pervertendo os sentidos, afastaram seus olhos do céu, para não meditarem nele e não se recordarem dos justos juízos de Deus. Aconteceu, pois, que, esperando eles um dia propício, Susana entrou no jardim, segundo o costume, acompanhada só por duas criadas, com intenção de se banhar no jardim, pois estava calor e não havia lá ninguém, além dos dous velhos escondidos, que a espreitavam. Disse, pois, ela às criadas: «Trazei-me óleo e perfumes e fechai as portas do jardim, para que eu me banhe». Logo que as criadas saíram, levantaram-se os dous velhos, correram para Susana e disseram-lhe: «Eis que as portas do jardim estão fechadas; ninguém nos vê. Nós ardemos de paixão por ti; consente, pois, no nosso desejo, entregando-te a nós. Se recusas, daremos contra ti testemunho de que esteve contigo um mancebo e que, por isso, despediste as criadas». Suspirou Susana e disse: «De todos os lados me oprimem angústias! Se eu fizer isso, será a morte para mim; e, se o não fizer, não me livro das vossas mãos. Mas é melhor para mim cair nas vossas mãos, sem praticar o pecado, do que pecar diante de Deus». Então Susana gritou com voz forte, fazendo o mesmo os velhos contra ela. Um deles correu à porta do jardim e abriu-a. Ouvindo os criados da casa este ruído, precipitaram-se pelo postigo para verem o que lá haveria. Depois que os velhos falaram, envergonharam-se muito os servos, porque ninguém dissera igual cousa de Susana. No dia seguinte, vindo o povo a casa de Joaquim, seu marido, vieram também os dous velhos, com o iníquo pensamento de conseguir a morte de Susana. Disseram, então, diante do povo: «Mandai buscar Susana, filha de Helcias e mulher de Joaquim». Imediatamente foram buscá-la. Veio ela com seus pais, filhos e todos que a conheciam, chorando todos. Levantando-se, pois, os dous velhos no meio do povo, puseram as mãos sobre a cabeça de Susana. Ela, chorando, ergueu os olhos ao céu, pois o seu coração estava cheio de confiança no Senhor! Disseram os velhos: «Passeando nós no jardim, a sós, esta mulher entrou com duas criadas, fechou as portas do jardim e despediu as criadas. Então veio ter com ela um mancebo, que estava escondido, e praticou acções más com ela. Porém, nós, que estávamos a um canto do jardim, vendo esta iniquidade, corremos para eles e vimo-los praticar o mal. Ao mancebo, porém, não pudemos agarrar, porque era mais forte do que nós, e, tendo aberto a porta, fugiu. Logo prendemos esta e lhe perguntámos quem era o mancebo; mas não quis dizer-nos. Disto somos testemunhas». Deu-lhes crédito a multidão, porque eram velhos e juízes do povo, e condenaram Susana à morte. Então exclamou ela em voz alta: «Deus eterno, que conheceis o que é oculto, bem como todas as cousas antes que elas aconteçam, sabeis que se apresenta um falso testemunho contra mim e que vou morrer, sem nunca ter feito nenhuma das cousas que malignamente me imputam!». O Senhor ouviu a sua oração. E, sendo ela conduzida para a morte, despertou o Senhor o Espírito Santo num jovem, chamado Daniel, o qual gritou em alta voz: «Eu sou inocente do sangue desta mulher». Então, voltando-se todos para ele, disseram-lhe: «Que disseste?». Daniel, ficando de pé, no meio deles, disse: «Sois vós tão insensatos, ó filhos de Israel, que condeneis à morte uma filha de Israel sem procurar conhecer a verdade? Repeti o julgamento, porque se apresentou um falso testemunho contra ela». Voltou-se imediatamente o povo; e Daniel continuou: «Separai os juízes um do outro e eu os julgarei». Sendo separados um do outro, chamou Daniel um deles e disse-lhe: «Homem envelhecido no mal, os pecados que outrora cometeste caem hoje sobre ti! Tu, que proferias juízos iníquos, oprimindo inocentes e absolvendo culpados (quando o Senhor diz: «Não matarás o inocente e o justo»), diz-me: debaixo de que árvore viste Susana e o mancebo, falando juntos?». Ele respondeu: «Debaixo duma aroeira». E Daniel disse: «Claramente mentiste contra a tua cabeça! Eis que o Anjo de Deus, que de Deus recebeu tua sentença, vai rachar-te de meio a meio». Foi este homem retirado, sendo mandado vir o outro, a quem disse: «Raça de Canaan (e não de Judá), a formosura cativou-te e a paixão perverteu-te o coração. Assim fazias às filhas de Israel, que, temendo-te, falavam contigo; mas a filha de Judá não quis suportar a tua iniquidade! Agora, pois, dize-me: debaixo de que árvore os surpreendeste, quando falavam os dous?». Respondeu ele: «Debaixo duma azinheira». Daniel retorquiu: «Claramente mentiste contra a tua cabeça; pois o Anjo do Senhor está preparado, com a espada já erguida para te passar pelo meio e vos matar a ambos!». Então toda a multidão clamou em voz alta e louvou Deus, porque Ele salva aqueles que n’Ele confiam. E voltaram-se contra os dous velhos (pois Daniel os convencera pelas suas próprias palavras) e fizeram-lhes o mal que eles queriam fazer ao seu próximo, isto é, mataram-nos. E naquele dia foi salvo o sangue inocente.
}\end{paracol}

\paragraphinfo{Gradual}{Sl. 22, 4}
\begin{paracol}{2}\latim{
\rlettrine{S}{i} ámbulem in médio umbræ mortis, non timébo mala: quóniam tu mecum es, Dómine. ℣. Virga tua et báculus tuus, ipsa me consoláta sunt.
}\switchcolumn\portugues{
\rlettrine{A}{inda} que eu ande no meio das sombras da morte, não recearei mal algum, pois Vós, Senhor, estais comigo. ℣. Vossa vara e o vosso báculo confortam-me.
}\end{paracol}

\paragraphinfo{Evangelho}{Jo. 8, 1-11}
\begin{paracol}{2}\latim{
\cruz Sequéntia sancti Evangélii secúndum Joánnem.
}\switchcolumn\portugues{
\cruz Continuação do santo Evangelho segundo S. João.
}\switchcolumn*\latim{
\blettrine{I}{n} illo témpore: Perréxit Jesus in montem Olivéti: et dilúculo íterum venit in templum, et omnis pópulus venit ad eum, et sedens docébat eos. Addúcunt autem scribæ et pharisǽi mulíerem in adultério deprehénsam: et statuérunt eam in médio, et dixérunt ei: Magister, hæc mulier modo deprehénsa est in adultério. In lege autem Moyses mandávit nobis hujúsmodi lapidáre. Tu ergo quid dicis? Hoc autem dicébant tentántes eum, ut possent accusáre eum. Jesus autem inclínans se deórsum, dígito scribébat in terra. Cum ergo perseverárent interrogántes eum, eréxit se, et dixit eis: Qui sine peccáto est vestrum, primus in illam lápidem mittat. Et íterum se inclínans, scribébat in terra. Audiéntes autem unus post unum exíbant, incipiéntes a senióribus: et remánsit solus Jesus, et múlier in médio stans. Erigens autem se Jesus, dixit ei: Múlier, ubi sunt, qui te accusábant? nemo te condemnávit? Quæ dixit: Nemo, Dómine. Dixit autem Jesus: Nec ego te condemnábo: Vade, et jam ámplius noli peccáre.
}\switchcolumn\portugues{
\blettrine{N}{aquele} tempo, Jesus retirou-se para o monte das Oliveiras, voltando novamente ao templo ao romper da manhã. Logo o povo se aproximou d’Ele, sentando-se. E Jesus ensinava a multidão. Os escribas e fariseus trouxeram-Lhe, então, uma mulher, apanhada em adultério, e perguntaram-Lhe: «Mestre, esta mulher acaba de ser surpreendida em adultério. Moisés manda na lei que os adúlteros sejam apedrejados; tu, porém, que dizes?». Isto diziam, tentando-O, para ver se podiam acusá-l’O. Mas Jesus, tendo-se inclinado para o chão, escrevia com o dedo na terra. Como continuassem a interrogá-l’O, endireitou-se Jesus, dizendo-lhes: «Aquele de vós que estiver sem pecado seja o primeiro a atirar-lhe a pedra». E, inclinando-se outra vez, continuou a escrever na terra. Eles, porém, ouvindo estas palavras, retiraram-se uns após outros (a começar pelos mais velhos), tendo ficado só Jesus e a mulher, que estava de pé. Erguendo-se, então, Jesus disse-lhe: «Mulher, onde estão os que te acusavam? Ninguém te condenou?». Ela disse: «Ninguém, Senhor». Jesus continuou: «Nem eu também te condenarei. Vai e não tornes a pecar».
}\end{paracol}

\paragraphinfo{Ofertório}{Sl. 118, 133}
\begin{paracol}{2}\latim{
\rlettrine{G}{ressus} meos dírige secúndum elóquium tuum: ut non dominétur mei omnis injustítia, Dómine.
}\switchcolumn\portugues{
\rlettrine{D}{irigi} os meus passos segundo os vossos preceitos, Senhor; permiti que não domine em mim iniquidade alguma.
}\end{paracol}

\paragraph{Secreta}
\begin{paracol}{2}\latim{
\rlettrine{C}{oncéde,} quǽsumus, omnípotens Deus: ut hujus sacrifícii munus oblátum, fragilitátem nostram ab omni malo purget semper et múniat. Per Dóminum \emph{\&c.}
}\switchcolumn\portugues{
\rlettrine{V}{os} suplicamos, ó Deus omnipotente, concedei-nos que a oblação deste sacrifício, que Vos oferecemos, nos livre de todos os males da nossa fraqueza e nos fortaleça. Por nosso Senhor \emph{\&c.}
}\end{paracol}

\paragraphinfo{Comúnio}{Jo. 8, 10 \& 11}
\begin{paracol}{2}\latim{
\rlettrine{N}{emo} te condemnávit, mulier? Nemo, Dómine. Nec ego te condemnábo: jam ámplius noli peccáre.
}\switchcolumn\portugues{
\rlettrine{M}{ulher,} ninguém te condenou? Ninguém, Senhor. Nem Eu, também, te condenarei. Vai e não tornes a pecar.
}\end{paracol}

\paragraph{Postcomúnio}
\begin{paracol}{2}\latim{
\qlettrine{Q}{uǽsumus,} omnípotens Deus: ut inter ejus membra numerémur, cujus córpori communicámus et sánguini: Qui tecum \emph{\&c.}
}\switchcolumn\portugues{
\slettrine{Ó}{} Deus omnipotente, Vos rogamos, permiti que sejamos contados no número dos membros d’Aquele que nos admitiu à comunhão do seu Corpo e do seu Sangue. Ele \emph{\&c.}
}\end{paracol}

\paragraph{Oração sobre o povo}
\begin{paracol}{2}\latim{
\begin{nscenter} Orémus. \end{nscenter}
}\switchcolumn\portugues{
\begin{nscenter} Oremos. \end{nscenter}
}\switchcolumn*\latim{
Humiliáte cápita vestra Deo.
}\switchcolumn\portugues{
Inclinai as vossas cabeças diante de Deus.
}\switchcolumn*\latim{
Præténde, Dómine, fidélibus tuis déxteram cœléstis auxílii: ut te toto corde perquírant; et, quæ digne póstulant, cónsequi mereántur. Per Dóminum \emph{\&c.}
}\switchcolumn\portugues{
Senhor, estendei sobre os vossos fiéis a vossa dextra, assistindo-lhes com o auxílio celestial, para que Vos procurem com todo o coração e consigam alcançar o que dignamente suplicam. Por nosso Senhor \emph{\&c.}
}\end{paracol}

\subsectioninfo{Sábado da l.ª Semana da Quaresma}{Estação em S. Pedro}

\paragraphinfo{Intróito}{Sl. 87, 3}
\begin{paracol}{2}\latim{
\rlettrine{I}{ntret} orátio mea in conspéctu tuo: inclína aurem tuam ad precem meam, Dómine. \emph{Ps. ibid., 2} Dómine, Deus salútis meæ: in die clamávi, et nocte coram te.
℣. Gloria Patri \emph{\&c.}
}\switchcolumn\portugues{
\qlettrine{Q}{ue} a minha oração chegue à vossa presença! Senhor, ouvi benigno a minha súplica. \emph{Sl. ibid., 2} Senhor, meu Deus e meu Salvador, de dia e de noite clamo diante de Vós.
℣. Glória ao Pai \emph{\&c.}
}\end{paracol}

\begin{paracol}{2}\latim{
\begin{nscenter} Orémus. Flectámus génua. \end{nscenter}
}\switchcolumn\portugues{
\begin{nscenter} Oremos. Ajoelhemos! \end{nscenter}
}\switchcolumn*\latim{
℟. Leváte.
}\switchcolumn\portugues{
℟. Levantai-vos!
}\end{paracol}

\paragraph{Oração}
\begin{paracol}{2}\latim{
\rlettrine{P}{ópulum} tuum, quǽsumus, Dómine, propítius réspice: atque ab eo flagella tuæ iracúndiæ cleménter avérte. Per Dóminum \emph{\&c.}
}\switchcolumn\portugues{
\rlettrine{S}{enhor,} Vos suplicamos, olhai propício para o vosso povo e afastai piedosamente dele os flagelos da vossa ira. Por nosso Senhor \emph{\&c.}
}\end{paracol}

\paragraphinfo{1.ª Lição}{Dt. 26, 12-19}
\begin{paracol}{2}\latim{
Léctio libri Deuteronómii.
}\switchcolumn\portugues{
Lição do Livro do Deuteronómio.
}\switchcolumn*\latim{
\rlettrine{I}{n} diébus illis: Locútus est Móyses ad pópulum, dicens: Quando compléveris décimam cunctárum frugum tuárum, loquéris in conspéctu Dómini, Dei tui: Abstuli, quod sanctificátum est de domo mea, et dedi illud levítæ et ádvenæ et pupíllo ac víduæ, sicut jussísti mihi: non præterívi mandáta tua, nec sum oblítus impérii tui. Obœdívi voci Dómini, Dei mei, et feci ómnia, sicut præcepísti mihi. Réspice de sanctuário tuo et de excélso cœlórum habitáculo, et benedic pópulo tuo Israël, et terræ, quam dedísti nobis, sicut jurásti pátribus nostris, terræ lacte et melle manánti. Hódie Dóminus, Deus tuus, præcépit tibi, ut fácias mandáta hæc atque judícia: et custódias et ímpleas ex toto corde tuo et ex tota ánima tua. Dóminum elegísti hódie, ut sit tibi Deus, et ámbules in viis ejus, et custódias cæremónias illíus et mandáta atque judícia, et obǿdias ejus império. Et Dóminus elegit te hódie, ut sis ei pópulus peculiáris, sicut locútus est tibi, et custódias ómnia præcépta illíus: et fáciat te excelsiórem cunctis géntibus, quas creávit in laudem et nomen et glóriam suam: ut sis populus sanctus Dómini, Dei tui, sicut locútus est.
}\switchcolumn\portugues{
\rlettrine{N}{aqueles} dias, Moisés falou ao povo nestes termos: «Quando acabardes de pagar o dízimo de todos vossos frutos, falareis assim diante do Senhor, vosso Deus: «Tirei de minha casa o que era Consagrado, entreguei-o ao Levita, ao peregrino, ao órfão e à viúva, como mandastes; não transgredi, nem desprezei os vossos Mandamentos; obedeci à voz do Senhor, meu Deus, e tenho procedido segundo o que ordenastes. Então, Senhor, olhai lá do santuário e do alto dos céus, onde habitais, e abençoai o vosso povo de Israel e a terra, que nos destes, como jurastes a nossos pais, a terra onde brota o leite e o mel». Hoje, o Senhor, vosso Deus, manda guardar e cumprir de todo o coração e de toda a alma estas ordens e estes preceitos. Escolhestes hoje o Senhor para que seja vosso Deus, sigais os seus caminhos, guardeis as suas regras, preceitos e ordens e obedeçais à sua voz? Pois o Senhor também vos escolheu para que sejais o seu povo predilecto, como Ele disse, para que observeis todos seus preceitos e para vos tornar o povo mais ilustre de todas as nações, que criou, paraseu louvor, honra e glória, e para que sejais o povo santo do Senhor, vosso Deus, segundo a sua palavra».
}\end{paracol}

\paragraphinfo{Gradual}{Sl. 78, 9 \& 10}
\begin{paracol}{2}\latim{
\rlettrine{P}{ropítius} esto, Dómine, peccátis nostris: ne quando dicant gentes: Ubi est Deus eórum? ℣. Adjuva nos, Deus, salutáris noster: et propter honórem nóminis tui, Dómine, líbera nos.
}\switchcolumn\portugues{
\rlettrine{P}{erdoai} os nossos pecados, Senhor, para que os povos não digam: «Onde está o seu Deus?». ℣. Auxiliai-nos, ó Deus, nosso Salvador; para honra do vosso nome, Senhor, livrai-nos.
}\end{paracol}

\begin{paracol}{2}\latim{
\begin{nscenter} Orémus. Flectámus génua. \end{nscenter}
}\switchcolumn\portugues{
\begin{nscenter} Oremos. Ajoelhemos! \end{nscenter}
}\switchcolumn*\latim{
℟. Leváte.
}\switchcolumn\portugues{
℟. Levantai-vos!
}\end{paracol}

\paragraph{Oração}
\begin{paracol}{2}\latim{
\rlettrine{P}{otéctor} noster, áspice, Deus: ut, qui malórum nostrórum póndere prémimur, percépta misericórdia, líbera tibi mente famulémur. Per Dóminum \emph{\&c.}
}\switchcolumn\portugues{
\slettrine{Ó}{} Deus, nosso protector, dignai-Vos olhar propício para nós, que estamos oprimidos com o peso dos nossos males, a fim de que, por efeito da vossa misericórdia, Vos sirvamos com o espírito livre. Por nosso Senhor \emph{\&c.}
}\end{paracol}

\paragraphinfo{2.ª Lição}{Dt. 11, 22-25}
\begin{paracol}{2}\latim{
Léctio libri Deuteronómii.
}\switchcolumn\portugues{
Lição do Livro do Deuteronómio.
}\switchcolumn*\latim{
\rlettrine{I}{n} diébus illis: Dixit Móyses fíliis Israël: Si custodiéritis mandáta, quæ ego præcípio vobis, et fecéritis ea, ut diligátis Dóminum, Deum vestrum, et ambulétis in ómnibus viis ejus, adhærén-tes ei, dispérdet Dóminus omnes gentes istas ante fáciem vestram, et possidébitis eas, quæ majóres et fortióres vobis sunt. Omnis locus quem calcáverit pes vester, vester erit. A desérto et a Líbano, a flúmine magno Euphráte usque ad mare Occidentále, erunt términi vestri. Nullus stabit contra vos: terrórem vestrum et formídinem dabit Dóminus, Deus vester, super omnem terram, quam calcatúri estis, sicut locútus est vobis Dóminus, Deus vester.
}\switchcolumn\portugues{
\rlettrine{N}{aqueles} dias, Moisés disse aos filhos de Israel: «Se guardardes os Mandamentos, que vos entrego, amando o Senhor, vosso Deus, seguindo os seus caminhos e unindo-vos a Ele, o Senhor afastará de vós todas as outras nações e vos tornareis senhores de povos que são maiores e mais fortes do qu e vós. Todos os lugares que vossos pés pisarem serão vossos. Vossa fronteira estender-se-á desde o deserto do Líbano e do grande rio Eufrates até ao mar Ocidental. Ninguém poderá subsistir diante de vós, O Senhor, vosso Deus, como Ele disse, espalhará o pânico do vosso nome em todos os povos por onde passardes.
}\end{paracol}

\paragraphinfo{Gradual}{Sl. 83, 10 \& 9}
\begin{paracol}{2}\latim{
\rlettrine{P}{rotéctor} noster, áspice, Deus, et réspice super servos tuos. ℣. Dómine, Deus virtútum, exáudi preces servórum tuórum.
}\switchcolumn\portugues{
\rlettrine{O}{lhai} para nós, ó Deus, que sois o nosso protector; lançai os olhos para os vossos servos. ℣. Senhor, Deus dos exércitos, ouvi as súplicas dos vossos servos.
}\end{paracol}

\begin{paracol}{2}\latim{
\begin{nscenter} Orémus. Flectámus génua. \end{nscenter}
}\switchcolumn\portugues{
\begin{nscenter} Oremos. Ajoelhemos! \end{nscenter}
}\switchcolumn*\latim{
℟. Leváte.
}\switchcolumn\portugues{
℟. Levantai-vos!
}\end{paracol}

\paragraph{Oração}
\begin{paracol}{2}\latim{
\rlettrine{A}{désto,} quǽsumus, Dómine, supplicatiónibus nostris: ut esse, te largiénte, mereámur et inter próspera húmiles, et inter advérsa secúri. Per Dóminum \emph{\&c.}
}\switchcolumn\portugues{
\rlettrine{S}{enhor,} Vos pedimos, dignai-Vos atender às nossas súplicas, para que, por efeito da vossa graça, possamos ser humildes nas prosperidades e confiantes nas contrariedades. Por nosso Senhor \emph{\&c.}
}\end{paracol}

\paragraphinfo{3.ª Lição}{2 Mac. l, 23-26 et 27}
\begin{paracol}{2}\latim{
Léctio libri Machabæórum.
}\switchcolumn\portugues{
Lição do Livro dos Macabeus.
}\switchcolumn*\latim{
\rlettrine{I}{n} diébus illis: Oratiónem faciebant omnes sacerdotes, dum consummarétur sacrifícium, Jónatha inchoánte, céteris autem respondéntibus. Et Nehemíæ erat orátio hunc habens modum: Dómine Deus, ómnium Creátor, terríbilis et fortis, justus et miséricors, qui solus es bonus rex, solus præstans, solus justus et omnípotens et ætérnus, qui líberas Israël de omni malo, qui fecísti patres electos et sanctificásti eos: accipe sacrifícium pro univérso pópulo tuo Israël, et custódi partem tuam et sanctífica: ut sciant gentes, quia tu es Deus noster.
}\switchcolumn\portugues{
\rlettrine{N}{aqueles} dias, enquanto se consumia o sacrifício das vítimas, todos os sacerdotes oravam, principiando Jónatas e respondendo os outros. E a oração de Nehemias era assim: «Senhor Deus, criador de todas as cousas, temível e forte, justo e misericordioso: só Vós sois rei suave; rei excelente; rei justo, omnipotente e eterno! Vós, que livrastes Israel de todo o mal; Vós, que escolhestes entre os outros homens os nossos pais e os santificastes, aceitai este sacrifício em nome do vosso povo de Israel e conservai e santificai a vossa herança, para que todos os povos conheçam que sois nosso Deus».
}\end{paracol}

\paragraphinfo{Gradual}{Sl. 89, 13 \& 1}
\begin{paracol}{2}\latim{
\rlettrine{C}{onvértere,} Dómine, aliquántulum, et deprecáre super servos tuos. ℣. Dómine, refúgium factus es nobis, a generatióne et progénie.
}\switchcolumn\portugues{
\rlettrine{V}{inde} a nós quanto antes, Senhor, e deixai-Vos aplacar com as preces dos vossos servos. ℣. Senhor, tendes sido o nosso refúgio de geração em geração.
}\end{paracol}

\begin{paracol}{2}\latim{
\begin{nscenter} Orémus. Flectámus génua. \end{nscenter}
}\switchcolumn\portugues{
\begin{nscenter} Oremos. Ajoelhemos! \end{nscenter}
}\switchcolumn*\latim{
℟. Leváte.
}\switchcolumn\portugues{
℟. Levantai-vos!
}\end{paracol}

\paragraph{Oração}
\begin{paracol}{2}\latim{
\rlettrine{P}{reces} pópuli tui, quǽsumus, Dómine, cleménter exáudi: ut, qui juste pro peccátis nostris afflígimur, pro tui nóminis glória misericórditer liberémur. Per Dóminum nostrum \emph{\&c.}
}\switchcolumn\portugues{
\rlettrine{O}{uvi} benigno, Senhor, Vos suplicamos, as preces do vosso povo, a fim de que nós, que fomos justamente castigados por causa dos nossos pecados, sejamos misericordiosamente livres pela glória do vosso nome. Por nosso Senhor \emph{\&c.}
}\end{paracol}

\paragraphinfo{4.ª Lição}{Ecl. 36, 1-10}
\begin{paracol}{2}\latim{
Léctio libri Sapientiae.
}\switchcolumn\portugues{
Lição do Livro da Sabedoria.
}\switchcolumn*\latim{
\rlettrine{M}{iserére} nostri, Deus ómnium, et réspice nos, et osténde nobis lucem miseratiónum tuárum: et immítte timórem tuum super gentes, quæ non exquisiérunt te, ut cognóscant, quia non est Deus nisi tu, et enárrent magnália tua. Alleva manum tuam super gentes aliénas, ut vídeant poténtiam tuam. Sicut enim in conspéctu eórum sanctificátus es in nobis, sic in conspéctu nostro magnificáberis in eis, ut cognóscant te, sicut et nos cognóvimus, quóniam non est Deus præter te, Dómine. Innova signa et immúta mirabília. Glorífica manum et bráchium dextrum. Excita furórem et effúnde iram. Tolle adversárium et afflíge inimícum. Festína tempus et meménto finis, ut enárrent mirabília tua, Dómine, Deus noster.
}\switchcolumn\portugues{
\rlettrine{T}{ende} piedade de nós, ó Deus de todos os viventes, lançai benigno os vossos olhares sobre nós e fazei resplandecer a nossos olhos a luz das vossas misericórdias! Espalhai o vosso temor nos povos que se não aproximam de Vós, para que conheçam que não existe senão um só Deus, que sois Vós, e publiquem as vossas grandezas. Erguei a vossa mão contra os povos estrangeiros, para que conheçam o vosso poder; porquanto, assim como lhes mostrastes a vossa santidade, castigando-nos pelas nossas faltas, assim também nos mostreis a vossa grandeza, punindo-os, a fim de que conheçam, como nós, que não há outro Deus senão Vós, Senhor. Repeti os milagres; renovai as maravilhas; glorificai a vossa mão e o vosso braço direito. Excitai a vossa indignação; manifestai a vossa ira; destruí o adversário; importunai o inimigo. Apressai o tempo, para que chegue depressa o fim e se publiquem as vossas maravilhas, ó Senhor, nosso Deus.
}\end{paracol}

\paragraphinfo{Gradual}{Sl. 140, 2}
\begin{paracol}{2}\latim{
\rlettrine{D}{irigátur} orátio mea sicut incénsum in conspéctu tuo, Dómine. ℣. Elevátio mánuum meárum sacrifícium vespertínum.
}\switchcolumn\portugues{
\qlettrine{Q}{ue} a minha oração suba até à vossa presença, como incenso, Senhor! ℣. E que o erguer das minhas mãos Vos seja agradável, como o sacrifício da tarde.
}\end{paracol}

\begin{paracol}{2}\latim{
\begin{nscenter} Orémus. Flectámus génua. \end{nscenter}
}\switchcolumn\portugues{
\begin{nscenter} Oremos. Ajoelhemos! \end{nscenter}
}\switchcolumn*\latim{
℟. Leváte.
}\switchcolumn\portugues{
℟. Levantai-vos!
}\end{paracol}

\paragraph{Oração}
\begin{paracol}{2}\latim{
\rlettrine{A}{ctiónes} nostras, quǽsumus, Dómine, aspirándo prǽveni, et adjuvándo proséquere: ut cuncta nostra orátio et operátio a te semper incípiat, et per te cœpta finiátur. Per Dóminum \emph{\&c.}
}\switchcolumn\portugues{
\rlettrine{D}{ignai-Vos,} Senhor, insuflar as nossas acções com vosso espírito e acompanhá-las com vossa graça, a fim de que as nossas orações e obras, tendo princípio em Deus, tenham também n’Ele sua finalidade. Por nosso Senhor \emph{\&c.}
}\end{paracol}

\paragraphinfo{5.ª Lição}{Dn. 3, 47-51}
\begin{paracol}{2}\latim{
Léctio Daniélis Prophétæ.
}\switchcolumn\portugues{
Lição do Profeta Daniel.
}\switchcolumn*\latim{
\rlettrine{I}{n} diébus illis: Angelus Dómini descéndit cum Azaría et sóciis ejus in fornácem: et excússit flammam ignis de fornáce, et fecit médium fornácis quasi ventum roris flantem. Flamma autem effundebátur super fornácem cúbitis quadragínta novem: et erúpit, et incéndit, quos répperit juxta fornácem de Chaldǽis, minístros regis, qui eam incendébant. Et non tétigit eos omníno ignis, neque contristavit, nec quidquam moléstiæ íntulit. Tunc hi tres quasi ex uno ore lau-dábant, et glorificábant, et benedicébant Deum in fornáce, dicéntes:
}\switchcolumn\portugues{
\rlettrine{N}{aqueles} dias, o Anjo do Senhor desceu à fornalha com Azarias e os seus companheiros e afastou da fornalha as chamas do fogo, soprando no meio delas como que um vento de orvalho. As chamas do fogo, porém, cresciam acima da fornalha quarenta e nove côvados; e, irrompendo fora dela, queimaram os Caldeus, ministros do rei, que estavam perto da fornalha e atiçavam o fogo; e não queimaram nenhum dos três jovens hebreus, nem os feriram, nem lhes causaram qualquer incómodo! Então, estes três jovens louvavam, glorificavam e bendiziam Deus na fornalha, em voz uníssona, dizendo:
}\end{paracol}

\paragraphinfo{Hino Benedictus Es}{Página \pageref{benedictuses}}

\paragraph{Oração}
\begin{paracol}{2}\latim{
\rlettrine{D}{eus,} qui tribus púeris mitigásti flammas ígnium: concéde propítius; ut nos fámulos tuos non exúrat flamma vitiórum. Per Dóminum nostrum \emph{\&c.}
}\switchcolumn\portugues{
\slettrine{Ó}{} Deus, que mitigastes as chamas do fogo aos três jovens, concedei, misericordiosamente, a nós, que somos vossos servos, que não sejamos queimados pelas chamas dos vícios. Por nosso Senhor \emph{\&c.}
}\end{paracol}

\paragraphinfo{Epístola}{1 Ts. 5, 14-23}
\begin{paracol}{2}\latim{
Léctio Epístolæ beáti Pauli Apóstoli ad Thessalonicénses.
}\switchcolumn\portugues{
Lição da Ep.ª do B. Ap.º Paulo aos Tessalonicenses.
}\switchcolumn*\latim{
\rlettrine{F}{ratres:} Rogámus vos, corrípite inquiétos, consolámini pusillánimes, suscípite infirmos, patiéntes estóte ad omnes. Vidéte, ne quis malum pro malo alicui reddat: sed semper quod bonum est sectámini in ínvicem, et in omnes. Semper gaudéte. Sine intermissióne oráte. In ómnibus grátias ágite: hæc est enim volúntas Dei in Christo Jesu in ómnibus vobis. Spíritum nolíte exstínguere. Prophetías nolíte spérnere. Omnia autem probáte: quod bonum est tenéte. Ab omni spécie mala abstinéte vos. Ipse autem Deus pacis sanctíficet vos per ómnia: ut ínteger spíritus vester, et ánima, et corpus sine queréla, in advéntu Dómini nostri Jesu Christi servétur.
}\switchcolumn\portugues{
\rlettrine{M}{eus} rmãos, vos pedimos, admoestai os que perturbam a ordem; confortai os pusilânimes; amparai os fracos; sede pacientes para com todos. Reparai que ninguém retribua com o mal o mal que recebe de outrem; mas procurai sempre com ardor que os vossos irmãos pratiquem o bem uns para com os outros e para com todos. Regozijai-vos sempre. Orai sem cessar. Em todas as cousas dai graças a Deus, pois esta é a vontade de Deus, em Jesus Cristo, a vosso respeito. Não apagueis o Espírito, nem desprezeis as Profecias. Examinai todas as cousas e conservai o que é bom. Abstende-vos de toda a espécie do mal. Que o mesmo Deus de paz vos santifique em todas as cousas, a fim de que o vosso espírito, alma e corpo se conservem irrepreensíveis até à vinda de N. S. Jesus Cristo.
}\end{paracol}

\paragraphinfo{Trato}{Sl. 116, 1-2}
\begin{paracol}{2}\latim{
\rlettrine{L}{audáte} Dóminum, omnes gentes: et collaudáte eum, omnes pópuli. ℣. Quóniam confirmáta est super nos misericórdia ejus: et véritas Dómini manet in ætérnum.
}\switchcolumn\portugues{
\qlettrine{Q}{ue} todas as nações louvem o Senhor; que todos os povos O glorifiquem: ℣. Porque a sua misericórdia é infinita para connosco e a verdade do Senhor permanece eternamente.
}\end{paracol}

\paragraphinfo{Evangelho}{Página \pageref{2domingoquaresma}}

\paragraphinfo{Ofertório}{Sl. 87, 2-3}
\begin{paracol}{2}\latim{
\rlettrine{D}{ómine,} Deus salútis meæ, in die clamávi et nocte coram te: intret orátio mea in conspéctu tuo, Dómine.
}\switchcolumn\portugues{
\rlettrine{S}{enhor,} meu Deus e meu Salvador, clamo dia e noite diante de Vós! Que minha oração se eleve até à vossa presença.
}\end{paracol}

\paragraph{Secreta}
\begin{paracol}{2}\latim{
\rlettrine{P}{ræséntibus} sacrifíciis, quǽsumus, Dómine, jejúnia nostra sanctífica: ut, quod observántia nostra profitétur extrínsecus, intérius operétur. Per Dóminum \emph{\&c.}
}\switchcolumn\portugues{
\rlettrine{S}{antificai,} Senhor, Vos suplicamos, os nossos jejuns pelo presente sacrifício, a fim de que esta observância exterior produza efeitos espirituais interiores. Por nosso Senhor \emph{\&c.}
}\end{paracol}

\paragraphinfo{Comúnio}{Sl. 7, 2}
\begin{paracol}{2}\latim{
\rlettrine{D}{ómine,} Deus meus, in te sperávi: líbera me ab ómnibus persequéntibus me, et éripe me.
}\switchcolumn\portugues{
\rlettrine{S}{enhor,} meu Deus, em Vós esperei; salvai-me e livrai-me de todos meus perseguidores.
}\end{paracol}

\paragraph{Postcomúnio}
\begin{paracol}{2}\latim{
\rlettrine{S}{anctificatiónibus} tuis, omnípotens Deus, et vítia nostra curéntur, et remédia nobis ætérna provéniant. Per Dóminum \emph{\&c.}
}\switchcolumn\portugues{
\slettrine{Ó}{} omnipotente Deus, permiti que os nossos vícios sejam curados pela virtude deste vosso sacrifício e concedei-nos o remédio que produz a salvação eterna. Por nosso Senhor \emph{\&c.}
}\end{paracol}

\paragraph{Oração sobre o povo}
\begin{paracol}{2}\latim{
\begin{nscenter} Orémus. \end{nscenter}
}\switchcolumn\portugues{
\begin{nscenter} Oremos. \end{nscenter}
}\switchcolumn*\latim{
Humiliáte cápita vestra Deo.
}\switchcolumn\portugues{
Inclinai as vossas cabeças diante de Deus.
}\switchcolumn*\latim{
Fidéles tuos, Deus, benedíctio desideráta confírmet: quæ eos et a tua voluntáte numquam fáciat discrepáre, et tuis semper indúlgeat benefíciis gratulári. Per Dóminum \emph{\&c.}
}\switchcolumn\portugues{
Ó Deus, fortificai os fiéis com vossa bênção, que eles imploram; e que, por efeito dela, nunca se afastem da vossa vontade e lhes alcance a graça de se alegrarem sempre com vossos benefícios. Por nosso Senhor \emph{\&c.}
}\end{paracol}

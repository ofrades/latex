\subsectioninfo{Sábado depois das Cinzas}{Estação em S. Trifão}

\paragraphinfo{Intróito, Gradual, Trato, Ofertório e Comúnio}{Página \pageref{sextafeiradepoiscinzas}}

\paragraph{Oração}
\begin{paracol}{2}\latim{
\rlettrine{A}{désto,} Dómine, supplicatiónibus nostris: et concéde; ut hoc sollémne jejúnium, quod animábus corporibúsque curándis salúbriter institútum est, devóto servítio celebrémus. Per Dóminum \emph{\&c.}
}\switchcolumn\portugues{
\rlettrine{A}{tendei,} Senhor, às nossas súplicas e concedei-nos a graça de celebrarmos como servos
devotos este solene jejum, que foi salutarmente instituído para curar as nossas almas e os nossos corpos. Por nosso Senhor \emph{\&c.}
}\end{paracol}

\paragraphinfo{Epístola}{Is. 58, 9-14}
\begin{paracol}{2}\latim{
Léctio Isaíæ Prophétæ.
}\switchcolumn\portugues{
Lição do Profeta Isaías.
}\switchcolumn*\latim{
\rlettrine{H}{æc} dicit Dóminus Deus: Si abstúleris de médio tui caténam, et desíeris exténdere dígitum, et loqui quod non prodest. Cum effúderis esuriénti ánimam tuam, et ánimam afflíctam repléveris, oriétur in ténebris lux tua, et ténebræ tuæ erunt sicut merídies. Et réquiem tibi dabit Dóminus semper, et implébit splendóribus ánimam tuam, et ossa tua liberábit, et eris quasi hortus irríguus, et sicut fons aquárum, cujus non defícient aquæ. Et ædificabúntur in te desérta sæculórum: fundaménta generatiónis et generatiónis suscitábis: et vocáberis ædificátor sépium, avértens sémitas in quiétem. Si avérteris a sábbato pedem tuum, fácere voluntátem tuam in die sancto meo, et vocáveris sábbatum delicátum, et sanctum Dómini gloriósum, et glorificáveris eum, dum non facis vias tuas, et non invénitur volúntas tua, ut loquáris sermónem: tunc delectáberis super Dómino: et sustóllam te super altitúdines terræ, et cibábo te hereditáte Jacob, patris tui. Os enim Dómini locútum est.
}\switchcolumn\portugues{
\rlettrine{A}{ssim} fala o Senhor Deus: «Se acabar no meio de vós o jugo; se cessardes de estender o dedo e de falar em cousas que não são proveitosas; se assistirdes ao faminto com o carinho da vossa alma e consolardes o coração aflito: então nascerá nas trevas a vossa luz e as vossas trevas brilharão, como o sol ao meio-dia. E o Senhor vos dará repouso para sempre; encherá a vossa alma com seus esplendores; livrará os vossos ossos da corrupção; e sereis como um jardim regado e como uma fonte abundante, cujas águas nunca mais se esgotarão. Por vós serão edificados os lugares desertos; levantados os alicerces, postos pelas gerações passadas, que estavam abandonados; dir-se-á que reparastes os muros e tornastes os caminhos em lugares de paz. Se deixardes em descanso os vossos pés ao sábado; se não fizerdes a vossa vontade no meu dia sagrado; se chamardes ao sábado dia agradável, como dia santo e glorioso do Senhor, e o glorificardes, não seguindo as vossas inclinações, nem vos entregando aos vossos negócios e às vossas palavras insensatas: então encontrareis as delícias no Senhor e sereis erguidos às culminâncias da terra, e vos darei o gozo da herança de vosso pai Jacob, pois a palavra do Senhor o prometeu».
}\end{paracol}

\paragraphinfo{Evangelho}{Mc. 6, 47-56}
\begin{paracol}{2}\latim{
\cruz Sequéntia sancti Evangélii secúndum Marcum.
}\switchcolumn\portugues{
\cruz Continuação do santo Evangelho segundo S. Marcos.
}\switchcolumn*\latim{
\blettrine{I}{n} illo témpore: Cum sero esset, erat navis in médio mari, et Jesus solus in terra. Et videns discípulos suos laborántes in remigándo (erat enim ventus contrárius eis), et circa quartam vigíliam noctis venit ad eos ámbulans supra mare: et volébat præteríre eos. At illi, ut vidérunt eum ambulántem supra mare, putavérunt phantásma esse, et exclamavérunt. Omnes enim vidérunt eum, et conturbáti sunt. Et statim locútus est cum eis, et dixit eis: Confídite, ego sum, nolíte timére. Et ascéndit ad illos in navim, et cessávit ventus. Et plus magis intra se stupébant: non enim intellexérunt de pánibus: erat enim cor eórum obcæcátum. Et cum transfretássent, venérunt in terram Genésareth, et applicuérunt. Cumque egréssi essent de navi, contínuo cognovérunt eum: et percurréntes univérsam regiónem illam, cœpérunt in grabátis eos, qui se male habébant, circumférre ubi audiébant eum esse. Et quocúmque introíbat, in vicos vel in villas aut civitátes, in platéis ponébant infírmos, et deprecabántur eum, ut vel fímbriam vestiménti ejus tángerent: et quotquot tangébant eum, salvi fiébant.
}\switchcolumn\portugues{
\blettrine{N}{aquele} tempo, sendo quase noite, a barca estava no meio do mar; e Jesus estava, só, em terra. Vendo Ele, então, os seus discípulos fatigados de remar (pois o vento era contrário) cerca da quarta vigília da noite, veio a eles, caminhando por cima do mar e querendo ultrapassá-los • Vendo-O os discípulos a andar no mar, julgaram que era um fantasma e deram gritos de pavor, pois todos O viram e se perturbaram. Logo, Jesus lhes falou e disse: «Tende confiança, sou eu; não tenhais nenhum receio». Em seguida subiu para a barca, junto deles, e logo o vento cessou. E cada vez mais se admiravam intimamente, porquanto não haviam compreendido o milagre dos pães, pois o seu coração estava obcecado. Entretanto, tendo passado para a outra banda, vieram à terra de Genesaré, onde abordaram. Logo que saíram da barca, conheceram Jesus. Então percorreram toda aquela região; e começaram a trazer nas camas os que se achavam doentes, onde quer que Ele estava. E em qualquer lugar em que Ele entrava (aldeias, vilas ou cidades) punham nas praças os enfermos e suplicavam-Lhe que ao menos lhes deixasse tocar na orla do seu vestido. E todos aqueles que Lhe tocavam eram curados.
}\end{paracol}

\paragraph{Secreta}
\begin{paracol}{2}\latim{
\rlettrine{S}{úscipe,} Dómine, sacrifícium, cujus te voluísti dignánter immolatióne placári: præsta, quǽsumus; ut, hujus operatióne mundáti, beneplácitum tibi nostræ mentis offerámus afféctum. Per Dóminum \emph{\&c.}
}\switchcolumn\portugues{
\rlettrine{A}{ceitai,} Senhor, este sacrifício, cuja imolação quisestes que possuísse a virtude de aplacar-Vos, e permiti, Vos suplicamos, que, purificados pela sua virtude, queirais aceitar o afecto do nosso coração, como uma oblação agradável. Por nosso Senhor \emph{\&c.}
}\end{paracol}

\paragraph{Postcomúnio}
\begin{paracol}{2}\latim{
\rlettrine{C}{œléstis} vitæ múnere vegetáti, quǽsumus, Dómine: ut, quod est nobis in præsénti vita mystérium, fiat æternitátis auxílium. Per Dóminum nostrum \emph{\&c.}
}\switchcolumn\portugues{
\rlettrine{A}{gora,} que fomos alimentados com o pão da vida celestial, Vos suplicamos, Senhor, permiti que este dom, que é para nós mystério nesta vida, seja nosso auxílio na eternidade. Por nosso Senhor \emph{\&c.}
}\end{paracol}

\paragraph{Oração sobre o povo}
\begin{paracol}{2}\latim{
\begin{nscenter} Orémus. \end{nscenter}
}\switchcolumn\portugues{
\begin{nscenter} Oremos. \end{nscenter}
}\switchcolumn*\latim{
Humiliáte cápita vestra Deo.
}\switchcolumn\portugues{
Inclinai as vossas cabeças diante de Deus.
}\switchcolumn*\latim{
Fidéles tui, Deus, per tua dona firméntur: ut éadem et percipiéndo requírant, et quæréndo sine fine percípiant. Per Dóminum \emph{\&c.}
}\switchcolumn\portugues{
Que os vossos fiéis, ó Deus, sejam fortificados com vossos dons, a fim de que, recebendo-os, os procurem, e, procurando-os, os alcancem perpetuamente. Por nosso Senhor \emph{\&c.}
}\end{paracol}

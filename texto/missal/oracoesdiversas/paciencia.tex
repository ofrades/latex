\subsection{28.ª Para pedir a paciência}\label{paciencia}

\paragraph{Oração}
\begin{paracol}{2}\latim{
\rlettrine{D}{eus,} qui Unigéniti tui patiéntia antíqui hostis contrivísti supérbiam: da nobis, quǽsumus, quæ idem pie pro nobis pértulit, digne recólere; sicque, exémplo ejus, nobis adversántia æquanímiter toleráre. Per eúndem Dóminum nostrum \emph{\&c.}
}\switchcolumn\portugues{
\slettrine{Ó}{} Deus, que, em virtude da paciência que o vosso Filho Unigénito praticou, esmagastes o «homem antigo», permiti, Vos suplicamos, que meditemos gravemente em tudo quanto Jesus sofreu com tanta bondade por nós, a fim de que, sustentados com seu exemplo, possamos suportar com resignação as adversidades. Por nosso Senhor \emph{\&c.}
}\end{paracol}

\paragraph{Secreta}
\begin{paracol}{2}\latim{
\rlettrine{M}{únera} nostræ oblatiónis, quǽsumus, Dómine, súscipe placátus: quæ, ut nobis patiéntiæ donum largíri dignéris, majestáti tuæ devota offérimus actióne. Per Dóminum nostrum \emph{\&c.}
}\switchcolumn\portugues{
\rlettrine{S}{enhor,} Vos suplicamos, recebei os dons da nossa oblação, e deixai-Vos aplacar; oferecemos devotadamente estes dons à vossa majestade, para que Vos digneis conceder-nos a virtude da paciência. Por nosso Senhor \emph{\&c.}
}\end{paracol}

\paragraph{Postcomúnio}
\begin{paracol}{2}\latim{
\rlettrine{M}{ystéria,} Dómine, sacrosáncta, quæ súmpsimus, amíssam nobis, quǽsumus, reconcílient grátiam: atque munus patiéntiæ in illátis ómnibus, semper et ubíque protegéndo, impértiant. Per Dóminum nostrum \emph{\&c.}
}\switchcolumn\portugues{
\rlettrine{S}{enhor,} Vos imploramos, permiti que os sacrossantos mistérios que recebemos nos restaurem na graça, que havíamos perdido, e que, fazendo-nos sentir a vossa protecção, nos concedam sempre e em toda a parte o dom da paciência em todas as adversidades. Por nosso Senhor \emph{\&c.}
}\end{paracol}
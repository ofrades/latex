\subsection{23.ª Pelos penitentes públicos}\label{penitentespublicos}

\paragraph{Oração}
\begin{paracol}{2}\latim{
\rlettrine{O}{mnípotens} sempitérne Deus, confidéntibus tibi fámulis tuis pro tua pietáte reláxa peccáta: ut non ámplius eis nóceat consciéntias reátus ad pœnam, quam indulgéntia tuæ propitiatiónis prosit ad véniam. Per Dóminum \emph{\&c.}
}\switchcolumn\portugues{
\slettrine{Ó}{} Deus omnipotente e sempiterno, dignai-Vos misericordiosamente conceder a remissão dos pecados a estes vossos servos, que são pecadores confessos, a fim de que a culpa, contraída pela consciência, lhes não seja mais perniciosa pela pena em que incorreram do que a vossa misericórdia lhes foi útil para o perdão. Por nosso Senhor \emph{\&c.}
}\end{paracol}

\paragraph{Secreta}
\begin{paracol}{2}\latim{
\rlettrine{P}{ræsta,} quǽsumus, omnípotens et miséricors Deus: ut hæc salutáris oblátio fámulos tuos et a própriis reátibus indesinénter expédiat, et ab ómnibus tueátur advérsis. Per Dóminum \emph{\&c.}
}\switchcolumn\portugues{
\rlettrine{S}{enhor} omnipotente e misericordioso, permiti que esta hóstia de salvação purifique inteiramente os vossos servos das culpas que contraíram e os proteja contra tudo o que lhes seja nocivo. Por nosso Senhor \emph{\&c.}
}\end{paracol}

\paragraph{Postcomúnio}
\begin{paracol}{2}\latim{
\rlettrine{O}{mnípotens} et miséricors Deus, qui omnem ánimam pœniténtem et confiténtem tibi magis vis emendáre, quam pérdere: réspice super hos fámulos tuos; et per hæc sancta sacraménta, quæ súmpsimus, avérte ab eis iram indignatiónis tuæ, et ómnia eórum peccáta dimítte. Per Dóminum nostrum \emph{\&c.}
}\switchcolumn\portugues{
\rlettrine{D}{eus} omnipotente e misericordioso, que quereis não a perda mas a conversão da alma penitente, que confessa as suas faltas, dignai-Vos lançar vossos olhares para estes vossos servos; e, pela virtude destes sacramentos, que acabamos de receber, afastai de cima das suas cabeças a ira da vossa indignação e perdoai-lhes os pecados. Por nosso Senhor \emph{\&c.}
}\end{paracol}
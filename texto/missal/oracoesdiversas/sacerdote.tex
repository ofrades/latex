\subsection{20.ª Pelo próprio Sacerdote}\label{sacerdote}

\paragraph{Oração}
\begin{paracol}{2}\latim{
\rlettrine{O}{mnípotens} et miséricors Deus, humilitátis meæ preces benígnus inténde: ei me fámulum tuum, quem, nullis suffragántibus méritis, sed imménsa cleméntiæ tuæ largitáte cœléstibus mystériis servíre tribuísti, dignum sacris altáribus fac minístrum; ut, quod mea voce deprómitur, tua sanctificatióne firmétur. Per Dóminum \emph{\&c.}
}\switchcolumn\portugues{
\rlettrine{O}{mnipotente} e misericordioso Deus, atendei benigno às preces que humildemente Vos dirijo, e tornai digno ministro dos vossos altares sagrados este vosso servo (que elevastes ao ministério dos dons celestes, não pelos seus próprios méritos, mas pela vossa imensa clemência), a fim de que as palavras que saem da minha boca sejam por Vós confirmadas e santificadas. Por nosso Senhor \emph{\&c.}
}\end{paracol}

\paragraph{Secreta}
\begin{paracol}{2}\latim{
\rlettrine{H}{ujus,} Dómine, virtúte sacraménti, peccatórum meórum máculas abstérge: et præsta; ut, ad exsequéndum injúncti officii ministérium, me tua grátia dignum effíciat. Per Dóminum \emph{\&c.}
}\switchcolumn\portugues{
\rlettrine{S}{enhor,} pela virtude deste mistério, purificai-me das máculas dos meus pecados; e, Vos suplico insistenternente, pela vossa graça, tornai-me digno das sagradas funções do ministério que me foi imposto. Por nosso Senhor \emph{\&c.}
}\end{paracol}

\paragraph{Postcomúnio}
\begin{paracol}{2}\latim{
\rlettrine{O}{mnípotens} sempitérne Deus, qui me peccatórem sacris altáribus astáre voluísti, et sancti nóminis tui laudáre poténtiam: concéde propítius, per hujus sacraménti mystérium, meórum mihi véniam peccatórum; ut tuæ majestáti digne mérear famulári. Per Dóminum \emph{\&c.}
}\switchcolumn\portugues{
\rlettrine{D}{eus} omnipotente e sempiterno, que destinastes este indigno pecador para servir os vossos sacrossantos altares e louvar a majestade do vosso santo nome, concedei-me misericordiosa-mente, pelo ministério deste sacrifício, a remissão dos meus pecados, para que possa dignamente servir a vossa majestade. Por nosso Senhor \emph{\&c.}
}\end{paracol}
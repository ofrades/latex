\subsection{5.ª Pelo Imperador}\label{imperador}

\paragraph{Oração}
\begin{paracol}{2}\latim{
\rlettrine{D}{eus,} regnórum ómnium, et christiáni maxime protéctor impérii: da servo tuo imperatóri nostro {\redx N.} triúmphum virtútis tuæ sciénter excólere; ut, qui tua institutióne est princeps, tuo sit semper múnere potens. Per Dóminum nostrum \emph{\&c.}
}\switchcolumn\portugues{
\slettrine{Ó}{} Deus, que sois o protector de todos os reinos e principalmente do império cristão, concedei ao imperador {\redx N.} vosso servo, que governe sempre com sabedoria, para o triunfo do vosso poder, a fim de que, sendo príncipe em virtude da vossa instituição, seja sempre poderoso em virtude da vossa graça. Por nosso Senhor \emph{\&c.}
}\end{paracol}

\paragraph{Secreta}
\begin{paracol}{2}\latim{
\rlettrine{S}{úscipe,} Dómine, preces et hóstias Ecclésiæ tuæ, pro salúte fámuli tui supplicántis: et in protectióne fidélium populórum antiqua bráchii tui operáre mirácula; ut, superátis pacis inimícis, secúra tibi sérviat christiána libértas. Per Dóminum \emph{\&c.}
}\switchcolumn\portugues{
\rlettrine{R}{ecebei,} Senhor, as preces e as hóstias da vossa Igreja em favor da salvação do vosso servo suplicante, e operai os costumados prodígios do vosso poder para proteger os povos fiéis, a fim de que, sendo vencidos os inimigos da paz, a liberdade cristã permita que Vos possam servir com segurança. Por nosso Senhor \emph{\&c.}
}\end{paracol}

\paragraph{Postcomúnio}
\begin{paracol}{2}\latim{
\rlettrine{D}{eus,} qui ad prædicándum ætérni Regis Evangélium, Románum impérium præparásti: præténde fámulo tuo imperatóri nostro {\redx N.} arma cœléstia; ut pax ecclesiárum nulla turbátur tempestáte bellórum. Per Dóminum \emph{\&c.}
}\switchcolumn\portugues{
\slettrine{Ó}{} Deus, que predestinastes o império romano para pregar o Evangelho do eterno Rei, entregai ao imperador {\redx N.}, vosso servo, as armas celestiais, para que a paz das igrejas não seja perturbada pela tempestade feroz das guerras. Por nosso Senhor \emph{\&c.}
}\end{paracol}
\subsection{22.ª Para obter perdão dos pecados}\label{perdaopecados}

\paragraph{Oração}
\begin{paracol}{2}\latim{
\rlettrine{D}{eus,} qui nullum réspuis, sed, quantúmvis peccántibus, per pœniténtiam pia miseratióne placáris: réspice propítius ad preces humilitátis nostræ, et illúmina corda nostra; ut tua valeámus implére præcépta. Per Dóminum nostrum \emph{\&c.}
}\switchcolumn\portugues{
\slettrine{Ó}{} Deus, que não repelis homem algum, mas antes, em vossa misericordiosa bondade, Vos deixais aplacar pela penitência dos pecadores, por mais que Vos tenham ofendido, aceitai benigno as nossas humildes orações e iluminai os nossos corações, para que possamos cumprir os vossos preceitos. Por nosso Senhor \emph{\&c.}
}\end{paracol}

\paragraph{Secreta}
\begin{paracol}{2}\latim{
\rlettrine{P}{ræsens} sacrifícium, Dómine, quod tibi pro delíctis nostris offérimus, sit tibi munus accéptum: et tam vivéntibus quam defúnctis profíciat ad salútem. Per Dóminum \emph{\&c.}
}\switchcolumn\portugues{
\qlettrine{Q}{ue} este sacrifício, que Vos oferecemos em reparação dos nossos pecados, Vos seja agradável, Senhor, e que, tanto aos vivos como aos mortos, seja proveitoso para a sua salvação. Por nosso Senhor \emph{\&c.}
}\end{paracol}

\paragraph{Postcomúnio}
\begin{paracol}{2}\latim{
\rlettrine{E}{xáudi} preces famíliæ tuæ, omnípotens Deus: et præsta; ut sancta hæc, quæ a te súmpsimus, incorrúpta in nobis, te donánte, servántur. Per Dóminum \emph{\&c.}
}\switchcolumn\portugues{
\slettrine{Ó}{} Deus omnipotente, ouvi as preces da vossa família; e pela vossa graça concedei-nos, Vos suplicamos, que estes sacrossantos mistérios, que recebemos de vossas mãos, não sejam manchados no nosso íntimo. Por nosso Senhor \emph{\&c.}
}\end{paracol}

\subsection{10.ª Contra os perseguidores da Igreja}\label{perseguidoresigreja}

\paragraph{Oração}
\begin{paracol}{2}\latim{
\rlettrine{E}{cclésiæ} tuæ, quǽsumus, Dómine, preces placátus admítte: ut, destrúctis adversitátibus et erróribus univérsis, secúra tibi sérviat libertáte. Per Dóminum \emph{\&c.}
}\switchcolumn\portugues{
\rlettrine{S}{enhor,} Vos suplicamos, dignai-Vos acolher benigno as preces da vossa Igreja, para que, destruídas todas as adversidades e todos os obstáculos, ela Vos sirva com liberdade e segurança. Por nosso Senhor \emph{\&c.}
}\end{paracol}

\paragraph{Secreta}
\begin{paracol}{2}\latim{
\rlettrine{P}{rótege} nos, Dómine, tuis mystériis serviéntes: ut, divinis rebus inhæréntes, et córpore tibi famulémur et mente. Per Dóminum \emph{\&c.}
}\switchcolumn\portugues{
\rlettrine{A}{} nós, que celebramos os vossos mistérios, protegei-nos, Senhor, a fim de que, unindo-nos aos mistérios divinos, Vos sirvamos com o corpo e com a alma. Por nosso Senhor \emph{\&c.}
}\end{paracol}

\paragraph{Postcomúnio}
\begin{paracol}{2}\latim{
\qlettrine{Q}{uǽsumus,} Dómine, Deus noster: ut, quos divína tríbuis participatióne gaudére, humánis non sinas subjacére perículis. Per Dóminum nostrum \emph{\&c.}
}\switchcolumn\portugues{
\slettrine{Ó}{} Senhor, nosso Deus, Vos pedimos, não consintais que aqueles a quem concedestes a graça de participar do divino banquete sejam expostos aos perigos que ameaçam os homens. Por nosso Senhor \emph{\&c.}
}\end{paracol}
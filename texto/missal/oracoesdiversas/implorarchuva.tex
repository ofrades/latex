\subsection{16.ª Para implorar a chuva}\label{implorarchuva}

\paragraph{Oração}
\begin{paracol}{2}\latim{
\rlettrine{D}{eus,} in quo vívimus, movémur et sumus: plúviam nobis tríbue congruéntem; ut, præséntibus subsídiis sufficiénter adjuti, sempitérna fiduciálius appetámus. Per Dóminum nostrum \emph{\&c.}
}\switchcolumn\portugues{
\slettrine{Ó}{} Deus, em quem nos movemos, vivemos e existimos, concedei-nos a chuva necessária, para que, auxiliados suficientemente com os subsídios temporais, aspiremos com mais confiança às coisas eternas. Por nosso Senhor \emph{\&c.}
}\end{paracol}

\paragraph{Secreta}
\begin{paracol}{2}\latim{
\rlettrine{O}{blátis,} quǽsumus, Dómine, placáre munéribus: et opportúnum nobis tríbue plúviæ sufficiéntis auxílium. Per Dóminum \emph{\&c.}
}\switchcolumn\portugues{
\rlettrine{S}{enhor,} Vos imploramos, deixai-Vos aplacar com estas ofertas e concedei-nos o auxílio da chuva, segundo as necessidades presentes. Por nosso Senhor \emph{\&c.}
}\end{paracol}

\paragraph{Postcomúnio}
\begin{paracol}{2}\latim{
\rlettrine{D}{a} nobis, quǽsumus, Dómine, plúviam salutárem: et áridam terræ fáciem fluéntis cœléstibus dignánter infúnde. Per Dóminum \emph{\&c.}
}\switchcolumn\portugues{
\rlettrine{D}{ai-nos,} Senhor, Vos pedimos, a chuva salutar; e espalhai misericordiosamente as águas do céu pela superfície seca da terra. Por nosso Senhor \emph{\&c.}
}\end{paracol}
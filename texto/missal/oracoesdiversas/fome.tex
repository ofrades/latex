\subsection{14.ª Para os tempos de fome}\label{fome}

\paragraph{Oração}
\begin{paracol}{2}\latim{
\rlettrine{D}{a} nobis, quǽsumus, Dómine, piæ supplicatiónis efféctum: et famem propitiátus avérte; ut mortálium corda cognóscant, et te indignánte tália flagélla prodíre, et te miseránte cessáre. Per Dóminum \emph{\&c.}
}\switchcolumn\portugues{
\rlettrine{V}{os} suplicamos, Senhor, concedei-nos a graça de alcançarmos o que de Vós imploramos com nossas súplicas piedosas; e, pela vossa bondade, afastai de nós a fome, a fim de que os corações mortais conheçam que, assim como estes flagelos provêm da vossa indignação, assim também a vossa misericórdia pode fazê-los cessar. Por nosso Senhor \emph{\&c.}
}\end{paracol}

\paragraph{Secreta}
\begin{paracol}{2}\latim{
\rlettrine{D}{eus,} qui humáni generis utrámque substántiam, præséntium númerum et aliménto végetas et rénovas sacraménto: tríbue, quǽsumus; ut eórum et corpóribus nostris subsídium non desit et méntibus. Per Dóminum \emph{\&c.}
}\switchcolumn\portugues{
\slettrine{Ó}{} Deus, que com os dons aqui presentes assistis ao género humano nas suas duas substâncias, sustentando-o com o alimento e renovando-o com o sacramento, concedei-nos, Vos suplicamos, que a assistência, que esperamos, não falte nem aos nossos corpos, nem às nossas almas. Por nosso Senhor \emph{\&c.}
}\end{paracol}

\paragraph{Postcomúnio}
\begin{paracol}{2}\latim{
\rlettrine{G}{ubérna,} quǽsumus, Dómine, temporálibus aliméntis: quos dignáris ætérnis informáre mystériis. Per Dóminum \emph{\&c.}
}\switchcolumn\portugues{
\rlettrine{V}{os} suplicamos, Senhor, dignai-Vos manifestar a vossa providência, concedendo os alimentos temporais àqueles que Vos dignastes robustecer com mistérios eternos. Por nosso Senhor \emph{\&c.}
}\end{paracol}
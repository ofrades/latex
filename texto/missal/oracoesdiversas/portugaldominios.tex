\subsection{Para Portugal e seus Domínios}\label{portugaldominios}

\paragraph{Colecta\footnote{Por Determinação da Sagrada Congregação dos Ritos (19 de Maio de 1952) em todas as Missas, excepto nas de Réquiem, a seguir à última Oração, à última Secreta e ao último Postcomúnio.}}

\rlettrine{E}{} livrai de todas as adversidades no ar, na terra e no mar os vossos servos: o nosso Papa {\redx N.} o nosso Patriarca {\redx N.} (Arcebispo {\redx N.} ou Bispo {\redx N.})\footnote{Acrescenta-se o nome do Antístite do lugar.}, o nosso Presidente e os nossos Governantes, assim como o povo que lhes foi confiado e o seu exército; concedei a paz e a prosperidade aos nossos tempos e afastai da nossa Igreja toda a maldade, destruindo a soberba dos pagãos e dos hereges com o poder da vossa dextra. (Por nosso Senhor \emph{\&c.} ou Pelo mesmo nosso Senhor segundo a conclusão da Oração, da Secreta ou do Postcomúnio a que fica acrescentada)

ADVERTÊNCIAS:
\begin{compactitem}
\item Se a Missa tiver uma única Oração, dir-se-á esta Colecta imediatamente à Oração e sob a mesma e única conclusão; o mesmo quanto à Secreta e Postcomúnio.
\item Quando na Missa se recitar a Ovação «Pelo Papa», omitir-se-ão as Palavras «o nosso Papa»; o mesmo quanto ao nome do Antístite, quando a Sé Diocesana estiver vaga.
\item Se se recitar a Oração «A Cunctis», omitir-se-ão as Palavras: «concedei a paz e a prosperidade aos nossos tempos e afastai da nossa Igreja toda a maldade».
\item Quando se diz a Oração «Pela Igreja omitem-se estas Palavras: «afastai da nossa Igreja toda a maldade».
\item Quando se diz a Ovação «Pela Paz», omitem-se as Palavras: «concedei a paz e a prosperidade aos nossos tempos».
\item Nas Secretas e Postcomúnios não há alteração.
\end{compactitem}

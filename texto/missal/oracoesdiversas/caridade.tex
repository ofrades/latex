\subsection{29.ª Para pedir a caridade}\label{caridade}

\paragraph{Oração}
\begin{paracol}{2}\latim{
\rlettrine{D}{eus,} qui diligéntibus te facis cuncta prodésse: da córdibus nostris inviolábilem tuæ cantátis afféctum; ut desidéria, de tua inspiratióne concépta, nulla possint tentatióne imitári. Per Dóminum nostrum \emph{\&c.}
}\switchcolumn\portugues{
\slettrine{Ó}{} Deus, que fazeis tender todas as cousas para benefício daqueles que Vos amam, gravai perpetuamente nos nossos corações os afectos da vossa caridade, a fim de que os desejos, que concebemos por vossa inspiração, permaneçam invariavelmente em nós, a despeito de todas as tentações. Por nosso Senhor \emph{\&c.}
}\end{paracol}

\paragraph{Secreta}
\begin{paracol}{2}\latim{
\rlettrine{D}{eus,} qui nos ad imáginem tuam sacraméntis rénovas et præcéptis: pérfice gressus nostros in sémitis tuis; ut cantátis donum, quod fecísti a nobis sperári, per hæc, quæ offérimus sacrifícia, fácias veráciter apprehéndi. Per Dóminum \emph{\&c.}
}\switchcolumn\portugues{
\slettrine{Ó}{} Deus, que com vossos sacramentos e preceitos nos renovais à «Vossa imagem», fazei-nos avançar no caminho da vossa perfeição, a fim de que, pela virtude deste sacrifício, que Vos oferecemos, possamos alcançar verdadeiramente o dom da caridade, que nos ensinastes a esperar de Vós. Por nosso Senhor \emph{\&c.}
}\end{paracol}

\paragraph{Postcomúnio}
\begin{paracol}{2}\latim{
\rlettrine{S}{ancti} Spíritus grátia, quǽsumus, Dómine, corda 
nostra illúminet: et perféctæ cantátis dulcédine abundánter refíciat. Per Dóminum \emph{\&c.} in unitáte ejusdem Spíritus Sancti.
}\switchcolumn\portugues{
\rlettrine{S}{enhor,} Vos suplicamos, que a graça do vosso Espírito Santo ilumine os nossos corações e os reconforte e sacie com a doçura da caridade perfeita. Por nosso Senhor \emph{\&c.}
}\end{paracol}
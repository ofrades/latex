\subsection{9.ª Pela concórdia na Congregação}\label{concordiacongregacao}

\paragraph{Oração}
\begin{paracol}{2}\latim{
\rlettrine{D}{eus,} lárgiter pacis et amátor cantátis: da fámulis tuis veram cum tua voluntáte concórdiam; ut ab ómnibus, quæ nos pulsant, tentatiónibus liberémur. Per Dóminum nostrum \emph{\&c.}
}\switchcolumn\portugues{
\slettrine{Ó}{} Deus, que dais a paz e amais a caridade, concedei aos vossos servos a verdadeira união com vossa vontade a fim de que sejamos livres de todas as tentações que nos perseguem. Por nosso Senhor \emph{\&c.}
}\end{paracol}

\paragraph{Secreta}
\begin{paracol}{2}\latim{
\rlettrine{H}{is} sacrifíciis, Dómine, quǽsumus, concéde placátus: ut, qui própriis orámus absólvi delíctis, non gravémur extérnis. Per Dóminum nostrum \emph{\&c.}
}\switchcolumn\portugues{
\rlettrine{A}{placado} com este sacrifício, Senhor, concedei-nos, Vos pedimos, que nós, querendo ser absolvidos dos nossos próprios pecados, não sejamos sobrecarregados com os alheios. Por nosso Senhor \emph{\&c.}
}\end{paracol}

\paragraph{Postcomúnio}
\begin{paracol}{2}\latim{
\rlettrine{S}{píritum} nobis, Dómine, tuæ cantátis infúnde: ut, quos uno pane cœlésti satiásti, tua fácias pietáte concórdes. Per Dóminum \emph{\&c.} in unitáte ejusdem.
}\switchcolumn\portugues{
\rlettrine{I}{nfundi} em nós, Senhor, o espírito da vossa caridade, para que misericordiosamente torneis unidos de coração aqueles a quem saciastes com o mesmo Pão celestial. Por nosso Senhor \emph{\&c.}
}\end{paracol}
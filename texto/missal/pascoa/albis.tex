\subsection{Domingo In Albis}

\paragraphinfo{Intróito}{1. Pe. 2, 2}
\begin{paracol}{2}\latim{
\qlettrine{Q}{uasi} modo géniti infántes, allelúja: rationabiles, sine dolo lac concupíscite, allelúja, allelúja allelúja. \emph{Ps. 80, 2} Exsultáte Deo, adjutóri nostro: jubiláte Deo Jacob.
℣. Gloria Patri \emph{\&c.}
}\switchcolumn\portugues{
\rlettrine{C}{omo} meninos recém-nascidos, aleluia, e raciocinando convenientemente, desejai com ardor beber o leite puro, aleluia, aleluia, aleluia. \emph{Sl. 80, 2} Aclamai a Deus, que é o nosso Sustentáculo: cantai hinos de júbilo em honra de Deus de Jacob.
℣. Glória ao Pai \emph{\&c.}
}\end{paracol}

\paragraph{Oração}
\begin{paracol}{2}\latim{
\rlettrine{P}{ræsta,} quǽsumus, omnípotens Deus: ut, qui paschália festa perégimus, hæc, te largiénte, móribus et vita teneámus. Per Dóminum \emph{\&c.}
}\switchcolumn\portugues{
\slettrine{Ó}{} Deus omnipotente, Vos suplicamos, agora, que celebrámos as festas pascais, concedei-nos a graça de conservarmos sempre o seu espírito, tanto nos nossos costumes, como na nossa vida. Por nosso Senhor \emph{\&c.}
}\end{paracol}

\paragraphinfo{Epístola}{1. Jo. 5, 4-10}
\begin{paracol}{2}\latim{
Léctio Epístolæ beáti Joannis Apóstoli.
}\switchcolumn\portugues{
Lição da Ep.ª do B. Ap.º S. João.
}\switchcolumn*\latim{
\rlettrine{C}{aríssimi:} Omne, quod natum est ex Deo, vincit mundum: et hæc est victoria, quæ vincit mundum, fides nostra. Quis est, qui vincit mundum, nisi qui credit, quóniam Jesus est Fílius Dei? Hic est, qui venit per aquam et sánguinem, Jesus Christus: non in aqua solum, sed in aqua et sánguine. Et Spíritus est, qui testificátur, quóniam Christus est véritas. Quóniam tres sunt, qui testimónium dant in cœlo: Pater, Verbum, et Spíritus Sanctus: et hi tres unum sunt. Et tres sunt, qui testimónium dant in terra: Spíritus, et aqua, et sanguis: et hi tres unum sunt. Si testimónium hóminum accípimus, testimónium Dei majus est: quóniam hoc est testimónium Dei, quod majus est: quóniam testificátus est de Fílio suo. Qui credit in Fílium Dei, habet testimónium Dei in se.
}\switchcolumn\portugues{
\rlettrine{C}{aríssimos:} Todo aquele que nasceu de Deus é vencedor do mundo. O que alcança vitória contra o mundo é a nossa fé. Quem é que vence o mundo, senão o que crê que Jesus Cristo é o Filho de Deus? Este mesmo Jesus Cristo é que veio com a água e o sangue: não somente com a água, mas com a água e o sangue. E é o Espírito quem atesta que Jesus Cristo é a verdade. Porquanto são três os que dão testemunho do céu: o Pai, o Verbo, e o Espírito Santo: e estes três são um só. E três são os que dão testemunho na terra: o espírito, a água e o sangue: e estes três são um só. Se aceitamos o testemunho dos homens, muito maior é o testemunho de Deus. Este testemunho de Deus, que é maior, foi por Ele dado a respeito do seu Filho. Aquele que crê no Filho de Deus tem este testemunho de Deus em si mesmo.
}\end{paracol}


\begin{paracol}{2}\latim{
Allelúja, allelúja. ℣. \emph{Matth. 28, 7} In die resurrectiónis meæ, dicit Dóminus, præcédam vos in Galilǽam. Allelúja. ℣. \emph{Joann. 20, 26} Post dies octo, jánuis clausis, stetit Jesus in médio discipulórum suórum, et dixit: Pax vobis. Allelúja.
}\switchcolumn\portugues{
Aleluia, aleluia. ℣. \emph{Mt. 28, 7} No dia da minha ressurreição, diz o Senhor, irei adiante de vós para a Galileia, aleluia. ℣. \emph{Jo. 20, 26} Oito dias depois, estando as portas fechadas, Jesus veio e, estando no meio dos seus discípulos, disse: «A paz seja convosco», aleluia.
}\end{paracol}

\paragraphinfo{Evangelho}{Jo. 20, 19-31}
\begin{paracol}{2}\latim{
\cruz Sequéntia sancti Evangélii secúndum Joánnem.
}\switchcolumn\portugues{
\cruz Continuação do santo Evangelho segundo S. João.
}\switchcolumn*\latim{
\blettrine{I}{n} illo témpore: Cum sero esset die illo, una sabbatórum, et fores essent clausæ, ubi erant discípuli congregáti propter metum Judæórum: venit Jesus, et stetit in médio, et dixit eis: Pax vobis. Et cum hoc dixísset, osténdit eis manus et latus. Gavísi sunt ergo discípuli, viso Dómino. Dixit ergo eis íterum: Pax vobis. Sicut misit me Pater, et ego mitto vos. Hæc cum dixísset, insufflávit, et dixit eis: Accípite Spíritum Sanctum: quorum remiseritis peccáta, remittúntur eis; et quorum retinuéritis, reténta sunt. Thomas autem unus ex duódecim, qui dícitur Dídymus, non erat cum eis, quando venit Jesus. Dixérunt ergo ei alii discípuli: Vídimus Dóminum. Ille autem dixit eis: Nisi vídero in mánibus ejus fixúram clavórum, et mittam dígitum meum in locum clavórum, et mittam manum meam in latus ejus, non credam. Et post dies octo, íterum erant discípuli ejus intus, et Thomas cum eis. Venit Jesus, jánuis clausis, et stetit in médio, et dixit: Pax vobis. Deinde dicit Thomæ: Infer dígitum tuum huc et vide manus meas, et affer manum tuam et mitte in latus meum: et noli esse incrédulus, sed fidélis. Respóndit Thomas et dixit ei: Dóminus meus et Deus meus. Dixit ei Jesus: Quia vidísti me, Thoma, credidísti: beáti, qui non vidérunt, et credidérunt. Multa quidem et alia signa fecit Jesus in conspéctu discipulórum suórum, quæ non sunt scripta in libro hoc. Hæc autem scripta sunt, ut credátis, quia Jesus est Christus, Fílius Dei: et ut credéntes vitam habeátis in nómine ejus.
}\switchcolumn\portugues{
\blettrine{N}{aquele} tempo, chegada a tarde daquele dia, que era o primeiro da semana, encontrando-se os discípulos reunidos em um lugar, cujas portas estavam fechadas, por causa do medo que tinham dos judeus, veio Jesus, e, estando no meio deles, disse: «A paz seja convosco!». Depois de dizer isto, mostrou as suas mãos e o seu lado. E alegraram-se os discípulos, vendo o Senhor. E disse-lhes novamente: «A paz seja convosco! Assim como meu Pai me enviou, assim também vos envio». Ditas estas palavras, soprou sobre eles, dizendo: «Recebei o Espírito Santo. Àqueles a quem perdoardes os pecados, ser-lhes-ão perdoados, e àqueles a quem os retiverdes, ser-lhes-ão retidos». Porém Tomé, um dos Doze, que era chamado Dídimo, não estava com eles. Disseram-lhe, então, os outros discípulos: «Vimos o Senhor!». Ele respondeu-lhes: «Se não vir nas suas mãos o sinal dos cravos, se não meter o meu dedo no lugar dos cravos e não meter a minha mão no seu lado, não acreditarei». Passados oito dias, encontravam-se os discípulos outra vez, no mesmo lugar, estando Tomé com eles. E veio Jesus, estando as portas fechadas; e, pondo-se no meio deles, disse: «A paz seja convosco». Em seguida disse a Tomé: «Mete aqui o teu dedo e vê as minhas mãos; aproxima, também, a tua mão e mete-a no meu lado; e não sejas incrédulo, mas fiel». Respondeu Tomé: «Meu Senhor e meu Deus!». Disse-lhe Jesus: «Porque me viste, ó Tomé, acreditaste: bem-aventurados aqueles que não viram e acreditaram». Jesus fez ainda na presença de seus discípulos muitos outros milagres, que não foram escritos neste livro; mas estes foram, a fim de que acrediteis que Jesus Cristo é Filho de Deus; e, acreditando, alcanceis a vida eterna em seu nome.
}\end{paracol}

\paragraphinfo{Ofertório}{Mt. 28, 2, 5 \& 6. }
\begin{paracol}{2}\latim{
\rlettrine{A}{ngelus} Dómini descéndit de cœlo, et dixit muliéribus: Quem quǽritis, surréxit, sicut dixit, allelúja.
}\switchcolumn\portugues{
\rlettrine{U}{m} anjo do Senhor desceu do céu e disse às mulheres: «Aquele a quem procurais ressuscitou, como havia anunciado», aleluia.
}\end{paracol}

\paragraph{Secreta}
\begin{paracol}{2}\latim{
\rlettrine{S}{uscipe} múnera, Dómine, quǽsumus, exsultántis Ecclésiæ: et, cui causam tanti gáudii præstitísti, perpétuæ fructum concéde lætítiæ. Per Dóminum \emph{\&c.}
}\switchcolumn\portugues{
\rlettrine{A}{ceitai,} Senhor, Vos suplicamos, as ofertas que a vossa Igreja alegremente Vos consagra; e, assim como lhe proporcionastes a graça de tão grande gozo; concedei-lhe, também o fruto da eterna alegria. Por nosso Senhor \emph{\&c.}
}\end{paracol}

\paragraphinfo{Comúnio}{Jo. 20, 27}
\begin{paracol}{2}\latim{
\rlettrine{M}{itte} manum tuam, et cognósce loca clavórum, allelúja: et noli esse incrédulus, sed fidélis, allelúja, allelúja.
}\switchcolumn\portugues{
\rlettrine{M}{ete} aqui o teu dedo e toca no lugar dos cravos, aleluia; não sejas incrédulo, mas fiel, aleluia, aleluia.
}\end{paracol}

\paragraph{Postcomúnio}
\begin{paracol}{2}\latim{
\qlettrine{Q}{uǽsumus,} Dómine, Deus noster: ut sacrosáncta mystéria, quæ pro reparatiónis nostræ munímine contulísti; et præsens nobis remédium esse fácias et futúrum. Per Dóminum \emph{\&c.}
}\switchcolumn\portugues{
\rlettrine{V}{os} suplicamos, Senhor, nosso Deus, permiti que estes sacrossantos mystérios, que instituístes para alcançarmos a regeneração, sejam nosso remédio salutar no presente e no futuro. Por nosso Senhor \emph{\&c.}
}\end{paracol}
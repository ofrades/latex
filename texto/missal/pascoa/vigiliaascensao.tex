\subsection{Vigília da Festa da Ascensão de N. S. J. C.}

\textit{Como no 5.º Domingo depois da Páscoa, página \pageref{5depoispascoa}, excepto:}

\paragraphinfo{Epístola}{Ef. 4, 7-13}
\begin{paracol}{2}\latim{
Léctio Epístolæ beáti Pauli Apóstoli ad Ephésios.
}\switchcolumn\portugues{
Lição da Ep.ª do B. Ap.º Paulo aos Efésios.
}\switchcolumn*\latim{
\rlettrine{F}{ratres:} Unicuíque nostrum data est grátia secúndum mensúram donatiónis Christi. Propter quod dicit: Ascéndens in altum, captívam duxit captivitátem: dedit dona homínibus. Quod autem ascéndit, quid est, nisi quia et descéndit primum in inferióres partes terræ? Qui descéndit, ipse est et qui ascéndit super omnes cœlos, ut impléret ómnia. Et ipse dedit quosdam quidem apóstolos, quosdam autem prophétas,
álios vero evangelístas, álios autem pastóres et doctóres, ad consummatiónem sanctórum in opus ministérii, in ædificatiónem córporis Christi: donec occurrámus omnes in unitátem fídei, et agnitiónis Fílii Dei, in virum perféctum, in mensúram ætátis plenitúdinis Christi.
}\switchcolumn\portugues{
\rlettrine{M}{eus} irmãos: A graça é dada a cada um de nós, segundo a medida do dom de Cristo. Por isso se diz na Escritura: «Subindo ao alto, levou consigo os cativos e repartiu os seus dons pelos homens». Que quer dizer «subiu» senão que antes tinha descido às regiões inferiores da terra? Aquele que desceu é o mesmo que subiu acima de todos os céus, a fim de completar todas as cousas. Foi Ele, também, quem deu à sua Igreja uns como Apóstolos, outros como Profetas, outros como Evangelistas, outros como Pastores e Doutores, para que uns e outros trabalhem para a perfeição dos santos, nas funções do ministério e na edificação do corpo de Cristo, até que todos cheguemos à unidade de uma só fé e do mesmo conhecimento do Filho de Deus, ao estado de um homem perfeito, a um desenvolvimento orgânico proporcionado à plenitude de Cristo.
}\end{paracol}

\begin{paracol}{2}\latim{
Allelúja, allelúja. ℣. Surréxit Christus, et illúxit nobis, quos rédemit sánguine suo. Allelúja. ℣. \emph{Joann. 16, 28} Exívi a Patre, et veni in mundum: íterum relínquo mundum, et vado ad Patrem. Allelúja.
}\switchcolumn\portugues{
Aleluia, aleluia. ℣. Ressuscitou Cristo: e fez brilhar a sua luz sobre nós, a quem remiu com o sen sangue. Aleluia. ℣. \emph{Jo. 16, 28} Saí do Pai e vim ao mundo; agora, outra vez, deixo o mundo e vou ao Pai. Aleluia.
}\end{paracol}

\paragraphinfo{Evangelho}{Jo. 17, 1-11}
\begin{paracol}{2}\latim{
\cruz Sequéntia sancti Evangélii secúndum Joánnem.
}\switchcolumn\portugues{
\cruz Continuação do santo Evangelho segundo S. João.
}\switchcolumn*\latim{
\blettrine{I}{n} illo témpore: Sublevátis Jesus oculis in cœlum, dixit: Pater, venit hora, clarífica Fílium tuum, ut Fílius tuus claríficet te: sicut dedísti ei potestátem omnis carnis, ut omne, quod dedísti ei, det eis vitam ætérnam. Hæc est autem vita ætérna: ut cognóscant te, solum Deum verum, et quem misísti Jesum Christum. Ego te clarificávi super terram: opus consummávi, quod dedísti mihi, ut fáciam: et nunc clarífica me tu, Pater, apud temetípsum, claritáte, quam hábui, priúsquam mundus esset, apud te. Manifestávi nomen tuum homínibus, quos dedísti mihi de mundo. Tui erant, et mihi eos dedísti; et sermónem tuum servavérunt. Nunc cognovérunt, quia ómnia, quæ dedísti mihi, abs te sunt: quia verba, quæ dedísti mihi, dedi eis: et ipsi accepérunt, et cognovérunt vere, quia a te exivi, et credidérunt, quia tu me misísti. Ego pro eis rogo, non pro mundo rogo, sed pro his, quos dedísti mihi: quia tui sunt: et mea ómnia tua sunt, et tua mea sunt: et clarificátus sum in eis. Et jam non sum in mundo, et hi in mundo sunt, et ego ad te vénio.
}\switchcolumn\portugues{
\blettrine{N}{aquele} tempo, levantou Jesus os olhos ao céu e disse: «Pai, veio a hora: glorificai o vosso Filho, para que o vosso Filho Vos glorifique, pois que Lhe concedestes poder sobre toda a carne humana, para que a todos quantos Lhe destes Ele dê a vida eterna! Ora a vida eterna consiste em que Vos conheçam como o único verdadeiro Deus, e a Jesus Cristo, que enviastes. Já Vos glorifiquei na terra; já cumpri a missão que me distribuístes. Agora, pois, ó Pai, glorificai-me junto de Vós com aquela glória que Eu possuía antes que o mundo fosse criado. Já manifestei o vosso nome aos homens, que me destes, no meio do mundo. Eles eram vossos, e Vós mos destes; e eles guardaram a vossa palavra. Agora já conhecem que tudo o que me destes vem de Vós; pois as palavras que me destes Eu lhas dei e as receberam: reconheceram verdadeiramente que saí de Vós e acreditaram que me enviastes. Eu rogo por eles. Não rogo pelo mundo, mas por aqueles que me destes, pois são vossos: todas minhas cousas são vossas, e sou glorificado neles. Já não pertenço mais ao mundo, porém eles ficam no mundo. E Eu vou para Vós».
}\end{paracol}

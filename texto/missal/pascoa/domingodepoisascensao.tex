\subsection{Domingo depois da Ascensão}

\paragraphinfo{Intróito}{Sl. 26, 7, 8 \& 9}
\begin{paracol}{2}\latim{
\rlettrine{E}{xáudi,} Dómine, vocem meam, qua clamávi ad te, allelúja: tibi dixit cor meum, quæsívi vultum tuum, vultum tuum, Dómine, requíram: ne avértas fáciem tuam a me, allelúja, allelúja. \emph{Ps. ibidem, 1} Dóminus illuminátio mea et salus mea: quem timébo?
℣. Gloria Patri \emph{\&c.}
}\switchcolumn\portugues{
\rlettrine{O}{uvi,} Senhor, a minha voz: ouvi o clamor com que Vos imploro, aleluia, aleluia! Meu coração dirigiu-Vos estas palavras: Procurei a vossa presença, Senhor; não cessarei de procurar a vossa presença; não afasteis, pois, de mim a vossa face, aleluia, aleluia. \emph{Sl. ibidem, 1} Ó Senhor é a minha luz e a minha salvação. A quem, pois, temerei?
℣. Glória ao Pai \emph{\&c.}
}\end{paracol}

\paragraph{Oração}
\begin{paracol}{2}\latim{
\rlettrine{O}{mnípotens} sempitérne Deus: fac nos tibi semper et devótam gérere voluntátem; et majestáti tuæ sincéro corde servíre. Per Dóminum nostrum \emph{\&c.}
}\switchcolumn\portugues{
\rlettrine{O}{mnipotente} e eterno Deus, permiti que a nossa vontade seja sempre fervorosa para convosco e que sirvamos a vossa majestade com sinceridade de coração. Por nosso Senhor \emph{\&c.}
}\end{paracol}

\paragraphinfo{Epístola}{1. Pe. 4, 7-11}
\begin{paracol}{2}\latim{
Léctio Epístolæ beáti Petri Apóstoli.
}\switchcolumn\portugues{
Lição da Ep.ª do B. Ap.º Pedro.
}\switchcolumn*\latim{
\rlettrine{C}{aríssimi:} Estóte prudéntes et vigiláte in oratiónibus. Ante ómnia autem mútuam in vobismetípsis caritátem contínuam habéntes: quia cáritas óperit multitúdinem peccatórum. Hospitáles ínvicem sine murmuratióne: unusquísque, sicut accépit grátiam, in altérutrum illam administrántes, sicut boni dispensatóres multifórmis grátiæ Dei. Si quis lóquitur, quasi sermónes Dei: si quis minístrat, tamquam ex virtúte, quam adminístrat Deus: ut in ómnibus honorificétur Deus per Jesum Christum, Dóminum nostrum.
}\switchcolumn\portugues{
\rlettrine{C}{aríssimos:} Sede prudentes e vigiai na oração. Porém, primeiro do que tudo, sede ardentemente caritativos uns para com os outros, pois a caridade apaga uma multidão de pecados. Hospedai-vos uns aos outros sem murmuração. Que cada um seja caritativo para com o próximo, segundo a abundância dos dons, que recebeu, como dispensadores da multiforme graça de Deus. Se alguém falar, que o faça com palavras de Deus; se alguém exercer um ministério, que o faça pela virtude que Deus lhe dá, a fim de que Deus seja glorificado em tudo por Nosso Senhor Jesus Cristo.
}\end{paracol}

\begin{paracol}{2}\latim{
 Allelúja, allelúja. ℣. \emph{Ps. 46, 9} Regnávit Dóminus super omnes gentes: Deus sedet super sedem sanctam suam. Allelúja. ℣. \emph{Joann. 14, 18} Non vos relínquam órphanos: vado, et vénio ad vos, et gaudébit cor vestrum. Allelúja.
}\switchcolumn\portugues{
Aleluia, aleluia. ℣. \emph{Sl. 46, 9} O Senhor reina em todos Os povos: Deus está assentado sobre o seu trono Sagrado, aleluia. ℣. \emph{Jo. 14, 18} Não vos deixarei órfãos: vou, e voltarei para vós e o vosso coração alegrar-se-á. Aleluia.
}\end{paracol}

\paragraphinfo{Evangelho}{Jo. 15, 26-27; 16, 1-4}
\begin{paracol}{2}\latim{
\cruz Sequéntia sancti Evangélii secúndum Joánnem.
}\switchcolumn\portugues{
\cruz Continuação do santo Evangelho segundo S. João.
}\switchcolumn*\latim{
\blettrine{I}{n} illo témpore: Dixit Jesus discípulis suis: Cum vénerit Paráclitus, quem ego mittam vobis a Patre, Spíritum veritátis, qui a Patre procédit, ille testimónium perhibébit de me: et vos testimónium perhibébitis, quia ab inítio mecum estis. Hæc locútus sum vobis, ut non scandalizémini. Absque synagógis fácient vos: sed venit hora, ut omnis, qui intérficit vos, arbitrétur obséquium se præstáre Deo. Et hæc fácient vobis, quia non novérunt Patrem neque me. Sed hæc locútus sum vobis: ut, cum vénerit hora eórum, reminiscámini, quia ego dixi vobis.
}\switchcolumn\portugues{
\blettrine{N}{aquele} tempo, disse Jesus aos seus discípulos: «Quando vier o Paráclito, que vos enviarei da parte do Pai Espírito da verdade, que procede do Pai Ele dará testemunho de mim, e também vós dareis testemunho de mim, porque estais comigo desde o princípio. Digo-vos estas coisas para que vos não escandalizeis. Expulsar-vos-ão das sinagogas, e até se aproxima a hora em que qualquer que vos matar cuidará que presta homenagem a Deus. Procederão assim, porque não conhecem nem o Pai, nem a mim. Digo-Vos estas coisas para que, quando chegar a hora, vos lembreis de que vo-las disse.
}\end{paracol}

\paragraphinfo{Ofertório}{Sl. 46, 6}
\begin{paracol}{2}\latim{
\rlettrine{A}{scéndit} Deus in jubilatióne, et Dóminus in voce tubæ, allelúja.
}\switchcolumn\portugues{
\rlettrine{D}{eus} elevou-se por entre aclamações de júbilo; o Senhor elevou-se ao som da trombeta, aleluia.
}\end{paracol}

\paragraph{Secreta}
\begin{paracol}{2}\latim{
\rlettrine{S}{acrifícia} nos, Dómine, immaculáta puríficent: et méntibus nostris supérnæ grátiæ dent vigórem. Per Dóminum nostrum \emph{\&c.}
}\switchcolumn\portugues{
\rlettrine{S}{enhor,} que estes sacrifícios imaculados nos purifiquem e que comuniquem às nossas almas o vigor da graça celestial. Por nosso Senhor \emph{\&c.}
}\end{paracol}

\paragraphinfo{Comúnio}{Jo. 17,12-13 \& 15}
\begin{paracol}{2}\latim{
\rlettrine{P}{ater,} cum essem cum eis, ego servábam eos, quos dedísti mihi, allelúja: nunc autem ad te vénio: non rogo, ut tollas eos de mundo, sed ut serves eos a malo, allelúja, allelúja.
}\switchcolumn\portugues{
\rlettrine{P}{ai,} quando estava com eles, guardei aqueles que me entregastes, aleluia; mas agora, que venho a Vós, não Vos peço que os tireis do mundo, mas que os livreis do mal, aleluia, aleluia.
}\end{paracol}

\paragraph{Postcomúnio}
\begin{paracol}{2}\latim{
\rlettrine{R}{epléti,} Dómine, munéribus sacris: da, quǽsumus; ut in gratiárum semper actióne maneámus. Per Dóminum nostrum \emph{\&c.}
}\switchcolumn\portugues{
\rlettrine{H}{avendo} sido saciados com estes dons sacratíssimos, fazei, Senhor, Vos suplicamos, que por este motivo Vos rendamos contínuas acções de graças. Por nosso Senhor \emph{\&c.}
}\end{paracol}

\subsection{Terceiro Domingo depois da Páscoa}

\paragraphinfo{Intróito}{Sl. 65, 1-2}
\begin{paracol}{2}\latim{
\qlettrine{J}{ubiláte} Deo, omnis terra, allelúja: psalmum dícite nómini ejus, allelúja: date glóriam laudi ejus, allelúja, allelúja, allelúja. \emph{Ps. ibid., 3} Dícite Deo, quam terribília sunt ópera tua, Dómine! in multitúdine virtútis tuæ mentiéntur tibi inimíci tui.
℣. Gloria Patri \emph{\&c.}
}\switchcolumn\portugues{
\qlettrine{Q}{ue} todos os povos da terra entoem cânticos de alegria em honra de Deus, aleluia: cantai a glória do seu nome, aleluia: rendei honra e louvor ao Senhor, aleluia, aleluia. Dizei a Deus: \emph{Sl. ibid., 3} «Quão admiráveis, Senhor, são as vossas obras! Tão grande é o vosso poder, que sois glorificado até pelos vossos inimigos!».
℣. Glória ao Pai \emph{\&c.}
}\end{paracol}

\paragraph{Oração}
\begin{paracol}{2}\latim{
\rlettrine{D}{eus,} qui errántibus, ut in viam possint redíre justítiæ, veritátis tuæ lumen osténdis: da cunctis, qui christiána professióne censéntur, et illa respúere, quæ huic inimíca sunt nómini; et ea, quæ sunt apta, sectári. Per Dóminum nostrum \emph{\&c.}
}\switchcolumn\portugues{
\slettrine{Ó}{} Deus, que com a luz da verdade esclareceis aqueles que estão no caminho do erro para que possam voltar ao caminho da justiça, concedei aos que professam a fé cristã, que rejeitem tudo quanto é contrário ao nome cristão e sigam o que lhe é conforme. Por nosso Senhor \emph{\&c.}
}\end{paracol}

\paragraphinfo{Epístola}{1. Pe. 2, 11-19}
\begin{paracol}{2}\latim{
Léctio Epístolæ beáti Petri Apóstoli.
}\switchcolumn\portugues{
Lição da Ep.ª do B. Ap.º Pedro.
}\switchcolumn*\latim{
\rlettrine{C}{aríssimi:} Obsecro vos tamquam ádvenas et peregrínos abstinére vos a carnálibus desidériis, quæ mílitant advérsus ánimam, conversatiónem vestram inter gentes habéntes bonam: ut in eo, quod detréctant de vobis tamquam de malefactóribus, ex bonis opéribus vos considerántes, gloríficent Deum in die visitatiónis. Subjécti ígitur estóte omni humánæ creatúræ propter Deum: sive regi, quasi præcellénti: sive dúcibus, tamquam ab eo missis ad vindíctam malefactórum, laudem vero bonórum: quia sic est volúntas Dei, ut benefaciéntes obmutéscere faciátis imprudéntium hóminum ignorántiam: quasi líberi, et non quasi velámen habéntes malítiæ libertátem, sed sicut servi Dei. Omnes honoráte: fraternitátem dilígite: Deum timéte: regem honorificáte Servi, súbditi estóte in omni timóre dóminis, non tantum bonis et modéstis, sed étiam dýscolis. Hæc est enim grátia: in Christo Jesu, Dómino nostro.
}\switchcolumn\portugues{
\rlettrine{C}{aríssimos:} Como estrangeiros e peregrinos que sois, exorto-vos a que vos abstenhais dos apetites carnais, que são contrários à alma. Tende uma conduta honesta entre os povos, a fim de que, em vez de vos caluniarem, como a malfeitores, observem as vossas boas obras, e por causa delas glorifiquem o Senhor, no dia da sua visita. Sede, pois, sujeitos por amor de Deus a toda a instituição humana, seja ao rei, como soberano, seja aos governadores, como delegados dele, para castigo dos malfeitores e louvor dos beneméritos; pois é vontade de Deus que, praticando o bem, caleis a ignorância dos homens insensatos. Portai-vos como homens livres, não para fazer da liberdade capa para tapar a malícia, mas para mostrardes que sois servos de Deus. Honrai todos os homens; amai os vossos irmãos; temei a Deus; honrai o rei. Vós, como servos, sede submissos com todo o respeito a vossos senhores, não somente àqueles que são bons e pacíficos, mas também aos que são rigorosos; porquanto isto agrada a Deus, em nosso Senhor Jesus Cristo.
}\end{paracol}

\begin{paracol}{2}\latim{
Allelúja, allelúja. ℣. \emph{Ps. 110, 9} Redemptiónem misit Dóminus pópulo suo. Allelúja. ℣. \emph{Luc. 24, 46} Oportebat pati Christum, et resúrgere a mórtuis: et ita intráre in glóriam suam. Allelúja.
}\switchcolumn\portugues{
Aleluia, aleluia. ℣. \emph{Sl. 110, 9} O Senhor mandou a salvação ao seu povo. Aleluia. ℣. \emph{Lc. 24, 46} Era preciso que Cristo sofresse, para que depois ressuscitasse dos mortos e entrasse assim na glória, aleluia.
}\end{paracol}

\paragraphinfo{Evangelho}{Jo. 16, 16-22}
\begin{paracol}{2}\latim{
\cruz Sequéntia sancti Evangélii secúndum Joánnem.
}\switchcolumn\portugues{
\cruz Continuação do santo Evangelho segundo S. João.
}\switchcolumn*\latim{
\blettrine{I}{n} illo témpore: Dixit Jesus discípulis suis: Módicum, et jam non vidébitis me: et íterum módicum, et vidébitis me: quia vado ad Patrem. Dixérunt ergo ex discípulis ejus ad ínvicem: Quid est hoc, quod dicit nobis: Módicum, et non vidébitis me: et íterum módicum, et vidébitis me, et quia vado ad Patrem? Dicébant ergo: Quid est hoc, quod dicit: Modicum? nescímus, quid lóquitur. Cognóvit autem Jesus, quia volébant eum interrogáre, et dixit eis: De hoc quǽritis inter vos, quia dixi: Modicum, et non vidébitis me: et íterum módicum, et vidébitis me. Amen, amen, dico vobis: quia plorábitis et flébitis vos, mundus autem gaudébit: vos autem contristabímini, sed tristítia vestra vertétur in gáudium. Múlier cum parit, tristítiam habet, quia venit hora ejus: cum autem pepérerit púerum, jam non méminit pressúræ propter gáudium, quia natus est homo in mundum. Et vos igitur nunc quidem tristítiam habétis, íterum autem vidébo vos, et gaudébit cor vestrum: et gáudium vestrum nemo tollet a vobis.
}\switchcolumn\portugues{
\blettrine{N}{aquele} tempo, disse Jesus aos discípulos: «Ainda um pouco de tempo e me não vereis mais; e ainda um pouco de tempo e me tornareis a ver, porque vou ao Pai». Disseram, então, alguns discípulos uns aos outros: «Que significa isto que nos diz: «Ainda um pouco de tempo e me não vereis mais; e mais um pouco de tempo e me tornareis a ver, porque vou ao Pai?». E diziam os discípulos: «Que quer Ele dizer com estas palavras: «Ainda um pouco?». Ignoramos o que quer dizer». Então Jesus, conhecendo que queriam interrogá-l’O, disse-lhes: «Interrogai-vos uns aos outros, porque disse: «Ainda um pouco de tempo e me não vereis mais; e ainda um pouco de tempo e me tornareis a ver?». Em verdade, em verdade vos digo: Chorareis e vos lamentareis, enquanto o mundo se regozija; afligir-vos-eis; mas a vossa tristeza tornar-se-á em gozo. A mulher, quando dá à luz o filho, está triste, porque chegou a sua hora; mas, logo que a criança nasce, esquece as dores que sofreu, com a alegria de ter trazido ao mundo uma criatura humana. Assim, vós, agora, estais tristes; mas outra vez vos verei e o vosso coração estará cheio de gozo; e ninguém vos tirará a alegria».
}\end{paracol}

\paragraphinfo{Ofertório}{Sl. 145, 2}
\begin{paracol}{2}\latim{
\rlettrine{L}{auda,} anima mea, Dóminum: laudábo Dóminum in vita mea: psallam Deo meo, quámdiu ero, allelúja.
}\switchcolumn\portugues{
\rlettrine{L}{ouvai} o Senhor, ó minha alma. Louvarei o Senhor durante toda a vida: cantarei hinos ao meu Deus, enquanto viver, aleluia.
}\end{paracol}

\paragraph{Secreta}
\begin{paracol}{2}\latim{
\rlettrine{H}{is} nobis, Dómine, mystériis conferátur, quo, terréna desidéria mitigántes, discámus amáre cœléstia. Per Dóminum nostrum \emph{\&c.}
}\switchcolumn\portugues{
\slettrine{Ó}{} Senhor, por meio destes sagrados mistérios mitigai-nos os desejos terrenos e ensinai-nos a amar as coisas celestiais. Por nosso Senhor \emph{\&c.}
}\end{paracol}

\paragraphinfo{Comúnio}{Jo. 16, 16}
\begin{paracol}{2}\latim{
\rlettrine{M}{ódicum,} et non vidébitis me, allelúja: íterum módicum, et vidébitis me, quia vado ad Patrem, allelúja, allelúja.
}\switchcolumn\portugues{
\rlettrine{A}{inda} um pouco de tempo e me não vereis mais, aleluia; e ainda um pouco de tempo e me tornareis a ver, porque vou para meu Pai, aleluia, aleluia.
}\end{paracol}

\paragraph{Postcomúnio}
\begin{paracol}{2}\latim{
\rlettrine{S}{acramenta} quæ súmpsimus, quǽsumus, Dómine: et spirituálibus nos instáurent aliméntis, et corporálibus tueántur auxíliis. Per Dóminum nostrum \emph{\&c.}
}\switchcolumn\portugues{
\rlettrine{S}{enhor,} Vos suplicamos, permiti que os sacramentos, que recebemos, nos restaurem, tornando-se em alimento para as nossas almas e em auxílio para os nossos corpos. Por nosso Senhor \emph{\&c.}
}\end{paracol}

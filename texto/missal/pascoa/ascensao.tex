\subsection{Ascensão de Nosso Senhor Jesus Cristo}

\paragraphinfo{Intróito}{Act. 1, 11}
\begin{paracol}{2}\latim{
\rlettrine{V}{iri} Galilǽi, quid admirámini aspiciéntes in cœlum? allelúja: quemádmodum vidístis eum ascendéntem in cœlum, ita véniet, allelúja, allelúja, allelúja. \emph{Ps. 46, 2} Omnes gentes, pláudite mánibus: jubiláte Deo in voce exsultatiónis.
℣. Gloria Patri \emph{\&c.}
}\switchcolumn\portugues{
\rlettrine{V}{arões} galileus, porque estais a olhar, admirados, para o céu? Aleluia! Como O vistes subir para o céu, assim Ele virá: aleluia, aleluia, aleluia. \emph{Sl. 46, 2} Ó povos, batei palmas com júbilo: aclamai Deus com vossos hinos de alegria!
℣. Glória ao Pai \emph{\&c.}
}\end{paracol}

\paragraph{Oração}
\begin{paracol}{2}\latim{
\rlettrine{C}{oncéde,} quǽsumus, omnípotens Deus: ut, qui hodiérna die Unigénitum tuum, Redemptórem nostrum, ad cœlos ascendísse crédimus; ipsi quoque mente in cœléstibus habitémus. Per eúndem Dóminum \emph{\&c.}
}\switchcolumn\portugues{
\slettrine{Ó}{} Deus omnipotente, assim como acreditamos que o vosso Filho Unigénito, nosso Redentor, subiu aos céus neste dia, assim também, Vos suplicamos, tenhamos a nossa morada em espírito lá no céu. Pelo mesmo nosso Senhor \emph{\&c.}
}\end{paracol}

\paragraphinfo{Epístola}{Act. 1, 1-11}
\begin{paracol}{2}\latim{
Léctio Actuum Apostólorum.
}\switchcolumn\portugues{
Lição dos Actos dos Apóstolos.
}\switchcolumn*\latim{
\rlettrine{P}{rimum} quidem sermónem feci de ómnibus, o Theóphile, quæ cœpit Jesus facere et docére usque in diem, qua, præcípiens Apóstolis per Spíritum Sanctum, quos elégit, assúmptus est: quibus et prǽbuit seípsum vivum post passiónem suam in multas arguméntis, per dies quadragínta appárens eis et loquens de regno Dei. Et convéscens, præcépit eis, ab Jerosólymis ne discéderent, sed exspectárent promissiónem Patris, quam audístis (inquit) per os meum: quia Joánnes quidem baptizávit aqua, vos autem baptizabímini Spíritu Sancto non post multos hos dies. Igitur qui convénerant, interrogábant eum, dicéntes: Dómine, si in témpore hoc restítues regnum Israël? Dixit autem eis: Non est vestrum nosse témpora vel moménta, quæ Pater pósuit in sua potestáte: sed accipiétis virtútem superveniéntis Spíritus Sancti in vos, et éritis mihi testes in Jerúsalem et in omni Judǽa et Samaría et usque ad últimum terræ. Et cum hæc dixísset, vidéntibus illis, elevátus est, et nubes suscépit eum ab óculis eórum. Cumque intuerétur in cœlum eúntem illum, ecce, duo viri astitérunt juxta illos in véstibus albis, qui et dixérunt: Viri Galilǽi, quid statis aspiciéntes in cœlum? Hic Jesus, qui assúmptus est a vobis in cœlum, sic véniet, quemádmodum vidístis eum eúntem in cœlum.
}\switchcolumn\portugues{
\qlettrine{J}{á} vos narrei no meu primeiro livro, ó Teófilo, tudo quanto Jesus fez e ensinou, até ao dia em que, após ter dado as suas instruções pelo Espírito Santo aos Apóstolos, que Ele escolhera, subiu aos céus. A estes, também, depois da sua paixão, se apresentou vivo com muitas e infalíveis provas, aparecendo-lhes durante quarenta dias, e falando-lhes do reino de Deus. Em uma vez, comendo com eles à mesa, ordenou-lhes que se não afastassem de Jerusalém, mas que esperassem a realização da promessa do Pai, «que ouvistes (disse Ele) da minha boca; pois João baptizou na água, mas vós sereis baptizados com o Espírito Santo, pisados não muitos dias». Então, aqueles que estavam reunidos perguntaram: «Senhor, será nessa ocasião que restaurareis o reino de Israel?». Ele respondeu: «Não vos é dado conhecerdes nem os tempos, nem os momentos que o Pai fixou com sua autoridade; mas recebereis a virtude do Espírito Santo, que descerá sobre vós, e vos tornareis meus testemunhos em Jerusalém, na Judeia, na Samaria e até aos confins do mundo». Havendo dito isto, elevou-se ao alto na sua presença, envolvendo-O uma nuvem, que O escondeu a seus olhos. E, como contemplassem atentamente o céu, para onde Jesus se elevara, apareceram ao pé deles dous homens, vestidos de branco, que lhes disseram: «Varões galileus, porque estais admirados a olhar para o céu? Este Jesus, que se elevou ao céu no meio de vós, de lá virá da mesma maneira que O vistes elevar-se».
}\end{paracol}

\begin{paracol}{2}\latim{
Allelúja, allelúja. ℣. \emph{Ps. 46, 6} Ascéndit Deus in jubilatióne, et Dóminus in voce tubæ. Allelúja. ℣. Ps. 67, 18-19. Dóminus in Sina in sancto, ascéndens in altum, captívam duxit captivitátem. Allelúja.
}\switchcolumn\portugues{
Aleluia, aleluia. ℣. \emph{Sl. 46, 6} Subiu Deus ao céu no meio de aclamações de júbilo: o Senhor subiu ao som da trombeta. Aleluia. O Senhor do Sinai entrou no seu santuário; e, subindo «ao alto», levou consigo os cativos. Aleluia.
}\end{paracol}

\paragraphinfo{Evangelho}{Mc. 16, 14-20}
\begin{paracol}{2}\latim{
\cruz Sequéntia sancti Evangélii secúndum Marcum.
}\switchcolumn\portugues{
\cruz Continuação do santo Evangelho segundo S. Marcos.
}\switchcolumn*\latim{
\blettrine{I}{n} illo témpore: Recumbéntibus úndecim discípulis, appáruit illis Jesus: et exprobrávit incredulitátem eórum et durítiam cordis: quia iis, qui víderant eum resurrexísse, non credidérunt. Et dixit eis: Eúntes in mundum univérsum, prædicáte Evangélium omni creatúræ. Qui credíderit et baptizátus fúerit, salvus erit: qui vero non credíderit, condemnábitur. Signa autem eos, qui credíderint, hæc sequéntur: In nómine meo dæmónia ejícient: linguis loquantur novis: serpentes tollent: et si mortíferum quid bíberint, non eis nocébit: super ægros manus impónent, et bene habébunt. Et Dóminus quidem Jesus, postquam locútus est eis, assúmptus est in cœlum, et sedet a dextris Dei. Illi autem profécti, prædicavérunt ubíque, Dómino cooperánte et sermónem confirmánte, sequéntibus signis.
}\switchcolumn\portugues{
\blettrine{N}{aquele} tempo, estando à mesa os Onze discípulos, apareceu-lhes Jesus e repreendeu-os por causa da sua incredulidade e dureza de coração; pois não tinham acreditado naqueles que já haviam visto Jesus depois da Ressurreição. E disse-lhes: «Ide por todo o mundo e pregai o Evangelho a todas as criaturas. Quem acreditar e for baptizado será salvo; e quem não acreditar, será condenado. Eis os prodígios que operarão aqueles que tiverem acreditado: Em meu nome expulsarão os demónios; falarão novas línguas; dominarão as serpentes; se tomarem alguma bebida mortífera, nenhum mal lhes acontecerá; e imporão as mãos sobre os enfermos e curá-los-ão». Então o Senhor Jesus, depois de haver assim falado, subiu ao céu, onde está assentado à direita de Deus. E eles partiram e pregaram em todos os lugares, cooperando o Senhor com eles e confirmando as suas palavras com milagres, que as acompanhavam.
}\end{paracol}

\paragraphinfo{Ofertório}{Sl. 46, 6}
\begin{paracol}{2}\latim{
\rlettrine{A}{scéndit} Deus in jubilatióne, et Dóminus in voce tubæ, allelúja.
}\switchcolumn\portugues{
\rlettrine{S}{ubiu} Deus no meio de aclamações jubilosas: o Senhor elevou-se ao som da trombeta, aleluia.
}\end{paracol}

\paragraph{Secreta}
\begin{paracol}{2}\latim{
\rlettrine{S}{úscipe,} Dómine, múnera, quæ pro Fílii tui gloriósa censióne deférimus: et concéde propítius; ut a præséntibus perículis liberémur, et ad vitam per veniámus ætérnam. Per eúndem Dóminum \emph{\&c.}
}\switchcolumn\portugues{
\rlettrine{A}{ceitai,} Senhor, os dons que Vos oferecemos em memória da gloriosa Ascensão de vosso Filho e concedei-nos benigno que sejamos livres dos perigos presentes e alcancemos a vida eterna. Pelo mesmo nosso Senhor \emph{\&c.}
}\end{paracol}

\paragraphinfo{Comúnio}{Sl. 67, 33-34}
\begin{paracol}{2}\latim{
\rlettrine{P}{sállite} Dómino, qui ascéndit super cœlos cœlórum ad Oriéntem, allelúja.
}\switchcolumn\portugues{
\rlettrine{C}{antai} louvores ao Senhor, que sobe no Oriente ao mais alto dos céus, aleluia.
}\end{paracol}

\paragraph{Postcomúnio}
\begin{paracol}{2}\latim{
\rlettrine{P}{ræsta} nobis, quǽsumus, omnípotens et miséricors Deus: ut, quæ visibílibus mystériis suménda percépimus, invisíbili consequámur efféctu. Per Dóminum \emph{\&c.}
}\switchcolumn\portugues{
\slettrine{Ó}{} Deus omnipotente e misericordioso, Vos suplicamos, permiti que sejamos participantes dos efeitos invisíveis dos mystérios em que visivelmente tomámos parte. Por nosso Senhor \emph{\&c.}
}\end{paracol}

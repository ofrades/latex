\subsection{Domingo da Ressurreição}

\paragraphinfo{Intróito}{Sl. 138, 18 \& 5-6}
\begin{paracol}{2}\latim{
\rlettrine{R}{esurréxi,} et adhuc tecum sum, allelúja: posuísti super me manum tuam, allelúja: mirábilis facta est sciéntia tua, allelúja, allelúja. \emph{Ps. ibid., 1-2} Dómine, probásti me et cognovísti me: tu cognovísti sessiónem meam et resurrectiónem meam.
℣. Gloria Patri \emph{\&c.}
}\switchcolumn\portugues{
\rlettrine{R}{essuscitei} e ainda estou convosco, aleluia. Pusestes sobre mim a vossa mão, aleluia. Vossa sabedoria mostrou-se admirável, aleluia, aleluia. \emph{Sl. ibid., 1-2} Experimentastes-me, Senhor, e ficastes-me conhecendo: Ficastes conhecendo quando me deito e quando me levanto.
℣. Glória ao Pai \emph{\&c.}
}\end{paracol}

\paragraph{Oração}
\begin{paracol}{2}\latim{
\rlettrine{D}{eus,} qui hodiérna die per Unigénitum tuum æternitátis nobis áditum, devícta morte, reserásti: vota nostra, quæ præveniéndo aspíras, étiam adjuvándo proséquere. Per eúndem Dóminum \emph{\&c.}
}\switchcolumn\portugues{
\slettrine{Ó}{} Deus, que neste dia pelo triunfo do vosso Filho Unigénito, vencedor da morte, nos abristes o caminho da eternidade, auxiliai-nos com vosso socorro a realizar os votos que nos inspirais com vossa graça. Pelo mesmo nosso Senhor \emph{\&c.}
}\end{paracol}

\paragraphinfo{Epístola}{1. Cor. 5, 7-8}
\begin{paracol}{2}\latim{
Léctio Epístolæ beáti Pauli Apóstoli ad Corinthios.
}\switchcolumn\portugues{
Lição da Ep.ª do B. Ap.º Paulo aos Coríntios.
}\switchcolumn*\latim{
\rlettrine{F}{atres:} Expurgáte vetus ferméntum, ut sitis nova conspérsio, sicut estis ázymi. Etenim Pascha nostrum immolátus est Christus. Itaque epulémur: non in ferménto véteri, neque in ferménto malítiae et nequitiæ: sed in ázymis sinceritátis et veritátis.
}\switchcolumn\portugues{
\rlettrine{M}{eus} irmãos: Agora, que já sois ázimos, limpai-vos do fermento velho, para que vos torneis massa nova; pois Cristo, nossa Páscoa, foi imolado. Celebremos, então, a festa, não com o fermento da malícia e perversidade, mas com os ázimos da sinceridade e verdade.
}\end{paracol}

\paragraphinfo{Gradual}{Sl. 117, 24 \& 1}
\begin{paracol}{2}\latim{
\rlettrine{H}{æc} dies, quam fecit Dóminus: exsultémus et lætémur in ea. ℣. Confitémini Dómino, quóniam bonus: quóniam in sǽculum misericórdia ejus.
}\switchcolumn\portugues{
\rlettrine{E}{is} o dia que o Senhor fez: exultemos e alegremo-nos n’Ele. ℣. Louvai o Senhor, porque Ele é bom: e porque a sua misericórdia é eterna.
}\switchcolumn*\latim{
Allelúja, allelúja. ℣. \emph{1. Cor. 5, 7} Pascha nostrum immolátus est Christus.
}\switchcolumn\portugues{
Aleluia, aleluia. ℣. \emph{1. Cor. 5, 7} Cristo nossa Páscoa, foi imolado.
}\end{paracol}

\subsubsection{Sequência}
\begin{paracol}{2}\latim{
\rlettrine{V}{íctimæ} pascháli laudes ímmolent Christiáni. Agnus rédemit oves: Christus ínnocens Patri reconciliávit peccatóres. Mors et vita duéllo conflixére mirándo: dux vitæ mórtuus regnat vivus. Dic nobis, María, quid vidísti in via? Sepúlcrum Christi vivéntis et glóriam vidi resurgéntis. Angélicos testes, sudárium et vestes. Surréxit Christus, spes mea: præcédet vos in Galilǽam. Scimus Christum surrexísse a mórtuis vere: tu nobis, victor Rex, miserére. Amen. Allelúja
}\switchcolumn\portugues{
\rlettrine{V}{enham} os cristãos oferecer louvores à Vítima Pascal! O Cordeiro remiu as ovelhas; Cristo inocente reconciliou os pecadores com o Pai. A morte e a vida travaram combate estupendo: o autor da vida morreu, mas Ele vive e reina. Diz-nos, ó Maria, o que viste no caminho? Vi o sepulcro de Cristo, que está vivo; vi a glória de Cristo ressuscitado. Vi presentes os Anjos, o sudário e as vestes. Ressuscitou Cristo, minha esperança, que precederá na Galileia os discípulos. Sabemos que Cristo ressuscitou dos mortos, verdadeiramente. Ó Rei vencedor, tende piedade de nós. Amen. Aleluia.
}\end{paracol}

\paragraphinfo{Evangelho}{Mc. 16, 1-7}
\begin{paracol}{2}\latim{
\cruz Sequéntia sancti Evangélii secúndum Marcum.
}\switchcolumn\portugues{
\cruz Continuação do santo Evangelho segundo S. Marcos.
}\switchcolumn*\latim{
\blettrine{I}{n} illo témpore: María Magdaléne et María Jacóbi et Salóme emérunt arómata, ut venientes úngerent Jesum. Et valde mane una sabbatórum, veniunt ad monuméntum, orto jam sole. Et dicébant ad ínvicem: Quis revólvet nobis lápidem ab óstio monuménti? Et respiciéntes vidérunt revolútum lápidem. Erat quippe magnus valde. Et introëúntes in monuméntum vidérunt júvenem sedéntem in dextris, coopértum stola cándida, et obstupuérunt. Qui dicit illis: Nolíte expavéscere: Jesum quǽritis Nazarénum, crucifíxum: surréxit, non est hic, ecce locus, ubi posuérunt eum. Sed ite, dícite discípulis ejus et Petro, quia præcédit vos in Galilǽam: ibi eum vidébitis, sicut dixit vobis.
}\switchcolumn\portugues{
\blettrine{N}{aquele} tempo, Maria Madalena, Maria, mãe de Tiago, e Salomé compraram perfumes para ungir Jesus. Partindo, pois, de manhã cedo, no primeiro dia depois do sábado, chegaram ao sepulcro, tendo já nascido o sol. E diziam umas às outras: «Quem nos tirará a pedra da entrada do sepulcro?». Olhando, então, viram que a pedra estava desviada para o lado: a qual, com efeito, era muito grande. Entrando no sepulcro, viram um jovem, assentado ao lado direito, vestido com uma túnica branca. Ficaram atónitas! Mas o jovem disse-lhes: «Não tenhais medo! Procurais a Jesus Nazareno, que foi crucificado? Ressuscitou, não está aqui: eis o lugar onde fora colocado! Ide, pois, dizer a seus discípulos e a Pedro, que Ele vos precede na Galileia e lá O vereis, como vos disse».
}\end{paracol}

\paragraphinfo{Ofertório}{Sl. 75, 9-10}
\begin{paracol}{2}\latim{
\rlettrine{T}{erra} trémuit, et quiévit, dum resúrgeret in judício Deus, allelúja.
}\switchcolumn\portugues{
\rlettrine{A}{} terra tremeu e aquietou-se, logo que Deus se ergueu para a julgar, aleluia.
}\end{paracol}

\paragraph{Secreta}
\begin{paracol}{2}\latim{
\rlettrine{S}{úscipe,} quǽsumus, Dómine, preces pópuli tui cum oblatiónibus hostiárum: ut, paschálibus initiáta mystériis, ad æternitátis nobis medélam, te operánte, profíciant. Per Dóminum \emph{\&c.}
}\switchcolumn\portugues{
\rlettrine{S}{enhor,} dignai-Vos aceitar com a oferta destas hóstias as preces do vosso povo, para que os mystérios da Páscoa, agora iniciados, nos sirvam, com vosso auxílio, de remédio para a eternidade. Por nosso Senhor \emph{\&c.}
}\end{paracol}

\paragraphinfo{Comúnio}{1. Cor. 5, 7-8}
\begin{paracol}{2}\latim{
\rlettrine{P}{ascha} nostrum immolátus est Christus, allelúja: itaque epulémur in ázymis sinceritátis et veritátis, allelúja, allelúja, allelúja.
}\switchcolumn\portugues{
\rlettrine{C}{risto,} nossa Páscoa, foi imolado, aleluia. Celebremos, pois, a Páscoa com os ázimos da sinceridade e verdade, aleluia, aleluia, aleluia.
}\end{paracol}

\paragraph{Postcomúnio}
\begin{paracol}{2}\latim{
\rlettrine{S}{píritum} nobis, Dómine, tuæ caritátis infúnde: ut, quos sacraméntis paschálibus satiásti, tua fácias pietáte concordes. Per Dóminum \emph{\&c.}
}\switchcolumn\portugues{
\rlettrine{I}{nfundi} em nós, Senhor, o espírito da vossa caridade, para que todos aqueles que foram alimentados com vossos sacramentos nesta festa pascal vivam unidos em perfeita concórdia, pela vossa bondade. Por nosso Senhor \emph{\&c.}
}\end{paracol}

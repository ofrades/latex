\subsectioninfo{Dias de Rogação}{25 de Abril e nos 3 dias antes da Ascensão}\label{rogacoes}

\subsubsection{Procissão}

\paragraph{Antífona}
\begin{paracol}{2}\latim{
\rlettrine{E}{xsurge,} Dómine, ádjuva nos, et líbera nos propter nomen tuum. \emph{Psalm} Deus, áuribus nostris audívimus, patres nostri annuntiavérunt nobis.
℣. Glória ao Pai \emph{\&c.}
}\switchcolumn\portugues{
\slettrine{L}{evantai-Vos,} ó Senhor, ajudai-nos, e resgatai-nos por causa do vosso nome. \emph{Salmo} Nós ouvimos, ó Deus, com os nossos próprios ouvidos, anunciaram-nos nossos pais a obra que fizestes nos seus dias e nos dias antigos.
℣. Glória ao Pai \emph{\&c.}
}\end{paracol}

\textit{Salmo 69 encontra-se logo após a Ladaínha}

\paragraph{Ladainha de Todos os Santos}
\begin{paracol}{2}\latim{
Kyrie eleison
}\switchcolumn\portugues{
Senhor, tende piedade de nós.
}\switchcolumn*\latim{
Christe, eléison.
}\switchcolumn\portugues{
Cristo, tende piedade de nós.
}\switchcolumn*\latim{
Kyrie, eléison.
}\switchcolumn\portugues{
Senhor, tende piedade de nós.
}\switchcolumn*\latim{
Christe, audi nos.
}\switchcolumn\portugues{
Cristo, ouvi-nos.
}\switchcolumn*\latim{
Christe, exáudi nos.
}\switchcolumn\portugues{
Cristo, atendei-nos.
}\switchcolumn*\latim{
Pater de cælis, Deus, miserére nobis
}\switchcolumn\portugues{
Deus pai do céu, tende piedade de nós.
}\switchcolumn*\latim{
Fili, Redémptor mundi, Deus,\footnote[2]{miserere nobis.\label{miserere}}
}\switchcolumn\portugues{
Filho Redentor do mundo, que sois Deus,\footnote[2]{tende piedade de nós.\label{tende}}
}\switchcolumn*\latim{
Spíritus Sancte, Deus,\footref{miserere}
}\switchcolumn\portugues{
Deus Espírito Santo,\footref{tende}
}\switchcolumn*\latim{
Sancta Trínitas, unus Deus,\footref{miserere}
}\switchcolumn\portugues{
Santíssima Trindade, que sois um só Deus,\footref{tende}
}\switchcolumn*\latim{
Sancta María, ora pro nobis.
}\switchcolumn\portugues{
Santa Maria, rogai por nós.
}\switchcolumn*\latim{
Sancta Dei Génetrix,\footnote[8]{ora pro nobis.\label{ora}}
}\switchcolumn\portugues{
Santa Mãe de Deus,\footnote[8]{rogai por nós.\label{rogai}}
}\switchcolumn*\latim{
Sancta Virgo vírginum,\footref{ora}
}\switchcolumn\portugues{
Santa Virgem das virgens,\footref{rogai}
}\switchcolumn*\latim{
Sancte Michæl,\footref{ora}
}\switchcolumn\portugues{
São Miguel,\footref{rogai}
}\switchcolumn*\latim{
Sancte Gabriel,\footref{ora}
}\switchcolumn\portugues{
São Gabriel,\footref{rogai}
}\switchcolumn*\latim{
Sancte Raphæl,\footref{ora}
}\switchcolumn\portugues{
São Rafæl,\footref{rogai}
}\switchcolumn*\latim{
Omnes sancti Angeli et Archangeli,\footnote[3]{orate pro nobis.\label{orate}}
}\switchcolumn\portugues{
Todos os santos Anjos e Arcanjos,\footref{rogai}
}\switchcolumn*\latim{
Omnes sancti beatórum Spírituum ordines,\footref{orate}
}\switchcolumn\portugues{
Todos as santas ordens de Espíritos bem-aventurados,\footref{rogai}
}\switchcolumn*\latim{
Sancte Joánnes Baptista,\footref{ora}
}\switchcolumn\portugues{
São João Batista,\footref{rogai}
}\switchcolumn*\latim{
Sancte Josephe,\footref{ora}
}\switchcolumn\portugues{
São José,\footref{rogai}
}\switchcolumn*\latim{
Omnes sancti Patriárchæ et Prophetæ,\footref{orate}
}\switchcolumn\portugues{
Todos os santos patriarcas e profetas,\footref{rogai}
}\switchcolumn*\latim{
Sancte Petre,\footref{ora}
}\switchcolumn\portugues{
São Pedro,\footref{rogai}
}\switchcolumn*\latim{
Sancte Paule,\footref{ora}
}\switchcolumn\portugues{
São Paulo,\footref{rogai}
}\switchcolumn*\latim{
Sancte Andrea,\footref{ora}
}\switchcolumn\portugues{
Santo André,\footref{rogai}
}\switchcolumn*\latim{
Sancte Jacobe,\footref{ora}
}\switchcolumn\portugues{
São Tiago,\footref{rogai}
}\switchcolumn*\latim{
Sancte Joánnes,\footref{ora}
}\switchcolumn\portugues{
São João,\footref{rogai}
}\switchcolumn*\latim{
Sancte Thoma,\footref{ora}
}\switchcolumn\portugues{
São Tomé,\footref{rogai}
}\switchcolumn*\latim{
Sancte Jacobe,\footref{ora}
}\switchcolumn\portugues{
São Tiago,\footref{rogai}
}\switchcolumn*\latim{
Sancte Philippe,\footref{ora}
}\switchcolumn\portugues{
São Felipe,\footref{rogai}
}\switchcolumn*\latim{
Sancte Bartholomæe,\footref{ora}
}\switchcolumn\portugues{
São Bartolomeu,\footref{rogai}
}\switchcolumn*\latim{
Sancte Matthæe,\footref{ora}
}\switchcolumn\portugues{
São Mateus,\footref{rogai}
}\switchcolumn*\latim{
Sancte Simon,\footref{ora}
}\switchcolumn\portugues{
São Simão,\footref{rogai}
}\switchcolumn*\latim{
Sancte Thaddæe,\footref{ora}
}\switchcolumn\portugues{
São Tadeu,\footref{rogai}
}\switchcolumn*\latim{
Sancte Matthia,\footref{ora}
}\switchcolumn\portugues{
São Matias,\footref{rogai}
}\switchcolumn*\latim{
Sancte Barnaba,\footref{ora}
}\switchcolumn\portugues{
São Barnabé,\footref{rogai}
}\switchcolumn*\latim{
Sancte Luca,\footref{ora}
}\switchcolumn\portugues{
São Lucas,\footref{rogai}
}\switchcolumn*\latim{
Sancte Marce,\footref{ora}
}\switchcolumn\portugues{
São Marcos,\footref{rogai}
}\switchcolumn*\latim{
Omnes sancti Apóstoli et Evangelistæ,\footref{orate}
}\switchcolumn\portugues{
Todos os santos apóstolos e evangelistas,\footref{rogai}
}\switchcolumn*\latim{
Omnes sancti Discípuli Dómini,\footref{orate}
}\switchcolumn\portugues{
Todos os santos Discípulos do Senhor,\footref{rogai}
}\switchcolumn*\latim{
Omnes sancti Innocéntes,\footref{orate}
}\switchcolumn\portugues{
Todos os santos inocentes,\footref{rogai}
}\switchcolumn*\latim{
Sancte Stephane,\footref{ora}
}\switchcolumn\portugues{
Santo Estêvão,\footref{rogai}
}\switchcolumn*\latim{
Sancte Laurénti,\footref{ora}
}\switchcolumn\portugues{
São Lourenço,\footref{rogai}
}\switchcolumn*\latim{
Sancte Vincenti,\footref{ora}
}\switchcolumn\portugues{
São Vicente,\footref{rogai}
}\switchcolumn*\latim{
Sancti Fabiane et Sebastiane,\footref{orate}
}\switchcolumn\portugues{
Santos Fabiano e São Sebastião,\footref{rogai}
}\switchcolumn*\latim{
Sancti Joánnes et Paule,\footref{orate}
}\switchcolumn\portugues{
Santos João e Paulo,\footref{rogai}
}\switchcolumn*\latim{
Sancti Cosma et Damiane,\footref{orate}
}\switchcolumn\portugues{
Santos Cosme e Damião,\footref{rogai}
}\switchcolumn*\latim{
Sancti Gervasi et Protasi,\footref{orate}
}\switchcolumn\portugues{
Santos Gervásio e Protásio,\footref{rogai}
}\switchcolumn*\latim{
Omnes sancti Mártyres,\footref{orate}
}\switchcolumn\portugues{
Todos os santos Mártires,\footref{rogai}
}\switchcolumn*\latim{
Sancte Silvester,\footref{ora}
}\switchcolumn\portugues{
São Silvestre,\footref{rogai}
}\switchcolumn*\latim{
Sancte Gregóri,\footref{ora}
}\switchcolumn\portugues{
São Gregório,\footref{rogai}
}\switchcolumn*\latim{
Sancte Ambrósi,\footref{ora}
}\switchcolumn\portugues{
Santo Ambrósio,\footref{rogai}
}\switchcolumn*\latim{
Sancte Augustine,\footref{ora}
}\switchcolumn\portugues{
Santo Agostinho,\footref{rogai}
}\switchcolumn*\latim{
Sancte Hieronyme,\footref{ora}
}\switchcolumn\portugues{
São Jerônimo,\footref{rogai}
}\switchcolumn*\latim{
Sancte Martine,\footref{ora}
}\switchcolumn\portugues{
São Martinho,\footref{rogai}
}\switchcolumn*\latim{
Sancte Nicolaë,\footref{ora}
}\switchcolumn\portugues{
São Nicolau,\footref{rogai}
}\switchcolumn*\latim{
Omnes sancti Pontifices et Confessores,\footref{orate}
}\switchcolumn\portugues{
Todos os santos pontífices e confessores,\footref{rogai}
}\switchcolumn*\latim{
Omnes sancti Doctores,\footref{orate}
}\switchcolumn\portugues{
Todos os santos doutores,\footref{rogai}
}\switchcolumn*\latim{
Sancte Antoni,\footref{ora}
}\switchcolumn\portugues{
Santo Antônio,\footref{rogai}
}\switchcolumn*\latim{
Sancte Benedicte,\footref{orate}
}\switchcolumn\portugues{
São Bento,\footref{rogai}
}\switchcolumn*\latim{
Sancte Bernarde,\footref{ora}
}\switchcolumn\portugues{
São Bernardo,\footref{rogai}
}\switchcolumn*\latim{
Sancte Dominice,\footref{ora}
}\switchcolumn\portugues{
São Domingos,\footref{rogai}
}\switchcolumn*\latim{
Sancte Francisce,\footref{ora}
}\switchcolumn\portugues{
São Francisco,\footref{rogai}
}\switchcolumn*\latim{
Omnes sancti Sacerdótes et Levitæ,\footref{orate}
}\switchcolumn\portugues{
Todos os santos sacerdotes e levitas,\footref{rogai}
}\switchcolumn*\latim{
Omnes sancti Monachi et Eremitæ,\footref{orate}
}\switchcolumn\portugues{
Todos os santos Monges e eremitas,\footref{rogai}
}\switchcolumn*\latim{
Sancta María Magdalena,\footref{ora}
}\switchcolumn\portugues{
Santa Maria Madalena,\footref{rogai}
}\switchcolumn*\latim{
Sancta Agatha,\footref{ora}
}\switchcolumn\portugues{
Santa Águeda,\footref{rogai}
}\switchcolumn*\latim{
Sancta Lucia,\footref{ora}
}\switchcolumn\portugues{
Santa Lúcia,\footref{rogai}
}\switchcolumn*\latim{
Sancta Agnes,\footref{ora}
}\switchcolumn\portugues{
Santa Inês,\footref{rogai}
}\switchcolumn*\latim{
Sancta Cæcilia,\footref{ora}
}\switchcolumn\portugues{
Santa Cecília,\footref{rogai}
}\switchcolumn*\latim{
Sancta Catharina,\footref{ora}
}\switchcolumn\portugues{
Santa Catarina,\footref{rogai}
}\switchcolumn*\latim{
Sancta Anastasia,\footnote[8]{ora pro nobis.\label{ora}}
}\switchcolumn\portugues{
Santa Anastasia,\footref{rogai}
}\switchcolumn*\latim{
Omnes sanctæ Vírgines et Víduæ,\footnote[3]{orate pro nobis.\label{orate}}
}\switchcolumn\portugues{
Todas as Santas Virgens e Viúvas,\footnote[8]{rogai por nós.\label{rogai}}
}\switchcolumn*\latim{
Omnes Sancti et Sanctæ Dei, intercédite pro nobis.
}\switchcolumn\portugues{
Todas os Santos e Santas de Deus, intercedam por nós.
}\switchcolumn*\latim{
Propitius esto, parce nobis, Dómine.
}\switchcolumn\portugues{
Sede-nos propício, perdoai-nos, Senhor.
}\switchcolumn*\latim{
Propitius esto, Exáudi nos, Dómine.
}\switchcolumn\portugues{
Sede-nos propício, Atendei-nos, Senhor.
}\switchcolumn*\latim{
Ab omni malo, líbera nos, Dómine.
}\switchcolumn\portugues{
De todo o mal, livrai-nos, Senhor.
}\switchcolumn*\latim{
Ab omni peccáto,\footnote[4]{líbera nos, Dómine.\label{libera}}
}\switchcolumn\portugues{
De todo o pecado,\footnote[4]{livrai-nos, Senhor.\label{livrai}}
}\switchcolumn*\latim{
Ab ira tua,\footref{libera}
}\switchcolumn\portugues{
Da sua ira,\footref{livrai}
}\switchcolumn*\latim{
A subitanea et improvisa morte,\footref{libera}
}\switchcolumn\portugues{
Da morte repentina e imprevista,\footref{livrai}
}\switchcolumn*\latim{
Ab insídiis diaboli,\footref{libera}
}\switchcolumn\portugues{
Das ciladas do demônio,\footref{livrai}
}\switchcolumn*\latim{
Ab ira, et ódio, et omni mala voluntáte,\footref{libera}
}\switchcolumn\portugues{
De toda a ira, ódio e má vontade,\footref{livrai}
}\switchcolumn*\latim{
A spíritu fornicatiónis,\footref{libera}
}\switchcolumn\portugues{
Do espírito da fornicação,\footref{livrai}
}\switchcolumn*\latim{
A fulgure et tempestáte,\footref{libera}
}\switchcolumn\portugues{
Do raio e da tempestade,\footref{livrai}
}\switchcolumn*\latim{
A flagello terræmotus,\footref{libera}
}\switchcolumn\portugues{
Do flagelo do terremoto,\footref{livrai}
}\switchcolumn*\latim{
A peste, fame et bello,\footref{libera}
}\switchcolumn\portugues{
Da peste da fome e da guerra,\footref{livrai}
}\switchcolumn*\latim{
A morte perpetua,\footref{libera}
}\switchcolumn\portugues{
Da morte eterna,\footref{livrai}
}\switchcolumn*\latim{
Per mystérium sanctæ Incarnatiónis tuæ,\footref{libera}
}\switchcolumn\portugues{
Pelo mystério de vossa santa encarnação.
}\switchcolumn*\latim{
Per advéntum tuum,\footref{libera}
}\switchcolumn\portugues{
Pela vossa vinda,\footref{livrai}
}\switchcolumn*\latim{
Per nativitátem tuam,\footref{libera}
}\switchcolumn\portugues{
Pelo vosso nascimento,\footref{livrai}
}\switchcolumn*\latim{
Per baptismum et sanctum jejunium tuum,\footref{libera}
}\switchcolumn\portugues{
Por vosso batismo e santo jejum,\footref{livrai}
}\switchcolumn*\latim{
Per crucem et passiónem tuam,\footref{libera}
}\switchcolumn\portugues{
Por vossa cruz e paixão,\footref{livrai}
}\switchcolumn*\latim{
Per mortem et sepultúram tuam,\footref{libera}
}\switchcolumn\portugues{
Por vossa morte e sepultura,\footref{livrai}
}\switchcolumn*\latim{
Per sanctam resurrectiónem tuam,\footref{libera}
}\switchcolumn\portugues{
Por vossa santa ressurreição,\footref{livrai}
}\switchcolumn*\latim{
Per admirábilem ascensiónem tuam,\footref{libera}
}\switchcolumn\portugues{
Por vossa admirável ascensão,\footref{livrai}
}\switchcolumn*\latim{
Per advéntum Spíritus Sancti Paracliti,\footref{libera}
}\switchcolumn\portugues{
Pela vinda do Espírito Santo Consolador,\footref{livrai}
}\switchcolumn*\latim{
In die judícii,\footnote[4]{líbera nos, Dómine.\label{libera}}
}\switchcolumn\portugues{
No dia do juízo,\footnote[4]{livrai-nos, Senhor.\label{livrai}}
}\switchcolumn*\latim{
Peccatóres, te rogamus, audi nos.
}\switchcolumn\portugues{
Pecadores que somos, nós vos rogamos: ouvi-nos.
}\switchcolumn*\latim{
Ut nobis parcas,\footnote[1]{te rogamus, audi nos.\label{terogamus}}
}\switchcolumn\portugues{
Que nos perdoeis,\footnote[1]{nós vos rogamos: ouvi-nos.\label{nosvos}}
}\switchcolumn*\latim{
Ut nobis indulgeas,\footref{terogamus}
}\switchcolumn\portugues{
Que useis de indulgência conosco,\footref{nosvos}
}\switchcolumn*\latim{
Ut ad veram pœniténtiam nos perducere dignéris,\footref{terogamus}
}\switchcolumn\portugues{
Que nos digneis conduzi-nos a verdadeira penitência,\footref{nosvos}
}\switchcolumn*\latim{
Ut Ecclésiam tuam sanctam regere et conservare dignéris,\footref{terogamus}
}\switchcolumn\portugues{
Que nos digneis reagir e conservar a vossa santa igreja,\footref{nosvos}
}\switchcolumn*\latim{
Ut domnum Apostolicum et omnes ecclesiásticos ordines in sancta religióne conservare dignéris,\footref{terogamus}
}\switchcolumn\portugues{
Que nos digneis conservar a vossa santa religião o Senhor Apostólico e a todos as ordens da hierarquia eclesiástica,\footref{nosvos}
}\switchcolumn*\latim{
Ut inimícos sanctæ Ecclésiæ humiliare dignéris,\footref{terogamus}
}\switchcolumn\portugues{
Que nos digneis humilhar os inimigos da santa igreja,\footref{nosvos}
}\switchcolumn*\latim{
Ut régibus et princípibus christiánis pacem et veram concordiam donare dignéris,\footref{terogamus}
}\switchcolumn\portugues{
Que nos digneis conceder a verdadeira paz e concórdia entre os reis e príncipes cristãos,\footref{nosvos}
}\switchcolumn*\latim{
Ut cuncto pópulo christiáno pacem et unitátem largiri dignéris,\footref{terogamus}
}\switchcolumn\portugues{
Que nos digneis conceder a paz e a união a todo o povo cristão,\footref{nosvos}
}\switchcolumn*\latim{
Ut omnes errántes ad unitátem Ecclésiæ revocare, et infidéles univérsos ad Evangélii lumen perducere dignéris,\footref{terogamus}
}\switchcolumn\portugues{
Que nos digneis chamar à unidade da Igreja, a todos os que estão alheios a ela, para iluminar todos os infiéis com a luz do Evangelho,\footref{nosvos}
}\switchcolumn*\latim{
Ut nosmetípsos in tuo sancto servítio confortare et conservare dignéris,\footref{terogamus}
}\switchcolumn\portugues{
Que vos digneis confortar-nos e conservar-nos em vosso santo serviço,\footref{nosvos}
}\switchcolumn*\latim{
Ut mentes nostras ad cæléstia desidéria erigas,\footref{terogamus}
}\switchcolumn\portugues{
Que levanteis nossos corações a desejar as cousas celestiais,\footref{nosvos}
}\switchcolumn*\latim{
Ut ómnibus benefactóribus nostris sempitérna bona retríbuas,\footref{terogamus}
}\switchcolumn\portugues{
Que nos digneis retribuir, com os bens eternos a todos nossos benfeitores,\footref{nosvos}
}\switchcolumn*\latim{
Ut ánimas nostras, fratrum, propinquorum et benefactórum nostrórum ab ætérna damnatióne erípias,\footref{terogamus}
}\switchcolumn\portugues{
Que livreis da morte eterna nossas almas e as de nossos irmãos, parentes e benfeitores,\footref{nosvos}
}\switchcolumn*\latim{
Ut fructus terræ dare et conservare dignéris,\footref{terogamus}
}\switchcolumn\portugues{
Que nos digneis dar e conservar os frutos da terra,\footref{nosvos}
}\switchcolumn*\latim{
Ut ómnibus fidelibus defunctis réquiem ætérnam donare dignéris,\footref{terogamus}
}\switchcolumn\portugues{
 Que nos digneis conceder o eterno descanso a todos os fiéis defuntos,\footref{nosvos}
}\switchcolumn*\latim{
Ut nos exáudire dignéris,\footref{terogamus}
}\switchcolumn\portugues{
Que nos digneis atender-nos,\footref{nosvos}
}\switchcolumn*\latim{
Fili Dei,\footnote[1]{te rogamus, audi nos.\label{terogamus}}
}\switchcolumn\portugues{
Filho de Deus,\footnote[1]{nós vos rogamos: ouvi-nos.\label{nosvos}}
}\switchcolumn*\latim{
Agnus Dei, qui tollis peccáta mundi, parce nobis, Dómine.
}\switchcolumn\portugues{
Cordeiro de Deus que tirais os pecados do mundo, perdoai-nos, Senhor.
}\switchcolumn*\latim{
Agnus Dei, qui tollis peccáta mundi, exáudi nos, Dómine.
}\switchcolumn\portugues{
Cordeiro de Deus que tirais os pecados do mundo, atendei-nos, Senhor.
}\switchcolumn*\latim{
Agnus Dei, qui tollis peccáta mundi, miserére nobis.
}\switchcolumn\portugues{
Cordeiro de Deus que tirais os pecados do mundo, tende piedade de nós.
}\switchcolumn*\latim{
Christe, audi nos.
}\switchcolumn\portugues{
Cristo, ouvi-nos.
}\switchcolumn*\latim{
Christe, exáudi nos.
}\switchcolumn\portugues{
Cristo, atendei-nos.
}\switchcolumn*\latim{
Kyrie, eléison.
}\switchcolumn\portugues{
Senhor, tende piedade de nós.
}\switchcolumn*\latim{
Christe, eléison. Kyrie, eléison.
}\switchcolumn\portugues{
Cristo, tende piedade de nós. Senhor, tende piedade de nós
}\switchcolumn*\latim{
Pater noster (secréto)
}\switchcolumn\portugues{
Pai nosso (silêncio)
}\switchcolumn*\latim{
Et ne nos indúcas in tentatiónem.
}\switchcolumn\portugues{
E não nos dexeis cair em tentação.
}\switchcolumn*\latim{
Sed líbera nos a malo.
}\switchcolumn\portugues{
Mais livrai-nos do mal.
}\end{paracol}

\paragraphinfo{Salmo 69}{Página \pageref{salmo69}}

\begin{paracol}{2}\latim{
Glória Patri, \&c.
}\switchcolumn\portugues{
Glória ao Pai, \&c.
}\switchcolumn*\latim{
℣. Salvos fac servos tuos.
}\switchcolumn\portugues{
℣. Meu Deus, salvai os vossos servos.
}\switchcolumn*\latim{
℟. Deus meus, sperántes in te.
}\switchcolumn\portugues{
℟. Quem esperam em Vós.
}\switchcolumn*\latim{
℣. Esto nobis, Dómine, turris fortitúdinis.
}\switchcolumn\portugues{
℣. Sede para nós, Senhor, uma torre forte.
}\switchcolumn*\latim{
℟. A fácie inimíci.
}\switchcolumn\portugues{
℟. Contra os ataques do inimigo.
}\switchcolumn*\latim{
℣. Nihil profíciat intimícus in nobis.
}\switchcolumn\portugues{
℣. Nada possa o inimigo contra nós.
}\switchcolumn*\latim{
℟. Et fílius iniquitátis non appónat nocére nobis.
}\switchcolumn\portugues{
℟. E o filho da iniquidade não consiga fazer-nos mal.
}\switchcolumn*\latim{
℣. Dómine, non secúndum peccáta nostra fácias nobis.
}\switchcolumn\portugues{
℣. Senhor, nos não trateis como merecem os nossos pecados.
}\switchcolumn*\latim{
℟. Neque secúndum iniquitátes nostras retribuas nobis.
}\switchcolumn\portugues{
℟. Nem nos castigueis como pedem as nossas iniquidades.
}\switchcolumn*\latim{
℣. Orémus pro Pontífice nostro {\redx N.}
}\switchcolumn\portugues{
℣. Oremos pelo nosso pontífice {\redx N.}
}\switchcolumn*\latim{
℟. Dóminus consérvet eum, et vivíficet eum, et beátum fáciat eum in terra, et non tradat eum in ánimam inimicórum éjus.
}\switchcolumn\portugues{
℟. O Senhor o conserve, lhe dê vida, o faça feliz na terra e o não entregue à violência dos seus inimigos.
}\switchcolumn*\latim{
℣. Orémus pro benefactóribus nostris.
}\switchcolumn\portugues{
℣. Oremos pelos nossos benfeitores.
}\switchcolumn*\latim{
℟. Retribúere dignare, Dómine, ómnibus nobis bona faciéntibus, propter nomen tuum, vitam ætérnam. ℟. Amen.
}\switchcolumn\portugues{
℟. Dignai-Vos, Senhor, para glória do vosso Nome, conceder a vida eterna a todos os que nos fazem bem. Amen
}\switchcolumn*\latim{
℣. Orémus pro fidélibus defúnctis.
}\switchcolumn\portugues{
℣. Oremos pelos fiéis defuntos.
}\switchcolumn*\latim{
℟. Réquiem ætérnam dona eis, Dómine, et lux perpétua lúceat eis.
}\switchcolumn\portugues{
℟. Dai-lhes, Senhor, o eterno descanso, entre os esplendores da luz perpétua.
}\switchcolumn*\latim{
℣. Requiéscant in pace.
}\switchcolumn\portugues{
℣. Descansem em paz.
}\switchcolumn*\latim{
℟. Amen.
}\switchcolumn\portugues{
℟. Amen.
}\switchcolumn*\latim{
℣. Pro frátribus nostris abséntibus.
}\switchcolumn\portugues{
℣. Oremos pelos nossos irmãos ausentes.
}\switchcolumn*\latim{
℟. Salvos fac servos tuos, Deus meus, sperántes in te.
}\switchcolumn\portugues{
℟. Salvai, meu Deus, os vossos servos que esperam em Vós.
}\switchcolumn*\latim{
℣. Mitte eis, Dómine, auxílium de sancto.
}\switchcolumn\portugues{
℣. Socorrei-os, Senhor, lá do vosso santuário.
}\switchcolumn*\latim{
℟. Et de Sion tuére eos.
}\switchcolumn\portugues{
℟. E protegei-os da celestial Sião.
}\switchcolumn*\latim{
℣. Dómine, exáudi oratiónem meam.
}\switchcolumn\portugues{
℣. Ouvi, Senhor, a minha oração.
}\switchcolumn*\latim{
℟. Et clamor meus ad te vénita.
}\switchcolumn\portugues{
℟. E o meu clamor chegue até Vós.
}\switchcolumn*\latim{
℣. Dóminus vobíscum.
}\switchcolumn\portugues{
℣. O Senhor seja convosco.
}\switchcolumn*\latim{
℟. Et cum spíritu tuo.
}\switchcolumn\portugues{
℟. E com vosso espírito.
}\switchcolumn*\latim{
\begin{nscenter} Orémus. \end{nscenter}
}\switchcolumn\portugues{
\begin{nscenter} Oremos. \end{nscenter}
}\switchcolumn*\latim{
\rlettrine{D}{eus,} cui proprium est misereri semper et parcere: suscipe deprecationem nostram; ut nos, et omnes famulos tuos, quos delictorum catena constringit, miseratio tuæ pietatis clementer absolvat.
}\switchcolumn\portugues{
\slettrine{Ó}{} Deus, de quem sempre é próprio compadecer-Vos e perdoar, recebei a nossa humilde súplica; e fazei, por benefício da vossa clementíssima piedade, que, assim nós, como os outros vossos servos, sejamos inteiramente livres da injúriosa cadeia dos nossos delitos.
}\switchcolumn*\latim{
Exaudi, quǽsumus, Domine, supplicum preces, et confitentium tibi parce peccatis: ut pariter nobis indulgentiam tribuas benignus et pacem.
}\switchcolumn\portugues{
Ouvi, Senhor, os humildes rogos e perdoai todos os pecados dos que Vo-lo confessam para que ao mesmo tempo recebamos da vossa bondade o perdão e a graça duma completa paz.
}\switchcolumn*\latim{
Ineffabilem nobis, Domine, misericordiam tuam clementer ostende: ut simul nos et a peccatis omnibus exuas, et a poenis quas pro his meremur, eripias.
}\switchcolumn\portugues{
Ostentai sobre nós, Senhor, a vossa inefável misericórdia, de modo que, absolvendo-nos de todos nossos pecados, nos livreis juntamente das gravíssimas penas, que por eles havemos merecido.
}\switchcolumn*\latim{
Deus, qui culpa offenderis, pænitentia placaris: preces populi tui supplicantis propitius respice; et flagella tuæ iracundiæ, quæ pro peccatis nostris meremur, averte.
}\switchcolumn\portugues{
Ó Deus, a quem a culpa ofende e a penitência aplaca, recebei propício as humildes súplicas do vosso povo e apartai de nós os flagelos da vossa ira, que merecemos pelas nossas culpas.
}\switchcolumn*\latim{
Omnipotens sempiterne Deus, miserere famulo tuo Pontifici nostro {\redx N.}, et dirige eum secundum tuam clementiam in viam salutis æternæ: ut, te donante, tibi placita cupiat, et tota virtute perficiat.
}\switchcolumn\portugues{
Ó Deus eterno e omnipotente, tende piedade do vosso servo, o nosso Santo Padre {\redx N.} e conduzi-o, segundo a clemência, pelo caminho da salvação eterna para que, mediante a vossa graça execute sempre com todo o esforço o que for mais do vosso agrado.
}\switchcolumn*\latim{
Deus, a quo sancta desideria, recta consilia, et iusta sunt opera: da servis tuis illam, quam mundus dare non potest, pacem; ut et corda nostra mandatis tuis dedita, et, hostium sublata formidine, tempora sint tua protectione tranquilla.
}\switchcolumn\portugues{
Ó Deus, de quem dependem os santos desejos, os rectos conselhos e as boas obras, concedei aos aos vossos servos aquela paz que o mundo não pode dar, para que, aplicados os nossos corações à observância dos vossos preceitos e desterrado o temor dos vossos inimigos, gozemos com vossa protecção, em nossos dias, uma feliz tranquilidade.
}\switchcolumn*\latim{
Ure igne Sancti Spiritus renes nostros et cor nostrum, Domine: ut tibi casto corpore serviamus, et mundo corde placeamus.
}\switchcolumn\portugues{
Queimai, Senhor, os nossos rins e o nosso coração com o fogo do Espírito Santo, para que Vos servamos com o corpo casto e Vos agrademos com a pureza do nosso coração.
}\switchcolumn*\latim{
Fidelium, Deus omnium Conditor et Redemptor, animabus famulorum famularumque tuarum remissionem cunctorum tribue peccatorum: ut indulgentiam, quam semper optaverunt, piis supplicationibus consequantur
}\switchcolumn\portugues{
Ó Deus, criador e rendentor de todos os fiéis, concedei ás almas dos vossos servos e servas a benigna remissão dos seus pecados, para que alcancem pelas pais súplicas da vossa Igreja o perdão a que sempre aspiraram.
}\switchcolumn*\latim{
Actiones nostras, quǽsumus, Domine, aspirando præveni et adiuvando prosequere: ut cuncta oratio et operatio a te semper incipiat et per te coepta finiatur.
}\switchcolumn\portugues{
Vos suplicamos, Senhor, que Vos antecipeis a insuflar e a ajudar as nossas obras, para que todas nossas orações sempre de Vós procedam e em Vós se completem.
}\switchcolumn*\latim{
Omnipotens sempiterne Deus, qui vivorum dominaris simul et mortuorum, omniumque misereris, quos tuos fide et opere futuros esse prænoscis: te supplices exoramus; ut pro quibus effundere preces decrevimus, quosque vel præsens sæculum adhuc in carne retinet vel futurum iam exutos corpore suscepit, intercedentibus omnibus Sanctis tuis, pietatis tuæ clementia, omnium delictorum suorum veniam consequantur. Per Dominum nostrum Jesum Christum. Amen.
}\switchcolumn\portugues{
Ó Deus omnipotente e eterno, que sois supremo Senhor dos vivos e dos mortos e usais de misericórdia para com todos aqueles que pela fé e boas obras antecipadamente conheceis que serão do glorioso número dos vossos predestinados, Vos suplicamos, que os mesmos por quem Vos pedimos, ou estejam ainda em carne mortal neste mundo ou, despidos já dos seus corpos, hajam passado para a outra vida, alcancem da vossa pia clemência, pela intercessão de todos vossos Santos, o benigno perdão dos seus pecados. Por Jesus Cristo, vosso Filho e Senhor nosso, que convosco vive e reina em unidade de Deus Espírito Santo, por todos os séculos dos séculos. Amen.
}\switchcolumn*\latim{
℣. Dómine, exáudi orationem meam.
}\switchcolumn\portugues{
℣. Senhor, ouvi a minha oração.
}\switchcolumn*\latim{
℟. Et clamor meus ad te véniat.
}\switchcolumn\portugues{
℟. E que meu clamor chegue até Vós.
}\switchcolumn*\latim{
℣. Dóminus vobíscum.
}\switchcolumn\portugues{
℣. O Senhor seja convosco.
}\switchcolumn*\latim{
℟. Et cum spíritu tuo.
}\switchcolumn\portugues{
℟. E com vosso espírito.
}\switchcolumn*\latim{
℣. Exáudiat nos omnípotens et miséricors Dóminus.
}\switchcolumn\portugues{
℣. Que o Senhor omnipotente e misericordioso se digne ouvir-nos.
}\switchcolumn*\latim{
℟. Amen.
}\switchcolumn\portugues{
℟. Amen.
}\switchcolumn*\latim{
℣. Et fidélium ánimæ per misericórdiam Dei requiéscant in pace.
}\switchcolumn\portugues{
℣. E que as almas dos fiéis defuntos por misericórdia de Deus descansem em paz.
}\switchcolumn*\latim{
℟. Amen.
}\switchcolumn\portugues{
℟. Amen.
}\switchcolumn*\latim{
}\end{paracol}


\paragraph{Salmo 69}
\begin{paracol}{2}\latim{
\rlettrine{D}{eus,} in adjutórium meum inténde: * Dómine, ad adjuvándum me festína.
}\switchcolumn\portugues{
\slettrine{Ó}{} Deus, vinde em meu auxílio: * ó Senhor, apressai-Vos em ajudar-me.
}\switchcolumn*\latim{
Confundántur et revereántur, * qui quǽrunt ánimam meam.
}\switchcolumn\portugues{
Sejam confundidos e envergonhados, * os que a vida me procuram tirar.
}\switchcolumn*\latim{
Avertántur retrórsum, et erubéscant, * qui volunt mihi mala.
}\switchcolumn\portugues{
Deixai que recuem e sejam envergonhados, * os que mal me desejam.
}\switchcolumn*\latim{
Avertántur statim erubescéntes, * qui dicunt mihi: euge, euge.
}\switchcolumn\portugues{
Deixai que sejam imediatamente envergonhados, * os que me dizem: bem, bem!
}\switchcolumn*\latim{
Exsúltent et læténtur in Te omnes qui quǽrunt Te, * et dicant semper: magnificétur Dóminus: qui díligunt salutáre tuum.
}\switchcolumn\portugues{
Regozijem-se e alegrem-se em Vós todos os que Vos buscam, * e digam sempre os que amam a vossa salvação: glorificado seja o Senhor.
}\switchcolumn*\latim{
Ego vero egénus, et pauper sum: * Deus, ádjuva me.
}\switchcolumn\portugues{
Eu, contudo, sou necessitado e pobre: * ó Deus, ajudai-me.
}\switchcolumn*\latim{
Adjútor meus, et liberátor meus es Tu: * Dómine, ne moréris.
}\switchcolumn\portugues{
Vós sois o meu auxiliador e o meu libertador: * ó Senhor, Vos não demoreis.
}\end{paracol}


\subsection{Missa das Rogações}

\paragraphinfo{Intróito}{Sl. 17, 7}
\begin{paracol}{2}\latim{
\rlettrine{E}{xaudívit} de templo sancto suo vocem meam, allelúja: et clamor meus in conspectu ejus, introívit in aures ejus, allelúja, allelúja. \emph{Ps. ibid., 2-3} Díligam te, Dómine, virtus mea: Dóminus firmaméntum meum et refúgium meum et liberátor meus.
℣. Gloria Patri \emph{\&c.}
}\switchcolumn\portugues{
\rlettrine{L}{á} no seu santo templo ouviu Ele a minha voz, aleluia: e o meu clamor chegou à sua presença e soou a seus ouvidos: aleluia, aleluia. \emph{Sl. ibid., 2-3} Eu vos amo, Senhor, que sois a minha força; sois, ó Senhor, o meu sustentáculo, o meu refúgio e a minha salvação.
℣. Glória ao Pai \emph{\&c.}
}\end{paracol}

\paragraph{Oração}
\begin{paracol}{2}\latim{
\rlettrine{P}{ræsta,} quǽsumus, omnípotens Deus: ut, qui in afflictióne nostra de tua pietáte confídimus; contra advérsa ómnia, tua semper protectióne muniámur. Per Dóminum nostrum \emph{\&c.}
}\switchcolumn\portugues{
\slettrine{Ó}{} Deus omnipotente, Vos suplicamos, concedei-nos a graça de confiarmos sempre na vossa bondade no meio das nossas tribulações e de estarmos sempre munidos com vosso socorro no meio das adversidades. Por nosso Senhor \emph{\&c.}
}\end{paracol}

\paragraphinfo{Epístola}{Tg. 5, 16-20}
\begin{paracol}{2}\latim{
Léctio Epístolæ beáti Jacóbi Apóstoli.
}\switchcolumn\portugues{
Lição da Ep.ª do B. Ap.º Tiago.
}\switchcolumn*\latim{
\rlettrine{C}{aríssimi:} Confitémini altérutrum peccáta vestra, et oráte pro ínvicem, ut salvémini: multum enim valet deprecátio justi assídua. Elías homo erat símilis nobis passíbilis: et oratióne orávit, ut non plúeret super terram, et non pluit annos tres et menses sex. Et rursum orávit: et cœlum dedit plúviam et terra dedit fructum suum. Fratres mei, si quis ex vobis erráverit a veritáte et convérterit quis eum: scire debet, quóniam, qui convérti fécerit peccatórem ab erróre viæ suæ, salvábit ánimam ejus a morte, et opériet multitúdinem peccatórum.
}\switchcolumn\portugues{
\rlettrine{C}{aríssimos:} Confessai as vossas culpas uns aos outros e orai uns pelos outros, a fim de que sejais salvos; pois a oração assídua do justo vale muito. Elias era um homem sujeito às mesmas misérias do que nós. Pois ele pediu instantemente que não chovesse, e durante três anos e seis meses não choveu. E tornou a orar, pedindo chuva, e esta caiu do céu e a terra produziu frutos. Meus irmãos: se algum de vós se afastou da verdade e alguém o tenha convertido, saiba que aquele que afastar um pecador do caminho do erro salvará a sua alma da morte e apagará uma multidão de pecados.
}\end{paracol}

\begin{paracol}{2}\latim{
Allelúja. ℣. \emph{Ps. 117, 1} Confitémini Dómino, quóniam bonus: quóniam in sǽculum misericórdia ejus.
}\switchcolumn\portugues{
Aleluia. ℣. \emph{Sl. 117, 1} Louvai o Senhor, pois Ele é bom: a sua misericórdia é eterna.
}\end{paracol}

\paragraphinfo{Evangelho}{Lc. 11, 5-13}
\begin{paracol}{2}\latim{
\cruz Sequéntia sancti Evangélii secúndum Lucam.
}\switchcolumn\portugues{
\cruz Continuação do santo Evangelho segundo S. Lucas.
}\switchcolumn*\latim{
\blettrine{I}{n} illo témpore: Dixit Jesus discípulis suis: Quis vestrum habébit amícum, et íbit ad illum média nocte, et dicet illi: Amíce, cómmoda mihi tres panes, quóniam amícus meus venit de via ad me, et non hábeo quod ponam ante illum: et ille deíntus respóndens, dicat: Noli mihi moléstus esse, jam óstium clausum est, et púeri mei mecum sunt in cubíli, non possum súrgere et dare tibi. Et si ille perseveráverit pulsans: dico vobis, etsi non dabit illi surgens, eo quod amícus ejus sit, propter improbitátem tamen ejus surget et dabit illi, quotquot habet necessários. Et ego dico vobis: Pétite, et dábitur vobis: quǽrite, et inveniétis: pulsáte, et aperiétur vobis. Omnis enim, qui petit, áccipit: et qui quærit, invénit: et pulsánti aperietur, Quis autem ex vobis patrem pétii panem, numquid lápidem dabit illi? Aut piscem: numquid pro pisce serpéntem dabit illi? Aut si petíerit ovum: numquid pórriget illi scorpiónem? Si ergo vos, cum sitis mali, nostis bona data dare fíliis vestris: quanto magis Pater vester de cœlo dabit spíritum bonum peténtibus se?
}\switchcolumn\portugues{
\blettrine{N}{aquele} tempo, disse Jesus aos seus discípulos: «Se algum de vós tiver um amigo, e for à meia-noite encontrá-lo, dizendo-lhe: «Empresta-me três pães, porque um dos meus amigos chegou agora de viagem a minha casa e não tenho que dar-lhe» ; e esse amigo responder de dentro de casa: «Não me importunes, pois a porta já está fechada; eu e os meus filhos estamos deitados e não posso levantar-me», digo-vos que, se o primeiro continuar a bater à porta, ainda que o outro se não levante para lhe dar o pão por ser seu amigo, levantar-se-á para não ser importunado; e dar-lhe-á tanto quanto carecer. Assim vos digo eu: pedi e recebereis; buscai e encontrareis; batei e abrir-se-vos-á. Porquanto todo aquele que pedir receberá; todo aquele, que procurar achará; todo aquele que bater abrir-se-lhe-á. Se algum de vós pedir um pão a seu pai, porventura este lhe dará uma pedra? Ou, se lhe pedir um peixe, dar-lhe-á uma serpente? Ou, se lhe pedir um ovo, dar-lhe-á um escorpião? Pois se vós, sendo maus, sabeis, contudo, dar cousas boas a vossos filhos, quanto mais vosso Pai celestial dará o Espírito Santo àqueles que lho pedirem».
}\end{paracol}

\paragraphinfo{Ofertório}{Sl. 108, 30-31}
\begin{paracol}{2}\latim{
\rlettrine{C}{onfitébor} Dómino nimis in ore meo: et in médio multórum laudábo eum, qui ástitit a dextris páuperis: ut salvam fáceret a persequéntibus ánimam meam, allelúja.
}\switchcolumn\portugues{
\rlettrine{L}ouvarei{} o Senhor com a minha boca, sonoramente: louvá-l’O-ci perante a multidão: pois esteve à direita do pobre para salvar a minha alma dos que a perseguiam, aleluia.
}\end{paracol}

\paragraph{Secreta}
\begin{paracol}{2}\latim{
\rlettrine{H}{æc} múnera, quǽsumus, Dómine, et víncula nostræ pravitátis absólvant, et tuæ nobis misericórdiæ dona concílient. Per Dóminum \emph{\&c.}
}\switchcolumn\portugues{
\qlettrine{Q}{ue} estas oblações, Senhor, Vos suplicamos, nos livrem dos laços da nossa malícia e nos alcancem os dons da vossa misericórdia. Por nosso Senhor \emph{\&c.}
}\end{paracol}

\paragraphinfo{Comúnio}{Lc. 11, 9-10}
\begin{paracol}{2}\latim{
\rlettrine{P}{etite,} et accipiétis: quǽrite, et inveniétis: pulsáte, et aperiétur vobis: omnis enim qui pétii, áccipit: et qui quærit, invénit: et pulsánti aperiétur, allelúja.
}\switchcolumn\portugues{
\rlettrine{P}{edi} e recebereis; buscai e achareis; batei e abrir-se-vos-á. Pois todo o que pede recebe; todo o que procura acha; todo o que bate abrir-se-lhe-á, aleluia.
}\end{paracol}

\paragraph{Postcomúnio}
\begin{paracol}{2}\latim{
\rlettrine{V}{ota} nostra, quǽsumus, Dómine, pio favóre proséquere: ut, dum dona tua in tribulatióne percípimus, de consolatióne nostra in tuo amóre crescámus. Per Dóminum nostrum \emph{\&c.}
}\switchcolumn\portugues{
\rlettrine{D}{ignai-Vos} acolher favoravelmente, Senhor, Vos suplicamos, os nossos votos, a fim de que, recebendo os vossos dons na tribulação, cresçamos no vosso amor com a consolação que nos dais. Por nosso Senhor \emph{\&c.}
}\end{paracol}

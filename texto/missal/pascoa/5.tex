\subsection{Quinto Domingo depois da Páscoa}\label{5depoispascoa}

\paragraphinfo{Intróito}{Is. 48, 20}
\begin{paracol}{2}\latim{
\rlettrine{V}{ocem} jucunditátis annuntiáte, et audiátur, allelúja: annuntiáte usque ad extrémum terræ: liberávit Dóminus pópulum suum, allelúja, allelúja. \emph{Ps. 65, 1-2} Jubiláte Deo, omnis terra, psalmum dícite nómini ejus: date glóriam laudi ejus.
℣. Gloria Patri \emph{\&c.}
}\switchcolumn\portugues{
\rlettrine{C}{om} voz de alegria anunciai, fazei ouvir: aleluia; proclamai até aos confins da terra; o Senhor libertou o povo, aleluia, aleluia. \emph{Sl. 65, 1-2} Ó povos de toda a terra, aclamai Deus com júbilo: cantai hinos em honra do seu nome: rendei-Lhe honras e louvores.
℣. Glória ao Pai \emph{\&c.}
}\end{paracol}

\paragraph{Oração}
\begin{paracol}{2}\latim{
\rlettrine{D}{eus,} a quo bona cuncta procédunt, largíre supplícibus tuis: ut cogitémus, te inspiránte, quæ recta sunt; et, te gubernánte, eadem faciámus. Per Dóminum \emph{\&c.}
}\switchcolumn\portugues{
\slettrine{Ó}{} Deus, origem de todos os bens, dignai-Vos conceder a estes fiéis suplicantes que por vossa inspiração meditemos naquilo que é recto, e, sob o vosso império, cumpramos aquilo em que meditámos. Por nosso Senhor \emph{\&c.}
}\end{paracol}

\paragraphinfo{Epístola}{Tg. 1, 22-27}
\begin{paracol}{2}\latim{
Léctio Epístolæ beáti Jacóbi Apóstoli.
}\switchcolumn\portugues{
Lição da Ep.ª do B. Ap.º Tiago.
}\switchcolumn*\latim{
\rlettrine{C}{aríssimi:} Estóte factóres verbi, et non auditóres tantum: falléntes vosmetípsos. Quia si quis audítor est verbi et non factor: hic comparábitur viro consideránti vultum nativitátis suæ in spéculo: considerávit enim se et ábiit, et statim oblítus est, qualis fúerit. Qui autem perspéxerit in legem perfectam libertátis et permánserit in ea, non audítor obliviósus factus, sed factor óperis: hic beátus in facto suo erit. Si quis autem putat se religiósum esse, non refrénans linguam suam, sed sedúcens cor suum, hujus vana est relígio. Relígio munda et immaculáta apud Deum et Patrem hæc est: Visitáre pupíllos et viduas in tribulatióne eórum, et immaculátum se custodíre ab hoc sǽculo.
}\switchcolumn\portugues{
\rlettrine{C}{aríssimos:} Sede praticantes do ensino da palavra, e não vos contenteis em ouvi-la, pois vos enganais. Se alguém ouve a palavra e a não cumpre, é semelhante a um homem que vê em um espelho o seu rosto natural e, apenas o vê, retira-se, esquecendo logo o que era. Aquele, porém, que grava bem no espírito a lei perfeita da liberdade, e a enraíza em si, não ouvindo para logo esquecer, mas para praticar com suas acções o que ouviu, esse encontrará a felicidade nas suas acções. Se alguém pensa ser religioso, e não refreia a língua, engana o seu Próprio coração, e a sua religião é vã. A religião pura e imaculada para com o nosso Deus ai consiste em visitar os órfãos e as viúvas nas tribulações e em conservar-se puro no meio da corrupção do mundo.
}\end{paracol}

\begin{paracol}{2}\latim{
Allelúja, allelúja. ℣. Surréxit Christus, et illúxit nobis, quos rédemit sánguine suo. Allelúja. ℣. \emph{Joann. 16, 28} Exívi a Patre, et veni in mundum: íterum relínquo mundum, et vado ad Patrem. Allelúja.
}\switchcolumn\portugues{
Aleluia, aleluia. ℣. Cristo ressuscitou e fez brilhar a sua luz sobre nós, que fomos remidos com seu sangue. Aleluia. ℣. \emph{Jo. 16, 28} Eu saí do Pai e vim ao mundo: e agora deixo o mundo e volto para o Pai. Aleluia.
}\end{paracol}

\paragraphinfo{Evangelho}{Jo. 16, 23-30}
\begin{paracol}{2}\latim{
\cruz Sequéntia sancti Evangélii secúndum Joánnem.
}\switchcolumn\portugues{
\cruz Continuação do santo Evangelho segundo S. João.
}\switchcolumn*\latim{
\blettrine{I}{n} illo témpore: Dixit Jesus discípulis suis: Amen, amen, dico vobis: si quid petiéritis Patrem in nómine meo, dabit vobis. Usque modo non petístis quidquam in nómine meo: Pétite, et accipiétis, ut gáudium vestrum sit plenum. Hæc in provérbiis locútus sum vobis. Venit hora, cum jam non in provérbiis loquar vobis, sed palam de Patre annuntiábo vobis. In illo die in nómine meo petétis: et non dico vobis, quia ego rogábo Patrem de vobis: ipse enim Pater amat vos, quia vos me amástis, et credidístis quia ego a Deo exívi. Exívi a Patre et veni in mundum: íterum relínquo mundum et vado ad Patrem. Dicunt ei discípuli ejus: Ecce, nunc palam loquéris et provérbium nullum dicis. Nunc scimus, quia scis ómnia et non opus est tibi, ut quis te intérroget: in hoc crédimus, quia a Deo exísti.
}\switchcolumn\portugues{
\blettrine{N}{aquele} tempo, disse Jesus aos seus discípulos: «Em verdade, em verdade vos digo: Se pedirdes ao meu Pai alguma cousa em meu nome, Ele vo-la dará. Até agora nada pedistes em meu nome. Pedi e recebereis, para que a vossa alegria seja perfeita. Tenho-vos dito estas cousas, servindo-me de parábolas. Chegou a hora em que vos não falarei por meio de parábolas a respeito do Pai, mas claramente. Nesse dia pedireis em meu nome, e não digo que pedirei por vós ao Pai, porque o próprio Pai vos ama, pois me amais e acreditastes que saí do Pai. Eu saí do Pai e vim ao mundo; e, agora, deixo o mundo e volto para o Pai». Então os discípulos disseram-Lhe: «Agora falais claramente e sem parábolas; agora conhecemos que sabeis tudo e que não é mister que ninguém Vos interrogue. Eis porque cremos que saístes do Pai».
}\end{paracol}

\paragraphinfo{Ofertório}{Sl. 65, 8-9 \& 20.}
\begin{paracol}{2}\latim{
\rlettrine{B}{enedícite,} gentes, Dóminum, Deum nostrum, et obaudíte vocem laudis ejus: qui pósuit ánimam meam ad vitam, et non dedit commovéri pedes meos: benedíctus Dóminus, qui non amóvit deprecatiónem meam et misericórdiam suam a me, allelúja.
}\switchcolumn\portugues{
\slettrine{Ó}{} povos, bendizei o Senhor, nosso Deus, e fazei ressoar cânticos em seu louvor; pois foi Ele quem conservou a vida à minha alma e não permitiu que meus pés tropeçassem. Bendito seja Deus, que não rejeitou a minha oração, nem me faltou com sua misericórdia, aleluia.
}\end{paracol}

\paragraph{Secreta}
\begin{paracol}{2}\latim{
\rlettrine{S}{úscipe,} Dómine, fidélium preces cum oblatiónibus hostiárum: ut, per hæc piæ devotiónis offícia, ad cœléstem glóriam transeámus. Per Dóminum \emph{\&c.}
}\switchcolumn\portugues{
\rlettrine{D}{ignai-Vos,} Senhor, aceitar as preces dos fiéis juntamente com as hóstias que Vos são oferecidas; e, em recompensa deste dever da nossa pia devoção, permiti que alcancemos a glória celestial. Por nosso Senhor \emph{\&c.}
}\end{paracol}

\paragraphinfo{Comúnio}{Sl. 95, 2}
\begin{paracol}{2}\latim{
\rlettrine{C}{antáte} Dómino, allelúja: cantáte Dómino et benedícite nomen ejus: bene nuntiáte de die in diem salutáre ejus, allelúja, allelúja.
}\switchcolumn\portugues{
\rlettrine{C}{antai} hinos em honra do Senhor: cantai hinos em honra do Senhor: bendizei o seu nome; proclamai constantemente, dia a dia, a salvação que nos concede, aleluia, aleluia.
}\end{paracol}

\paragraph{Postcomúnio}
\begin{paracol}{2}\latim{
\rlettrine{T}{ríbue} nobis, Dómine, cæ léstis mensæ virtúte satiátis: et desideráre, quæ recta sunt, et desideráta percípere. Per Dóminum \emph{\&c.}
}\switchcolumn\portugues{
\rlettrine{H}{avendo} nós sido fortalecidos com o Pão da mesa celestial, concedei-nos, Senhor, a graça de desejarmos o que é justo e de alcançarmos o que desejamos. Por nosso Senhor \emph{\&c.}
}\end{paracol}

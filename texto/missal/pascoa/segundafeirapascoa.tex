\subsectioninfo{Segunda-feira Pascal}{Estação em S. Pedro}

\paragraphinfo{Intróito}{Ex. 13, 5 \& 9}
\begin{paracol}{2}\latim{
\rlettrine{I}{ntrodúxit} vos Dóminus in terram fluéntem lac et mel, allelúja: et ut lex Dómini semper sit in ore vestro, allelúja, allelúja. \emph{Ps. 104, 1} Confitémini Dómino et invocáte nomen ejus: annuntiáte inter gentes ópera ejus.
℣. Gloria Patri \emph{\&c.}
}\switchcolumn\portugues{
\rlettrine{O}{} Senhor vos introduziu em um país onde correm leite e mel, aleluia: para que a Lei do Senhor esteja sempre na vossa boca, aleluia, aleluia. \emph{Sl. 104, 1} Louvai o Senhor, invocai o seu nome e anunciai as suas obras em todos os povos.
℣. Glória ao Pai \emph{\&c.}
}\end{paracol}

\paragraph{Oração}
\begin{paracol}{2}\latim{
\rlettrine{D}{eus,} qui sollemnitáte pascháli, mundo remédia contulísti: pópulum tuum, quǽsumus, cœlésti dono proséquere; ut et perféctam libertátem consequi mereátur, et ad vitam profíciat sempitérnam. Per Dóminum \emph{\&c.}
}\switchcolumn\portugues{
\slettrine{Ó}{} Deus, que nesta solenidade pascal trouxestes ao mundo os remédios eficazes, continuai a espalhar, Vos suplicamos, os dons celestiais sobre o vosso povo, para que mereça alcançar a perfeita liberdade e consiga a vida eterna. Por nosso Senhor \emph{\&c.}
}\end{paracol}

\paragraphinfo{Epístola}{Act. 10, 37-43}
\begin{paracol}{2}\latim{
Léctio Actuum Apostólorum.
}\switchcolumn\portugues{
Lição dos Actos dos Apóstolos.
}\switchcolumn*\latim{
\rlettrine{I}{n} diébus illis: Stans Petrus in médio plebis, dixit: Viri fratres, vos scitis, quod factum est verbum per universam Judǽam: incípiens enim a Galilǽa, post baptísmum, quod prædicávit Joánnes, Jesum a Názareth: quómodo unxit eum Deus Spíritu Sancto et virtúte, qui pertránsiit benefaciéndo, et sanándo omnes oppréssos a diábolo, quóniam Deus erat cum illo. Et nos testes sumus ómnium, quæ fecit in regióne Judæórum et Jerúsalem, quem occidérunt suspendéntes in ligno. Hunc Deus suscitávit tértia die, et dedit eum maniféstum fíeri, non omni pópulo, sed téstibus præordinátis a Deo: nobis, qui mandticávimus et bíbimus cum illo, postquam resurréxit a mórtuis. Et præcépit nobis prædicáre populo et testificári, quia ipse est, qui constitútus est a Deo judex vivórum et mortuórum. Huic omnes Prophétæ testimónium pérhibent, remissiónem peccatórum accípere per nomen ejus omnes, qui credunt in eum.
}\switchcolumn\portugues{
\rlettrine{N}{aqueles} dias, estando Pedro no meio do povo, disse: «Meus irmãos: Sabeis o que se passou em toda a Judeia, a começar pela Galileia, a respeito de Jesus de Nazaré, desde, o baptismo, que João pregou; como Deus O ungiu com o Espírito Santo e com o poder; como Ele passou pela terra, praticando o bom e curando os que estavam possessos pelo demónio, pois Deus estava com Ele. Nós somos testemunhas de tudo o que praticou na região dos judeus em Jerusalém, onde O mataram, pregando-O numa cruz. Mas Deus ressuscitou-O ao terceiro dia e quis que Ele se manifestasse, não a todo o povo, mas às testemunhas que antecipadamente escolhera, entre as quais nós, que comemos e bebemos com Ele, após a sua ressurreição dos mortos. Ele nos mandou pregar ao povo e afirmar que foi estabelecido por Deus como Juiz dos vivos e dos mortos. É a Ele que todos os Profetas se referem quando testemunham: «que todos os que acreditaram n’Ele alcançarão em virtude do seu nome a remissão de seus pecados».
}\end{paracol}

\paragraphinfo{Gradual}{Sl. 117, 24 \& 2}
\begin{paracol}{2}\latim{
\rlettrine{H}{æc} dies, quam fecit Dóminus: exsultémus et lætémur in ea. ℣. Dicat nunc Israël, quóniam bonus: quóniam in sǽculum misericórdia ejus.
}\switchcolumn\portugues{
\rlettrine{E}{is} o dia que o -Senhor fez: exultemos e alegremo-nos nele. Diga, agora, Israel: Deus é bom; a sua misericórdia é eterna.
}\switchcolumn*\latim{
Allelúja, allelúja. ℣. \emph{Matth. 28, 2} Angelus Dómini descéndit de cœlo: et accédens revólvit lápidem, et sedébat super eum.
}\switchcolumn\portugues{
Aleluia, aleluia. ℣. \emph{Mt. 28, 2} Um anjo do Senhor desceu do céu, e, chegando-se, revolveu a pedra e assentou-se sobre ela.
}\end{paracol}

\paragraphinfo{Evangelho}{Lc. 24, 13-35}
\begin{paracol}{2}\latim{
\cruz Sequéntia sancti Evangélii secúndum Lucam.
}\switchcolumn\portugues{
\cruz Continuação do santo Evangelho segundo S. Lucas.
}\switchcolumn*\latim{
\blettrine{I}{n} illo témpore: Duo ex discípulis Jesu ibant ipsa die in castéllum, quod erat in spátio stadiórum sexagínta ab Jerúsalem, nómine Emmaus. Et ipsi loquebántur ad ínvicem de his ómnibus, quæ accíderant. Et factum est, dum fabularéntur et secum quǽrerent: et ipse Jesus appropínquans ibat cum illis: óculi autem illórum tenebántur, ne eum agnóscerent. Et ait ad illos: Qui sunt hi sermónes, quos confértis ad ínvicem ambulántes, et estis tristes? Et respóndens unus, cui nomen Cléophas, dixit ei: Tu solus peregrínus es in Jerúsalem, et non cognovísti, quæ facta sunt in illa his diébus? Quibus ille dixit: Quæ? Et dixérunt: De Jesu Nazaréno, qui fuit vir Prophéta potens in ópere et sermóne, coram Deo et omni pópulo: et quómodo eum tradidérunt summi sacerdótes et príncipes nostri in damnatiónem mortis, et crucifixérunt eum. Nos autem sperabámus, quia ipse esset redemptúrus Israël: et nunc super hæc ómnia tértia dies est hódie, quod hæc facta sunt. Sed et mulíeres quædam ex nostris terruérunt nos, quæ ante lucem fuérunt ad monuméntum, et, non invénto córpore ejus, venérunt, dicéntes se étiam visiónem Angelórum vidísse, qui dicunt eum vívere. Et abiérunt quidam ex nostris ad monuméntum: et ita invenérunt, sicut mulíeres dixérunt, ipsum vero non invenérunt. Et ipse dixit ad eos: O stulti et tardi corde ad credéndum in ómnibus, quæ locúti sunt Prophétæ! Nonne hæc opórtuit pati Christum, et ita intráre in glóriam suam? Et incípiens a Móyse et ómnibus Prophétis, interpretabátur illis in ómnibus Scriptúris, quæ de ipso erant. Et appropinquavérunt castéllo, quo ibant: et ipse se finxit lóngius ire. Et coëgérunt illum, dicéntes: Mane nobiscum, quóniam advesperáscit et inclináta est jam dies. Et intrávit cum illis. Et factum est, dum recúmberet cum eis, accépit panem, et benedíxit, ac fregit, et porrigébat illis. Et apérti sunt óculi eórum, et cognovérunt eum: et ipse evánuit ex óculis eórum. Et dixérunt ad ínvicem: Nonne cor nostrum ardens erat in nobis, dum loquerétur in via, et aperíret nobis Scriptúras? Et surgéntes eádem hora regréssi sunt in Jerúsalem: et invenérunt congregátas úndecim, et eos, qui cum illis erant, dicéntes: Quod surréxit Dóminus vere, et appáruit Simóni. Et ipsi narrábant, quæ gesta erant in via: et quómodo cognovérunt eum in fractióne panis.
}\switchcolumn\portugues{
\blettrine{N}{aquele} tempo, iam dois discípulos, naquele mesmo dia, para uma aldeia chamada Emaús, distante de Jerusalém sessenta estádios, os quais iam conversando nas coisas que haviam sucedido. Ora aconteceu que, enquanto falavam e discutiam, aproximou-se Jesus e acompanhou-os; mas os olhos deles estavam tão entenebrecidos que O não conheceram. Jesus disse-lhes, então: «Que assuntos são esses em que vos entretendes, mesmo pelo caminho, e que vos mostram tão tristes?». Respondendo um deles, chamado Cléofas, disse-Lhe: «Só Vós sois tão alheio a Jerusalém, que ignorais o que se passou durante estes dias?». Ele disse: «Que foi?». E responderam: «O que diz respeito a Jesus Nazareno, que era um Profeta poderoso em obras e em palavras, diante de Deus e de todo o povo: e como os nossos príncipes dos sacerdotes e magistrados O entregaram, para ser condenado à morte, e O crucificaram. Nós esperávamos que fosse Ele quem resgatasse Israel, mas com tudo isto já hoje é o terceiro dia, depois que estas coisas aconteceram. É verdade que algumas mulheres, que são da nossa grei, nos causaram espanto, dizendo que, havendo ido de madrugada ao sepulcro, não tinham encontrado o corpo de Jesus; e vieram dizer que tinham visto um Anjo que lhes dissera estar Ele vivo. Alguns dos nossos foram ao sepulcro e encontraram as coisas como as mulheres haviam dito: mas O não viram». Então Jesus disse-lhes: «Ó insensatos, cujo coração é lento em acreditar o que os Profetas anunciaram! Porventura não convinha que Cristo padecesse essas coisas e que entrasse na sua glória?». E, começando em Moisés e percorrendo todos os Profetas, explicou-lhes tudo o que as Escrituras continham a seu respeito. Quando já estavam perto da aldeia para onde se dirigiam, Jesus deu a entender que ia mais longe. Porém, eles suplicaram-Lhe: «Ficai connosco, pois faz-se tarde e anoitece». Então, Jesus entrou com eles. Ora, aconteceu que, enquanto estava à mesa com eles, tomou o pão, benzeu-o e, partindo-o, apresentou-lho. Então abriram-se-lhes os olhos e reconhceram-n’O: mas Ele desapareceu da sua vista. E disseram um ao outro: «Porventura, enquanto Ele nos falava no caminho e explicava as Escrituras os nossos corações não pulsavam ardentemente?». E, erguendo-se imediatamente, voltaram a Jerusalém, onde estavam reunidos os Onze e os que estavam com estes, os quais lhes disseram: «O Senhor ressuscitou, verdadeiramente; já apareceu a Simão». E eles contaram o que se havia passado no caminho e, como O tinham reconhecido pela fracção do pão.
}\end{paracol}

\paragraphinfo{Ofertório}{Mt. 28, 2, 5 \& 6}
\begin{paracol}{2}\latim{
\rlettrine{A}{ngelus} Dómini descéndit de cœlo, et dixit muliéribus: Quem quǽritis, surréxit, sicut dixit, allelúja.
}\switchcolumn\portugues{
\rlettrine{U}{m} Anjo do Senhor desceu do céu e disse às mulheres: «Aquele a quem buscais ressuscitou, como Ele havia dito», aleluia.
}\end{paracol}

\paragraphinfo{Secreta e Postcomúnio}{Como no dia precedente}

\paragraphinfo{Comúnio}{Lc. 24, 34}
\begin{paracol}{2}\latim{
\rlettrine{S}{urréxit} Dóminus, et appáruit Petro, allelúja.
}\switchcolumn\portugues{
\rlettrine{R}{essuscitou} o Senhor e apareceu a Pedro, aleluia.
}\end{paracol}

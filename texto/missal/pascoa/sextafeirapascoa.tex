\subsectioninfo{Sexta-feira Pascal}{Estação em Santa Maria dos Mártires }

\paragraphinfo{Intróito}{Sl. 77, 53}
\begin{paracol}{2}\latim{
\rlettrine{E}{dúxit} eos Dóminus in spe, allelúja: et inimícos eórum opéruit mare, allelúja, allelúja, allelúja. \emph{Ps. ibid., 1} Attendite, pópule meus, legem meam: inclináte aurem vestram in verba oris mei.
℣. Gloria Patri \emph{\&c.}
}\switchcolumn\portugues{
\rlettrine{O}{} Senhor conduziu-os cheios de esperança, aleluia: enquanto que aos seus inimigos submergiu-os no mar, aleluia, aleluia, aleluia. \emph{Sl. ibid., 1} Atende à minha lei, ó meu povo: escuta as palavras da minha boca.
℣. Glória ao Pai \emph{\&c.}
}\end{paracol}

\paragraph{Oração}
\begin{paracol}{2}\latim{
\rlettrine{O}{mnípotens} sempitérne Deus, qui paschále sacramén
tum in reconciliatiónis humánæ fǿdere contulísti: da méntibus nostris; ut, quod professióne celebrámus, imitémur efféctu. Per Dóminum \emph{\&c.}
}\switchcolumn\portugues{
\rlettrine{D}{eus} omnipotente e eterno, que por meio do sacramento pascal estabelecestes uma aliança de reconciliação com a humanidade, concedei às nossas almas a graça de imitarem em suas acções o que confessam durante esta festa. Por nosso Senhor \emph{\&c.}
}\end{paracol}

\paragraphinfo{Epístola}{1. Pe. 3, 18-22}
\begin{paracol}{2}\latim{
Léctio Epístolæ beáti Petri Apóstoli.
}\switchcolumn\portugues{
Lição da Ep.ª do B. Ap.º Pedro.
}\switchcolumn*\latim{
\rlettrine{C}{aríssimi:} Christus semel pro peccátis nostris mórtuus est, justus pro injústis, ut nos offérret Deo, mortificátus quidem carne, vivificátus autem spíritu. In quo et his, qui in cárcere erant, spirítibus véniens prædicávit: qui incréduli fúerant aliquándo, quando exspectábant Dei patiéntiam in diébus Noë, cum fabricarétur arca, in qua pauci, id est octo ánimæ salvæ factæ sunt per aquam. Quod et vos nunc símilis formæ salvos facit baptísma: non carnis deposítio sórdium, sed consciéntiæ bonæ interrogátio in Deum per resurrectiónem Jesu Christi, Dómini nostri, qui est in dextera Dei.
}\switchcolumn\portugues{
\rlettrine{C}{aríssimos:} Cristo morreu uma só vez pelos nossos pecados: Ele, o justo, pelos injustos, a fim de nos oferecer a Deus, depois de haver sido morto na carne, mas voltou à vida pelo Espírito, Foi também com este espírito que pregou aos espíritos, que estavam na prisão, e que outrora haviam sido incrédulos, quando nos dias de Noé esperavam a paciência de Deus, enquanto se construía a arca, na qual poucas pessoas, isto é, somente oito, foram salvas das águas. Isto era a figura do Baptismo, pelo qual deveis ser salvos. Este Baptismo não consiste na purificação das manchas da carne, mas na promessa que fazeis quando vos perguntam se quereis guardar em Deus a vossa consciência pura a qual salvação vos é dada pela ressurreição de nosso Senhor Jesus Cristo, que está à dextra de Deus.
}\end{paracol}

\paragraphinfo{Gradual}{Sl. 117, 24 \& 26-27}
\begin{paracol}{2}\latim{
\rlettrine{H}{æc} dies, quam fecit Dóminus: exsultémus et lætémur in ea. ℣. Benedíctus, qui venit in nómine Dómini: eus Dóminus, et illúxit nobis.
}\switchcolumn\portugues{
\rlettrine{E}{is} o dia que o Senhor fez: exultemos e alegremo-nos nele. Bendito seja Aquele que vem em nome do Senhor: o Senhor e Deus fez resplandecer sobre nós a sua luz.
}\switchcolumn*\latim{
Allelúja, allelúja. ℣. \emph{Ps. 95, 10} Dícite in géntibus: quia Dóminus regnávit a ligno.
}\switchcolumn\portugues{
Aleluia, aleluia. ℣. \emph{Sl. 95, 10} Dizei aos povos: o Senhor reinou pelo madeiro.
}\end{paracol}

\paragraphinfo{Evangelho}{Mt. 28, 16-20}
\begin{paracol}{2}\latim{
\cruz Sequéntia sancti Evangélii secúndum Matthǽum.
}\switchcolumn\portugues{
\cruz Continuação do santo Evangelho segundo S. Mateus.
}\switchcolumn*\latim{
\blettrine{I}{n} illo témpore: Undecim discípuli abiérunt in Galilǽam, in montem, ubi constitúerat illis Jesus. Et vidéntes eum adoravérunt: quidam autem dubitavérunt. Et accédens Jesus locútus est eis, dicens: Data est mihi omnis potéstas in cœlo et in terra. Eúntes ergo, docéte omnes gentes, baptizántes eos in nómine Patris, et Fílii, et Spíritus Sancti: docentes eos serváre ómnia, quæcúmque mandávi vobis. Et ecce, ego vobíscum sum ómnibus diébus usque ad consummatiónem sǽculi.
}\switchcolumn\portugues{
\blettrine{N}{aquele} tempo, partiram os Onze discípulos para a Galileia, dirigindo-se ao monte onde o Senhor os havia mandado ir. E, vendo eles Jesus, logo O adoraram; ficando, contudo, alguns deles na dúvida. Entretanto, Jesus foi-se aproximando e falou-lhes, dizendo: «Todo o poder me foi dado no céu e na terra. Ide, pois, ensinai todos os povos e baptizai-os em nome do Pai, e do Filho, e do Espírito Santo, ensinando-os a observar tudo o que vos ordenei. Eu permanecerei convosco todos os dias até à consumação dos séculos».
}\end{paracol}

\paragraphinfo{Ofertório}{Ex. 12, 14}
\begin{paracol}{2}\latim{
\rlettrine{E}{rit} vobis hæc dies memoriális, allelúja: et diem festum celebrábitis sollémnem Dómino in progénies vestras: legítimum sempitérnum diem, allelúja, allelúja, allelúja. 
}\switchcolumn\portugues{
\rlettrine{E}{ste} dia ficar-vos-á memorável, aleluia: celebrá-lo-eis de geração em geração com uma festa solene em honra do Senhor: e ficará uma instituição perpétua, aleluia, aleluia, aleluia.
}\end{paracol}

\paragraph{Secreta}
\begin{paracol}{2}\latim{
\rlettrine{H}{óstias,} quǽsumus, Dómine, placátus assúme: quas et pro renatórum expiatióne peccáti deférimus, et pro acceleratióne cœléstis auxílii. Per Dóminum \emph{\&c.}
}\switchcolumn\portugues{
\rlettrine{V}{os} suplicamos, Senhor, recebei benigno as hóstias, que Vos oferecemos; e que elas sirvam de expiação dos pecados dos recém-nascidos e para alcançarmos o socorro celestial. Por nosso Senhor \emph{\&c.}
}\end{paracol}

\paragraphinfo{Comúnio}{Mt. 28, 18-19}
\begin{paracol}{2}\latim{
\rlettrine{D}{ata} est mihi omnis potéstas in cœlo et in terra, allelúja: eúntes, docéte omnes gentes, baptizántes eos in nómine Patris, et Fílii, et Spíritus Sancti, allelúja, allelúja.
}\switchcolumn\portugues{
\rlettrine{T}{odo} o poder me foi dado no céu e na terra, aleluia: ide, pois, ensinai todos os povos e baptizai-os em nome do Pai, e do Filho, e do Espírito Santo, aleluia, aleluia. 
}\end{paracol}

\paragraph{Postcomúnio}
\begin{paracol}{2}\latim{
\rlettrine{R}{éspice,} quǽsumus, Dómine, pópulum tuum: et, quem ætérnis dignátus es renováre mystériis, a temporálibus culpis dignánter absólve. Per Dóminum \emph{\&c.}
}\switchcolumn\portugues{
\rlettrine{O}{lhai} para o vosso povo, Senhor; e, assim como Vos dignastes renová-lo com os mistérios eternos, assim também Vos digneis absolvê-lo das ofensas cometidas neste mundo. Por nosso Senhor \emph{\&c.}
}\end{paracol}

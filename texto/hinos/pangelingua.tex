\gregorioscore{scores/pangelingua1verse}

\begin{nscenter}
Canta, ó minha língua, o mystério do Corpo e do Sangue precioso que foi derramado para resgate do mundo, fruto dum seio fecundo, o Rei dos povos.
\end{nscenter}

\begin{paracol}{2}\latim{
\rlettrine{N}{obis} datus, nobis natus
Ex intácta Vírgine,
Et in mundo conversátus,
Sparso verbi sémine,
Sui moras incolátus
Miro clausit órdine.
}\switchcolumn\portugues{
\rlettrine{F}{oi-nos} dado; para nós nasceu da Virgem Imaculada; viveu no mundo, e, depois de haver espalhado a semente da palavra, terminou a sua passagem neste mundo com uma admirável instituição.
}\switchcolumn*\latim{
In suprémæ nocte coenæ
Recúmbens cum frátribus
Observáta lege plene
Cibis in legálibus,
Cibum turbæ duodénæ
Se dat suis mánibus.
}\switchcolumn\portugues{
Na noite da última ceia, estando à mesa com seus irmãos depois de haver observado os ritos legais, Ele próprio se deu com suas mãos em alimento aos Doze.
}\switchcolumn*\latim{
Verbum caro, panem verum
Verbo carnem éfficit:
Fitque sanguis Christi merum,
Et si sensus déficit,
Ad firmándum cor sincérum
Sola fides súfficit.
}\switchcolumn\portugues{
O Verbo feito carne mudou pela sua palavra um pão verdadeiro na própria Carne, e o vinho no Sangue de Cristo; e se a razão desfalece, não podendo compreender isto, a fé basta para corroborar esta crença nos corações sinceros.
}\end{paracol}

\gregorioscore{scores/tantum}

\begin{nscenter}
Adoremos, pois, prostrados este tão grande Sacramento: cedam os ritos antigos o lugar ao novo mystério e que a fé supra a fraqueza dos nossos sentidos.
\end{nscenter}

\gregorioscore{scores/genitori}

\begin{nscenter}
Glória, honra, louvor, poder, acção de graças e bênçãos sejam dadas ao Pai e ao Filho: e dêem-se iguais louvores ao que procede de um e do outro. Amen.
\end{nscenter}

\gregorioscore{scores/sextafeirasanta/cruxfidelis}

\begin{nscenter}
Ó Cruz fiel, entre todas a árvore mais nobre: Nenhum bosque produz igual, em ramagens, frutos e flores. Ó doce lenho, que os doces cravos e o doce peso sustentas.
\end{nscenter}

\begin{paracol}{2}\latim{
\rlettrine{P}{ange,} lingua, gloriósi prǽlium certáminis, et super crucis trophǽo dic triúmphum nóbilem: Quáliter Redémptor orbis Immolátus vícerit.
}\switchcolumn\portugues{
\rlettrine{C}{anta,} ó língua, o glorioso combate, e, diante do troféu da Cruz, proclama o nobre triunfo: a vitória conseguida pelo Redentor, vítima imolada para o mundo.
}\switchcolumn*\latim{
Crux fidélis, inter omnes Arbor una nóbilis: Nulla talem silva profert, fronde, flore, gérmine.
}\switchcolumn\portugues{
Ó Cruz fiel, entre todas a árvore mais nobre: Nenhum bosque produz igual, em ramagens, frutos e flores.
}\switchcolumn*\latim{
De paréntis protoplásti Fráude Fáctor cóndolens, quando pómi noxiális morte morsu córruit: Ipse lígnum tunc notávit, Dámna lígni ut sólveret.
}\switchcolumn\portugues{
O Criador teve pena do primitivo casal, que foi ferido de morte, comendo o fruto fatal, e marcou logo outra árvore para curar-se do mal.
}\switchcolumn*\latim{
Dulce lignum, dulce clavo, dulce pondus sústinens.
}\switchcolumn\portugues{
Ó doce lenho, que os doces cravos e o doce peso sustentas.
}\switchcolumn*\latim{
Aequa Patri Filioque, inclito Paraclito, sempiterna sit beatæ Trinitati gloria, cuius alma nos redemit atque servat gratia.
}\switchcolumn\portugues{
Glória e poder à Trindade, ao Pai e ao Filho louvor, honra ao Espírito Santo, eterna glória ao Senhor, que nos salvou pela graça e nos reuniu no amor.
}\switchcolumn*\latim{
Crux fidélis, inter omnes Arbor una nóbilis: Nulla talem silva profert, fronde, flore, gérmine.
}\switchcolumn\portugues{
Ó Cruz fiel, entre todas a árvore mais nobre: Nenhum bosque produz igual, em ramagens, frutos e flores.
}\end{paracol}

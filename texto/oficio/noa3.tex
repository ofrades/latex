\subsection{Noa 2}

\textit{Tudo como no primeiro oficio, excepto o seguinte:}

\begin{paracol}{2}\latim{
\emph{Ant.} Ecce, Maria genuit nobis Salvatorem, quem Joannes videns exclamavit, dicens: Ecce Agnus Dei, ecce qui tollit peccata mundi, (allelúja).
}\switchcolumn\portugues{
\emph{Ant.} Maria deu à luz o nosso Salvador, que João reconheceu e exclamou: eis o Cordeiro de Deus, Aquele que tira o pecado do mundo, (aleluia).
}\end{paracol}

\paragraphinfo{Pequeno Capítulo}{Ecl. 24, 19-20}
\begin{paracol}{2}\latim{
\rlettrine{I}{n} plateis sicut cinnamomum et balsamum aromatizans odorem dedi: quasi myrrha electa, dedi suavitatem odoris.
}\switchcolumn\portugues{
\rlettrine{N}{as} praças assim como o cinamomo e o bálsamo, que difundem cheiro, dei eu fragrância; como a mirra, dei cheiro de suavidade.
}\switchcolumn*\latim{
℟. Deo grátias.
}\switchcolumn\portugues{
℟. Graças a Deus.
}\switchcolumn*\latim{
℣. Post partum, Virgo, invioláta permansísti.
}\switchcolumn\portugues{
℣. Despois do parto permanecestes imaculada.
}\switchcolumn*\latim{
℟. Dei Génetrix, intercéde pro nobis.
}\switchcolumn\portugues{
℟. Intercedei por nós, ó Mãe de Deus.
}\end{paracol}

\begin{paracol}{2}\latim{
\begin{nscenter} Orémus. \end{nscenter}
}\switchcolumn\portugues{
\begin{nscenter} Oremos. \end{nscenter}
}\switchcolumn*\latim{
\rlettrine{D}{eus,} qui salutis æternæ, beatæ Mariæ virginitate fœcunda, humano generi præmia præstitisti: tribue, quǽsumus; ut ipsam pro nobis intercedere sentiamus, per quam meruimus auctorem vitæ suscipere, Dominum nostrum Jesum Christum Filium tuum. Qui tecum vivit et regnat in unitate Spiritus Sancti, Deus, per omnia sæcula sæculorum.
}\switchcolumn\portugues{
\slettrine{Ó}{} Deus, que pela virgindade fecunda da bem-aventurada Maria, destes ao género humano as gratificações da salvação eterna: concedei-nos, Vos rogamos, que experienciemos sua intercessão por nós, dela pela qual recebemos o autor da vida, Nosso Senhor Jesus Cristo, vosso Filho. Que convosco, e com o Espírito Santo, vive e reina por todos os séculos.
}\switchcolumn*\latim{
℟. Amen.
}\switchcolumn\portugues{
℟. Amen.
}\end{paracol}
\subsection{Completas 3}

\paragraphinfo{Pequeno Capítulo}{Ecl. 24}
\begin{paracol}{2}\latim{
\rlettrine{E}{go} mater pulchræ dilectionis, et timoris, et agnitionis, et sanctæ spei.
}\switchcolumn\portugues{
\rlettrine{E}{u} sou a Mãe do amor belo e do temor, e do conhecimento antigo, e da santa esperança.
}\switchcolumn*\latim{
℟. Deo grátias.
}\switchcolumn\portugues{
℟. Graças a Deus.
}\switchcolumn*\latim{
℣. Ora pro nobis sancta Dei Génetrix.
}\switchcolumn\portugues{
℣. Rogai por nós, Santa Mãe de Deus.
}\switchcolumn*\latim{
℟. Ut digni efficiamur promissionibus Christi.
}\switchcolumn\portugues{
℟. Para que sejamos dignos das promessas de Cristo.
}\switchcolumn*\latim{
\emph{Nunc. Ant.} Magnum hæreditatis mysterium: templum Dei factus est uterus nescientis virum: non est pollutus ex ea carnem assumens; omnes gentes venient, dicentes: Gloria tibi, Domine.
}\switchcolumn\portugues{
\emph{Nunc. Ant.} Grande mistério de herança: o ventre daquela que não conheceu varão, é feito templo de Deus; o qual se não manchou, tomando dela carne humana. Virão todas as gentes, dizendo: Glória a Vós, ó Senhor.
}\switchcolumn*\latim{
\begin{nscenter} Orémus. \end{nscenter}
}\switchcolumn\portugues{
\begin{nscenter} Oremos. \end{nscenter}
}\switchcolumn*\latim{
\rlettrine{D}{eus,} qui salutis æternæ, beatæ Mariæ virginitate fœcunda, humano generi præmia præstitisti: tribue, quǽsumus; ut ipsam pro nobis intercedere sentiamus, per quam meruimus auctorem vitæ suscipere, Dominum nostrum Jesum Christum Filium tuum. Qui tecum vivit et regnat in unitate Spiritus Sancti, Deus, per omnia sæcula sæculorum.
}\switchcolumn\portugues{
\slettrine{Ó}{} Deus, que pela virgindade fecunda da bem-aventurada Maria, destes ao género humano as gratificações da salvação eterna: concedei-nos, Vos rogamos, que experienciemos sua intercessão por nós, dela pela qual recebemos o autor da vida, Nosso Senhor Jesus Cristo, vosso Filho. Que convosco, e com o Espírito Santo, vive e reina por todos os séculos.
}\switchcolumn*\latim{
℟. Amen.
}\switchcolumn\portugues{
℟. Amen.
}\end{paracol}

\emph{Acabar com uma Antífona de Nossa Senhora na página \pageref{antifonasnossasenhora}.}

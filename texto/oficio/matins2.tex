\subsection{Matinas 2}

\emph{Tudo como no primeiro oficio, excepto o seguinte:
No Terceiro Nocturno, na Quarta-feira, no Sábado e na Festa da Anunciação, a última Antífona é a seguinte; são próprias as Lições e Responsórios que seguem.}

\begin{paracol}{2}\latim{
\emph{Ant.} Angelus Domini nuntiavit Mariæ, et concepit de Spiritu Sancto, (Allelúja)
}\switchcolumn\portugues{
\emph{Ant.} O Anjo do Senhor anunciou a Maria, e ela concebeu do Espírito Santo.
}\end{paracol}

\paragraphinfo{Salmo 97}{Página \pageref{salmo97}}

\begin{paracol}{2}\latim{
\emph{Ant.} Angelus Domini nuntiavit Mariæ, et concepit de Spiritu Sancto, (Allelúja)
}\switchcolumn\portugues{
\emph{Ant.} O Anjo do Senhor anunciou a Maria, e ela concebeu do Espírito Santo.
}\end{paracol}

\paragraph{Absolvição}
\begin{paracol}{2}\latim{
\rlettrine{P}{recibus} et meritis beatæ Mariæ semper Virginis, et omnium Sanctorum, perducat nos Dominus ad regna cælorum.
}\switchcolumn\portugues{
\rlettrine{P}{elos} rogos e merecimentos da bem-aventurada Virgem Maria, e de todos os Santos, nos conduza o Senhor ao reino dos céus.
}\switchcolumn*\latim{
℟. Amen.
}\switchcolumn\portugues{
℟. Amen.
}\switchcolumn*\latim{
℣. Jube, Domine, benedicere.
}\switchcolumn\portugues{
℣. Dai-me, ó Senhor, a vossa bênção.
}\end{paracol}

\paragraph{Benção}
\begin{paracol}{2}\latim{
Nos cum prole pia benedicat Virgo Maria.
}\switchcolumn\portugues{
Benza-nos a Virgem Maria com seu piíssimo Filho.
}\switchcolumn*\latim{
℟. Amen.
}\switchcolumn\portugues{
℟. Amen.
}\end{paracol}

\paragraphinfo{Lição 1}{Lc. 1, 26-28}
\begin{paracol}{2}\latim{
\rlettrine{M}{issus} est Angelus Gabriel a Deo in civitatem Galilææ, cui nomen Nazareth, ad virginem desponsatam viro, cui nomen erat Joseph, de domo David: et nomen virginis Maria. Et ingressus Angelus ad eam dixit: Ave gratia plena: Dominus tecum: benedicta tu in mulieribus.
}\switchcolumn\portugues{
\rlettrine{O}{} Anjo Gabriel foi mandado por Deus à cidade de Galileia chamada Nazaré, a uma virgem denominada Maria, desposada com um varão cujo nome era José, da casa de David. E entrando o Anjo onde ela estava, lhe disse: Deus vos salve, ó cheia de graça, o Senhor é convosco; bendita sois vós entre as mulheres.
}\switchcolumn*\latim{
℣. Tu autem, Dómine, miserére nobis.
}\switchcolumn\portugues{
℣. E vós, Senhor, tende misericórdia de nós.
}\switchcolumn*\latim{
℟. Deo grátias.
}\switchcolumn\portugues{
℟. Demos graças a Deus.
}\switchcolumn*\latim{
℟. Missus est Gabriel Angelus ad Maríam Vírginem desponsatam Joseph, nuntians ei verbum; et expavescit Virgo de lúmine: ne timeas, María, invenísti grátiam apud Dóminum: Ecce concipies et paries, et vocábitur Altíssimi Fílius.
}\switchcolumn\portugues{
℟. O Anjo Gabriel foi enviado a Maria Virgem, desposada com José, para lhe anunciar o verbo; e a Virgem assustou-se com o esplendor da sua luz. Não temas, Maria, que achaste graça para com o Senhor. Conceberás, e darás à luz um filho que será chamado o filho do Altíssimo.
}\switchcolumn*\latim{
℣. Dabit ei Dóminus Deus sedem David, patris ejus, et regnábit in domo Jacob in ætérnum.
}\switchcolumn\portugues{
℣. O Senhor Deus lhe dará o trono de David seu Pai, e reinará eternamente na casa de Jacob.
}\switchcolumn*\latim{
℟. Ecce concipies et paries, et vocábitur Altíssimi Fílius.
}\switchcolumn\portugues{
℟. Conceberás, e darás á luz um filho que será chamado o Filho do Altíssimo.
}\switchcolumn*\latim{
℣. Iube domne benedicere.
}\switchcolumn\portugues{
℣. Dai-me, ó Senhor, a vossa bênção.
}\switchcolumn*\latim{
Ipsa Virgo Vírginum intercédat pro nobis ad Dóminum.
}\switchcolumn\portugues{
A mesma Virgem das virgens interceda por nós ao Senhor.
}\switchcolumn*\latim{
℟. Amen.
}\switchcolumn\portugues{
℟. Amen.
}\end{paracol}

\paragraphinfo{Lição 2}{Lc. 1, 29-33}
\begin{paracol}{2}\latim{
\qlettrine{Q}{uæ} cum audisset, turbata est in sermone ejus, et cogitabat qualis esset ista salutatio. Et ait Angelus ei: Ne timeas, Maria: invenisti enim gratiam apud Deum: ecce concipies in utero, et paries filium, et vocabis nomen ejus Jesum: hic erit magnus, et Filius Altissimi vocabitur, et dabit illi Dominus Deus sedem David patris ejus: et regnabit in domo Jacob in æternum, et regni ejus non erit finis.
}\switchcolumn\portugues{
\rlettrine{O}{uvindo} ela estas palavras, perturbou-se pelo que se lhe dizia; e considerava que saudação seria. Então o Anjo disse-lhe: Não temas, Maria, porque achaste graça para com Deus. Conceberás no teu ventre, e darás à luz um filho a quem darás o nome de Jesus. Este será grande e se chamará Filho do Altíssimo, e o Senhor Deus lhe dará o trono de David seu Pai, e reinará eternamente na casa de Jacob, e o seu Reino não terá fim.
}\switchcolumn*\latim{
℣. Tu autem, Dómine, miserére nobis.
}\switchcolumn\portugues{
℣. E vós, Senhor, tende misericórdia de nós.
}\switchcolumn*\latim{
℟. Deo grátias.
}\switchcolumn\portugues{
℟. Demos graças a Deus.
}\switchcolumn*\latim{
℟. Ave, María, grátia plena; Dóminus tecum: Spíritus Sanctus supervéniet in te, et virtus Altíssimi obumbrábit tibi: quod enim ex te nascétur Sanctum, vocábitur Fílius Dei.
}\switchcolumn\portugues{
℟. Ave, Maria, cheia de graça; o Senhor é convosco. Virá sobre vós o Espírito Santo e a virtude do Altíssimo vos fará sombra: por isso o santo que nascerá de vós será chamado Filho de Deus.
}\switchcolumn*\latim{
℣. Quómodo fiet istud, quóniam virum non cognósco? Et respóndens Angelus, dixit ei.
}\switchcolumn\portugues{
℣. Como se fará isto, pois não conheço varão? E respondendo o Anjo, lhe disse:
}\switchcolumn*\latim{
℟. Spíritus Sanctus supervéniet in te, et virtus Altíssimi obumbrábit tibi: quod enim ex te nascétur Sanctum, vocábitur Fílius Dei.
}\switchcolumn\portugues{
℟. Virá sobre vós o Espírito Santo, e a virtude do Altíssimo vos fará sombra; por isso o santo que nascerá de vós será chamado Filho de Deus.
}\end{paracol}

\emph{Quando o Te Deum (página \pageref{tedeum}) é dito depois da Terceira Lição, adiciona-se o seguinte no fim do Responsório:}

\begin{paracol}{2}\latim{
℣. Glória Patri, et Fílio, et Spirítui Sancto.
}\switchcolumn\portugues{
℣. Glória ao Pai e ao Filho e ao Espírito Santo.
}\switchcolumn*\latim{
℟. Spíritus Sanctus supervéniet in te, et virtus Altíssimi obumbrábit tibi: quod enim ex te nascétur Sanctum, vocábitur Fílius Dei.
}\switchcolumn\portugues{
℟. Virá sobre vós o Espírito Santo, e a virtude do Altíssimo vos fará sombra; por isso o santo que nascerá de vós será chamado Filho de Deus.
}\switchcolumn*\latim{
℣. Iube domne benedicere.
}\switchcolumn\portugues{
℣. Dai-me, Senhor, a vossa bênção.
}\end{paracol}

\paragraph{Benção}
\begin{paracol}{2}\latim{
Per Vírginem Matrem concédat nobis Dóminus salútem et pacem.
}\switchcolumn\portugues{
Pela Virgem Maria, nos conceda o Senhor a paz e a salvação.
}\switchcolumn*\latim{
℟. Amen.
}\switchcolumn\portugues{
℟. Amen.
}\end{paracol}

\paragraphinfo{Lição 3}{Lc. 1, 34-38}
\begin{paracol}{2}\latim{
\rlettrine{D}{ixit} autem Maria ad Angelum: Quomodo fiet istud, quoniam virum non cognosco? Et respondens Angelus dixit ei: Spiritus Sanctus superveniet in te, et virtus Altissimi obumbrabit tibi. Ideoque et quod nascetur ex te Sanctum, vocabitur Filius Dei. Et ecce Elisabeth cognata tua, et ipsa concepit filium in senectute sua: et hic mensis sextus est illi, quæ vocatur sterilis: quia non erit impossibile apud Deum omne verbum. Dixit autem Maria: Ecce ancilla Domini: fiat mihi secundum verbum tuum.
}\switchcolumn\portugues{
\rlettrine{D}{isse} então Maria ao Anjo: Como se fará isto, por quando não conheço varão? E respondendo o Anjo, lhe disse: Virá sobre vós o Espírito Santo, e a virtude do Altíssimo vos fará sombra; e por isso o santo que nascerá de vós se chamará Filho de Deus. E também Isabel, vossa parenta, que é chamada estéril, concebeu um filho na sua velhice, está já no sexto mês; porque a Deus nada é impossível. Disse então Maria: Eis aqui a escrava do Senhor, faça-se em mim segundo a vossa palavra.
}\switchcolumn*\latim{
℣. Tu autem, Dómine, miserére nobis.
}\switchcolumn\portugues{
℣. E vós, Senhor, tende misericórdia de nós.
}\switchcolumn*\latim{
℟. Deo grátias.
}\switchcolumn\portugues{
℟. Demos graças a Deus.
}\end{paracol}

\emph{O Te Deum não é dito no Advento, excluindo as Festas de Nossa Senhora. O seguinte Responsório é dito quando o Te Deum é omitido:}

\begin{paracol}{2}\latim{
℟. Súscipe verbum, Virgo María, quod tibi a Dómino per Angelum transmíssum est: concípies et páries Deum páriter et hóminem, ut benedícta dicáris inter omnes mulíeres.
}\switchcolumn\portugues{
℟. Recebei, Maria Virgem, a palavra que Senhor vos transmite pelo seu Anjo. Concebereis, e dareis à luz a Deus e Homem juntamente: pelo que sereis chamada Bendita entre todas as mulheres.
}\switchcolumn*\latim{
℣. Paries quidem fílium, et virginitátis non patiéris detriméntum: efficiéris grávida, et eris mater semper intácta.
}\switchcolumn\portugues{
℣. Dareis à luz um filho, e ficareis sempre Virgem. Concebereis e ficareis mãe, continuareis sempre pura e imaculada.
}\switchcolumn*\latim{
℟. Ut benedícta dicáris inter omnes mulíeres.
}\switchcolumn\portugues{
℟. Pelo que sereis chamada Bendita entre todas as mulheres.
}\switchcolumn*\latim{
℣. Glória Patri, et Fílio, et Spirítui Sancto.
}\switchcolumn\portugues{
℣. Glória ao Pai e ao Filho e ao Espírito Santo.
}\switchcolumn*\latim{
℟. Ut benedícta dicáris inter omnes mulíeres.
}\switchcolumn\portugues{
℟. Pelo que sereis chamada Bendita entre todas as mulheres.
}\end{paracol}

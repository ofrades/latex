\subsection{Sexta 2}

\textit{Tudo como no primeiro oficio, excepto o seguinte:}

\begin{paracol}{2}\latim{
\emph{Ant.} Rubum, quem viderat Moyses incombustum, conservatam agnovimus tuam laudabilem virginitatem: Dei Genitrix, intercede pro nobis.
}\switchcolumn\portugues{
\emph{Ant.} Na sarça que Moisés via sem se consumirr, reconhecemos a vossa admirável virgindade conservada: rogai por nós, Santa Mãe de Deus.
}\end{paracol}

\paragraphinfo{Pequeno Capítulo}{Ecl. 24, 16}
\begin{paracol}{2}\latim{
\rlettrine{E}{t} radicavi in populo honorificato, et in parte Dei mei hereditas illius et in plenitudine sanctorum detentio mea.
}\switchcolumn\portugues{
\rlettrine{E}{} lancei raízes no povo honorificado, e na parte de meu Deus, herança sua; e na congregação dos santos fiz a minha morada.
}\switchcolumn*\latim{
℟. Deo grátias.
}\switchcolumn\portugues{
℟. Graças a Deus.
}\switchcolumn*\latim{
℣. Benedicta tu in mulieribus.
}\switchcolumn\portugues{
℣. Bendita sois vóo entre as mulheres.
}\switchcolumn*\latim{
℟. Et benedictus fructus ventris tui.
}\switchcolumn\portugues{
℟. E bendito é o fruto do vosso ventre.
}\end{paracol}

\begin{paracol}{2}\latim{
\begin{nscenter} Orémus. \end{nscenter}
}\switchcolumn\portugues{
\begin{nscenter} Oremos. \end{nscenter}
}\switchcolumn*\latim{
\rlettrine{D}{eus,} qui salutis æternæ, beatæ Mariæ virginitate fœcunda, humano generi præmia præstitisti: tribue, quǽsumus; ut ipsam pro nobis intercedere sentiamus, per quam meruimus auctorem vitæ suscipere, Dominum nostrum Jesum Christum Filium tuum. Qui tecum vivit et regnat in unitate Spiritus Sancti, Deus, per omnia sæcula sæculorum.
}\switchcolumn\portugues{
\slettrine{Ó}{} Deus, que pela virgindade fecunda da bem-aventurada Maria, destes ao género humano as gratificações da salvação eterna: concedei-nos, Vos rogamos, que experienciemos sua intercessão por nós, dela pela qual recebemos o autor da vida, Nosso Senhor Jesus Cristo, vosso Filho. Que convosco, e com o Espírito Santo, vive e reina por todos os séculos.
}\switchcolumn*\latim{
℟. Amen.
}\switchcolumn\portugues{
℟. Amen.
}\end{paracol}
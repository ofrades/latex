\subsection{Completas 2}

\emph{Tudo como no primeiro oficio, excepto o seguinte:}

\paragraphinfo{Pequeno Capítulo}{Is. 7, 14-15}
\begin{paracol}{2}\latim{
\rlettrine{E}{cce} Virgo concipiet, et pariet filium, et vocabitur nomen ejus Emmanuel. Butyrum et mel comedet, ut sciat reprobare malum, et eligere bonum.
}\switchcolumn\portugues{
\rlettrine{P}{ois} por isso o mesmo Senhor vos dará este sinal: Uma virgem conceberá e dará à luz um filho, e o seu nome será Emanuel. Ele comerá manteiga e mel, até que saiba rejeitar o mal e escolher o bem.
}\switchcolumn*\latim{
℟. Deo gratias.
}\switchcolumn\portugues{
℟. Graças a Deus.
}\switchcolumn*\latim{
℣. Angelus Dómini nuntiávit Maríæ.
}\switchcolumn\portugues{
℣. O Anjo do Senhor anunciou a Maria.
}\switchcolumn*\latim{
℟. Et concépit de Spíritu Sancto.
}\switchcolumn\portugues{
℟. E Ela concebeu do Espírito Santo.
}\switchcolumn*\latim{
\emph{Nunc. Ant.} Spiritus Sanctus in te descendet, Maria: ne timeas, habebis in utero Filium Dei,(Allelúja).
}\switchcolumn\portugues{
\emph{Nunc. Ant.} O Espírito Santo descerá sobre vós, ó Maria; não temais: concebereis, e tereis no ventre o Filho de Deus, (Aleluia).
}\switchcolumn*\latim{
\begin{nscenter} Orémus. \end{nscenter}
}\switchcolumn\portugues{
\begin{nscenter} Oremos. \end{nscenter}
}\switchcolumn*\latim{
\rlettrine{D}{eus,} qui de beatæ Mariæ Virginis utero Verbum tuum, Angelo nuntiante, carnem suscipere voluisti: præsta supplicibus tuis; ut qui vere eam Genetricem Dei credimus, ejus apud te intercessionibus adjuvemur. Per eundem Dominum nostrum Jesum Christum.
}\switchcolumn\portugues{
\slettrine{Ó}{} Deus, que pela anunciação do Anjo quisestes que o vosso Verbo se vestisse da nossa carne nas entranhas da bem-aventurada Virgem Maria: nós, vossos humildes servos, cremos ser ela a verdadeira Mãe de Deus, concedei-nos que nos ajudem as suas intercessões para convosco. Pelo mesmo Jesus Cristo Senhor Nosso.
}\switchcolumn*\latim{
℟. Amen.
}\switchcolumn\portugues{
℟. Amen.
}\end{paracol}

\emph{Acabar com uma Antífona de Nossa Senhora na página \pageref{antifonasnossasenhora}.}

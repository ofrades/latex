\subsection{Matinas 1}

\begin{paracol}{2}\latim{
℣. Domine, \cruz labia mea aperies.
}\switchcolumn\portugues{
℣. Abri, Senhor, \cruz os meus lábios.
}\switchcolumn*\latim{
℟. Et os meum annuntiabit laudem tuam.
}\switchcolumn\portugues{
℟. E a minha boca anunciará o vosso louvor.
}\switchcolumn*\latim{
℣. Deus \cruz in adjutórium meum inténde.
}\switchcolumn\portugues{
℣. Deus, \cruz vinde em meu auxílio.
}\switchcolumn*\latim{
℟. Dómine, ad adjuvándum me festína.
}\switchcolumn\portugues{
℟. Senhor, apressai-Vos em socorrer-me.
}\switchcolumn*\latim{
Glória Patri, \&c.
}\switchcolumn\portugues{
Glória ao Pai, \&c.
}\end{paracol}

\emph{Desde o Sábado antes do Domingo da Septuagésima até ás vésperas do Sábado Santo, em vez de Allelúja, é dito:}

\begin{paracol}{2}\latim{
Laus tibi, Domine, Rex æternæ gloriæ.
}\switchcolumn\portugues{
Louvado sejais, ó Senhor, Rei da glória eterna.
}\end{paracol}

\paragraph{Invitatório}
\begin{paracol}{2}\latim{
Ave Maria, gratia plena, Dominus tecum.
}\switchcolumn\portugues{
Ave, Maria, cheia de graça, o Senhor é convosco.
}\end{paracol}

\paragraph{Salmo 94}
\begin{paracol}{2}\latim{
\rlettrine{V}{eníte,} exsultémus Dómino: * jubilémus Deo salutári nostro:
}\switchcolumn\portugues{
\rlettrine{V}{inde,} exultemos no Senhor: * cantemos alegres a de Deus nosso salvador:
}\switchcolumn*\latim{
Præoccupémus fáciem ejus in confessióne: * et in psalmis jubilémus ei.
}\switchcolumn\portugues{
Apresentemo-nos diante d’Ele em acção de graças: * e celebremo-l’O com salmos.
}\switchcolumn*\latim{
Ave Maria, gratia plena, Dominus tecum.
}\switchcolumn\portugues{
Ave, Maria, cheia de graça, o Senhor é convosco.
}\switchcolumn*\latim{
Quóniam Deus magnus Dóminus: * et Rex magnus super omnes deos.
}\switchcolumn\portugues{
Porque o Senhor é o Deus grande: * e o Rei grande sobre todos os deuses.
}\switchcolumn*\latim{
Quia in manu ejus sunt omnes fines terræ: * et altitúdines móntium ipsíus sunt.
}\switchcolumn\portugues{
Pois na sua mão estão todos os confins da terra: * e as alturas dos montes são suas.
}\switchcolumn*\latim{
Dominus tecum.
}\switchcolumn\portugues{
O Senhor é convosco.
}\switchcolumn*\latim{
Quóniam ipsíus est mare, et ipse fecit illud: * et siccam manus ejus formavérunt.
}\switchcolumn\portugues{
Seu é o mar e Ele o fez: * e as suas mãos formaram a terra árida.
}\switchcolumn*\latim{
Veníte, adorémus, et procidámus, * et plorémus ante Dóminum qui fecit nos.
}\switchcolumn\portugues{
Vinde, adoremos e prostremo-nos, * e choremos diante do Senhor que nos criou.
}\switchcolumn*\latim{
Quia ipse est Dóminus Deus noster, * et nos pópulus páscuæ ejus, et oves manus ejus.
}\switchcolumn\portugues{
Pois Ele é o Senhor nosso Deus, * e nós somos o povo do seu pasto e as ovelhas da sua manada.
}\switchcolumn*\latim{
Ave Maria, gratia plena, Dominus tecum.
}\switchcolumn\portugues{
Ave, Maria, cheia de graça, o Senhor é convosco.
}\switchcolumn*\latim{
Hódie si vocem ejus audiéritis, * nolíte obduráre corda vestra:
}\switchcolumn\portugues{
Se hoje ouvirdes a sua voz, * não endureceis os vossos corações:
}\switchcolumn*\latim{
Sicut in irritatióne secúndum diem tentatiónis in desérto: * ubi tentavérunt me patres vestri, probavérunt me, et vidérunt ópera mea.
}\switchcolumn\portugues{
Como quando me provocaram à ira, no dia da tentação no deserto: * onde vossos pais me tentaram, me testaram e viram as minhas obras.
}\switchcolumn*\latim{
Dominus tecum.
}\switchcolumn\portugues{
O Senhor é convosco.
}\switchcolumn*\latim{
Quadragínta annis offénsus fui generatióni illi, * et dixi: semper hi errant corde.
}\switchcolumn\portugues{
Quarenta anos estive irritado contra esta geração, * e disse: é um povo de coração errante.
}\switchcolumn*\latim{
Et isti non cognovérunt vias meas, ut jurávi in ira mea: * Si introíbunt in réquiem meam.
}\switchcolumn\portugues{
Eles não conheceram os meus caminhos, pelo que jurei na minha ira: * no meu repouso não entrarão.
}\switchcolumn*\latim{
Ave Maria, gratia plena, Dominus tecum.
}\switchcolumn\portugues{
Ave, Maria, cheia de graça, o Senhor é convosco.
}\switchcolumn*\latim{
Gloria Patri, et Filio, et Spiritui sancto: Sicut erat in principio, et nunc, et semper, et in sæcula sæculorum. Amen.
}\switchcolumn\portugues{
Glória ao Pai, e ao Filho e ao Espírito Santo. Assim como era no princípio, agora e sempre, e por todos os séculos dos séculos. Amen.
}\switchcolumn*\latim{
Dominus tecum.
}\switchcolumn\portugues{
O Senhor é convosco.
}\switchcolumn*\latim{
Ave Maria, gratia plena, Dominus tecum.
}\switchcolumn\portugues{
Ave, Maria, cheia de graça, o Senhor é convosco.
}\end{paracol}

\paragraphinfo{Hino Quem terra}{Página \pageref{quemterra}}

\emph{Os três Salmos seguintes, com suas Antífonas, dizem-se no Domingo, Segunda-feira e Quinta-feira}

\subsubsection{Primeiro Nocturno}
\begin{paracol}{2}\latim{
\emph{Ant.} Benedicta tu in mulieribus, et benedictus fructus ventris tui.
}\switchcolumn\portugues{
\emph{Ant.} Bendita sois vós entre as mulheres, e bendito é o fruto do vosso ventre.
}\end{paracol}

\paragraphinfo{Salmo 8}{Página \pageref{salmo8}}

\begin{paracol}{2}\latim{
\emph{Ant.} Benedicta tu in mulieribus, et benedictus fructus ventris tui.
}\switchcolumn\portugues{
\emph{Ant.} Bendita sois vós entre as mulheres, e bendito é o fruto do vosso ventre.
}\end{paracol}

\begin{paracol}{2}\latim{
\emph{Ant.} Sicut myrrha electa, odorem dedisti suavitatis, sancta Dei Genitrix.
}\switchcolumn\portugues{
\emph{Ant.} Como a preciosa mirra, exalastes suavíssima fragrância, ó santa Mãe de Deus.
}\end{paracol}

\paragraphinfo{Salmo 18}{Página \pageref{salmo18}}

\begin{paracol}{2}\latim{
\emph{Ant.} Sicut myrrha electa, odorem dedisti suavitatis, sancta Dei Genitrix.
}\switchcolumn\portugues{
\emph{Ant.} Como a preciosa mirra, exalastes suavíssima fragrância, ó santa Mãe de Deus.
}\end{paracol}

\begin{paracol}{2}\latim{
\emph{Ant.} Ante torum hujus Virginis frequentate nobis dulcia cantica dramatis.
}\switchcolumn\portugues{
\emph{Ant.} Multiplicai-nos doces cânticos ante o precioso leito desta Virgem.
}\end{paracol}

\paragraphinfo{Salmo 23}{Página \pageref{salmo23}}

\begin{paracol}{2}\latim{
\emph{Ant.} Ante torum hujus Virginis frequentate nobis dulcia cantica dramatis.
}\switchcolumn\portugues{
\emph{Ant.} Multiplicai-nos doces cânticos ante o precioso leito desta Virgem.
}\end{paracol}

\emph{Absolvições, Lições e Responsórios, como no fim do terceiro Nocturno.}

\emph{Os três Salmos seguintes, com suas Antífonas, dizem-se na Terça-Feira e Sexta-feira}

\subsubsection{Segundo Nocturno}
\begin{paracol}{2}\latim{
\emph{Ant.} Specie tua et pulchritudine tua intende, prospere procede, et regna.
}\switchcolumn\portugues{
\emph{Ant.} Ornada de glória e de formosura, caminhai prosperamente e reinai.
}\end{paracol}

\paragraphinfo{Salmo 44}{Página \pageref{salmo44}}

\begin{paracol}{2}\latim{
\emph{Ant.} Specie tua et pulchritudine tua intende, prospere procede, et regna.
}\switchcolumn\portugues{
\emph{Ant.} Ornada de glória e de formosura, caminhai prosperamente e reinai.
}\end{paracol}

\begin{paracol}{2}\latim{
\emph{Ant.} Adjuvabit eam Deus vultu suo: Deus in medio ejus, non commovebitur.
}\switchcolumn\portugues{
\emph{Ant.} Ajudou-a Deus com seu favorável aspecto; e como Deus assiste no meio dela, não se verá perturbada.
}\end{paracol}

\paragraphinfo{Salmo 45}{Página \pageref{salmo45}}

\begin{paracol}{2}\latim{
\emph{Ant.} Adjuvabit eam Deus vultu suo: Deus in medio ejus, non commovebitur.
}\switchcolumn\portugues{
\emph{Ant.} Ajudou-a Deus com seu favorável aspecto; e como Deus assiste no meio dela, não se verá perturbada.
}\end{paracol}

\begin{paracol}{2}\latim{
\emph{Ant.} Sicut lætantium omnium nostrum habitatio est in te, sancta Dei Genitrix.
}\switchcolumn\portugues{
\emph{Ant.} Santa Mãe de Deus, todos nossos que por amor habitam convosco estão cheios de alegria.
}\end{paracol}

\paragraphinfo{Salmo 86}{Página \pageref{salmo86}}

\begin{paracol}{2}\latim{
\emph{Ant.} Sicut lætantium omnium nostrum habitatio est in te, sancta Dei Genitrix.
}\switchcolumn\portugues{
\emph{Ant.} Santa Mãe de Deus, todos nossos que por amor habitam convosco estão cheios de alegria.
}\end{paracol}

\emph{Absolvições, Lições e Responsórios, como no fim do terceiro Nocturno.}

\emph{Os três Salmos seguintes, com suas Antífonas, dizem-se na Quarta-Feira e Sábado}

\subsubsection{Terceiro Nocturno}
\begin{paracol}{2}\latim{
\emph{Ant.} Gaude, Maria Virgo: cunctas hæreses sola intermenisti in universo mundo.
}\switchcolumn\portugues{
\emph{Ant.} Alegrai-vos, Virgem Maria: porque só vós haveis destruído todas as heresias em todo o mundo.
}\end{paracol}

\paragraphinfo{Salmo 95}{Página \pageref{salmo95}}

\begin{paracol}{2}\latim{
\emph{Ant.} Gaude, Maria Virgo: cunctas hæreses sola intermenisti in universo mundo.
}\switchcolumn\portugues{
\emph{Ant.} Alegrai-vos, Virgem Maria: porque só vós haveis destruído todas as heresias em todo o mundo.
}\end{paracol}

\begin{paracol}{2}\latim{
\emph{Ant.} Dignare me laudare te, Virgo sacrata: da mihi virtutem contra hostes tuos.
}\switchcolumn\portugues{
\emph{Ant.} Dignai-vos, sagrada Virgem, de que eu vos louve; dai-me esforço contra vossos inimigos.
}\end{paracol}

\paragraphinfo{Salmo 96}{Página \pageref{salmo96}}

\begin{paracol}{2}\latim{
\emph{Ant.} Dignare me laudare te, Virgo sacrata: da mihi virtutem contra hostes tuos.
}\switchcolumn\portugues{
\emph{Ant.} Dignai-vos, sagrada Virgem, de que eu vos louve; dai-me esforço contra vossos inimigos.
}\end{paracol}

\begin{paracol}{2}\latim{
\emph{Ant.} Post partum virgo inviolata permansisti: Dei Genitrix, intercede pro nobis.
}\switchcolumn\portugues{
\emph{Ant.} Depois do parto permanecestes virgem imaculada; Mãe de Deus, intercedei por nós.
}\end{paracol}

\paragraphinfo{Salmo 97}{Página \pageref{salmo97}}

\begin{paracol}{2}\latim{
\emph{Ant.} Post partum virgo inviolata permansisti: Dei Genitrix, intercede pro nobis.
}\switchcolumn\portugues{
\emph{Ant.} Depois do parto permanecestes virgem imaculada; Mãe de Deus, intercedei por nós.
}\end{paracol}

\paragraph{Versículo}
\begin{paracol}{2}\latim{
℣. Diffusa est gratia in labiis tuis.
}\switchcolumn\portugues{
℣. Estão cheios de graça vossos lábios.
}\switchcolumn*\latim{
℟. Propterea benedixit te Deum in æternum.
}\switchcolumn\portugues{
℟. Por isso Deus vos abençoou para sempre.
}\switchcolumn*\latim{
Pater Noster (secreto usque ad).
}\switchcolumn\portugues{
Pai Nosso (em silêncio).
}\switchcolumn*\latim{
℣. Et ne nos inducas in tentationem.
}\switchcolumn\portugues{
℣. E nos não deixeis cair em tentação.
}\switchcolumn*\latim{
℟. Sed libera nos a malo.
}\switchcolumn\portugues{
℟. Mas livrai-nos do mal.
}\end{paracol}

\paragraph{Absolvição}
\begin{paracol}{2}\latim{
\rlettrine{P}{recibus} et meritis beatæ Mariæ semper Virginis, et omnium Sanctorum, perducat nos Dominus ad regna cælorum.
}\switchcolumn\portugues{
\rlettrine{P}{elos} rogos e merecimentos da bem-aventurada Virgem Maria, e de todos os Santos, nos conduza o Senhor ao reino dos céus.
}\switchcolumn*\latim{
℟. Amen.
}\switchcolumn\portugues{
℟. Amen.
}\switchcolumn*\latim{
℣. Jube, Domine, benedicere.
}\switchcolumn\portugues{
℣. Dai-me, Senhor, a vossa bênção.
}\end{paracol}

\paragraph{Benção}
\begin{paracol}{2}\latim{
Nos cum prole pia benedicat Virgo Maria.
}\switchcolumn\portugues{
Nos abençoe a Virgem Maria com seu piíssimo Filho.
}\switchcolumn*\latim{
℟. Amen.
}\switchcolumn\portugues{
℟. Amen.
}\end{paracol}

\paragraphinfo{Lição 1}{Ecl. 24, 11-13}
\begin{paracol}{2}\latim{
\rlettrine{I}{n} omnibus requiem quæsivi, et in hereditate Domini morabor. Tunc præcepit, et dixit mihi Creator omnium: et qui creavit me, requievit in tabernaculo meo. Et dixit mihi: In Jacob inhabita, et in Israël hereditare, et in electis meis mitte radices.
}\switchcolumn\portugues{
\rlettrine{E}{m} todas as cousas procurei descanso, e na herança do Senhor farei morada. Então ordenou, e me disse o Criador de tudo; e O que me criou descansou no meu Tabernáculo, e disse-me: Tem a tua morada em Jacob, e a tua herança em Israel, e nos meus escolhidos lança raízes.
}\switchcolumn*\latim{
℣. Tu autem, Dómine, miserére nobis.
}\switchcolumn\portugues{
℣. E Vós, Senhor, tende misericórdia de nós.
}\switchcolumn*\latim{
℟. Deo grátias.
}\switchcolumn\portugues{
℟. Graças a Deus.
}\switchcolumn*\latim{
℟. Sancta et immaculáta virginitas, quibus te laudibus efferam nescio: Quia quem cæli cápere non póterant, tuo gremio contulísti.
}\switchcolumn\portugues{
℟. Santa e imaculada Virgindade, não sei com que louvores possa exaltar-vos. Porque encerrastes no vosso seio Aquele a quem os céus não podiam abranger.
}\switchcolumn*\latim{
℣. Benedicta tu in muliéribus, et benedíctus fructus ventris tui.
}\switchcolumn\portugues{
℣. Bendita sois vós entre as mulheres e bendito é o fruto do vosso ventre.
}\switchcolumn*\latim{
℟. Quia quem cæli cápere non póterant, tuo gremio contulisti.
}\switchcolumn\portugues{
℟. Porque encerrastes no vosso seio Aquele a quem os céus não podiam abranger.
}\switchcolumn*\latim{
℣. Iube domne benedicere.
}\switchcolumn\portugues{
℣. Dai-me, Senhor, a vossa bênção.
}\switchcolumn*\latim{
Ipsa Virgo Vírginum intercédat pro nobis ad Dóminum.
}\switchcolumn\portugues{
A mesma Virgem das virgens interceda por nós ao Senhor.
}\switchcolumn*\latim{
℟. Amen.
}\switchcolumn\portugues{
℟. Amen.
}\end{paracol}

\paragraphinfo{Lição 2}{Ecl. 24, 15-16}
\begin{paracol}{2}\latim{
\rlettrine{E}{t} sic in Sion firmata sum, et in civitate sanctificata similiter requievi, et in Ierúsalem potestas mea. Et radicavi in populo honorificato, et in parte Dei mei hereditas illius, et in plenitudine sanctorum detentio mea.
}\switchcolumn\portugues{
\rlettrine{E}{} desta maneira estou fundada em Sião, e semelhantemente repousei na cidade santificada; e em Jerusalém é o meu poder. E lancei raízes no povo honorificado, e na parte do meu Deus, herança sua, e na congregação dos santos fiz a minha morada.
}\switchcolumn*\latim{
℣. Tu autem, Dómine, miserére nobis.
}\switchcolumn\portugues{
℣. E Vós, Senhor, tende misericórdia de nós.
}\switchcolumn*\latim{
℟. Deo grátias.
}\switchcolumn\portugues{
℟. Demos graças a Deus.
}\switchcolumn*\latim{
℟. Beata es, Virgo Maria, quæ Dominum portasti, Creatorem mundi: Genuisti qui te fecit, et in æternum permanes Virgo.
}\switchcolumn\portugues{
℟. Bem-aventurada sois, ó Virgem Maria, que trouxestes no vosso ventre o Criador do mundo. Gerastes o que vos deu o ser, e ficastes para sempre Virgem.
}\switchcolumn*\latim{
℣. Ave Maria, gratia plena, Dominus tecum.
}\switchcolumn\portugues{
℣. Ave Maria, cheia de graça, o Senhor é convosco.
}\switchcolumn*\latim{
℟. Genuisti qui te fecit, et in æternum permanes Virgo.
}\switchcolumn\portugues{
℟. Gerastes O que vos deu o ser, e ficastes para sempre Virgem.
}\end{paracol}

\emph{Quando o Te Deum é dito depois da Terceira Lição, adiciona-se o seguinte no fim do Responsório:}

\begin{paracol}{2}\latim{
℣. Glória Patri, et Fílio, et Spirítui Sancto.
}\switchcolumn\portugues{
℣. Glória ao Pai e ao Filho e ao Espírito Santo.
}\switchcolumn*\latim{
℟. Genuisti qui te fecit, et in æternum permanes Virgo.
}\switchcolumn\portugues{
℟. Gerastes O que vos deu o ser, e ficastes para sempre Virgem.
}\switchcolumn*\latim{
℣. Iube domne benedicere.
}\switchcolumn\portugues{
℣. Dai-me, Senhor, a vossa bênção.
}\end{paracol}

\paragraph{Benção}
\begin{paracol}{2}\latim{
Per Vírginem matrem concédat nobis Dóminus salútem et pacem.
}\switchcolumn\portugues{
Pela Virgem Maria, nos conceda o Senhor a paz e a salvação.
}\switchcolumn*\latim{
℟. Amen.
}\switchcolumn\portugues{
℟. Amen.
}\end{paracol}

\paragraphinfo{Lição 3}{Ecl. 24, 17-20}
\begin{paracol}{2}\latim{
\qlettrine{Q}{uasi} cedrus exaltata sum in Libano, et quasi cypressus in monte Sion: Quasi palma exaltata sum in Cades, et quasi plantatio rosæ in Iericho: Quasi oliva speciosa in campis, et quasi platanus exaltata sum iuxta aquam in plateis. Sicut cinnamomum et balsamum aromatizans odorem dedi; quasi myrrha electa dedi suavitatem odoris:
}\switchcolumn\portugues{
\rlettrine{E}{xaltada} sou, qual cedro no Líbano, e qual cipreste no monte Sião. Exaltada sou, qual palma em Cades e como as rosas em Jericó. Qual especial oliveira nos campos, e qual plátano, sou exaltada junto da água nas praças. Assim como o cinamomo e o bálsamo, que difundem cheiro, dei eu fragrância; como a mirra, dei cheiro de suavidade.
}\switchcolumn*\latim{
℣. Tu autem, Dómine, miserére nobis.
}\switchcolumn\portugues{
℣. E Vós, Senhor, tende misericórdia de nós.
}\switchcolumn*\latim{
℟. Deo grátias.
}\switchcolumn\portugues{
℟. Demos graças a Deus.
}\end{paracol}

\emph{O Te Deum não é dito no Advento, ou da Septuagésima até à Páscoa, excluindo as Festas de Nossa Senhora. O seguinte Responsório é dito quando o Te Deum é omitido:}

\begin{paracol}{2}\latim{
℟. Felix namque es, sacra Virgo Maria, et omni laude dignissima: Quia ex te ortus est sol justitiæ, Christus Deus noster.
}\switchcolumn\portugues{
℟. Ditosa sois, ó sagrada Virgem Maria, e digníssima de todo o louvor. Porque de vós nasceu o sol de justiça, Jesus Cristo nosso Deus.
}\switchcolumn*\latim{
℣. Ora pro populo, interveni pro clero, intercede pro devoto femineo sexu: sentiant omnes tuum juvamen, quicumque celebrant tuam sanctam commemorationem.
}\switchcolumn\portugues{
℣. Rogai pelo povo, intercedei pelo clero, advogai pelo devoto sexo feminino; experimentem o vosso patrocínio os que celebram a vossa santa memória. Porque de vós nasceu o Sol de justiça, Jesus Cristo, nosso Deus.
}\switchcolumn*\latim{
℟. Quia ex te ortus est sol justitiæ.
}\switchcolumn\portugues{
℟. Porque de ti nasceu o Sol de justiça.
}\switchcolumn*\latim{
℣. Glória Patri, et Fílio, et Spirítui Sancto.
}\switchcolumn\portugues{
℣. Glória ao Pai e ao Filho e ao Espírito Santo.
}\switchcolumn*\latim{
℟. Christus Deus noster.
}\switchcolumn\portugues{
℟. Jesus Cristo, nosso Deus.
}\end{paracol}

\paragraphinfo{Te Deum}{Página \pageref{tedeum}}

\emph{As Matinas acabam depois do Terceiro Responsório ou do Te Deum, porque é usual depois das Matinas passar-se directamente para as Laudes. No entanto, se não continuar para as Laudes diz:}

\begin{paracol}{2}\latim{
℣. Domine, exaudi orationem meam.
}\switchcolumn\portugues{
℣. Ouvi, Senhor, a minha oração.
}\switchcolumn*\latim{
℟. Et clamor meus ad te veniat.
}\switchcolumn\portugues{
℟. E o meu clamor chegue até Vós.
}\switchcolumn*\latim{
\begin{nscenter} Orémus. \end{nscenter}
}\switchcolumn\portugues{
\begin{nscenter} Oremos. \end{nscenter}
}\switchcolumn*\latim{
\rlettrine{C}{oncede} nos famulos tuos, quǽsumus, Domine Deus, perpetua mentis et corporis sanitate gaudere: et gloriosa beatæ Mariæ semper Virginis intercessione, a præsenti liberari tristitia, et æterna perfrui lætitia. Per Dominum nostrum Jesum Christum.
}\switchcolumn\portugues{
\rlettrine{S}{enhor} Deus, nós Vos suplicamos que concedais a vossos servos lograr uma perpétua saúde de corpo e alma, e que pela intercessão gloriosa da bem-aventurada sempre Virgem Maria sejamos livres da presente tristeza, e gozemos da eterna alegria. Por Jesus Cristo nosso Senhor.
}\switchcolumn*\latim{
℟. Amen.
}\switchcolumn\portugues{
℟. Amen.
}\switchcolumn*\latim{
℣. Domine, exaudi orationem meam.
}\switchcolumn\portugues{
℣. Ouvi, Senhor, a minha oração.
}\switchcolumn*\latim{
℟. Et clamor meus ad te veniat.
}\switchcolumn\portugues{
℟. E o meu clamor chegue até Vós.
}\switchcolumn*\latim{
℣. Benedicamus Domino.
}\switchcolumn\portugues{
℣. Bendigamos o Senhor.
}\switchcolumn*\latim{
℟. Deo gratias.
}\switchcolumn\portugues{
℟. Graças a Deus.
}\switchcolumn*\latim{
℣. Fidelium animæ per misericordiam Dei, requiescant in pace.
}\switchcolumn\portugues{
℣. E que as almas dos fiéis, pela misericórdia de Deus, descansem em paz.
}\switchcolumn*\latim{
℟. Amen.
}\switchcolumn\portugues{
℟. Amen.
}\end{paracol}

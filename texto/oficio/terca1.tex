\subsection{Terça 1}

\begin{paracol}{2}\latim{
℣. Deus \cruz in adjutórium meum inténde.
}\switchcolumn\portugues{
℣. Deus, \cruz vinde em meu auxílio.
}\switchcolumn*\latim{
℟. Dómine, ad adjuvándum me festína.
}\switchcolumn\portugues{
℟. Senhor, apressai-Vos em socorrer-me.
}\switchcolumn*\latim{
Glória Patri, \&c.
}\switchcolumn\portugues{
Glória ao Pai, \&c.
}\end{paracol}

\paragraphinfo{Hino Memento rerum conditor}{Página \pageref{mementorerumconditor}}

\begin{paracol}{2}\latim{
\emph{Ant.} Maria Virgo assumpta est ad ætherum thalamum, in quo Rex regum stellato sedet solio.
}\switchcolumn\portugues{
\emph{Ant.} Maria Virgem foi sublimada ao tálamo celeste, onde o Rei dos reis está sentado num trono de estrelas.
}\end{paracol}

\paragraphinfo{Salmo 119}{Página \pageref{salmo119}}

\paragraphinfo{Salmo 120}{Página \pageref{salmo120}}

\paragraphinfo{Salmo 121}{Página \pageref{salmo121}}

\begin{paracol}{2}\latim{
\emph{Ant.} Maria Virgo assumpta est ad ætherum thalamum, in quo Rex regum stellato sedet solio.
}\switchcolumn\portugues{
\emph{Ant.} Maria Virgem foi sublimada ao tálamo celeste, onde o Rei dos reis está sentado num trono de estrelas.
}\end{paracol}

\paragraphinfo{Pequeno Capítulo}{Ecl. 24, 15}
\begin{paracol}{2}\latim{
\rlettrine{E}{t} sic in Sion firmata sum, et in civitate sanctificata similiter requievi, et in Jerúsalem potestas mea.
}\switchcolumn\portugues{
\rlettrine{E}{} desta maneira estou fundada em Sião, e semelhantemente repousei na cidade santificada, e em Jerúsalem é o meu poder.
}\switchcolumn*\latim{
℟. Deo grátias.
}\switchcolumn\portugues{
℟. Graças a Deus.
}\switchcolumn*\latim{
℣. Diffusa est gratia in labiis tuis.
}\switchcolumn\portugues{
℣. A graça derramou-se nos vossos lábios.
}\switchcolumn*\latim{
℟. Propterea benedixit te Deus in æternum.
}\switchcolumn\portugues{
℟. Por isso vos abençoou Deus para sempre.
}\end{paracol}


\begin{paracol}{2}\latim{
\emph{(Hic genuflectitur)} Kyrie eleison
}\switchcolumn\portugues{
\emph{(Genuflectir)} Senhor, tende piedade de nós.
}\switchcolumn*\latim{
Christe, eléison.
}\switchcolumn\portugues{
Cristo, tende piedade de nós.
}\switchcolumn*\latim{
Kyrie, eléison.
}\switchcolumn\portugues{
Senhor, tende piedade de nós.
}\switchcolumn*\latim{
℣. Domine, exaudi orationem meam.
}\switchcolumn\portugues{
℣. Ouvi, Senhor, a minha oração.
}\switchcolumn*\latim{
℟. Et clamor meus ad te veniat.
}\switchcolumn\portugues{
℟. E o meu clamor chegue até Vós.
}\end{paracol}

\begin{paracol}{2}\latim{
\begin{nscenter} Orémus. \end{nscenter}
}\switchcolumn\portugues{
\begin{nscenter} Oremos. \end{nscenter}
}\switchcolumn*\latim{
\rlettrine{D}{eus,} qui salutis aeternae, beatae Mariae virginitate fecunda, humano generi praemia praestitisti: tribue, quaesumus; ut ipsam pro nobis intercedere sentiamus, per quam meruimus auctorem vitae suscipere, Dominum nostrum Jesum Christum Filium tuum: Qui tecum vivit et regnat \emph{\&c.}
}\switchcolumn\portugues{
\slettrine{Ó}{} Deus, que pela virgindade fecunda da B. Maria, participastes ao género humano os prémios da salvação eterna: concedei-nos, Vos rogamos, que experimentemos quanto é poderosa a nosso favor a intercessão daquela Virgem, pela qual merecemos receber o autor da vida nosso Senhor Jesus Cristo, Filho vosso: que convosco Vive e reina \emph{\&c.}
}\switchcolumn*\latim{
℟. Amen.
}\switchcolumn\portugues{
℟. Amen.
}\switchcolumn*\latim{
℣. Domine, exaudi orationem meam.
}\switchcolumn\portugues{
℣. Ouvi, Senhor, a minha oração.
}\switchcolumn*\latim{
℟. Et clamor meus ad te veniat.
}\switchcolumn\portugues{
℟. E o meu clamor chegue até Vós.
}\switchcolumn*\latim{
℣. Benedicamus Domino.
}\switchcolumn\portugues{
℣. Bendigamos o Senhor.
}\switchcolumn*\latim{
℟. Deo gratias.
}\switchcolumn\portugues{
℟. Graças a Deus.
}\switchcolumn*\latim{
℣. Fidelium animæ per misericordiam Dei, requiescant in pace.
}\switchcolumn\portugues{
℣. E que as almas dos fiéis, pela misericórdia de Deus, descansem em paz.
}\switchcolumn*\latim{
℟. Amen.
}\switchcolumn\portugues{
℟. Amen.
}\end{paracol}

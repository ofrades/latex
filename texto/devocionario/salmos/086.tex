\begin{paracol}{2}\latim{
\rlettrine{F}{undaménta} ejus in móntibus sanctis: * díligit Dóminus portas Sion super ómnia tabernácula Jacob.
}\switchcolumn\portugues{
\rlettrine{O}{s} seus fundamentos estão sobre os montes santos: * o Senhor ama as portas de Sião mais que todos os tabernáculos de Jacob.
}\switchcolumn*\latim{
Gloriósa dicta sunt de te, * cívitas Dei.
}\switchcolumn\portugues{
Cousas gloriosas se têm dito de ti, * ó cidade de Deus.
}\switchcolumn*\latim{
Memor ero Rahab, et Babylónis * sciéntium me.
}\switchcolumn\portugues{
Lembrar-me-ei de Raab e de Babilónia, * que me conhecem.
}\switchcolumn*\latim{
Ecce, alienígenæ, et Tyrus, et pópulus Æthíopum, * hi fuérunt illic.
}\switchcolumn\portugues{
Eis os estrangeiros, Tiro e o povo dos Etíopes, * todos estes estarão lá.
}\switchcolumn*\latim{
Numquid Sion dicet: homo, et homo natus est in ea: * et ipse fundávit eam Altíssimus?
}\switchcolumn\portugues{
Porventura se não dirá a Sião: um grande número de homens nasceu nela: * e a fundou o mesmo Altíssimo?
}\switchcolumn*\latim{
Dóminus narrábit in scriptúris populórum, et príncipum: * horum, qui fuérunt in ea.
}\switchcolumn\portugues{
O Senhor poderá contar, no registo dos povos e dos príncipes: * o número daqueles que nela estiveram.
}\switchcolumn*\latim{
Sicut lætántium ómnium * habitátio est in te.
}\switchcolumn\portugues{
Estão cheios de alegria todos * os que habitam dentro de ti.
}\end{paracol}

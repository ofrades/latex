\begin{paracol}{2}\latim{
\rlettrine{L}{audáte} Dóminum quóniam bonus est psalmus: * Deo nostro sit jucúnda, decóraque laudátio.
}\switchcolumn\portugues{
\rlettrine{L}{ouvai} o Senhor, porque é bom salmodiar: * sê alegre para o nosso Deus, louvai-O graciosamente.
}\switchcolumn*\latim{
Ædíficans Jerúsalem Dóminus: * dispersiónes Israélis congregábit.
}\switchcolumn\portugues{
O Senhor que edifica Jerusalém: * congregará os dispersos de Israel.
}\switchcolumn*\latim{
Qui sanat contrítos corde: * et álligat contritiónes eórum.
}\switchcolumn\portugues{
É Ele que sara os de coração contrito: * e liga as suas chagas.
}\switchcolumn*\latim{
Qui númerat multitúdinem stellárum: * et ómnibus eis nómina vocat.
}\switchcolumn\portugues{
É Ele que conta a multidão das estrelas: * e as chama todas pelos seus nomes.
}\switchcolumn*\latim{
Magnus Dóminus noster, et magna virtus ejus: * et sapiéntiæ ejus non est númerus.
}\switchcolumn\portugues{
Grande é o nosso Senhor e grande o seu poder: * e a sua sabedoria não tem limites.
}\switchcolumn*\latim{
Suscípiens mansuétos Dóminus: * humílians autem peccatóres usque ad terram.
}\switchcolumn\portugues{
O Senhor é quem ampara os mansos: * e abate os pecadores até à terra.
}\switchcolumn*\latim{
Præcínite Dómino in confessióne: * psállite Deo nostro in cíthara.
}\switchcolumn\portugues{
Entoai cânticos ao Senhor em seu louvor: * cantai ao nosso Deus com a cítara.
}\switchcolumn*\latim{
Qui óperit cælum núbibus: * et parat terræ plúviam.
}\switchcolumn\portugues{
É Ele que cobre o céu de nuvens: * e prepara assim chuva para a terra.
}\switchcolumn*\latim{
Qui prodúcit in móntibus fænum: * et herbam servitúti hóminum.
}\switchcolumn\portugues{
É Ele que produz feno nos montes: * e erva para serviço dos homens.
}\switchcolumn*\latim{
Qui dat juméntis escam ipsórum: * et pullis corvórum invocántibus eum.
}\switchcolumn\portugues{
É Ele que dá aos animais o seu alimento próprio: * e aos filhinhos dos corvos que O chamam.
}\switchcolumn*\latim{
Non in fortitúdine equi voluntátem habébit: * nec in tíbiis viri beneplácitum erit ei.
}\switchcolumn\portugues{
Não se agradará da força do cavalo: * nem se agradará nos pés robustos do varão.
}\switchcolumn*\latim{
Beneplácitum est Dómino super timéntes eum: * et in eis, qui sperant super misericórdia ejus.
}\switchcolumn\portugues{
O Senhor agradou-se sempre dos que O temem: * e daqueles que esperam na sua misericórdia.
}\end{paracol}

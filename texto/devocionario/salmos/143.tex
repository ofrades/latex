\begin{paracol}{2}\latim{
\rlettrine{B}{enedíctus} Dóminus, Deus meus, qui docet manus meas ad prǽlium, * et dígitos meos ad bellum.
}\switchcolumn\portugues{
\rlettrine{B}{endito} seja o Senhor meu Deus, que adestra as minhas mãos para a batalha, * e os meus dedos para a guerra.
}\switchcolumn*\latim{
Misericórdia mea, et refúgium meum: * suscéptor meus, et liberátor meus:
}\switchcolumn\portugues{
Minha misericórdia e o meu refúgio: * meu defensor e meu libertador:
}\switchcolumn*\latim{
Protéctor meus, et in ipso sperávi: * qui subdit pópulum meum sub me.
}\switchcolumn\portugues{
Meu protector e é n’Ele que espero: * é quem submete o meu povo à minha autoridade.
}\switchcolumn*\latim{
Dómine, quid est homo, quia innotuísti ei? * aut fílius hóminis, quia réputas eum?
}\switchcolumn\portugues{
Que é o homem, ó Senhor, para que a ele Vos tenhais manifestado? * Ou o filho do homem, para assim o estimardes?
}\switchcolumn*\latim{
Homo vanitáti símilis factus est: * dies ejus sicut umbra prætéreunt.
}\switchcolumn\portugues{
O homem fez-se semelhante à vaidade: * seus dias passam como a sombra.
}\switchcolumn*\latim{
Dómine, inclína cælos tuos, et descénde: * tange montes, et fumigábunt.
}\switchcolumn\portugues{
Senhor, inclinai os vossos céus e descei: * tocai os montes e fumegarão.
}\switchcolumn*\latim{
Fúlgura coruscatiónem, et dissipábis eos: * emítte sagíttas tuas, et conturbábis eos.
}\switchcolumn\portugues{
Desferi raios e os dissipareis: * disparai as vossas setas e conturbá-los-eis.
}\switchcolumn*\latim{
Emítte manum tuam de alto, éripe me, et líbera me de aquis multis: * de manu filiórum alienórum.
}\switchcolumn\portugues{
Enviai a vossa mão lá do alto, tirai-me e livrai-me das muitas águas: * da mão dos filhos estranhos.
}\switchcolumn*\latim{
Quorum os locútum est vanitátem: * et déxtera eórum, déxtera iniquitátis.
}\switchcolumn\portugues{
Cuja boca falou vaidade: * e cuja direita é uma direita de iniquidade.
}\switchcolumn*\latim{
Deus, cánticum novum cantábo tibi: * in psaltério decachórdo psallam tibi.
}\switchcolumn\portugues{
A Vós, ó Deus, cantarei um cântico novo: * com o saltério de dez cordas Vos louvarei.
}\switchcolumn*\latim{
Qui das salútem régibus: * qui redemísti David, servum tuum, de gládio malígno: éripe me.
}\switchcolumn\portugues{
Vós que dais saúde aos reis: * que livrastes vosso servo David da espada maligna: livrai-me.
}\switchcolumn*\latim{
Et érue me de manu filiórum alienórum, quorum os locútum est vanitátem: * et déxtera eórum, déxtera iniquitátis:
}\switchcolumn\portugues{
E tirai-me da mão dos filhos estranhos, cuja boca falou vaidade: * e cuja direita é uma direita de iniquidade.
}\switchcolumn*\latim{
Quorum fílii, sicut novéllæ plantatiónes * in juventúte sua.
}\switchcolumn\portugues{
Cujos filhos são como novas plantas * na sua mocidade.
}\switchcolumn*\latim{
Fíliæ eórum compósitæ: * circumornátæ ut similitúdo templi.
}\switchcolumn\portugues{
Suas filhas decoradas: * adornadas como um templo.
}\switchcolumn*\latim{
Promptuária eórum plena, * eructántia ex hoc in illud.
}\switchcolumn\portugues{
Seus celeiros estão cheios, * a trasbordar duns para outros.
}\switchcolumn*\latim{
Oves eórum fœtósæ, abundántes in egréssibus suis: * boves eórum crassæ.
}\switchcolumn\portugues{
Suas ovelhas são fecundas, vão pastar abundantemente: * as suas vacas são gordas.
}\switchcolumn*\latim{
Non est ruína macériæ, neque tránsitus: * neque clamor in platéis eórum.
}\switchcolumn\portugues{
Não há ruína de muro, nem passagem na sua cerca: * nem gritos nas suas praças.
}\switchcolumn*\latim{
Beátum dixérunt pópulum, cui hæc sunt: * beátus pópulus, cujus Dóminus Deus ejus.
}\switchcolumn\portugues{
Bem-aventurado chamarão ao povo que tem estes bens: * bem-aventurado o povo que tem o Senhor por seu Deus.
}\end{paracol}

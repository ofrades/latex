\begin{paracol}{2}\latim{
\blettrine{B}{eátus} vir, qui non ábiit in consílio impiórum, et in via peccatórum non stetit, * et in cáthedra pestiléntiæ non sedit:
}\switchcolumn\portugues{
\blettrine{B}{em-aventurado} o varão que não foi no conselho dos ímpios, nem ficou no caminho dos pecadores, * e na cadeira pestilencial se não sentou:
}\switchcolumn*\latim{
Sed in lege Dómini volúntas ejus, * et in lege ejus meditábitur die ac nocte.
}\switchcolumn\portugues{
Mas sua vontade está na lei do Senhor, * e dia e noite meditará na sua lei.
}\switchcolumn*\latim{
Et erit tamquam lignum, quod plantátum est secus decúrsus aquárum, * quod fructum suum dabit in témpore suo:
}\switchcolumn\portugues{
Ele será como a árvore, que está plantada junto ao curso das águas, * que a seu tempo dará seu fruto:
}\switchcolumn*\latim{
Et fólium ejus non défluet: * et ómnia quæcúmque fáciet, prosperabúntur.
}\switchcolumn\portugues{
Cuja folha não murchará: * e prosperará tudo quanto fizer.
}\switchcolumn*\latim{
Non sic ímpii, non sic: * sed tamquam pulvis, quem proícit ventus a fácie terræ.
}\switchcolumn\portugues{
Não assim os ímpios, não assim: * mas serão como o pó que o vento dispersa da face da terra.
}\switchcolumn*\latim{
Ideo non resúrgent ímpii in judício: * neque peccatóres in concílio justórum.
}\switchcolumn\portugues{
Por isso os ímpios não ressuscitarão no juízo: * nem os pecadores no concílio dos justos.
}\switchcolumn*\latim{
Quóniam novit Dóminus viam justórum: * et iter impiórum períbit.
}\switchcolumn\portugues{
Porque o Senhor conhece o caminho dos justos: * e o caminho dos ímpios perecerá.
}\end{paracol}

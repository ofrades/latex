\begin{paracol}{2}\latim{
\rlettrine{C}{onfitébor} tibi, Dómine, in toto corde meo: * narrábo ómnia mirabília tua.
}\switchcolumn\portugues{
\rlettrine{E}{u} Vos glorificarei, ó Senhor, com todo meu coração: * narrarei todas vossas maravilhas.
}\switchcolumn*\latim{
Lætábor et exsultábo in Te: * psallam nómini tuo, Altíssime.
}\switchcolumn\portugues{
Alegrar-me-ei e em Vós exultarei: * cantarei o vosso nome, ó Altíssimo.
}\switchcolumn*\latim{
In converténdo inimícum meum retrórsum: * infirmabúntur, et períbunt a fácie tua.
}\switchcolumn\portugues{
Quando baterem em retirada os meus inimigos: * cairão e perecerão ante Vós.
}\switchcolumn*\latim{
Quóniam fecísti judícium meum et causam meam: * sedísti super thronum, qui júdicas justítiam.
}\switchcolumn\portugues{
Porque julgastes e defendestes a minha causa: * sentastes-Vos sobre o trono, Vós que justamente julgais.
}\switchcolumn*\latim{
Increpásti gentes, et périit ímpius: * nomen eórum delésti in ætérnum, et in sǽculum sǽculi.
}\switchcolumn\portugues{
Repreendestes as gentes e o ímpio pereceu: * apagastes o nome delas para sempre e por todos os séculos dos séculos.
}\switchcolumn*\latim{
Inimíci defecérunt frámeæ in finem: * et civitátes eórum destruxísti.
}\switchcolumn\portugues{
As espadas do inimigo ficaram embotadas para sempre: * e as suas cidades destruístes.
}\switchcolumn*\latim{
Périit memória eórum cum sónitu: * et Dóminus in ætérnum pérmanet.
}\switchcolumn\portugues{
Com estrondo pereceu a memória deles: * mas o Senhor permanece eternamente.
}\switchcolumn*\latim{
Parávit in judício thronum suum: * et ipse judicábit orbem terræ in æquitáte, judicábit pópulos in justítia.
}\switchcolumn\portugues{
Preparou o seu trono para o juízo: * e Ele mesmo julgará com equidade toda a terra, julgará os povos com justiça.
}\switchcolumn*\latim{
Et factus est Dóminus refúgium páuperi: * adjútor in opportunitátibus, in tribulatióne.
}\switchcolumn\portugues{
O Senhor fez-se o refúgio do pobre: * socorrendo-o oportunamente na tribulação.
}\switchcolumn*\latim{
Et sperent in Te qui novérunt nomen tuum: * quóniam non dereliquísti quæréntes Te, Dómine.
}\switchcolumn\portugues{
Em Vós esperem os que conhecem o vosso nome: * porque Vós, ó Senhor, não desamparastes os que Vos buscam.
}\switchcolumn*\latim{
Psállite Dómino, qui hábitat in Sion: * annuntiáte inter gentes stúdia ejus:
}\switchcolumn\portugues{
Cantai ao Senhor que habita em Sião: * anunciai os seus desígnios entre as gentes:
}\switchcolumn*\latim{
Quóniam requírens sánguinem eórum recordátus est: * non est oblítus clamórem páuperum.
}\switchcolumn\portugues{
Porque, vingando o seu sangue, mostrou que delas se lembra: * do clamor dos pobres se não esqueceu.
}\switchcolumn*\latim{
Miserére mei, Dómine: * vide humilitátem meam de inimícis meis.
}\switchcolumn\portugues{
Tende compaixão de mim, Senhor: * vede o meu abatimento que vem dos meus inimigos.
}\switchcolumn*\latim{
Qui exáltas me de portis mortis, * ut annúntiem omnes laudatiónes tuas in portis fíliæ Sion.
}\switchcolumn\portugues{
Que me ergueis das portas da morte, * para que anuncie todos vossos louvores às portas da filha de Sião.
}\switchcolumn*\latim{
Exsultábo in salutári tuo: * infíxæ sunt gentes in intéritu, quem fecérunt.
}\switchcolumn\portugues{
Exultarei na salvação que me obtivestes: * as gentes caíram na ruína que me tinham preparado.
}\switchcolumn*\latim{
In láqueo isto, quem abscondérunt, * comprehénsus est pes eórum.
}\switchcolumn\portugues{
No laço que me tinham preparado, * o seu pé ficou preso.
}\switchcolumn*\latim{
Cognoscétur Dóminus judícia fáciens: * in opéribus mánuum suárum comprehénsus est peccátor.
}\switchcolumn\portugues{
Conhecer-se-á que o Senhor faz justiça: * nas obras das suas mãos foi preso o pecador.
}\switchcolumn*\latim{
Convertántur peccatóres in inférnum, * omnes gentes quæ obliviscúntur Deum.
}\switchcolumn\portugues{
Sejam precipitados no inferno todos os pecadores, * todos as gentes que de Deus se esquecem.
}\switchcolumn*\latim{
Quóniam non in finem oblívio erit páuperis: * patiéntia páuperum non períbit in finem.
}\switchcolumn\portugues{
Porque não estará para sempre esquecido o pobre: * nem a paciência dos infelizes será para sempre frustrada.
}\switchcolumn*\latim{
Exsúrge, Dómine, non confortétur homo: * judicéntur gentes in conspéctu tuo.
}\switchcolumn\portugues{
Levantai-Vos, ó Senhor, não triunfe o homem: * sejam julgadas as gentes em vossa presença.
}\switchcolumn*\latim{
Constítue, Dómine, legislatórem super eos: * ut sciant gentes quóniam hómines sunt.
}\switchcolumn\portugues{
Senhor, estabelecei sobre elas um legislador: * para que as gentes saibam que são apenas homens.
}\switchcolumn*\latim{
Ut quid, Dómine, recessísti longe, * déspicis in opportunitátibus, in tribulatióne?
}\switchcolumn\portugues{
Senhor, porque Vos apartastes para longe, * desamparais-nos nas necessidades e na tribulação?
}\switchcolumn*\latim{
Dum supérbit ímpius, incénditur pauper: * comprehendúntur in consíliis quibus cógitant.
}\switchcolumn\portugues{
Enquanto o ímpio se envaidece, o pobre é abrasado: * são apanhados nas intrigas que teceram.
}\switchcolumn*\latim{
Quóniam laudátur peccátor in desidériis ánimæ suæ: * et iníquus benedícitur.
}\switchcolumn\portugues{
Pois o pecador vangloria-se nos desejos da sua alma: * e o iníquo é felicitado.
}\switchcolumn*\latim{
Exacerbávit Dóminum peccátor, * secúndum multitúdinem iræ suæ non quǽret.
}\switchcolumn\portugues{
O pecador exacerbou o Senhor, * devido à sua grande ira Ele o não procurará.
}\switchcolumn*\latim{
Non est Deus in conspéctu ejus: * inquinátæ sunt viæ illíus in omni témpore.
}\switchcolumn\portugues{
Não há Deus diante dele: * os seus caminhos são sempre viciosos.
}\switchcolumn*\latim{
Auferúntur judícia tua a fácie ejus: * ómnium inimicórum suórum dominábitur.
}\switchcolumn\portugues{
Não estão ante sua vista Vossos juízos: * dominará ele todos seus inimigos.
}\switchcolumn*\latim{
Dixit enim in corde suo: * Non movébor a generatióne in generatiónem sine malo.
}\switchcolumn\portugues{
Pois disse no seu coração: * não serei movido de geração em geração e do mal estarei livre.
}\switchcolumn*\latim{
Cujus maledictióne os plenum est, et amaritúdine, et dolo: * sub lingua ejus labor et dolor.
}\switchcolumn\portugues{
Sua boca está cheia de maledicência, de amargura e de dolo: * debaixo da sua língua estão o trabalho e a dor.
}\switchcolumn*\latim{
Sedet in insídiis cum divítibus in occúltis: * ut interfíciat innocéntem.
}\switchcolumn\portugues{
Senta-se em emboscada com os ricos em lugares ocultos: * para o inocente matar.
}\switchcolumn*\latim{
Óculi ejus in páuperem respíciunt: * insidiátur in abscóndito, quasi leo in spelúnca sua.
}\switchcolumn\portugues{
Seus olhos estão sobre o pobre: * aguarda escondido como o leão na sua cova.
}\switchcolumn*\latim{
Insidiátur ut rápiat páuperem: * rápere páuperem, dum áttrahit eum.
}\switchcolumn\portugues{
Arma ciladas para arrebatar o pobre: * para arrebatar o pobre, atraindo-o a si.
}\switchcolumn*\latim{
In láqueo suo humiliábit eum: * inclinábit se, et cadet, cum dominátus fúerit páuperum.
}\switchcolumn\portugues{
No seu laço ele fá-lo-á cair: * inclinar-se-á e cairá sobre os pobres, logo que se apoderar deles.
}\switchcolumn*\latim{
Dixit enim in corde suo: oblítus est Deus, * avértit fáciem suam ne vídeat in finem.
}\switchcolumn\portugues{
Pois disse no seu coração: Deus esqueceu-se, * virou o seu rosto para até ao fim não ver.
}\switchcolumn*\latim{
Exsúrge, Dómine Deus, exaltétur manus tua: * ne obliviscáris páuperum.
}\switchcolumn\portugues{
Levantai-Vos, ó Senhor Deus, eleve-se a vossa mão: * e dos pobres Vos não esqueçais.
}\switchcolumn*\latim{
Propter quid irritávit ímpius Deum? * Dixit enim in corde suo: non requíret.
}\switchcolumn\portugues{
Por que motivo o ímpio irritou a Deus? * Porque disse no seu coração: Ele não exige.
}\switchcolumn*\latim{
Vides quóniam Tu labórem et dolórem consíderas: * ut tradas eos in manus tuas.
}\switchcolumn\portugues{
Porém, Vós o vedes, considerais o trabalho e a dor: * para o tomardes nas vossas mãos.
}\switchcolumn*\latim{
Tibi derelíctus est pauper: * órphano Tu eris adjútor.
}\switchcolumn\portugues{
A Vós se abandona o infeliz: * sereis Vós o amparo do órfão.
}\switchcolumn*\latim{
Cóntere brácchium peccatóris et malígni: * quærétur peccátum illíus, et non inveniétur.
}\switchcolumn\portugues{
Quebrai o braço do pecador e do maligno: * buscar-se-á o seu pecado e se não achará.
}\switchcolumn*\latim{
Dóminus regnábit in ætérnum, et in sǽculum sǽculi: * períbitis, gentes, de terra illíus.
}\switchcolumn\portugues{
O Senhor reinará eternamente e pelos séculos dos séculos: * vós, ó gentes, sereis exterminadas da sua terra.
}\switchcolumn*\latim{
Desidérium páuperum exaudívit Dóminus: * præparatiónem cordis eórum audívit auris tua.
}\switchcolumn\portugues{
O Senhor ouviu o desejo dos pobres: * o vosso ouvido atendeu à prece do seu coração.
}\switchcolumn*\latim{
Judicáre pupíllo et húmili, * ut non appónat ultra magnificáre se homo super terram.
}\switchcolumn\portugues{
Para fazerdes justiça ao órfão e ao humilde, * a fim de que o homem cesse de se engrandecer sobre a terra.
}\end{paracol}

\begin{paracol}{2}\latim{
\rlettrine{E}{xsúrgat} Deus, et dissipéntur inimíci ejus, * et fúgiant qui odérunt eum, a fácie ejus.
}\switchcolumn\portugues{
\rlettrine{L}{evante-se} Deus e sejam dispersos os seus inimigos, * e da sua presença fujam os que o odeiam.
}\switchcolumn*\latim{
Sicut déficit fumus, defíciant: * sicut fluit cera a fácie ignis, sic péreant peccatóres a fácie Dei.
}\switchcolumn\portugues{
Se desvaneçam assim como se desvanece o fumo: * assim como se derrete a cera diante do fogo, assim pereçam os pecadores ante Deus.
}\switchcolumn*\latim{
Et justi epuléntur, et exsúltent in conspéctu Dei: * et delecténtur in lætítia.
}\switchcolumn\portugues{
Os justos, porém, banqueteiem-se e regozijem-se na presença de Deus: * e com alegria se alegrem.
}\switchcolumn*\latim{
Cantáte Deo, psalmum dícite nómini ejus: * iter fácite ei, qui ascéndit super occásum: \emph{(fit reverentia)} Dóminus nomen illi.
}\switchcolumn\portugues{
Cantai a Deus, cantai salmos ao seu nome: * abri o caminho Àquele que sobe para o ocidente: \emph{(inclinar a cabeça)} o Senhor é o seu nome.
}\switchcolumn*\latim{
Exsultáte in conspéctu ejus: * turbabúntur a fácie ejus, patris orphanórum et júdicis viduárum.
}\switchcolumn\portugues{
Regozijai-vos diante d’Ele: * perturbar-se-ão diante d’Ele, Ele é o pai dos órfãos e o juiz das viúvas.
}\switchcolumn*\latim{
Deus in loco sancto suo: * Deus, qui inhabitáre facit uníus moris in domo:
}\switchcolumn\portugues{
Deus está no seu lugar santo: * é Deus que faz habitar na casa os solitários:
}\switchcolumn*\latim{
Qui edúcit vinctos in fortitúdine, * simíliter eos qui exásperant, qui hábitant in sepúlcris.
}\switchcolumn\portugues{
Que põe em liberdade os cativos com seu poder, * mesmo aqueles que o irritam, os quais moram nos sepulcros.
}\switchcolumn*\latim{
Deus, cum egrederéris in conspéctu pópuli tui, * cum pertransíres in desérto:
}\switchcolumn\portugues{
Ó Deus, quando saíeis à frente de vosso povo, * quando atravessáveis o deserto:
}\switchcolumn*\latim{
Terra mota est, étenim cæli distillavérunt a fácie Dei Sínai, * a fácie Dei Israël.
}\switchcolumn\portugues{
A terra tremeu e os céus destilaram, ante a face do Deus do Sinai, * diante do Deus de Israel.
}\switchcolumn*\latim{
Plúviam voluntáriam segregábis, Deus, hereditáti tuæ: * et infirmáta est, Tu vero perfecísti eam.
}\switchcolumn\portugues{
Ó Deus, reservastes uma chuva abundante para a vossa herança: * e, quando ela enfraqueceu, Vós a aperfeiçoastes.
}\switchcolumn*\latim{
Animália tua habitábunt in ea: * parásti in dulcédine tua páuperi, Deus.
}\switchcolumn\portugues{
Nela morarão as vossas criaturas: * na vossa doçura, ó Deus, provestes o pobre.
}\switchcolumn*\latim{
Dóminus dabit verbum evangelizántibus, * virtúte multa.
}\switchcolumn\portugues{
O Senhor dará a palavra aos que anunciam a boa nova, * com grande coragem.
}\switchcolumn*\latim{
Rex virtútum dilécti dilécti: * et speciéi domus divídere spólia.
}\switchcolumn\portugues{
Rei dos exércitos será do amado, do amado: * e a formosura da casa repartirá os despojos.
}\switchcolumn*\latim{
Si dormiátis inter médios cleros, pennæ colúmbæ deargentátæ, * et posterióra dorsi ejus in pallóre auri.
}\switchcolumn\portugues{
Se dormirdes entre vossos despojos, sereis como as penas prateadas da pomba, * e o brilho flavo do ouro na extremidade do seu dorso.
}\switchcolumn*\latim{
Dum discérnit cæléstis reges super eam, nive dealbabúntur in Selmon: * mons Dei, mons pinguis.
}\switchcolumn\portugues{
Enquanto o Altíssimo dispersa os reis sobre a terra, ficarão brancos com neve em Selmon: * o monte de Deus é monte farto.
}\switchcolumn*\latim{
Mons coagulátus, mons pinguis: * ut quid suspicámini montes coagulátos?
}\switchcolumn\portugues{
Monte escarpado, monte fecundo: * porém, porque pensais em outros montes escarpados?
}\switchcolumn*\latim{
Mons, in quo beneplácitum est Deo habitáre in eo: * étenim Dóminus habitábit in finem.
}\switchcolumn\portugues{
Um monte em que aprouve a Deus morar: * de facto, lá o Senhor habitará perpetuamente.
}\switchcolumn*\latim{
Currus Dei decem míllibus múltiplex, míllia lætántium: * Dóminus in eis in Sina in sancto.
}\switchcolumn\portugues{
O carro de Deus é assistido por dez milhares, milhares alegram-se: * o Senhor está entre eles em Sinai, no seu santuário.
}\switchcolumn*\latim{
Ascendísti in altum, cepísti captivitátem: * accepísti dona in homínibus.
}\switchcolumn\portugues{
Subistes ao alto, cativos levastes convosco: * pelos homens recebestes dons.
}\switchcolumn*\latim{
Étenim non credéntes, * inhabitáre Dóminum Deum.
}\switchcolumn\portugues{
Mesmo pelos descrentes, * habitava o Senhor Deus.
}\switchcolumn*\latim{
Benedíctus Dóminus die quotídie: * prósperum iter fáciet nobis Deus salutárium nostrórum.
}\switchcolumn\portugues{
Bendito seja o Senhor quotidianamente: * o Deus da nossa salvação fazer-nos-á a jornada próspera.
}\switchcolumn*\latim{
Deus noster, Deus salvos faciéndi: * et Dómini Dómini éxitus mortis.
}\switchcolumn\portugues{
Nosso Deus é o Deus que salva: * e ao Senhor, ao Senhor pertence livrar da morte.
}\switchcolumn*\latim{
Verúmtamen Deus confrínget cápita inimicórum suórum: * vérticem capílli perambulántium in delíctis suis.
}\switchcolumn\portugues{
Contudo, Deus quebrará as cabeças dos seus inimigos: * a moleira cabeluda dos que passeiam nos seus pecados.
}\switchcolumn*\latim{
Dixit Dóminus: ex Basan convértam, * convértam in profúndum maris:
}\switchcolumn\portugues{
O Senhor disse: de Basã os farei volver, * do fundo do mar volver os farei:
}\switchcolumn*\latim{
Ut intingátur pes tuus in sánguine: * lingua canum tuórum ex inimícis, ab ipso.
}\switchcolumn\portugues{
Para que o teu pé seja mergulhado no sangue: * de teus inimigos e também a língua de teus cães.
}\switchcolumn*\latim{
Vidérunt ingréssus tuos, Deus: * ingréssus Dei mei: regis mei qui est in sancto.
}\switchcolumn\portugues{
Eles viram a vossas procissões, ó Deus: * as procissões do meu Deus: do meu rei, que está no santuário.
}\switchcolumn*\latim{
Prævenérunt príncipes conjúncti psalléntibus: * in médio juvenculárum tympanistriárum.
}\switchcolumn\portugues{
Adiante foram os príncipes, juntamente com os cantores: * no meio das donzelas que tocavam timbales.
}\switchcolumn*\latim{
In ecclésiis benedícite Deo Dómino, * de fóntibus Israël.
}\switchcolumn\portugues{
Nas igrejas bendizei o Senhor Deus, * vós da estirpe de Israel.
}\switchcolumn*\latim{
Ibi Bénjamin adolescéntulus: * in mentis excéssu.
}\switchcolumn\portugues{
Ali estava o jovem Benjamim: * em êxtase mental.
}\switchcolumn*\latim{
Príncipes Juda, duces eórum: * príncipes Zábulon, príncipes Néphtali.
}\switchcolumn\portugues{
Os príncipes de Judá, seus comandantes: * os príncipes de Zabulon, os príncipes de Neftali.
}\switchcolumn*\latim{
Manda, Deus, virtúti tuæ: * confírma hoc, Deus, quod operátus es in nobis.
}\switchcolumn\portugues{
Ó Deus, mostrai o vosso poder: * confirmai, ó Deus, aquilo que fizestes entre nós.
}\switchcolumn*\latim{
A templo tuo in Jerúsalem, * tibi ófferent reges múnera.
}\switchcolumn\portugues{
Desde o vosso templo em Jerusalém, * os reis oferecer-Vos-ão dons.
}\switchcolumn*\latim{
Íncrepa feras arúndinis, congregátio taurórum in vaccis populórum: * ut exclúdant eos, qui probáti sunt argénto.
}\switchcolumn\portugues{
Reprimi essas feras dos canaviais, esses povos congregados como touros entre vacas: * para lançar fora os que foram provados como a prata.
}\switchcolumn*\latim{
Díssipa gentes, quæ bella volunt: vénient legáti ex Ægýpto: * Æthiópia prævéniet manus ejus Deo.
}\switchcolumn\portugues{
Dissipai as gentes que querem guerras: virão embaixadores do Egipto: * a Etiópia adiantar-se-á a estender as mãos para Deus.
}\switchcolumn*\latim{
Regna terræ, cantáte Deo: * psállite Dómino.
}\switchcolumn\portugues{
Reinos da terra, cantai a Deus: * salmodiai ao Senhor.
}\switchcolumn*\latim{
Psállite Deo, qui ascéndit super cælum cæli, * ad Oriéntem.
}\switchcolumn\portugues{
Salmodiai a Deus, que se eleva sobre todos os céus, * para Oriente.
}\switchcolumn*\latim{
Ecce dabit voci suæ vocem virtútis, date glóriam Deo super Israël, * magnificéntia ejus, et virtus ejus in núbibus.
}\switchcolumn\portugues{
Eis dará à sua voz força, dai glória a Deus pelo que fez em Israel, * a sua magnificência e o seu poder está nas nuvens.
}\switchcolumn*\latim{
Mirábilis Deus in sanctis suis, Deus Israël ipse dabit virtútem, et fortitúdinem plebi suæ, * benedíctus Deus.
}\switchcolumn\portugues{
Deus é admirável nos seus santos, o Deus de Israel, Ele mesmo dará poder e fortaleza ao seu povo, * bendito seja Deus.
}\end{paracol}

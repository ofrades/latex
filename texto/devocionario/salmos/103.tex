\begin{paracol}{2}\latim{
\rlettrine{B}{énedic,} ánima mea, Dómino: * Dómine, Deus meus, magnificátus es veheménter.
}\switchcolumn\portugues{
\rlettrine{B}{endiz} o Senhor, ó minha alma: * ó Senhor meu Deus, Vos engrandecestes sumamente.
}\switchcolumn*\latim{
Confessiónem, et decórem induísti: * amíctus lúmine sicut vestiménto:
}\switchcolumn\portugues{
Com glória e majestade Vos revestistes: * como um traje coberto de luz.
}\switchcolumn*\latim{
Exténdens cælum sicut pellem: * qui tegis aquis superióra ejus.
}\switchcolumn\portugues{
Como a tenda, estendeis o céu: * que cobris de água a sua cobertura.
}\switchcolumn*\latim{
Qui ponis nubem ascénsum tuum: * qui ámbulas super pennas ventórum.
}\switchcolumn\portugues{
Que subis sobre as nuvens: * e sobre as asas dos ventos andeis.
}\switchcolumn*\latim{
Qui facis ángelos tuos, spíritus: * et minístros tuos ignem uréntem.
}\switchcolumn\portugues{
Que fazeis os vossos anjos espíritos: * e que os vossos ministros sejam fogo ardente.
}\switchcolumn*\latim{
Qui fundásti terram super stabilitátem suam: * non inclinábitur in sǽculum sǽculi.
}\switchcolumn\portugues{
Que fundastes a terra sobre as suas bases: * ela se não desnivelará pelos séculos dos séculos.
}\switchcolumn*\latim{
Abýssus, sicut vestiméntum, amíctus ejus: * super montes stabunt aquæ.
}\switchcolumn\portugues{
O abysmo cinge-a como um traje: * as águas elevam-se acima das montanhas.
}\switchcolumn*\latim{
Ab increpatióne tua fúgient: * a voce tonítrui tui formidábunt.
}\switchcolumn\portugues{
À vossa ameaça fugirão: * temerão à voz de vosso trovão.
}\switchcolumn*\latim{
Ascéndunt montes: et descéndunt campi * in locum, quem fundásti eis.
}\switchcolumn\portugues{
Elevam-se as montanhas e os vales descem, * ao lugar que lhes estabelecestes.
}\switchcolumn*\latim{
Términum posuísti, quem non transgrediéntur: * neque converténtur operíre terram.
}\switchcolumn\portugues{
Instituístes-lhes limites, que não ultrapassarão: * e não volverão a cobrir a terra.
}\switchcolumn*\latim{
Qui emíttis fontes in convállibus: * inter médium móntium pertransíbunt aquæ.
}\switchcolumn\portugues{
Vós fazeis sair as fontes nos vales: * as águas passam por meio dos montes.
}\switchcolumn*\latim{
Potábunt omnes béstiæ agri: * exspectábunt ónagri in siti sua.
}\switchcolumn\portugues{
Todos os animais do campo beberão: * suspiram os asnos selvagens na sua sede.
}\switchcolumn*\latim{
Super ea vólucres cæli habitábunt: * de médio petrárum dabunt voces.
}\switchcolumn\portugues{
Sobre elas habitam as aves do céu: * do meio dos rochedos, farão ouvir as suas vozes.
}\switchcolumn*\latim{
Rigans montes de superióribus suis: * de fructu óperum tuórum satiábitur terra:
}\switchcolumn\portugues{
Regais os montes dos altos: * com o fruto de vossas obras a terra será saciada:
}\switchcolumn*\latim{
Prodúcens fænum juméntis, * et herbam servitúti hóminum:
}\switchcolumn\portugues{
Feno produzis para os animais, * e plantas para uso dos homens:
}\switchcolumn*\latim{
Ut edúcas panem de terra: * et vinum lætíficet cor hóminis:
}\switchcolumn\portugues{
Pão fazeis sair do seio da terra: * e vinho que alegra o coração do homem:
}\switchcolumn*\latim{
Ut exhílaret fáciem in óleo: * et panis cor hóminis confírmet.
}\switchcolumn\portugues{
Azeite para espalhar a alegria sobre o rosto: * e pão para fortificar o coração.
}\switchcolumn*\latim{
Saturabúntur ligna campi, et cedri Líbani, quas plantávit: * illic pásseres nidificábunt.
}\switchcolumn\portugues{
Encher-se-ão de seiva as árvores do campo e os cedros do Líbano que plantou: * ali farão ninhos as aves.
}\switchcolumn*\latim{
Heródii domus dux est eórum: * montes excélsi cervis: petra refúgium herináciis.
}\switchcolumn\portugues{
A casa da cegonha lhes serve de guia: * os montes altos são refúgio dos veados e os penhascos dos ouriços.
}\switchcolumn*\latim{
Fecit lunam in témpora: * sol cognóvit occásum suum.
}\switchcolumn\portugues{
Fez a lua para marcar os tempos: * o sol conhece o seu ocaso.
}\switchcolumn*\latim{
Posuísti ténebras, et facta est nox: * in ipsa pertransíbunt omnes béstiæ silvæ.
}\switchcolumn\portugues{
Espalhastes as trevas e a noite se fez: * vagueiam então todos os animais da selva.
}\switchcolumn*\latim{
Cátuli leónum rugiéntes, ut rápiant, * et quǽrant a Deo escam sibi.
}\switchcolumn\portugues{
Os leõezinhos rugem em busca da presa, * e pedem a Deus o seu sustento.
}\switchcolumn*\latim{
Ortus est sol, et congregáti sunt: * et in cubílibus suis collocabúntur.
}\switchcolumn\portugues{
Desponta o sol e reúnem-se: * e vão esconder-se nos seus covis.
}\switchcolumn*\latim{
Exíbit homo ad opus suum: * et ad operatiónem suam usque ad vésperum.
}\switchcolumn\portugues{
Sairá o homem para a sua obra: * e para os seus trabalhos até entardecer.
}\switchcolumn*\latim{
Quam magnificáta sunt ópera tua, Dómine! * ómnia in sapiéntia fecísti: impléta est terra possessióne tua.
}\switchcolumn\portugues{
Quão magníficas são as vossas obras, ó Senhor! * Fizestes com sabedoria todas as cousas: a terra está cheia das vossas riquezas.
}\switchcolumn*\latim{
Hoc mare magnum, et spatiósum mánibus: * illic reptília, quorum non est númerus.
}\switchcolumn\portugues{
Este mar grande e de longos braços: * nele existem peixes sem número.
}\switchcolumn*\latim{
Animália pusílla cum magnis: * illic naves pertransíbunt.
}\switchcolumn\portugues{
Animais pequenos e grandes: * por ele transitam os navios.
}\switchcolumn*\latim{
Draco iste, quem formásti ad illudéndum ei: * ómnia a Te exspéctant ut des illis escam in témpore.
}\switchcolumn\portugues{
Lá brinca esse dragão que formastes: * todos esperam de Vós que lhes deis de comer a seu tempo.
}\switchcolumn*\latim{
Dante Te illis, cólligent: * aperiénte Te manum tuam, ómnia implebúntur bonitáte.
}\switchcolumn\portugues{
Dando-lho Vós, eles o recolhem: * abrindo Vós vossa mão, todos se encherão de bens.
}\switchcolumn*\latim{
Averténte autem Te fáciem, turbabúntur: * áuferes spíritum eórum, et defícient, et in púlverem suum reverténtur.
}\switchcolumn\portugues{
Mas, se apartardes o vosso rosto, turvar-se-ão: * tirar-lhes-eis o espírito, deixarão de ser e ao pó retornarão.
}\switchcolumn*\latim{
Emíttes spíritum tuum, et creabúntur: * et renovábis fáciem terræ.
}\switchcolumn\portugues{
Enviareis o vosso espírito e serão criados: * e renovareis a face da terra.
}\switchcolumn*\latim{
Sit glória Dómini in sǽculum: * lætábitur Dóminus in opéribus suis:
}\switchcolumn\portugues{
Seja celebrada a glória do Senhor para sempre: * alegrar-se-á o Senhor nas suas obras:
}\switchcolumn*\latim{
Qui réspicit terram, et facit eam trémere: * qui tangit montes, et fúmigant.
}\switchcolumn\portugues{
Olha para a terra e tremer a faz: * toca os montes e eles fumegam.
}\switchcolumn*\latim{
Cantábo Dómino in vita mea: * psallam Deo meo, quámdiu sum.
}\switchcolumn\portugues{
Cantarei ao Senhor durante a minha vida: * cantarei hinos a meu Deus enquanto existir.
}\switchcolumn*\latim{
Jucúndum sit ei elóquium meum: * ego vero delectábor in Dómino.
}\switchcolumn\portugues{
Sejam-Lhe agradáveis as minhas palavras: * quanto a mim, deleitar-me-ei no Senhor.
}\switchcolumn*\latim{
Defíciant peccatóres a terra, et iníqui ita ut non sint: * bénedic, ánima mea, Dómino.
}\switchcolumn\portugues{
Desapareçam da terra os pecadores e os iníquos não mais existam: * bendiz o Senhor, ó minha alma.
}\end{paracol}

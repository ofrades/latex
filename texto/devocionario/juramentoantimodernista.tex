\section{Juramento Anti-Modernista}\label{juramentoantimodernista}

\begin{paracol}{2}\latim{
Ego {\redx N.} firmiter amplector ac recipio omnia et singula, quæ ab inerranti Ecclesiæ magisterio definita, adserta ac dedarata sunt, præsertim ea doctrinæ capita, quæ huius temporis erroribus directo adversantur.
}\switchcolumn\portugues{
Eu, {\redx N.}, firmemente aceito e creio em todas e em cada uma das verdades definidas, afirmadas e declaradas pelo magistério infalível da Igreja, sobretudo aqueles princípios doutrinais que contradizem directamente os erros do tempo presente.
}\switchcolumn*\latim{
Ac primum quidem: Deum, rerum omnium principium et finem, naturali rationis lumine per ea quæ facta sunt (Rom 1, 20), hoc est, per visibilia creationis opera, tamquam causam per effectus, certo cognosci, ideoque demonstrari etiam posse, profiteor.
}\switchcolumn\portugues{
Primeiro: creio que Deus, princípio e fim de todas as coisas, pode ser conhecido com certeza e pode também ser demonstrado, com as luzes da razão natural, nas obras por Ele realizadas (Cf. Rm I 20), isto é, nas criaturas visíveis, como (se conhece) a causa pelos seus efeitos.
}\switchcolumn*\latim{
Secundo: externa revelationis argumenta, hoc est facta divina, in primisque miracula et prophetias admitto et agnosco tamquam signa certissima divinitus ortæ Christianæ religionis, eademque teneo ætatum omnium atque hominum, etiam huius temporis, intellegentiæ esse maxime accommodata.
}\switchcolumn\portugues{
Segundo: admito e reconheço as provas exteriores da revelação, isto é, as intervenções divinas, e sobretudo os milagres e as profecias, como sinais certíssimos da origem sobrenatural da razão cristã, e as considero perfeitamente adequadas a todos os homens de todos os tempos, inclusive aquele no qual vivemos.
}\switchcolumn*\latim{
Tertio: firma pariter fide credo Ecclesiam, verbi revelati custodem et magistram, per ipsum verum atque historicum Christum, cum apud nos degeret, proxime ac directo institutam eamdemque super Petrum, apostolicæ hierarchiæ principem, ejusque in ævum successores ædificatam.
}\switchcolumn\portugues{
Terceiro: com a mesma firme fé creio que a Igreja, guardiã e mestra da palavra revelada, foi instituída imediatamente e directamente pelo próprio Cristo verdadeiro e histórico, enquanto vivia entre nós, e que foi edificada sobre Pedro, chefe da hierarquia eclesiástica, e sobre os seus sucessores através dos séculos.
}\switchcolumn*\latim{
Quarto: fidei doctrinam ab apostolis per orthodoxos patres eodem sensu eademque semper sententia ad nos usque transmissam, sincere recipio; ideoque prorsus reicio hæreticum commentum evolutionis dogmatum, ab uno in alium sensum transeuntium, diversum ab eo, quem prius habuit Ecclesia; pariterque damno errorem omnem quo divino deposito, Christi sponsæ tradito ab eaque fideliter custodiendo, sufficitur philosophicum inventum, vel creatio humanæ conscientiæ, hominum conatu sensim efformatæ et in posterum indefinito progressu perficiendæ.
}\switchcolumn\portugues{
Quarto: acolho sinceramente a doutrina da fé transmitida a nós pelos apóstolos através dos padres ortodoxos, sempre com o mesmo sentido e igual conteúdo, e rejeito totalmente a fantasiosa heresia da evolução dos dogmas de um significado a outro, diferente daquele que a Igreja professava primeiro; condeno de igual modo todo o erro que pretenda substituir o depósito divino confiado por Cristo à Igreja, para que o guardasse fielmente, por uma hipótese filosófica ou uma criação da consciência que se tivesse ido formando lentamente mediante esforços humanos e contínuo aperfeiçoamento, com um progresso indefinido.
}\switchcolumn*\latim{
Quinto: certissime teneo ac sincere profiteor, fidem non esse cæcum sensum religionis e latebris «subconscientiæ» erumpentem, sub pressione cordis et inflexionis voluntatis moraliter informatæ, sed verum assensum intellectus veritati extrinsecus acceptæ ex auditu, quo nempe, quæ a Deo personali, creatore ac Domino nostro dicta, testata et revelata sunt, vera esse credimus, propter Dei auctoritatem summe veracis.
}\switchcolumn\portugues{
Quinto: estou absolutamente convencido e sinceramente declaro que a fé não é um cego sentimento religioso que emerge da obscuridade do subconsciente por impulso do coração e inclinação da vontade moralmente educada, mas um verdadeiro assentimento do intelecto a uma verdade recebida de fora pela pregação, pelo qual, confiantes na sua autoridade supremamente veraz, nós cremos tudo aquilo que, pessoalmente, Deus, criador e senhor nosso, disse, atestou e revelou.
}\switchcolumn*\latim{
Me etiam, qua par est reverentia, subicio totoque animo adhæreo damnationibus, declarationibus, præscriptis omnibus, quæ in encyclicis litteris Pascendi et in decreto Lamentabili continentur, præsertim circa eam quam historiam dogmatum vocant.
}\switchcolumn\portugues{
Submeto-me também com o devido respeito, e de todo o coração adiro a todas as condenações, declarações e prescrições da encíclica Pascendi e do decreto Lamentabili, particularmente acerca da dita história dos dogmas.
}\switchcolumn*\latim{
Idem reprobo errorem affirmantium, propositam ab Ecclesia fidem posse historiæ repugnare, et catholica dogmata, quo sensu nunc intelleguntur, cum verioribus Christianæ religionis originibus componi non posse.
}\switchcolumn\portugues{
Reprovo outrossim o erro de quem sustenta que a fé proposta pela Igreja pode ser contrária à história, e que os dogmas católicos, no sentido que hoje lhes é atribuído, são inconciliáveis com as reais origens da razão cristã.
}\switchcolumn*\latim{
Damno quoque ac reicio eorum sententiam, qui dicunt Christianum hominem eruditiorem induere personam duplicem, aliam credentis, aliam historici, quasi liceret historico ea retinere, quæ credentis fidei contradicant, aut præmissas adstruere, ex quibus consequatur, dogmata esse aut falsa aut dubia, modo hæc directo non denegentur.
}\switchcolumn\portugues{
Desaprovo também e rejeito a opinião de quem pensa que o homem cristão mais instruído se reveste da dupla personalidade do crente e do histórico, como se ao histórico fosse lícito defender teses que contradizem a fé o crente ou fixar premissas das quais se conclui que os dogmas são falsos ou dúbios, desde que não sejam positivamente negados.
}\switchcolumn*\latim{
Reprobo pariter eam Scripturæ sanctæ diiudicandæ atque interpretandæ rationem, quæ, Ecclesiæ traditione, analogia fidei et apostolicæ Sedis normis posthabitis, rationalistarum commentis inhæret, et criticam textus velut unicam supremamque regulam haud minus licenter quam temere amplectitur.
}\switchcolumn\portugues{
Condeno igualmente aquele sistema de julgar e de interpretar a sagrada Escritura que, desdenhando a tradição da Igreja, a analogia da fé e as normas da Sé apostólica, recorre ao método dos racionalistas e com desenvoltura não menos que audácia, aplica a crítica textual como regra única e suprema.
}\switchcolumn*\latim{
Sententiam præterea illorum reiicio, qui tenent, doctori disciplinæ historicæ theologicæ tradendæ aut iis de rebus scribenti seponendam prius esse opinionem ante conceptam sive de supernaturali origine catholicæ traditionis, sive de promissa divinitus ope ad perennem conservationem uniuscuiusque revelati veri; deinde scripta patrum singulorum interpretanda solis scientiæ principiis, sacra qualibet auctoritate seclusa eaque iudicii libertate, qua profana quævis monumenta solent investigari.
}\switchcolumn\portugues{
Refuto ainda a sentença de quem sustenta que o ensinamento de disciplinas histórico-teológicas ou quem delas trata por escrito deve inicialmente prescindir de qualquer ideia pré-concebida, seja quanto à origem sobrenatural da tradição católica, seja quanto à ajuda prometida por Deus para a perene salvaguarda de cada uma das verdades reveladas, e então interpretar os textos patrísticos somente sobre as bases científicas, expulsando toda autoridade religiosa, e com a mesma autonomia crítica admitida para o exame de qualquer outro documento profano.
}\switchcolumn*\latim{
In universum denique me alienissimum ab errore profiteor, quo modernistæ tenent in sacra traditione nihil inesse divini, aut, quad longe deterius, pantheistico sensu illud admittunt, ita ut nihil iam restet nisi nudum factum et simplex, communibus historice factis æquandum: hominum nempe sua industria, solertia, ingenio scholam a Christo ejusque apostolis inchoatam per subsequentes ætates continuantium.
}\switchcolumn\portugues{
Declaro-me enfim totalmente alheio a todos os erros dos modernistas, segundo os quais na sagrada tradição não há nada de divino ou, pior ainda, admitem-no, mas em sentido panteísta, reduzindo-o a um evento pura e simplesmente análogo àqueles ocorridos na história, pelos quais os homens com o próprio empenho, habilidade e engenho prolongam nas eras posteriores a escola inaugurada por Cristo e pelos apóstolos.
}\switchcolumn*\latim{
Proinde fidem patrum firmissime retineo et ad extremum vitæ spiritum retinebo, de charismate veritatis certo, quad est, fuit eritque semper in episcopatus ab apostolis successione, non ut id teneatur, quod melius et aptius videri possit secundum suam cuiusque ætatis culturam, sed ut numquam aliter credatur, numquam aliter intellegatur absoluta et immutabilis veritas ab initio per apostolos prædicata.
}\switchcolumn\portugues{
Mantenho, portanto, e até o último suspiro manterei a fé dos pais no carisma certo da verdade, que esteve, está e sempre estará na sucessão do episcopado aos apóstolos, não para que se assuma aquilo que pareça melhor e mais consoante à cultura própria e particular de cada época, mas para que a verdade absoluta e imutável, pregada no princípio pelos apóstolos, não seja jamais crida de modo diferente nem entendida de outro modo.
}\switchcolumn*\latim{
Hæc omnia spondeo me fideliter, integre sincereque servaturum et inviolabiliter custoditurum, nusquam ab us sive in docendo sive quomodolibet verbis scriptisque deflectendo. Sic spondeo, sic iuro, sic me Deus adiuvet, et hæc sancta Dei Evangelia.
}\switchcolumn\portugues{
Empenho-me em observar tudo isto fielmente, integralmente e sinceramente, e em guardá-lo inviolavelmente, sem jamais disso me separar nem no ensinamento nem em género algum de discursos ou de escritos. Assim prometo, assim juro, assim me ajudem Deus e esses santos Evangelhos de Deus.
}\end{paracol}
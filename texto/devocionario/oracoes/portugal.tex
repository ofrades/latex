\section{Portugal}

\subsection{Anjo Custódio de Portugal}
\slettrine{Ó}{} Deus omnipotente e sempiterno, que com inefável providência destinais para cada nação um Anjo, que a guarde, concedei-nos, Vos suplicamos, que, pelas súplicas e pelo patrocínio do Anjo Custódio da nossa Nação, sejamos sempre livres de todas as adversidades. Amen.

\subsection{Os Nossos Santos}

\subsubsection{Santo Francisco e Santa Jacinta}
\begin{paracol}{2}\latim{
\rlettrine{D}{eus,} qui innocentiam diligis et humiles exaltas, da nobis, ut exemplo beatorum Francisci et Hyacinthæ, puritate cordis tibi servientes, in regnum cælorum introire mereamur. Amen.
}\switchcolumn\portugues{
\rlettrine{D}{eus} de infinita bondade, que amais a inocência e exaltais os humildes, concedei, pela intercessão da Imaculada Mãe do vosso Filho, que, à imitação dos bem-aventurados Francisco e Jacinta, Vos sirvamos em pureza de coração, para podermos entrar no reino dos Céus. Amen.
}\end{paracol}

\subsubsection{Oração a Santa Beatriz da Silva}
\begin{paracol}{2}\latim{
\rlettrine{D}{eus,} qui ad cultum Imaculatæ Virginis promovendum, beatam Beatricem Virginem singulari castitatis prærogativa decorasti, et per eam novam in Ecclesia tua familiam suscitasti; da nobis ejus intercessione et exemplo ita innocenter vivere, ut terrenis omni bus abdicatis, gaudiis perfrui mereamur æternis. Per Dominum nostrum. Amen.
}\switchcolumn\portugues{
\slettrine{Ó}{} Deus, que para promover o culto da Virgem Imaculada, adornastes a B. Beatriz com a singular prerrogativa da castidade, e fundastes por meio dela uma Ordem na vossa Igreja, concedei-nos que por sua intercessão e exemplo, vivamos tão santamente que, desprezando as coisas da terra, mereçamos desfrutar os gozos eternos. Por Nosso Senhor Jesus Cristo. Amen.
}\end{paracol}

\subsection{Santo António}

\subsubsection{Objectos Perdidos}
\blettrine{E}{u,} vos saúdo, glorioso Santo António, fiel protector dos que em vós esperam. Já que recebestes de Deus o poder especial de fazer achar os objectos perdidos, socorrei-me neste momento, a fim de que, mediante vosso auxílio, eu encontre o objecto que procuro...\par
Alcançai-me, sobretudo, uma fé viva, uma esperança firme, uma caridade ardente e uma docilidade sempre pronta aos desejos de Deus. Que eu me não detenha apenas nas coisas deste mundo. Saiba valorizá-las e utilizá-las como algo que nos foi emprestado e lute sobretudo por aquelas coisas que ladrão nenhum pode nos arrebatar e nem iremos perder jamais. Amen.

\subsubsection{Responsório de Santo António}\label{respstantonio}

\begin{paracol}{2}\latim{
\rlettrine{S}{i} quæris mirácula, mors, error, calámitas, dæmon, lepra fúgiunt, ægri surgunt sani.
}\switchcolumn\portugues{
\rlettrine{S}{e} milagres procurais, a morte, o erro, a calamidade, o demónio, e a lepra fogem, os enfermos saudáveis se levantam.
}\switchcolumn*\latim{
\emph{Ant.} Cedunt mare, víncula: membra, resque pérditas, pétunt et accípiunt juvénes et cani.
}\switchcolumn\portugues{
\emph{Ant.} Cede o mar embravecido, recupera-se o perdido, pedem e recebem, tanto velhos como mancebos.
}\switchcolumn*\latim{
Péreunt perícula, cessat et necéssitas, narrent hi qui séntiunt, dicant Paduáni.
}\switchcolumn\portugues{
Desaparecem os perigos e cessa a indigência, digam-no aqueles que o sentiram, e digam-no os Paduanos.
}\switchcolumn*\latim{
\emph{Ant.} Cedunt mare, víncula: membra, resque pérditas, pétunt et accípiunt juvénes et cani.
}\switchcolumn\portugues{
\emph{Ant.} Cede o mar embravecido, recupera-se o perdido, pedem e recebem, tanto velhos como mancebos.
}\switchcolumn*\latim{
Glória Patri et Fílio et Spirítui Sancto.
}\switchcolumn\portugues{
Glória ao Pai, e ao Filho e ao Espírito Santo.
}\switchcolumn*\latim{
\emph{Ant.} Cedunt mare, víncula: membra, resque pérditas, pétunt et accípiunt juvénes et cani.
}\switchcolumn\portugues{
\emph{Ant.} Cede o mar embravecido, recupera-se o perdido, pedem e recebem, tanto velhos como mancebos.
}\switchcolumn*\latim{
℣. Ora pro nobis, beate Antoni.
}\switchcolumn\portugues{
℣. Rogai por nós, bem-aventurado António.
}\switchcolumn*\latim{
℟. Ut digni efficiamur promissionibus Christi.
}\switchcolumn\portugues{
℟. Para que sejamos dignos das promessas de Cristo.
}\switchcolumn*\latim{
\begin{nscenter} Orémus. \end{nscenter}
}\switchcolumn\portugues{
\begin{nscenter} Oremos. \end{nscenter}
}\switchcolumn*\latim{
\rlettrine{E}{cclesiam} tuam, Deus, beati Antonii Confessoris tui atque Doctoris solemnitas votiva lætificet, ut spiritualibus semper muniatur auxiliis, et gáudiis perfrui mereatur æternis. Per Christum Dóminum nostrum. ℟. Amen.
}\switchcolumn\portugues{
\slettrine{Ó}{} Deus, nós Vos suplicamos, que alegre à vossa Igreja a solenidade votiva do bem-aventurado Santo António, vosso Confessor e Doutor, para que, fortalecida sempre com os espirituais auxílios, mereça gozar os prazeres eternos. Por Jesus Cristo, Nosso Senhor. ℟. Amen.
}\end{paracol}

\subsection{Portuguesa Nação Nossa}

\subsubsection{Oração a Nossa Senhora de Fátima}
\blettrine{V}{irgem} Imaculada, que pelo vosso Santo Rosário extinguistes outrora no seio da Igreja a nefasta heresia dos albigenses, por ele libertastes a cristandade do perigo muçulmano e robustecestes a piedade dos fiéis, extingui também no povo português, pela prática mais intensa da vossa devoção, os gérmenes de morte que fazem definhar a sua Fé, libertai-o de todos os perigos internos e externos que ameaçam a pureza dos seus costumes, fortalecei-o mais e mais, fazendo rejuvenescer nele o genuíno espírito de piedade que no Passado o fez um povo cristianíssimo, fidelíssimo e evangelizador.\par
E já que, por uma inefável prova de celestial predilecção, Vos dignastes visitar este povo que se ufana de ser vassalo vosso, mostrando-lhe dos montes da Fátima quão caro é ao vosso Coração, não deixeis nunca, Mãe amorosíssima, de o acalentar com esse mesmo amor de predilecção. Descansai sobre ele olhares de misericórdia, fazei-lhe sentir mais e mais a vossa suavíssima protecção e os doces atractivos do vosso Coração, que é coração de mãe.\par
Abençoai, ó Virgem Imaculada, a terra que Vos dignastes visitar, atraí a Vós todos os portugueses, patenteai-lhes os tesouros do vosso amor, revelai-lhes os arcanos do vosso Coração materno, fazei de cada coração português um órgão que vibre de amor por Vós e de Portugal inteiro um santuário de amor que corresponda com seu filial afecto ao vosso carinho maternal; e assim mereça agora e sempre ser chamado a Terra de Santa Maria. Amen.

\subsubsection{Exército de Almas}
\rlettrine{M}{ajestade} Divina, Senhor da vida e da morte, dos que Vos amam e dos que Vos perseguem! Por intercessão da Santíssima Virgem de Fátima, Rainha da Paz e nossa Mãe, Vos pedimos que não deixeis a nossa Pátria onde Maria ergueu seu trono, venha a ser dominada e destruída por obra dos vossos inimigos. Enviai os vossos Santos Anjos a todos os locais da nossa terra e permiti que eles possam desenvolver as suas potências em todos seus recantos, para que o inimigo não venha a triunfar na nossa Pátria. Nós queremos formar um exército de almas que rezam para que Vós, Deus Uno e Trino, estendais a vossa Mão poderosa sobre este povo que é de Maria vossa Mãe. Permiti, ó Deus, que as nuvens tempestuosas que pairam sobre a humanidade e tendem a espalhar-se e a submergir a nossa Pátria, sejam afastadas. Só Vós podeis salvar-nos! Pela vossa graça e especial protecção da nossa Padroeira Maria Imaculada e do Anjo Custódio de Portugal, permiti, ó Deus, que a nossa terra nunca seja aniquilada pelo inimigo. Deus Santo, Deus Forte, Deus Todo-Poderoso, Deus Imortal, em união com todos os Santos Anjos, pedimo-Vos auxílio e Bênção para a nossa Pátria, por Jesus Cristo Nosso Senhor. Amen.

\subsubsection{Rocha sobre a qual a nossa nação se fundou}
\rlettrine{S}{enhor} Pai santo, confiamos o nosso Portugal à vossa misericórdia e protecção. Vós sois a rocha sobre a qual a nossa nação se fundou. Só Vós sois a fonte da verdade e do amor. Reclamai esta terra para a vossa glória e habitai no meio do vosso povo. Enviai o vosso Espírito e tocai os corações dos líderes da nossa nação. Abri os seus corações ao grande valor da vida humana e às responsabilidades que acompanham a liberdade humana. Relembrai o vosso povo que a verdadeira felicidade está enraizada na procura e no cumprimento da vossa vontade. Por intercessão de Maria Imaculada. Padroeira da nossa terra, concedei-nos a coragem de levar o Evangelho do vosso Filho Jesus a todos aqueles com quem convivemos e de o testemunhar com uma vida santa. Por Cristo Nosso Senhor. Amen.

\subsection{Oração de Desagravo}
\slettrine{Ó}{} Cruz adorável do meu amantíssimo Jesus!... Como vós sois bela!... Como vós faleis ao meu pobre coração!...\par
De vós pendeu o meu Deus feito homem por meu amor!... Pregado em vós deu-me até à última gota o sangue preciosíssimo do seu coração!... Em vós morreu, e como morreu!..., o meu Jesus!...\par
E desde então, ó Cruz bendita, vós ficastes sendo a nossa maior glória, ficastes sendo a nossa vida, o nosso amparo e consolação.\par
Feliz o homem, que nas agruras deste exilio se deixa guiar por vós! Felizes as nações, que vos tomam por norte, ó Cruz do meu Senhor!...\par
E, no entanto, há quem vos esqueça, há quem vos ultraje e quem vos ofenda, e, nos dias tristes que vamos atravessando, há até quem vos derrube, quem vos calque aos pés!...\par
E nem sequer a estes insultos vos poupam neste vosso querido Portugal!...\par
Neste Portugal, que vós abençoastes no berço com tanto amor; neste Portugal, que dominou os mares, porque vós lhe guiastes as caravelas; neste Portugal, que assombrou o mundo com seus feitos, porque vós, ó Cruz, lhe destes como a ninguém heroicidade e valor!...\par
E eu que sou português, e que vos amo, não hei-de cair a vossos pés para reparar tantas ingratidões?!...\par
Perdoai-lhes, oh!, perdoai-lhes e iluminai-os para que vejam o abismo que com vosso desprezo estão cavando, vejam o terrível futuro que sem vós estão preparando para este pobre Portugal!...\par
Perdoai-lhes e que todos vos amem, ó Cruz da minha alma, ó Cruz do meu Jesus!...\par
E, se para ser ouvido é necessário um sacrifício, eu, aqui estou por vós pronto para tudo, contando que me salveis e salveis Portugal. Amen.

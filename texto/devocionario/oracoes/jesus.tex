\ilustra{media/jesus1}

\section{Jesus}

\subsection{Anima Christi}\label{animachristi}
\begin{paracol}{2}\latim{
\rlettrine{A}{nima} Christi, sanctífica me.\\
Corpus Christi, salve me.
}\switchcolumn\portugues{
\rlettrine{A}{lma} de Cristo, santificai-me.\\
Corpo de Cristo, salvai-me.
}\switchcolumn*\latim{
Sanguis Christi, inébria me.
}\switchcolumn\portugues{
Sangue de Cristo, inebriai-me.
}\switchcolumn*\latim{
Aqua láteris Christi, lava me.
}\switchcolumn\portugues{
Água do lado de Cristo, lavai-me.
}\switchcolumn*\latim{
Pássio Christi, conforta me.
}\switchcolumn\portugues{
Paixão de Cristo, confortai-me.
}\switchcolumn*\latim{
O bone Jesu, exáudi me.
}\switchcolumn\portugues{
Ó bom Jesus, ouvi-me.
}\switchcolumn*\latim{
Intra tua vúlnera abscónde me.
}\switchcolumn\portugues{
Dentro das vossas Chagas, escondei-me.
}\switchcolumn*\latim{
Ne permittas me separári a te.
}\switchcolumn\portugues{
Não permitais que de Vós me separe.
}\switchcolumn*\latim{
Ab hoste maligno defénde me.
}\switchcolumn\portugues{
Do espírito maligno, defendei-me.
}\switchcolumn*\latim{
In hora mortis meæ voca me.
}\switchcolumn\portugues{
Na hora da minha morte, chamai-me.
}\switchcolumn*\latim{
Et jube me venire ad te, ut cum Sanctis tuis laudem te in sǽcula
sæculórum.
}\switchcolumn\portugues{
E mandai-me ir para Vós, para que Vos louve com vossos Santos, por
todos os séculos.
}\switchcolumn*\latim{
Amen.
}\switchcolumn\portugues{
Amen.
}\end{paracol}

\subsection{Jesus misericordioso, tende compaixão de mim}
\qlettrine{J}{esus} Cristo, nosso Senhor, Deus de bondade e misericórdia, aqui me tendes em vossa presença, humilhado e contrito de coração; recomendo-Vos a minha hora derradeira, e a sorte que depois dela me espera.\par
Quando os meus pés imóveis me avisarem de que meu caminho neste mundo está prestes a terminar: \textit{Jesus misericordioso, tende compaixão de mim.}\par
Quando as minhas mãos, trémulas e entorpecidas, não puderem já apertar o Crucifixo e, contra a minha vontade, o deixarem cair sobre o meu leito de dor: \textit{Jesus misericordioso, tende compaixão de mim.}\par
Quando os meus olhos, apagados e amortecidos, horrorizados à vista da morte iminente, cravarem em vossa imagem seus olhares abatidos e moribundos: \textit{Jesus misericordioso, tende compaixão de mim.}\par
Quando os meus lábios, frios e trémulos, pronunciarem pela derradeira vez o vosso adorável nome: \textit{Jesus misericordioso, tende compaixão de mim.}\par
Quando o meu rosto, pálido e arroxeado, já mover à compaixão e ao susto as pessoas presentes, e os meus cabelos parados, banhados do suor da morte, derem o sinal de que se apressa o termo dos meus dias: \textit{Jesus misericordioso, tende compaixão de mim.}\par
Quando os meus ouvidos, prestes a fecharem-se para sempre às conversações dos homens, se abrirem para ouvir de vossa boca a irrevogável sentença, que decidirá a minha sorte por toda a eternidade: \textit{Jesus misericordioso, tende compaixão de mim.}\par
Quando a minha imaginação, perturbada por fantasmas horrendos e aterradores, cair em mortal angústia, e meu espírito, abalado e confuso à vista das próprias iniquidades e receoso da vossa justiça, lutar contra o anjo das trevas, que há-de querer tirar-me a esperança na vossa misericórdia e precipitar-me no abysmo do desespero: \textit{Jesus misericordioso, tende compaixão de mim.}\par
Quando o meu coração, fraco e angustiado com as dores da doença, for surpreendido pelos horrores da morte e se achar exausto e cansado com os esforços feitos para triunfar dos inimigos da minha salvação: \textit{Jesus misericordioso, tende compaixão de mim.}\par
Quando deitar as últimas lágrimas, prenúncio da minha destruição, recebei-as, ó meu Jesus, em sacrifício de expiação, para que assim morra vítima de penitência; e naquele terrível momento: \textit{Jesus misericordioso, tende compaixão de mim.}\par
Quando os parentes e amigos, apinhados ao redor de mim, se enternecerem à vista do meu lastimoso estado, e invocarem vossa misericórdia em meu favor: \textit{Jesus misericordioso, tende compaixão de mim.}\par
Quando, perdido o uso dos sentidos e apagada de toda minha vista, gemer no meio das ânsias da agonia extrema e na crise da morte: \textit{Jesus misericordioso, tende compaixão de mim.}\par
Quando os últimos impulsos do meu coração obrigarem a minha alma a sair do corpo, recebei-os como prova de um vivo anseio de ir ter convosco; e Vós: \textit{Jesus misericordioso, tende compaixão de mim.}\par
Quando a minha alma sair para sempre deste mundo, e deixar o meu corpo pálido, frio e sem vida, aceitai a destruição do meu ser como uma homenagem que desde há ofereço à vossa Divina Majestade, e naquela hora: \textit{Jesus misericordioso, tende compaixão de mim.}\par
Quando, finalmente, a minha comparecer diante de Vós e contemplar pela primeira vez o imortal esplendor da vossa Majestade, as não expulseis de vossa presença: mas dignai-Vos receber-me no seio amoroso da vossa misericórdia, para que possa cantar eternamente os vossos louvores: \textit{Jesus misericordioso, tende compaixão de mim.}\par

\subsection{Acto de Reparação}
\rlettrine{C}{om} aquele profundíssimo respeito que a Fé me inspira, ó meu Deus e meu Salvador, Jesus Cristo, verdadeiro Deus e Homem, eu Vos adoro e amo com todo o coração no Augustíssimo Sacramento do Altar, em reparação de todas as irreverências, profanações e sacrilégios que por minha desgraça tenha cometido até agora, assim como de todos os que no passado se têm feito ou possam (tal não permita Deus) fazer-se no futuro. Adoro-Vos, pois, ó meu Deus, não só pelo muito que sois digno de ser amado e adorado, mas, ao menos, conforme o que posso; e quisera poder fazê-lo com aquela perfeição de que são capazes todas as criaturas racionais. Deste modo, tenho intenção de Vos adorar agora e sempre, não só por aqueles Católicos que Vos não adoram nem amam, mas ainda em compensação da adoração que Vos devem os infiéis, os hereges, os cismáticos, os ímpios, os blasfemos, os profanadores, os idólatras, os judeus, os maometanos e todos os outros que Vos injuriam e perseguem, e pela conversão de todos eles. Ah! Sim, meu Jesus, permiti que todos Vos conheçam, adorem e amem, e Vos dêem graças a todo o momento no Santíssimo e diviníssimo Sacramento. Amen.

\subsection{Concede Mihi}
\begin{paracol}{2}\latim{
\rlettrine{C}{oncede} mihi, benignissime Jesu, grátiam tuam, ut mecum sit et mecum laboret, mecum que in finem usque persevéret.
}\switchcolumn\portugues{
\rlettrine{I}{nfinitamente} bom Jesus, eu Vos peço que me concedeis a vossa graça; fazei que ela permaneça em mim, trabalhe comigo e se mantenha comigo até ao fim.
}\switchcolumn*\latim{
Da mihi hoc semper desiderare et velle, quod tibi magis acceptum est et carius placet.
}\switchcolumn\portugues{
Concedei-me sempre a vontade e o desejo daquilo que for mais agradável e mais aceitável para Vós.
}\switchcolumn*\latim{
Tua voluntas mea sit, et mea voluntas tuam semper sequatur et optime ei concordet.
}\switchcolumn\portugues{
Que a vossa vontade seja a minha, e que minha vontade esteja sempre em conformidade com vossa.
}\switchcolumn*\latim{
Sit mihi unum velle et nolle tecum, nec aliud posse velle aut nolle, nisi quod Tu vis et nolis. Amen.
}\switchcolumn\portugues{
Fazei que tudo aquilo que eu queira ou não queira seja aquilo que Vós quereis ou não quereis. Amen.
}\end{paracol}

\subsection{Oração a Nosso Senhor dos Passos}
\slettrine{Ó}{} Jesus, Filho Unigénito de Deus e da Virgem Imaculada, que pela salvação do mundo quisestes ser condenado, traído, atado a uma coluna, conduzido como um cordeiro ao matadouro, acusado injustamente num tribunal, ferido com pancadas, saturado de opróbrios e injúrias, cuspido no rosto, barbaramente açoitado, coroado de espinhos, condenado à morte, despojado das vestes, pregado numa cruz com toda a crueldade, suspenso entre dous ladrões, vexado com fel e vinagre, abandonado em tormentosa agonia e finalmente trespassado por uma lança: por estes tormentos, Senhor, dos quais nós, indignos filhos vosso, agora com devoção, gratidão e amor nos lembramos, e pela vossa santíssima morte na cruz, livrai-nos das penas eternas do inferno e dignai-Vos conduzir-nos ao paraíso, para onde levastes convosco o bom ladrão.\par
Tende piedade de nós, Senhor, que com o Pai e o Espírito Santo viveis e reinais pelos séculos dos séculos. Amen.

\subsection{Pureza}
\rlettrine{D}{ulcíssimo} Menino Jesus, Cordeiro imaculado, cheio de bondade, misericórdia e amor! Para nos restituirdes a santa inocência, vieste do céu à terra, sofrestes pobreza e perseguições. Eu Vos agradeço e Vos amo de todo meu coração. E por vosso amor proponho hoje firmemente guardar com todo o cuidado a santa pureza do coração. Ó meu Jesus, abençoa o meu corpo, para que seja sempre um santuário de inocência e pureza. Fazei que eu evite com cuidado todo o pecado moral, tal como a uma peste contagiosa. Ó Jesus inocentíssimo e todo imaculado, pelo vosso amor e pela vossa inocência concedei-me a virtude da santa pureza, para que eu, depois da minha morte, tenha a felicidade de ver Vos no céu.

\subsection{Consagração Pessoal a Jesus Cristo}
\emph{Nos 30 dias anteriores à consagração devem-se rezar Ladainhas, o Veni Creator Spíritus e a Ave Maris Stella, deve-se ler o Santo Evangelho e a Imitação de Cristo, assim como rezar o terço.}\par

\slettrine{Ó}{} sabedoria eterna e encarnada! Ó Amabilíssimo e adorável Jesus, verdadeiro Deus e verdadeiro homem, Filho Unigénito do Pai Eterno e da sempre Virgem Maria.
Adoro-Vos profundamente, no seio e nos esplendores do vosso Pai, durante toda a eternidade, e no seio virginal de Maria, vossa Mãe digníssima, no tempo da vossa Encarnação.
Dou-Vos graças por Vos terdes aniquilado a Vós mesmo, tomando a forma de escravo, para livrar-me da cruel escravidão do demónio.
Eu Vos louvo e glorifico por Vos terdes querido submeter em tudo a Maria, vossa Mãe Santíssima, a fim de, por Ela, tornar-me vosso fiel escravo.
Entretanto, ai de mim, criatura ingrata e infiel! Não guardei os votos e promessas que tão solenemente Vos fiz no meu Baptismo. Não cumpri as minhas obrigações; não mereço ser chamado vosso filho, nem vosso escravo; e, como nada há em mim que não mereça a vossa repulsa e a vossa cólera, não ouso aproximar-me por mim mesmo da vossa Santíssima e Augustíssima Majestade.
Recorro, pois, à intercessão e à misericórdia de vossa Mãe Santíssima, que me destes por medianeira junto de Vós. É por intermédio d’Ela que espero obter de Vós a contrição e o perdão dos meus pecados, a aquisição e conservação da Sabedoria.
Ave, pois, ó Maria Imaculada, Tabernáculo Vivo da Divindade, onde a Eterna Sabedoria escondida quer ser adorada pelos anjos e pelos homens.
Ave, ó Rainha do Céu e da Terra, a cujo Império é submetido tudo o que há abaixo de Deus.
Ave, ó Seguro Refúgio dos pecadores, cuja misericórdia a ninguém despreza. Atendei ao desejo que tenho da Divina Sabedoria, e recebei, para isso, os votos e ofertas apresentados pela minha baixeza.

Eu, {\redx N.}, infiel pecador, renovo e ratifico hoje, nas vossas mãos, as promessas do meu Baptismo: renuncio para sempre a Satanás, às suas pompas e suas obras, e dou-me inteiramente a Jesus Cristo, a Sabedoria Encarnada, para o seguir, levando a minha Cruz, todos os dias da minha vida. E para lhe ser mais fiel do que até agora tenho sido, escolho-Vos hoje, ó Maria, na presença de toda a Corte Celeste, por minha Mãe e Senhora. Entrego-Vos e consagro-Vos, na qualidade de escravo, o meu corpo e a minha alma, os meus bens interiores e exteriores, e o próprio valor das minhas boas obras passadas, presentes e futuras, deixando-Vos pleno e inteiro direito de dispor de mim e de tudo o que me pertence, sem excepção alguma, segundo o vosso agrado e para maior glória de Deus, no tempo e na eternidade.
Recebei, ó Benigníssima Virgem, esta pequenina oferta da minha escravidão, em união e em honra à submissão que a Sabedoria Eterna quis ter à vossa Maternidade; em homenagem ao poder que ambos tendes sobre este vermezinho e miserável pecador; em acção de graças pelos privilégios com que largamente Vos favoreceu a Trindade Santíssima.
Protesto que quero, de hoje em diante e firmemente, como vosso verdadeiro escravo, buscar a vossa honra e obedecer-Vos em todas as cousas.
Ó Mãe Admirável, apresentai-me ao vosso amado Filho na condição de escravo perpétuo, a fim de que, tendo-me resgatado por Vós, por Vós também me receba propiciamente.
Ó Mãe de Misericórdia, concedei-me a graça de obter a Verdadeira Sabedoria de Deus, e de colocar-me, para isso, entre o número daqueles que amais, ensinais, guiais, sustentais e protegeis como filhos e escravos vossos.
Ó Virgem Fiel, tornai-me em tudo um tão perfeito discípulo, imitador e escravo da Sabedoria Encarnada, Jesus Cristo, vosso Filho, que eu chegue um dia, por vossa intercessão e a vosso exemplo, à plenitude da sua idade na Terra e da sua glória no Céu. Amen.

\subsection{Consagração ao Sagrado Coração de Jesus}
\slettrine{Ó}{} Dulcíssimo Jesus, ó Redentor do género humano, lançai um olhar sobre nós, humildemente prostrados diante do vosso Altar! Somos vossos e vossos queremos ser; e para podermos viver mais estreitamente unidos a Vós, eis que cada um de nós se consagra ao vosso Sacratíssimo Coração. Muitos, porém, já vos não conhecem; muitos, ao desprezar os vossos Mandamentos, repudiam-Vos. Ó Benigníssimo Jesus, tende piedade de uns e de outros; e atraí todos ao vosso Coração Santíssimo.\par
Ó Senhor, sede o Rei não só dos fiéis que se não distanciaram de Vós, mas também destes filhos pródigos que Vos abandonaram; fazei com que estes retornem à Casa Paterna o quanto antes para não morrerem de miséria e fome. Sede o Rei de todos os que vivem no engano do erro ou que por discordarem de Vós se separaram; chamai-os ao Porto da Verdade e da Unidade da Fé para que assim, em breve, mais não haja que um só rebanho sob um só Pastor.\par
Sede finalmente o Rei de todos os que estão envoltos nas superstições do paganismo e não recuseis tirá-los das trevas para traze-los à Luz do Reino de Deus.\par
Obtende, ó Senhor, a integridade e liberdade segura para a vossa Igreja; dai a todo o povo a tranquilidade da ordem; fazei com que de uma extremidade à outra da Terra ressoe esta única voz:\par
℣. Seja louvado este Coração do qual provém a nossa salvação!\par
℟. A Ele a Honra e a Glória por todos os séculos. Amen.

\subsection{Coroinha do Sagrado Coração de Jesus}

\rlettrine{I}{} Amorosíssimo Jesus, quando medito no vosso Santíssimo Coração e O vejo todo piedade e bondade para com os pecadores, sinto o meu coração encher-se de alegria e de confiança de que será por Vós bem acolhido. Ai de mim, quantos pecados tenho cometido!... Mas, agora, penetrado de contrição, choro-os e detesto-os, como Pedro e Madalena, por serem ofensas a Vós, ó Sumo Bem. Eu Vo-lo suplico encarecidamente, pelo vosso Santíssimo Coração. Oxalá eu antes morra do que Vos ofenda! Que eu não viva senão para Vos Amar!

\rlettrine{II}{} Bendigo o vosso humilíssimo Coração, ó meu bom Jesus, e Vos dou graças, porque ao dardes-m’O como exemplo, não só com veementes desejos me incitastes a imitá-l’O, senão que, à custa de tantas humilhações vossas, me proporcionastes e aplanastes o caminho. Como tenho sido insensato e ingrato!... Quanto me tenho extraviado!... Perdoai-me. Não mais quero ser soberbo e ambicioso; quero somente seguir-Vos com o coração humilde entre as humilhações, e alcançar a paz e a salvação. Dai-me Vós a graça para isto, e bendizei sempre o vosso pacientíssimo Coração.

\rlettrine{III}{} Ao meditar no vosso pacientíssimo Coração. ó meu Jesus, fico confundido, e Vos dou graças por tantos exemplos maravilhosos de invicto sofrimentos, que nos deixastes. Eu me arrependo da minha indigna delicadeza, que se impacienta com a menor contrariedade. Ah! meu amado Jesus, infundi no meu coração um constante e forte amor às tribulações, às cruzes, às mortificações e à penitência, a fim de que, acompanhando-Vos ao Calvário, chegue convosco à glória e alegria do Paraíso.

\rlettrine{IV}{} Que horror sinto de mim, ó meu amado Jesus, ao contemplar o vosso amantíssimo Coração e ao ver como o meu coração é tão diverso do vosso; pois eu me inquieto, agasto e lamento à menor sombra, gesto ou palavra que contrarie. Ah! Senhor, perdoai-me todos estes defeitos, e concedei-me para o futuro a graça de imitar em qualquer contrariedade a vossa inalterável mansidão e por tal modo gozar santa e perpétua paz.

\rlettrine{V}{}Ó meu amado Jesus, vencedor da morte e do inferno, entoem-se louvores ao vosso generosíssimo Coração, que bem os merece. Quanto a mim, fico confundido ao ver o meu coração tão pusilânime, que estremece com qualquer injúria. Mas não serei mais assim! De Vós imploro tanta fortaleza e valor para sofrer as injúrias, que, combatendo e vencendo na terra, possa triunfar convosco no céu!\par

\emph{Volvamo-nos para o maternal Coração de Maria S.\textsuperscript{ma}}\par

\rlettrine{P}{elas} singulares prerrogativas do vosso dulcíssimo Coração, alcançai-me, ó Maria, Mãe de Deus e minha Mãe, uma verdadeira e permanente devoção ao Santíssimo Coração de Jesus, vosso Filho; e, assim, eu cumpra fielmente os meus deveres e com alegria sirva sempre, mas especialmente hoje, nosso Senhor Jesus Cristo.\par
℣. Coração de Jesus, abrasado em amor por nós.\par
℟. Inflamai os nossos corações de amor por Vós.\par
\begin{nscenter} Oremos. \end{nscenter}
Vos suplicamos, ó Senhor, que o Divino Espírito Santo nos inflame naquele fogo que nosso Senhor Jesus Cristo do íntimo doseu Coração lançou no mundo e quis que se acendesses em labaredas por toda a parte. Ele, que vive e reina em todos os séculos dos séculos.
℟. Amen.

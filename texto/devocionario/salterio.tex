\ilustra{gravuras/hieronymus/jeronimo}

\subsection{Introdução}

\rlettrine{T}{radução} latina por São Jerónimo e tradução portuguesa por Miguel Pereira da Silva, tendo como base as versões do Padre António de Pereira Figueiredo e do Padre Matos Soares. Foi minha intenção ser fiel ao texto latino e o uso de uma estilo literário intemporal. Uma linguagem litúrgica, que se quer bela, não mundana mas singular, própria e constante requer conhecimentos que infelizmente não possuo e não sou digno da tão grande tarefa, mas como não existe o que eu entendo por ser essencial, resolvi com preocupação, começar a trabalhar neste projecto. Começou por ser um livro de orações tradicionais para rezar em família. Logo veio a ideia do Saltério, parte central da oração cristã. Depois adicionei o Breve Ofício da Nossa Senhora e também o Missal tradicional completo. Quem sabe se um dia a obra não fica completa com o Breviário. Aqui apenas apresento o Saltério, uma obra mais literária, que se desenvolveu a partir dos salmos dos dous autores referidos anteriormente.

Para ser uma boa tradução para uma linguagem litúrgica é necessária musicalidade no texto, ritmo, poesia, gentileza e cadência, fidelidade ao texto original e sacralidade, reverência. Esta minha versão é a minha humilde tentativa, um primeiro passo, para esse fim. Este texto, assim como os outros que recolhi, irão estar disponíveis num sítio web para que se possa melhorar.

Os textos cristãos do Antigo Testamento têm como mãe a Septuaginta, versão grega das Sagradas Escrituras. O este e o oeste seguem a Septuaginta e o Saltério que aqui apresento é a segunda revisão de São Jerónimo do Saltério Vetus Latina, a mais antiga tradução latina que foi revista por São Jerónimo na que veio a ser conhecida como a Vulgata. A revisão de São Jerónimo foi feita na Palestina, ganhou fama na Gália e depois de Carlos Magno, usada em todo o ocidente, especialmente com a sua utilização no Breviário compilado por São Pio V.

São estes textos que os Pais da Igreja comentam, assim como os Doutores e Santos. Bento XVI afirma, no famoso discurso de Regensburg em 2006, que a Septuaginta é mais do que uma simples tradução, mas que se trata de um testemunho textual único e que acabou por ser significativo para o nascimento do cristianismo e sua difusão. Ao contrário dos protestantes, que escolheram o texto hebreu medieval, os cristãos antigos, orientais e ocidentais, viram na Septuaginta a sua rocha, a sua casa. Com o racionalismo e iluminismo, assim como pelo modernismo, veio também uma procura pelos textos hebreus, mas a tradição cristã antiga neste ponto é clara: a Septuaginta é o nosso texto, oficializada pelo Concílio de Trento e considerada livre de erro.

É por intermédio do Espírito Santo que nos abre os Salmos ao encontro do tesouro escondido no campo, Jesus Cristo. Os textos antigos são tesouros que não deveriamos ter arrogância de apagar. A versão aqui proposta é magnífica, mas encontramos tesouros também na Vetus Latina, por exemplo, no Salmo 95 encontramos uma referência a Deus reinando da árvore, que também encrontraomos no hino Vexilla Regis: «Cumpriu-se o oráculo de David, que nos seus cânticos inspirados havia anunciado às nações: «Deus reinará pelo madeiro».». Os salmos são hinos sagrados, por meio dos quais o povo de Deus louva o Senhor, implora a sua misericórdia, agradece os benefícios recebidos, e recorda os prodígios da sua paternal providência em favor de Israel.

Os ricos queriam ter belas versões do Saltério, com coloridas iluminuras e os pobres nos campos ouviam-no ser cantado pelos monges à distãncia. Tornou-se o Saltério da Vulgata e a base para o Canto Gregoriano, por ele muitos aprenderam a ler, outros tantos a cantar e muitos mais a louvar o Senhor.

No saltério encontra-se tudo o que de útil e salutar está espalhado pelos outros livros do Antigo Testamento. «Quando leio os salmos», diz Santo Ambrósio, «descubro neles todos os mystérios da nossa santa Religião, e tudo o que os profetas vaticinaram: reconheço a graça das revelações, os testemunhos da ressurreição de Jesus Cristo, os prémios e castigos da outra vida; e aprendo a confundir-me e a envergonhar-me dos meus pecados, a detestá-los e a evitá-los».

Miguel Filipe Bastos Pereira da Silva, Abril de 2019.

\newpage\null\thispagestyle{empty}\newpage

\section{Saltério}

\subsectioninfo{Salmo 1}{Beatus vir}\label{salmo1}
\begin{paracol}{2}\latim{
\rlettrine{B}{eátus} vir, qui non ábiit in consílio impiórum, et in via peccatórum non stetit, * et in cáthedra pestiléntiæ non sedit:
}\switchcolumn\portugues{
\rlettrine{B}{em-aventurado} o varão que não foi no conselho dos ímpios, nem ficou no caminho dos pecadores, * e na cadeira pestilencial se não sentou:
}\switchcolumn*\latim{
Sed in lege Dómini volúntas ejus, * et in lege ejus meditábitur die ac nocte.
}\switchcolumn\portugues{
Mas sua vontade está na lei do Senhor, * e na sua lei dia e noite meditará.
}\switchcolumn*\latim{
Et erit tamquam lignum, quod plantátum est secus decúrsus aquárum, * quod fructum suum dabit in témpore suo:
}\switchcolumn\portugues{
Ele será como a árvore, que está plantada junto às correntes das águas, * que dará fruto a seu tempo:
}\switchcolumn*\latim{
Et fólium ejus non défluet: * et ómnia quæcúmque fáciet, prosperabúntur.
}\switchcolumn\portugues{
Cuja folha não murchará: * e prosperará tudo quanto fizer.
}\switchcolumn*\latim{
Non sic ímpii, non sic: * sed tamquam pulvis, quem proícit ventus a fácie terræ.
}\switchcolumn\portugues{
Não assim os ímpios, não assim: * mas serão como o pó que o vento dispersa da face da terra.
}\switchcolumn*\latim{
Ideo non resúrgent ímpii in judício: * neque peccatóres in concílio justórum.
}\switchcolumn\portugues{
Por isso os ímpios não ressuscitarão no juízo: * nem os pecadores no concílio dos justos.
}\switchcolumn*\latim{
Quóniam novit Dóminus viam justórum: * et iter impiórum períbit.
}\switchcolumn\portugues{
Porque o Senhor conhece o caminho dos justos: * e o caminho dos ímpios perecerá.
}\end{paracol}


\subsectioninfo{Salmo 2}{Quare fremuerunt gentes}\label{salmo2}
\begin{paracol}{2}\latim{
\qlettrine{Q}{uare} fremuérunt gentes: * et pópuli meditáti sunt inánia?
}\switchcolumn\portugues{
\rlettrine{P}{or} que razão se embraveceram as gentes: * e os povos cousas vãs meditaram?
}\switchcolumn*\latim{
Astitérunt reges terræ, et príncipes convenérunt in unum * advérsus Dóminum, et advérsus Christum ejus.
}\switchcolumn\portugues{
Os reis da terra levantaram-se e os príncipes reuniram-se * contra o Senhor e contra o seu Cristo.
}\switchcolumn*\latim{
Dirumpámus víncula eórum: * et proiciámus a nobis jugum ipsórum.
}\switchcolumn\portugues{
Rompamos os seus laços: * e sacudamos de nós o seu jugo.
}\switchcolumn*\latim{
Qui hábitat in cælis, irridébit eos: * et Dóminus subsannábit eos.
}\switchcolumn\portugues{
Aquele que habita no céu rir-se-á deles: * e o Senhor os ridicularizá.
}\switchcolumn*\latim{
Tunc loquétur ad eos in ira sua, * et in furóre suo conturbábit eos.
}\switchcolumn\portugues{
Ele então lhes falará na sua ira, * e conturbá-los-á na sua fúria.
}\switchcolumn*\latim{
Ego autem constitútus sum Rex ab eo super Sion montem sanctum ejus, * prǽdicans præcéptum ejus.
}\switchcolumn\portugues{
Eu, porém, fui por Ele constituído Rei sobre Sião, seu santo monte, * para pregar a sua doutrina.
}\switchcolumn*\latim{
Dóminus dixit ad me: * Fílius meus es tu, ego hódie génui te.
}\switchcolumn\portugues{
Disse-me o Senhor: * tu és meu filho, eu hoje te gerei.
}\switchcolumn*\latim{
Póstula a me, et dabo tibi gentes hereditátem tuam, * et possessiónem tuam términos terræ.
}\switchcolumn\portugues{
Pede-me e dar-te-ei as gentes como tua herança, * e estenderei o teu domínio aos confins da terra.
}\switchcolumn*\latim{
Reges eos in virga férrea, * et tamquam vas fíguli confrínges eos.
}\switchcolumn\portugues{
Governá-las-ás com vara de ferro, * e quebrá-las-ás como um vaso do oleiro.
}\switchcolumn*\latim{
Et nunc, reges, intellégite: * erudímini, qui judicátis terram.
}\switchcolumn\portugues{
Agora, ó reis, entendei: * instruí-vos, vós que julgais a terra.
}\switchcolumn*\latim{
Servíte Dómino in timóre: * et exsultáte ei cum tremóre.
}\switchcolumn\portugues{
Servi o Senhor com temor: * e com tremor alegrai-vos n’Ele.
}\switchcolumn*\latim{
Apprehéndite disciplínam, nequándo irascátur Dóminus, * et pereátis de via justa.
}\switchcolumn\portugues{
Abraçai a disciplina, para que o Senhor se não irrite, * e não pereçais fora do caminho da justiça.
}\switchcolumn*\latim{
Cum exárserit in brevi ira ejus: * beáti omnes qui confídunt in eo.
}\switchcolumn\portugues{
Quando brevemente se incendiar a sua ira: * bem-aventurados todos os que n’Ele confiam.
}\end{paracol}


\subsectioninfo{Salmo 3}{Domine, quid multiplicati}\label{salmo3}
\begin{paracol}{2}\latim{
\rlettrine{D}{ómine,} quid multiplicáti sunt qui tríbulant me? * Multi insúrgunt advérsum me.
}\switchcolumn\portugues{
\rlettrine{S}{enhor,} porque tantos são os que me atribulam? * Muitos se insurgem contra mim.
}\switchcolumn*\latim{
Multi dicunt ánimæ meæ: * Non est salus ipsi in Deo ejus.
}\switchcolumn\portugues{
Muitos dizem à minha alma: * não há salvação para ele no seu Deus.
}\switchcolumn*\latim{
Tu autem, Dómine, suscéptor meus es, * glória mea, et exáltans caput meum.
}\switchcolumn\portugues{
Vós, porém, Senhor, sois o meu protector, * minha glória e exaltais a minha cabeça.
}\switchcolumn*\latim{
Voce mea ad Dóminum clamávi: * et exaudívit me de monte sancto suo.
}\switchcolumn\portugues{
Com minha voz ao Senhor clamei: * e Ele me ouviu do seu santo monte.
}\switchcolumn*\latim{
Ego dormívi, et soporátus sum: * et exsurréxi, quia Dóminus suscépit me.
}\switchcolumn\portugues{
Deitei-me para descansar e adormeci: * e levantei-me, pois me acolheu o Senhor.
}\switchcolumn*\latim{
Non timébo míllia pópuli circumdántis me: * exsúrge, Dómine, salvum me fac, Deus meus.
}\switchcolumn\portugues{
Não temerei milhares de pessoas me cercando: * levantai-Vos, ó Senhor, salvai-me, ó Deus meu!
}\switchcolumn*\latim{
Quóniam Tu percussísti omnes adversántes mihi sine causa: * dentes peccatórum contrivísti.
}\switchcolumn\portugues{
Porque Vós tendes ferido todos os que sem causa me perseguem: * quebrastes os dentes dos pecadores.
}\switchcolumn*\latim{
Dómini est salus: * et super pópulum tuum benedíctio tua.
}\switchcolumn\portugues{
A salvação é do Senhor: * e sua bênção está sobre seu povo.
}\end{paracol}


\subsectioninfo{Salmo 4}{Cum invocarem}\label{salmo4}
\begin{paracol}{2}\latim{
\rlettrine{C}{um} invocárem exaudívit me Deus justítiæ meæ: * in tribulatióne dilatásti mihi.
}\switchcolumn\portugues{
\qlettrine{Q}{uando} O invoquei, me ouviu o Deus da minha justiça: * na tribulação me dilatastes.
}\switchcolumn*\latim{
Miserére mei, * et exáudi oratiónem meam.
}\switchcolumn\portugues{
Tende compaixão de mim, * e escutai a minha oração.
}\switchcolumn*\latim{
Fílii hóminum, úsquequo gravi corde? * Ut quid dilígitis vanitátem, et quǽritis mendácium?
}\switchcolumn\portugues{
Filhos dos homens, até quando duros de coração sereis? * Porque amais a vaidade e mentiras buscais?
}\switchcolumn*\latim{
Et scitóte quóniam mirificávit Dóminus sanctum suum: * Dóminus exáudiet me cum clamávero ad eum.
}\switchcolumn\portugues{
Sabei, pois, que o Senhor fez maravilhoso o seu santo: * o Senhor escutar-me-á, quando a Ele clamar.
}\switchcolumn*\latim{
Irascímini, et nolíte peccáre: * quæ dícitis in córdibus vestris, in cubílibus vestris compungímini.
}\switchcolumn\portugues{
Irai-vos e não pequeis: * do que dizeis nos vossos corações, nos vossos leitos arrependei-vos.
}\switchcolumn*\latim{
Sacrificáte sacrifícium justítiæ, et speráte in Dómino. * Multi dicunt: quis osténdit nobis bona?
}\switchcolumn\portugues{
Oferecei sacrifícios de justiça e esperai no Senhor. * Muitos dizem: quem nos mostrará o bem?
}\switchcolumn*\latim{
Signátum est super nos lumen vultus tui, Dómine: * dedísti lætítiam in corde meo.
}\switchcolumn\portugues{
Gravada está sobre nós a luz de vossa face, ó Senhor: * no meu coração infundistes alegria.
}\switchcolumn*\latim{
A fructu fruménti, vini, et ólei sui * multiplicáti sunt.
}\switchcolumn\portugues{
Pelo fruto do seu trigo, vinho e azeite * se multiplicam.
}\switchcolumn*\latim{
In pace in idípsum * dórmiam, et requiéscam;
}\switchcolumn\portugues{
Em paz dormirei * e tranquilo descansarei;
}\switchcolumn*\latim{
Quóniam Tu, Dómine, singuláriter in spe * constituísti me.
}\switchcolumn\portugues{
Porque Vós, ó Senhor, de forma singular * na esperança me firmastes.
}\end{paracol}


\subsectioninfo{Salmo 5}{Verba mea auribus}\label{salmo5}
\begin{paracol}{2}\latim{
\rlettrine{V}{erba} mea áuribus pércipe, Dómine, * intéllege clamórem meum.
}\switchcolumn\portugues{
\rlettrine{S}{enhor,} dai ouvidos às minhas palavras, * escutai o meu clamor.
}\switchcolumn*\latim{
Inténde voci oratiónis meæ, * Rex meus et Deus meus.
}\switchcolumn\portugues{
Atendei à voz da minha súplica, * meu Rei e meu Deus.
}\switchcolumn*\latim{
Quóniam ad Te orábo: * Dómine, mane exáudies vocem meam.
}\switchcolumn\portugues{
Porque a Vós orarei: * Senhor, de manhã ouvireis a minha voz.
}\switchcolumn*\latim{
Mane astábo tibi et vidébo: * quóniam non Deus volens iniquitátem Tu es.
}\switchcolumn\portugues{
De manhã ficarei ante Vós e contemplarei: * porque não sois um Deus que ame a iniquidade.
}\switchcolumn*\latim{
Neque habitábit juxta Te malígnus: * neque permanébunt injústi ante óculos tuos.
}\switchcolumn\portugues{
Nem o maligno habitará junto de Vós: * nem os injustos poderão permanecer ante vossos olhos.
}\switchcolumn*\latim{
Odísti omnes, qui operántur iniquitátem: * perdes omnes, qui loquúntur mendácium.
}\switchcolumn\portugues{
Odieis todos os que obram a iniquidade: * exterminareis todos os que dizem a mentira.
}\switchcolumn*\latim{
Virum sánguinum et dolósum abominábitur Dóminus: * ego autem in multitúdine misericórdiæ tuæ.
}\switchcolumn\portugues{
O Senhor abominará o homem sanguinário e doloso: * eu, porém, confiado na abundância de vossa misericórdia.
}\switchcolumn*\latim{
Introíbo in domum tuam: * adorábo ad templum sanctum tuum in timóre tuo.
}\switchcolumn\portugues{
Entrarei na vossa casa: * e pelo vosso temor, no vosso santo templo Vos adorarei.
}\switchcolumn*\latim{
Dómine, deduc me in justítia tua: * propter inimícos meos dírige in conspéctu tuo viam meam.
}\switchcolumn\portugues{
Senhor, na vossa justiça guiai-me: * por causa dos meus inimigos dirigis ante vossos olhos o meu caminho.
}\switchcolumn*\latim{
Quóniam non est in ore eórum véritas: * cor eórum vanum est.
}\switchcolumn\portugues{
Porque não há verdade na boca deles: * vão é o seu coração.
}\switchcolumn*\latim{
Sepúlcrum patens est guttur eórum, linguis suis dolóse agébant, * júdica illos, Deus.
}\switchcolumn\portugues{
Sua garganta é um sepulcro aberto, urdem enganos com suas línguas, * julgai-os, ó Deus.
}\switchcolumn*\latim{
Décidant a cogitatiónibus suis, secúndum multitúdinem impietátum eórum expélle eos, * quóniam irritavérunt Te, Dómine.
}\switchcolumn\portugues{
Frustrem-se os seus desígnios, expulsai-os segundo a multidão das suas impiedades, * porque Vos irritaram, Senhor.
}\switchcolumn*\latim{
Et læténtur omnes, qui sperant in Te, * in ætérnum exsultábunt: et habitábis in eis.
}\switchcolumn\portugues{
Alegrem-se todos aqueles que em Vós esperam, * exultarão eternamente: e neles habitareis.
}\switchcolumn*\latim{
Et gloriabúntur in Te omnes, qui díligunt nomen tuum, * quóniam Tu benedíces justo.
}\switchcolumn\portugues{
Em Vós gloriar-se-ão todos os que amam o vosso nome, * porque o justo Vós o abençoareis.
}\switchcolumn*\latim{
Dómine, ut scuto bonæ voluntátis tuæ * coronásti nos.
}\switchcolumn\portugues{
Ó Senhor, como um escudo de vossa boa vontade * nos coroastes.
}\end{paracol}


\subsectioninfo{Salmo 6}{Domine, ne in furore tuo}\label{salmo6}
\begin{paracol}{2}\latim{
\rlettrine{D}{ómine,} ne in furóre tuo árguas me, * neque in ira tua corrípias me.
}\switchcolumn\portugues{
\rlettrine{S}{enhor,} me não acuseis na vossa indignação, * nem me castigueis na vossa ira.
}\switchcolumn*\latim{
Miserére mei, Dómine, quóniam infírmus sum: * sana me, Dómine, quóniam conturbáta sunt ossa mea.
}\switchcolumn\portugues{
Tende misericórdia de mim, ó Senhor, porque estou fraco: * sarai-me Senhor, porque estão abalados os meus ossos.
}\switchcolumn*\latim{
Et ánima mea turbáta est valde: * sed Tu, Dómine, úsquequo?
}\switchcolumn\portugues{
Turvou-se-me a alma profundamente: * mas Vós, ó Senhor, até quando?
}\switchcolumn*\latim{
Convértere, Dómine, et éripe ánimam meam: * salvum me fac propter misericórdiam tuam.
}\switchcolumn\portugues{
Volvei Senhor e livrai a minha alma: * salvai-me pela vossa misericórdia.
}\switchcolumn*\latim{
Quóniam non est in morte qui memor sit tui: * in inférno autem quis confitébitur tibi?
}\switchcolumn\portugues{
Porque na morte não há quem se recorde de Vós: * no inferno quem Vos louvará?
}\switchcolumn*\latim{
Laborávi in gémitu meo, lavábo per síngulas noctes lectum meum: * lácrimis meis stratum meum rigábo.
}\switchcolumn\portugues{
Esgotei-me com meus gemidos, lavarei o meu leito todas as noites: * com lágrimas regarei a minha cama.
}\switchcolumn*\latim{
Turbátus est a furóre óculus meus: * inveterávi inter omnes inimícos meos.
}\switchcolumn\portugues{
Turvou-se-me o olho devido à indignação: * envelheci no meio de todos meus inimigos.
}\switchcolumn*\latim{
Discédite a me, omnes, qui operámini iniquitátem: * quóniam exaudívit Dóminus vocem fletus mei.
}\switchcolumn\portugues{
Apartai-vos de mim todos os que praticais a iniquidade: * porque o Senhor ouviu a voz do meu pranto.
}\switchcolumn*\latim{
Exaudívit Dóminus deprecatiónem meam, * Dóminus oratiónem meam suscépit.
}\switchcolumn\portugues{
O Senhor ouviu a minha súplica, * o Senhor acolheu a minha oração.
}\switchcolumn*\latim{
Erubéscant, et conturbéntur veheménter omnes inimíci mei: * convertántur et erubéscant valde velóciter.
}\switchcolumn\portugues{
Envergonhados e extremadamente conturbados sejam todos meus inimigos: * retirem-se e sejam velozmente cobertos de vergonha.
}\end{paracol}


\subsectioninfo{Salmo 7}{Domine Deus meus}\label{salmo7}
\begin{paracol}{2}\latim{
\rlettrine{D}{ómine,} Deus meus, in Te sperávi: * salvum me fac ex ómnibus persequéntibus me, et líbera me.
}\switchcolumn\portugues{
\rlettrine{S}{enhor,} Deus meu, em Vós esperei: * salvai-me de todos os que me perseguem e livrai-me.
}\switchcolumn*\latim{
Nequándo rápiat ut leo ánimam meam, * dum non est qui rédimat, neque qui salvum fáciat.
}\switchcolumn\portugues{
Para que ninguém rasgue como um leão a minha alma, * sem que haja quem me livre, nem quem me salve.
}\switchcolumn*\latim{
Dómine, Deus meus, si feci istud, * si est iníquitas in mánibus meis:
}\switchcolumn\portugues{
Ó Senhor meu Deus, se fiz isso, * se há iniquidade nas minhas mãos:
}\switchcolumn*\latim{
Si réddidi retribuéntibus mihi mala, * décidam mérito ab inimícis meis inánis.
}\switchcolumn\portugues{
Se retribuí maldades aos que mas faziam, * caia justamente debaixo dos meus inimigos.
}\switchcolumn*\latim{
Persequátur inimícus ánimam meam, et comprehéndat, et concúlcet in terra vitam meam, * et glóriam meam in púlverem dedúcat.
}\switchcolumn\portugues{
Persiga o inimigo a minha alma, apodere-a e calque na terra a minha vida * e a pó reduza a minha glória.
}\switchcolumn*\latim{
Exsúrge, Dómine, in ira tua: * et exaltáre in fínibus inimicórum meórum.
}\switchcolumn\portugues{
Levantai-Vos, ó Senhor, na vossa ira: * e exaltai-Vos nas fronteiras dos meus inimigos.
}\switchcolumn*\latim{
Et exsúrge, Dómine, Deus meus, in præcépto quod mandásti: * et synagóga populórum circúmdabit Te.
}\switchcolumn\portugues{
Levantai-Vos, ó Senhor meu Deus, na lei que ordenastes: * e rodear-Vos-á a congregação dos povos.
}\switchcolumn*\latim{
Et propter hanc in altum regrédere: * Dóminus júdicat pópulos.
}\switchcolumn\portugues{
Por esta ao alto retornai: * o Senhor é que julga os povos.
}\switchcolumn*\latim{
Júdica me, Dómine, secúndum justítiam meam, * et secúndum innocéntiam meam super me.
}\switchcolumn\portugues{
Julgai-me, ó Senhor, segundo a minha justiça, * e segundo a inocência que há em mim.
}\switchcolumn*\latim{
Consumétur nequítia peccatórum, et díriges justum, * scrutans corda et renes, Deus.
}\switchcolumn\portugues{
Será consumida a malícia dos pecadores e encaminhareis o justo, * sondais os corações e as entranhas, ó Deus.
}\switchcolumn*\latim{
Justum adjutórium meum a Dómino, * qui salvos facit rectos corde.
}\switchcolumn\portugues{
Justo é o meu auxílio que vem do Senhor, * que salva os rectos de coração.
}\switchcolumn*\latim{
Deus judex justus, fortis, et pátiens: * numquid iráscitur per síngulos dies?
}\switchcolumn\portugues{
Deus é um juiz justo, forte e paciente: * ira-se todos os dias porventura?
}\switchcolumn*\latim{
Nisi convérsi fuéritis, gládium suum vibrábit: * arcum suum teténdit, et parávit illum.
}\switchcolumn\portugues{
Se vos não converterdes, vibrará a sua espada: * armou o seu arco e tem-no preparado.
}\switchcolumn*\latim{
Et in eo parávit vasa mortis: * sagíttas suas ardéntibus effécit.
}\switchcolumn\portugues{
Pôs nele dardos mortais: * preparou as suas setas ardentes.
}\switchcolumn*\latim{
Ecce, partúriit injustítiam: * concépit dolórem, et péperit iniquitátem.
}\switchcolumn\portugues{
Eis que pariu a injustiça: * concebeu dor e nasceu a iniquidade.
}\switchcolumn*\latim{
Lacum apéruit, et effódit eum: * et íncidit in fóveam quam fecit.
}\switchcolumn\portugues{
Fosso abriu e o cavou: * e caiu na cova que fez.
}\switchcolumn*\latim{
Convertétur dolor ejus in caput ejus: * et in vérticem ipsíus iníquitas ejus descéndet.
}\switchcolumn\portugues{
A dor volver-se-á contra a sua cabeça: * e sobre a sua fronte recairá a sua iniquidade.
}\switchcolumn*\latim{
Confitébor Dómino secúndum justítiam ejus: * et psallam nómini Dómini altíssimi.
}\switchcolumn\portugues{
Glorificarei o Senhor segundo a sua justiça: * e cantarei o nome do altíssimo Senhor.
}\end{paracol}


\subsectioninfo{Salmo 8}{Domine, Dominus noster}\label{salmo8}
\begin{paracol}{2}\latim{
\rlettrine{D}{ómine,} Dóminus noster, * quam admirábile est nomen tuum in univérsa terra!
}\switchcolumn\portugues{
\slettrine{Ó}{} Senhor, Senhor nosso, * quão admirável é o vosso nome em toda a terra!
}\switchcolumn*\latim{
Quóniam eleváta est magnificéntia tua, * super cælos.
}\switchcolumn\portugues{
Pois se elevou a vossa majestade * sobre os céus.
}\switchcolumn*\latim{
Ex ore infántium et lacténtium perfecísti laudem propter inimícos tuos, * ut déstruas inimícum et ultórem.
}\switchcolumn\portugues{
Da boca dos meninos e lactentes fizestes sair um louvor perfeito, devido aos vossos inimigos, * para destruirdes o inimigo e o vingativo.
}\switchcolumn*\latim{
Quóniam vidébo cælos tuos, ópera digitórum tuórum: * lunam et stellas, quæ Tu fundásti.
}\switchcolumn\portugues{
Contemplarei os vossos céus, obra de vossos dedos: * a lua e as estrelas que Vós fundastes.
}\switchcolumn*\latim{
Quid est homo quod memor es ejus? * Aut fílius hóminis, quóniam vísitas eum?
}\switchcolumn\portugues{
Que é o homem, para Vos lembrardes dele? * Ou que é o filho do homem, para o visitardes?
}\switchcolumn*\latim{
Minuísti eum paulo minus ab Ángelis, glória et honóre coronásti eum: * et constituísti eum super ópera mánuum tuárum.
}\switchcolumn\portugues{
Pouco inferior aos anjos Vós o fizestes, de glória e de honra o coroastes: * e lhe destes o poder sobre as obras de vossas mãos.
}\switchcolumn*\latim{
Omnia subjecísti sub pédibus ejus, * oves et boves univérsas: ínsuper et pécora campi.
}\switchcolumn\portugues{
Tudo sob seus pés sujeitastes, * todas as ovelhas e bois: e, além destes, os outros animais do campo.
}\switchcolumn*\latim{
Vólucres cæli, et pisces maris, * qui perámbulant sémitas maris.
}\switchcolumn\portugues{
As aves do céu e os peixes do mar, * que percorrem as veredas do oceano.
}\switchcolumn*\latim{
Dómine, Dóminus noster, * quam admirábile est nomen tuum in univérsa terra!
}\switchcolumn\portugues{
Ó Senhor, Senhor nosso, * quão admirável é o vosso nome em toda a terra!
}\end{paracol}


\subsectioninfo{Salmo 9}{Confitebor tibi}\label{salmo9}
\begin{paracol}{2}\latim{
\rlettrine{C}{onfitébor} tibi, Dómine, in toto corde meo: * narrábo ómnia mirabília tua.
}\switchcolumn\portugues{
\rlettrine{E}{u} Vos glorificarei, ó Senhor, com todo meu coração: * narrarei todas vossas maravilhas.
}\switchcolumn*\latim{
Lætábor et exsultábo in Te: * psallam nómini tuo, Altíssime.
}\switchcolumn\portugues{
Alegrar-me-ei e em Vós exultarei: * cantarei o vosso nome, ó Altíssimo.
}\switchcolumn*\latim{
In converténdo inimícum meum retrórsum: * infirmabúntur, et períbunt a fácie tua.
}\switchcolumn\portugues{
Quando baterem em retirada os meus inimigos: * cairão e perecerão ante Vós.
}\switchcolumn*\latim{
Quóniam fecísti judícium meum et causam meam: * sedísti super thronum, qui júdicas justítiam.
}\switchcolumn\portugues{
Porque julgastes e defendestes a minha causa: * sentastes-Vos sobre o trono, Vós que justamente julgais.
}\switchcolumn*\latim{
Increpásti gentes, et périit ímpius: * nomen eórum delésti in ætérnum, et in sǽculum sǽculi.
}\switchcolumn\portugues{
Repreendestes as gentes e o ímpio pereceu: * apagastes o nome delas para sempre e por todos os séculos dos séculos.
}\switchcolumn*\latim{
Inimíci defecérunt frámeæ in finem: * et civitátes eórum destruxísti.
}\switchcolumn\portugues{
As espadas do inimigo ficaram embotadas para sempre: * e as suas cidades destruístes.
}\switchcolumn*\latim{
Périit memória eórum cum sónitu: * et Dóminus in ætérnum pérmanet.
}\switchcolumn\portugues{
Com estrondo pereceu a memória deles: * mas o Senhor permanece eternamente.
}\switchcolumn*\latim{
Parávit in judício thronum suum: * et ipse judicábit orbem terræ in æquitáte, judicábit pópulos in justítia.
}\switchcolumn\portugues{
Preparou o seu trono para o juízo: * e Ele mesmo julgará com equidade toda a terra, julgará os povos com justiça.
}\switchcolumn*\latim{
Et factus est Dóminus refúgium páuperi: * adjútor in opportunitátibus, in tribulatióne.
}\switchcolumn\portugues{
O Senhor fez-se o refúgio do pobre: * socorrendo-o oportunamente na tribulação.
}\switchcolumn*\latim{
Et sperent in Te qui novérunt nomen tuum: * quóniam non dereliquísti quæréntes Te, Dómine.
}\switchcolumn\portugues{
Em Vós esperem os que conhecem o vosso nome: * porque Vós, ó Senhor, não desamparastes os que Vos buscam.
}\switchcolumn*\latim{
Psállite Dómino, qui hábitat in Sion: * annuntiáte inter gentes stúdia ejus:
}\switchcolumn\portugues{
Cantai ao Senhor que habita em Sião: * anunciai os seus desígnios entre as gentes:
}\switchcolumn*\latim{
Quóniam requírens sánguinem eórum recordátus est: * non est oblítus clamórem páuperum.
}\switchcolumn\portugues{
Porque, vingando o seu sangue, mostrou que delas se lembra: * do clamor dos pobres se não esqueceu.
}\switchcolumn*\latim{
Miserére mei, Dómine: * vide humilitátem meam de inimícis meis.
}\switchcolumn\portugues{
Tende compaixão de mim, Senhor: * vede o meu abatimento que vem dos meus inimigos.
}\switchcolumn*\latim{
Qui exáltas me de portis mortis, * ut annúntiem omnes laudatiónes tuas in portis fíliæ Sion.
}\switchcolumn\portugues{
Que me ergueis das portas da morte, * para que anuncie todos vossos louvores às portas da filha de Sião.
}\switchcolumn*\latim{
Exsultábo in salutári tuo: * infíxæ sunt gentes in intéritu, quem fecérunt.
}\switchcolumn\portugues{
Exultarei na salvação que me obtivestes: * as gentes caíram na ruína que me tinham preparado.
}\switchcolumn*\latim{
In láqueo isto, quem abscondérunt, * comprehénsus est pes eórum.
}\switchcolumn\portugues{
No laço que me tinham preparado, * o seu pé ficou preso.
}\switchcolumn*\latim{
Cognoscétur Dóminus judícia fáciens: * in opéribus mánuum suárum comprehénsus est peccátor.
}\switchcolumn\portugues{
Conhecer-se-á que o Senhor faz justiça: * nas obras das suas mãos foi preso o pecador.
}\switchcolumn*\latim{
Convertántur peccatóres in inférnum, * omnes gentes quæ obliviscúntur Deum.
}\switchcolumn\portugues{
Sejam precipitados no inferno todos os pecadores, * todos as gentes que de Deus se esquecem.
}\switchcolumn*\latim{
Quóniam non in finem oblívio erit páuperis: * patiéntia páuperum non períbit in finem.
}\switchcolumn\portugues{
Porque não estará para sempre esquecido o pobre: * nem a paciência dos infelizes será para sempre frustrada.
}\switchcolumn*\latim{
Exsúrge, Dómine, non confortétur homo: * judicéntur gentes in conspéctu tuo.
}\switchcolumn\portugues{
Levantai-Vos, ó Senhor, não triunfe o homem: * sejam julgadas as gentes em vossa presença.
}\switchcolumn*\latim{
Constítue, Dómine, legislatórem super eos: * ut sciant gentes quóniam hómines sunt.
}\switchcolumn\portugues{
Senhor, estabelecei sobre elas um legislador: * para que as gentes saibam que são apenas homens.
}\switchcolumn*\latim{
Ut quid, Dómine, recessísti longe, * déspicis in opportunitátibus, in tribulatióne?
}\switchcolumn\portugues{
Senhor, porque Vos apartastes para longe, * desamparais-nos nas necessidades e na tribulação?
}\switchcolumn*\latim{
Dum supérbit ímpius, incénditur pauper: * comprehendúntur in consíliis quibus cógitant.
}\switchcolumn\portugues{
Enquanto o ímpio se envaidece, o pobre é abrasado: * são apanhados nas intrigas que teceram.
}\switchcolumn*\latim{
Quóniam laudátur peccátor in desidériis ánimæ suæ: * et iníquus benedícitur.
}\switchcolumn\portugues{
Pois o pecador vangloria-se nos desejos da sua alma: * e o iníquo é felicitado.
}\switchcolumn*\latim{
Exacerbávit Dóminum peccátor, * secúndum multitúdinem iræ suæ non quǽret.
}\switchcolumn\portugues{
O pecador exacerbou o Senhor, * devido à sua grande ira Ele o não procurará.
}\switchcolumn*\latim{
Non est Deus in conspéctu ejus: * inquinátæ sunt viæ illíus in omni témpore.
}\switchcolumn\portugues{
Não há Deus diante dele: * os seus caminhos são sempre viciosos.
}\switchcolumn*\latim{
Auferúntur judícia tua a fácie ejus: * ómnium inimicórum suórum dominábitur.
}\switchcolumn\portugues{
Não estão ante sua vista Vossos juízos: * dominará ele todos seus inimigos.
}\switchcolumn*\latim{
Dixit enim in corde suo: * Non movébor a generatióne in generatiónem sine malo.
}\switchcolumn\portugues{
Pois disse no seu coração: * não serei movido de geração em geração e do mal estarei livre.
}\switchcolumn*\latim{
Cujus maledictióne os plenum est, et amaritúdine, et dolo: * sub lingua ejus labor et dolor.
}\switchcolumn\portugues{
Sua boca está cheia de maledicência, de amargura e de dolo: * debaixo da sua língua estão o trabalho e a dor.
}\switchcolumn*\latim{
Sedet in insídiis cum divítibus in occúltis: * ut interfíciat innocéntem.
}\switchcolumn\portugues{
Senta-se em emboscada com os ricos em lugares ocultos: * para o inocente matar.
}\switchcolumn*\latim{
Óculi ejus in páuperem respíciunt: * insidiátur in abscóndito, quasi leo in spelúnca sua.
}\switchcolumn\portugues{
Seus olhos estão sobre o pobre: * aguarda escondido como o leão na sua cova.
}\switchcolumn*\latim{
Insidiátur ut rápiat páuperem: * rápere páuperem, dum áttrahit eum.
}\switchcolumn\portugues{
Arma ciladas para arrebatar o pobre: * para arrebatar o pobre, atraindo-o a si.
}\switchcolumn*\latim{
In láqueo suo humiliábit eum: * inclinábit se, et cadet, cum dominátus fúerit páuperum.
}\switchcolumn\portugues{
No seu laço ele fá-lo-á cair: * inclinar-se-á e cairá sobre os pobres, logo que se apoderar deles.
}\switchcolumn*\latim{
Dixit enim in corde suo: oblítus est Deus, * avértit fáciem suam ne vídeat in finem.
}\switchcolumn\portugues{
Pois disse no seu coração: Deus esqueceu-se, * virou o seu rosto para até ao fim não ver.
}\switchcolumn*\latim{
Exsúrge, Dómine Deus, exaltétur manus tua: * ne obliviscáris páuperum.
}\switchcolumn\portugues{
Levantai-Vos, ó Senhor Deus, eleve-se a vossa mão: * e dos pobres Vos não esqueçais.
}\switchcolumn*\latim{
Propter quid irritávit ímpius Deum? * Dixit enim in corde suo: non requíret.
}\switchcolumn\portugues{
Por que motivo o ímpio irritou a Deus? * Porque disse no seu coração: Ele não exige.
}\switchcolumn*\latim{
Vides quóniam Tu labórem et dolórem consíderas: * ut tradas eos in manus tuas.
}\switchcolumn\portugues{
Porém, Vós o vedes, considerais o trabalho e a dor: * para o tomardes nas vossas mãos.
}\switchcolumn*\latim{
Tibi derelíctus est pauper: * órphano Tu eris adjútor.
}\switchcolumn\portugues{
A Vós se abandona o infeliz: * sereis Vós o amparo do órfão.
}\switchcolumn*\latim{
Cóntere brácchium peccatóris et malígni: * quærétur peccátum illíus, et non inveniétur.
}\switchcolumn\portugues{
Quebrai o braço do pecador e do maligno: * buscar-se-á o seu pecado e se não achará.
}\switchcolumn*\latim{
Dóminus regnábit in ætérnum, et in sǽculum sǽculi: * períbitis, gentes, de terra illíus.
}\switchcolumn\portugues{
O Senhor reinará eternamente e pelos séculos dos séculos: * vós, ó gentes, sereis exterminadas da sua terra.
}\switchcolumn*\latim{
Desidérium páuperum exaudívit Dóminus: * præparatiónem cordis eórum audívit auris tua.
}\switchcolumn\portugues{
O Senhor ouviu o desejo dos pobres: * o vosso ouvido atendeu à prece do seu coração.
}\switchcolumn*\latim{
Judicáre pupíllo et húmili, * ut non appónat ultra magnificáre se homo super terram.
}\switchcolumn\portugues{
Para fazerdes justiça ao órfão e ao humilde, * a fim de que o homem cesse de se engrandecer sobre a terra.
}\end{paracol}


\subsectioninfo{Salmo 10}{In Domino confido}\label{salmo10}
\begin{paracol}{2}\latim{
\rlettrine{I}{n} Dómino confído: quómodo dícitis ánimæ meæ: * Tránsmigra in montem sicut passer?
}\switchcolumn\portugues{
\rlettrine{N}{o} Senhor confio: porque dizeis à minha alma: * migra para o monte como a ave?
}\switchcolumn*\latim{
Quóniam ecce peccatóres intendérunt arcum, paravérunt sagíttas suas in pháretra, * ut sagíttent in obscúro rectos corde.
}\switchcolumn\portugues{
Eis que os pecadores mostraram o seu arco, prepararam as suas setas na aljava, * para no escuro dispararem aos rectos de coração.
}\switchcolumn*\latim{
Quóniam quæ perfecísti, destruxérunt: * justus autem quid fecit?
}\switchcolumn\portugues{
Porque eles destruíram o que fizestes de bom: * mas que fez o justo?
}\switchcolumn*\latim{
Dóminus in templo sancto suo, * Dóminus in cælo sedes ejus.
}\switchcolumn\portugues{
O Senhor habita no seu santo templo, * o trono do Senhor está no céu.
}\switchcolumn*\latim{
Óculi ejus in páuperem respíciunt: * pálpebræ ejus intérrogant fílios hóminum.
}\switchcolumn\portugues{
Seus olhos olham para o pobre: * suas pálpebras inquirem os filhos dos homens.
}\switchcolumn*\latim{
Dóminus intérrogat justum et ímpium: * qui autem díligit iniquitátem, odit ánimam suam.
}\switchcolumn\portugues{
O Senhor interroga o justo e o ímpio: * mas aquele que ama a iniquidade, odeia a sua alma.
}\switchcolumn*\latim{
Pluet super peccatóres láqueos: * ignis, et sulphur, et spíritus procellárum pars cálicis eórum.
}\switchcolumn\portugues{
Fará chover laços sobre os pecadores: * o fogo e o enxofre e as tempestades são a parte que lhes toca.
}\switchcolumn*\latim{
Quóniam justus Dóminus, et justítias diléxit: * æquitátem vidit vultus ejus.
}\switchcolumn\portugues{
Porque o Senhor é justo e ama a justiça: * o seu rosto olha para a equidade.
}\end{paracol}


\subsectioninfo{Salmo 11}{Salvum me fac}\label{salmo11}
\begin{paracol}{2}\latim{
\rlettrine{S}{alvum} me fac, Dómine, quóniam defécit sanctus: * quóniam diminútæ sunt veritátes a fíliis hóminum.
}\switchcolumn\portugues{
\rlettrine{S}{alvai-me,} ó Senhor, porque se dissipou o santo: * porque as verdades são depreciadas entre os filhos dos homens.
}\switchcolumn*\latim{
Vana locúti sunt unusquísque ad próximum suum: * lábia dolósa, in corde et corde locúti sunt.
}\switchcolumn\portugues{
Cada um deles diz vãs cousas ao seu próximo: * fala com os lábios enganosos e com coração dúplice.
}\switchcolumn*\latim{
Dispérdat Dóminus univérsa lábia dolósa, * et linguam magníloquam.
}\switchcolumn\portugues{
Destrua o Senhor todos os lábios enganosos, * e a língua que fala com arrogância.
}\switchcolumn*\latim{
Qui dixérunt: Linguam nostram magnificábimus, lábia nostra a nobis sunt, * quis noster Dóminus est?
}\switchcolumn\portugues{
Os que disseram: faremos grandes cousas com a nossa língua, somos donos dos nossos lábios, * o nosso Senhor quem é?
}\switchcolumn*\latim{
Propter misériam ínopum, et gémitum páuperum, * nunc exsúrgam, dicit Dóminus.
}\switchcolumn\portugues{
Pela miséria dos desvalidos e o gemido dos pobres, * agora me levantarei, diz o Senhor.
}\switchcolumn*\latim{
Ponam in salutári: * fiduciáliter agam in eo.
}\switchcolumn\portugues{
A salvo os porei: * nisto procederei confiadamente.
}\switchcolumn*\latim{
Elóquia Dómini, elóquia casta: * argéntum igne examinátum, probátum terræ purgátum séptuplum.
}\switchcolumn\portugues{
As palavras do Senhor são palavras castas: * como prata refinada num forno de barro, sete vezes purificada.
}\switchcolumn*\latim{
Tu, Dómine, servábis nos: et custódies nos * a generatióne hac in ætérnum.
}\switchcolumn\portugues{
Vós, ó Senhor, nos guardareis e nos preservareis * para sempre desta geração.
}\switchcolumn*\latim{
In circúitu ímpii ámbulant: * secúndum altitúdinem tuam multiplicásti fílios hóminum.
}\switchcolumn\portugues{
Os ímpios em circuito ambulam: * segundo a vossa altitude, multiplicastes os filhos dos homens.
}\end{paracol}


\subsectioninfo{Salmo 12}{Usquequo, Domine}\label{salmo12}
\begin{paracol}{2}\latim{
\slettrine{Ú}{squequo,} Dómine, obliviscéris me in finem? * Úsquequo avértis fáciem tuam a me?
}\switchcolumn\portugues{
\rlettrine{A}{té} quando, ó Senhor, me esquecereis para sempre? * Até quando afastareis de mim a vossa face?
}\switchcolumn*\latim{
Quámdiu ponam consília in ánima mea, * dolórem in corde meo per diem?
}\switchcolumn\portugues{
Até quando trarei a minha alma com planos, * e o meu coração todo o dia em dor?
}\switchcolumn*\latim{
Úsquequo exaltábitur inimícus meus super me? * Réspice, et exáudi me, Dómine, Deus meus.
}\switchcolumn\portugues{
Até quando o meu inimigo será exaltado sobre mim? * Olhai para mim e escutai-me, ó Senhor meu Deus.
}\switchcolumn*\latim{
Illúmina óculos meos ne umquam obdórmiam in morte: * nequándo dicat inimícus meus: præválui advérsus eum.
}\switchcolumn\portugues{
Iluminai os meus olhos para que nunca durma na morte: * para que nunca o meu inimigo diga: prevaleci contra ele.
}\switchcolumn*\latim{
Qui tríbulant me, exsultábunt si motus fúero: * ego autem in misericórdia tua sperávi.
}\switchcolumn\portugues{
Os que me atribulam exultarão se for amotinado: * eu, porém, esperei na vossa misericórdia.
}\switchcolumn*\latim{
Exsultábit cor meum in salutári tuo: cantábo Dómino qui bona tríbuit mihi: * et psallam nómini Dómini altíssimi.
}\switchcolumn\portugues{
Meu coração exultará na vossa salvação: cantarei ao Senhor que bem me fez: * e salmos entoarei ao nome do Senhor altíssimo.
}\end{paracol}


\subsectioninfo{Salmo 13}{Dixit insipiens}\label{salmo13}
\begin{paracol}{2}\latim{
\rlettrine{D}{ixit} insípiens in corde suo: * non est Deus.
}\switchcolumn\portugues{
\rlettrine{O}{} insensato disse no seu coração: * Não há Deus.
}\switchcolumn*\latim{
Corrúpti sunt, et abominábiles facti sunt in stúdiis suis: * non est qui fáciat bonum, non est usque ad unum.
}\switchcolumn\portugues{
Corromperam-se e tornaram-se abomináveis nos seus desejos: * não há quem o bem faça, não há nem sequer um.
}\switchcolumn*\latim{
Dóminus de cælo prospéxit super fílios hóminum, * ut vídeat si est intéllegens, aut requírens Deum.
}\switchcolumn\portugues{
O Senhor olhou do céu para os filhos dos homens, * para ver se há quem tenha inteligência, ou busque a Deus.
}\switchcolumn*\latim{
Omnes declinavérunt, simul inútiles facti sunt: * non est qui fáciat bonum, non est usque ad unum.
}\switchcolumn\portugues{
Todos se extraviaram, todos se tornaram inúteis: * não há quem o bem faça, não há nem sequer um.
}\switchcolumn*\latim{
Sepúlcrum patens est guttur eórum: linguis suis dolóse agébant * venénum áspidum sub lábiis eórum.
}\switchcolumn\portugues{
Sua garganta é um sepulcro aberto; com suas línguas urdem enganos, * debaixo dos seus lábios há áspides venenosas.
}\switchcolumn*\latim{
Quorum os maledictióne et amaritúdine plenum est: * velóces pedes eórum ad effundéndum sánguinem.
}\switchcolumn\portugues{
Sua boca está cheia de maldição e de amargura: * os seus pés são velozes para derramar sangue.
}\switchcolumn*\latim{
Contrítio et infelícitas in viis eórum, et viam pacis non cognovérunt: * non est timor Dei ante óculos eórum.
}\switchcolumn\portugues{
Há tormento e desgraça nos seus caminhos e não conheceram o caminho da paz: * não há temor de Deus ante seus olhos.
}\switchcolumn*\latim{
Nonne cognóscent omnes qui operántur iniquitátem, * qui dévorant plebem meam sicut escam panis?
}\switchcolumn\portugues{
Não terão porventura conhecimento todos os que obram a iniquidade, * os que devoram o meu povo como a um pão?
}\switchcolumn*\latim{
Dóminum non invocavérunt, * illic trepidavérunt timóre, ubi non erat timor.
}\switchcolumn\portugues{
Não invocaram o Senhor, * ali tremeram de medo, onde não havia que temer.
}\switchcolumn*\latim{
Quóniam Dóminus in generatióne justa est, consílium ínopis confudístis: * quóniam Dóminus spes ejus est.
}\switchcolumn\portugues{
Porque o Senhor está com a geração dos justos, confundistes os planos do pobre: * mas o Senhor é a sua esperança.
}\switchcolumn*\latim{
Quis dabit ex Sion salutáre Israël? * Cum avérterit Dóminus captivitátem plebis suæ, exsultábit Jacob, et lætábitur Israël.
}\switchcolumn\portugues{
Quem enviará de Sião a salvação de Israel? * Quando o Senhor puser fim ao cativeiro do seu povo, exultará Jacob e alegrar-se-á Israel.
}\end{paracol}


\subsectioninfo{Salmo 14}{Domine, quis habitabit}\label{salmo14}
\begin{paracol}{2}\latim{
\rlettrine{D}{ómine,} quis habitábit in tabernáculo tuo? * Aut quis requiéscet in monte sancto tuo?
}\switchcolumn\portugues{
\rlettrine{S}{enhor,} quem habitará no vosso tabernáculo? * Ou quem descansará no vosso santo monte?
}\switchcolumn*\latim{
Qui ingréditur sine mácula, * et operátur justítiam:
}\switchcolumn\portugues{
O que vive sem mácula, * e pratica a justiça:
}\switchcolumn*\latim{
Qui lóquitur veritátem in corde suo, * qui non egit dolum in lingua sua:
}\switchcolumn\portugues{
O que fala verdade no seu coração, * o que não forjou dolos com sua língua:
}\switchcolumn*\latim{
Nec fecit próximo suo malum, * et oppróbrium non accépit advérsus próximos suos.
}\switchcolumn\portugues{
Nem mal fez ao seu próximo, * nem consentiu que seus próximos fossem desonrados.
}\switchcolumn*\latim{
Ad níhilum dedúctus est in conspéctu ejus malígnus: * timéntes autem Dóminum gloríficat:
}\switchcolumn\portugues{
Na sua apreciação considera o malvado como um nada, * mas honra os que temem o Senhor:
}\switchcolumn*\latim{
Qui jurat próximo suo, et non décipit, * qui pecúniam suam non dedit ad usúram, et múnera super innocéntem non accépit.
}\switchcolumn\portugues{
Faz juramento ao seu próximo e o não engana, * não empresta o seu dinheiro com usura, nem aceita subornos contra o inocente.
}\switchcolumn*\latim{
Qui facit hæc: * non movébitur in ætérnum.
}\switchcolumn\portugues{
Quem procede assim: * jamais será abalado.
}\end{paracol}


\subsectioninfo{Salmo 15}{Conserva me}\label{salmo15}
\begin{paracol}{2}\latim{
\rlettrine{C}{onsérva} me, Dómine, quóniam sperávi in Te. * Dixi Dómino: Deus meus es Tu, quóniam bonórum meórum non eges.
}\switchcolumn\portugues{
\rlettrine{G}{uardai-me,} ó Senhor, porque em Vós esperei. * Disse ao Senhor: Vós sois o meu Deus, que não tem necessidade dos meus bens.
}\switchcolumn*\latim{
Sanctis, qui sunt in terra ejus, * mirificávit omnes voluntátes meas in eis.
}\switchcolumn\portugues{
Para com os santos que estão sobre a sua terra, * maravilhosos fez neles todos meus desejos.
}\switchcolumn*\latim{
Multiplicátæ sunt infirmitátes eórum: * póstea acceleravérunt.
}\switchcolumn\portugues{
Multiplicaram-se suas enfermidades: * depois correram aceleradamente.
}\switchcolumn*\latim{
Non congregábo conventícula eórum de sanguínibus, * nec memor ero nóminum eórum per lábia mea.
}\switchcolumn\portugues{
Não me juntarei a eles nas suas reuniões sanguinários, * nem terei nos meus lábios a memória dos seus nomes.
}\switchcolumn*\latim{
Dóminus pars hereditátis meæ, et cálicis mei: * Tu es, qui restítues hereditátem meam mihi.
}\switchcolumn\portugues{
O Senhor é a porção da minha herança e do meu cálice: * Vós sois quem restituirá a minha herança.
}\switchcolumn*\latim{
Funes cecidérunt mihi in præcláris: * étenim heréditas mea præclára est mihi.
}\switchcolumn\portugues{
Caíram-me as linhas demarcatórias em boa região: * de facto, a minha herança é-me excelente.
}\switchcolumn*\latim{
Benedícam Dóminum, qui tríbuit mihi intelléctum: * ínsuper et usque ad noctem increpuérunt me renes mei.
}\switchcolumn\portugues{
Louvarei o Senhor, que me deu inteligência: * além disto, mesmo durante a noite, acusaram-me as minhas entranhas.
}\switchcolumn*\latim{
Providébam Dóminum in conspéctu meo semper: * quóniam a dextris est mihi, ne commóvear.
}\switchcolumn\portugues{
Contemplava sempre o Senhor ante mim: * porque Ele está à minha direita para que não seja afligido.
}\switchcolumn*\latim{
Propter hoc lætátum est cor meum, et exsultávit lingua mea: * ínsuper et caro mea requiéscet in spe.
}\switchcolumn\portugues{
Alegrou-se, portanto, o meu coração e exultou a minha língua: * também a minha carne repousará na esperança.
}\switchcolumn*\latim{
Quóniam non derelínques ánimam meam in inférno: * nec dabis sanctum tuum vidére corruptiónem.
}\switchcolumn\portugues{
Porque não deixareis a minha alma no inferno: * nem permitireis que o vosso santo veja corrupção.
}\switchcolumn*\latim{
Notas mihi fecísti vias vitæ, adimplébis me lætítia cum vultu tuo: * delectatiónes in déxtera tua usque in finem.
}\switchcolumn\portugues{
Fizestes-me conhecer os caminhos da vida, com vosso rosto encher-me-eis de alegria: * estão delícias eternas à vossa direita.
}\end{paracol}


\subsectioninfo{Salmo 16}{Exaudi, Domine}\label{salmo16}
\begin{paracol}{2}\latim{
\rlettrine{E}{xáudi,} Dómine, justítiam meam: * inténde deprecatiónem meam.
}\switchcolumn\portugues{
\rlettrine{O}{uvi,} ó Senhor, a minha justiça; atendei a minha súplica.
}\switchcolumn*\latim{
Áuribus pércipe oratiónem meam, * non in lábiis dolósis.
}\switchcolumn\portugues{
Chegue aos vossos ouvidos a minha oração, * não com lábios dolosos.
}\switchcolumn*\latim{
De vultu tuo judícium meum pródeat: * óculi tui vídeant æquitátes.
}\switchcolumn\portugues{
De vosso rosto benigno saia a minha sentença: * vejam vossos olhos a justiça.
}\switchcolumn*\latim{
Probásti cor meum, et visitásti nocte: * igne me examinásti, et non est invénta in me iníquitas.
}\switchcolumn\portugues{
Provastes o meu coração e o visitastes de noite: * no fogo me purificastes e não foi encontrada em mim iniquidade.
}\switchcolumn*\latim{
Ut non loquátur os meum ópera hóminum: * propter verba labiórum tuórum ego custodívi vias duras.
}\switchcolumn\portugues{
Para que minha boca não fale as obras dos homens: * por causa das palavras de vossos lábios, mantive caminhos penosos.
}\switchcolumn*\latim{
Pérfice gressus meos in sémitis tuis: * ut non moveántur vestígia mea.
}\switchcolumn\portugues{
Firmai os meus passos nas vossas veredas: * para que meus pés não vacilem.
}\switchcolumn*\latim{
Ego clamávi, quóniam exaudísti me, Deus: * inclína aurem tuam mihi, et exáudi verba mea.
}\switchcolumn\portugues{
Eu clamei, porque me tendes ouvido, ó Deus: * inclinai para mim a vossa orelha e ouvi as minhas palavras.
}\switchcolumn*\latim{
Mirífica misericórdias tuas, * qui salvos facis sperántes in Te.
}\switchcolumn\portugues{
Manifestai as vossas maravilhosas misericórdias, * Vós que salvais aqueles que em Vós esperam.
}\switchcolumn*\latim{
A resisténtibus déxteræ tuæ custódi me, * ut pupíllam óculi.
}\switchcolumn\portugues{
Guardai-me dos que à vossa direita resistem, * como à menina do olho.
}\switchcolumn*\latim{
Sub umbra alárum tuárum prótege me: * a fácie impiórum qui me afflixérunt.
}\switchcolumn\portugues{
Protegei-me à sombra de vossas asas: * da face dos ímpios que me afligem.
}\switchcolumn*\latim{
Inimíci mei ánimam meam circumdedérunt, ádipem suum conclusérunt: * os eórum locútum est supérbiam.
}\switchcolumn\portugues{
Meus inimigos cercaram a minha alma, estão fechados nas suas entranhas: * a sua boca falou com soberba.
}\switchcolumn*\latim{
Proiciéntes me nunc circumdedérunt me: * óculos suos statuérunt declináre in terram.
}\switchcolumn\portugues{
Lançaram-me fora e agora me cercam: * resolveram baixar para a terra os seus olhos.
}\switchcolumn*\latim{
Suscepérunt me sicut leo parátus ad prædam: * et sicut cátulus leónis hábitans in ábditis.
}\switchcolumn\portugues{
Arrebataram-me como um leão preparado para a presa: * e como um jovem leão que habita esconderijos.
}\switchcolumn*\latim{
Exsúrge, Dómine, prævéni eum, et supplánta eum: * éripe ánimam meam ab ímpio, frámeam tuam ab inimícis manus tuæ.
}\switchcolumn\portugues{
Levantai-Vos, ó Senhor, desapontai-o e suplantai-o: * livrai a minha alma do ímpio, vossa espada dos inimigos de vossa mão.
}\switchcolumn*\latim{
Dómine, a paucis de terra dívide eos in vita eórum: * de abscónditis tuis adimplétus est venter eórum.
}\switchcolumn\portugues{
Ó Senhor, separai os bons ainda em vida, que são poucos sobre a terra: * o seu ventre está cheio de vossos tesouros.
}\switchcolumn*\latim{
Saturáti sunt fíliis: * et dimisérunt relíquias suas párvulis suis.
}\switchcolumn\portugues{
Saturados estão de filhos: * e deixam o resto dos seus bens às suas crianças.
}\switchcolumn*\latim{
Ego autem in justítia apparébo conspéctui tuo: * satiábor cum apparúerit glória tua.
}\switchcolumn\portugues{
Eu, porém, comparecerei com justiça na vossa presença: * saciar-me-ei quando aparecer a vossa glória.
}\end{paracol}


\subsectioninfo{Salmo 17}{Diligam Te, Domine}\label{salmo17}
\begin{paracol}{2}\latim{
\rlettrine{D}{íligam} Te, Dómine, fortitúdo mea: * Dóminus firmaméntum meum, et refúgium meum, et liberátor meus.
}\switchcolumn\portugues{
\rlettrine{E}{u} Vos amarei, ó Senhor, minha fortaleza: * o Senhor é o meu firmamento, o meu refúgio e o meu libertador.
}\switchcolumn*\latim{
Deus meus adjútor meus, * et sperábo in eum.
}\switchcolumn\portugues{
Meu Deus é meu auxílio, * e n’Ele esperarei.
}\switchcolumn*\latim{
Protéctor meus, et cornu salútis meæ, * et suscéptor meus.
}\switchcolumn\portugues{
É o meu protector, a minha poderosa salvação * e o meu defensor.
}\switchcolumn*\latim{
Laudans invocábo Dóminum: * et ab inimícis meis salvus ero.
}\switchcolumn\portugues{
Invocarei o Senhor, louvando-o, * e serei salvo dos meus inimigos.
}\switchcolumn*\latim{
Circumdedérunt me dolóres mortis: * et torréntes iniquitátis conturbavérunt me.
}\switchcolumn\portugues{
Cercaram-me dores de morte, * e torrentes de iniquidade me conturbaram.
}\switchcolumn*\latim{
Dolóres inférni circumdedérunt me: * præoccupavérunt me láquei mortis.
}\switchcolumn\portugues{
Dores de inferno me cercaram: * me prenderam laços de morte.
}\switchcolumn*\latim{
In tribulatióne mea invocávi Dóminum, * et ad Deum meum clamávi.
}\switchcolumn\portugues{
Na minha tribulação invoquei o Senhor, * e clamei ao meu Deus.
}\switchcolumn*\latim{
Et exaudívit de templo sancto suo vocem meam: * et clamor meus in conspéctu ejus, introívit in aures ejus.
}\switchcolumn\portugues{
Ele ouviu a minha voz do seu santo templo: * e o clamor, que elevei na sua presença, entrou nos seus ouvidos.
}\switchcolumn*\latim{
Commóta est, et contrémuit terra: * fundaménta móntium conturbáta sunt, et commóta sunt, quóniam irátus est eis.
}\switchcolumn\portugues{
Comoveu-se a terra e tremeu: * os fundamentos dos montes estremeceram e abalaram-se, porque contra eles se indignou.
}\switchcolumn*\latim{
Ascéndit fumus in ira ejus: et ignis a fácie ejus exársit: * carbónes succénsi sunt ab eo.
}\switchcolumn\portugues{
Subiu fumo por causa da sua ira e saiu fogo ardente do seu rosto: * carvões foram por Ele acesos.
}\switchcolumn*\latim{
Inclinávit cælos, et descéndit: * et calígo sub pédibus ejus.
}\switchcolumn\portugues{
Inclinou os céus e desceu: * e a névoa estava sob os seus pés.
}\switchcolumn*\latim{
Et ascéndit super Chérubim, et volávit: * volávit super pennas ventórum.
}\switchcolumn\portugues{
Subiu sobre Querubins e voou: * voou sobre as asas dos ventos.
}\switchcolumn*\latim{
Et pósuit ténebras latíbulum suum, in circúitu ejus tabernáculum ejus: * tenebrósa aqua in núbibus áëris.
}\switchcolumn\portugues{
Fez das trevas o lugar do seu abrigo, à volta da sua tenda cercavam-n’O: * as águas tenebrosas das nuvens do ar.
}\switchcolumn*\latim{
Præ fulgóre in conspéctu ejus nubes transiérunt, * grando et carbónes ignis.
}\switchcolumn\portugues{
Diante do esplendor da sua presença, das nuvens caíram * saraiva e carvões ardentes.
}\switchcolumn*\latim{
Et intónuit de cælo Dóminus, et Altíssimus dedit vocem suam: * grando et carbónes ignis.
}\switchcolumn\portugues{
Dos céus trovejou o Senhor e o Altíssimo fez ouvir sua voz: * saraiva e carvões ardentes.
}\switchcolumn*\latim{
Et misit sagíttas suas, et dissipávit eos: * fúlgura multiplicávit, et conturbávit eos.
}\switchcolumn\portugues{
Enviou as suas setas e desbaratou-os: * multiplicou os relâmpagos e aterrou-os.
}\switchcolumn*\latim{
Et apparuérunt fontes aquárum, * et reveláta sunt fundaménta orbis terrárum:
}\switchcolumn\portugues{
Apareceram as fontes das águas, * e ficaram descobertos os fundamentos da terra:
}\switchcolumn*\latim{
Ab increpatióne tua, Dómine, * ab inspiratióne spíritus iræ tuæ.
}\switchcolumn\portugues{
Devido às vossas ameaças, ó Senhor, * e ao sopro impetuoso de vossa ira.
}\switchcolumn*\latim{
Misit de summo, et accépit me: * et assúmpsit me de aquis multis.
}\switchcolumn\portugues{
Estendeu do alto a sua mão e tomou-me: * e retirou-me de muitas águas.
}\switchcolumn*\latim{
Erípuit me de inimícis meis fortíssimis, et ab his qui odérunt me: * quóniam confortáti sunt super me.
}\switchcolumn\portugues{
Livrou-me dos meus fortíssimos inimigos e dos que me odiavam: * porque eram fortíssimos para mim.
}\switchcolumn*\latim{
Prævenérunt me in die afflictiónis meæ: * et factus est Dóminus protéctor meus.
}\switchcolumn\portugues{
Eles me impediram no dia do meu tormento: * e o Senhor fez-se meu protector.
}\switchcolumn*\latim{
Et edúxit me in latitúdinem: * salvum me fecit, quóniam vóluit me.
}\switchcolumn\portugues{
Retirou-me e pôs-me ao largo: * salvou-me, porque lhe era querido.
}\switchcolumn*\latim{
Et retríbuet mihi Dóminus secúndum justítiam meam: * et secúndum puritátem mánuum meárum retríbuet mihi:
}\switchcolumn\portugues{
O Senhor retribuir-me-á segundo a minha justiça: * e recompensar-me-á segundo a pureza das minhas mãos:
}\switchcolumn*\latim{
Quia custodívi vias Dómini, * nec ímpie gessi a Deo meo.
}\switchcolumn\portugues{
Pois guardei os caminhos do Senhor, * e não procedi impiamente contra o meu Deus.
}\switchcolumn*\latim{
Quóniam ómnia judícia ejus in conspéctu meo: * et justítias ejus non répuli a me.
}\switchcolumn\portugues{
Porque todos seus juízos estão ante mim: * e porque não repeli de mim as suas justiças.
}\switchcolumn*\latim{
Et ero immaculátus cum eo: * et observábo me ab iniquitáte mea.
}\switchcolumn\portugues{
Conservar-me-ei sem mácula diante d’Ele: * e guardar-me-ei da minha iniquidade.
}\switchcolumn*\latim{
Et retríbuet mihi Dóminus secúndum justítiam meam: * et secúndum puritátem mánuum meárum in conspéctu oculórum ejus.
}\switchcolumn\portugues{
O Senhor retribuir-me-á segundo a minha justiça: * e segundo a pureza das minhas mãos ante seus olhos.
}\switchcolumn*\latim{
Cum sancto sanctus eris, * et cum viro innocénte ínnocens eris:
}\switchcolumn\portugues{
Sereis santo com o santo, * e com o homem inocente sereis inocente:
}\switchcolumn*\latim{
Et cum elécto eléctus eris: * et cum pervérso pervertéris.
}\switchcolumn\portugues{
Com o eleito, eleito sereis: * com o perverso sereis perverso.
}\switchcolumn*\latim{
Quóniam Tu pópulum húmilem salvum fácies: * et óculos superbórum humiliábis.
}\switchcolumn\portugues{
Porque salvareis o povo humilde: * e humilhareis os olhos dos soberbos.
}\switchcolumn*\latim{
Quóniam Tu illúminas lucérnam meam, Dómine: * Deus meus, illúmina ténebras meas.
}\switchcolumn\portugues{
Visto que Vós, ó Senhor, iluminais a minha candeia: * esclarecei, meu Deus, as minhas trevas.
}\switchcolumn*\latim{
Quóniam in Te erípiar a tentatióne, * et in Deo meo transgrédiar murum.
}\switchcolumn\portugues{
Porque por Vós sairei livre da tentação, * e com meu Deus passarei a muralha.
}\switchcolumn*\latim{
Deus meus, impollúta via ejus: elóquia Dómini igne examináta: * protéctor est ómnium sperántium in se.
}\switchcolumn\portugues{
Sem mácula é o caminho do meu Deus; as suas palavras são examinadas no fogo: * Ele é o protector de todos os que esperam n’Ele.
}\switchcolumn*\latim{
Quóniam quis Deus præter Dóminum? * Aut quis Deus præter Deum nostrum?
}\switchcolumn\portugues{
Porque, quem é Deus senão o Senhor? * Ou que deus há para além do nosso Deus?
}\switchcolumn*\latim{
Deus, qui præcínxit me virtúte: * et pósuit immaculátam viam meam.
}\switchcolumn\portugues{
Ele é o Deus que me revestiu de força: * e fez que meu caminho fosse imaculado.
}\switchcolumn*\latim{
Qui perfécit pedes meos tamquam cervórum, * et super excélsa státuens me.
}\switchcolumn\portugues{
Que fez os meus pés como os dos veados, * e me estabeleceu sobre lugares altos.
}\switchcolumn*\latim{
Qui docet manus meas ad prǽlium: * et posuísti, ut arcum ǽreum, brácchia mea.
}\switchcolumn\portugues{
Que adestra as minhas mãos para a luta: * e fizestes dos meus braços como um arco de bronze.
}\switchcolumn*\latim{
Et dedísti mihi protectiónem salútis tuæ: * et déxtera tua suscépit me:
}\switchcolumn\portugues{
Destes-me a vossa protecção para me salvar: * e a vossa direita me susteve:
}\switchcolumn*\latim{
Et disciplína tua corréxit me in finem: * et disciplína tua ipsa me docébit.
}\switchcolumn\portugues{
Vossa disciplina corrigiu-me até ao fim: * e essa vossa mesma disciplina ensinar-me-á.
}\switchcolumn*\latim{
Dilatásti gressus meos subtus me: * et non sunt infirmáta vestígia mea:
}\switchcolumn\portugues{
Abristes o caminho sob os meus passos: * e se não enfraqueceram os meus pés:
}\switchcolumn*\latim{
Pérsequar inimícos meos et comprehéndam illos: * et non convértar, donec defíciant.
}\switchcolumn\portugues{
Perseguirei os meus inimigos e apanhá-los-ei: * e não recuarei até que eles acabem.
}\switchcolumn*\latim{
Confríngam illos, nec póterunt stare: * cadent subtus pedes meos.
}\switchcolumn\portugues{
Eu quebrar-lhes-ei as forças, então não conseguirão manter-se em pé: * cairão debaixo dos meus pés.
}\switchcolumn*\latim{
Et præcinxísti me virtúte ad bellum: * et supplantásti insurgéntes in me subtus me.
}\switchcolumn\portugues{
Porque me guarnecestes de força para a guerra: * e suplantastes os insurgentes debaixo de mim.
}\switchcolumn*\latim{
Et inimícos meos dedísti mihi dorsum, * et odiéntes me disperdidísti.
}\switchcolumn\portugues{
Fizestes os meus inimigos me voltarem as costas, * e aniquilastes os que me odiavam.
}\switchcolumn*\latim{
Clamavérunt, nec erat qui salvos fáceret ad Dóminum: * nec exaudívit eos.
}\switchcolumn\portugues{
Gritaram e não havia quem os salvasse para o Senhor: * e Ele os não ouviu.
}\switchcolumn*\latim{
Et commínuam illos, ut púlverem ante fáciem venti: * ut lutum plateárum delébo eos.
}\switchcolumn\portugues{
Os vencerei como o pó atirado ao vento: * os esmagarei como à lama das ruas.
}\switchcolumn*\latim{
Erípies me de contradictiónibus pópuli: * constítues me in caput géntium.
}\switchcolumn\portugues{
Livrar-me-eis das contradições do povo: * estabelecer-me-eis chefe das gentes.
}\switchcolumn*\latim{
Pópulus quem non cognóvi servívit mihi: * in audítu auris obedívit mihi.
}\switchcolumn\portugues{
Um povo que não conhecia me serviu: * ao ouvir a minha voz, foi-me obediente.
}\switchcolumn*\latim{
Fílii aliéni mentíti sunt mihi, * fílii aliéni inveteráti sunt, et claudicavérunt a sémitis suis.
}\switchcolumn\portugues{
Os filhos alheios me mentiram, * os filhos alheios esvaneceram e claudicaram dos seus caminhos.
}\switchcolumn*\latim{
Vivit Dóminus, et benedíctus Deus meus: * et exaltétur Deus salútis meæ.
}\switchcolumn\portugues{
Viva o Senhor e seja bendito o meu Deus: * e seja exaltado o Deus da minha salvação!
}\switchcolumn*\latim{
Deus, qui das vindíctas mihi, et subdis pópulos sub me: * liberátor meus de inimícis meis iracúndis.
}\switchcolumn\portugues{
Deus, que me vingais e que sujeitais os povos debaixo de mim: * que me livrais dos meus inimigos enfurecidos.
}\switchcolumn*\latim{
Et ab insurgéntibus in me exaltábis me: * a viro iníquo erípies me.
}\switchcolumn\portugues{
Elevar-me-eis acima dos que se insurgem contra mim: * livrar-me-eis do homem iníquo.
}\switchcolumn*\latim{
Proptérea confitébor tibi in natiónibus, Dómine: * et nómini tuo psalmum dicam.
}\switchcolumn\portugues{
Por isso eu, ó Senhor, Vos louvarei entre as nações: * e cantarei um salmo ao vosso nome.
}\switchcolumn*\latim{
Magníficans salútes Regis ejus, et fáciens misericórdiam Christo suo David: * et sémini ejus usque in sǽculum.
}\switchcolumn\portugues{
Dando ao seu Rei grandes vitórias, mostrando misericórdia a David seu Ungido: * e com sua descendência por todos os séculos.
}\end{paracol}


\subsectioninfo{Salmo 18}{Cæli enarrant gloriam Dei}\label{salmo18}
\begin{paracol}{2}\latim{
\rlettrine{C}{æli} enárrant glóriam Dei: * et ópera mánuum ejus annúntiat firmaméntum.
}\switchcolumn\portugues{
\rlettrine{O}{s} céus proclamam a glória de Deus: * e o firmamento anuncia a obra das suas mãos.
}\switchcolumn*\latim{
Dies diéi erúctat verbum, * et nox nocti índicat sciéntiam.
}\switchcolumn\portugues{
Um dia transmite ao outro esta mensagem, * e a noite mostra sabedoria a outra noite.
}\switchcolumn*\latim{
Non sunt loquélæ, neque sermónes, * quorum non audiántur voces eórum.
}\switchcolumn\portugues{
Não há discursos nem línguas, * em que não sejam ouvidas suas vozes.
}\switchcolumn*\latim{
In omnem terram exívit sonus eórum: * et in fines orbis terræ verba eórum.
}\switchcolumn\portugues{
Seu eco estendeu-se por toda a terra: * e as suas palavras até aos confins do mundo.
}\switchcolumn*\latim{
In sole pósuit tabernáculum suum: * et ipse tamquam sponsus procédens de thálamo suo:
}\switchcolumn\portugues{
Estabeleceu o seu tabernáculo no sol: * e Ele mesmo é como um esposo que sai do tálamo:
}\switchcolumn*\latim{
Exsultávit ut gigas ad curréndam viam, * a summo cælo egréssio ejus:
}\switchcolumn\portugues{
Dá saltos como gigante para percorrer o seu caminho, * a sua saída é de uma extremidade do céu:
}\switchcolumn*\latim{
Et occúrsus ejus usque ad summum ejus: * nec est qui se abscóndat a calóre ejus.
}\switchcolumn\portugues{
Seu curso vai até à outra extremidade: * e não há quem se esconda do seu calor.
}\switchcolumn*\latim{
Lex Dómini immaculáta, convértens ánimas: * testimónium Dómini fidéle, sapiéntiam præstans párvulis.
}\switchcolumn\portugues{
A lei do Senhor é imaculada, convertendo a alma: * o testemunho do Senhor é fiel, dando sabedoria aos pequeninos.
}\switchcolumn*\latim{
Justítiæ Dómini rectæ, lætificántes corda: * præcéptum Dómini lúcidum, illúminans óculos.
}\switchcolumn\portugues{
As justiças do Senhor são rectas, alegram os corações: * os mandamentos do Senhor são claros, iluminam os olhos.
}\switchcolumn*\latim{
Timor Dómini sanctus, pérmanens in sǽculum sǽculi: * judícia Dómini vera, justificáta in semetípsa.
}\switchcolumn\portugues{
O temor do Senhor é santo, permanece pelos séculos dos séculos: * os juízos do Senhor são verdadeiros, cheios de justiça em si mesmos.
}\switchcolumn*\latim{
Desiderabília super aurum et lápidem pretiósum multum: * et dulcióra super mel et favum.
}\switchcolumn\portugues{
Mais preciosos que o ouro e as muitas pedras preciosas: * e mais doces do que o mel e o favo.
}\switchcolumn*\latim{
Étenim servus tuus custódit ea, * in custodiéndis illis retribútio multa.
}\switchcolumn\portugues{
De facto, o vosso servo os guarda, * e em os guardar há grande recompensa.
}\switchcolumn*\latim{
Delícta quis intéllegit? ab occúltis meis munda me: * et ab aliénis parce servo tuo.
}\switchcolumn\portugues{
Quem os seus delitos conhece? Dos que me são ocultos purificai-me: * e as alheias, perdoai ao vosso servo.
}\switchcolumn*\latim{
Si mei non fúerint domináti, tunc immaculátus ero: * et emundábor a delícto máximo.
}\switchcolumn\portugues{
Se elas me não dominarem, serei imaculado: * e serei purificado dum delito desmedido.
}\switchcolumn*\latim{
Et erunt ut compláceant elóquia oris mei: * et meditátio cordis mei in conspéctu tuo semper.
}\switchcolumn\portugues{
Então as palavras da minha boca ser-Vos-ão agradáveis: * e a meditação do meu coração esteja sempre na vossa presença.
}\switchcolumn*\latim{
Dómine, adjútor meus, * et redémptor meus.
}\switchcolumn\portugues{
Ó Senhor, meu amparo * e meu redentor.
}\end{paracol}


\subsectioninfo{Salmo 19}{Exaudiat te Dominus}\label{salmo19}
\begin{paracol}{2}\latim{
\rlettrine{E}{xáudiat} te Dóminus in die tribulatiónis: * prótegat te nomen Dei Jacob.
}\switchcolumn\portugues{
\rlettrine{O}{} Senhor te ouça no dia da tribulação: * o nome de Deus de Jacob te proteja.
}\switchcolumn*\latim{
Mittat tibi auxílium de sancto: * et de Sion tueátur te.
}\switchcolumn\portugues{
Envie-te auxílio do seu santuário: * e de Sião te proteja.
}\switchcolumn*\latim{
Memor sit omnis sacrifícii tui: * et holocáustum tuum pingue fiat.
}\switchcolumn\portugues{
Tenha presentes todos teus sacrifícios: * e o teu holocausto Lhe seja agradável.
}\switchcolumn*\latim{
Tríbuat tibi secúndum cor tuum: * et omne consílium tuum confírmet.
}\switchcolumn\portugues{
Ele te dê segundo o teu coração: * e cumpra todos teus planos.
}\switchcolumn*\latim{
Lætábimur in salutári tuo: * et in nómine Dei nostri magnificábimur.
}\switchcolumn\portugues{
Alegrar-nos-emos na tua salvação: * e em nome do nosso Deus seremos engrandecidos.
}\switchcolumn*\latim{
Impleat Dóminus omnes petitiónes tuas: * nunc cognóvi quóniam salvum fecit Dóminus Christum suum.
}\switchcolumn\portugues{
Ouça o Senhor todas as tuas petições: * pois sei agora que o Senhor salvou o seu Ungido.
}\switchcolumn*\latim{
Exáudiet illum de cælo sancto suo: * in potentátibus salus déxteræ ejus.
}\switchcolumn\portugues{
Ele ouvi-lo-á do céu, sua santa morada: * em sua poderosa direita está a salvação.
}\switchcolumn*\latim{
Hi in cúrribus, et hi in equis: * nos autem in nómine Dómini, Dei nostri invocábimus.
}\switchcolumn\portugues{
Uns confiam nos carros, outros nos cavalos: * nós, porém, invocaremos o nome do Senhor nosso Deus.
}\switchcolumn*\latim{
Ipsi obligáti sunt, et cecidérunt: * nos autem surréximus et erécti sumus.
}\switchcolumn\portugues{
Eles ficaram atados e caíram: * mas nós nos levantámos e ficámos de pé.
}\switchcolumn*\latim{
Dómine, salvum fac regem: * et exáudi nos in die, qua invocavérimus te.
}\switchcolumn\portugues{
Ó Senhor, salvai o rei: * e ouvi-nos no dia em que Vos invocarmos.
}\end{paracol}


\subsectioninfo{Salmo 20}{Domine, in virtute tua}\label{salmo20}
\begin{paracol}{2}\latim{
\rlettrine{D}{ómine,} in virtúte tua lætábitur rex: * et super salutáre tuum exsultábit veheménter.
}\switchcolumn\portugues{
\rlettrine{S}{enhor,} o rei alegrar-se-á na vossa fortaleza: * e muito regozijará na vossa salvação.
}\switchcolumn*\latim{
Desidérium cordis ejus tribuísti ei: * et voluntáte labiórum ejus non fraudásti eum.
}\switchcolumn\portugues{
Satisfizestes-lhe os anseios do coração: * e não defraudastes os pedidos de seus lábios.
}\switchcolumn*\latim{
Quóniam prævenísti eum in benedictiónibus dulcédinis: * posuísti in cápite ejus corónam de lápide pretióso.
}\switchcolumn\portugues{
Porque o prevenistes com bênçãos de doçura: * cingistes a sua cabeça com uma coroa de pedras preciosas.
}\switchcolumn*\latim{
Vitam pétiit a Te: * et tribuísti ei longitúdinem diérum in sǽculum, et in sǽculum sǽculi.
}\switchcolumn\portugues{
Vida Vos pediu: * e concedestes-lhes largos dias pelos séculos dos séculos.
}\switchcolumn*\latim{
Magna est glória ejus in salutári tuo: * glóriam et magnum decórem impónes super eum.
}\switchcolumn\portugues{
Grande é a sua glória, devido à salvação que lhe destes: * glória e grande esplendor poreis sobre ele.
}\switchcolumn*\latim{
Quóniam dabis eum in benedictiónem in sǽculum sǽculi: * lætificábis eum in gáudio cum vultu tuo.
}\switchcolumn\portugues{
Porque dele fareis uma fonte de bênçãos perpétuas: * enchê-lo-eis de alegria, mostrando-lhe o vosso rosto.
}\switchcolumn*\latim{
Quóniam rex sperat in Dómino: * et in misericórdia Altíssimi non commovébitur.
}\switchcolumn\portugues{
Porque o rei no Senhor espera: * e a misericórdia do Altíssimo torná-lo-á inabalável.
}\switchcolumn*\latim{
Inveniátur manus tua ómnibus inimícis tuis: * déxtera tua invéniat omnes, qui Te odérunt.
}\switchcolumn\portugues{
Caia a vossa mão sobre todos vossos inimigos: * caia a vossa direita sobre todos os que Vos aborrecem.
}\switchcolumn*\latim{
Pones eos ut clíbanum ignis in témpore vultus tui: * Dóminus in ira sua conturbábit eos, et devorábit eos ignis.
}\switchcolumn\portugues{
Os poreis em fornalha acesa ao mostrar-lhes vosso rosto: * o Senhor na sua ira conturbá-los-á e o fogo devorá-los-á.
}\switchcolumn*\latim{
Fructum eórum de terra perdes: * et semen eórum a fíliis hóminum.
}\switchcolumn\portugues{
Exterminareis o seu fruto da terra: * e a sua descendência de entre os filhos dos homens.
}\switchcolumn*\latim{
Quóniam declinavérunt in Te mala: * cogitavérunt consília, quæ non potuérunt stabilíre.
}\switchcolumn\portugues{
Porque urdiram contra Vós males: * formaram planos que não puderam estabelecer.
}\switchcolumn*\latim{
Quóniam pones eos dorsum: * in relíquiis tuis præparábis vultum eórum.
}\switchcolumn\portugues{
Vós, porém, os poreis em fuga: * nos vossos resquícios preparareis o rosto deles.
}\switchcolumn*\latim{
Exaltáre, Dómine, in virtúte tua: * cantábimus et psallémus virtútes tuas.
}\switchcolumn\portugues{
Exaltai-Vos, ó Senhor, no vosso poder: * nós cantaremos e louvaremos as vossas maravilhas.
}\end{paracol}


\subsectioninfo{Salmo 21}{Deus, Deus meus}\label{salmo21}
\begin{paracol}{2}\latim{
\rlettrine{D}{eus,} Deus meus, réspice in me: quare me dereliquísti? * longe a salúte mea verba delictórum meórum.
}\switchcolumn\portugues{
\rlettrine{D}{eus,} ó meu Deus, olhai para mim, porque me abandonastes? * Os clamores dos meus pecados afastam de mim a salvação.
}\switchcolumn*\latim{
Deus meus, clamábo per diem, et non exáudies: * et nocte, et non ad insipiéntiam mihi.
}\switchcolumn\portugues{
Meu Deus, clamarei durante o dia e me não ouvireis: * clamarei de noite e não por minha culpa.
}\switchcolumn*\latim{
Tu autem in sancto hábitas, * laus Israël.
}\switchcolumn\portugues{
Mas Vós morais no lugar santo, * ó glória de Israel.
}\switchcolumn*\latim{
In te speravérunt patres nostri: * speravérunt, et liberásti eos.
}\switchcolumn\portugues{
Em Vós esperaram nossos pais: * esperaram e os libertastes.
}\switchcolumn*\latim{
Ad te clamavérunt, et salvi facti sunt: * in te speravérunt, et non sunt confúsi.
}\switchcolumn\portugues{
A Vós clamaram e foram salvos: * em Vós esperaram e não foram confundidos.
}\switchcolumn*\latim{
Ego autem sum vermis, et non homo: * oppróbrium hóminum, et abjéctio plebis.
}\switchcolumn\portugues{
Eu, porém, sou um verme e não um homem: * opróbio dos homens e abjecção da plebe.
}\switchcolumn*\latim{
Omnes vidéntes me, derisérunt me: * locúti sunt lábiis, et movérunt caput.
}\switchcolumn\portugues{
Todos os que me viram escarneceram de mim: * falaram com os lábios e menearam a cabeça.
}\switchcolumn*\latim{
Sperávit in Dómino, erípiat eum: * salvum fáciat eum, quóniam vult eum.
}\switchcolumn\portugues{
Esperou no Senhor, livre-o: * salve-o, se é que o ama.
}\switchcolumn*\latim{
Quóniam tu es, qui extraxísti me de ventre: * spes mea ab ubéribus matris meæ. In te projéctus sum ex útero:
}\switchcolumn\portugues{
Pois Vós sois quem do ventre me tirou: * minha esperança desde o seio de minha mãe. Fui do útero lançado para Vós:
}\switchcolumn*\latim{
De ventre matris meæ Deus meus es tu, * ne discésseris a me:
}\switchcolumn\portugues{
Vós sois o meu Deus desde o ventre materno, * de mim Vos não retireis:
}\switchcolumn*\latim{
Quóniam tribulátio próxima est: * quóniam non est qui ádjuvet.
}\switchcolumn\portugues{
Porque a tribulação está próxima: * porque não há quem me ajude.
}\switchcolumn*\latim{
Circumdedérunt me vítuli multi: * tauri pingues obsedérunt me.
}\switchcolumn\portugues{
Um grande número de vitelos me cercara: * vi-me sitiado de gordos touros.
}\switchcolumn*\latim{
Aperuérunt super me os suum, * sicut leo rápiens et rúgiens.
}\switchcolumn\portugues{
Abriram sobre mim sua boca, * como um leão arrebatador e que ruge.
}\switchcolumn*\latim{
Sicut aqua effúsus sum: * et dispérsa sunt ómnia ossa mea.
}\switchcolumn\portugues{
Derramei-me como água: * e todos meus ossos se desconjuntaram.
}\switchcolumn*\latim{
Factum est cor meum tamquam cera liquéscens * in médio ventris mei.
}\switchcolumn\portugues{
Meu coração tornou-se como cera derretida * no meio das minhas entranhas.
}\switchcolumn*\latim{
Aruit tamquam testa virtus mea, et lingua mea adhǽsit fáucibus meis: * et in púlverem mortis deduxísti me.
}\switchcolumn\portugues{
Meu vigor secou-se como barro queimado e minha língua pegou-se ao paladar: * e conduzistes-me até ao pó da sepultura.
}\switchcolumn*\latim{
Quóniam circumdedérunt me canes multi: * concílium malignántium obsédit me.
}\switchcolumn\portugues{
Porquanto me rodearam muitos cães raivosos: * uma turba de malignos me assaltou.
}\switchcolumn*\latim{
Fodérunt manus meas et pedes meos: * dinumeravérunt ómnia ossa mea.
}\switchcolumn\portugues{
Traspassaram as minhas mãos e os meus pés: * contaram todos meus ossos.
}\switchcolumn*\latim{
Ipsi vero consideravérunt et inspexérunt me: * divisérunt sibi vestiménta mea, et super vestem meam misérunt sortem.
}\switchcolumn\portugues{
Estiveram-me veramente considerando e olhando: * repartiram entre si as minhas vestes e lançaram sortes sobre a minha túnica.
}\switchcolumn*\latim{
Tu autem, Dómine, ne elongáveris auxílium tuum a me: * ad defensiónem meam cónspice.
}\switchcolumn\portugues{
Mas Vós, ó Senhor, não afasteis de mim o vosso auxílio: * atendei à minha defesa.
}\switchcolumn*\latim{
Erue a frámea, Deus, ánimam meam: * et de manu canis únicam meam:
}\switchcolumn\portugues{
Livrai a minha alma da espada, ó Deus: * e minha única das garras dos cães:
}\switchcolumn*\latim{
Salva me ex ore leónis: * et a córnibus unicórnium humilitátem meam.
}\switchcolumn\portugues{
Salvai-me da boca do leão: * e a minha humildade das hastes dos unicórnios.
}\switchcolumn*\latim{
Narrábo nomen tuum frátribus meis: * in médio ecclésiæ laudábo te.
}\switchcolumn\portugues{
Narrarei o vosso nome aos meus irmãos: * no meio da igreja Vos louvarei.
}\switchcolumn*\latim{
Qui timétis Dóminum, laudáte eum: * univérsum semen Jacob, glorificáte eum.
}\switchcolumn\portugues{
Vós que temeis o Senhor, louvai-O: * vós todos, descendência de Jacób, glorificai-O.
}\switchcolumn*\latim{
Tímeat eum omne semen Israël: * quóniam non sprevit, neque despéxit deprecatiónem páuperis:
}\switchcolumn\portugues{
Tema-O toda a posteridade de Israel: * porque Ele não desprezou nem desatendeu a súplica do pobre:
}\switchcolumn*\latim{
Nec avértit fáciem suam a me: * et cum clamárem ad eum, exaudívit me.
}\switchcolumn\portugues{
Nem escondeu de mim a sua face: * mas me ouviu quando O chamava.
}\switchcolumn*\latim{
Apud te laus mea in ecclésia magna: * vota mea reddam in conspéctu timéntium eum.
}\switchcolumn\portugues{
A Vós dirigir-se-á o meu louvor numa grande igreja: * cumprirei os meus votos em presença dos que O temem.
}\switchcolumn*\latim{
Edent páuperes, et saturabúntur: et laudábunt Dóminum qui requírunt eum: * vivent corda eórum in sǽculum sǽculi.
}\switchcolumn\portugues{
Os pobres comerão e serão saciados: e os que buscam o Senhor louvá-l’O-ão: * os seus corações viverão pelos séculos dos séculos.
}\switchcolumn*\latim{
Reminiscéntur et converténtur ad Dóminum * univérsi fines terræ:
}\switchcolumn\portugues{
Lembrar-se-ão e converter-se-ão ao Senhor * todos os limites da terra:
}\switchcolumn*\latim{
Et adorábunt in conspéctu ejus * univérsæ famíliæ géntium.
}\switchcolumn\portugues{
E adorá-l'O-ão na sua presença * todas as famílias das gentes.
}\switchcolumn*\latim{
Quóniam Dómini est regnum: * et ipse dominábitur géntium.
}\switchcolumn\portugues{
Porque o reino pertence ao Senhor: * e Ele reinará sobre as gentes.
}\switchcolumn*\latim{
Manducavérunt et adoravérunt omnes pingues terræ: * in conspéctu ejus cadent omnes qui descéndunt in terram.
}\switchcolumn\portugues{
Comeram e adoraram todos os ricos da terra: * diante d’Ele se prostraram todos os mortais.
}\switchcolumn*\latim{
Et ánima mea illi vivet: * et semen meum sérviet ipsi.
}\switchcolumn\portugues{
E a minha alma viverá para Ele: * e a minha descendência servi-l'O-á.
}\switchcolumn*\latim{
Annuntiábitur Dómino generátio ventúra: * et annuntiábunt cæli justítiam ejus pópulo qui nascétur, quem fecit Dóminus.
}\switchcolumn\portugues{
A geração vindoura será anunciada ao Senhor: * e o que fez o Senhor, os céus anunciarão a sua justiça ao povo que há-de nascer.
}\end{paracol}


\subsectioninfo{Salmo 22}{Dominus regit me}\label{salmo22}
\begin{paracol}{2}\latim{
\rlettrine{D}{óminus} regit me, et nihil mihi déerit: * in loco páscuæ ibi me collocávit.
}\switchcolumn\portugues{
\rlettrine{O}{} Senhor é meu pastor e nada me faltará: * num lugar de pastos, Ele me colocou.
}\switchcolumn*\latim{
Super aquam refectiónis educávit me: * ánimam meam convértit.
}\switchcolumn\portugues{
Conduziu-me junto a uma água refrescante: * converteu a minha alma.
}\switchcolumn*\latim{
Dedúxit me super sémitas justítiæ, * propter nomen suum.
}\switchcolumn\portugues{
Levou-me por veredas de justiça, * por causa do seu nome.
}\switchcolumn*\latim{
Nam, et si ambulávero in médio umbræ mortis, non timébo mala: * quóniam Tu mecum es.
}\switchcolumn\portugues{
Pois, ainda que ande no meio da sombra da morte, não temerei mal algum: * porque Vós estais comigo.
}\switchcolumn*\latim{
Virga tua, et báculus tuus: * ipsa me consoláta sunt.
}\switchcolumn\portugues{
Vossa vara e o vosso báculo: * me consolaram.
}\switchcolumn*\latim{
Parásti in conspéctu meo mensam, * advérsus eos, qui tríbulant me.
}\switchcolumn\portugues{
Preparastes uma mesa ante mim, * à vista daqueles que me atribulam.
}\switchcolumn*\latim{
Impinguásti in óleo caput meum: * et calix meus inébrians quam præclárus est!
}\switchcolumn\portugues{
Ungistes com óleo a minha cabeça: * e o meu cálice que embriaga, quão precioso é!
}\switchcolumn*\latim{
Et misericórdia tua subsequétur me * ómnibus diébus vitæ meæ:
}\switchcolumn\portugues{
Vossa misericórdia seguir-me-á * todos os dias da minha vida:
}\switchcolumn*\latim{
Et ut inhábitem in domo Dómini, * in longitúdinem diérum.
}\switchcolumn\portugues{
A fim de que habite na casa do Senhor, * durante longos dias.
}\end{paracol}


\subsectioninfo{Salmo 23}{Domini est terra}\label{salmo23}
\begin{paracol}{2}\latim{
\rlettrine{D}{ómini} est terra, et plenitúdo ejus: * orbis terrárum, et univérsi qui hábitant in eo.
}\switchcolumn\portugues{
\rlettrine{D}{o} Senhor é a terra e toda sua plenitude: * a órbita terrestre e quantos nela habitam.
}\switchcolumn*\latim{
Quia ipse super mária fundávit eum: * et super flúmina præparávit eum.
}\switchcolumn\portugues{
Pois Ele a fundou sobre os mares: * e a estabeleceu sobre os rios.
}\switchcolumn*\latim{
Quis ascéndet in montem Dómini? * Aut quis stabit in loco sancto ejus?
}\switchcolumn\portugues{
Quem ao monte do Senhor subirá? * Ou quem no seu lugar santo estará?
}\switchcolumn*\latim{
Ínnocens mánibus et mundo corde, * qui non accépit in vano ánimam suam, nec jurávit in dolo próximo suo.
}\switchcolumn\portugues{
O inocente de mãos e puro de coração, * o que não recebeu em vão sua alma, nem juramentos dolosos fez ao seu próximo.
}\switchcolumn*\latim{
Hic accípiet benedictiónem a Dómino: * et misericórdiam a Deo, salutári suo.
}\switchcolumn\portugues{
Este receberá a bênção do Senhor: * e a misericórdia de Deus, seu Salvador.
}\switchcolumn*\latim{
Hæc est generátio quæréntium eum, * quæréntium fáciem Dei Jacob.
}\switchcolumn\portugues{
Tal é a geração dos que O buscam, * dos que buscam a face do Deus de Jacob.
}\switchcolumn*\latim{
Attóllite portas, príncipes, vestras, et elevámini, portæ æternáles: * et introíbit Rex glóriæ.
}\switchcolumn\portugues{
Levantai, ó príncipes, as vossas portas, levantai-vos, ó portas eternas: * e entrará o Rei da glória.
}\switchcolumn*\latim{
Quis est iste Rex glóriæ? * Dóminus fortis et potens: Dóminus potens in prǽlio.
}\switchcolumn\portugues{
Quem é este Rei da glória? * É o Senhor forte e poderoso, o Senhor poderoso nas batalhas.
}\switchcolumn*\latim{
Attóllite portas, príncipes, vestras, et elevámini, portæ æternáles: * et introíbit Rex glóriæ.
}\switchcolumn\portugues{
Levantai, ó príncipes, as vossas portas, levantai-vos, ó portas eternas: * e entrará o Rei da glória.
}\switchcolumn*\latim{
Quis est iste Rex glóriæ? * Dóminus virtútum ipse est Rex glóriæ.
}\switchcolumn\portugues{
Quem é este Rei da glória? * O Senhor dos exércitos, é Ele o Rei da glória.
}\end{paracol}


\subsectioninfo{Salmo 24}{Ad Te, Domine}\label{salmo24}
\begin{paracol}{2}\latim{
\rlettrine{A}{d} Te, Dómine, levávi ánimam meam: * Deus meus, in Te confído, non erubéscam.
}\switchcolumn\portugues{
\rlettrine{A}{} Vós, ó Senhor, elevei a minha alma: * Deus meu, em Vós confio, não seja eu envergonhado.
}\switchcolumn*\latim{
Neque irrídeant me inimíci mei: * étenim univérsi, qui sústinent Te, non confundéntur.
}\switchcolumn\portugues{
Não me escarneçam os meus inimigos: * de facto, nem um dos que esperam em Vós será confundido.
}\switchcolumn*\latim{
Confundántur omnes iníqua agéntes * supervácue.
}\switchcolumn\portugues{
Sejam confundidos todos os que a iniquidade * cometem em vão.
}\switchcolumn*\latim{
Vias tuas, Dómine, demónstra mihi: * et sémitas tuas édoce me.
}\switchcolumn\portugues{
Mostrai-me, ó Senhor, os vossos caminhos: * e ensinai-me as vossas veredas.
}\switchcolumn*\latim{
Dírige me in veritáte tua, et doce me: * quia Tu es, Deus, salvátor meus, et Te sustínui tota die.
}\switchcolumn\portugues{
Dirigi-me na vossa verdade e ensinai-me: * pois Vós sois Deus, meu Salvador, e esperei em Vós todo o dia.
}\switchcolumn*\latim{
Reminíscere miseratiónum tuárum, Dómine, * et misericordiárum tuárum, quæ a sǽculo sunt.
}\switchcolumn\portugues{
Lembrai-Vos, ó Senhor, de vossas bondades, * e de vossas misericórdias, que datam dos séculos passados.
}\switchcolumn*\latim{
Delícta juventútis meæ, * et ignorántias meas ne memíneris.
}\switchcolumn\portugues{
Dos delitos da minha mocidade, * e das minhas ignorâncias, Vos não recordeis.
}\switchcolumn*\latim{
Secúndum misericórdiam tuam meménto mei Tu: * propter bonitátem tuam, Dómine.
}\switchcolumn\portugues{
De acordo com vossa misericórdia lembrai-Vos de mim: * ó Senhor, segundo a vossa bondade.
}\switchcolumn*\latim{
Dulcis et rectus Dóminus: * propter hoc legem dabit delinquéntibus in via.
}\switchcolumn\portugues{
Doce e recto é o Senhor: * por isso Ele dará lei aos que pecam pelo caminho.
}\switchcolumn*\latim{
Díriget mansuétos in judício: * docébit mites vias suas.
}\switchcolumn\portugues{
Aos mansos conduzirá em justiça: * os seus caminhos ensinará aos humildes.
}\switchcolumn*\latim{
Univérsæ viæ Dómini, misericórdia et véritas, * requiréntibus testaméntum ejus et testimónia ejus.
}\switchcolumn\portugues{
Todos os caminhos do Senhor são misericórdia e verdade, * para os que buscam a sua aliança e os seus mandamentos.
}\switchcolumn*\latim{
Propter nomen tuum, Dómine, propitiáberis peccáto meo: * multum est enim.
}\switchcolumn\portugues{
Por causa de vosso nome, ó Senhor, me haveis de perdoar o meu pecado, * é veramente grande.
}\switchcolumn*\latim{
Quis est homo qui timet Dóminum? * Legem státuit ei in via, quam elégit.
}\switchcolumn\portugues{
Quem é o homem que teme o Senhor? * Fixou-lhe Ele uma lei no caminho que escolheu.
}\switchcolumn*\latim{
Ánima ejus in bonis demorábitur: * et semen ejus hereditábit terram.
}\switchcolumn\portugues{
Sua alma repousará em bens: * e a sua descendência herdará a terra.
}\switchcolumn*\latim{
Firmaméntum est Dóminus timéntibus eum: * et testaméntum ipsíus ut manifestétur illis.
}\switchcolumn\portugues{
O Senhor é o firme apoio dos que O temem: * e manifestar-lhes-á a sua aliança.
}\switchcolumn*\latim{
Óculi mei semper ad Dóminum: * quóniam ipse evéllet de láqueo pedes meos.
}\switchcolumn\portugues{
Meus olhos estão sempre voltados para o Senhor: * porque Ele tirará os meus pés do laço.
}\switchcolumn*\latim{
Réspice in me, et miserére mei: * quia únicus et pauper sum ego.
}\switchcolumn\portugues{
Olhai para mim e tende misericórdia de mim: * pois vejo-me só e pobre.
}\switchcolumn*\latim{
Tribulatiónes cordis mei multiplicátæ sunt: * de necessitátibus meis érue me.
}\switchcolumn\portugues{
As tribulações do meu coração multiplicaram-se: * livrai-me das minhas aflições.
}\switchcolumn*\latim{
Vide humilitátem meam, et labórem meum: * et dimítte univérsa delícta mea.
}\switchcolumn\portugues{
Olhai para o meu abatimento e para o meu trabalho: * e perdoai todos meus pecados.
}\switchcolumn*\latim{
Réspice inimícos meos quóniam multiplicáti sunt, * et ódio iníquo odérunt me.
}\switchcolumn\portugues{
Vede quanto os meus inimigos se têm multiplicado, * e com que ódio iníquo me odeiam.
}\switchcolumn*\latim{
Custódi ánimam meam, et érue me: * non erubéscam quóniam sperávi in Te.
}\switchcolumn\portugues{
Guardai a minha alma e livrai-me: * não seja eu envergonhado, tendo em Vós esperado.
}\switchcolumn*\latim{
Innocéntes et recti adhæsérunt mihi: * quia sustínui Te.
}\switchcolumn\portugues{
Os inocentes e os justos uniram-se comigo: * pois esperei em Vós.
}\switchcolumn*\latim{
Líbera, Deus, Israël, * ex ómnibus tribulatiónibus suis.
}\switchcolumn\portugues{
Livrai Israel, ó Deus, * de todas suas tribulações.
}\end{paracol}


\subsectioninfo{Salmo 25}{Judica me, Domine}\label{salmo25}
\begin{paracol}{2}\latim{
\qlettrine{J}{údica} me, Dómine, quóniam ego in innocéntia mea ingréssus sum: * et in Dómino sperans non infirmábor.
}\switchcolumn\portugues{
\qlettrine{J}{ulgai-me,} ó Senhor, porque andei na minha inocência: * e, esperando no Senhor, não vacilarei.
}\switchcolumn*\latim{
Proba me, Dómine, et tenta me: * ure renes meos et cor meum.
}\switchcolumn\portugues{
Testai-me, ó Senhor, e tentai-me: * purificai-me os rins e o meu coração.
}\switchcolumn*\latim{
Quóniam misericórdia tua ante óculos meos est: * et complácui in veritáte tua.
}\switchcolumn\portugues{
Porque a vossa misericórdia está ante meus olhos: * e com vossa verdade estou satisfeito.
}\switchcolumn*\latim{
Non sedi cum concílio vanitátis: * et cum iníqua geréntibus non introíbo.
}\switchcolumn\portugues{
Não me sentei no concílio da vaidade: * e não entrarei com os que praticam a iniquidade.
}\switchcolumn*\latim{
Odívi ecclésiam malignántium: * et cum ímpiis non sedébo.
}\switchcolumn\portugues{
Odeio a igreja dos malignos: * e me não sentarei com os ímpios.
}\switchcolumn*\latim{
Lavábo inter innocéntes manus meas: * et circúmdabo altáre tuum, Dómine:
}\switchcolumn\portugues{
Lavarei as minhas mãos entre os inocentes: * e estarei, ó Senhor, ao redor de vosso altar.
}\switchcolumn*\latim{
Ut áudiam vocem laudis, * et enárrem univérsa mirabília tua.
}\switchcolumn\portugues{
Para ouvir a voz dos louvores, * e narrar todas vossas maravilhas.
}\switchcolumn*\latim{
Dómine, diléxi decórem domus tuæ, * et locum habitatiónis glóriæ tuæ.
}\switchcolumn\portugues{
Senhor, amei o decoro de vossa casa, * e o lugar onde habita a vossa glória.
}\switchcolumn*\latim{
Ne perdas cum ímpiis, Deus, ánimam meam, * et cum viris sánguinum vitam meam:
}\switchcolumn\portugues{
Não percais, ó Deus, a minha alma com os ímpios, * nem a minha vida com os homens sanguinários.
}\switchcolumn*\latim{
In quorum mánibus iniquitátes sunt: * déxtera eórum repléta est munéribus.
}\switchcolumn\portugues{
Em cujas mãos está a iniquidade: * e cuja dextra está cheia de subornos.
}\switchcolumn*\latim{
Ego autem in innocéntia mea ingréssus sum: * rédime me, et miserére mei.
}\switchcolumn\portugues{
Eu, porém, andei na minha inocência: * salvai-me e tende compaixão de mim.
}\switchcolumn*\latim{
Pes meus stetit in dirécto: * in ecclésiis benedícam Te, Dómine.
}\switchcolumn\portugues{
Meu pé esteve no recto caminho: * nas igrejas Vos bem-direi, ó Senhor.
}\end{paracol}


\subsectioninfo{Salmo 26}{Dominus illuminatio mea}\label{salmo26}
\begin{paracol}{2}\latim{
\rlettrine{D}{óminus} illuminátio mea, et salus mea, * quem timébo?
}\switchcolumn\portugues{
\rlettrine{O}{} Senhor é a minha luz e a minha salvação, * a quem temerei?
}\switchcolumn*\latim{
Dóminus protéctor vitæ meæ, * a quo trepidábo?
}\switchcolumn\portugues{
O Senhor é o defensor da minha vida, * ante quem trepidarei?
}\switchcolumn*\latim{
Dum apprópiant super me nocéntes, * ut edant carnes meas:
}\switchcolumn\portugues{
Enquanto se aproximam de mim os malvados, * para devorar as minhas carnes:
}\switchcolumn*\latim{
Qui tríbulant me inimíci mei, * ipsi infirmáti sunt, et cecidérunt.
}\switchcolumn\portugues{
Meus inimigos que me atribulam, * eles mesmos se debilitaram e caíram.
}\switchcolumn*\latim{
Si consístant advérsum me castra, * non timébit cor meum.
}\switchcolumn\portugues{
Se contra mim exércitos fizerem cerco, * o meu coração não temerá.
}\switchcolumn*\latim{
Si exsúrgat advérsum me prǽlium, * in hoc ego sperábo.
}\switchcolumn\portugues{
Ainda que se levante batalha contra mim, * mesmo assim esperarei.
}\switchcolumn*\latim{
Unam pétii a Dómino, hanc requíram, * ut inhábitem in domo Dómini ómnibus diébus vitæ meæ:
}\switchcolumn\portugues{
Uma cousa só pedi ao Senhor, esta solicitarei, * é que habite na casa do Senhor todos os dias da minha vida:
}\switchcolumn*\latim{
Ut vídeam voluptátem Dómini, * et vísitem templum ejus.
}\switchcolumn\portugues{
Para ver as delícias do Senhor, * e visitar o seu templo.
}\switchcolumn*\latim{
Quóniam abscóndit me in tabernáculo suo: * in die malórum protéxit me in abscóndito tabernáculi sui.
}\switchcolumn\portugues{
Porque me escondeu no seu tabernáculo: * no dia do tormento me protegeu no recôndito do seu tabernáculo.
}\switchcolumn*\latim{
In petra exaltávit me: * et nunc exaltávit caput meum super inimícos meos.
}\switchcolumn\portugues{
Ergueu-me numa pedra: * e agora ergueu a minha cabeça sobre os meus inimigos.
}\switchcolumn*\latim{
Circuívi, et immolávi in tabernáculo ejus hóstiam vociferatiónis: * cantábo, et psalmum dicam Dómino.
}\switchcolumn\portugues{
Circundei e no seu tabernáculo ofereci uma hóstia de júbilo: * cantarei e entoarei um salmo ao Senhor.
}\switchcolumn*\latim{
Exáudi, Dómine, vocem meam, qua clamávi ad Te: * miserére mei, et exáudi me.
}\switchcolumn\portugues{
Ouvi, ó Senhor, a minha voz, com que clamei a Vós: * de mim tende compaixão e ouvi-me.
}\switchcolumn*\latim{
Tibi dixit cor meum, exquisívit Te fácies mea: * fáciem tuam, Dómine, requíram.
}\switchcolumn\portugues{
Meu coração Vos falou, meus olhos Vos buscaram: * Senhor, hei-de procurar o vosso rosto.
}\switchcolumn*\latim{
Ne avértas fáciem tuam a me: * ne declínes in ira a servo tuo.
}\switchcolumn\portugues{
Não escondeis de mim o vosso rosto: * e Vos não retireis com ira de vosso servo.
}\switchcolumn*\latim{
Adjútor meus esto: * ne derelínquas me, neque despícias me, Deus, salutáris meus.
}\switchcolumn\portugues{
Sede a minha ajuda: * me não deixeis, nem me desprezeis, ó Deus meu Salvador.
}\switchcolumn*\latim{
Quóniam pater meus, et mater mea dereliquérunt me: * Dóminus autem assúmpsit me.
}\switchcolumn\portugues{
Porque meu pai e minha mãe me abandonaram: * mas o Senhor me acolheu.
}\switchcolumn*\latim{
Legem pone mihi, Dómine, in via tua: * et dírige me in sémitam rectam propter inimícos meos.
}\switchcolumn\portugues{
Prescreve-me, ó Senhor, uma lei no vosso caminho: * e conduzi-me pela vereda direita, por causa dos meus inimigos.
}\switchcolumn*\latim{
Ne tradíderis me in ánimas tribulántium me: * quóniam insurrexérunt in me testes iníqui, et mentíta est iníquitas sibi.
}\switchcolumn\portugues{
Não me entregueis à mercê das almas que me atribulam: * pois contra mim se levantaram falsas testemunhas, mas a iniquidade mentiu contra si própria.
}\switchcolumn*\latim{
Credo vidére bona Dómini * in terra vivéntium.
}\switchcolumn\portugues{
Creio ver as maravilhas do Senhor * na terra dos viventes.
}\switchcolumn*\latim{
Exspécta Dóminum, viríliter age: * et confortétur cor tuum, et sústine Dóminum.
}\switchcolumn\portugues{
Espera o Senhor, porta-te virilmente: * fortifique-se o teu coração e espera no Senhor.
}\end{paracol}


\subsectioninfo{Salmo 27}{Ad Te, Domine, clamabo}\label{salmo27}
\begin{paracol}{2}\latim{
\rlettrine{A}{d} Te, Dómine, clamábo, Deus meus, ne síleas a me: * nequándo táceas a me, et assimilábor descendéntibus in lacum.
}\switchcolumn\portugues{
\rlettrine{A}{} Vós, ó Senhor, clamarei; Deus meu, não ficais em silêncio comigo: * não suceda que, se me não ouvirdes, seja semelhante àqueles na cova.
}\switchcolumn*\latim{
Exáudi, Dómine, vocem deprecatiónis meæ dum oro ad Te: * dum extóllo manus meas ad templum sanctum tuum.
}\switchcolumn\portugues{
Ouvi, ó Senhor, a voz da minha súplica, quando Vos rogo: * quando ergo as minhas mãos para o vosso santo templo.
}\switchcolumn*\latim{
Ne simul trahas me cum peccatóribus: * et cum operántibus iniquitátem ne perdas me.
}\switchcolumn\portugues{
Não me arrasteis juntamente com os pecadores: * e me não percais com os que praticam a iniquidade.
}\switchcolumn*\latim{
Qui loquúntur pacem cum próximo suo, * mala autem in córdibus eórum.
}\switchcolumn\portugues{
Os quais falam de paz com seu próximo, * e maldade têm em seus corações.
}\switchcolumn*\latim{
Da illis secúndum ópera eórum, * et secúndum nequítiam adinventiónum ipsórum.
}\switchcolumn\portugues{
Dai-lhes segundo as suas obras, * e segundo a malignidade dos seus projectos.
}\switchcolumn*\latim{
Secúndum ópera mánuum eórum tríbue illis: * redde retributiónem eórum ipsis.
}\switchcolumn\portugues{
Dai-lhes segundo as obras das suas mãos: * dai-lhes a recompensa que merecem.
}\switchcolumn*\latim{
Quóniam non intellexérunt ópera Dómini, et in ópera mánuum ejus * déstrues illos, et non ædificábis eos.
}\switchcolumn\portugues{
Porquanto não compreenderam as obras do Senhor, nem as obras das suas mãos; * Vós destruireis e os não restabelecereis.
}\switchcolumn*\latim{
Benedíctus Dóminus: * quóniam exaudívit vocem deprecatiónis meæ.
}\switchcolumn\portugues{
Bendito seja o Senhor: * porque ouviu a voz da minha súplica.
}\switchcolumn*\latim{
Dóminus adjútor meus, et protéctor meus: * in ipso sperávit cor meum, et adjútus sum.
}\switchcolumn\portugues{
O Senhor é a minha ajuda e o meu protector: * n’Ele esperou o meu coração e fui ajudado.
}\switchcolumn*\latim{
Et reflóruit caro mea: * et ex voluntáte mea confitébor ei.
}\switchcolumn\portugues{
Refloresceu a minha carne: * e O louvarei de todo meu coração.
}\switchcolumn*\latim{
Dóminus fortitúdo plebis suæ: * et protéctor salvatiónum Christi sui est.
}\switchcolumn\portugues{
O Senhor é a fortaleza do seu povo: * e o protector que salva o seu Cristo.
}\switchcolumn*\latim{
Salvum fac pópulum tuum, Dómine, et bénedic hereditáti tuæ: * et rege eos, et extólle illos usque in ætérnum.
}\switchcolumn\portugues{
Salvai, ó Senhor, o vosso povo e abençoai a vossa herança: * conduzi-os e exaltai-os por toda a eternidade.
}\end{paracol}


\subsectioninfo{Salmo 28}{Afferte Domino}\label{salmo28}
\begin{paracol}{2}\latim{
\rlettrine{A}{fférte} Dómino, fílii Dei: * afférte Dómino fílios aríetum.
}\switchcolumn\portugues{
\rlettrine{O}{ferecei} ao Senhor, ó filhos de Deus: * oferecei ao Senhor tenros cordeiros.
}\switchcolumn*\latim{
Afférte Dómino glóriam et honórem, afférte Dómino glóriam nómini ejus: * adoráte Dóminum in átrio sancto ejus.
}\switchcolumn\portugues{
Rendei ao Senhor glória e honra, rendei ao Senhor glória ao seu nome: * adorai o Senhor no átrio do seu santuário.
}\switchcolumn*\latim{
Vox Dómini super aquas, Deus majestátis intónuit: * Dóminus super aquas multas.
}\switchcolumn\portugues{
A voz do Senhor está sobre as águas, o Deus da majestade trovejou: * o Senhor está sobre muitas águas.
}\switchcolumn*\latim{
Vox Dómini in virtúte: * vox Dómini in magnificéntia.
}\switchcolumn\portugues{
A voz do Senhor é poderosa: * a voz do Senhor é majestosa.
}\switchcolumn*\latim{
Vox Dómini confringéntis cedros: * et confrínget Dóminus cedros Líbani:
}\switchcolumn\portugues{
A voz do Senhor quebra os cedros: * e o Senhor quebrará os cedros do Líbano:
}\switchcolumn*\latim{
Et commínuet eas tamquam vítulum Líbani: * et diléctus quemádmodum fílius unicórnium.
}\switchcolumn\portugues{
Fá-los-á em pequenos pedaços como a um bezerro do Líbano: * e o bem-amado será como o filho do unicórnio.
}\switchcolumn*\latim{
Vox Dómini intercidéntis flammam ignis: * vox Dómini concutiéntis desértum: et commovébit Dóminus desértum Cades.
}\switchcolumn\portugues{
A voz do Senhor divide as chamas do fogo: * a voz do Senhor abala o deserto e o Senhor fará tremer o deserto de Cades.
}\switchcolumn*\latim{
Vox Dómini præparántis cervos, et revelábit condénsa: * et in templo ejus omnes dicent glóriam.
}\switchcolumn\portugues{
A voz do Senhor prepara os veados e descobre os lugares sombrios: * e no seu templo todos anunciarão a sua glória.
}\switchcolumn*\latim{
Dóminus dilúvium inhabitáre facit: * et sedébit Dóminus Rex in ætérnum.
}\switchcolumn\portugues{
O Senhor faz do dilúvio a sua habitação: * o Senhor sentar-se-á como Rei para sempre.
}\switchcolumn*\latim{
Dóminus virtútem pópulo suo dabit: * Dóminus benedícet pópulo suo in pace.
}\switchcolumn\portugues{
O Senhor dará fortaleza ao seu povo: * o Senhor abençoará o seu povo com paz.
}\end{paracol}


\subsectioninfo{Salmo 29}{Exaltabo Te, Domine}\label{salmo29}
\begin{paracol}{2}\latim{
\rlettrine{E}{xaltábo} Te, Dómine, quóniam suscepísti me: * nec delectásti inimícos meos super me.
}\switchcolumn\portugues{
\rlettrine{E}{u} Vos glorificarei, ó Senhor, porque me recebestes: * e não permitistes que meus inimigos se alegrassem à minha custa.
}\switchcolumn*\latim{
Dómine, Deus meus, clamávi ad Te, * et sanásti me.
}\switchcolumn\portugues{
Ó Senhor meu Deus, clamei a Vós, * e me sarastes.
}\switchcolumn*\latim{
Dómine, eduxísti ab inférno ánimam meam: * salvásti me a descendéntibus in lacum.
}\switchcolumn\portugues{
Senhor, tirastes do inferno a minha alma: * pusestes-me a salvo dos que descem à cova.
}\switchcolumn*\latim{
Psállite Dómino, sancti ejus: * et confitémini memóriæ sanctitátis ejus.
}\switchcolumn\portugues{
Cantai ao Senhor, ó seus santos: * e celebrai a sua santa memória.
}\switchcolumn*\latim{
Quóniam ira in indignatióne ejus: * et vita in voluntáte ejus.
}\switchcolumn\portugues{
Porque a ira está na sua indignação: * e a vida na sua boa vontade.
}\switchcolumn*\latim{
Ad vésperum demorábitur fletus: * et ad matutínum lætítia.
}\switchcolumn\portugues{
De tarde estaremos em lágrimas: * e de manhã em alegria.
}\switchcolumn*\latim{
Ego autem dixi in abundántia mea: * Non movébor in ætérnum.
}\switchcolumn\portugues{
Eu, porém, disse na minha abundância: * jamais serei mudado.
}\switchcolumn*\latim{
Dómine, in voluntáte tua, * præstitísti decóri meo virtútem.
}\switchcolumn\portugues{
Senhor, por vossa vontade, * destes força ao meu decoro.
}\switchcolumn*\latim{
Avertísti fáciem tuam a me, * et factus sum conturbátus.
}\switchcolumn\portugues{
Afastastes de mim a vossa face, * e fiquei conturbado.
}\switchcolumn*\latim{
Ad Te, Dómine, clamábo: * et ad Deum meum deprecábor.
}\switchcolumn\portugues{
A Vós, ó Senhor, clamarei: * e implorarei ao meu Deus.
}\switchcolumn*\latim{
Quæ utílitas in sánguine meo, * dum descéndo in corruptiónem?
}\switchcolumn\portugues{
Que utilidade haverá na minha morte, * enquanto eu à corrupção descer?
}\switchcolumn*\latim{
Numquid confitébitur tibi pulvis, * aut annuntiábit veritátem tuam?
}\switchcolumn\portugues{
Porventura o pó professar-Vos-á * ou anunciará a vossa verdade?
}\switchcolumn*\latim{
Audívit Dóminus, et misértus est mei: * Dóminus factus est adjútor meus.
}\switchcolumn\portugues{
O Senhor me ouviu e teve misericórdia de mim: * o Senhor fez-se meu auxílio.
}\switchcolumn*\latim{
Convertísti planctum meum in gáudium mihi: * conscidísti saccum meum, et circumdedísti me lætítia:
}\switchcolumn\portugues{
Vós convertestes o meu pranto em júbilo: * rasgastes o meu luto e me cercastes de alegria:
}\switchcolumn*\latim{
Ut cantet tibi glória mea, et non compúngar: * Dómine, Deus meus, in ætérnum confitébor tibi.
}\switchcolumn\portugues{
Para que até ao fim a minha glória Vos cante e me não abale: * Ó Senhor meu Deus, Vos louvarei eternamente.
}\end{paracol}


\subsectioninfo{Salmo 30}{In Te, Domine}\label{salmo30}
\begin{paracol}{2}\latim{
\rlettrine{I}{n} Te, Dómine, sperávi non confúndar in ætérnum: * in justítia tua líbera me.
}\switchcolumn\portugues{
\rlettrine{E}{m} Vós esperei, ó Senhor, não permitais que seja jamais confundido: * livrai-me na vossa justiça.
}\switchcolumn*\latim{
Inclína ad me aurem tuam, * accélera ut éruas me.
}\switchcolumn\portugues{
Inclinai para mim os vossos ouvidos, * acudi prontamente a livrar-me.
}\switchcolumn*\latim{
Esto mihi in Deum protectórem, et in domum refúgii: * ut salvum me fácias.
}\switchcolumn\portugues{
Sede para mim um Deus protector e uma casa de refúgio: * para me salvares.
}\switchcolumn*\latim{
Quóniam fortitúdo mea, et refúgium meum es Tu: * et propter nomen tuum dedúces me, et enútries me.
}\switchcolumn\portugues{
Porque Vós sois a minha força e o meu refúgio: * e por causa de vosso nome me conduzíreis e me nutrireis.
}\switchcolumn*\latim{
Edúces me de láqueo hoc, quem abscondérunt mihi: * quóniam Tu es protéctor meus.
}\switchcolumn\portugues{
Tirareis-me deste laço, que esconderam de mim: * porque Vós sois o meu protector.
}\switchcolumn*\latim{
In manus tuas comméndo spíritum meum: * redemísti me, Dómine, Deus veritátis.
}\switchcolumn\portugues{
Em vossas mãos entrego o meu espírito: * me redimistes, Senhor Deus de verdade.
}\switchcolumn*\latim{
Odísti observántes vanitátes, * supervácue.
}\switchcolumn\portugues{
Odieis os que observam cousas vãs * inutilmente.
}\switchcolumn*\latim{
Ego autem in Dómino sperávi: * exsultábo, et lætábor in misericórdia tua.
}\switchcolumn\portugues{
Eu, porém, esperei no Senhor: * exultar-me-ei e alegrar-me-ei na vossa misericórdia.
}\switchcolumn*\latim{
Quóniam respexísti humilitátem meam, * salvásti de necessitátibus ánimam meam.
}\switchcolumn\portugues{
Porque considerastes o meu abatimento, * salvastes das angústias a minha alma.
}\switchcolumn*\latim{
Nec conclusísti me in mánibus inimíci: * statuísti in loco spatióso pedes meos.
}\switchcolumn\portugues{
Não me entregastes nas mãos do inimigo: * antes pusestes os meus pés num terreiro.
}\switchcolumn*\latim{
Miserére mei, Dómine, quóniam tríbulor: * conturbátus est in ira óculus meus, ánima mea, et venter meus:
}\switchcolumn\portugues{
Tende piedade de mim, ó Senhor, porque estou aflito: * conturbados com ira estão meus olhos, minha alma e meu ventre.
}\switchcolumn*\latim{
Quóniam defécit in dolóre vita mea: * et anni mei in gemítibus.
}\switchcolumn\portugues{
Porque a minha vida vai-se consumindo com a mágoa: * e meus anos em gemidos.
}\switchcolumn*\latim{
Infirmáta est in paupertáte virtus mea: * et ossa mea conturbáta sunt.
}\switchcolumn\portugues{
Com pobreza tem-se debilitado a minha força: * e os meus ossos estão abalados.
}\switchcolumn*\latim{
Super omnes inimícos meos factus sum oppróbrium et vicínis meis valde: * et timor notis meis.
}\switchcolumn\portugues{
Mais que todos meus inimigos, tornei-me o escárnio, sobretudo para os meus vizinhos: * e o terror dos meus conhecidos.
}\switchcolumn*\latim{
Qui vidébant me, foras fugérunt a me: * oblivióni datus sum, tamquam mórtuus a corde.
}\switchcolumn\portugues{
Os que me viam, fugiam para longe de mim: * fui esquecido como um morto pelos seus corações.
}\switchcolumn*\latim{
Factus sum tamquam vas pérditum: * quóniam audívi vituperatiónem multórum commorántium in circúitu.
}\switchcolumn\portugues{
Fiquei como um vaso quebrado: * porque no meio deles ouvi as injúrias de muitos.
}\switchcolumn*\latim{
In eo dum convenírent simul advérsum me, * accípere ánimam meam consiliáti sunt.
}\switchcolumn\portugues{
Quando deliberavam juntos contra mim, * resolveram tirar-me a vida.
}\switchcolumn*\latim{
Ego autem in Te sperávi, Dómine: * dixi: Deus meus es Tu: in mánibus tuis sortes meæ.
}\switchcolumn\portugues{
Eu, porém, esperei em Vós, ó Senhor: * disse: o meu Deus sois Vós, nas vossas mãos está o meu fado.
}\switchcolumn*\latim{
Éripe me de manu inimicórum meórum, * et a persequéntibus me.
}\switchcolumn\portugues{
Livrai-me das mãos dos meus inimigos, * e dos que me perseguem.
}\switchcolumn*\latim{
Illústra fáciem tuam super servum tuum, salvum me fac in misericórdia tua: * Dómine, non confúndar, quóniam invocávi Te.
}\switchcolumn\portugues{
Brilhe a claridade de vossa face sobre o vosso servo, salvai-me na vossa misericórdia: * Senhor, não seja confundido, porque Vos invoquei.
}\switchcolumn*\latim{
Erubéscant ímpii, et deducántur in inférnum: * muta fiant lábia dolósa.
}\switchcolumn\portugues{
Os ímpios envergonhem-se e sejam conduzidos ao inferno: * mudos se tornem os lábios dolosos.
}\switchcolumn*\latim{
Quæ loquúntur advérsus justum iniquitátem: * in supérbia, et in abusióne.
}\switchcolumn\portugues{
Que proferem contra o justo palavras de iniquidade: * com soberba e abuso.
}\switchcolumn*\latim{
Quam magna multitúdo dulcédinis tuæ, Dómine, * quam abscondísti timéntibus Te.
}\switchcolumn\portugues{
Quão grande é, ó Senhor, a abundância de vossa doçura, * que tendes escondida para os que Vos temem!
}\switchcolumn*\latim{
Perfecísti eis, qui sperant in Te, * in conspéctu filiórum hóminum.
}\switchcolumn\portugues{
Concedeste-la àqueles que em Vós esperam, * à vista dos filhos dos homens.
}\switchcolumn*\latim{
Abscóndes eos in abscóndito faciéi tuæ * a conturbatióne hóminum.
}\switchcolumn\portugues{
Os escondereis ao abrigo de vossa face contra * as conturbações dos homens.
}\switchcolumn*\latim{
Próteges eos in tabernáculo tuo * a contradictióne linguárum.
}\switchcolumn\portugues{
Os defendereis no vosso tabernáculo * da contradição de suas línguas.
}\switchcolumn*\latim{
Benedíctus Dóminus: * quóniam mirificávit misericórdiam suam mihi in civitáte muníta.
}\switchcolumn\portugues{
Bendito seja o Senhor: * usou maravilhosamente comigo a sua misericórdia numa cidade fortificada.
}\switchcolumn*\latim{
Ego autem dixi in excéssu mentis meæ: * Projéctus sum a fácie oculórum tuórum.
}\switchcolumn\portugues{
Eu, porém, disse no excesso do meu espírito: * fui expulso de ante vossos olhos.
}\switchcolumn*\latim{
Ideo exaudísti vocem oratiónis meæ, * dum clamárem ad Te.
}\switchcolumn\portugues{
Portanto ouvistes a voz da minha oração, * quando a Vós clamava.
}\switchcolumn*\latim{
Dilígite Dóminum omnes sancti ejus: * quóniam veritátem requíret Dóminus, et retríbuet abundánter faciéntibus supérbiam.
}\switchcolumn\portugues{
Amai o Senhor, vós todos seus santos: * porque o Senhor requererá a verdade e severamente retribuirá os que com soberba procedem.
}\switchcolumn*\latim{
Viríliter ágite, et confortétur cor vestrum, * omnes, qui sperátis in Dómino.
}\switchcolumn\portugues{
Portai-vos virilmente e deixei o vosso coração ser fortalecido, * vós todos os que esperais no Senhor.
}\end{paracol}


\subsectioninfo{Salmo 31}{Beati quorum remissæ}\label{salmo31}
\begin{paracol}{2}\latim{
\rlettrine{B}{eáti} quorum remíssæ sunt iniquitátes: * et quorum tecta sunt peccáta.
}\switchcolumn\portugues{
\rlettrine{B}{em-aventurados} aqueles cujas iniquidades foram perdoadas: * e cujos pecados são cobertos.
}\switchcolumn*\latim{
Beátus vir, cui non imputávit Dóminus peccátum, * nec est in spíritu ejus dolus.
}\switchcolumn\portugues{
Bem-aventurado o varão a quem o Senhor não imputou o pecado, * e cujo espírito é isento de dolo.
}\switchcolumn*\latim{
Quóniam tácui, inveteravérunt ossa mea, * dum clamárem tota die.
}\switchcolumn\portugues{
Porque me calei, os meus ossos envelheceram, * enquanto clamava todo o dia.
}\switchcolumn*\latim{
Quóniam die ac nocte graváta est super me manus tua: * convérsus sum in ærúmna mea, dum confígitur spina.
}\switchcolumn\portugues{
Porque a vossa mão tornou-se pesada sobre mim de dia e de noite: * revolvia-me na minha miséria, enquanto a espinha se cravava.
}\switchcolumn*\latim{
Delíctum meum cógnitum tibi feci: * et injustítiam meam non abscóndi.
}\switchcolumn\portugues{
Eu Vos manifestei o meu pecado: * e não ocultei a minha injustiça.
}\switchcolumn*\latim{
Dixi: confitébor advérsum me injustítiam meam Dómino: * et Tu remisísti impietátem peccáti mei.
}\switchcolumn\portugues{
Disse: confessarei contra mim mesmo ao Senhor a minha injustiça: * e Vós perdoastes a impiedade do meu pecado.
}\switchcolumn*\latim{
Pro hac orábit ad Te omnis sanctus, * in témpore opportúno.
}\switchcolumn\portugues{
Por isto orará a Vós todo o santo * no tempo oportuno.
}\switchcolumn*\latim{
Verúmtamen in dilúvio aquárum multárum, * ad eum non approximábunt.
}\switchcolumn\portugues{
E, na inundação das muitas águas, * estas se não aproximarão dele.
}\switchcolumn*\latim{
Tu es refúgium meum a tribulatióne, quæ circúmdedit me: * exsultátio mea, érue me a circumdántibus me.
}\switchcolumn\portugues{
Vós sois o meu refúgio na tribulação que me cercou: * ó alegria minha, livrai-me dos que me cercam.
}\switchcolumn*\latim{
Intelléctum tibi dabo, et ínstruam te in via hac, qua gradiéris: * firmábo super te óculos meos.
}\switchcolumn\portugues{
Inteligência dar-te-ei e ensinar-te-ei o caminho que deves seguir: * fixarei sobre ti os meus olhos.
}\switchcolumn*\latim{
Nolíte fíeri sicut equus et mulus, * quibus non est intelléctus.
}\switchcolumn\portugues{
Não queirais ser como o cavalo e o mulo, * que não têm entendimento.
}\switchcolumn*\latim{
In camo et freno maxíllas eórum constrínge, * qui non appróximant ad Te.
}\switchcolumn\portugues{
Com o cabresto e o freio sujeitai as queixadas, * dos que se não aproximão de Vós.
}\switchcolumn*\latim{
Multa flagélla peccatóris, * sperántem autem in Dómino misericórdia circúmdabit.
}\switchcolumn\portugues{
Muitos flagelos esperam o pecador, * mas o que espera no Senhor será cercado de misericórdia.
}\switchcolumn*\latim{
Lætámini in Dómino et exsultáte, justi, * et gloriámini, omnes recti corde.
}\switchcolumn\portugues{
Ó justos, alegrai-vos no Senhor e exultai-vos, * gloriai vós todos os que sois rectos de coração.
}\end{paracol}


\subsectioninfo{Salmo 32}{Exultate, justi, in Domino}\label{salmo32}
\begin{paracol}{2}\latim{
\rlettrine{E}{xsuláte,} justi, in Dómino: * rectos decet collaudátio.
}\switchcolumn\portugues{
\rlettrine{E}{xultai} no Senhor, ó justos: * aos rectos convém que O louvem.
}\switchcolumn*\latim{
Confitémini Dómino in cíthara: * in psaltério decem chordárum psállite illi.
}\switchcolumn\portugues{
Louvai o Senhor com a cítara: * cantai-Lhe com o saltério de dez cordas.
}\switchcolumn*\latim{
Cantáte ei cánticum novum: * bene psállite ei in vociferatióne.
}\switchcolumn\portugues{
Cantai-Lhe um cântico novo: * cantai-Lhe bem com alta voz.
}\switchcolumn*\latim{
Quia rectum est verbum Dómini, * et ómnia ópera ejus in fide.
}\switchcolumn\portugues{
Pois a palavra do Senhor é recta, * e a sua fidelidade brilha em todas suas obras.
}\switchcolumn*\latim{
Díligit misericórdiam et judícium: * misericórdia Dómini plena est terra.
}\switchcolumn\portugues{
Ele ama a misericórdia e a justiça: * a terra está cheia da misericórdia do Senhor.
}\switchcolumn*\latim{
Verbo Dómini cæli firmáti sunt: * et spíritu oris ejus omnis virtus eórum.
}\switchcolumn\portugues{
Pela palavra do Senhor os céus foram criados: * e todo seu poder pelo espírito da sua boca.
}\switchcolumn*\latim{
Cóngregans sicut in utre aquas maris: * ponens in thesáuris abýssos.
}\switchcolumn\portugues{
Ele junta como num odre as águas do mar: * Ele põe os abysmos nos tesouros.
}\switchcolumn*\latim{
Tímeat Dóminum omnis terra: * ab eo autem commoveántur omnes inhabitántes orbem.
}\switchcolumn\portugues{
Toda a terra tema o Senhor: * e todos os que habitam o universo, tremam diante d’Ele.
}\switchcolumn*\latim{
Quóniam ipse dixit, et facta sunt: * ipse mandávit, et creáta sunt.
}\switchcolumn\portugues{
Porque Ele disse e foi feito: * mandou e foi criado.
}\switchcolumn*\latim{
Dóminus díssipat consília géntium: * réprobat autem cogitatiónes populórum, et réprobat consília príncipum.
}\switchcolumn\portugues{
O Senhor dissipa os conselhos das gentes: * reprova os intentos dos povos e rejeita os conselhos dos príncipes.
}\switchcolumn*\latim{
Consílium autem Dómini in ætérnum manet: * cogitatiónes cordis ejus in generatióne et generatiónem.
}\switchcolumn\portugues{
Porém, os conselhos do Senhor permanecem eternamente: * os intentos do seu coração de geração em geração.
}\switchcolumn*\latim{
Beáta gens, cujus est Dóminus, Deus ejus: * pópulus, quem elégit in hereditátem sibi.
}\switchcolumn\portugues{
Bem-aventurada a nação que tem o Senhor por seu Deus: * o povo que Ele escolheu para sua herança.
}\switchcolumn*\latim{
De cælo respéxit Dóminus: * vidit omnes fílios hóminum.
}\switchcolumn\portugues{
O Senhor olhou do céu: * viu todos os filhos dos homens.
}\switchcolumn*\latim{
De præparáto habitáculo suo * respéxit super omnes, qui hábitant terram.
}\switchcolumn\portugues{
Da morada que Ele preparou para si * olhou sobre todos os que habitam a terra:
}\switchcolumn*\latim{
Qui finxit sigillátim corda eórum: * qui intéllegit ómnia ópera eórum.
}\switchcolumn\portugues{
Foi Ele que formou o coração de cada um deles: * é Ele que conhece todas suas obras.
}\switchcolumn*\latim{
Non salvátur rex per multam virtútem: * et gigas non salvábitur in multitúdine virtútis suæ.
}\switchcolumn\portugues{
Não é pelo seu muito poder que o rei se salva: * nem o gigante se salvará pela sua enormíssima força.
}\switchcolumn*\latim{
Fallax equus ad salútem: * in abundántia autem virtútis suæ non salvábitur.
}\switchcolumn\portugues{
Ilude-se quem do cavalo espera a salvação: * e o não salvará a abundância da sua força.
}\switchcolumn*\latim{
Ecce, óculi Dómini super metuéntes eum: * et in eis, qui sperant super misericórdia ejus:
}\switchcolumn\portugues{
Eis os olhos do Senhor postos sobre os que O temem: * e sobre aqueles que esperam na sua misericórdia:
}\switchcolumn*\latim{
Ut éruat a morte ánimas eórum: * et alat eos in fame.
}\switchcolumn\portugues{
Para livrar da morte as suas almas: * e para os sustentar na fome.
}\switchcolumn*\latim{
Ánima nostra sústinet Dóminum: * quóniam adjútor et protéctor noster est.
}\switchcolumn\portugues{
A nossa alma espera o Senhor: * porque é nosso auxílio e protector.
}\switchcolumn*\latim{
Quia in eo lætábitur cor nostrum: * et in nómine sancto ejus sperávimus.
}\switchcolumn\portugues{
Pois n’Ele alegrar-se-á o nosso coração: * e no seu santo nome temos esperado.
}\switchcolumn*\latim{
Fiat misericórdia tua, Dómine, super nos: * quemádmodum sperávimus in Te.
}\switchcolumn\portugues{
Venha sobre nós, ó Senhor, a vossa misericórdia: * segundo temos esperado em Vós.
}\end{paracol}


\subsectioninfo{Salmo 33}{Benedicam Dominum in omni tempore}\label{salmo33}
\begin{paracol}{2}\latim{
\rlettrine{B}{enedícam} Dóminum in omni témpore: * semper laus ejus in ore meo.
}\switchcolumn\portugues{
\rlettrine{B}{endirei} o Senhor a toda a hora: * o seu louvor estará sempre na minha boca.
}\switchcolumn*\latim{
In Dómino laudábitur ánima mea: * áudiant mansuéti, et læténtur.
}\switchcolumn\portugues{
Minha alma louvar-se-á no Senhor: * ouçam-n’O os mansos e se alegrem.
}\switchcolumn*\latim{
Magnificáte Dóminum mecum: * et exaltémus nomen ejus in idípsum.
}\switchcolumn\portugues{
Comigo engrandecei o Senhor: * e exaltemos juntos o seu nome.
}\switchcolumn*\latim{
Exquisívi Dóminum, et exaudívit me: * et ex ómnibus tribulatiónibus meis erípuit me.
}\switchcolumn\portugues{
Procurei o Senhor e Ele me ouviu: * e me livrou de todas minhas tribulações.
}\switchcolumn*\latim{
Accédite ad eum, et illuminámini: * et fácies vestræ non confundéntur.
}\switchcolumn\portugues{
Aproximai-vos d’Ele e sereis iluminados: * e os vossos rostos não serão confundidos.
}\switchcolumn*\latim{
Iste pauper clamávit, et Dóminus exaudívit eum: * et de ómnibus tribulatiónibus ejus salvávit eum.
}\switchcolumn\portugues{
Este pobre clamou e o Senhor o ouviu: * e o salvou de todas suas tribulações.
}\switchcolumn*\latim{
Immíttet Ángelus Dómini in circúitu timéntium eum: * et erípiet eos.
}\switchcolumn\portugues{
O anjo do Senhor andará à volta dos que O temem: * e resgatá-los-á.
}\switchcolumn*\latim{
Gustáte, et vidéte quóniam suávis est Dóminus: * beátus vir, qui sperat in eo.
}\switchcolumn\portugues{
Provai e vede quão suave é o Senhor: * feliz o varão que n’Ele espera.
}\switchcolumn*\latim{
Timéte Dóminum, omnes sancti ejus: * quóniam non est inópia timéntibus eum.
}\switchcolumn\portugues{
Temei o Senhor, todos seus santos: * porque não há indigência aos que O temem.
}\switchcolumn*\latim{
Dívites eguérunt et esuriérunt: * inquiréntes autem Dóminum non minuéntur omni bono.
}\switchcolumn\portugues{
Os ricos tiveram necessidade e fome: * mas os que buscam o Senhor, não terão falta de bem algum.
}\switchcolumn*\latim{
Veníte, fílii, audíte me: * timórem Dómini docébo vos.
}\switchcolumn\portugues{
Vinde, ó filhos, ouvi-me: * vos ensinarei o temor do Senhor.
}\switchcolumn*\latim{
Quis est homo qui vult vitam: * díligit dies vidére bonos?
}\switchcolumn\portugues{
Quem é o homem que a vida quer: * e que dias felizes deseja ver?
}\switchcolumn*\latim{
Próhibe linguam tuam a malo: * et lábia tua ne loquántur dolum.
}\switchcolumn\portugues{
Guarda a tua língua do mal: * e dolos não espalhem os teus lábios.
}\switchcolumn*\latim{
Divérte a malo, et fac bonum: * inquíre pacem, et perséquere eam.
}\switchcolumn\portugues{
Desvia-te do mal e faz o bem: * busca a paz e persegue-a.
}\switchcolumn*\latim{
Óculi Dómini super justos: * et aures ejus in preces eórum.
}\switchcolumn\portugues{
Os olhos do Senhor estão sobre os justos: * e seus ouvidos nas suas preces.
}\switchcolumn*\latim{
Vultus autem Dómini super faciéntes mala: * ut perdat de terra memóriam eórum.
}\switchcolumn\portugues{
Contudo, o rosto do Senhor está sobre os que fazem o mal: * para apagar da terra a sua memória.
}\switchcolumn*\latim{
Clamavérunt justi, et Dóminus exaudívit eos: * et ex ómnibus tribulatiónibus eórum liberávit eos.
}\switchcolumn\portugues{
Clamaram os justos e o Senhor os ouviu: * e os salvou de todas suas tribulações.
}\switchcolumn*\latim{
Juxta est Dóminus iis, qui tribuláto sunt corde: * et húmiles spíritu salvábit.
}\switchcolumn\portugues{
O Senhor está perto daqueles que têm o coração atribulado: * e salvará os humildes de espírito.
}\switchcolumn*\latim{
Multæ tribulatiónes justórum: * et de ómnibus his liberábit eos Dóminus.
}\switchcolumn\portugues{
Muitas são as tribulações dos justos: * e de todas elas livrá-los-á o Senhor.
}\switchcolumn*\latim{
Custódit Dóminus ómnia ossa eórum: * unum ex his non conterétur.
}\switchcolumn\portugues{
O Senhor guarda todos os ossos deles: * e nem um só se quebrará.
}\switchcolumn*\latim{
Mors peccatórum péssima: * et qui odérunt justum, delínquent.
}\switchcolumn\portugues{
A morte dos pecadores é péssima: * e castigados serão os que ao justo odeiam.
}\switchcolumn*\latim{
Rédimet Dóminus ánimas servórum suórum: * et non delínquent omnes qui sperant in eo.
}\switchcolumn\portugues{
O Senhor resgatará as almas dos seus servos: * e não castigará todos aqueles que n’Ele esperam.
}\end{paracol}


\subsectioninfo{Salmo 34}{Judica, Domine}\label{salmo34}
\begin{paracol}{2}\latim{
\qlettrine{J}{údica,} Dómine, nocéntes me, * expúgna impugnántes me.
}\switchcolumn\portugues{
\qlettrine{J}{ulgai,} ó Senhor, os que me fazem mal, * expugnai os que me combatem.
}\switchcolumn*\latim{
Apprehénde arma et scutum: * et exsúrge in adjutórium mihi.
}\switchcolumn\portugues{
Tomai as vossas armas e o vosso escudo: * e levantai-Vos em meu socorro.
}\switchcolumn*\latim{
Effúnde frámeam, et conclúde advérsus eos, qui persequúntur me: * dic ánimæ meæ: salus tua ego sum.
}\switchcolumn\portugues{
Tirai da espada e cortai a passagem àqueles que me perseguem: * dizei à minha alma: eu sou a tua salvação.
}\switchcolumn*\latim{
Confundántur et revereántur, * quæréntes ánimam meam.
}\switchcolumn\portugues{
Sejam confundidos e envergonhados * os que buscam a minha vida.
}\switchcolumn*\latim{
Avertántur retrórsum, et confundántur * cogitántes mihi mala.
}\switchcolumn\portugues{
Retrocedam e sejam confundidos * os que tramam males contra mim.
}\switchcolumn*\latim{
Fiant tamquam pulvis ante fáciem venti: * et Ángelus Dómini coárctans eos.
}\switchcolumn\portugues{
Tornem-se como o pó levado pelo vento: * e o anjo do Senhor os restrinja.
}\switchcolumn*\latim{
Fiat via illórum ténebræ et lúbricum: * et Ángelus Dómini pérsequens eos.
}\switchcolumn\portugues{
Torne-se o seu caminho tenebroso e escorregadio: * e o anjo do Senhor os persiga.
}\switchcolumn*\latim{
Quóniam gratis abscondérunt mihi intéritum láquei sui: * supervácue exprobravérunt ánimam meam.
}\switchcolumn\portugues{
Porquanto sem causa e para minha ruina eles esconderam um laço: * sem causa insultaram a minha alma.
}\switchcolumn*\latim{
Véniat illi láqueus, quem ignórat: et cáptio, quam abscóndit, apprehéndat eum: * et in láqueum cadat in ipsum.
}\switchcolumn\portugues{
Venha sobre ele a ruina que ignora e a rede que escondeu o prenda a ele: * e caia no próprio laço que armou.
}\switchcolumn*\latim{
Ánima autem mea exsultábit in Dómino: * et delectábitur super salutári suo.
}\switchcolumn\portugues{
Minha alma, porém, exultar-se-á no Senhor: * e porá as suas delícias na sua salvação.
}\switchcolumn*\latim{
Omnia ossa mea dicent: * Dómine, quis símilis tibi?
}\switchcolumn\portugues{
Todos meus ossos dirão: * Senhor, quem a Vós é semelhante?
}\switchcolumn*\latim{
Erípiens ínopem de manu fortiórum ejus: * egénum et páuperem a diripiéntibus eum.
}\switchcolumn\portugues{
Livrais o desvalido das mãos dos mais fortes que ele: * o necessitado e o pobre dos que o roubam.
}\switchcolumn*\latim{
Surgéntes testes iníqui, * quæ ignorábam interrogábant me.
}\switchcolumn\portugues{
Levantaram-se testemunhas iníquas, * me interrogaram sobre o que ignorava.
}\switchcolumn*\latim{
Retribuébant mihi mala pro bonis: * sterilitátem ánimæ meæ.
}\switchcolumn\portugues{
Repagaram-me o bem com o mal: * para a esterilização da minha alma.
}\switchcolumn*\latim{
Ego autem cum mihi molésti essent, * induébar cilício.
}\switchcolumn\portugues{
Eu, porém, quando eles me eram incómodo, * vestia-me de cilício.
}\switchcolumn*\latim{
Humiliábam in jejúnio ánimam meam: * et orátio mea in sinu meo convertétur.
}\switchcolumn\portugues{
Humilhava a minha alma com o jejum: * e a minha oração dava voltas no meu peito.
}\switchcolumn*\latim{
Quasi próximum, et quasi fratrem nostrum, sic complacébam: * quasi lugens et contristátus, sic humiliábar.
}\switchcolumn\portugues{
Como a um próximo e um amigo, assim fazia: * humilhava-me assim como quem está em lamentação e tristeza.
}\switchcolumn*\latim{
Et advérsum me lætáti sunt, et convenérunt: * congregáta sunt super me flagélla, et ignorávi.
}\switchcolumn\portugues{
Alegraram-se e juntaram-se contra mim: * amontoaram-se sobre mim flagelos, que ignorava.
}\switchcolumn*\latim{
Dissipáti sunt, nec compúncti, tentavérunt me, subsannavérunt me subsannatióne: * frenduérunt super me déntibus suis.
}\switchcolumn\portugues{
Foram dissipados, mas se não arrependeram, me tentaram, me insultaram com escárnios: * rangeram contra mim os seus dentes.
}\switchcolumn*\latim{
Dómine, quando respícies? * Restítue ánimam meam a malignitáte eórum, a leónibus únicam meam.
}\switchcolumn\portugues{
Senhor, olhareis até quando? * Resgatai a minha alma da sua malícia: a minha única dos leões.
}\switchcolumn*\latim{
Confitébor tibi in ecclésia magna, * in pópulo gravi laudábo Te.
}\switchcolumn\portugues{
Glorificar-Vos-ei numa grande igreja, * num povo sério Vos louvarei.
}\switchcolumn*\latim{
Non supergáudeant mihi qui adversántur mihi iníque: * qui odérunt me gratis et ánnuunt óculis.
}\switchcolumn\portugues{
Não se regozijem sobre mim os que me atacam injustamente: * os que me odeiam sem causa e piscam os olhos.
}\switchcolumn*\latim{
Quóniam mihi quidem pacífice loquebántur: * et in iracúndia terræ loquéntes, dolos cogitábant.
}\switchcolumn\portugues{
Porque, de facto, me dirigiam palavras de paz: * mas, falando na ira da terra, maquinavam enganos.
}\switchcolumn*\latim{
Et dilatavérunt super me os suum: * dixérunt: euge, euge, vidérunt óculi nostri.
}\switchcolumn\portugues{
Sua boca alargaram contra mim: * e disseram: bem, bem, os nossos olhos viram!
}\switchcolumn*\latim{
Vidísti, Dómine, ne síleas: * Dómine, ne discédas a me.
}\switchcolumn\portugues{
Vós o vistes, ó Senhor, não caleis: * ó Senhor, Vos não aparteis de mim.
}\switchcolumn*\latim{
Exsúrge et inténde judício meo: * Deus meus, et Dóminus meus in causam meam.
}\switchcolumn\portugues{
Levantai-Vos e ao meu julgamento atendei: * à minha causa, Deus meu e Senhor meu.
}\switchcolumn*\latim{
Júdica me secúndum justítiam tuam, Dómine, Deus meus, * et non supergáudeant mihi.
}\switchcolumn\portugues{
Julgai-me segundo a vossa justiça, Senhor Deus meu, * e se não alegrem eles de mim.
}\switchcolumn*\latim{
Non dicant in córdibus suis: euge, euge, ánimæ nostræ: * nec dicant: devorávimus eum.
}\switchcolumn\portugues{
Não digam em seus corações: bem, bem, conseguimos o que desejávamos: * nem digam: nós o devorámos!
}\switchcolumn*\latim{
Erubéscant et revereántur simul, * qui gratulántur malis meis.
}\switchcolumn\portugues{
Fiquem envergonhados e confundidos todos * os que se congratulam dos meus males.
}\switchcolumn*\latim{
Induántur confusióne et reveréntia * qui magna loquúntur super me.
}\switchcolumn\portugues{
Vestidos sejam de confusão e de vergonha * os que falam com orgulho contra mim.
}\switchcolumn*\latim{
Exsúltent et læténtur qui volunt justítiam meam: * et dicant semper: magnificétur Dóminus qui volunt pacem servi ejus.
}\switchcolumn\portugues{
Exultem-se e alegrem-se os que querem a minha justiça: * e digam sempre os que desejam a paz do seu servo: glorificado seja o Senhor.
}\switchcolumn*\latim{
Et lingua mea meditábitur justítiam tuam, * tota die laudem tuam.
}\switchcolumn\portugues{
Minha língua proclamará a vossa justiça, * o vosso louvor todo o dia.
}\end{paracol}


\subsectioninfo{Salmo 35}{Dixit injustus}\label{salmo35}
\begin{paracol}{2}\latim{
\rlettrine{D}{ixit} injústus ut delínquat in semetípso: * non est timor Dei ante óculos ejus.
}\switchcolumn\portugues{
\rlettrine{O}{} injusto disse em si mesmo que pecar queria: * não há temor de Deus ante seus olhos.
}\switchcolumn*\latim{
Quóniam dolóse egit in conspéctu ejus: * ut inveniátur iníquitas ejus ad ódium.
}\switchcolumn\portugues{
Porque procedeu ele enganosamente na sua presença: * e a sua iniquidade mais odiosa se tornou.
}\switchcolumn*\latim{
Verba oris ejus iníquitas, et dolus: * nóluit intellégere ut bene ágeret.
}\switchcolumn\portugues{
As palavras da sua boca são de iniquidade e dolo: * não quis instruir-se para o bem fazer.
}\switchcolumn*\latim{
Iniquitátem meditátus est in cubíli suo: * ástitit omni viæ non bonæ, malítiam autem non odívit.
}\switchcolumn\portugues{
Meditou a iniquidade no seu leito: * deteve-se em todos os maus caminhos, a malícia ele não odiou.
}\switchcolumn*\latim{
Dómine, in cælo misericórdia tua: * et véritas tua usque ad nubes.
}\switchcolumn\portugues{
Senhor, a vossa misericórdia está no céu: * e a vossa verdade eleva-se até às nuvens.
}\switchcolumn*\latim{
Justítia tua sicut montes Dei: * judícia tua abýssus multa.
}\switchcolumn\portugues{
Vossa justiça é como os montes de Deus: * vossos juízos são um abysmo profundo.
}\switchcolumn*\latim{
Hómines, et juménta salvábis, Dómine: * quemádmodum multiplicásti misericórdiam tuam, Deus.
}\switchcolumn\portugues{
Ó Senhor, salvareis homens e animais: * quanto multiplicastes a vossa misericórdia, ó Deus!
}\switchcolumn*\latim{
Fílii autem hóminum, * in tégmine alárum tuárum sperábunt.
}\switchcolumn\portugues{
Por isso os filhos dos homens, * esperarão à sombra de vossas asas.
}\switchcolumn*\latim{
Inebriabúntur ab ubertáte domus tuæ: * et torrénte voluptátis tuæ potábis eos.
}\switchcolumn\portugues{
Inebriar-se-ão com a abundância de vossa casa: * e os fareis beber na torrente de vossas delícias.
}\switchcolumn*\latim{
Quóniam apud Te est fons vitæ: * et in lúmine tuo vidébimus lumen.
}\switchcolumn\portugues{
Porque em Vós está a fonte da vida: * e na vossa luz veremos a luz.
}\switchcolumn*\latim{
Præténde misericórdiam tuam sciéntibus Te, * et justítiam tuam his, qui recto sunt corde.
}\switchcolumn\portugues{
Estendei a vossa misericórdia sobre os que Vos conhecem, * e a vossa justiça sobre aqueles que têm o coração recto.
}\switchcolumn*\latim{
Non véniat mihi pes supérbiæ: * et manus peccatóris non móveat me.
}\switchcolumn\portugues{
Não venha sobre mim o pé do soberbo: * e a mão do pecador me não comova.
}\switchcolumn*\latim{
Ibi cecidérunt qui operántur iniquitátem: * expúlsi sunt, nec potuérunt stare.
}\switchcolumn\portugues{
Ali caíram os que cometem a iniquidade: * foram empurrados e se não puderam mais levantar.
}\end{paracol}


\subsectioninfo{Salmo 36}{Noli æmulari in malignantibus}\label{salmo36}
\begin{paracol}{2}\latim{
\rlettrine{N}{oli} æmulári in malignántibus: * neque zeláveris faciéntes iniquitátem.
}\switchcolumn\portugues{
\rlettrine{N}{ão} imites os malignos: * nem invejes os que obram a iniquidade.
}\switchcolumn*\latim{
Quóniam tamquam fænum velóciter aréscent: * et quemádmodum ólera herbárum cito décident.
}\switchcolumn\portugues{
Porque eles velozmente secarão como feno: * e como as verdes ervas logo murcharão.
}\switchcolumn*\latim{
Spera in Dómino, et fac bonitátem: * et inhábita terram, et pascéris in divítiis ejus.
}\switchcolumn\portugues{
No Senhor espera e faz o bem: * e habitarás na terra e as suas riquezas sustentar-te-ão.
}\switchcolumn*\latim{
Delectáre in Dómino: * et dabit tibi petitiónes cordis tui.
}\switchcolumn\portugues{
Põe as tuas delícias no Senhor: * e Ele dar-te-á as petições de teu coração.
}\switchcolumn*\latim{
Revéla Dómino viam tuam, et spera in eo: * et ipse fáciet.
}\switchcolumn\portugues{
Expõe o teu caminho ao Senhor e n’Ele espera: * e Ele procederá.
}\switchcolumn*\latim{
Et edúcet quasi lumen justítiam tuam: et judícium tuum tamquam merídiem: * súbditus esto Dómino, et ora eum.
}\switchcolumn\portugues{
Fará brilhar como luz a tua justiça e o teu juízo como o meio-dia: * sê obediente ao Senhor e roga-Lhe.
}\switchcolumn*\latim{
Noli æmulári in eo, qui prosperátur in via sua: * in hómine faciénte injustítias.
}\switchcolumn\portugues{
Não invejes o que tem prosperidade no seu caminho: * o homem que comete injustiças.
}\switchcolumn*\latim{
Désine ab ira, et derelínque furórem: * noli æmulári ut malignéris.
}\switchcolumn\portugues{
Guarda-te da ira e deixa a fúria: * não queiras ser rival em vileza.
}\switchcolumn*\latim{
Quóniam qui malignántur, exterminabúntur: * sustinéntes autem Dóminum, ipsi hereditábunt terram.
}\switchcolumn\portugues{
Porque os que cometem maldades serão exterminados: * mas os que esperam no Senhor herdarão a terra.
}\switchcolumn*\latim{
Et adhuc pusíllum, et non erit peccátor: * et quǽres locum ejus et non invénies.
}\switchcolumn\portugues{
Ainda um pouco e não mais existirá o pecador: * e procurarás o seu lugar e o não acharás.
}\switchcolumn*\latim{
Mansuéti autem hereditábunt terram: * et delectabúntur in multitúdine pacis.
}\switchcolumn\portugues{
Porém, os mansos a terra herdarão: * e deleitar-se-ão na abundância da paz.
}\switchcolumn*\latim{
Observábit peccátor justum: * et stridébit super eum déntibus suis.
}\switchcolumn\portugues{
O pecador observará o justo: * e rangerá com os dentes contra ele.
}\switchcolumn*\latim{
Dóminus autem irridébit eum: * quóniam próspicit quod véniet dies ejus.
}\switchcolumn\portugues{
O Senhor, porém, zombará dele: * porque vê que seu dia há-de chegar.
}\switchcolumn*\latim{
Gládium evaginavérunt peccatóres: * intendérunt arcum suum,
}\switchcolumn\portugues{
Os pecadores desembainharam a espada: * estenderam o seu arco,
}\switchcolumn*\latim{
Ut deíciant páuperem et ínopem: * ut trucídent rectos corde.
}\switchcolumn\portugues{
Para arruinarem o pobre e o indigente: * para assassinarem os rectos de coração.
}\switchcolumn*\latim{
Gládius eórum intret in corda ipsórum: * et arcus eórum confringátur.
}\switchcolumn\portugues{
Sua espada trespasse o seu próprio coração: * e seja quebrado o seu arco.
}\switchcolumn*\latim{
Mélius est módicum justo, * super divítias peccatórum multas.
}\switchcolumn\portugues{
Mais vale o pouco do justo, * que as muitas riquezas aos pecadores.
}\switchcolumn*\latim{
Quóniam brácchia peccatórum conteréntur: * confírmat autem justos Dóminus.
}\switchcolumn\portugues{
Porque os braços dos pecadores serão quebrados: * mas o Senhor fortalece os justos.
}\switchcolumn*\latim{
Novit Dóminus dies immaculatórum: * et heréditas eórum in ætérnum erit.
}\switchcolumn\portugues{
O Senhor conhece os dias dos que são imaculados: * e eterna será a herança deles.
}\switchcolumn*\latim{
Non confundéntur in témpore malo, et in diébus famis saturabúntur: * quia peccatóres períbunt.
}\switchcolumn\portugues{
Não serão confundidos no tempo mau e nos dias de fome estarão saciados: * pois os pecadores perecerão.
}\switchcolumn*\latim{
Inimíci vero Dómini mox ut honorificáti fúerint et exaltáti: * deficiéntes, quemádmodum fumus defícient.
}\switchcolumn\portugues{
Os inimigos do Senhor, tanto que tiverem sido honrados e exaltados: * cairão e se desvanecerão como o fumo.
}\switchcolumn*\latim{
Mutuábitur peccátor, et non solvet: * justus autem miserétur et tríbuet.
}\switchcolumn\portugues{
O pecador pedirá emprestado e não pagará: * o justo, porém, doa e é misericordioso.
}\switchcolumn*\latim{
Quia benedicéntes ei hereditábunt terram: * maledicéntes autem ei disperíbunt.
}\switchcolumn\portugues{
Pois os que bendizem a Deus herdarão a terra: * mas os que O maldizem perecerão.
}\switchcolumn*\latim{
Apud Dóminum gressus hóminis dirigéntur: * et viam ejus volet.
}\switchcolumn\portugues{
Os passos do homem serão dirigidos pelo Senhor: * e o seu caminho será aprovado por ele.
}\switchcolumn*\latim{
Cum cecíderit non collidétur: * quia Dóminus suppónit manum suam.
}\switchcolumn\portugues{
Quando cair, se não ferirá: * pois o Senhor lhe põe a mão por baixo.
}\switchcolumn*\latim{
Júnior fui, étenim sénui: * et non vidi justum derelíctum, nec semen ejus quǽrens panem.
}\switchcolumn\portugues{
Jovem fui e sou já velho: * e nunca vi o justo desamparado, nem sua descendência mendigando pão.
}\switchcolumn*\latim{
Tota die miserétur et cómmodat: * et semen illíus in benedictióne erit.
}\switchcolumn\portugues{
Passa o dia sempre misericordioso e dando emprestado: * e a sua descendência será abençoada.
}\switchcolumn*\latim{
Declína a malo, et fac bonum: * et inhábita in sǽculum sǽculi.
}\switchcolumn\portugues{
Desvia-te do mal e faz o bem: * e terás uma eterna morada.
}\switchcolumn*\latim{
Quia Dóminus amat judícium, et non derelínquet sanctos suos: * in ætérnum conservabúntur.
}\switchcolumn\portugues{
Pois o Senhor ama a justiça e não desampara os seus santos: * eles serão conservados eternamente.
}\switchcolumn*\latim{
Injústi puniéntur: * et semen impiórum períbit.
}\switchcolumn\portugues{
Os injustos serão punidos: * e perecerá a descendência dos ímpios.
}\switchcolumn*\latim{
Justi autem hereditábunt terram: * et inhabitábunt in sǽculum sǽculi super eam.
}\switchcolumn\portugues{
Os justos, porém, a terra herdarão: * e habitarão sobre ela por todos os séculos.
}\switchcolumn*\latim{
Os justi meditábitur sapiéntiam, * et lingua ejus loquétur judícium.
}\switchcolumn\portugues{
A boca do justo meditará sabedoria: * e a sua língua falará prudência.
}\switchcolumn*\latim{
Lex Dei ejus in corde ipsíus, * et non supplantabúntur gressus ejus.
}\switchcolumn\portugues{
A lei do seu Deus está no seu coração: * e seus passos não serão suplantados.
}\switchcolumn*\latim{
Consíderat peccátor justum: * et quǽrit mortificáre eum.
}\switchcolumn\portugues{
O pecador observa o justo: * e procura dar-lhe a morte.
}\switchcolumn*\latim{
Dóminus autem non derelínquet eum in mánibus ejus: * nec damnábit eum, cum judicábitur illi.
}\switchcolumn\portugues{
O Senhor, contudo, o não abandonará nas suas mãos: * nem o condenará quando for julgado.
}\switchcolumn*\latim{
Exspécta Dóminum, et custódi viam ejus: et exaltábit te ut hereditáte cápias terram: * cum períerint peccatóres vidébis.
}\switchcolumn\portugues{
Espera no Senhor, guarda o seu caminho e Ele exaltar-te-á para que a terra possuas em herança: * o verás quando perecerem os pecadores.
}\switchcolumn*\latim{
Vidi ímpium superexaltátum, * et elevátum sicut cedros Líbani.
}\switchcolumn\portugues{
Vi o ímpio bastante exaltado, * e elevado como os cedros do Líbano.
}\switchcolumn*\latim{
Et transívi, et ecce non erat: * et quæsívi eum, et non est invéntus locus ejus.
}\switchcolumn\portugues{
Passei e eis que já não existia: * e procurei-o e não encontrei o seu lugar.
}\switchcolumn*\latim{
Custódi innocéntiam, et vide æquitátem: * quóniam sunt relíquiæ hómini pacífico.
}\switchcolumn\portugues{
Guarda a inocência e atende à equidade: * porque ficarão restos para o homem pacífico.
}\switchcolumn*\latim{
Injústi autem disperíbunt simul: * relíquiæ impiórum interíbunt.
}\switchcolumn\portugues{
Os injustos, porém, perecerão igualmente: * o que restar dos ímpios será destruído.
}\switchcolumn*\latim{
Salus autem justórum a Dómino: * et protéctor eórum in témpore tribulatiónis.
}\switchcolumn\portugues{
A salvação dos justos vem do Senhor: * e é Ele o seu protector no tempo da tribulação.
}\switchcolumn*\latim{
Et adjuvábit eos Dóminus et liberábit eos: * et éruet eos a peccatóribus, et salvábit eos: quia speravérunt in eo.
}\switchcolumn\portugues{
O Senhor ajudá-los-á e livrá-los-á: * tirá-los-á da mão dos pecadores e salvá-los-á, pois n’Ele esperam.
}\end{paracol}


\subsectioninfo{Salmo 37}{Domine, ne in furore tuo arguas me}\label{salmo37}
\begin{paracol}{2}\latim{
\rlettrine{D}{ómine,} ne in furóre tuo árguas me, * neque in ira tua corrípias me.
}\switchcolumn\portugues{
\rlettrine{N}{ão} me repreendais, ó Senhor, na vossa indignação, * nem me castigueis na vossa ira.
}\switchcolumn*\latim{
Quóniam sagíttæ tuæ infíxæ sunt mihi: * et confirmásti super me manum tuam.
}\switchcolumn\portugues{
Porque em mim se cravaram as vossas setas: * e sobre mim caiu a vossa mão.
}\switchcolumn*\latim{
Non est sánitas in carne mea a fácie iræ tuæ: * non est pax óssibus meis a fácie peccatórum meórum.
}\switchcolumn\portugues{
Não há parte sã na minha carne devido à vossa ira: * não há paz nos meus ossos, à face dos meus pecados.
}\switchcolumn*\latim{
Quóniam iniquitátes meæ supergréssæ sunt caput meum: * et sicut onus grave gravátæ sunt super me.
}\switchcolumn\portugues{
Porque as minhas iniquidades se elevaram acima da minha cabeça: * e me esmagam como uma pesada carga.
}\switchcolumn*\latim{
Putruérunt et corrúptæ sunt cicatríces meæ, * a fácie insipiéntiæ meæ.
}\switchcolumn\portugues{
Apodreceram e corromperam-se as minhas chagas, * à face da minha ignorância.
}\switchcolumn*\latim{
Miser factus sum, et curvátus sum usque in finem: * tota die contristátus ingrediébar.
}\switchcolumn\portugues{
Tornei-me miserável e totalmente curvado: * todo o dia cheio de tristeza andava.
}\switchcolumn*\latim{
Quóniam lumbi mei impléti sunt illusiónibus: * et non est sánitas in carne mea.
}\switchcolumn\portugues{
Porque as minhas entranhas estão cheias de ilusões: * e não há parte alguma sã na minha carne.
}\switchcolumn*\latim{
Afflíctus sum, et humiliátus sum nimis: * rugiébam a gémitu cordis mei.
}\switchcolumn\portugues{
Estou aflito e sumamente humilhado: * rugi com o gemido do meu coração.
}\switchcolumn*\latim{
Dómine, ante Te omne desidérium meum: * et gémitus meus a Te non est abscónditus.
}\switchcolumn\portugues{
Ó Senhor, bem vedes todos meus desejos: * e o meu gemido Vos não é oculto.
}\switchcolumn*\latim{
Cor meum conturbátum est, derelíquit me virtus mea: * et lumen oculórum meórum, et ipsum non est mecum.
}\switchcolumn\portugues{
Meu coração está abalado, a minha força desamparou-me: * e a própria luz dos meus olhos comigo já não está.
}\switchcolumn*\latim{
Amíci mei, et próximi mei * advérsum me appropinquavérunt, et stetérunt.
}\switchcolumn\portugues{
Meus amigos e meus próximos * avançaram e puseram-se contra mim.
}\switchcolumn*\latim{
Et qui juxta me erant, de longe stetérunt: * et vim faciébant qui quærébant ánimam meam.
}\switchcolumn\portugues{
Meus parentes puseram-se ao longe: * e usavam de violência, os que buscavam a minha vida.
}\switchcolumn*\latim{
Et qui inquirébant mala mihi, locúti sunt vanitátes: * et dolos tota die meditabántur.
}\switchcolumn\portugues{
Os que me procuravam males coisas vãs falaram: * e todo o dia maquinavam enganos.
}\switchcolumn*\latim{
Ego autem tamquam surdus non audiébam: * et sicut mutus non apériens os suum.
}\switchcolumn\portugues{
Eu, porém, como um surdo, não ouvia: * e, como um mudo, não abria a boca.
}\switchcolumn*\latim{
Et factus sum sicut homo non áudiens: * et non habens in ore suo redargutiónes.
}\switchcolumn\portugues{
Tornei-me como um homem surdo: * e que não tem réplica na sua boca.
}\switchcolumn*\latim{
Quóniam in Te, Dómine, sperávi: * Tu exáudies me, Dómine, Deus meus.
}\switchcolumn\portugues{
Porque em Vós, ó Senhor, esperei: * Vós me ouvireis, ó Senhor meu Deus.
}\switchcolumn*\latim{
Quia dixi: nequándo supergáudeant mihi inimíci mei: * et dum commovéntur pedes mei, super me magna locúti sunt.
}\switchcolumn\portugues{
Pois disse: nunca triunfem sobre mim os meus inimigos: * eles que, tendo visto os meus pés vacilantes, falaram de mim insolentemente.
}\switchcolumn*\latim{
Quóniam ego in flagélla parátus sum: * et dolor meus in conspéctu meo semper.
}\switchcolumn\portugues{
Porque estou preparado para o castigo: * e a minha dor está sempre ante mim.
}\switchcolumn*\latim{
Quóniam iniquitátem meam annuntiábo: * et cogitábo pro peccáto meo.
}\switchcolumn\portugues{
Porque confessarei a minha iniquidade: * e pensarei no meu pecado.
}\switchcolumn*\latim{
Inimíci autem mei vivunt, et confirmáti sunt super me: * et multiplicáti sunt qui odérunt me iníque.
}\switchcolumn\portugues{
Meus inimigos vivem e têm-se tornado mais fortes do que eu: * e os que injustamente me odeiam têm-se multiplicado.
}\switchcolumn*\latim{
Qui retríbuunt mala pro bonis, detrahébant mihi: * quóniam sequébar bonitátem.
}\switchcolumn\portugues{
Os que pagam o bem com o mal, desdiziam de mim: * porque a bondade seguia.
}\switchcolumn*\latim{
Ne derelínquas me, Dómine, Deus meus: * ne discésseris a me.
}\switchcolumn\portugues{
Não me desampareis, ó Senhor meu Deus: * de mim Vos não aparteis.
}\switchcolumn*\latim{
Inténde in adjutórium meum, * Dómine, Deus, salútis meæ.
}\switchcolumn\portugues{
Acudi em meu socorro, * ó Senhor Deus da minha salvação.
}\end{paracol}


\subsectioninfo{Salmo 38}{Dixit: custodiam vias meas}\label{salmo38}
\begin{paracol}{2}\latim{
\rlettrine{D}{ixi:} custódiam vias meas: * ut non delínquam in lingua mea.
}\switchcolumn\portugues{
\rlettrine{D}{isse:} meus caminhos velarei: * para que não peque com minha língua.
}\switchcolumn*\latim{
Pósui ori meo custódiam, * cum consísteret peccátor advérsum me.
}\switchcolumn\portugues{
Pus guarda à minha boca, * quando o pecador estava contra mim.
}\switchcolumn*\latim{
Obmútui, et humiliátus sum, et sílui a bonis: * et dolor meus renovátus est.
}\switchcolumn\portugues{
Permaneci mudo, humilhado e mantive silêncio do bem: * e renovou-se a minha dor.
}\switchcolumn*\latim{
Concáluit cor meum intra me: * et in meditatióne mea exardéscet ignis.
}\switchcolumn\portugues{
Dentro de mim ardia o meu coração: * e na minha meditação acendiam-se chamas de fogo.
}\switchcolumn*\latim{
Locútus sum in lingua mea: * Notum fac mihi, Dómine, finem meum.
}\switchcolumn\portugues{
Falei com minha língua: * ó Senhor, fazei-me conhecer o meu fim.
}\switchcolumn*\latim{
Et númerum diérum meórum quis est: * ut sciam quid desit mihi.
}\switchcolumn\portugues{
Qual é o número dos meus dias: * para que saiba o quanto me resta.
}\switchcolumn*\latim{
Ecce mensurábiles posuísti dies meos: * et substántia mea tamquam níhilum ante Te.
}\switchcolumn\portugues{
Eis que pusestes os meus dias em medida: * e ante Vós a minha existência nada é.
}\switchcolumn*\latim{
Verúmtamen univérsa vánitas, * omnis homo vivens.
}\switchcolumn\portugues{
Realmente tudo é vaidade, * todo o homem vivente.
}\switchcolumn*\latim{
Verúmtamen in imágine pertránsit homo: * sed et frustra conturbátur.
}\switchcolumn\portugues{
Certamente que o homem como uma sombra passa: * e em vão se conturba.
}\switchcolumn*\latim{
Thesaurízat: * et ignórat cui congregábit ea.
}\switchcolumn\portugues{
Acumula: * e ignora para quem junta.
}\switchcolumn*\latim{
Et nunc quæ est exspectátio mea? Nonne Dóminus? * Et substántia mea apud Te est.
}\switchcolumn\portugues{
Agora, qual é a minha esperança? A não é o Senhor? * Em Vós está a minha substância.
}\switchcolumn*\latim{
Ab ómnibus iniquitátibus meis érue me: * oppróbrium insipiénti dedísti me.
}\switchcolumn\portugues{
Livrai-me de todas minhas iniquidades: * um objecto de escárnio para o insensato me fizestes.
}\switchcolumn*\latim{
Obmútui, et non apérui os meum, quóniam Tu fecísti: * ámove a me plagas tuas.
}\switchcolumn\portugues{
Calei-me e não abri a minha boca, porque Vós o fizestes: * afastai de mim os vossos flagelos.
}\switchcolumn*\latim{
A fortitúdine manus tuæ ego deféci in increpatiónibus: * propter iniquitátem corripuísti hóminem.
}\switchcolumn\portugues{
Repreendestes-me e debaixo da força de vossa mão desfaleci: * por causa da iniquidade castigastes o homem.
}\switchcolumn*\latim{
Et tabéscere fecísti sicut aráneam ánimam ejus: * verúmtamen vane conturbátur omnis homo.
}\switchcolumn\portugues{
Fizestes que sua vida se consumisse como uma aranha: * é contudo em vão que todo o homem se inquieta.
}\switchcolumn*\latim{
Exáudi oratiónem meam, Dómine, et deprecatiónem meam: * áuribus pércipe lácrimas meas.
}\switchcolumn\portugues{
Senhor, escutai a minha oração e a minha súplica: * atendei às minhas lágrimas.
}\switchcolumn*\latim{
Ne síleas: quóniam ádvena ego sum apud Te, et peregrínus, * sicut omnes patres mei.
}\switchcolumn\portugues{
Não Vos caleis, porque ante Vós eu sou um peregrino, * e um estranho como foram todos meus pais.
}\switchcolumn*\latim{
Remítte mihi, ut refrígerer priúsquam ábeam, * et ámplius non ero.
}\switchcolumn\portugues{
Perdoai-me, para que possa ser refrescado, * antes que parta e deixe de existir.
}\end{paracol}


\subsectioninfo{Salmo 39}{Exspectans exspectavi Dominum}\label{salmo39}
\begin{paracol}{2}\latim{
\rlettrine{E}{xspéctans} exspectávi Dóminum, * et inténdit mihi.
}\switchcolumn\portugues{
\rlettrine{A}{guardei} expectante o Senhor, * e Ele me atendeu.
}\switchcolumn*\latim{
Et exaudívit preces meas: * et edúxit me de lacu misériæ, et de luto fæcis.
}\switchcolumn\portugues{
Ouviu as minhas súplicas: * e tirou-me do abysmo da miséria e do lodo profundo.
}\switchcolumn*\latim{
Et státuit super petram pedes meos: * et diréxit gressus meos.
}\switchcolumn\portugues{
Meus pés pôs sobre pedra: * e dirigiu os meus passos.
}\switchcolumn*\latim{
Et immísit in os meum cánticum novum, * carmen Deo nostro.
}\switchcolumn\portugues{
Um cântico novo pôs na minha boca, * uma canção ao nosso Deus.
}\switchcolumn*\latim{
Vidébunt multi, et timébunt: * et sperábunt in Dómino.
}\switchcolumn\portugues{
Muitos vê-l'O-ão e temerão: * e esperarão no Senhor.
}\switchcolumn*\latim{
Beátus vir, cujus est nomen Dómini spes ejus: * et non respéxit in vanitátes et insánias falsas.
}\switchcolumn\portugues{
Bem-aventurado o varão, cuja esperança é o nome do Senhor: * e que não olhou para vaidades e falsas loucuras.
}\switchcolumn*\latim{
Multa fecísti Tu, Dómine, Deus meus, mirabília tua: * et cogitatiónibus tuis non est qui símilis sit tibi.
}\switchcolumn\portugues{
Ó Senhor meu Deus, tendes feito muitas obras maravilhosas: * e nos vossos desígnios não há quem Vos seja semelhante.
}\switchcolumn*\latim{
Annuntiávi et locútus sum: * multiplicáti sunt super númerum.
}\switchcolumn\portugues{
Quis anunciá-los e falar deles: * é inumerável o seu número.
}\switchcolumn*\latim{
Sacrifícium et oblatiónem noluísti: * aures autem perfecísti mihi.
}\switchcolumn\portugues{
Não quisestes sacrifício nem oblação: * mas ouvidos me formastes.
}\switchcolumn*\latim{
Holocáustum et pro peccáto non postulásti: * tunc dixi: ecce, vénio.
}\switchcolumn\portugues{
Não pedistes holocausto pelo pecado: * então disse: eis que aqui venho.
}\switchcolumn*\latim{
In cápite libri scriptum est de me ut fácerem voluntátem tuam: * Deus meus, vólui, et legem tuam in médio cordis mei.
}\switchcolumn\portugues{
Está escrito de mim na capa do livro, para fazer a vossa vontade: * ó Deus meu, assim o quis e a vossa lei está no íntimo do meu coração.
}\switchcolumn*\latim{
Annuntiávi justítiam tuam in ecclésia magna, * ecce, lábia mea non prohibébo: Dómine, Tu scisti.
}\switchcolumn\portugues{
Anunciei a vossa justiça numa grande igreja, * eis que não fecharei os meus lábios: ó Senhor, Vós o sabeis.
}\switchcolumn*\latim{
Justítiam tuam non abscóndi in corde meo: * veritátem tuam et salutáre tuum dixi.
}\switchcolumn\portugues{
Não escondi a vossa justiça no meu coração: * declarei a vossa verdade e a salvação que vem de Vós.
}\switchcolumn*\latim{
Non abscóndi misericórdiam tuam et veritátem tuam * a concílio multo.
}\switchcolumn\portugues{
Não escondi a vossa misericórdia e a vossa verdade * ao numeroso concílio.
}\switchcolumn*\latim{
Tu autem, Dómine, ne longe fácias miseratiónes tuas a me: * misericórdia tua et véritas tua semper suscepérunt me.
}\switchcolumn\portugues{
Vós, ó Senhor, não afasteis de mim as vossas misericórdias: * a vossa misericórdia e a vossa verdade sempre me ampararam.
}\switchcolumn*\latim{
Quóniam circumdedérunt me mala, quorum non est númerus: * comprehendérunt me iniquitátes meæ, et non pótui ut vidérem.
}\switchcolumn\portugues{
Um sem número de males me cercaram: * me surpreenderam as minhas iniquidades e não pude vê-las.
}\switchcolumn*\latim{
Multiplicátæ sunt super capíllos cápitis mei: * et cor meum derelíquit me.
}\switchcolumn\portugues{
Multiplicaram-se mais do que os cabelos da minha cabeça: * e o meu coração desfaleceu.
}\switchcolumn*\latim{
Compláceat tibi, Dómine, ut éruas me: * Dómine, ad adjuvándum me réspice.
}\switchcolumn\portugues{
Seja de vosso agrado me livrardes, ó Senhor: * ó Senhor, voltai os olhos para me socorrerdes.
}\switchcolumn*\latim{
Confundántur et revereántur simul, qui quærunt ánimam meam, * ut áuferant eam.
}\switchcolumn\portugues{
Simultaneamente sejam confundidos e envergonhados, os que minha vida * procuram tirar.
}\switchcolumn*\latim{
Convertántur retrórsum, et revereántur, * qui volunt mihi mala.
}\switchcolumn\portugues{
Recuem e fiquem confundidos, * os que me desejam males.
}\switchcolumn*\latim{
Ferant conféstim confusiónem suam, * qui dicunt mihi: euge, euge.
}\switchcolumn\portugues{
Sofram imediatamente a sua confusão, * aqueles que me dizem: bem, bem!
}\switchcolumn*\latim{
Exsúltent et læténtur super Te omnes quæréntes Te: * et dicant semper: magnificétur Dóminus: qui díligunt salutáre tuum.
}\switchcolumn\portugues{
Regozijem-se e alegrem-se em Vós todos os que Vos buscam: * e os que amam a vossa salvação digam sempre: o Senhor seja glorificado.
}\switchcolumn*\latim{
Ego autem mendícus sum, et pauper: * Dóminus sollícitus est mei.
}\switchcolumn\portugues{
Quanto a mim sou mendigo e pobre: * o Senhor, porém, de mim tem cuidado.
}\switchcolumn*\latim{
Adjútor meus, et protéctor meus Tu es: * Deus meus, ne tardáveris.
}\switchcolumn\portugues{
Vós sois o meu auxílio e o meu protector: * não tardeis, ó meu Deus.
}\end{paracol}


\subsectioninfo{Salmo 40}{Beatus qui intelligit}\label{salmo40}
\begin{paracol}{2}\latim{
\rlettrine{B}{eátus} qui intéllegit super egénum, et páuperem: * in die mala liberábit eum Dóminus.
}\switchcolumn\portugues{
\rlettrine{B}{em-aventurado} o que tem em consideração o necessitado e o pobre: * no mau dia livrá-lo-á o Senhor.
}\switchcolumn*\latim{
Dóminus consérvet eum, et vivíficet eum, et beátum fáciat eum in terra: * et non tradat eum in ánimam inimicórum ejus.
}\switchcolumn\portugues{
O guarde o Senhor e lhe dê vida e o faça feliz na terra: * e o não entregue ao poder dos seus inimigos.
}\switchcolumn*\latim{
Dóminus opem ferat illi super lectum dolóris ejus: * univérsum stratum ejus versásti in infirmitáte ejus.
}\switchcolumn\portugues{
O Senhor lhe dê auxílio sobre o leito da sua dor: * na doença revirastes toda sua cama.
}\switchcolumn*\latim{
Ego dixi: Dómine, miserére mei: * sana ánimam meam, quia peccávi tibi.
}\switchcolumn\portugues{
Eu disse: ó Senhor, compadecei-Vos de mim: * sarai a minha alma, pois pequei contra Vós.
}\switchcolumn*\latim{
Inimíci mei dixérunt mala mihi: * Quando moriétur, et períbit nomen ejus?
}\switchcolumn\portugues{
Maldades os meus inimigos falaram contra mim: * quando morrerá e perecerá o seu nome?
}\switchcolumn*\latim{
Et si ingrediebátur ut vidéret, vana loquebátur: * cor ejus congregávit iniquitátem sibi.
}\switchcolumn\portugues{
E, se entrava para me ver, diria vãs cousas: * o seu coração acumulava em si a iniquidade.
}\switchcolumn*\latim{
Egrediebátur foras, * et loquebátur in idípsum.
}\switchcolumn\portugues{
Ele saía para fora, * e falava para o mesmo fim.
}\switchcolumn*\latim{
Advérsum me susurrábant omnes inimíci mei: * advérsum me cogitábant mala mihi.
}\switchcolumn\portugues{
Murmuravam contra mim todos meus inimigos: * teciam males contra mim.
}\switchcolumn*\latim{
Verbum iníquum constituérunt advérsum me: * Numquid qui dormit non adíciet ut resúrgat?
}\switchcolumn\portugues{
Decretaram contra mim uma injusta palavra: * o que dorme não poderá porventura volver a erguer-se?
}\switchcolumn*\latim{
Étenim homo pacis meæ, in quo sperávi: * qui edébat panes meos, magnificávit super me supplantatiónem.
}\switchcolumn\portugues{
De facto, o homem da minha paz, em quem esperei: * que comia o meu pão, engrandeceu contra mim a sua traição.
}\switchcolumn*\latim{
Tu autem, Dómine, miserére mei, et resúscita me: * et retríbuam eis.
}\switchcolumn\portugues{
Vós, porém, ó Senhor, tende compaixão de mim e elevai-me: * e lhes retribuirei.
}\switchcolumn*\latim{
In hoc cognóvi quóniam voluísti me: * quóniam non gaudébit inimícus meus super me.
}\switchcolumn\portugues{
Nisto conhecerei que Vós me quereis bem: * porque sobre mim o meu inimigo se não alegrará.
}\switchcolumn*\latim{
Me autem propter innocéntiam suscepísti: * et confirmásti me in conspéctu tuo in ætérnum.
}\switchcolumn\portugues{
Porque Vós me suportastes por causa da minha inocência: * e me fortificastes ante Vós para sempre.
}\switchcolumn*\latim{
Benedíctus Dóminus, Deus Israël, a sǽculo et usque in sǽculum: * fiat, fiat.
}\switchcolumn\portugues{
Seja bendito o Senhor Deus de Israel por todos os séculos dos séculos: * assim seja, assim seja.
}\end{paracol}


\subsectioninfo{Salmo 41}{Quemadmodum desiderat cervus}\label{salmo41}
\begin{paracol}{2}\latim{
\qlettrine{Q}{uemádmodum} desíderat cervus ad fontes aquárum: * ita desíderat ánima mea ad Te, Deus.
}\switchcolumn\portugues{
\rlettrine{A}{ssim} como o veado suspira pelas fontes das águas: * assim por Vós suspira a minha alma, ó Deus.
}\switchcolumn*\latim{
Sitívit ánima mea ad Deum fortem vivum: * quando véniam, et apparébo ante fáciem Dei?
}\switchcolumn\portugues{
Minha alma tem sede do Deus forte e vivo: * quando irei e aparecerei ante a face de Deus?
}\switchcolumn*\latim{
Fuérunt mihi lácrimæ meæ panes die ac nocte: * dum dícitur mihi quotídie: ubi est Deus tuus?
}\switchcolumn\portugues{
Noite e dia foram as minhas lágrimas o meu pão: * enquanto todos os dias me dizem: onde está o teu Deus?
}\switchcolumn*\latim{
Hæc recordátus sum, et effúdi in me ánimam meam: * quóniam transíbo in locum tabernáculi admirábilis, usque ad domum Dei.
}\switchcolumn\portugues{
Lembrei-me destas coisas e dentro de mim mesmo derramei a minha alma: * porque irei ao lugar do admirável tabernáculo, até à casa de Deus.
}\switchcolumn*\latim{
In voce exsultatiónis, et confessiónis: * sonus epulántis.
}\switchcolumn\portugues{
Entre vozes de alegria e louvor: * o ruído dum festim.
}\switchcolumn*\latim{
Quare tristis es, ánima mea? * Et quare contúrbas me?
}\switchcolumn\portugues{
Porque estás triste, alma minha? * E porque me conturbas?
}\switchcolumn*\latim{
Spera in Deo, quóniam adhuc confitébor illi: * salutáre vultus mei, et Deus meus.
}\switchcolumn\portugues{
Espera em Deus, porque ainda O hei-de louvar: * a Ele que é a salvação da minha face e meu Deus.
}\switchcolumn*\latim{
Ad meípsum ánima mea conturbáta est: * proptérea memor ero tui de terra Jordánis, et Hermóniim a monte módico.
}\switchcolumn\portugues{
Minha alma está abalada dentro de mim mesmo: * portanto lembrei-me de Vós, na terra do Jordão e de Hermon e desde o pequeno monte.
}\switchcolumn*\latim{
Abýssus abýssum ínvocat, * in voce cataractárum tuárum.
}\switchcolumn\portugues{
Abismo atrai abismo, * à voz de vossas cataratas.
}\switchcolumn*\latim{
Omnia excélsa tua, et fluctus tui * super me transiérunt.
}\switchcolumn\portugues{
Todas vossas vagas e vossas ondas * passaram sobre mim.
}\switchcolumn*\latim{
In die mandávit Dóminus misericórdiam suam: * et nocte cánticum ejus.
}\switchcolumn\portugues{
Durante o dia enviou o Senhor a sua misericórdia: * e de noite o seu cântico.
}\switchcolumn*\latim{
Apud me orátio Deo vitæ meæ, * dicam Deo: suscéptor meus es.
}\switchcolumn\portugues{
Orarei dentro de mim ao Deus da minha vida: * direi a Deus: sois o meu protector.
}\switchcolumn*\latim{
Quare oblítus es mei? * Et quare contristátus incédo, dum afflígit me inimícus?
}\switchcolumn\portugues{
Porque de mim Vos esquecestes? * E porque hei-de andar triste, enquanto o inimigo me aflige?
}\switchcolumn*\latim{
Dum confringúntur ossa mea, * exprobravérunt mihi qui tríbulant me inimíci mei.
}\switchcolumn\portugues{
Enquanto os meus ossos são quebrados, * insultam-me os meus inimigos que me atribulam.
}\switchcolumn*\latim{
Dum dicunt mihi per síngulos dies: ubi est Deus tuus? * Quare tristis es, ánima mea? et quare contúrbas me?
}\switchcolumn\portugues{
Dizendo-me todos os dias: o teu Deus onde está? * Porque triste estás, alma minha? E porque me conturbas?
}\switchcolumn*\latim{
Spera in Deo, quóniam adhuc confitébor illi: * salutáre vultus mei, et Deus meus.
}\switchcolumn\portugues{
Espera em Deus, porque O ainda hei-de louvar: * a Ele que é a salvação do meu rosto e o meu Deus.
}\end{paracol}


\subsectioninfo{Salmo 42}{Judica me, Deus}\label{salmo42}
\begin{paracol}{2}\latim{
\qlettrine{J}{údica} me, Deus, et discérne causam meam de gente non sancta, * ab hómine iníquo, et dolóso érue me.
}\switchcolumn\portugues{
\qlettrine{J}{ulgai-me,} ó Deus, e defendei a minha causa da gente infiel, * livrai-me do homem iníquo e ardiloso.
}\switchcolumn*\latim{
Quia Tu es, Deus, fortitúdo mea: * quare me repulísti? et quare tristis incédo, dum afflígit me inimícus?
}\switchcolumn\portugues{
Pois Vós sois a minha fortaleza, ó Deus: * porque me repelistes? E porque hei-de andar triste, enquanto me aflige o inimigo?
}\switchcolumn*\latim{
Emítte lucem tuam et veritátem tuam: * ipsa me deduxérunt, et adduxérunt in montem sanctum tuum, et in tabernácula tua.
}\switchcolumn\portugues{
Enviai a vossa luz e a vossa verdade: * elas me conduziram e me levaram ao vosso santo monte e aos vossos tabernáculos.
}\switchcolumn*\latim{
Et introíbo ad altáre Dei: * ad Deum, qui lætíficat juventútem meam.
}\switchcolumn\portugues{
Irei até ao Altar de Deus: * até Deus, que é a alegria da minha juventude.
}\switchcolumn*\latim{
Confitébor tibi in cíthara, Deus, Deus meus: * quare tristis es, ánima mea? et quare contúrbas me?
}\switchcolumn\portugues{
Ó Deus, ó meu Deus, louvar-Vos-ei com a cítara: * porque estás triste, alma minha? E porque me conturbas?
}\switchcolumn*\latim{
Spera in Deo, quóniam adhuc confitébor illi: * salutáre vultus mei, et Deus meus.
}\switchcolumn\portugues{
Confia em Deus, porque ainda O louvarei: * a Ele que é a salvação do meu rosto e o meu Deus.
}\end{paracol}


\subsectioninfo{Salmo 43}{Deus, auribus nostris audivimus}\label{salmo43}
\begin{paracol}{2}\latim{
\rlettrine{D}{eus,} áuribus nostris audívimus: * patres nostri annuntiavérunt nobis.
}\switchcolumn\portugues{
\rlettrine{N}{ós} ouvimos, ó Deus, com os nossos próprios ouvidos: * nossos pais nos anunciaram.
}\switchcolumn*\latim{
Opus, quod operátus es in diébus eórum, * et in diébus antíquis.
}\switchcolumn\portugues{
A obra que fizestes nos seus dias, * e nos antigos dias.
}\switchcolumn*\latim{
Manus tua gentes dispérdidit, et plantásti eos: * afflixísti pópulos, et expulísti eos.
}\switchcolumn\portugues{
Plantaste-os a eles e a vossa mão exterminou as gentes: * afligistes aqueles povos e os expelistes.
}\switchcolumn*\latim{
Nec enim in gládio suo possedérunt terram, * et brácchium eórum non salvávit eos:
}\switchcolumn\portugues{
Porque não foi com sua espada que conquistaram esta terra, * e não foi o seu braço que os salvou:
}\switchcolumn*\latim{
Sed déxtera tua, et brácchium tuum, et illuminátio vultus tui: * quóniam complacuísti in eis.
}\switchcolumn\portugues{
Senão a vossa dextra, o vosso braço e a luz de vosso rosto: * porque com eles Vos agradastes.
}\switchcolumn*\latim{
Tu es ipse Rex meus et Deus meus: * qui mandas salútes Jacob.
}\switchcolumn\portugues{
Vós mesmo sois o meu Rei e o meu Deus: * que destes a salvação a Jacob.
}\switchcolumn*\latim{
In Te inimícos nostros ventilábimus cornu: * et in nómine tuo spernémus insurgéntes in nobis.
}\switchcolumn\portugues{
Através de Vós investiremos contra os nossos inimigos: * e em vosso nome desprezaremos os que se levantaram contra nós.
}\switchcolumn*\latim{
Non enim in arcu meo sperábo: * et gládius meus non salvábit me.
}\switchcolumn\portugues{
Porque no meu arco não confiarei: * e não é a minha espada que me salvará.
}\switchcolumn*\latim{
Salvásti enim nos de affligéntibus nos: * et odiéntes nos confudísti.
}\switchcolumn\portugues{
Salvastes-nos dos que nos afligiam: * e humilhastes os que nos tinham ódio.
}\switchcolumn*\latim{
In Deo laudábimur tota die: * et in nómine tuo confitébimur in sǽculum.
}\switchcolumn\portugues{
Todo o dia celebraremos em Deus: * e no vosso nome louvaremos eternamente.
}\switchcolumn*\latim{
Nunc autem repulísti et confudísti nos: * et non egrediéris, Deus, in virtútibus nostris.
}\switchcolumn\portugues{
Agora, contudo, Vós repelistes-nos e humilhastes-nos: * Vós já não saís à frente dos nossos exércitos, ó Deus.
}\switchcolumn*\latim{
Avertísti nos retrórsum post inimícos nostros: * et qui odérunt nos, diripiébant sibi.
}\switchcolumn\portugues{
Fizestes-nos volver as costas aos nossos inimigos: * e os que nos odeiam saquearam para si mesmos.
}\switchcolumn*\latim{
Dedísti nos tamquam oves escárum: * et in géntibus dispersísti nos.
}\switchcolumn\portugues{
Entregastes-nos como ovelhas para o matadouro: * e dispersastes-nos entre as gentes.
}\switchcolumn*\latim{
Vendidísti pópulum tuum sine prétio: * et non fuit multitúdo in commutatiónibus eórum.
}\switchcolumn\portugues{
Vendestes o vosso povo sem preço: * e não houve lucro na sua troca.
}\switchcolumn*\latim{
Posuísti nos oppróbrium vicínis nostris, * subsannatiónem et derísum his, qui sunt in circúitu nostro.
}\switchcolumn\portugues{
Tornastes-nos a vergonha dos nossos vizinhos, * e objecto de escárnio e zombaria para aqueles que nos rodeiam.
}\switchcolumn*\latim{
Posuísti nos in similitúdinem géntibus: * commotiónem cápitis in pópulis.
}\switchcolumn\portugues{
Pusestes-nos como parábola entre as gentes: * um abanar de cabeça entre os povos.
}\switchcolumn*\latim{
Tota die verecúndia mea contra me est, * et confúsio faciéi meæ coopéruit me.
}\switchcolumn\portugues{
Minha ignomínia está todo o dia ante mim, * e o meu rosto cobriu-se de confusão.
}\switchcolumn*\latim{
A voce exprobrántis, et obloquéntis: * a fácie inimíci, et persequéntis.
}\switchcolumn\portugues{
À voz do que me insulta e destrói: * à vista do inimigo e do que me persegue.
}\switchcolumn*\latim{
Hæc ómnia venérunt super nos, nec oblíti sumus Te: * et iníque non égimus in testaménto tuo.
}\switchcolumn\portugues{
Tudo isto veio sobre nós, contudo, Vos não esquecemos: * e na vossa aliança não cometemos iniquidade.
}\switchcolumn*\latim{
Et non recéssit retro cor nostrum: * et declinásti sémitas nostras a via tua:
}\switchcolumn\portugues{
Nosso coração não recuou: * nem desviastes Vós nossos passos de vosso caminho:
}\switchcolumn*\latim{
Quóniam humiliásti nos in loco afflictiónis, * et coopéruit nos umbra mortis.
}\switchcolumn\portugues{
Porque nos humilhastes no lugar do tormento, * e a sombra da morte nos cobriu.
}\switchcolumn*\latim{
Si oblíti sumus nomen Dei nostri, * et si expándimus manus nostras ad deum aliénum:
}\switchcolumn\portugues{
Se nos esquecemos do nome do nosso Deus, * e se estendemos as mãos para algum deus estranho:
}\switchcolumn*\latim{
Nonne Deus requíret ista? * Ipse enim novit abscóndita cordis.
}\switchcolumn\portugues{
Não há-de Deus pedir conta disso? * Pois Ele conhece os segredos do coração.
}\switchcolumn*\latim{
Quóniam propter Te mortificámur tota die: * æstimáti sumus sicut oves occisiónis.
}\switchcolumn\portugues{
Somos por Vós entregues à morte todos os dias: * somos estimados como ovelhas para o matadouro.
}\switchcolumn*\latim{
Exsúrge, quare obdórmis, Dómine? * Exsúrge, et ne repéllas in finem.
}\switchcolumn\portugues{
Levantai-Vos, porque dormis, ó Senhor? * Levantai-Vos e nos não desampareis para sempre.
}\switchcolumn*\latim{
Quare fáciem tuam avértis, * oblivísceris inópiæ nostræ, et tribulatiónis nostræ?
}\switchcolumn\portugues{
Porque desviais de nós o vosso rosto, * e Vos esqueceis da nossa miséria e da nossa tribulação?
}\switchcolumn*\latim{
Quóniam humiliáta est in púlvere ánima nostra: * conglutinátus est in terra venter noster.
}\switchcolumn\portugues{
Porquanto a nossa alma está prostrada até ao pó: * e o nosso ventre está colado à terra.
}\switchcolumn*\latim{
Exsúrge, Dómine, ádjuva nos: * et rédime nos propter nomen tuum.
}\switchcolumn\portugues{
Levantai-Vos, ó Senhor, ajudai-nos: * e resgatai-nos por causa de vosso nome.
}\end{paracol}


\subsectioninfo{Salmo 44}{Eructavit cor meum verbum bonum}\label{salmo44}
\begin{paracol}{2}\latim{
\rlettrine{E}{ructávit} cor meum verbum bonum: * dico ego ópera mea Regi.
}\switchcolumn\portugues{
\rlettrine{D}{o} meu coração saiu uma boa palavra: * minhas obras as digo ao Rei.
}\switchcolumn*\latim{
Lingua mea cálamus scribæ: * velóciter scribéntis.
}\switchcolumn\portugues{
Minha língua é a pena do escriba: * que escreve velozmente.
}\switchcolumn*\latim{
Speciósus forma præ fíliis hóminum, diffúsa est grátia in lábiis tuis: * proptérea benedíxit Te Deus in ætérnum.
}\switchcolumn\portugues{
Sois o mais belo dos filhos dos homens, a graça derramou-se nos vossos lábios: * por isso Vos abençoou Deus para sempre.
}\switchcolumn*\latim{
Accíngere gládio tuo super femur tuum, * potentíssime.
}\switchcolumn\portugues{
Cingi a vossa espada à cintura, * ó poderosíssimo.
}\switchcolumn*\latim{
Spécie tua et pulchritúdine tua: * inténde, próspere procéde, et regna.
}\switchcolumn\portugues{
Na vossa majestade e no vosso esplendor: * caminhai, avançai vitoriosamente e reinai.
}\switchcolumn*\latim{
Propter veritátem, et mansuetúdinem, et justítiam: * et dedúcet Te mirabíliter déxtera tua.
}\switchcolumn\portugues{
Por meio da verdade, da mansidão e da justiça: * e a vossa dextra conduzir-Vos-á maravilhosamente.
}\switchcolumn*\latim{
Sagíttæ tuæ acútæ, pópuli sub Te cadent: * in corda inimicórum Regis.
}\switchcolumn\portugues{
Agudas são as vossas setas: os povos cairão debaixo de Vós: * traspassarão o coração dos inimigos do Rei.
}\switchcolumn*\latim{
Sedes tua, Deus, in sǽculum sǽculi: * virga directiónis virga regni tui.
}\switchcolumn\portugues{
Vosso trono, ó Deus, é pelos séculos dos séculos: * o ceptro de vosso reino é de rectidão.
}\switchcolumn*\latim{
Dilexísti justítiam, et odísti iniquitátem: * proptérea unxit Te, Deus, Deus tuus, óleo lætítiæ præ consórtibus tuis.
}\switchcolumn\portugues{
Amastes a justiça e odiastes a iniquidade: * por isso Deus, vosso Deus, Vos ungiu com óleo de alegria, sobre vossos companheiros.
}\switchcolumn*\latim{
Myrrha, et gutta, et cásia a vestiméntis tuis, a dómibus ebúrneis: * ex quibus delectavérunt Te fíliæ regum in honóre tuo.
}\switchcolumn\portugues{
De vossos vestes se exala Mirra, aloés e cássia, vêm das casas de marfim: * nas quais Vos alegraram as filhas dos reis na vossa glória.
}\switchcolumn*\latim{
Ástitit regína a dextris tuis in vestítu deauráto: * circúmdata varietáte.
}\switchcolumn\portugues{
A Rainha está à vossa dextra, com manto de ouro: * e variedamente ornada.
}\switchcolumn*\latim{
Audi fília, et vide, et inclína aurem tuam: * et oblivíscere populum tuum et domum patris tui.
}\switchcolumn\portugues{
Escutai, ó filha, vede e inclinai o vosso ouvido: * e esquecei-vos de vosso povo e da casa de vosso pai.
}\switchcolumn*\latim{
Et concupíscet Rex decórem tuum: * quóniam ipse est Dóminus Deus tuus, et adorábunt eum.
}\switchcolumn\portugues{
O Rei cobiçará a vossa beleza: * porque Ele é o Senhor vosso Deus e todos O adorarão.
}\switchcolumn*\latim{
Et fíliæ Tyri in munéribus * vultum tuum deprecabúntur: omnes dívites plebis.
}\switchcolumn\portugues{
As filhas de Tiro com dádivas * apresentar-vos-ão suas súplicas: e todos os ricos do povo.
}\switchcolumn*\latim{
Omnis glória ejus fíliæ Regis ab intus, * in fímbriis áureis circumamícta varietátibus.
}\switchcolumn\portugues{
Toda a glória da filha do Rei está no interior, * em franjas de ouro, ornada com variedade.
}\switchcolumn*\latim{
Adducéntur Regi vírgines post eam: * próximæ ejus afferéntur tibi.
}\switchcolumn\portugues{
Após ela as virgens serão apresentadas ao Rei: * as suas companheiras ser-Vos-ão conduzidas.
}\switchcolumn*\latim{
Afferéntur in lætítia et exsultatióne: * adducéntur in templum Regis.
}\switchcolumn\portugues{
Serão conduzidas com alegria e com regozijo: * conduzi-las-ão ao templo do Rei.
}\switchcolumn*\latim{
Pro pátribus tuis nati sunt tibi fílii: * constítues eos príncipes super omnem terram.
}\switchcolumn\portugues{
Em lugar de vossos pais, filhos vos nascerão: * estabelecê-los-eis príncipes sobre toda a terra.
}\switchcolumn*\latim{
Mémores erunt nóminis tui: * in omni generatióne et generatiónem.
}\switchcolumn\portugues{
Lembrar-se-ão de vosso nome: * por todas as gerações.
}\switchcolumn*\latim{
Proptérea pópuli confitebúntur tibi in ætérnum: * et in sǽculum sǽculi.
}\switchcolumn\portugues{
Por isso Vos louvarão eternamente os povos: * e pelos séculos dos séculos.
}\end{paracol}


\subsectioninfo{Salmo 45}{Deus noster refugium}\label{salmo45}
\begin{paracol}{2}\latim{
\rlettrine{D}{eus} noster refúgium, et virtus: * adjútor in tribulatiónibus, quæ invenérunt nos nimis.
}\switchcolumn\portugues{
\rlettrine{O}{} nosso Deus é o nosso refúgio e a nossa força: * o nosso auxílio nas muitas tribulações em que nos encontrávamos.
}\switchcolumn*\latim{
Proptérea non timébimus dum turbábitur terra: * et transferéntur montes in cor maris.
}\switchcolumn\portugues{
Por isso não temeremos, ainda que a terra seja perturbada: * e sejam precipitados os montes para o meio do mar.
}\switchcolumn*\latim{
Sonuérunt, et turbátæ sunt aquæ eórum: * conturbáti sunt montes in fortitúdine ejus.
}\switchcolumn\portugues{
Bradaram e turvaram-se suas águas: * os montes conturbaram-se com sua força.
}\switchcolumn*\latim{
Flúminis ímpetus lætíficat civitátem Dei: * sanctificávit tabernáculum suum Altíssimus.
}\switchcolumn\portugues{
A corrente do rio alegra a cidade de Deus: * o Altíssimo santificou o seu tabernáculo.
}\switchcolumn*\latim{
Deus in médio ejus, non commovébitur: * adjuvábit eam Deus mane dilúculo.
}\switchcolumn\portugues{
Deus está no meio dela, não será tremida: * Deus a ajudará ao raiar da manhã.
}\switchcolumn*\latim{
Conturbátæ sunt gentes, et inclináta sunt regna: * dedit vocem suam, mota est terra.
}\switchcolumn\portugues{
As gentes se conturbaram e os reinos se humilharam: * Ele fez ouvir a sua voz e a terra estremeceu.
}\switchcolumn*\latim{
Dóminus virtútum nobíscum: * suscéptor noster Deus Jacob.
}\switchcolumn\portugues{
O Senhor dos exércitos está connosco: * o Deus de Jacob é o nosso defensor.
}\switchcolumn*\latim{
Veníte, et vidéte ópera Dómini, quæ pósuit prodígia super terram: * áuferens bella usque ad finem terræ.
}\switchcolumn\portugues{
Vinde e vede as obras do Senhor, as maravilhas que operou sobre a terra fazendo cessar as guerras até à extremidade do mundo.
}\switchcolumn*\latim{
Arcum cónteret, et confrínget arma: * et scuta combúret igni.
}\switchcolumn\portugues{
Quebrará o arco e despedaçará as armas: * e queimará no fogo o escudo.
}\switchcolumn*\latim{
Vacáte, et vidéte quóniam ego sum Deus: * exaltábor in géntibus, et exaltábor in terra.
}\switchcolumn\portugues{
Parai e reconhecei que eu sou Deus: * hei-de ser exaltado entre as gentes e exaltado sobre terra.
}\switchcolumn*\latim{
Dóminus virtútum nobíscum: * suscéptor noster Deus Jacob.
}\switchcolumn\portugues{
O Senhor dos exércitos está connosco: * o Deus de Jacob é o nosso defensor.
}\end{paracol}


\subsectioninfo{Salmo 46}{Omnes gentes}\label{salmo46}
\begin{paracol}{2}\latim{
\rlettrine{O}{mnes} gentes, pláudite mánibus: * jubiláte Deo in voce exsultatiónis.
}\switchcolumn\portugues{
\rlettrine{B}{atei} palmas todas as gentes: * aclamai a Deus com vozes de alegria.
}\switchcolumn*\latim{
Quóniam Dóminus excélsus, terríbilis: * Rex magnus super omnem terram.
}\switchcolumn\portugues{
Porque o Senhor é excelso e terrível: * Rei supremo sobre toda a terra.
}\switchcolumn*\latim{
Subjécit pópulos nobis: * et gentes sub pédibus nostris.
}\switchcolumn\portugues{
Submeteu os povos a nós: * e as gentes debaixo dos nossos pés.
}\switchcolumn*\latim{
Elégit nobis hereditátem suam: * spéciem Jacob, quam diléxit.
}\switchcolumn\portugues{
Escolheu-nos para sua herança: * beleza de Jacob que tanto amou.
}\switchcolumn*\latim{
Ascéndit Deus in júbilo: * et Dóminus in voce tubæ.
}\switchcolumn\portugues{
Subiu Deus com júbilo: * e o Senhor com a voz da trombeta.
}\switchcolumn*\latim{
Psállite Deo nostro, psállite: * psállite Regi nostro, psállite.
}\switchcolumn\portugues{
Cantai ao nosso Deus, cantai: * cantai ao nosso Rei, cantai.
}\switchcolumn*\latim{
Quóniam Rex omnis terræ Deus: * psállite sapiénter.
}\switchcolumn\portugues{
Deus é o Rei de toda a terra: * cantai sabiamente.
}\switchcolumn*\latim{
Regnábit Deus super gentes: * Deus sedet super sedem sanctam suam.
}\switchcolumn\portugues{
Deus reinará sobre as gentes: * Deus está sentado no seu santo trono.
}\switchcolumn*\latim{
Príncipes populórum congregáti sunt cum Deo Ábraham: * quóniam dii fortes terræ veheménter eleváti sunt.
}\switchcolumn\portugues{
Os príncipes dos povos reuniram-se com o Deus de Abraão: * porque os fortes deuses da terra foram elevadíssimos.
}\end{paracol}


\subsectioninfo{Salmo 47}{Magnus Dominus}\label{salmo47}
\begin{paracol}{2}\latim{
\rlettrine{M}{agnus} Dóminus, et laudábilis nimis * in civitáte Dei nostri, in monte sancto ejus.
}\switchcolumn\portugues{
\rlettrine{G}{rande} é o Senhor e digníssimo de louvor * na cidade do nosso Deus, no seu santo monte.
}\switchcolumn*\latim{
Fundátur exsultatióne univérsæ terræ mons Sion, * látera Aquilónis, cívitas Regis magni.
}\switchcolumn\portugues{
Com júbilo de toda a terra foi fundado o monte de Sião, * a cidade do grande Rei ao lado do setentrião.
}\switchcolumn*\latim{
Deus in dómibus ejus cognoscétur: * cum suscípiet eam.
}\switchcolumn\portugues{
Deus far-se-á conhecer nas suas casas: * quando tiver de a defender.
}\switchcolumn*\latim{
Quóniam ecce reges terræ congregáti sunt: * convenérunt in unum.
}\switchcolumn\portugues{
Porque eis que os reis da terra se coligaram: * e se juntaram num só.
}\switchcolumn*\latim{
Ipsi vidéntes sic admiráti sunt, conturbáti sunt, commóti sunt: * tremor apprehéndit eos.
}\switchcolumn\portugues{
Eles, quando a viram, admiraram-se, conturbaram-se e afligidos ficaram: * o terror apoderou-se deles.
}\switchcolumn*\latim{
Ibi dolóres ut parturiéntis: * in spíritu veheménti cónteres naves Tharsis.
}\switchcolumn\portugues{
Ali sentiram dores como a mulher que dá à luz: * com vento impetuoso quebrareis as naus de Társis.
}\switchcolumn*\latim{
Sicut audívimus, sic vídimus in civitáte Dómini virtútum, in civitáte Dei nostri: * Deus fundávit eam in ætérnum.
}\switchcolumn\portugues{
Assim como ouvimos, assim vimos na cidade do Senhor dos exércitos, na cidade do nosso Deus: * Deus fundou-a para sempre.
}\switchcolumn*\latim{
Suscépimus, Deus, misericórdiam tuam, * in médio templi tui.
}\switchcolumn\portugues{
Recebemos a vossa misericórdia, ó Deus, * no meio de vosso templo.
}\switchcolumn*\latim{
Secúndum nomen tuum, Deus, sic et laus tua in fines terræ: * justítia plena est déxtera tua.
}\switchcolumn\portugues{
Como o vosso nome, ó Deus, também o vosso louvor se estende até aos confins da terra: * a vossa dextra está cheia de justiça.
}\switchcolumn*\latim{
Lætétur mons Sion, et exsúltent fíliæ Judæ: * propter judícia tua, Dómine.
}\switchcolumn\portugues{
Alegre-se o monte de Sião e regozijem-se as filhas de Judá: * devido aos vossos juízos, ó Senhor.
}\switchcolumn*\latim{
Circúmdate Sion, et complectímini eam: * narráte in túrribus ejus.
}\switchcolumn\portugues{
Dai voltas a Sião e considerai-a ao redor: * contai as suas torres.
}\switchcolumn*\latim{
Pónite corda vestra in virtúte ejus: * et distribúite domos ejus, ut enarrétis in progénie áltera.
}\switchcolumn\portugues{
Colocai o vosso coração na sua força: * e contemplai os seus baluartes, para que narreis à geração futura.
}\switchcolumn*\latim{
Quóniam hic est Deus, Deus noster in ætérnum et in sǽculum sǽculi: * ipse reget nos in sǽcula.
}\switchcolumn\portugues{
Porque Deus assim é, o nosso Deus para sempre e pelos séculos dos séculos: * Ele nos reinará eternamente.
}\end{paracol}


\subsectioninfo{Salmo 48}{Audite hæc}\label{salmo48}
\begin{paracol}{2}\latim{
\rlettrine{A}{udíte} hæc, omnes gentes: * áuribus percípite omnes, qui habitátis orbem:
}\switchcolumn\portugues{
\slettrine{Ó}{} todas as gentes ouvi isto: * estai atentas, vós todas que povoais a terra:
}\switchcolumn*\latim{
Quique terrígenæ, et fílii hóminum: * simul in unum dives et pauper.
}\switchcolumn\portugues{
Todas as que nasceram na terra e vós filhos dos homens: * o rico e o pobre juntamente.
}\switchcolumn*\latim{
Os meum loquétur sapiéntiam: * et meditátio cordis mei prudéntiam.
}\switchcolumn\portugues{
Sabedoria a minha boca proclamará: * e prudência da meditação do meu coração.
}\switchcolumn*\latim{
Inclinábo in parábolam aurem meam: * apériam in psaltério propositiónem meam.
}\switchcolumn\portugues{
Meu ouvido inclinarei à parábola: * revelarei ao som do saltério minha preposição.
}\switchcolumn*\latim{
Cur timébo in die mala? * Iníquitas calcánei mei circúmdabit me:
}\switchcolumn\portugues{
Que temerei no mau dia? * Rodear-me-á a iniquidade dos meus passos:
}\switchcolumn*\latim{
Qui confídunt in virtúte sua: * et in multitúdine divitiárum suárum gloriántur.
}\switchcolumn\portugues{
Eles confiam nas suas forças: * e glorificam-se na multidão das suas riquezas.
}\switchcolumn*\latim{
Frater non rédimit, rédimet homo: * non dabit Deo placatiónem suam.
}\switchcolumn\portugues{
O irmão não resgata, como resgatará o homem: * não dará a Deus a sua expiação.
}\switchcolumn*\latim{
Et prétium redemptiónis ánimæ suæ: * et laborábit in ætérnum, et vivet adhuc in finem.
}\switchcolumn\portugues{
Nem o preço da redenção de sua alma: * estará eternamente em labores e viverá, não obstante, até ao fim.
}\switchcolumn*\latim{
Non vidébit intéritum, cum víderit sapiéntes moriéntes: * simul insípiens, et stultus períbunt.
}\switchcolumn\portugues{
Ruína não verá, quando os sábios vir morrer: * o parvo e o tolo perecerão igualmente.
}\switchcolumn*\latim{
Et relínquent aliénis divítias suas: * et sepúlcra eórum domus illórum in ætérnum.
}\switchcolumn\portugues{
Deixarão aos estranhos as suas riquezas: * e os seus sepulcros serão para sempre as suas habitações.
}\switchcolumn*\latim{
Tabernácula eórum in progénie et progénie: * vocavérunt nómina sua in terris suis.
}\switchcolumn\portugues{
Sua morada de geração em geração: * eles que deram os seus nomes às suas terras.
}\switchcolumn*\latim{
Et homo, cum in honóre esset, non intelléxit: * comparátus est juméntis insipiéntibus, et símilis factus est illis.
}\switchcolumn\portugues{
O homem, em honra constituído, não entendeu: * foi comparado a bestas irracionais e como eles se tornou.
}\switchcolumn*\latim{
Hæc via illórum scándalum ipsis: * et póstea in ore suo complacébunt.
}\switchcolumn\portugues{
Este seu proceder é causa da sua ruína: * e, apesar disto, deleitam-se nos seus discursos.
}\switchcolumn*\latim{
Sicut oves in inférno pósiti sunt: * mors depáscet eos.
}\switchcolumn\portugues{
São postos no inferno como ovelhas: * e serão pasto da morte.
}\switchcolumn*\latim{
Et dominabúntur eórum justi in matutíno: * et auxílium eórum veteráscet in inférno a glória eórum.
}\switchcolumn\portugues{
Os justos terão domínio sobre eles na manhã: * e da sua glória, a ajuda que tiveram será destruída no inferno.
}\switchcolumn*\latim{
Verúmtamen Deus rédimet ánimam meam de manu ínferi: * cum accéperit me.
}\switchcolumn\portugues{
Deus, porém, resgatará a minha alma do poder do inferno: * quando me receber.
}\switchcolumn*\latim{
Ne timúeris, cum dives factus fúerit homo: * et cum multiplicáta fúerit glória domus ejus.
}\switchcolumn\portugues{
Não temas quando um homem se enriquecer: * e quando crescer a glória da sua casa.
}\switchcolumn*\latim{
Quóniam cum interíerit, non sumet ómnia: * neque descéndet cum eo glória ejus.
}\switchcolumn\portugues{
Porque, morrendo, nada levará consigo: * nem com ele descerá a sua glória.
}\switchcolumn*\latim{
Quia ánima ejus in vita ipsíus benedicétur: * confitébitur tibi cum beneféceris ei.
}\switchcolumn\portugues{
Pois, enquanto vive, será louvada a sua alma: * ele bendizer-Vos-á quando bem lhe fizerdes.
}\switchcolumn*\latim{
Introíbit usque in progénies patrum suórum: * et usque in ætérnum non vidébit lumen.
}\switchcolumn\portugues{
Entrará na geração de seus pais: * e não verá jamais a luz.
}\switchcolumn*\latim{
Homo, cum in honóre esset, non intelléxit: * comparátus est juméntis insipiéntibus, et símilis factus est illis.
}\switchcolumn\portugues{
O homem, constituído em honra, não entendeu: * foi comparado a bestas irracionais e tornou-se semelhante a elas.
}\end{paracol}


\subsectioninfo{Salmo 49}{Deus deorum}\label{salmo49}
\begin{paracol}{2}\latim{
\rlettrine{D}{eus} deórum, Dóminus locútus est: * et vocávit terram,
}\switchcolumn\portugues{
\rlettrine{F}{alou} o Senhor, Deus dos deuses: * e convocou a terra,
}\switchcolumn*\latim{
A solis ortu usque ad occásum: * ex Sion spécies decóris ejus.
}\switchcolumn\portugues{
Da aurora até ao crepúsculo: * de Sião virá o esplendor da sua formosura.
}\switchcolumn*\latim{
Deus maniféste véniet: * Deus noster et non silébit.
}\switchcolumn\portugues{
Manifestamente Deus virá: * nosso Deus e não manterá silêncio.
}\switchcolumn*\latim{
Ignis in conspéctu ejus exardéscet: * et in circúitu ejus tempéstas válida.
}\switchcolumn\portugues{
O fogo incendiar-se-á na sua presença: * e uma tempestade violenta cerca-l'O-á.
}\switchcolumn*\latim{
Advocábit cælum desúrsum: * et terram discérnere pópulum suum.
}\switchcolumn\portugues{
De alto chamará o céu: * e a terra, para julgar o seu povo.
}\switchcolumn*\latim{
Congregáte illi sanctos ejus: * qui órdinant testaméntum ejus super sacrifícia.
}\switchcolumn\portugues{
Reuni diante d’Ele os seus santos: * os quais fizeram aliança com Ele por meio de sacrifícios.
}\switchcolumn*\latim{
Et annuntiábunt cæli justítiam ejus: * quóniam Deus judex est.
}\switchcolumn\portugues{
Os céus anunciarão a sua justiça: * porquanto Deus é o juiz.
}\switchcolumn*\latim{
Audi, pópulus meus, et loquar: Israël, et testificábor tibi: * Deus, Deus tuus ego sum.
}\switchcolumn\portugues{
Ouve, ó meu povo, e falarei: ouve, ó Israel, e te darei testemunho: * Deus, o teu Deus sou eu.
}\switchcolumn*\latim{
Non in sacrifíciis tuis árguam te: * holocáusta autem tua in conspéctu meo sunt semper.
}\switchcolumn\portugues{
Por causa de teus sacrifícios te não acusarei: * os teus holocaustos estão sempre ante mim.
}\switchcolumn*\latim{
Non accípiam de domo tua vítulos: * neque de grégibus tuis hircos.
}\switchcolumn\portugues{
Não receberei de tua casa vitelos: * nem cabritos de teus rebanhos.
}\switchcolumn*\latim{
Quóniam meæ sunt omnes feræ silvárum: * juménta in móntibus et boves.
}\switchcolumn\portugues{
Porque são minhas todas as feras das selvas: * os animais dos montes e os bois.
}\switchcolumn*\latim{
Cognóvi ómnia volatília cæli: * et pulchritúdo agri mecum est.
}\switchcolumn\portugues{
Conheço todas as aves do céu: * e comigo está a formosura do campo.
}\switchcolumn*\latim{
Si esuríero, non dicam tibi: * meus est enim orbis terræ, et plenitúdo ejus.
}\switchcolumn\portugues{
Se fome tiver te não direi: * pois minha é a órbita da terra e o que ela contém.
}\switchcolumn*\latim{
Numquid manducábo carnes taurórum? * Aut sánguinem hircórum potábo?
}\switchcolumn\portugues{
Porventura comerei a carne dos touros? * Ou beberei o sangue dos cabritos?
}\switchcolumn*\latim{
Ímmola Deo sacrifícium laudis: * et redde Altíssimo vota tua.
}\switchcolumn\portugues{
Oferece a Deus um sacrifício de louvor: * e paga ao Altíssimo os teus votos.
}\switchcolumn*\latim{
Et ínvoca me in die tribulatiónis: * éruam te, et honorificábis me.
}\switchcolumn\portugues{
Invoca-me no dia da tribulação: * livrar-te-ei e tu me honrarás.
}\switchcolumn*\latim{
Peccatóri autem dixit Deus: * Quare tu enárras justítias meas, et assúmis testaméntum meum per os tuum?
}\switchcolumn\portugues{
Porém, ao pecador disse Deus: * porque falas tu dos meus mandamentos e tens a minha aliança na tua boca?
}\switchcolumn*\latim{
Tu vero odísti disciplínam: * et projecísti sermónes meos retrórsum:
}\switchcolumn\portugues{
Posto que tu aborreces a disciplina: * e rejeitaste as minhas palavras:
}\switchcolumn*\latim{
Si vidébas furem, currébas cum eo: * et cum adúlteris portiónem tuam ponébas.
}\switchcolumn\portugues{
Se vias um ladrão, corrias ao seu lado: * e com os adúlteros te juntavas.
}\switchcolumn*\latim{
Os tuum abundávit malítia: * et lingua tua concinnábat dolos.
}\switchcolumn\portugues{
Em malícia abundou a tua boca: * e a tua língua enganos urdia.
}\switchcolumn*\latim{
Sedens advérsus fratrem tuum loquebáris, et advérsus fílium matris tuæ ponébas scándalum: * hæc fecísti, et tácui.
}\switchcolumn\portugues{
Estando sentado, falavas contra teu irmão e lançavas escândalos ao filho de tua mãe: * isto fizeste e calei-me.
}\switchcolumn*\latim{
Existimásti, iníque, quod ero tui símilis: * árguam te, et státuam contra fáciem tuam.
}\switchcolumn\portugues{
Pensaste iniquamente que seria como tu: * acusar-te-ei e porei ante tua cara.
}\switchcolumn*\latim{
Intellégite hæc, qui obliviscímini Deum: * nequándo rápiat, et non sit qui erípiat.
}\switchcolumn\portugues{
Entendei isto, vós que vos esqueceis de Deus: * não suceda que vos arrebate e não haja quem vos livre.
}\switchcolumn*\latim{
Sacrifícium laudis honorificábit me: * et illic iter, quo osténdam illi salutáre Dei.
}\switchcolumn\portugues{
O sacrifício de louvor honrar-me-á: * e aí está o caminho, pelo qual lhe mostrarei a salvação de Deus.
}\end{paracol}


\subsectioninfo{Salmo 50}{Miserere mei}\label{salmo50}
\begin{paracol}{2}\latim{
\rlettrine{M}{iserére} mei, Deus, * secúndum magnam misericórdiam tuam.
}\switchcolumn\portugues{
\rlettrine{T}{ende} piedade de mim, ó Deus, * segundo a vossa grande misericórdia.
}\switchcolumn*\latim{
Et secúndum multitúdinem miseratiónum tuárum, * dele iniquitátem meam.
}\switchcolumn\portugues{
E, segundo a multidão de vossas bondades, * apagai a minha iniquidade.
}\switchcolumn*\latim{
Amplius lava me ab iniquitáte mea: * et a peccáto meo munda me.
}\switchcolumn\portugues{
Lavai-me inteiramente da minha iniquidade: * e purificai-me do meu pecado.
}\switchcolumn*\latim{
Quóniam iniquitátem meam ego cognósco: * et peccátum meum contra me est semper.
}\switchcolumn\portugues{
Porque reconheço a minha iniquidade: * e o meu pecado está sempre ante mim.
}\switchcolumn*\latim{
Tibi soli peccávi, et malum coram Te feci: * ut justificéris in sermónibus tuis, et vincas cum judicáris.
}\switchcolumn\portugues{
Contra Vós só pequei e ante Vós fiz o mal: * para que sejais justificado nas vossas palavras e vençais quando fores julgado.
}\switchcolumn*\latim{
Ecce enim, in iniquitátibus concéptus sum: * et in peccátis concépit me mater mea.
}\switchcolumn\portugues{
Eis que fui concebido em iniquidades: * e minha mãe no pecado me concebeu.
}\switchcolumn*\latim{
Ecce enim, veritátem dilexísti: * incérta et occúlta sapiéntiæ tuæ manifestásti mihi.
}\switchcolumn\portugues{
Eis que amastes a verdade: * e me revelastes o incerto e o oculto de vossa sabedoria.
}\switchcolumn*\latim{
Aspérges me hyssópo, et mundábor: * lavábis me, et super nivem dealbábor.
}\switchcolumn\portugues{
Aspergir-me-eis com o hissope e ficarei limpo: * lavar-me-eis e me tornarei mais branco que a neve.
}\switchcolumn*\latim{
Audítui meo dabis gáudium et lætítiam: * et exsultábunt ossa humiliáta.
}\switchcolumn\portugues{
Far-me-eis ouvir palavras de consolação e alegria: * e exultar-se-ão os ossos humilhados.
}\switchcolumn*\latim{
Avérte fáciem tuam a peccátis meis: * et omnes iniquitátes meas dele.
}\switchcolumn\portugues{
Afastai o vosso rosto dos meus pecados: * e apagai todas minhas iniquidades.
}\switchcolumn*\latim{
Cor mundum crea in me, Deus: * et spíritum rectum ínnova in viscéribus meis.
}\switchcolumn\portugues{
Criai um coração puro em mim, ó Deus: * e renovai nas minhas entranhas um espírito recto.
}\switchcolumn*\latim{
Ne proícias me a fácie tua: * et spíritum sanctum tuum ne áuferas a me.
}\switchcolumn\portugues{
Não me expulsais de vossa presença: * e de mim não afasteis o vosso espírito santo.
}\switchcolumn*\latim{
Redde mihi lætítiam salutáris tui: * et spíritu principáli confírma me.
}\switchcolumn\portugues{
Restaurai em mim a alegria de vossa salvação: * e confortai-me com um espírito magnânimo.
}\switchcolumn*\latim{
Docébo iníquos vias tuas: * et ímpii ad Te converténtur.
}\switchcolumn\portugues{
Ensinarei aos iníquos os vossos caminhos: * e a Vós converter-se-ão os ímpios.
}\switchcolumn*\latim{
Líbera me de sanguínibus, Deus, Deus salútis meæ: * et exsultábit lingua mea justítiam tuam.
}\switchcolumn\portugues{
Livrai-me das penas de sangue, ó Deus, Deus da minha salvação: * e a minha língua exaltará a vossa justiça.
}\switchcolumn*\latim{
Dómine, lábia mea apéries: * et os meum annuntiábit laudem tuam.
}\switchcolumn\portugues{
Abrireis os meus lábios, ó Senhor: * e a minha boca anunciará os vossos louvores.
}\switchcolumn*\latim{
Quóniam si voluísses sacrifícium, dedíssem útique: * holocáustis non delectáberis.
}\switchcolumn\portugues{
Porque se quisésseis um sacrifício, o teria oferecido: * mas Vos não deleitais com holocaustos.
}\switchcolumn*\latim{
Sacrifícium Deo spíritus contribulátus: * cor contrítum, et humiliátum, Deus, non despícies.
}\switchcolumn\portugues{
O sacrifício para Deus é um espírito contrito: * não desprezareis, ó Deus, um coração contrito e humilhado.
}\switchcolumn*\latim{
Benígne fac, Dómine, in bona voluntáte tua Sion: * ut ædificéntur muri Jerúsalem.
}\switchcolumn\portugues{
Pela vossa bondade, ó Senhor, sede benigno para com Sião: * para que se edifiquem os muros de Jerusalém.
}\switchcolumn*\latim{
Tunc acceptábis sacrifícium justítiæ, oblatiónes, et holocáusta: * tunc impónent super altáre tuum vítulos.
}\switchcolumn\portugues{
Aceitareis então os sacrifícios legítimos, oferendas e holocaustos: * então sobre o vosso altar serão colocados vitelos.
}\end{paracol}


\subsectioninfo{Salmo 51}{Quid gloriaris}\label{salmo51}
\begin{paracol}{2}\latim{
\qlettrine{Q}{uid} gloriáris in malítia, * qui potens es in iniquitáte?
}\switchcolumn\portugues{
\rlettrine{P}{orque} te glorias de tua malícia, * tu que és poderoso em iniquidade?
}\switchcolumn*\latim{
Tota die injustítiam cogitávit lingua tua: * sicut novácula acúta fecísti dolum.
}\switchcolumn\portugues{
Todo o dia a tua língua meditou injustiça: * como navalha afiada dolos fizeste.
}\switchcolumn*\latim{
Dilexísti malítiam super benignitátem: * iniquitátem magis quam loqui æquitátem.
}\switchcolumn\portugues{
Amaste o mal sobre o bem: * a linguagem da iniquidade mais que a da justiça.
}\switchcolumn*\latim{
Dilexísti ómnia verba præcipitatiónis, * lingua dolósa.
}\switchcolumn\portugues{
Amaste todas as palavras de ruína, * ó língua enganadora.
}\switchcolumn*\latim{
Proptérea Deus déstruet te in finem, * evéllet te, et emigrábit te de tabernáculo tuo: et radícem tuam de terra vivéntium.
}\switchcolumn\portugues{
Por isso Deus destruir-te-á para sempre: * arrancar-te-á, expulsar-te-á de tua morada e a tua estirpe da terra dos vivos.
}\switchcolumn*\latim{
Vidébunt justi, et timébunt, et super eum ridébunt, et dicent: * Ecce homo, qui non pósuit Deum adjutórem suum:
}\switchcolumn\portugues{
Vê-lo-ão os justos, temerão e dele se rirão, dizendo: * eis o homem que não tomou a Deus por seu protector:
}\switchcolumn*\latim{
Sed sperávit in multitúdine divitiárum suárum: * et præváluit in vanitáte sua.
}\switchcolumn\portugues{
Contudo, esperou na multidão das suas riquezas: * e prevaleceu na sua vaidade.
}\switchcolumn*\latim{
Ego autem, sicut olíva fructífera in domo Dei, * sperávi in misericórdia Dei in ætérnum: et in sǽculum sǽculi.
}\switchcolumn\portugues{
Eu, porém, sou como oliveira frutífera na casa de Deus, * espero na misericórdia de Deus para sempre e pelos séculos dos séculos.
}\switchcolumn*\latim{
Confitébor tibi in sǽculum, quia fecísti: * et exspectábo nomen tuum, quóniam bonum est in conspéctu sanctórum tuórum.
}\switchcolumn\portugues{
Louvar-Vos-ei eternamente, devido ao que fizestes: * e esperarei no vosso nome, porque é bom ante vossos santos.
}\end{paracol}


\subsectioninfo{Salmo 52}{Dixit insipiens in corde}\label{salmo52}
\begin{paracol}{2}\latim{
\rlettrine{D}{ixit} insípiens in corde suo: * Non est Deus.
}\switchcolumn\portugues{
\rlettrine{D}{isse} o parvo no seu coração: * não há Deus.
}\switchcolumn*\latim{
Corrúpti sunt, et abominábiles facti sunt in iniquitátibus: * non est qui fáciat bonum.
}\switchcolumn\portugues{
São corruptos e tornaram-se abomináveis nas suas iniquidades: * não há quem o bem faça.
}\switchcolumn*\latim{
Deus de cælo prospéxit super fílios hóminum: * ut vídeat si est intéllegens, aut requírens Deum.
}\switchcolumn\portugues{
Deus olhou do céu sobre os filhos dos homens: * para ver se há inteligentes, ou quem a Deus busque.
}\switchcolumn*\latim{
Omnes declinavérunt, simul inútiles facti sunt: * non est qui fáciat bonum, non est usque ad unum.
}\switchcolumn\portugues{
Todos se extraviaram, juntos tornaram-se inúteis: * não há quem o bem faça, não há sequer um só.
}\switchcolumn*\latim{
Nonne scient omnes qui operántur iniquitátem, * qui dévorant plebem meam ut cibum panis?
}\switchcolumn\portugues{
Porventura se não lembrarão todos os obreiros da iniquidade, * os que devoram o meu povo como quem pão come?
}\switchcolumn*\latim{
Deum non invocavérunt: * illic trepidavérunt timóre, ubi non erat timor.
}\switchcolumn\portugues{
Não invocaram a Deus: * tremeram de medo onde não havia que temer.
}\switchcolumn*\latim{
Quóniam Deus dissipávit ossa eórum qui homínibus placent: * confúsi sunt, quóniam Deus sprevit eos.
}\switchcolumn\portugues{
Porque dissipou Deus os ossos daqueles que aos homens agradam: * foram confundidos, porque Deus os desprezou.
}\switchcolumn*\latim{
Quis dabit ex Sion salutáre Israël? * Cum convérterit Deus captivitátem plebis suæ, exsultábit Jacob, et lætábitur Israël.
}\switchcolumn\portugues{
Quem enviará de Sião a salvação de Israel? * Quando Deus puser fim ao cativeiro do seu povo, regozijar-se-á Jacob e alegrar-se-á Israel.
}\end{paracol}


\subsectioninfo{Salmo 53}{Deus, in nomine tuo salvum}\label{salmo53}
\begin{paracol}{2}\latim{
\rlettrine{D}{eus,} in nómine tuo salvum me fac: * et in virtúte tua júdica me.
}\switchcolumn\portugues{
\rlettrine{S}{alvai-me,} ó Deus, por vosso nome: * e com vosso poder julgai-me.
}\switchcolumn*\latim{
Deus, exáudi oratiónem meam: * áuribus pércipe verba oris mei.
}\switchcolumn\portugues{
Ouvi, ó Deus, a minha oração: * atendei às palavras da minha boca.
}\switchcolumn*\latim{
Quóniam aliéni insurrexérunt advérsum me, et fortes quæsiérunt ánimam meam: * et non proposuérunt Deum ante conspéctum suum.
}\switchcolumn\portugues{
Porque os estranhos se levantaram contra mim e os fortes buscaram a minha vida: * e a Deus não puseram ante si.
}\switchcolumn*\latim{
Ecce enim, Deus ádjuvat me: * et Dóminus suscéptor est ánimæ meæ.
}\switchcolumn\portugues{
Eis que Deus vem em meu auxílio: * e o Senhor é o protector da minha vida.
}\switchcolumn*\latim{
Avérte mala inimícis meis: * et in veritáte tua dispérde illos.
}\switchcolumn\portugues{
Fazei recair os males sobre os meus inimigos: * e exterminai-os na vossa verdade.
}\switchcolumn*\latim{
Voluntárie sacrificábo tibi, * et confitébor nómini tuo, Dómine: quóniam bonum est:
}\switchcolumn\portugues{
Sacrificar-me-ei voluntariamente a Vós, * e o vosso nome louvarei, ó Senhor, porque é bom:
}\switchcolumn*\latim{
Quóniam ex omni tribulatióne eripuísti me: * et super inimícos meos despéxit óculus meus.
}\switchcolumn\portugues{
Porquanto me tendes livrado de toda a tribulação: * e com desdém olhei os meus inimigos.
}\end{paracol}


\subsectioninfo{Salmo 54}{Exaudi, Deus, orationem meam}\label{salmo54}
\begin{paracol}{2}\latim{
\rlettrine{E}{xáudi,} oratiónem meam, et ne despéxeris deprecatiónem meam: * inténde mihi, et exáudi me.
}\switchcolumn\portugues{
\rlettrine{O}{uvi,} ó Deus, a minha oração e não desprezeis a minha súplica: * atendei-me e ouvi-me.
}\switchcolumn*\latim{
Contristátus sum in exercitatióne mea: * et conturbátus sum a voce inimíci, et a tribulatióne peccatóris.
}\switchcolumn\portugues{
Estou triste na minha provação: * abalado estou pela voz do inimigo e pela perseguição do pecador.
}\switchcolumn*\latim{
Quóniam declinavérunt in me iniquitátes: * et in ira molésti erant mihi.
}\switchcolumn\portugues{
Porque me lançaram iniquidades: * e com ira me angustiaram.
}\switchcolumn*\latim{
Cor meum conturbátum est in me: * et formído mortis cécidit super me.
}\switchcolumn\portugues{
Meu coração está abalado dentro de mim: * e sobre mim caiu o pavor da morte.
}\switchcolumn*\latim{
Timor et tremor venérunt super me: * et contexérunt me ténebræ.
}\switchcolumn\portugues{
Temor e tremor sobre mim vieram: * e me rodearam as trevas.
}\switchcolumn*\latim{
Et dixi: quis dabit mihi pennas sicut colúmbæ, * et volábo, et requiéscam?
}\switchcolumn\portugues{
Então disse: quem me dará asas como as da pomba, * para voar e repousar?
}\switchcolumn*\latim{
Ecce, elongávi fúgiens: * et mansi in solitúdine.
}\switchcolumn\portugues{
Eis que me afastei fugindo: * e permaneci na solidão.
}\switchcolumn*\latim{
Exspectábam eum, qui salvum me fecit * a pusillanimitáte spíritus et tempestáte.
}\switchcolumn\portugues{
Aguardava Aquele que me salvou * da cobardia de espírito e da tempestade.
}\switchcolumn*\latim{
Præcípita, Dómine, dívide linguas eórum: * quóniam vidi iniquitátem, et contradictiónem in civitáte.
}\switchcolumn\portugues{
Precipitai-os, ó Senhor, dividi as suas línguas: * porque vejo a injustiça e a contradição na cidade.
}\switchcolumn*\latim{
Die ac nocte circúmdabit eam super muros ejus iníquitas: * et labor in médio ejus, et injustítia.
}\switchcolumn\portugues{
Dia e noite cercará sobre seus muros a iniquidade: * está no meio dela a labuta e a injustiça.
}\switchcolumn*\latim{
Et non defécit de platéis ejus * usúra, et dolus.
}\switchcolumn\portugues{
Não se afastam das suas praças * a usura e o dolo.
}\switchcolumn*\latim{
Quóniam si inimícus meus maledixísset mihi, * sustinuíssem útique.
}\switchcolumn\portugues{
Se o ultraje viesse do meu inimigo, * por certo o teria suportado.
}\switchcolumn*\latim{
Et si is, qui óderat me, super me magna locútus fuísset, * abscondíssem me fórsitan ab eo.
}\switchcolumn\portugues{
E, se o que me odiava tivesse falado de mim com insolência, * talvez me teria escondido dele.
}\switchcolumn*\latim{
Tu vero, homo unánimis: * dux meus, et notus meus:
}\switchcolumn\portugues{
Contudo, tu, ó homem unânime: * meu guia e meu amigo:
}\switchcolumn*\latim{
Qui simul mecum dulces capiébas cibos: * in domo Dei ambulávimus cum consénsu.
}\switchcolumn\portugues{
Que juntamente comigo tomavas doces manjares: * ambulávamos com consenso na casa do Senhor!
}\switchcolumn*\latim{
Véniat mors super illos: * et descéndant in inférnum vivéntes:
}\switchcolumn\portugues{
Venha a morte sobre eles: * e desçam vivos ao inferno:
}\switchcolumn*\latim{
Quóniam nequítiæ in habitáculis eórum: * in médio eórum.
}\switchcolumn\portugues{
Porque a malícia está nas suas moradas: * no meio deles.
}\switchcolumn*\latim{
Ego autem ad Deum clamávi: * et Dóminus salvábit me.
}\switchcolumn\portugues{
Eu, porém, clamei a Deus: * e o Senhor salvar-me-á.
}\switchcolumn*\latim{
Véspere, et mane, et merídie narrábo et annuntiábo: * et exáudiet vocem meam.
}\switchcolumn\portugues{
De tarde, de manhã e ao meio-dia narrarei e anunciarei: * e Ele ouvirá a minha voz.
}\switchcolumn*\latim{
Rédimet in pace ánimam meam ab his, qui appropínquant mihi: * quóniam inter multos erant mecum.
}\switchcolumn\portugues{
Em paz Ele salvará a minha vida daqueles que me assaltam: * porque são muitos contra mim.
}\switchcolumn*\latim{
Exáudiet Deus, et humiliábit illos, * qui est ante sǽcula.
}\switchcolumn\portugues{
Deus ouvirá e humilhá-los-á, * O que é antes dos séculos.
}\switchcolumn*\latim{
Non enim est illis commutátio, et non timuérunt Deum: * exténdit manum suam in retribuéndo.
}\switchcolumn\portugues{
Pois não há mudança neles e não temeram a Deus: * estendeu a sua mão para lhes retribuir.
}\switchcolumn*\latim{
Contaminavérunt testaméntum ejus, divísi sunt ab ira vultus ejus: * et appropinquávit cor illíus.
}\switchcolumn\portugues{
Profanaram a sua aliança, foram divididos pela ira do seu rosto: * e o seu coração se aproximou.
}\switchcolumn*\latim{
Mollíti sunt sermónes ejus super óleum: * et ipsi sunt jácula.
}\switchcolumn\portugues{
Suas palavras são mais suaves que o azeite: * e as mesmas são flechas.
}\switchcolumn*\latim{
Jacta super Dóminum curam tuam, et ipse te enútriet: * non dabit in ætérnum fluctuatiónem justo.
}\switchcolumn\portugues{
Descarrega sobre o Senhor os teus cuidados e Ele te sustentará: * não deixará o justo em perpétua agitação.
}\switchcolumn*\latim{
Tu vero, Deus, dedúces eos, * in púteum intéritus.
}\switchcolumn\portugues{
Contudo, Vós, ó Deus, os conduzireis * ao poço da perdição.
}\switchcolumn*\latim{
Viri sánguinum, et dolósi non dimidiábunt dies suos: * ego autem sperábo in Te, Dómine.
}\switchcolumn\portugues{
Homens sanguinários e enganadores não chegarão à metade dos seus dias: * eu, porém, esperei em Vós, ó Senhor.
}\end{paracol}


\subsectioninfo{Salmo 55}{Miserere mei, Deus}\label{salmo55}
\begin{paracol}{2}\latim{
\rlettrine{M}{iserére} mei, Deus, quóniam conculcávit me homo: * tota die impúgnans tribulávit me.
}\switchcolumn\portugues{
\rlettrine{T}{ende} misericórdia de mim, Deus, porque me calcou o homem: * angustiou-me combatendo-me todo o dia.
}\switchcolumn*\latim{
Conculcavérunt me inimíci mei tota die: * quóniam multi bellántes advérsum me.
}\switchcolumn\portugues{
Calcaram-me os meus inimigos todo o dia: * porque são muitos os que lutam contra mim.
}\switchcolumn*\latim{
Ab altitúdine diéi timébo: * ego vero in Te sperábo.
}\switchcolumn\portugues{
Temerei desde que o dia desponta: * mas esperarei em Vós.
}\switchcolumn*\latim{
In Deo laudábo sermónes meos, in Deo sperávi: * non timébo quid fáciat mihi caro.
}\switchcolumn\portugues{
Em Deus louvarei a minha palavra, em Deus espero: * não temerei o que me possa fazer a carne.
}\switchcolumn*\latim{
Tota die verba mea exsecrabántur: * advérsum me omnes cogitatiónes eórum in malum.
}\switchcolumn\portugues{
Todos os dias abominavam as minhas palavras: * para o mal, todos seus pensamentos eram contra mim.
}\switchcolumn*\latim{
Inhabitábunt et abscóndent: * ipsi calcáneum meum observábunt.
}\switchcolumn\portugues{
Juntar-se-ão e esconder-se-ão: * espiarão todos meus passos.
}\switchcolumn*\latim{
Sicut sustinuérunt ánimam meam, pro níhilo salvos fácies illos: * in ira pópulos confrínges.
}\switchcolumn\portugues{
Como disputaram a minha alma, por nada os salvareis: * na vossa ira despedaçareis estes povos.
}\switchcolumn*\latim{
Deus, vitam meam annuntiávi tibi: * posuísti lácrimas meas in conspéctu tuo.
}\switchcolumn\portugues{
Ó Deus, a Vós expus a minha vida: * tendes presente as minhas lágrimas.
}\switchcolumn*\latim{
Sicut et in promissióne tua: * tunc converténtur inimíci mei retrórsum:
}\switchcolumn\portugues{
Conforme a vossa promessa: * depois serão postos em fuga os meus inimigos.
}\switchcolumn*\latim{
In quacúmque die invocávero Te: * ecce, cognóvi, quóniam Deus meus es.
}\switchcolumn\portugues{
Em qualquer dia que Vos invocar: * eis que conhecerei que sois o meu Deus.
}\switchcolumn*\latim{
In Deo laudábo verbum, in Dómino laudábo sermónem: * in Deo sperávi, non timébo quid fáciat mihi homo.
}\switchcolumn\portugues{
Em Deus louvarei a palavra, no Senhor louvarei o seu discurso: * em Deus espero, não temerei o que o homem me possa fazer.
}\switchcolumn*\latim{
In me sunt, Deus, vota tua, * quæ reddam, laudatiónes tibi.
}\switchcolumn\portugues{
Em mim estão, ó Deus, os votos que Vos fiz, * os quais cumprirei com louvores.
}\switchcolumn*\latim{
Quóniam eripuísti ánimam meam de morte, et pedes meos de lapsu: * ut pláceam coram Deo in lúmine vivéntium.
}\switchcolumn\portugues{
Porque livrastes a minha alma da morte e os meus pés da queda: * para que eu seja agradável a Deus na luz dos viventes.
}\end{paracol}


\subsectioninfo{Salmo 56}{Miserere mei, Deus, miserere mei}\label{salmo56}
\begin{paracol}{2}\latim{
\rlettrine{M}{iserére} mei, Deus, miserére mei: * quóniam in Te confídit ánima mea.
}\switchcolumn\portugues{
\rlettrine{T}{ende} de mim piedade, ó Deus, tende de mim piedade: * porque em Vós confia a minha alma.
}\switchcolumn*\latim{
Et in umbra alárum tuárum sperábo, * donec tránseat iníquitas.
}\switchcolumn\portugues{
Na sombra de vossas asas esperarei, * até que a iniquidade passe.
}\switchcolumn*\latim{
Clamábo ad Deum altíssimum: * Deum qui benefécit mihi.
}\switchcolumn\portugues{
Clamarei ao Deus altíssimo: * ao Deus que tanto bem me tem feito.
}\switchcolumn*\latim{
Misit de cælo, et liberávit me: * dedit in oppróbrium conculcántes me.
}\switchcolumn\portugues{
Enviou do céu e me livrou: * cobriu de desonra os que me calcavam.
}\switchcolumn*\latim{
Misit Deus misericórdiam suam, et veritátem suam, * et erípuit ánimam meam de médio catulórum leónum: dormívi conturbátus.
}\switchcolumn\portugues{
Deus enviou a sua misericórdia e a sua verdade, * e tirou a minha alma do meio dos jovens leões: dormi conturbado.
}\switchcolumn*\latim{
Fílii hóminum dentes eórum arma et sagíttæ: * et lingua eórum gládius acútus.
}\switchcolumn\portugues{
Os filhos dos homens têm dentes que são armas e setas: * e a sua língua é uma espada aguda.
}\switchcolumn*\latim{
Exaltáre super cælos, Deus, * et in omnem terram glória tua.
}\switchcolumn\portugues{
Exaltai-Vos sobre os céus, ó Deus, * e a vossa glória sobre toda a terra.
}\switchcolumn*\latim{
Láqueum paravérunt pédibus meis: * et incurvavérunt ánimam meam.
}\switchcolumn\portugues{
Eles preparam laços para os meus pés: * e curvaram a minha alma.
}\switchcolumn*\latim{
Fodérunt ante fáciem meam fóveam: * et incidérunt in eam.
}\switchcolumn\portugues{
Cavaram ante mim uma cova: * e caíram nela.
}\switchcolumn*\latim{
Parátum cor meum, Deus, parátum cor meum: * cantábo, et psalmum dicam.
}\switchcolumn\portugues{
Meu coração, ó Deus, está preparado: * cantarei e entoarei salmos.
}\switchcolumn*\latim{
Exsúrge, glória mea, exsúrge, psaltérium et cíthara: * exsúrgam dilúculo.
}\switchcolumn\portugues{
Levanta-te, glória minha, levanta-te, saltério e cítara: * levantar-me-ei ao amanhecer.
}\switchcolumn*\latim{
Confitébor tibi in pópulis, Dómine: * et psalmum dicam tibi in géntibus:
}\switchcolumn\portugues{
Louvar-Vos-ei entre os povos, ó Senhor: * e entoar-Vos-ei salmos entre as gentes.
}\switchcolumn*\latim{
Quóniam magnificáta est usque ad cælos misericórdia tua, * et usque ad nubes véritas tua.
}\switchcolumn\portugues{
Porque a vossa misericórdia foi exaltada até aos céus * e a vossa verdade até às nuvens.
}\switchcolumn*\latim{
Exaltáre super cælos, Deus: * et super omnem terram glória tua.
}\switchcolumn\portugues{
Exaltai-Vos sobre os céus, ó Deus: * e a vossa glória acima de toda a terra.
}\end{paracol}


\subsectioninfo{Salmo 57}{Si vere utique justitiam loquimini}\label{salmo57}
\begin{paracol}{2}\latim{
\rlettrine{S}{i} vere útique justítiam loquímini: * recta judicáte, fílii hóminum.
}\switchcolumn\portugues{
\rlettrine{S}{e} veramente falais justiça: * julgai com rectidão, ó filhos dos homens.
}\switchcolumn*\latim{
Étenim in corde iniquitátes operámini: * in terra injustítias manus vestræ concínnant.
}\switchcolumn\portugues{
De facto, obrais iniquidade no vosso coração: * e as vossas mãos tramam injustiças na terra.
}\switchcolumn*\latim{
Alienáti sunt peccatóres a vulva, erravérunt ab útero: * locúti sunt falsa.
}\switchcolumn\portugues{
Os pecadores alienaram-se desde o ventre, vaguearam desde o útero: * disseram falsidades.
}\switchcolumn*\latim{
Furor illis secúndum similitúdinem serpéntis: * sicut áspidis surdæ, et obturántis aures suas,
}\switchcolumn\portugues{
Sua loucura é semelhante à da serpente: * e à da surda áspide, que fecha os seus ouvidos,
}\switchcolumn*\latim{
Quæ non exáudiet vocem incantántium: * et venéfici incantántis sapiénter.
}\switchcolumn\portugues{
Que não ouve a voz dos encantadores: * nem a do mago que encanta segundo a sua arte.
}\switchcolumn*\latim{
Deus cónteret dentes eórum in ore ipsórum: * molas leónum confrínget Dóminus.
}\switchcolumn\portugues{
Deus quebrar-lhes-á os dentes na sua boca: * o Senhor quebrará as queixadas desses leões.
}\switchcolumn*\latim{
Ad níhilum devénient tamquam aqua decúrrens: * inténdit arcum suum donec infirméntur.
}\switchcolumn\portugues{
Serão reduzidos a nada como água que passa: * curvará o seu arco até que sejam abatidos.
}\switchcolumn*\latim{
Sicut cera, quæ fluit, auferéntur: * supercécidit ignis, et non vidérunt solem.
}\switchcolumn\portugues{
Como a cera que se derrete serão destruídos: * caiu fogo em cima deles e não viram mais o sol.
}\switchcolumn*\latim{
Priúsquam intellégerent spinæ vestræ rhamnum: * sicut vivéntes, sic in ira absórbet eos.
}\switchcolumn\portugues{
Antes que os vossos espinhos se convertam num arbusto: * Ele devorá-los-á na sua ira ainda vivos.
}\switchcolumn*\latim{
Lætábitur justus cum víderit vindíctam: * manus suas lavábit in sánguine peccatóris.
}\switchcolumn\portugues{
Alegrar-se-á o justo ao ver a vingança: * lavará as suas mãos no sangue do pecador.
}\switchcolumn*\latim{
Et dicet homo: si útique est fructus justo: * útique est Deus júdicans eos in terra.
}\switchcolumn\portugues{
O homem dirá: se de certo há fruto para o justo: * de certo há um Deus que os julga sobre a terra.
}\end{paracol}


\subsectioninfo{Salmo 58}{Eripe me de inimicis meis}\label{salmo58}
\begin{paracol}{2}\latim{
\slettrine{É}{ripe} me de inimícis meis, Deus meus: * et ab insurgéntibus in me líbera me.
}\switchcolumn\portugues{
\rlettrine{S}{alvai-me,} meu Deus, dos meus inimigos: * e livrai-me dos que se levantam contra mim.
}\switchcolumn*\latim{
Éripe me de operántibus iniquitátem: * et de viris sánguinum salva me.
}\switchcolumn\portugues{
Livrai-me dos que praticam a iniquidade: * e salvai-me dos homens sanguinários.
}\switchcolumn*\latim{
Quia ecce cepérunt ánimam meam: * irruérunt in me fortes.
}\switchcolumn\portugues{
Pois eis que tomaram a minha alma: * vieram sobre mim homens fortes.
}\switchcolumn*\latim{
Neque iníquitas mea, neque peccátum meum, Dómine: * sine iniquitáte cucúrri, et diréxi.
}\switchcolumn\portugues{
Não, por minha iniquidade ou por pecado meu, ó Senhor: * sem iniquidade segui e ordenei os meus passos.
}\switchcolumn*\latim{
Exsúrge in occúrsum meum, et vide: * et Tu, Dómine, Deus virtútum, Deus Israël,
}\switchcolumn\portugues{
Levantai-Vos em meu encontro e considerai: * e Vós, Senhor, Deus dos exércitos, Deus de Israel,
}\switchcolumn*\latim{
Inténde ad visitándas omnes gentes: * non misereáris ómnibus, qui operántur iniquitátem.
}\switchcolumn\portugues{
Cuidai de visitar todas as gentes: * não useis de piedade com todos os que obram iniquidade.
}\switchcolumn*\latim{
Converténtur ad vésperam: et famem patiéntur ut canes, * et circuíbunt civitátem.
}\switchcolumn\portugues{
Retornarão à tarde e terão fome como cães: * e rodearão a cidade.
}\switchcolumn*\latim{
Ecce, loquéntur in ore suo, et gládius in lábiis eórum: * quóniam quis audívit?
}\switchcolumn\portugues{
Eis que falarão com sua boca e uma espada estará nos seus lábios: * porque quem ouviu?
}\switchcolumn*\latim{
Et Tu, Dómine, deridébis eos: * ad níhilum dedúces omnes gentes.
}\switchcolumn\portugues{
Vós, ó Senhor, zombareis deles: * reduzireis a nada todas as gentes.
}\switchcolumn*\latim{
Fortitúdinem meam ad Te custódiam, quia, Deus, suscéptor meus es: * Deus meus, misericórdia ejus prævéniet me.
}\switchcolumn\portugues{
Depositarei em Vós a minha fortaleza, pois, ó Deus, sois o meu defensor: * a misericórdia do meu Deus antecipar-se-á.
}\switchcolumn*\latim{
Deus osténdet mihi super inimícos meos, ne occídas eos: * nequándo obliviscántur pópuli mei.
}\switchcolumn\portugues{
Deus dar-me-á a ver sobre os meus inimigos, não os mateis: * para que se não esqueça o meu povo.
}\switchcolumn*\latim{
Dispérge illos in virtúte tua: * et depóne eos, protéctor meus, Dómine:
}\switchcolumn\portugues{
Dispersai-os com vosso poder: * e os abatei, ó Senhor, protector meu:
}\switchcolumn*\latim{
Delíctum oris eórum, sermónem labiórum ipsórum: * et comprehendántur in supérbia sua.
}\switchcolumn\portugues{
Pelo pecado da sua boca, pelas palavras dos seus lábios: * e fiquem presos na sua mesma soberba.
}\switchcolumn*\latim{
Et de exsecratióne et mendácio annuntiabúntur in consummatióne: * in ira consummatiónis, et non erunt.
}\switchcolumn\portugues{
Publicar-se-ão as suas execrações e mentiras, no dia da consumação: * serão convencidos pela vossa ira e não subsistirão mais.
}\switchcolumn*\latim{
Et scient quia Deus dominábitur Jacob: * et fínium terræ.
}\switchcolumn\portugues{
Saberão que Deus reinará sobre Jacob: * e até aos confins da terra.
}\switchcolumn*\latim{
Converténtur ad vésperam: et famem patiéntur ut canes, * et circuíbunt civitátem.
}\switchcolumn\portugues{
Retornarão à tarde e terão fome como cães, * e rodearão a cidade.
}\switchcolumn*\latim{
Ipsi dispergéntur ad manducándum: * si vero non fúerint saturáti, et murmurábunt.
}\switchcolumn\portugues{
Andarão dispersos à busca de comer: * e, se não forem veramente saciados, murmurarão.
}\switchcolumn*\latim{
Ego autem cantábo fortitúdinem tuam: * et exsultábo mane misericórdiam tuam.
}\switchcolumn\portugues{
Eu, porém, cantarei a vossa fortaleza: * e celebrarei com alegria desde manhã a vossa misericórdia.
}\switchcolumn*\latim{
Quia factus es suscéptor meus, * et refúgium meum, in die tribulatiónis meæ.
}\switchcolumn\portugues{
Pois Vos fizestes meu protector, * e meu refúgio no dia da minha tribulação.
}\switchcolumn*\latim{
Adjútor meus, tibi psallam, quia, Deus, suscéptor meus es: * Deus meus, misericórdia mea.
}\switchcolumn\portugues{
Vos cantarei, protector meu, pois, ó Deus, sois o meu defensor: * Deus meu, misericórdia minha.
}\end{paracol}


\subsectioninfo{Salmo 59}{Deus, repulisti nos}\label{salmo59}
\begin{paracol}{2}\latim{
\rlettrine{D}{eus,} repulísti nos, et destruxísti nos: * irátus es, et misértus es nobis.
}\switchcolumn\portugues{
\rlettrine{D}{eus,} repelistes-nos e destruístes-nos: * Vos irastes, porém, tivestes piedade de nós.
}\switchcolumn*\latim{
Commovísti terram, et conturbásti eam: * sana contritiónes ejus, quia commóta est.
}\switchcolumn\portugues{
Fizestes estremecer a terra e a conturbastes: * sarai as suas chagas, pois está abalada.
}\switchcolumn*\latim{
Ostendísti pópulo tuo dura: * potásti nos vino compunctiónis.
}\switchcolumn\portugues{
Mostrastes ao vosso povo dificuldades: * destes-nos a beber o vinho da amargura.
}\switchcolumn*\latim{
Dedísti metuéntibus Te significatiónem: * ut fúgiant a fácie arcus:
}\switchcolumn\portugues{
Destes aos que Vos temem um sinal: * para que fujam à face do arco:
}\switchcolumn*\latim{
Ut liberéntur dilécti tui: * salvum fac déxtera tua, et exáudi me.
}\switchcolumn\portugues{
Para que sejam livres os vossos amados: * salvai-me com vossa dextra e ouvi-me.
}\switchcolumn*\latim{
Deus locútus est in sancto suo: * lætábor, et partíbor Síchimam: et convállem tabernaculórum metíbor.
}\switchcolumn\portugues{
Deus falou no seu santuário, alegrar-me-ei: * e repartirei a Siquém e medirei o vale dos Tabernáculos.
}\switchcolumn*\latim{
Meus est Gálaad, et meus est Manásses: * et Éphraim fortitúdo cápitis mei.
}\switchcolumn\portugues{
Meu é Galaad e meu é Manassés: * e Efraim é a força da minha cabeça.
}\switchcolumn*\latim{
Juda rex meus: * Moab olla spei meæ.
}\switchcolumn\portugues{
Judá é o meu rei: * o Moab é o vaso da minha esperança.
}\switchcolumn*\latim{
In Idumǽam exténdam calceaméntum meum: * mihi alienígenæ súbditi sunt.
}\switchcolumn\portugues{
Estenderei o meu calçado sobre a Idumeia: * os estrangeiros estar-me-ão sujeitos.
}\switchcolumn*\latim{
Quis dedúcet me in civitátem munítam? * Quis dedúcet me usque in Idumǽam?
}\switchcolumn\portugues{
Quem me conduzirá à cidade fortificada? * Quem me conduzirá até à Idumeia?
}\switchcolumn*\latim{
Nonne Tu, Deus, qui repulísti nos? * Et non egrediéris, Deus, in virtútibus nostris?
}\switchcolumn\portugues{
Não fostes Vós, ó Deus, que nos repelistes? * Não vireis Vós, ó Deus, com os nossos exércitos?
}\switchcolumn*\latim{
Da nobis auxílium de tribulatióne: * quia vana salus hóminis.
}\switchcolumn\portugues{
Dai-nos socorro na tribulação: * pois é vã a salvação do homem.
}\switchcolumn*\latim{
In Deo faciémus virtútem: * et ipse ad níhilum dedúcet tribulántes nos.
}\switchcolumn\portugues{
Com Deus faremos proezas: * e Ele mesmo reduzirá a nada os que nos afligem.
}\end{paracol}


\subsectioninfo{Salmo 60}{Exaudi, Deus, deprecationem meam}\label{salmo60}
\begin{paracol}{2}\latim{
\rlettrine{E}{xáudi,} Deus, deprecatiónem meam: * inténde oratióni meæ.
}\switchcolumn\portugues{
\rlettrine{O}{uvi,} ó Deus, a minha súplica: * atendei à minha oração.
}\switchcolumn*\latim{
A fínibus terræ ad Te clamávi: * dum anxiarétur cor meum, in petra exaltásti me.
}\switchcolumn\portugues{
Dos confins da terra clamei a Vós: * quando o meu coração estava angustiado, numa rocha me erguestes.
}\switchcolumn*\latim{
Deduxísti me, quia factus es spes mea: * turris fortitúdinis a fácie inimíci.
}\switchcolumn\portugues{
Guiastes-me, pois Vos fizestes a minha esperança: * uma torre sólida contra o inimigo.
}\switchcolumn*\latim{
Inhabitábo in tabernáculo tuo in sǽcula: * prótegar in velaménto alárum tuárum.
}\switchcolumn\portugues{
Habitarei para sempre no vosso tabernáculo: * abrigar-me-ei à sombra de vossas asas.
}\switchcolumn*\latim{
Quóniam Tu, Deus meus, exaudísti oratiónem meam: * dedísti hereditátem timéntibus nomen tuum.
}\switchcolumn\portugues{
Porque Vós, Deus meu, ouvistes a minha oração: * destes uma herança aos que temem o vosso nome.
}\switchcolumn*\latim{
Dies super dies regis adícies: * annos ejus usque in diem generatiónis et generatiónis.
}\switchcolumn\portugues{
Acrescentareis dias aos dias do Rei: * os seus anos durarão de geração em geração.
}\switchcolumn*\latim{
Pérmanet in ætérnum in conspéctu Dei: * misericórdiam et veritátem ejus quis requíret?
}\switchcolumn\portugues{
Ele permanece eternamente na presença de Deus: * quem buscará a sua misericórdia e verdade?
}\switchcolumn*\latim{
Sic psalmum dicam nómini tuo in sǽculum sǽculi: * ut reddam vota mea de die in diem.
}\switchcolumn\portugues{
Assim cantarei um salmo ao vosso nome pelos séculos dos séculos: * para cumprir sem cessar os meus votos.
}\end{paracol}


\subsectioninfo{Salmo 61}{Nonne Deo subjecta erit anima mea}\label{salmo61}
\begin{paracol}{2}\latim{
\rlettrine{N}{onne} Deo subjécta erit ánima mea? * Ab ipso enim salutáre meum.
}\switchcolumn\portugues{
\rlettrine{P}{orventura} a minha alma não há-de estar sujeita a Deus? * Pois vem d’Ele a minha salvação.
}\switchcolumn*\latim{
Nam et ipse Deus meus, et salutáris meus: * suscéptor meus, non movébor ámplius.
}\switchcolumn\portugues{
Porquanto Ele é o meu Deus e o meu Salvador: * é minha defesa, não serei jamais abalado.
}\switchcolumn*\latim{
Quoúsque irrúitis in hóminem? * Interfícitis univérsi vos: tamquam paríeti inclináto et macériæ depúlsæ?
}\switchcolumn\portugues{
Até quando um homem confrontareis? * Todos matais, como a uma parede desnivelada e a um muro abalado?
}\switchcolumn*\latim{
Verúmtamen prétium meum cogitavérunt repéllere, cucúrri in siti: * ore suo benedicébant, et corde suo maledicébant.
}\switchcolumn\portugues{
Certamente meditaram tirar-me a minha dignidade, sedento corri: * com sua boca me bendiziam e com seu coração me maldiziam.
}\switchcolumn*\latim{
Verúmtamen Deo subjécta esto, ánima mea: * quóniam ab ipso patiéntia mea.
}\switchcolumn\portugues{
Porém, tu, ó alma minha, conserva-te sujeita a Deus: * porque d’Ele é que vem a minha paciência.
}\switchcolumn*\latim{
Quia ipse Deus meus, et salvátor meus: * adjútor meus, non emigrábo.
}\switchcolumn\portugues{
Pois Ele é meu Deus e meu salvador: * é minha defesa, não serei movido.
}\switchcolumn*\latim{
In Deo salutáre meum, et glória mea: * Deus auxílii mei, et spes mea in Deo est.
}\switchcolumn\portugues{
Em Deus está a minha salvação e a minha glória: * Deus é o meu socorro e em Deus está a minha esperança.
}\switchcolumn*\latim{
Speráte in eo omnis congregátio pópuli, effúndite coram illo corda vestra: * Deus adjútor noster in ætérnum.
}\switchcolumn\portugues{
Esperai n’Ele todos os constituintes do povo, expandi-Lhe vossos corações: * Deus é o nosso protector eternamente.
}\switchcolumn*\latim{
Verúmtamen vani fílii hóminum, mendáces fílii hóminum in statéris: * ut decípiant ipsi de vanitáte in idípsum.
}\switchcolumn\portugues{
Vãos, porém, são os filhos dos homens, mentirosos os filhos dos homens em balanças: * por vaidade conspiram juntos enganos.
}\switchcolumn*\latim{
Nolíte speráre in iniquitáte, et rapínas nolíte concupíscere: * divítiæ si áffluant, nolíte cor appónere.
}\switchcolumn\portugues{
Não confieis na iniquidade, nem cobiceis rapinas: * se abundardes em riquezas, não queirais pôr nelas vosso coração.
}\switchcolumn*\latim{
Semel locútus est Deus, duo hæc audívi, quia potéstas Dei est, et tibi, Dómine, misericórdia: * quia Tu reddes unicuíque juxta ópera sua.
}\switchcolumn\portugues{
Deus falou uma vez, estas duas cousas ouvi: que o poder pertence a Deus e a Vós, ó Senhor, a misericórdia: * pois dareis a cada um segundo as suas obras.
}\end{paracol}


\subsectioninfo{Salmo 62}{Deus, Deus meus, ad Te de luce vigilo}\label{salmo62}
\begin{paracol}{2}\latim{
\rlettrine{D}{eus,} Deus meus, * ad Te de luce vígilo.
}\switchcolumn\portugues{
\slettrine{Ó}{} Deus, ó meu Deus, * a Vós vigio desde a aurora.
}\switchcolumn*\latim{
Sitívit in Te ánima mea, * quam multiplíciter tibi caro mea.
}\switchcolumn\portugues{
De Vós está sedenta a minha alma, * e a minha carne, de quantas maneiras está!
}\switchcolumn*\latim{
In terra desérta, et ínvia, et inaquósa: * sic in sancto appárui tibi, ut vidérem virtútem tuam, et glóriam tuam.
}\switchcolumn\portugues{
Em terra deserta, intransitável e sem água: * no santuário me apresentei a Vós, para contemplar o vosso poder e a vossa glória.
}\switchcolumn*\latim{
Quóniam mélior est misericórdia tua super vitas: * lábia mea laudábunt Te.
}\switchcolumn\portugues{
Porque a vossa misericórdia é melhor que as vidas: * os meus lábios Vos louvarão.
}\switchcolumn*\latim{
Sic benedícam Te in vita mea: * et in nómine tuo levábo manus meas.
}\switchcolumn\portugues{
Assim Vos bendirei em minha vida: * e, invocando o vosso nome, levantarei as minhas mãos.
}\switchcolumn*\latim{
Sicut ádipe et pinguédine repleátur ánima mea: * et lábiis exsultatiónis laudábit os meum.
}\switchcolumn\portugues{
Como de banha e gordura seja farta a minha alma: * e com lábios de júbilo louvar-Vos-á a minha boca.
}\switchcolumn*\latim{
Si memor fui tui super stratum meum, in matutínis meditábor in Te: * quia fuísti adjútor meus.
}\switchcolumn\portugues{
Se me tenho lembrado de Vós sobre o meu leito, nas madrugadas meditarei em Vós: * pois fostes o meu defensor.
}\switchcolumn*\latim{
Et in velaménto alárum tuárum exsultábo, adhǽsit ánima mea post Te: * me suscépit déxtera tua.
}\switchcolumn\portugues{
À sombra de vossas asas me regozijarei, a minha alma está presa a Vós: * a vossa dextra me acolheu.
}\switchcolumn*\latim{
Ipsi vero in vanum quæsiérunt ánimam meam, introíbunt in inferióra terræ: * tradéntur in manus gládii, partes vúlpium erunt.
}\switchcolumn\portugues{
Eles em vão procuraram tirar-me a vida, entrarão nas profundidades da terra: * serão entregues ao poder da espada e virão a ser presa das raposas.
}\switchcolumn*\latim{
Rex vero lætábitur in Deo, laudabúntur omnes qui jurant in eo: * quia obstrúctum est os loquéntium iníqua.
}\switchcolumn\portugues{
Entretanto o rei alegrar-se-á em Deus, louvados serão todos os que juram por Ele: * pois foi fechada a boca aos que proferiam iniquidades.
}\end{paracol}


\subsectioninfo{Salmo 63}{Exaudi, Deus, orationem meam cum deprecor}\label{salmo63}
\begin{paracol}{2}\latim{
\rlettrine{E}{xáudi,} Deus, oratiónem meam cum déprecor: * a timóre inimíci éripe ánimam meam.
}\switchcolumn\portugues{
\rlettrine{O}{uvi,} ó Deus, a minha oração quando Vos rogo: * livrai a minha alma do temor do inimigo.
}\switchcolumn*\latim{
Protexísti me a convéntu malignántium: * a multitúdine operántium iniquitátem.
}\switchcolumn\portugues{
Defendestes-me da conspiração dos malignos: * da multidão dos que praticam a iniquidade.
}\switchcolumn*\latim{
Quia exacuérunt ut gládium linguas suas: * intendérunt arcum rem amáram, ut sagíttent in occúltis immaculátum.
}\switchcolumn\portugues{
Pois aguçaram as suas línguas como a espada: * curvaram o arco envenenado, para de emboscada assetear o inocente.
}\switchcolumn*\latim{
Súbito sagittábunt eum, et non timébunt: * firmavérunt sibi sermónem nequam.
}\switchcolumn\portugues{
De súbito o assetearão sem temor algum: * obstinaram-se na sua depravada resolução.
}\switchcolumn*\latim{
Narravérunt ut abscónderent láqueos: * dixérunt: quis vidébit eos?
}\switchcolumn\portugues{
Convencionaram esconder laços: * e disseram: quem os verá?
}\switchcolumn*\latim{
Scrutáti sunt iniquitátes: * defecérunt scrutántes scrutínio.
}\switchcolumn\portugues{
Inventaram crimes: * cansaram-se a esquadrinhar manhas.
}\switchcolumn*\latim{
Accédet homo ad cor altum: * et exaltábitur Deus.
}\switchcolumn\portugues{
O homem penetrará até ao fundo do coração: * e Deus será exaltado.
}\switchcolumn*\latim{
Sagíttæ parvulórum factæ sunt plagæ eórum: * et infirmátæ sunt contra eos linguæ eórum.
}\switchcolumn\portugues{
As flechas das crianças são as suas feridas: * e as suas línguas contra eles perderam a força.
}\switchcolumn*\latim{
Conturbáti sunt omnes qui vidébant eos: * et tímuit omnis homo.
}\switchcolumn\portugues{
Todos os que os viam ficaram abalados: * e todo o homem temeu.
}\switchcolumn*\latim{
Et annuntiavérunt ópera Dei, * et facta ejus intellexérunt.
}\switchcolumn\portugues{
Anunciaram as obras de Deus, * e compreenderam os seus actos.
}\switchcolumn*\latim{
Lætábitur justus in Dómino, et sperábit in eo, * et laudabúntur omnes recti corde.
}\switchcolumn\portugues{
Alegrar-se-á o justo no senhor e esperará n’Ele, * e serão louvados todos os de coração recto.
}\end{paracol}


\subsectioninfo{Salmo 64}{Te decet hymnus}\label{salmo64}
\begin{paracol}{2}\latim{
\rlettrine{T}{e} decet hymnus, Deus, in Sion: * et tibi reddétur votum in Jerúsalem.
}\switchcolumn\portugues{
\rlettrine{A}{} Vós, ó Deus, são devidos os hinos em Sião: * e a Vós serão prestados votos em Jerusalém.
}\switchcolumn*\latim{
Exáudi oratiónem meam: * ad Te omnis caro véniet.
}\switchcolumn\portugues{
Ouvi a minha oração: * a Vós toda a carne virá.
}\switchcolumn*\latim{
Verba iniquórum prævaluérunt super nos: * et impietátibus nostris Tu propitiáberis.
}\switchcolumn\portugues{
As palavras dos iníquos prevaleceram sobre nós: * mas Vós perdoareis as nossas impiedades.
}\switchcolumn*\latim{
Beátus, quem elegísti, et assumpsísti: * inhabitábit in átriis tuis.
}\switchcolumn\portugues{
Bem-aventurado o que elegestes e adoptastes: * ele habitará nos vossos átrios.
}\switchcolumn*\latim{
Replébimur in bonis domus tuæ: * sanctum est templum tuum, mirábile in æquitáte.
}\switchcolumn\portugues{
Seremos cheios dos bens da vossa casa: * santo é o vosso templo, maravilhoso em equidade.
}\switchcolumn*\latim{
Exáudi nos, Deus, salutáris noster, * spes ómnium fínium terræ, et in mari longe.
}\switchcolumn\portugues{
Ouvi-nos, ó Deus, Salvador nosso, * esperança de todos os confins da terra e no longínquo mar.
}\switchcolumn*\latim{
Prǽparans montes in virtúte tua, accínctus poténtia: * qui contúrbas profúndum maris sonum flúctuum ejus.
}\switchcolumn\portugues{
Dais firmeza aos montes com vossa força, cingido de poder: * conturbais o fundo do mar, o estrondo das suas ondas.
}\switchcolumn*\latim{
Turbabúntur gentes, et timébunt qui hábitant términos a signis tuis: * éxitus matutíni, et véspere delectábis.
}\switchcolumn\portugues{
Perturbar-se-ão as gentes e os que habitam os confins da terra temerão aos vossos prodígios: * dareis alegria às saídas da manhã e da tarde.
}\switchcolumn*\latim{
Visitásti terram, et inebriásti eam: * multiplicásti locupletáre eam.
}\switchcolumn\portugues{
Visitastes a terra e a embriagastes: * multiplicastes suas riquezas.
}\switchcolumn*\latim{
Flumen Dei replétum est aquis, parásti cibum illórum: * quóniam ita est præparátio ejus.
}\switchcolumn\portugues{
O rio de Deus encheu-se de águas, preparastes o seu sustento: * porque tal é a sua disposição.
}\switchcolumn*\latim{
Rivos ejus inébria, multíplica genímina ejus: * in stillicídiis ejus lætábitur gérminans.
}\switchcolumn\portugues{
Embriagai os seus ribeiros, multiplicai as suas produções: * com o destilar do orvalho alegrar-se-á nos frutos.
}\switchcolumn*\latim{
Benedíces corónæ anni benignitátis tuæ: * et campi tui replebúntur ubertáte.
}\switchcolumn\portugues{
Bendireis a coroa do ano da vossa bondade: * e os vossos campos se encherão de abundância.
}\switchcolumn*\latim{
Pinguéscent speciósa desérti: * et exsultatióne colles accingéntur.
}\switchcolumn\portugues{
O deserto ficará viçoso: * e as colinas vestir-se-ão de alegria.
}\switchcolumn*\latim{
Indúti sunt aríetes óvium, et valles abundábunt fruménto: * clamábunt, étenim hymnum dicent.
}\switchcolumn\portugues{
Os carneiros dos rebanhos se agasalharão e os vales estarão cheios de trigo: * clamarão e sim, cantarão hinos.
}\end{paracol}


\subsectioninfo{Salmo 65}{Jubilate Deo, omnis terra}\label{salmo65}
\begin{paracol}{2}\latim{
\qlettrine{J}{ubiláte} Deo, omnis terra, psalmum dícite nómini ejus: * date glóriam laudi ejus.
}\switchcolumn\portugues{
\rlettrine{A}{clamai} a Deus, habitantes todos da terra, cantai salmos ao seu nome: * tributai-Lhe gloriosos louvores.
}\switchcolumn*\latim{
Dícite Deo: quam terribília sunt ópera tua, Dómine! * in multitúdine virtútis tuæ mentiéntur tibi inimíci tui.
}\switchcolumn\portugues{
Dizei a Deus: quão terríveis são as vossas obras, ó Senhor! * No vosso imenso poder vossos inimigos Vos dirigem mentiras.
}\switchcolumn*\latim{
Omnis terra adóret Te, et psallat tibi: * psalmum dicat nómini tuo.
}\switchcolumn\portugues{
Toda a terra Vos adore e Vos cante: * que cante salmos ao vosso nome.
}\switchcolumn*\latim{
Veníte, et vidéte ópera Dei: * terríbilis in consíliis super fílios hóminum.
}\switchcolumn\portugues{
Vinde e vede as obras de Deus: * terrível nos planos sobre os filhos dos homens.
}\switchcolumn*\latim{
Qui convértit mare in áridam, in flúmine pertransíbunt pede: * ibi lætábimur in ipso.
}\switchcolumn\portugues{
Ele converteu o mar em terra seca, pelo rio passarão a pé: * ali com Ele nos alegraremos.
}\switchcolumn*\latim{
Qui dominátur in virtúte sua in ætérnum, óculi ejus super gentes respíciunt: * qui exásperant non exalténtur in semetípsis.
}\switchcolumn\portugues{
Ele domina pelo seu poder para sempre, os seus olhos contemplam as gentes: * os que o irritam se não exaltem a si mesmos.
}\switchcolumn*\latim{
Benedícite, gentes, Deum nostrum: * et audítam fácite vocem laudis ejus,
}\switchcolumn\portugues{
Bendizei, ó gentes, o nosso Deus: * e fazei que se ouça a voz do seu louvor,
}\switchcolumn*\latim{
Qui pósuit ánimam meam ad vitam: * et non dedit in commotiónem pedes meos.
}\switchcolumn\portugues{
É Ele que tem conservado a minha vida: * e não permitiu que meus pés vacilassem.
}\switchcolumn*\latim{
Quóniam probásti nos, Deus: * igne nos examinásti, sicut examinátur argéntum.
}\switchcolumn\portugues{
Porquanto nos provastes, ó Deus: * com fogo nos examinastes, como se examina a prata.
}\switchcolumn*\latim{
Induxísti nos in láqueum, posuísti tribulatiónes in dorso nostro: * imposuísti hómines super cápita nostra.
}\switchcolumn\portugues{
Deixastes-nos cair no laço, carregastes de tribulações as nossas costas: * pusestes homens sobre as nossas cabeças.
}\switchcolumn*\latim{
Transívimus per ignem et aquam: * et eduxísti nos in refrigérium.
}\switchcolumn\portugues{
Passámos pelo fogo e pela água: * mas conduzistes-nos a um lugar fresco.
}\switchcolumn*\latim{
Introíbo in domum tuam in holocáustis: * reddam tibi vota mea, quæ distinxérunt lábia mea.
}\switchcolumn\portugues{
Entrarei na vossa casa com holocaustos: * pagar-Vos-ei os meus votos, que meus lábios pronunciaram.
}\switchcolumn*\latim{
Et locútum est os meum, * in tribulatióne mea.
}\switchcolumn\portugues{
Que proferiu a minha boca, * na minha tribulação.
}\switchcolumn*\latim{
Holocáusta medulláta ófferam tibi cum incénso aríetum: * ófferam tibi boves cum hircis.
}\switchcolumn\portugues{
Oferecer-Vos-ei holocaustos gordos com o fumo dos carneiros: * oferecer-Vos-ei bois com cabritos.
}\switchcolumn*\latim{
Veníte, audíte, et narrábo, omnes, qui timétis Deum: * quanta fecit ánimæ meæ.
}\switchcolumn\portugues{
Vinde, ouvi e narrarei, a todos vós que temeis a Deus: * o que Ele fez à minha alma.
}\switchcolumn*\latim{
Ad ipsum ore meo clamávi, * et exaltávi sub lingua mea.
}\switchcolumn\portugues{
A Ele com minha boca clamei, * e com minha língua O exaltei.
}\switchcolumn*\latim{
Iniquitátem si aspéxi in corde meo, * non exáudiet Dóminus.
}\switchcolumn\portugues{
Se tivesse visto a iniquidade no meu coração, * o Senhor me não ouviria.
}\switchcolumn*\latim{
Proptérea exaudívit Deus, * et atténdit voci deprecatiónis meæ.
}\switchcolumn\portugues{
Por isso me ouviu Deus, * e atendeu à voz da minha súplica.
}\switchcolumn*\latim{
Benedíctus Deus, * qui non amóvit oratiónem meam, et misericórdiam suam a me.
}\switchcolumn\portugues{
Bendito seja Deus, * que não rejeitou a minha oração, nem retirou de mim a sua misericórdia.
}\end{paracol}


\subsectioninfo{Salmo 66}{Deus misereatur nostri}\label{salmo66}
\begin{paracol}{2}\latim{
\rlettrine{D}{eus} misereátur nostri, et benedícat nobis: * illúminet vultum suum super nos, et misereátur nostri.
}\switchcolumn\portugues{
\rlettrine{D}{eus} tenha piedade de nós e nos abençoe: * faça resplandecer a sua face sobre nós e tenha de nós piedade.
}\switchcolumn*\latim{
Ut cognoscámus in terra viam tuam, * in ómnibus géntibus salutáre tuum.
}\switchcolumn\portugues{
Para que conheçamos na terra o vosso caminho, * e entre todas as gentes a vossa salvação.
}\switchcolumn*\latim{
Confiteántur tibi pópuli, Deus: * confiteántur tibi pópuli omnes.
}\switchcolumn\portugues{
Que os povos Vos glorifiquem, ó Deus: * que todos os povos Vos glorifiquem.
}\switchcolumn*\latim{
Læténtur et exsúltent gentes: * quóniam júdicas pópulos in æquitáte, et gentes in terra dírigis.
}\switchcolumn\portugues{
Alegrem-se as gentes e regozijem-se: * porquanto julgais os povos com equidade e dirigis as gentes sobre a terra.
}\switchcolumn*\latim{
Confiteántur tibi pópuli, Deus, confiteántur tibi pópuli omnes: * terra dedit fructum suum.
}\switchcolumn\portugues{
Que os povos Vos glorifiquem, ó Deus, que os povos todos Vos glorifiquem: * a terra deu o seu fruto.
}\switchcolumn*\latim{
Benedícat nos Deus, Deus noster, benedícat nos Deus: * et métuant eum omnes fines terræ.
}\switchcolumn\portugues{
Abençoe-nos Deus, o nosso Deus, abençoe-nos Deus: * e temam-n’O todos os confins da terra!
}\end{paracol}


\subsectioninfo{Salmo 67}{Exsurgat Deus}\label{salmo67}
\begin{paracol}{2}\latim{
\rlettrine{E}{xsúrgat} Deus, et dissipéntur inimíci ejus, * et fúgiant qui odérunt eum, a fácie ejus.
}\switchcolumn\portugues{
\rlettrine{L}{evante-se} Deus e sejam dispersos os seus inimigos, * e da sua presença fujam os que o odeiam.
}\switchcolumn*\latim{
Sicut déficit fumus, defíciant: * sicut fluit cera a fácie ignis, sic péreant peccatóres a fácie Dei.
}\switchcolumn\portugues{
Assim como se desvanece o fumo, se desvaneçam: * assim como se derrete a cera diante do fogo, assim pereçam os pecadores ante Deus.
}\switchcolumn*\latim{
Et justi epuléntur, et exsúltent in conspéctu Dei: * et delecténtur in lætítia.
}\switchcolumn\portugues{
Os justos, porém, banqueteiem-se e regozijem-se na presença de Deus: * e que gozem com alegria.
}\switchcolumn*\latim{
Cantáte Deo, psalmum dícite nómini ejus: * iter fácite ei, qui ascéndit super occásum: \emph{(fit reverentia)} Dóminus nomen illi.
}\switchcolumn\portugues{
Cantai a Deus, cantai salmos ao seu nome: * abri o caminho Àquele que sobe para o ocidente: \emph{(inclinar a cabeça)} o Senhor é o seu nome.
}\switchcolumn*\latim{
Exsultáte in conspéctu ejus: * turbabúntur a fácie ejus, patris orphanórum et júdicis viduárum.
}\switchcolumn\portugues{
Regozijai-vos diante d’Ele: * perturbar-se-ão diante d’Ele, Ele é o pai dos órfãos e o juiz das viúvas.
}\switchcolumn*\latim{
Deus in loco sancto suo: * Deus, qui inhabitáre facit uníus moris in domo:
}\switchcolumn\portugues{
Deus está no seu lugar santo: * é o Deus que faz habitar na casa os solitários:
}\switchcolumn*\latim{
Qui edúcit vinctos in fortitúdine, * simíliter eos qui exásperant, qui hábitant in sepúlcris.
}\switchcolumn\portugues{
Que põe em liberdade os cativos com seu poder, * mesmo aqueles que o irritam, os quais moram nos sepulcros.
}\switchcolumn*\latim{
Deus, cum egrederéris in conspéctu pópuli tui, * cum pertransíres in desérto:
}\switchcolumn\portugues{
Ó Deus, quando saíeis à frente de vosso povo, * quando atravessáveis o deserto:
}\switchcolumn*\latim{
Terra mota est, étenim cæli distillavérunt a fácie Dei Sínai, * a fácie Dei Israël.
}\switchcolumn\portugues{
A terra tremeu e os céus destilaram, ante a face do Deus do Sinai, * diante do Deus de Israel.
}\switchcolumn*\latim{
Plúviam voluntáriam segregábis, Deus, hereditáti tuæ: * et infirmáta est, Tu vero perfecísti eam.
}\switchcolumn\portugues{
Ó Deus, reservastes uma chuva abundante para a vossa herança: * e, quando ela enfraqueceu, Vós a aperfeiçoastes.
}\switchcolumn*\latim{
Animália tua habitábunt in ea: * parásti in dulcédine tua páuperi, Deus.
}\switchcolumn\portugues{
Nela morarão as vossas criaturas: * na vossa bondade, ó Deus, para o pobre fornecestes.
}\switchcolumn*\latim{
Dóminus dabit verbum evangelizántibus, * virtúte multa.
}\switchcolumn\portugues{
O Senhor dará a palavra aos que anunciam a boa nova, * com grande coragem.
}\switchcolumn*\latim{
Rex virtútum dilécti dilécti: * et speciéi domus divídere spólia.
}\switchcolumn\portugues{
Rei dos exércitos será do amado, do amado: * e a formosura da casa repartirá os despojos.
}\switchcolumn*\latim{
Si dormiátis inter médios cleros, pennæ colúmbæ deargentátæ, * et posterióra dorsi ejus in pallóre auri.
}\switchcolumn\portugues{
Se dormirdes no meio de vossos despojos, sereis como as penas prateadas da pomba, * e o brilho flavo do ouro na extremidade do seu dorso.
}\switchcolumn*\latim{
Dum discérnit cæléstis reges super eam, nive dealbabúntur in Selmon: * mons Dei, mons pinguis.
}\switchcolumn\portugues{
Enquanto o Altíssimo dispersa os reis sobre a terra, ficarão brancos com neve em Selmon: * o monte de Deus é um monte farto.
}\switchcolumn*\latim{
Mons coagulátus, mons pinguis: * ut quid suspicámini montes coagulátos?
}\switchcolumn\portugues{
Monte escarpado, monte fecundo: * porém, porque pensais em outros montes escarpados?
}\switchcolumn*\latim{
Mons, in quo beneplácitum est Deo habitáre in eo: * étenim Dóminus habitábit in finem.
}\switchcolumn\portugues{
Um monte em que aprouve a Deus morar: * de facto, lá o Senhor habitará perpetuamente.
}\switchcolumn*\latim{
Currus Dei decem míllibus múltiplex, míllia lætántium: * Dóminus in eis in Sina in sancto.
}\switchcolumn\portugues{
O carro de Deus é assistido por dez milhares, milhares alegram-se: * o Senhor está entre eles em Sinai, no seu santuário.
}\switchcolumn*\latim{
Ascendísti in altum, cepísti captivitátem: * accepísti dona in homínibus.
}\switchcolumn\portugues{
Subistes ao alto, cativos levastes convosco: * pelos homens recebestes dons.
}\switchcolumn*\latim{
Étenim non credéntes, * inhabitáre Dóminum Deum.
}\switchcolumn\portugues{
Mesmo pelos descrentes, * habitava o Senhor Deus.
}\switchcolumn*\latim{
Benedíctus Dóminus die quotídie: * prósperum iter fáciet nobis Deus salutárium nostrórum.
}\switchcolumn\portugues{
Bendito seja o Senhor quotidianamente: * o Deus da nossa salvação fazer-nos-á a jornada próspera.
}\switchcolumn*\latim{
Deus noster, Deus salvos faciéndi: * et Dómini Dómini éxitus mortis.
}\switchcolumn\portugues{
Nosso Deus é o Deus que salva: * e ao Senhor, ao Senhor pertence o livrar da morte.
}\switchcolumn*\latim{
Verúmtamen Deus confrínget cápita inimicórum suórum: * vérticem capílli perambulántium in delíctis suis.
}\switchcolumn\portugues{
Contudo, Deus quebrará as cabeças dos seus inimigos: * a moleira cabeluda dos que passeiam nos seus pecados.
}\switchcolumn*\latim{
Dixit Dóminus: ex Basan convértam, * convértam in profúndum maris:
}\switchcolumn\portugues{
O Senhor disse: de Basã os farei volver, * do fundo do mar volver os farei:
}\switchcolumn*\latim{
Ut intingátur pes tuus in sánguine: * lingua canum tuórum ex inimícis, ab ipso.
}\switchcolumn\portugues{
Para que o teu pé seja mergulhado no sangue: * de teus inimigos e também a língua de teus cães.
}\switchcolumn*\latim{
Vidérunt ingréssus tuos, Deus: * ingréssus Dei mei: regis mei qui est in sancto.
}\switchcolumn\portugues{
Eles viram a vossas procissões, ó Deus: * as procissões do meu Deus: do meu rei, que está no santuário.
}\switchcolumn*\latim{
Prævenérunt príncipes conjúncti psalléntibus: * in médio juvenculárum tympanistriárum.
}\switchcolumn\portugues{
Adiante foram os príncipes, juntamente com os cantores: * no meio das donzelas que tocavam timbales.
}\switchcolumn*\latim{
In ecclésiis benedícite Deo Dómino, * de fóntibus Israël.
}\switchcolumn\portugues{
Nas igrejas bendizei o Senhor Deus, * vós da estirpe de Israel.
}\switchcolumn*\latim{
Ibi Bénjamin adolescéntulus: * in mentis excéssu.
}\switchcolumn\portugues{
Ali estava o jovem Benjamim: * em êxtase mental.
}\switchcolumn*\latim{
Príncipes Juda, duces eórum: * príncipes Zábulon, príncipes Néphtali.
}\switchcolumn\portugues{
Os príncipes de Judá, seus comandantes: * os príncipes de Zabulon, os príncipes de Neftali.
}\switchcolumn*\latim{
Manda, Deus, virtúti tuæ: * confírma hoc, Deus, quod operátus es in nobis.
}\switchcolumn\portugues{
Ó Deus, mostrai o vosso poder: * confirmai, ó Deus, aquilo que fizestes entre nós.
}\switchcolumn*\latim{
A templo tuo in Jerúsalem, * tibi ófferent reges múnera.
}\switchcolumn\portugues{
Desde o vosso templo em Jerusalém, * os reis oferecer-Vos-ão dons.
}\switchcolumn*\latim{
Íncrepa feras arúndinis, congregátio taurórum in vaccis populórum: * ut exclúdant eos, qui probáti sunt argénto.
}\switchcolumn\portugues{
Reprimi essas feras dos canaviais, esses povos congregados como touros entre vacas: * para lançar fora os que foram provados como a prata.
}\switchcolumn*\latim{
Díssipa gentes, quæ bella volunt: vénient legáti ex Ægýpto: * Æthiópia prævéniet manus ejus Deo.
}\switchcolumn\portugues{
Dissipai as gentes que querem guerras: virão embaixadores do Egipto: * a Etiópia adiantar-se-á a estender as mãos para Deus.
}\switchcolumn*\latim{
Regna terræ, cantáte Deo: * psállite Dómino.
}\switchcolumn\portugues{
Reinos da terra, cantai a Deus: * salmodiai ao Senhor.
}\switchcolumn*\latim{
Psállite Deo, qui ascéndit super cælum cæli, * ad Oriéntem.
}\switchcolumn\portugues{
Salmodiai a Deus, que se eleva sobre todos os céus, * para oriente.
}\switchcolumn*\latim{
Ecce dabit voci suæ vocem virtútis, date glóriam Deo super Israël, * magnificéntia ejus, et virtus ejus in núbibus.
}\switchcolumn\portugues{
Eis Ele dará à sua voz força, dai glória a Deus pelo que fez em Israel, * a sua magnificência e o seu poder está nas nuvens.
}\switchcolumn*\latim{
Mirábilis Deus in sanctis suis, Deus Israël ipse dabit virtútem, et fortitúdinem plebi suæ, * benedíctus Deus.
}\switchcolumn\portugues{
Deus é admirável nos seus santos, o Deus de Israel Ele mesmo dará poder e fortaleza ao seu povo, * bendito seja Deus!
}\end{paracol}


\subsectioninfo{Salmo 68}{Salvum me fac, Deus}\label{salmo68}
\begin{paracol}{2}\latim{
\rlettrine{S}{alvum} me fac, Deus: * quóniam intravérunt aquæ usque ad ánimam meam.
}\switchcolumn\portugues{
\rlettrine{S}{alvai-me,} ó Deus: * porque as águas quase inundaram a minha alma.
}\switchcolumn*\latim{
Infíxus sum in limo profúndi: * et non est substántia.
}\switchcolumn\portugues{
Estou atolado num profundo lodo: * e não há nele firmeza.
}\switchcolumn*\latim{
Veni in altitúdinem maris: * et tempéstas demérsit me.
}\switchcolumn\portugues{
Cheguei ao alto mar: * e a tempestade me afundou.
}\switchcolumn*\latim{
Laborávi clamans, raucæ factæ sunt fauces meæ: * defecérunt óculi mei, dum spero in Deum meum.
}\switchcolumn\portugues{
Cansei-me clamando, a minha garganta enrouqueceu-se: * os meus olhos desfaleceram à espera do meu Deus.
}\switchcolumn*\latim{
Multiplicáti sunt super capíllos cápitis mei, * qui odérunt me gratis.
}\switchcolumn\portugues{
Multiplicaram-se mais que os cabelos da minha cabeça, * aqueles que sem razão me aborrecem.
}\switchcolumn*\latim{
Confortáti sunt qui persecúti sunt me inimíci mei injúste: * quæ non rápui, tunc exsolvébam.
}\switchcolumn\portugues{
Tornaram-se fortes os inimigos que injustamente me perseguem: * paguei então o que não tinha roubado.
}\switchcolumn*\latim{
Deus, Tu scis insipiéntiam meam: * et delícta mea a Te non sunt abscóndita.
}\switchcolumn\portugues{
Ó Deus, Vós conheceis a minha insipiência: * e os meus delitos Vos não são ocultos.
}\switchcolumn*\latim{
Non erubéscant in me qui exspéctant Te, Dómine, * Dómine virtútum.
}\switchcolumn\portugues{
Não sejam envergonhados por minha causa os que esperam em Vós, Senhor, * ó Senhor dos exércitos.
}\switchcolumn*\latim{
Non confundántur super me * qui quǽrunt Te, Deus Israël.
}\switchcolumn\portugues{
Não sejam confundidos a meu respeito * aqueles que Vos buscam, ó Deus de Israel.
}\switchcolumn*\latim{
Quóniam propter Te sustínui oppróbrium: * opéruit confúsio fáciem meam.
}\switchcolumn\portugues{
Porque por vossa causa sofri afronta: * foi coberto de confusão o meu rosto.
}\switchcolumn*\latim{
Extráneus factus sum frátribus meis, * et peregrínus fíliis matris meæ.
}\switchcolumn\portugues{
Tornei-me um estranho para meus irmãos, * e um estrangeiro para os filhos de minha mãe.
}\switchcolumn*\latim{
Quóniam zelus domus tuæ comédit me: * et oppróbria exprobrántium tibi cecidérunt super me.
}\switchcolumn\portugues{
Porque o zelo de vossa casa me devorou: * e as ofensas dos que Vos insultavam recaíram sobre mim.
}\switchcolumn*\latim{
Et opérui in jejúnio ánimam meam: * et factum est in oppróbrium mihi.
}\switchcolumn\portugues{
Mortifiquei pelo jejum a minha alma: * e isto tornou-se em vergonha para mim.
}\switchcolumn*\latim{
Et pósui vestiméntum meum cilícium: * et factus sum illis in parábolam.
}\switchcolumn\portugues{
Fiz do cilício a minha vestimenta: * e passei a ser um parábola para eles.
}\switchcolumn*\latim{
Advérsum me loquebántur, qui sedébant in porta: * et in me psallébant qui bibébant vinum.
}\switchcolumn\portugues{
Falavam contra mim os que se sentavam ao portão: * e sobre mim cantavam os que bebiam vinho.
}\switchcolumn*\latim{
Ego vero oratiónem meam ad Te, Dómine: * tempus benepláciti, Deus.
}\switchcolumn\portugues{
Porém eu, ó Senhor, dirigia-Vos a minha oração: * eis o tempo favorável, ó Deus.
}\switchcolumn*\latim{
In multitúdine misericórdiæ tuæ exáudi me, * in veritáte salútis tuæ:
}\switchcolumn\portugues{
Ouvi-me segundo a multidão de vossa misericórdia, * segundo a verdade de vossa salvação:
}\switchcolumn*\latim{
Éripe me de luto, ut non infígar: * líbera me ab iis, qui odérunt me, et de profúndis aquárum.
}\switchcolumn\portugues{
Tirai-me do lodo, para que não fique afogado: * livrai-me daqueles que me odeiam e das águas profundas.
}\switchcolumn*\latim{
Non me demérgat tempéstas aquæ, neque absórbeat me profúndum: * neque úrgeat super me púteus os suum.
}\switchcolumn\portugues{
Não me afogue a tempestade de água, nem me absorva o mar profundo: * nem sobre mim se feche a boca do abysmo.
}\switchcolumn*\latim{
Exáudi me, Dómine, quóniam benígna est misericórdia tua: * secúndum multitúdinem miseratiónum tuárum réspice in me.
}\switchcolumn\portugues{
Ouvi-me, ó Senhor, porque é benigna a vossa misericórdia: * olhai para mim segundo a multidão de vossas misericórdias.
}\switchcolumn*\latim{
Et ne avértas fáciem tuam a púero tuo: * quóniam tríbulor, velóciter exáudi me.
}\switchcolumn\portugues{
Não aparteis de vosso servo o vosso rosto: * ouvi-me prontamente, porque estou angustiado.
}\switchcolumn*\latim{
Inténde ánimæ meæ, et líbera eam: * propter inimícos meos éripe me.
}\switchcolumn\portugues{
Atendei à minha alma e livrai-a: * salvai-me por causa dos meus inimigos.
}\switchcolumn*\latim{
Tu scis impropérium meum, et confusiónem meam, * et reveréntiam meam.
}\switchcolumn\portugues{
Vós conheceis a minha desgraça e a minha confusão, * e a minha vergonha.
}\switchcolumn*\latim{
In conspéctu tuo sunt omnes qui tríbulant me: * impropérium exspectávit cor meum, et misériam.
}\switchcolumn\portugues{
À vossa vista estão todos os que me atribulam: * o meu coração espera desgraças e misérias.
}\switchcolumn*\latim{
Et sustínui qui simul contristarétur, et non fuit: * et qui consolarétur, et non invéni.
}\switchcolumn\portugues{
Esperei que alguém se condoesse de mim e não houve ninguém: * esperei que alguém me consolasse e não achei.
}\switchcolumn*\latim{
Et dedérunt in escam meam fel: * et in siti mea potavérunt me acéto.
}\switchcolumn\portugues{
Por comida me deram veneno: * e na minha sede vinagre me apresentaram.
}\switchcolumn*\latim{
Fiat mensa eórum coram ipsis in láqueum, * et in retributiónes, et in scándalum.
}\switchcolumn\portugues{
Que sua mesa ante eles se torne um embuste, * uma recompensa e uma pedra no caminho.
}\switchcolumn*\latim{
Obscuréntur óculi eórum ne vídeant: * et dorsum eórum semper incúrva.
}\switchcolumn\portugues{
Obscureçam-se os seus olhos para que não vejam: * e o seu dorsal fique sempre curvo.
}\switchcolumn*\latim{
Effúnde super eos iram tuam: * et furor iræ tuæ comprehéndat eos.
}\switchcolumn\portugues{
Derramai sobre eles a vossa indignação: * e deixai que a vossa ira os alcance.
}\switchcolumn*\latim{
Fiat habitátio eórum desérta: * et in tabernáculis eórum non sit qui inhábitet.
}\switchcolumn\portugues{
Deserta fique a sua morada: * e não haja quem habite em suas tendas.
}\switchcolumn*\latim{
Quóniam quem Tu percussísti, persecúti sunt: * et super dolórem vúlnerum meórum addidérunt.
}\switchcolumn\portugues{
Porquanto perseguiram aquele que Vós feristes: * e agravaram a dor das minhas chagas.
}\switchcolumn*\latim{
Appóne iniquitátem super iniquitátem eórum: * et non intrent in justítiam tuam.
}\switchcolumn\portugues{
Somai-lhes iniquidade sobre iniquidade: * e não cheguem a entrar na vossa justiça.
}\switchcolumn*\latim{
Deleántur de libro vivéntium: * et cum justis non scribántur.
}\switchcolumn\portugues{
Sejam riscados do livro dos viventes: * e não sejam inscritos com os justos.
}\switchcolumn*\latim{
Ego sum pauper et dolens: * salus tua, Deus, suscépit me.
}\switchcolumn\portugues{
Eu sou pobre e cheio de dores: * mas a vossa salvação, ó Deus, me acolheu.
}\switchcolumn*\latim{
Laudábo nomen Dei cum cántico: * et magnificábo eum in laude:
}\switchcolumn\portugues{
Glorificarei o nome de Deus com cânticos: * e engrandecê-l’O-ei com louvores:
}\switchcolumn*\latim{
Et placébit Deo super vítulum novéllum: * córnua producéntem et úngulas.
}\switchcolumn\portugues{
Isto agradará a Deus mais do que o tenro novilho: * quando lhe vão nascendo as pontas e as unhas.
}\switchcolumn*\latim{
Vídeant páuperes et læténtur: * quǽrite Deum, et vivet ánima vestra.
}\switchcolumn\portugues{
Vejam os pobres e alegrem-se: * buscai a Deus e a vossa alma viverá.
}\switchcolumn*\latim{
Quóniam exaudívit páuperes Dóminus: * et vinctos suos non despéxit.
}\switchcolumn\portugues{
Porque o Senhor ouviu os pobres: * e não desprezou os seus prisioneiros.
}\switchcolumn*\latim{
Laudent illum cæli et terra, * mare et ómnia reptília in eis.
}\switchcolumn\portugues{
Louvem-n'O os céus e a terra, * o mar e tudo o que neles se move.
}\switchcolumn*\latim{
Quóniam Deus salvam fáciet Sion: * et ædificabúntur civitátes Juda.
}\switchcolumn\portugues{
Porque Deus salvará Sião: * e edificar-se-ão as cidades de Judá.
}\switchcolumn*\latim{
Et inhabitábunt ibi, * et hereditáte acquírent eam.
}\switchcolumn\portugues{
Morarão ali, * adquirindo-as como sua herança.
}\switchcolumn*\latim{
Et semen servórum ejus possidébit eam: * et qui díligunt nomen ejus, habitábunt in ea.
}\switchcolumn\portugues{
A descendência dos seus servos a possuirá: * e os que amam o seu nome habitarão nela.
}\end{paracol}


\subsectioninfo{Salmo 69}{Deus, in adjutorium meum intende}\label{salmo69}
\begin{paracol}{2}\latim{
\rlettrine{D}{eus,} in adjutórium meum inténde: * Dómine, ad adjuvándum me festína.
}\switchcolumn\portugues{
\slettrine{Ó}{} Deus, vinde em meu auxílio: * ó Senhor, apressai-Vos em ajudar-me.
}\switchcolumn*\latim{
Confundántur et revereántur, * qui quǽrunt ánimam meam.
}\switchcolumn\portugues{
Sejam confundidos e envergonhados, * os que a vida me procuram tirar.
}\switchcolumn*\latim{
Avertántur retrórsum, et erubéscant, * qui volunt mihi mala.
}\switchcolumn\portugues{
Deixai que recuem e sejam envergonhados, * os que mal me desejam.
}\switchcolumn*\latim{
Avertántur statim erubescéntes, * qui dicunt mihi: euge, euge.
}\switchcolumn\portugues{
Deixai que sejam imediatamente envergonhados, * os que me dizem: bem, bem!
}\switchcolumn*\latim{
Exsúltent et læténtur in Te omnes qui quǽrunt Te, * et dicant semper: magnificétur Dóminus: qui díligunt salutáre tuum.
}\switchcolumn\portugues{
Regozijem-se e alegrem-se em Vós todos os que Vos buscam, * e digam sempre os que amam a vossa salvação: glorificado seja o Senhor.
}\switchcolumn*\latim{
Ego vero egénus, et pauper sum: * Deus, ádjuva me.
}\switchcolumn\portugues{
Eu, contudo, sou necessitado e pobre: * ó Deus, ajudai-me.
}\switchcolumn*\latim{
Adjútor meus, et liberátor meus es Tu: * Dómine, ne moréris.
}\switchcolumn\portugues{
Vós sois o meu auxiliador e o meu libertador: * ó Senhor, Vos não demoreis.
}\end{paracol}


\subsectioninfo{Salmo 70}{In te, Domine, speravi}\label{salmo70}
\begin{paracol}{2}\latim{
\rlettrine{I}{n} te, Dómine, sperávi, non confúndar in ætérnum: * in justítia tua líbera me, et éripe me.
}\switchcolumn\portugues{
\rlettrine{E}{m} Vós, ó Senhor, tenho esperado, não seja jamais confundido: * livrai-me na vossa justiça e ponde-me a salvo.
}\switchcolumn*\latim{
Inclína ad me aurem tuam, * et salva me.
}\switchcolumn\portugues{
Inclinai para mim o vosso ouvido, * e salvai-me.
}\switchcolumn*\latim{
Esto mihi in Deum protectórem, et in locum munítum: * ut salvum me fácias,
}\switchcolumn\portugues{
Sede para mim um Deus protector e um asilo seguro: * para me salvar,
}\switchcolumn*\latim{
Quóniam firmaméntum meum, * et refúgium meum es tu.
}\switchcolumn\portugues{
Porque o meu apoio * e o meu refúgio sois Vós.
}\switchcolumn*\latim{
Deus meus, éripe me de manu peccatóris, * et de manu contra legem agéntis et iníqui:
}\switchcolumn\portugues{
Deus meu, livrai-me da mão do pecador, * da mão do transgressor da lei e do iníquo:
}\switchcolumn*\latim{
Quóniam tu es patiéntia mea, Dómine: * Dómine, spes mea a juventúte mea.
}\switchcolumn\portugues{
Porque Vós, ó Senhor, sois a minha paciência: * ó Senhor, sois a minha esperança desde a mocidade.
}\switchcolumn*\latim{
In te confirmátus sum ex útero: * de ventre matris meæ tu es protéctor meus.
}\switchcolumn\portugues{
Em Vós me sustentei desde o meu nascimento: * Vós sois o meu protector desde o ventre de minha mãe.
}\switchcolumn*\latim{
In te cantátio mea semper: * tamquam prodígium factus sum multis: et tu adjútor fortis.
}\switchcolumn\portugues{
Sobre Vós cantarei para sempre: * fui por muitos considerado como um prodígio, mas Vós sois um poderoso auxiliador.
}\switchcolumn*\latim{
Repleátur os meum laude, ut cantem glóriam tuam: * tota die magnitúdinem tuam.
}\switchcolumn\portugues{
Encha-se a minha boca de louvor, para cantar a vossa glória: * e para celebrar todo o dia a vossa grandeza.
}\switchcolumn*\latim{
Ne proícias me in témpore senectútis: * cum defécerit virtus mea, ne derelínquas me.
}\switchcolumn\portugues{
Não me desampareis no tempo da velhice: * quando faltarem as minhas forças me não abandoneis.
}\switchcolumn*\latim{
Quia dixérunt inimíci mei mihi: * et qui custodiébant ánimam meam, consílium fecérunt in unum.
}\switchcolumn\portugues{
Pois os meus inimigos falaram contra mim: * e insidiavam a minha vida, juntos, em conselho.
}\switchcolumn*\latim{
Dicéntes: Deus derelíquit eum, persequímini, et comprehéndite eum: * quia non est qui erípiat.
}\switchcolumn\portugues{
Dizendo: Deus desamparou-o, persegui-o e prendei-o: * pois não há quem o livre.
}\switchcolumn*\latim{
Deus, ne elongéris a me: * Deus meus, in auxílium meum réspice.
}\switchcolumn\portugues{
Ó Deus, Vos não afasteis de mim: * ó Deus meu, acudi em meu auxílio.
}\switchcolumn*\latim{
Confundántur, et defíciant detrahéntes ánimæ meæ: * operiántur confusióne, et pudóre qui quǽrunt mala mihi.
}\switchcolumn\portugues{
Confundidos sejam e pereçam, os que maldizem a minha alma: * sejam cobertos de confusão e de vergonha os que me procuram males.
}\switchcolumn*\latim{
Ego autem semper sperábo: * et adíciam super omnem laudem tuam.
}\switchcolumn\portugues{
Eu, porém, esperarei sempre: * e acrescentarei sobre todos vossos louvores.
}\switchcolumn*\latim{
Os meum annuntiábit justítiam tuam: * tota die salutáre tuum.
}\switchcolumn\portugues{
Minha boca anunciará a vossa justiça: * todo o dia publicará a vossa salvação.
}\switchcolumn*\latim{
Quóniam non cognóvi litteratúram, introíbo in poténtias Dómini: * Dómine, memorábor justítiæ tuæ solíus.
}\switchcolumn\portugues{
Visto que não conheço erudição, entrarei no domínio do Senhor: * ó Senhor, lembrar-me-ei somente de vossa justiça.
}\switchcolumn*\latim{
Deus, docuísti me a juventúte mea: * et usque nunc pronuntiábo mirabília tua.
}\switchcolumn\portugues{
Ensinastes-me, ó Deus, desde a minha mocidade: * e até agora publicarei as vossas maravilhas.
}\switchcolumn*\latim{
Et usque in senéctam et sénium: * Deus, ne derelínquas me,
}\switchcolumn\portugues{
E até à velhice e aos cabelos brancos: * ó Deus, não me desampareis,
}\switchcolumn*\latim{
Donec annúntiem brácchium tuum * generatióni omni, quæ ventúra est:
}\switchcolumn\portugues{
Até que anuncie o vosso braço * a toda a geração que há-de vir:
}\switchcolumn*\latim{
Poténtiam tuam, et justítiam tuam, Deus, usque in altíssima, quæ fecísti magnália: * Deus, quis símilis tibi?
}\switchcolumn\portugues{
Vosso poder e vossa justiça, ó Deus, que chegam até aos céus, nas maravilhas que fizestes: * ó Deus, quem é semelhante a Vós?
}\switchcolumn*\latim{
Quantas ostendísti mihi tribulatiónes multas et malas: et convérsus vivificásti me: * et de abýssis terræ íterum reduxísti me:
}\switchcolumn\portugues{
Quantas tribulações numerosas e amargas me fizestes provar: * mas, voltando-Vos para mim, destes-me a vida e dos abysmos da terra outra vez me tirastes:
}\switchcolumn*\latim{
Multiplicásti magnificéntiam tuam: * et convérsus consolátus es me.
}\switchcolumn\portugues{
Multiplicastes a vossa magnificência: * e, voltando-Vos para mim, me consolastes.
}\switchcolumn*\latim{
Nam et ego confitébor tibi in vasis psalmi veritátem tuam: * Deus, psallam tibi in cíthara, Sanctus Israël.
}\switchcolumn\portugues{
Por isso eu louvarei a vossa verdade com instrumentos de salmos: * ó Deus, Vos cantarei salmos com a cítara, ó Santo de Israel.
}\switchcolumn*\latim{
Exsultábunt lábia mea cum cantávero tibi: * et ánima mea, quam redemísti.
}\switchcolumn\portugues{
Ao cantar a Vós, regozijar-se-ão os meus lábios: * e a minha alma, que resgatastes.
}\switchcolumn*\latim{
Sed et lingua mea tota die meditábitur justítiam tuam: * cum confúsi et revériti fúerint, qui quærunt mala mihi.
}\switchcolumn\portugues{
E a minha língua anunciará todo o dia a vossa justiça: * quando forem confundidos e envergonhados os que procuram fazer-me mal.
}\end{paracol}


\subsectioninfo{Salmo 71}{Deus, judicium tuum regi da}\label{salmo71}
\begin{paracol}{2}\latim{
\rlettrine{D}{eus,} judícium tuum regi da: * et justítiam tuam fílio regis:
}\switchcolumn\portugues{
\rlettrine{D}{ai} o vosso juízo ao rei, ó Deus: * e a vossa justiça, ao filho do Rei:
}\switchcolumn*\latim{
Judicáre pópulum tuum in justítia, * et páuperes tuos in judício.
}\switchcolumn\portugues{
Para que julgue o vosso povo com justiça, * e vossos pobres com equidade.
}\switchcolumn*\latim{
Suscípiant montes pacem pópulo: * et colles justítiam.
}\switchcolumn\portugues{
Recebam os montes paz para o povo: * e as colinas justiça.
}\switchcolumn*\latim{
Judicábit páuperes pópuli, et salvos fáciet fílios páuperum: * et humiliábit calumniatórem.
}\switchcolumn\portugues{
Julgará os pobres do povo e salvará os filhos dos pobres: * e humilhará o caluniador.
}\switchcolumn*\latim{
Et permanébit cum sole, et ante lunam, * in generatióne et generatiónem.
}\switchcolumn\portugues{
Permanecerá com o sol e ante a lua, * de geração em geração.
}\switchcolumn*\latim{
Descéndet sicut plúvia in vellus: * et sicut stillicídia stillántia super terram.
}\switchcolumn\portugues{
Descerá como chuva sobre a lã: * e como orvalho que pinga sobre a terra.
}\switchcolumn*\latim{
Oriétur in diébus ejus justítia, et abundántia pacis: * donec auferátur luna.
}\switchcolumn\portugues{
Nos seus dias aparecerá a justiça e a abundância da paz: * até que a lua deixe de existir.
}\switchcolumn*\latim{
Et dominábitur a mari usque ad mare: * et a flúmine usque ad términos orbis terrárum.
}\switchcolumn\portugues{
Dominará de mar a mar: * e desde o rio até aos confins da órbita terrestre.
}\switchcolumn*\latim{
Coram illo prócident Æthíopes: * et inimíci ejus terram lingent.
}\switchcolumn\portugues{
Diante d’Ele prostrar-se-ão os Etíopes: * e os seus inimigos beijarão a terra.
}\switchcolumn*\latim{
Reges Tharsis, et ínsulæ múnera ófferent: * reges Árabum et Saba dona addúcent.
}\switchcolumn\portugues{
Os reis de Társis e as ilhas Lhe oferecerão dons: * os reis da Arábia e de Sabá Lhe trarão presentes.
}\switchcolumn*\latim{
Et adorábunt eum omnes reges terræ: * omnes gentes sérvient ei:
}\switchcolumn\portugues{
Adorá-l’O-ão todos os reis da terra: * todas as gentes o servirão:
}\switchcolumn*\latim{
Quia liberábit páuperem a poténte: * et páuperem, cui non erat adjútor.
}\switchcolumn\portugues{
Pois livrará o pobre do poderoso: * e o indigente que não tem quem lhe valha.
}\switchcolumn*\latim{
Parcet páuperi et ínopi: * et ánimas páuperum salvas fáciet.
}\switchcolumn\portugues{
Poupará o pobre e o desvalido: * e salvará as almas dos pobres.
}\switchcolumn*\latim{
Ex usúris et iniquitáte rédimet ánimas eórum: * et honorábile nomen eórum coram illo.
}\switchcolumn\portugues{
Resgatará as suas almas das usuras e da iniquidade: * e os seus nomes serão honrados na sua presença.
}\switchcolumn*\latim{
Et vivet, et dábitur ei de auro Arábiæ, et adorábunt de ipso semper: * tota die benedícent ei.
}\switchcolumn\portugues{
Viverá, apresentar-Lhe-ão do ouro da Arábia e adorá-l'O-ão sempre: * bendi-l'O-ão todo o dia.
}\switchcolumn*\latim{
Et erit firmaméntum in terra in summis móntium, superextollétur super Líbanum fructus ejus: * et florébunt de civitáte sicut fænum terræ.
}\switchcolumn\portugues{
Haverá mantimento na terra, no cume dos montes, erguer-se-á sobre o Líbano o seu fruto: * e florescerão os da cidade como a erva dos campos.
}\switchcolumn*\latim{
Sit nomen ejus benedíctum in sǽcula: * ante solem pérmanet nomen ejus.
}\switchcolumn\portugues{
Seja o seu nome bendito pelos séculos: * o seu nome existe antes do sol.
}\switchcolumn*\latim{
Et benedicéntur in ipso omnes tribus terræ: * omnes gentes magnificábunt eum.
}\switchcolumn\portugues{
Serão benditas n’Ele todas as tribos da terra: * todas as gentes O glorificarão.
}\switchcolumn*\latim{
Benedíctus Dóminus, Deus Israël, * qui facit mirabília solus:
}\switchcolumn\portugues{
Bendito seja o Senhor Deus de Israel, * é só Ele que faz maravilhas.
}\switchcolumn*\latim{
Et benedíctum nomen majestátis ejus in ætérnum: * et replébitur majestáte ejus omnis terra: fiat, fiat.
}\switchcolumn\portugues{
Bendito seja o nome da sua majestade para sempre: * e encher-se-á da sua majestade toda a terra: assim seja, assim seja.
}\end{paracol}


\subsectioninfo{Salmo 72}{Quam bonus Israël Deus}\label{salmo72}
\begin{paracol}{2}\latim{
\qlettrine{Q}{uam} bonus Israël Deus, * his, qui recto sunt corde!
}\switchcolumn\portugues{
\qlettrine{Q}{uão} bom é Deus para Israel, * para eles que são rectos de coração!
}\switchcolumn*\latim{
Mei autem pæne moti sunt pedes: * pæne effúsi sunt gressus mei.
}\switchcolumn\portugues{
Meus pés por pouco não vacilaram: * por pouco se não transviaram os meus passos.
}\switchcolumn*\latim{
Quia zelávi super iníquos, * pacem peccatórum videns.
}\switchcolumn\portugues{
Pois tive inveja dos iníquos, * vendo a paz dos pecadores.
}\switchcolumn*\latim{
Quia non est respéctus morti eórum: * et firmaméntum in plaga eórum.
}\switchcolumn\portugues{
Pois eles não têm medo da morte: * nem fortes são as suas feridas.
}\switchcolumn*\latim{
In labóre hóminum non sunt, * et cum homínibus non flagellabúntur:
}\switchcolumn\portugues{
Não participam dos trabalhos dos homens, * nem como os outros homens serão flagelados:
}\switchcolumn*\latim{
Ideo ténuit eos supérbia, * opérti sunt iniquitáte et impietáte sua.
}\switchcolumn\portugues{
Portanto ensoberbeceram-se, * estão cobertos da sua iniquidade e impiedade.
}\switchcolumn*\latim{
Pródiit quasi ex ádipe iníquitas eórum: * transiérunt in afféctum cordis.
}\switchcolumn\portugues{
Sua iniquidade nasce como que da sua gordura: * abandonaram-se às paixões do seu coração.
}\switchcolumn*\latim{
Cogitavérunt, et locúti sunt nequítiam: * iniquitátem in excélso locúti sunt.
}\switchcolumn\portugues{
Seus pensamentos e palavras são somente inutilidade: * iniquidade falaram altivamente.
}\switchcolumn*\latim{
Posuérunt in cælum os suum: * et lingua eórum transívit in terra.
}\switchcolumn\portugues{
Abriram a sua boca contra o céu: * e a sua língua foi discorrendo pela terra.
}\switchcolumn*\latim{
Ídeo convertétur pópulus meus hic: * et dies pleni inveniéntur in eis.
}\switchcolumn\portugues{
Por isto o meu povo retornará aqui: * e serão achados nele dias cheios.
}\switchcolumn*\latim{
Et dixérunt: quómodo scit Deus, * et si est sciéntia in excélso?
}\switchcolumn\portugues{
Chegam a dizer: porventura Deus saberá, * e tem conhecimento disto o Altíssimo?
}\switchcolumn*\latim{
Ecce, ipsi peccatóres, et abundántes in sǽculo, * obtinuérunt divítias.
}\switchcolumn\portugues{
Eis que estes pecadores, que têm tudo em abundância neste mundo, * adquiriram riquezas.
}\switchcolumn*\latim{
Et dixi: ergo sine causa justificávi cor meum, * et lavi inter innocéntes manus meas:
}\switchcolumn\portugues{
Disse: foi portanto inutilmente que justifiquei o meu coração, * e lavei entre os inocentes as minhas mãos:
}\switchcolumn*\latim{
Et fui flagellátus tota die, * et castigátio mea in matutínis.
}\switchcolumn\portugues{
Pois fui afligido todo o dia, * e castigado desde manhã.
}\switchcolumn*\latim{
Si dicébam: narrábo sic: * ecce, natiónem filiórum tuórum reprobávi.
}\switchcolumn\portugues{
Se dissesse: narrarei assim: * eis que condenava a nação de vossos filhos.
}\switchcolumn*\latim{
Existimábam ut cognóscerem hoc, * labor est ante me:
}\switchcolumn\portugues{
Reflecti para compreender isto, * porém, foi uma dificuldade a meus olhos:
}\switchcolumn*\latim{
Donec intrem in Sanctuárium Dei: * et intéllegam in novíssimis eórum.
}\switchcolumn\portugues{
Até que entrei no santuário de Deus: * e compreendi qual será o fim deles.
}\switchcolumn*\latim{
Verúmtamen propter dolos posuísti eis: * dejecísti eos dum allevaréntur.
}\switchcolumn\portugues{
Certamente em enganos os pusestes: * e os derrubastes quando se elevavam.
}\switchcolumn*\latim{
Quómodo facti sunt in desolatiónem, súbito defecérunt: * periérunt propter iniquitátem suam.
}\switchcolumn\portugues{
Como foram reduzidos a uma tal desolação, repentinamente murcharam: * pereceram pela sua iniquidade.
}\switchcolumn*\latim{
Velut sómnium surgéntium, Dómine, * in civitáte tua imáginem ipsórum ad níhilum rédiges.
}\switchcolumn\portugues{
Como o sonho dos que despertam, ó Senhor, * assim reduzireis a nada a sua imagem na vossa cidade.
}\switchcolumn*\latim{
Quia inflammátum est cor meum, et renes mei commutáti sunt: * et ego ad níhilum redáctus sum, et nescívi.
}\switchcolumn\portugues{
Pois se inflamou o meu coração e as minhas entranhas se comoveram: * e fiquei aniquilado sem saber por quê.
}\switchcolumn*\latim{
Ut juméntum factus sum apud Te: * et ego semper tecum.
}\switchcolumn\portugues{
Tornei-me ante Vós como um jumento: * e convosco estarei sempre.
}\switchcolumn*\latim{
Tenuísti manum déxteram meam: et in voluntáte tua deduxísti me, * et cum glória suscepísti me.
}\switchcolumn\portugues{
Tomastes-me pela minha mão direita e me conduzistes segundo a vossa vontade, * e com glória me acolhestes.
}\switchcolumn*\latim{
Quid enim mihi est in cælo? * Et a Te quid vólui super terram?
}\switchcolumn\portugues{
Pois que no céu há para mim? * E, além de Vós que desejei eu sobre a terra?
}\switchcolumn*\latim{
Defécit caro mea, et cor meum: * Deus cordis mei, et pars mea Deus in ætérnum.
}\switchcolumn\portugues{
Desfaleceu a minha carne e o meu coração: * ó Deus do meu coração, Deus é a minha herança para sempre.
}\switchcolumn*\latim{
Quia ecce, qui elóngant se a Te, períbunt: * perdidísti omnes, qui fornicántur abs Te.
}\switchcolumn\portugues{
Eis pois, os que se apartam de Vós perecerão: * aniquilastes todos os que Vos são infiéis.
}\switchcolumn*\latim{
Mihi autem adhærére Deo bonum est: * pónere in Dómino Deo spem meam:
}\switchcolumn\portugues{
Todavia, é para mim bom unir-me a Deus: * e pôr no Senhor Deus a minha esperança:
}\switchcolumn*\latim{
Ut annúntiem omnes prædicatiónes tuas, * in portis fíliæ Sion.
}\switchcolumn\portugues{
A fim de anunciar todos vossos louvores, * às portas da filha de Sião.
}\end{paracol}


\subsectioninfo{Salmo 73}{Ut quid, Deus}\label{salmo73}
\begin{paracol}{2}\latim{
\rlettrine{U}{t} quid, Deus, repulísti in finem: * irátus est furor tuus super oves páscuæ tuæ?
}\switchcolumn\portugues{
\rlettrine{P}{or} que razão, ó Deus, nos desamparastes até ao fim: * e se acendeu a vossa cólera contra as ovelhas de vosso pasto?
}\switchcolumn*\latim{
Memor esto congregatiónis tuæ, * quam possedísti ab inítio.
}\switchcolumn\portugues{
Lembrai-Vos de vossa congregação, * que possuístes desde o princípio.
}\switchcolumn*\latim{
Redemísti virgam hereditátis tuæ: * mons Sion, in quo habitásti in eo.
}\switchcolumn\portugues{
Vós recuperastes o ceptro de vossa herança: * o monte de Sião, em que habitastes.
}\switchcolumn*\latim{
Leva manus tuas in supérbias eórum in finem: * quanta malignátus est inimícus in sancto!
}\switchcolumn\portugues{
Levantai as vossas mãos contra a sua soberba sem limites: * quantas maldades cometeu o inimigo no santuário!
}\switchcolumn*\latim{
Et gloriáti sunt qui odérunt Te: * in médio solemnitátis tuæ.
}\switchcolumn\portugues{
Os que Vos odeiam, gloriam-se: * no meio de vossa solenidade.
}\switchcolumn*\latim{
Posuérunt signa sua, signa: * et non cognovérunt sicut in éxitu super summum.
}\switchcolumn\portugues{
Hastearam os seus estandartes como troféus: * e as não conheceram no cimo da porta de saída.
}\switchcolumn*\latim{
Quasi in silva lignórum secúribus excidérunt jánuas ejus in idípsum: * in secúri et áscia dejecérunt eam.
}\switchcolumn\portugues{
Como com machados num bosque de árvores, despedaçaram com afinco os seus portões: * com machado e martelo tudo derrubaram.
}\switchcolumn*\latim{
Incendérunt igni Sanctuárium tuum: * in terra polluérunt tabernáculum nóminis tui.
}\switchcolumn\portugues{
Puseram fogo ao vosso santuário: * na terra profanaram o tabernáculo de vosso nome.
}\switchcolumn*\latim{
Dixérunt in corde suo cognátio eórum simul: * Quiéscere faciámus omnes dies festos Dei a terra.
}\switchcolumn\portugues{
Com seus semelhantes disseram no seu coração: * façamos cessar na terra todos os dias de festa consagrados a Deus.
}\switchcolumn*\latim{
Signa nostra non vídimus, jam non est prophéta: * et nos non cognóscet ámplius.
}\switchcolumn\portugues{
Não vemos mais o nosso estandarte, já não há um profeta: * e Ele não mais nos conhecerá.
}\switchcolumn*\latim{
Úsquequo, Deus, improperábit inimícus: * irrítat adversárius nomen tuum in finem?
}\switchcolumn\portugues{
Até quando, ó Deus, o inimigo nos insultará: * o adversário há-de blasfemar para sempre?
}\switchcolumn*\latim{
Ut quid avértis manum tuam, et déxteram tuam, * de médio sinu tuo in finem?
}\switchcolumn\portugues{
Porque retraís a vossa mão e a vossa dextra, * do meio de vosso seio para sempre?
}\switchcolumn*\latim{
Deus autem Rex noster ante sǽcula: * operátus est salútem in médio terræ.
}\switchcolumn\portugues{
Deus, todavia, que é nosso Rei antes dos séculos: * operou a salvação no meio da terra.
}\switchcolumn*\latim{
Tu confirmásti in virtúte tua mare: * contribulásti cápita dracónum in aquis.
}\switchcolumn\portugues{
Vós com vosso poder destes solidez ao mar: * nas águas esmagastes as cabeças dos dragões.
}\switchcolumn*\latim{
Tu confregísti cápita dracónis: * dedísti eum escam pópulis Æthíopum.
}\switchcolumn\portugues{
Vós quebrastes as cabeças do dragão: * deste-o por comida aos povos da Etiópia.
}\switchcolumn*\latim{
Tu dirupísti fontes, et torréntes: * Tu siccásti flúvios Ethan.
}\switchcolumn\portugues{
Vós fizestes brotar fontes e torrentes: * Vós secastes os rios de Etan.
}\switchcolumn*\latim{
Tuus est dies, et tua est nox: * Tu fabricátus es auróram et solem.
}\switchcolumn\portugues{
Vosso é o dia e vossa é a noite: * Vós criastes a aurora e o sol.
}\switchcolumn*\latim{
Tu fecísti omnes términos terræ: * æstátem et ver Tu plasmásti ea.
}\switchcolumn\portugues{
Vós estabelecestes todos os limites da terra: * o Verão e a Primavera Vós os formastes.
}\switchcolumn*\latim{
Memor esto hujus, inimícus improperávit Dómino: * et pópulus insípiens incitávit nomen tuum.
}\switchcolumn\portugues{
Lembrai-Vos disto, o inimigo ultrajou o Senhor: * e um povo insensato blasfemou de vosso nome.
}\switchcolumn*\latim{
Ne tradas béstiis ánimas confiténtes tibi, * et ánimas páuperum tuórum ne obliviscáris in finem.
}\switchcolumn\portugues{
Não entregueis às feras as almas que Vos louvam, * e não esqueçais para sempre as almas de vossos pobres.
}\switchcolumn*\latim{
Réspice in testaméntum tuum: * quia repléti sunt, qui obscuráti sunt terræ dómibus iniquitátum.
}\switchcolumn\portugues{
Olhai para a vossa aliança: * pois todos os lugares obscuros do país estão cheios de antros de iniquidade.
}\switchcolumn*\latim{
Ne avertátur húmilis factus confúsus: * pauper et inops laudábunt nomen tuum.
}\switchcolumn\portugues{
Não se volte confundido o humilde: * o pobre e o desvalido louvarão o vosso nome.
}\switchcolumn*\latim{
Exsúrge, Deus, júdica causam tuam: * memor esto improperiórum tuórum, eórum quæ ab insipiénte sunt tota die.
}\switchcolumn\portugues{
Levantai-Vos, ó Deus, julgai a vossa causa: * lembrai-Vos dos ultrajes, com que um povo ignorante Vos injuria todo o dia.
}\switchcolumn*\latim{
Ne obliviscáris voces inimicórum tuórum: * supérbia eórum, qui Te odérunt, ascéndit semper.
}\switchcolumn\portugues{
Dos clamores de vossos inimigos Vos não esqueçais: * a soberba daqueles que Vos aborrecem aumenta continuamente.
}\end{paracol}


\subsectioninfo{Salmo 74}{Confitebimur tibi, Deus}\label{salmo74}
\begin{paracol}{2}\latim{
\rlettrine{C}{onfitébimur} tibi, Deus: * confitébimur, et invocábimus nomen tuum.
}\switchcolumn\portugues{
\rlettrine{N}{ós} Vos louvaremos, ó Deus: * nós Vos louvaremos e invocaremos o vosso nome.
}\switchcolumn*\latim{
Narrábimus mirabília tua: * cum accépero tempus, ego justítias judicábo.
}\switchcolumn\portugues{
Narraremos as vossas maravilhas: * quando decidir que é tempo, julgarei com justiça.
}\switchcolumn*\latim{
Liquefácta est terra, et omnes qui hábitant in ea: * ego confirmávi colúmnas ejus.
}\switchcolumn\portugues{
A terra dissolveu-se e todos os que a habitam: * eu fortaleci as suas colunas.
}\switchcolumn*\latim{
Dixi iníquis: nolíte iníque ágere: * et delinquéntibus: nolíte exaltáre cornu:
}\switchcolumn\portugues{
Disse aos iníquos: não pratiqueis iniquidade: * e aos pecadores: não ergueis a cabeça.
}\switchcolumn*\latim{
Nolíte extóllere in altum cornu vestrum: * nolíte loqui advérsus Deum iniquitátem.
}\switchcolumn\portugues{
Não levanteis com insolência as vossas cabeças: * não faleis iniquamente contra Deus.
}\switchcolumn*\latim{
Quia neque ab Oriénte, neque ab Occidénte, neque a desértis móntibus: * quóniam Deus judex est.
}\switchcolumn\portugues{
Pois nem do oriente, nem do ocidente, nem pelos desertos montes: * porque Deus é o juiz.
}\switchcolumn*\latim{
Hunc humíliat, et hunc exáltat: * quia calix in manu Dómini vini meri plenus misto.
}\switchcolumn\portugues{
A este humilha e àquele exalta: * pois na mão do Senhor há um cálice de vinho puro, cheio de mistura.
}\switchcolumn*\latim{
Et inclinávit ex hoc in hoc: verúmtamen fæx ejus non est exinaníta: * bibent omnes peccatóres terræ.
}\switchcolumn\portugues{
Inclina dum lado para o outro, e, todavia, suas fezes se não esgotaram: * delas beberão todos os pecadores da terra.
}\switchcolumn*\latim{
Ego autem annuntiábo in sǽculum: * cantábo Deo Jacob.
}\switchcolumn\portugues{
Eu, porém, anunciarei estas coisas sempre: * cantarei ao Deus de Jacob.
}\switchcolumn*\latim{
Et ómnia córnua peccatórum confríngam: * et exaltabúntur córnua justi.
}\switchcolumn\portugues{
Quebrarei todas as forças dos pecadores: * e será exaltada a cabeça do justo.
}\end{paracol}


\subsectioninfo{Salmo 75}{Notus in Judæa Deus}\label{salmo75}
\begin{paracol}{2}\latim{
\rlettrine{N}{otus} in Judǽa Deus: * in Israël magnum nomen ejus.
}\switchcolumn\portugues{
\rlettrine{D}{eus} é conhecido na Judeia: * grande é o seu nome em Israel.
}\switchcolumn*\latim{
Et factus est in pace locus ejus: * et habitátio ejus in Sion.
}\switchcolumn\portugues{
Na paz foi o seu lugar feito: * e a sua morada em Sião.
}\switchcolumn*\latim{
Ibi confrégit poténtias árcuum, * scutum, gládium, et bellum.
}\switchcolumn\portugues{
Ali quebrou a força do arco, * o escudo, a espada e a guerra.
}\switchcolumn*\latim{
Illúminans Tu mirabíliter a móntibus ætérnis: * turbáti sunt omnes insipiéntes corde.
}\switchcolumn\portugues{
Vós iluminais maravilhosamente dos montes eternos: * turvados ficaram todos os néscios de coração.
}\switchcolumn*\latim{
Dormiérunt somnum suum: * et nihil invenérunt omnes viri divitiárum in mánibus suis.
}\switchcolumn\portugues{
Dormiram o seu sono: * e todos estes homens de riquezas nada acharam nas suas mãos.
}\switchcolumn*\latim{
Ab increpatióne tua, Deus Jacob, * dormitavérunt qui ascendérunt equos.
}\switchcolumn\portugues{
Só com vossa ameaça, ó Deus de Jacob, * dormiram os que montavam em cavalos.
}\switchcolumn*\latim{
Tu terríbilis es, et quis resístet tibi? * Ex tunc ira tua.
}\switchcolumn\portugues{
Vós sois terrível e quem Vos resistirá? * No momento de vossa ira.
}\switchcolumn*\latim{
De cælo audítum fecísti judícium: * terra trémuit et quiévit,
}\switchcolumn\portugues{
Do céu fizestes ouvir o vosso juízo: * a terra tremeu e ficou em sossego,
}\switchcolumn*\latim{
Cum exsúrgeret in judícium Deus, * ut salvos fáceret omnes mansuétos terræ.
}\switchcolumn\portugues{
Quando Deus se levantou para fazer justiça, * para salvar todos os humildes da terra.
}\switchcolumn*\latim{
Quóniam cogitátio hóminis confitébitur tibi: * et relíquiæ cogitatiónis diem festum agent tibi.
}\switchcolumn\portugues{
Porque o homem que considere isto Vos louvará: * e da lembrança que lhe ficar fazer-Vos-á um dia de festa.
}\switchcolumn*\latim{
Vovéte, et réddite Dómino, Deo vestro: * omnes, qui in circúitu ejus affértis múnera.
}\switchcolumn\portugues{
Fazei votos e cumpri-os ao Senhor vosso Deus: * todos os que dos arredores Lhe trazeis oferendas.
}\switchcolumn*\latim{
Terríbili et ei qui aufert spíritum príncipum, * terríbili apud reges terræ.
}\switchcolumn\portugues{
Ao terrível e ao que tira a vida aos príncipes, * ao que é terrível para os reis da terra.
}\end{paracol}


\subsectioninfo{Salmo 76}{Voce mea ad Dominum}\label{salmo76}
\begin{paracol}{2}\latim{
\rlettrine{V}{oce} mea ad Dóminum clamávi: * voce mea ad Deum, et inténdit mihi.
}\switchcolumn\portugues{
\rlettrine{C}{om} a minha voz clamei ao Senhor: * levantei a minha voz a Deus e me atendeu.
}\switchcolumn*\latim{
In die tribulatiónis meæ Deum exquisívi, mánibus meis nocte contra eum: * et non sum decéptus.
}\switchcolumn\portugues{
No dia da minha tribulação busquei a Deus, estendi-Lhe de noite as minhas mãos: * e não fiquei defraudado.
}\switchcolumn*\latim{
Rénuit consolári ánima mea, * memor fui Dei, et delectátus sum, et exercitátus sum: et defécit spíritus meus.
}\switchcolumn\portugues{
Recusou consolar-se a minha alma, * lembrei-me de Deus e deleitei-me, ponderei e o meu espírito desfaleceu.
}\switchcolumn\portugues{
Anticipavérunt vigílias óculi mei: * turbátus sum, et non sum locútus.
}\switchcolumn\portugues{
Meus olhos anteciparam as vigílias: * fiquei perturbado e não falei.
}\switchcolumn*\latim{
Cogitávi dies antíquos: * et annos ætérnos in mente hábui.
}\switchcolumn\portugues{
Pensei nos dias antigos: * e tive na mente os anos eternos.
}\switchcolumn*\latim{
Et meditátus sum nocte cum corde meo, * et exercitábar, et scopébam spíritum meum.
}\switchcolumn\portugues{
Meditava de noite em meu coração, * reflectia e examinava o meu espírito.
}\switchcolumn*\latim{
Numquid in ætérnum proíciet Deus: * aut non appónet ut complacítior sit adhuc?
}\switchcolumn\portugues{
Porventura Deus há-de abandonar-nos para sempre: * e se não mostrará jamais favorável?
}\switchcolumn*\latim{
Aut in finem misericórdiam suam abscíndet, * a generatióne in generatiónem?
}\switchcolumn\portugues{
Ou há-de privar-nos para sempre da sua misericórdia, * de geração em geração?
}\switchcolumn*\latim{
Aut obliviscétur miseréri Deus? * Aut continébit in ira sua misericórdias suas?
}\switchcolumn\portugues{
Ou esquecer-se-á Deus de usar de clemência? * Ou deterá na sua ira suas misericórdias?
}\switchcolumn*\latim{
Et dixi: nunc cœpi: * hæc mutátio déxteræ Excélsi.
}\switchcolumn\portugues{
Então disse: agora começo: * esta mudança vem da dextra do Altíssimo.
}\switchcolumn*\latim{
Memor fui óperum Dómini: * quia memor ero ab inítio mirabílium tuórum.
}\switchcolumn\portugues{
Lembrei-me das obras do Senhor: * e recordar-me-ei de vossas maravilhas de outrora.
}\switchcolumn*\latim{
Et meditábor in ómnibus opéribus tuis: * et in adinventiónibus tuis exercébor.
}\switchcolumn\portugues{
Meditarei em todas vossas obras: * e considerarei os vossos desígnios.
}\switchcolumn*\latim{
Deus, in sancto via tua: quis Deus magnus sicut Deus noster? * Tu es Deus qui facis mirabília.
}\switchcolumn\portugues{
Vosso caminho, ó Deus, é santo: que Deus há grande como nosso Deus? * Vós sois o Deus que operais maravilhas.
}\switchcolumn*\latim{
Notam fecísti in pópulis virtútem tuam: * redemísti in brácchio tuo pópulum tuum, fílios Jacob et Joseph.
}\switchcolumn\portugues{
Fizestes conhecer entre os povos o vosso poder: * redimistes com vosso braço o vosso povo, os filhos de Jacob e de José.
}\switchcolumn*\latim{
Vidérunt Te aquæ, Deus, vidérunt Te aquæ: * et timuérunt, et turbátæ sunt abýssi.
}\switchcolumn\portugues{
Viram-Vos as águas, ó Deus, viram-Vos as águas: * temeram e foram turvados os abysmos.
}\switchcolumn*\latim{
Multitúdo sónitus aquárum: * vocem dedérunt nubes.
}\switchcolumn\portugues{
Grande foi o estrondo das águas: * as nuvens fizeram-se soar.
}\switchcolumn*\latim{
Étenim sagíttæ tuæ tránseunt: * vox tonítrui tui in rota.
}\switchcolumn\portugues{
Pois as vossas setas trespassaram: * a voz de vosso trovão rolou.
}\switchcolumn*\latim{
Illuxérunt coruscatiónes tuæ orbi terræ: * commóta est, et contrémuit terra.
}\switchcolumn\portugues{
Vossos relâmpagos iluminaram a terra: * vacilou e tremeu a terra.
}\switchcolumn*\latim{
In mari via tua, et sémitæ tuæ in aquis multis: * et vestígia tua non cognoscéntur.
}\switchcolumn\portugues{
No mar o vosso caminho e os vossos atalhos em muitas águas: * e não serão conhecidos os vossos vestígios.
}\switchcolumn*\latim{
Deduxísti sicut oves pópulum tuum, * in manu Móysi et Aaron.
}\switchcolumn\portugues{
Conduzistes o vosso povo como ovelhas, * pela mão de Moisés e de Arão.
}\end{paracol}


\subsectioninfo{Salmo 77}{Attendite, popule meus}\label{salmo77}
\begin{paracol}{2}\latim{
\rlettrine{A}{tténdite,} pópule meus, legem meam: * inclináte aurem vestram in verba oris mei.
}\switchcolumn\portugues{
\rlettrine{E}{scutai,} ó meu povo, a minha lei: * inclinai os vossos ouvidos às palavras da minha boca.
}\switchcolumn*\latim{
Apériam in parábolis os meum: * loquar propositiónes ab inítio.
}\switchcolumn\portugues{
Abrirei em parábolas a minha boca: * direi adágios escondidos desde o princípio.
}\switchcolumn*\latim{
Quanta audívimus et cognóvimus ea: * et patres nostri narravérunt nobis.
}\switchcolumn\portugues{
O que ouvimos e compreendemos: * e o que nossos pais nos contaram.
}\switchcolumn*\latim{
Non sunt occultáta a fíliis eórum: * in generatióne áltera.
}\switchcolumn\portugues{
Eles as não ocultaram a seus filhos: * nem à sua posteridade.
}\switchcolumn*\latim{
Narrántes laudes Dómini, et virtútes ejus: * et mirabília ejus, quæ fecit.
}\switchcolumn\portugues{
Publicaram os louvores do Senhor, o seu poder: * e as maravilhas que fez.
}\switchcolumn*\latim{
Et suscitávit testimónium in Jacob: * et legem pósuit in Israël.
}\switchcolumn\portugues{
Ele estabeleceu aliança com Jacob: * e pôs uma lei em Israel.
}\switchcolumn*\latim{
Quanta mandávit pátribus nostris nota fácere ea fíliis suis: * ut cognóscat generátio áltera.
}\switchcolumn\portugues{
Que ordenou a nossos pais para que dessem a conhecer a seus filhos: * para que a geração seguinte a conhecesse.
}\switchcolumn*\latim{
Fílii qui nascéntur, et exsúrgent, * et narrábunt fíliis suis.
}\switchcolumn\portugues{
Os filhos que hão-de nascer, erguer-se-ão, * e a contarão a seus filhos.
}\switchcolumn*\latim{
Ut ponant in Deo spem suam, et non obliviscántur óperum Dei: * et mandáta ejus exquírant.
}\switchcolumn\portugues{
Para que ponham em Deus a sua esperança e se não esqueçam das obras de Deus: * e busquem os seus mandamentos.
}\switchcolumn*\latim{
Ne fiant sicut patres eórum: * generátio prava et exásperans.
}\switchcolumn\portugues{
Para que não sejam como seus pais: * uma geração ruim e exasperada.
}\switchcolumn*\latim{
Generátio, quæ non diréxit cor suum: * et non est créditus cum Deo spíritus ejus.
}\switchcolumn\portugues{
Uma geração, que não encaminhou rectamente o seu coração: * nem seu espírito foi fiel a Deus.
}\switchcolumn*\latim{
Fílii Ephrem intendéntes et mitténtes arcum: * convérsi sunt in die belli.
}\switchcolumn\portugues{
Os filhos de Efraim, que curvam e disparam o arco: * viraram as costas no dia da batalha.
}\switchcolumn*\latim{
Non custodiérunt testaméntum Dei: * et in lege ejus noluérunt ambuláre.
}\switchcolumn\portugues{
Não guardaram a aliança feita com Deus: * e na sua lei não quiseram caminhar.
}\switchcolumn*\latim{
Et oblíti sunt benefactórum ejus: * et mirabílium ejus quæ osténdit eis.
}\switchcolumn\portugues{
Esqueceram-se dos seus benefícios: * e das maravilhas que fez à vista deles.
}\switchcolumn*\latim{
Coram pátribus eórum fecit mirabília in terra Ægýpti: * in campo Táneos.
}\switchcolumn\portugues{
Ante seus pais fez maravilhas, na terra do Egipto: * no campo de Tanis.
}\switchcolumn*\latim{
Interrúpit mare, et perdúxit eos: * et státuit aquas quasi in utre.
}\switchcolumn\portugues{
Dividiu o mar e por ele os fez passar: * e conteve as águas como num odre.
}\switchcolumn*\latim{
Et dedúxit eos in nube diéi: * et tota nocte in illuminatióne ignis.
}\switchcolumn\portugues{
Guiou-os de dia por meio de uma nuvem: * e toda a noite com a luz do fogo.
}\switchcolumn*\latim{
Interrúpit petram in erémo: * et adaquávit eos velut in abýsso multa.
}\switchcolumn\portugues{
Rachou a pedra no deserto: * e deu-lhes a beber águas como num rio caudaloso.
}\switchcolumn*\latim{
Et edúxit aquam de petra: * et dedúxit tamquam flúmina aquas.
}\switchcolumn\portugues{
Fez sair água da pedra: * e como rios a fez correr.
}\switchcolumn*\latim{
Et apposuérunt adhuc peccáre ei: * in iram excitavérunt Excélsum in inaquóso.
}\switchcolumn\portugues{
Continuaram a pecar contra Ele: * e incitaram a ira do Altíssimo no lugar árido.
}\switchcolumn*\latim{
Et tentavérunt Deum in córdibus suis, * ut péterent escas animábus suis.
}\switchcolumn\portugues{
A Deus tentaram nos seus corações, * pedindo iguarias que fossem do seu agrado.
}\switchcolumn*\latim{
Et male locúti sunt de Deo: * dixérunt: numquid póterit Deus paráre mensam in desérto?
}\switchcolumn\portugues{
Falaram mal de Deus: * e disseram: poderá porventura Deus preparar uma mesa no deserto?
}\switchcolumn*\latim{
Quóniam percússit petram, et fluxérunt aquæ: * et torréntes inundavérunt.
}\switchcolumn\portugues{
Sem dúvida Ele feriu a pedra e águas correram: * e as torrentes inundaram.
}\switchcolumn*\latim{
Numquid et panem póterit dare, * aut paráre mensam pópulo suo?
}\switchcolumn\portugues{
Poderá porventura também dar pão, * ou preparar a mesa para o seu povo?
}\switchcolumn*\latim{
Ideo audívit Dóminus, et dístulit: * et ignis accénsus est in Jacob, et ira ascéndit in Israël.
}\switchcolumn\portugues{
Ouviu isto o Senhor e irritou-se: * e um fogo acendeu-se contra Jacob e cresceu a ira contra Israel.
}\switchcolumn*\latim{
Quia non credidérunt in Deo: * nec speravérunt in salutári ejus:
}\switchcolumn\portugues{
Pois em Deus não creram: * nem d’Ele esperaram a salvação:
}\switchcolumn*\latim{
Et mandávit núbibus désuper: * et jánuas cæli apéruit.
}\switchcolumn\portugues{
Mandou de cima as nuvens: * e abriu as portas do céu.
}\switchcolumn*\latim{
Et pluit illis manna ad manducándum: * et panem cæli dedit eis.
}\switchcolumn\portugues{
Fez chover sobre eles maná para comerem: * e um pão do céu lhes deu.
}\switchcolumn*\latim{
Panem Angelórum manducávit homo, * cibária misit eis in abundántia.
}\switchcolumn\portugues{
O homem comeu o pão dos anjos, * enviou-lhes Ele manjares em abundância.
}\switchcolumn*\latim{
Tránstulit Austrum de cælo: * et indúxit in virtúte sua Áfricum.
}\switchcolumn\portugues{
Retirou do céu o vento do sul: * e enviou com seu poder o vento Áfrico.
}\switchcolumn*\latim{
Et pluit super eos sicut púlverem carnes: * et sicut arénam maris volatília pennáta.
}\switchcolumn\portugues{
Fez chover sobre eles carnes como pó: * e aves como areia do mar.
}\switchcolumn*\latim{
Et cecidérunt in médio castrórum eórum: * circa tabernácula eórum.
}\switchcolumn\portugues{
Caíram no meio dos seus acampamentos: * em redor das suas tendas.
}\switchcolumn*\latim{
Et manducavérunt, et saturáti sunt nimis, et desidérium eórum áttulit eis: * non sunt fraudáti a desidério suo.
}\switchcolumn\portugues{
Comeram, muito se fartaram e foi satisfeito o seu desejo: * não ficaram defraudados no que desejavam.
}\switchcolumn*\latim{
Adhuc escæ eórum erant in ore ipsórum: * et ira Dei ascéndit super eos.
}\switchcolumn\portugues{
Ainda estavam as iguarias na sua boca: * quando a ira de Deus se elevou contra eles.
}\switchcolumn*\latim{
Et occídit pingues eórum, * et eléctos Israël impedívit.
}\switchcolumn\portugues{
Matou os mais robustos, * e derrubou os escolhidos de Israel.
}\switchcolumn*\latim{
In ómnibus his peccavérunt adhuc: * et non credidérunt in mirabílibus ejus.
}\switchcolumn\portugues{
Depois de tudo isto ainda pecaram: * e não acreditaram nas suas maravilhas.
}\switchcolumn*\latim{
Et defecérunt in vanitáte dies eórum: * et anni eórum cum festinatióne.
}\switchcolumn\portugues{
Seus dias foram em vaidade dissipados: * e os seus anos depressa acabaram.
}\switchcolumn*\latim{
Cum occíderet eos, quærébant eum: * et revertebántur, et dilúculo veniébant ad eum.
}\switchcolumn\portugues{
Quando os matava, buscavam-n’O: * e convertiam-se e apressavam-se a volver para Ele.
}\switchcolumn*\latim{
Et rememoráti sunt quia Deus adjútor est eórum: * et Deus excélsus redémptor eórum est.
}\switchcolumn\portugues{
Lembravam-se que Deus era o seu defensor: * e que o altíssimo Deus era o seu redentor.
}\switchcolumn*\latim{
Et dilexérunt eum in ore suo, * et lingua sua mentíti sunt ei.
}\switchcolumn\portugues{
Amavam-n’O com a boca, * e com sua língua Lhe mentiam.
}\switchcolumn*\latim{
Cor autem eórum non erat rectum cum eo: * nec fidéles hábiti sunt in testaménto ejus.
}\switchcolumn\portugues{
Seu coração não era sincero com Ele: * nem se mantiveram fiéis à sua aliança.
}\switchcolumn*\latim{
Ipse autem est miséricors, et propítius fiet peccátis eórum: * et non dispérdet eos.
}\switchcolumn\portugues{
Ele, porém, é misericordioso e perdoava os seus pecados: * e os não destruía.
}\switchcolumn*\latim{
Et abundávit ut avérteret iram suam: * et non accéndit omnem iram suam:
}\switchcolumn\portugues{
Deteve muitas vezes a sua ira: * e não acendeu toda sua ira.
}\switchcolumn*\latim{
Et recordátus est quia caro sunt: * spíritus vadens et non rédiens.
}\switchcolumn\portugues{
Lembrou-se que eram carne: * um sopro que passa e não volta.
}\switchcolumn*\latim{
Quóties exacerbavérunt eum in desérto, * in iram concitavérunt eum in inaquóso?
}\switchcolumn\portugues{
Quantas vezes O exacerbaram no deserto, * e O moveram à ira naquele lugar árido?
}\switchcolumn*\latim{
Et convérsi sunt, et tentavérunt Deum: * et Sanctum Israël exacerbavérunt.
}\switchcolumn\portugues{
Voltaram a tentar a Deus: * e a exacerbar o Santo de Israel.
}\switchcolumn*\latim{
Non sunt recordáti manus ejus, * die qua redémit eos de manu tribulántis.
}\switchcolumn\portugues{
Não se recordaram da sua mão, * no dia em que os redimiu da mão do opressor.
}\switchcolumn*\latim{
Sicut pósuit in Ægýpto signa sua, * et prodígia sua in campo Táneos.
}\switchcolumn\portugues{
De como fez resplandecer no Egipto os seus milagres, * e os sues prodígios no campo de Tanis.
}\switchcolumn*\latim{
Et convértit in sánguinem flúmina eórum: * et imbres eórum, ne bíberent.
}\switchcolumn\portugues{
Ele converteu em sangue os seus rios: * e as suas águas para que as não pudessem beber.
}\switchcolumn*\latim{
Misit in eos cœnomyíam, et comédit eos: * et ranam, et dispérdidit eos.
}\switchcolumn\portugues{
Enviou contra eles todo o género de moscas, que os devoraram: * e rãs, que os destruíram.
}\switchcolumn*\latim{
Et dedit ærúgini fructus eórum: * et labóres eórum locústæ.
}\switchcolumn\portugues{
Entregou os seus frutos ao mofo: * e as suas searas aos gafanhotos.
}\switchcolumn*\latim{
Et occídit in grándine víneas eórum: * et moros eórum in pruína.
}\switchcolumn\portugues{
Destruiu com saraiva as suas vinhas: * e as suas amoreiras com geada.
}\switchcolumn*\latim{
Et trádidit grándini juménta eórum: * et possessiónem eórum igni.
}\switchcolumn\portugues{
Entregou à saraiva os seus animais: * e as suas possessões ao fogo.
}\switchcolumn*\latim{
Misit in eos iram indignatiónis suæ: * indignatiónem, et iram, et tribulatiónem: immissiónes per ángelos malos.
}\switchcolumn\portugues{
Descarregou sobre eles a violência da sua cólera: * a indignação, a ira e a tribulação, que enviou por anjos maus.
}\switchcolumn*\latim{
Viam fecit sémitæ iræ suæ, non pepércit a morte animábus eórum: * et juménta eórum in morte conclúsit.
}\switchcolumn\portugues{
Abriu um largo caminho à sua ira, não perdoou as suas vidas: * e envolveu na mortandade os seus animais.
}\switchcolumn*\latim{
Et percússit omne primogénitum in terra Ægýpti: * primítias omnis labóris eórum in tabernáculis Cham.
}\switchcolumn\portugues{
Feriu todo o primogénito na terra do Egipto: * e as primícias de todo seu trabalho nas tendas de Cam.
}\switchcolumn*\latim{
Et ábstulit sicut oves pópulum suum: * et perdúxit eos tamquam gregem in desérto.
}\switchcolumn\portugues{
Fez sair o seu povo como ovelhas: * e guiou-os como um rebanho no deserto.
}\switchcolumn*\latim{
Et dedúxit eos in spe, et non timuérunt: * et inimícos eórum opéruit mare.
}\switchcolumn\portugues{
Conduziu-os cheios de esperança e não temeram: * e o mar submergiu os seus inimigos.
}\switchcolumn*\latim{
Et indúxit eos in montem sanctificatiónis suæ: * montem, quem acquisívit déxtera ejus.
}\switchcolumn\portugues{
Os introduziu depois no monte da sua santificação: * monte que Ele adquiriu com sua dextra.
}\switchcolumn*\latim{
Et ejécit a fácie eórum gentes: * et sorte divísit eis terram in funículo distributiónis.
}\switchcolumn\portugues{
Ante eles expulsou as gentes: * e por sorte lhes dividiu a terra e destribuiu-as por linhas de medição.
}\switchcolumn*\latim{
Et habitáre fecit in tabernáculis eórum: * tribus Israël.
}\switchcolumn\portugues{
Fez habitar em suas tendas: * as tribos de Israel.
}\switchcolumn*\latim{
Et tentavérunt, et exacerbavérunt Deum excélsum: * et testimónia ejus non custodiérunt.
}\switchcolumn\portugues{
Eles, porém, tentaram e exacerbaram de novo o excelso Deus: * e não guardaram os seus preceitos.
}\switchcolumn*\latim{
Et avertérunt se, et non servavérunt pactum: * quemádmodum patres eórum convérsi sunt in arcum pravum.
}\switchcolumn\portugues{
Volveram-Lhe as costas e não observaram a aliança: * semelhantes a seus pais, falsearam como um arco torto.
}\switchcolumn*\latim{
In iram concitavérunt eum in cóllibus suis: * et in sculptílibus suis ad æmulatiónem eum provocavérunt.
}\switchcolumn\portugues{
Excitaram-n’O à ira nas suas colinas: * e com os ídolos que esculpiram inflamaram-Lhe o zelo.
}\switchcolumn*\latim{
Audivit Deus, et sprevit: * et ad níhilum redégit valde Israël.
}\switchcolumn\portugues{
Ouviu-os Deus e desprezou-os: * e reduziu Israël ao extremo nada.
}\switchcolumn*\latim{
Et répulit tabernáculum Silo: * tabernáculum suum, ubi habitávit in homínibus.
}\switchcolumn\portugues{
Rejeitou o tabernáculo de Silo: * o seu próprio tabernáculo, onde tinha habitado entre os homens.
}\switchcolumn*\latim{
Et trádidit in captivitátem virtútem eórum: * et pulchritúdinem eórum in manus inimíci.
}\switchcolumn\portugues{
Entregou ao cativeiro a força deles: * e a sua formusura nas mãos do inimigo.
}\switchcolumn*\latim{
Et conclúsit in gládio pópulum suum: * et hereditátem suam sprevit.
}\switchcolumn\portugues{
Entregou o seu povo à espada: * e desprezou a sua própria herança.
}\switchcolumn*\latim{
Júvenes eórum comédit ignis: * et vírgines eórum non sunt lamentátæ.
}\switchcolumn\portugues{
O fogo devorou os seus jovens: * e as suas virgens não foram lamentadas.
}\switchcolumn*\latim{
Sacerdótes eórum in gládio cecidérunt: * et víduæ eórum non plorabántur.
}\switchcolumn\portugues{
Seus sacerdotes pereceram à espada: * e ninguém chorava as suas viúvas.
}\switchcolumn*\latim{
Et excitátus est tamquam dórmiens Dóminus: * tamquam potens crapulátus a vino.
}\switchcolumn\portugues{
O Senhor despertou como quem dorme: * como um valente embriagado de vinho.
}\switchcolumn*\latim{
Et percússit inimícos suos in posterióra: * oppróbrium sempitérnum dedit illis.
}\switchcolumn\portugues{
Feriu os seus inimigos nas partes posteriores: * cobriu-os duma eterna ignomínia.
}\switchcolumn*\latim{
Et répulit tabernáculum Joseph: * et tribum Éphraim non elégit.
}\switchcolumn\portugues{
Rejeitou o tabernáculo de José: * e não escolheu a tribo de Efraim.
}\switchcolumn*\latim{
Sed elégit tribum Juda, * montem Sion quem diléxit.
}\switchcolumn\portugues{
Porém, escolheu a tribo de Judá, * o monte de Sião que amou.
}\switchcolumn*\latim{
Et ædificávit sicut unicórnium sanctifícium suum in terra, * quam fundávit in sǽcula.
}\switchcolumn\portugues{
Edificou o seu santuário como os do unicórnio na terra, * que tinha assegurado para sempre.
}\switchcolumn*\latim{
Et elégit David, servum suum, et sústulit eum de grégibus óvium: * de post fœtántes accépit eum,
}\switchcolumn\portugues{
Escolheu David, seu servo, e tomou-o do rebanho: * tirou-o do cuidado das ovelhas mães,
}\switchcolumn*\latim{
Páscere Jacob, servum suum, * et Israël, hereditátem suam:
}\switchcolumn\portugues{
Para que apascentasse Jacob, seu servo, * e Israel, sua herança:
}\switchcolumn*\latim{
Et pávit eos in innocéntia cordis sui: * et in intelléctibus mánuum suárum dedúxit eos.
}\switchcolumn\portugues{
Apascentou-os segundo a inocência do seu coração: * e os conduziu com a sabedoria das suas mãos.
}\end{paracol}


\subsectioninfo{Salmo 78}{Deus, venerunt gentes}\label{salmo78}
\begin{paracol}{2}\latim{
\rlettrine{D}{eus,} venérunt gentes in hereditátem tuam, polluérunt templum sanctum tuum: * posuérunt Jerúsalem in pomórum custódiam.
}\switchcolumn\portugues{
\slettrine{Ó}{} Deus, vieram as gentes à vossa herança, contaminaram o vosso santo templo: * e fizeram de Jerusalém uma despensa de frutas.
}\switchcolumn*\latim{
Posuérunt morticína servórum tuórum, escas volatílibus cæli: * carnes sanctórum tuórum béstiis terræ.
}\switchcolumn\portugues{
Deram os cadáveres de vossos servos em pasto às aves do céu: * as carnes de vossos santos aos animais da terra.
}\switchcolumn*\latim{
Effudérunt sánguinem eórum tamquam aquam in circúitu Jerúsalem: * et non erat qui sepelíret.
}\switchcolumn\portugues{
Derramaram o seu sangue como água à volta de Jerusalém: * e não havia quem lhes desse sepultura.
}\switchcolumn*\latim{
Facti sumus oppróbrium vicínis nostris: * subsannátio et illúsio his, qui in circúitu nostro sunt.
}\switchcolumn\portugues{
Chegámos a ser a maior desonra dos nossos vizinhos: * o escárnio e a troça daqueles que nos rodeiam.
}\switchcolumn*\latim{
Úsquequo, Dómine, irascéris in finem: * accendétur velut ignis zelus tuus?
}\switchcolumn\portugues{
Até quando, ó Senhor, Vos haveis de irar para sempre: * até quando acender-se-á como fogo o vosso zelo?
}\switchcolumn*\latim{
Effúnde iram tuam in gentes, quæ Te non novérunt: * et in regna quæ nomen tuum non invocavérunt:
}\switchcolumn\portugues{
Derramai a vossa ira sobre as gentes que Vos não conhecem: * e sobre os reinos que não invocaram o vosso nome:
}\switchcolumn*\latim{
Quia comedérunt Jacob: * et locum ejus desolavérunt.
}\switchcolumn\portugues{
Pois eles devoraram Jacob: * e desolaram a sua morada.
}\switchcolumn*\latim{
Ne memíneris iniquitátum nostrárum antiquárum, cito antícipent nos misericórdiæ tuæ: * quia páuperes facti sumus nimis.
}\switchcolumn\portugues{
De nossas antigas maldades Vos não lembreis, antecipem-se quanto antes as vossas misericórdias: * pois fomos reduzidos à última miséria.
}\switchcolumn*\latim{
Ádjuva nos, Deus, salutáris noster: et propter glóriam nóminis tui, Dómine, líbera nos: * et propítius esto peccátis nostris, propter nomen tuum:
}\switchcolumn\portugues{
Ajudai-nos, ó Deus, nosso Salvador, e pela glória de vosso nome, ó Senhor, livrai-nos: * e perdoai os nossos pecados, por amor de vosso nome:
}\switchcolumn*\latim{
Ne forte dicant in géntibus: ubi est Deus eórum? * Et innotéscat in natiónibus coram óculis nostris.
}\switchcolumn\portugues{
Para que se não diga entre as gentes: o Deus deles onde está? * Fazei brilhar entre as nações e ante nossos olhos.
}\switchcolumn*\latim{
Ultio sánguinis servórum tuórum, qui effúsus est: * intróeat in conspéctu tuo gémitus compeditórum.
}\switchcolumn\portugues{
A vingança do sangue de vossos servos, que tem sido derramado: * cheguem à vossa presença os gemidos dos cativos.
}\switchcolumn*\latim{
Secúndum magnitúdinem brácchii tui, * pósside fílios mortificatórum.
}\switchcolumn\portugues{
Com o poder de vosso braço, * conservai os filhos dos que foram mortos.
}\switchcolumn*\latim{
Et redde vicínis nostris séptuplum in sinu eórum: * impropérium ipsórum, quod exprobravérunt tibi, Dómine.
}\switchcolumn\portugues{
Pagai aos nossos vizinhos com males sete vezes maiores: * a desonra que eles Vos fizeram, ó Senhor.
}\switchcolumn*\latim{
Nos autem pópulus tuus, et oves páscuæ tuæ, * confitébimur tibi in sǽculum.
}\switchcolumn\portugues{
Nós, porém, vosso povo e ovelhas de vosso pasto, * nós Vos glorificaremos para sempre.
}\switchcolumn*\latim{
In generatiónem et generatiónem * annuntiábimus laudem tuam.
}\switchcolumn\portugues{
De geração em geração * publicaremos os vossos louvores.
}\end{paracol}


\subsectioninfo{Salmo 79}{Qui regis Israël}\label{salmo79}
\begin{paracol}{2}\latim{
\qlettrine{Q}{ui} regis Israël, inténde: * qui dedúcis velut ovem Joseph.
}\switchcolumn\portugues{
\rlettrine{V}{ós} que governais Israel, atendei: * que como uma ovelha conduzis José.
}\switchcolumn*\latim{
Qui sedes super Chérubim, * manifestáre coram Éphraim, Bénjamin, et Manásse.
}\switchcolumn\portugues{
Que estais sentado sobre os querubins, * manifestai ante Efraim, Benjamim e Manassés.
}\switchcolumn*\latim{
Éxcita poténtiam tuam, et veni, * ut salvos fácias nos.
}\switchcolumn\portugues{
Mostrai o vosso poder e vem, * para nos salvar.
}\switchcolumn*\latim{
Deus, convérte nos: * et osténde fáciem tuam, et salvi érimus.
}\switchcolumn\portugues{
Ó Deus, convertei-nos: * mostrai-nos o vosso rosto e seremos salvos.
}\switchcolumn*\latim{
Dómine, Deus virtútum, * quoúsque irascéris super oratiónem servi tui?
}\switchcolumn\portugues{
Senhor Deus dos exércitos, * até quando estareis furioso, sem ouvir a oração de vosso servo?
}\switchcolumn*\latim{
Cibábis nos pane lacrimárum: * et potum dabis nobis in lácrimis in mensúra?
}\switchcolumn\portugues{
Até quando nos sustentareis com pão de lágrimas: * e nos dareis a beber lágrimas com abundância?
}\switchcolumn*\latim{
Posuísti nos in contradictiónem vicínis nostris: * et inimíci nostri subsannavérunt nos.
}\switchcolumn\portugues{
Fizestes-nos um objecto de disputa para os nossos vizinhos: * e os nossos inimigos fizeram escárnio de nós.
}\switchcolumn*\latim{
Deus virtútum, convérte nos: * et osténde fáciem tuam, et salvi érimus.
}\switchcolumn\portugues{
Deus dos exércitos, restaurai-nos: * mostrai-nos o vosso rosto e seremos salvos.
}\switchcolumn*\latim{
Víneam de Ægýpto transtulísti: * ejecísti gentes, et plantásti eam.
}\switchcolumn\portugues{
Trasladastes a vossa vinha do Egipto: * plantaste-la em seu lugar e lançastes fora as gentes.
}\switchcolumn*\latim{
Dux itíneris fuísti in conspéctu ejus: * plantásti radíces ejus, et implévit terram.
}\switchcolumn\portugues{
Fostes guia no caminho diante dela: * plantastes as suas raízes e encheu a terra.
}\switchcolumn*\latim{
Opéruit montes umbra ejus: * et arbústa ejus cedros Dei.
}\switchcolumn\portugues{
Sua sombra cobriu os montes: * e os seus ramos os cedros de Deus.
}\switchcolumn*\latim{
Exténdit pálmites suos usque ad mare: * et usque ad flumen propágines ejus.
}\switchcolumn\portugues{
Estendeu a sua ramagem até ao mar: * e até ao rio os seus rebentos.
}\switchcolumn*\latim{
Ut quid destruxísti macériam ejus: * et vindémiant eam omnes, qui prætergrediúntur viam?
}\switchcolumn\portugues{
Para que destruístes o seu muro: * para que a vindimem todos os que pelo caminho passam?
}\switchcolumn*\latim{
Exterminávit eam aper de silva: * et singuláris ferus depástus est eam.
}\switchcolumn\portugues{
O javali da selva destruiu-a: * e a fera selvagem a devorou.
}\switchcolumn*\latim{
Deus virtútum, convértere: * réspice de cælo, et vide, et vísita víneam istam.
}\switchcolumn\portugues{
Ó Deus dos exércitos, voltai-Vos: * olhai do céu, vede e visitai esta vinha.
}\switchcolumn*\latim{
Et pérfice eam, quam plantávit déxtera tua: * et super fílium hóminis, quem confirmásti tibi.
}\switchcolumn\portugues{
Protegei aquela que a vossa dextra plantou: * e olhai para o filho do homem, a quem escolhestes.
}\switchcolumn*\latim{
Incénsa igni, et suffóssa * ab increpatióne vultus tui períbunt.
}\switchcolumn\portugues{
Ela foi queimada pelo fogo e escavada: * ante vosso rosto perecerá.
}\switchcolumn*\latim{
Fiat manus tua super virum déxteræ tuæ: * et super fílium hóminis, quem confirmásti tibi.
}\switchcolumn\portugues{
Estendei a vossa mão sobre o homem de vossa dextra: * e sobre o filho do homem que escolhestes para Vós.
}\switchcolumn*\latim{
Et non discédimus a Te, vivificábis nos: * et nomen tuum invocábimus.
}\switchcolumn\portugues{
Então nos não afastaremos de Vós, vida nos dareis: * e invocaremos o vosso nome.
}\switchcolumn*\latim{
Dómine, Deus virtútum, convérte nos: * et osténde fáciem tuam, et salvi érimus.
}\switchcolumn\portugues{
Ó Senhor Deus dos exércitos, convertei-nos: * mostrai-nos o vosso rosto e seremos salvos.
}\end{paracol}


\subsectioninfo{Salmo 80}{Exsultate Deo adjutori nostro}\label{salmo80}
\begin{paracol}{2}\latim{
\rlettrine{E}{xsultáte} Deo, adjutóri nostro: * jubiláte Deo Jacob.
}\switchcolumn\portugues{
\rlettrine{E}{xultai-vos} louvando a Deus, nosso protector: * cantai com alegria a Deus de Jacob.
}\switchcolumn*\latim{
Súmite psalmum, et date týmpanum: * psaltérium jucúndum cum cíthara.
}\switchcolumn\portugues{
Entoai um salmo e tocai os timbales: * o saltério harmonioso, com a cítara.
}\switchcolumn*\latim{
Buccináte in Neoménia tuba, * in insígni die solemnitátis vestræ.
}\switchcolumn\portugues{
Tocai a trombeta na lua nova, * no dia notável de vossa solenidade.
}\switchcolumn*\latim{
Quia præcéptum in Israël est: * et judícium Deo Jacob.
}\switchcolumn\portugues{
Pois é um preceito para Israel: * e uma ordem do Deus de Jacob.
}\switchcolumn*\latim{
Testimónium in Joseph pósuit illud, cum exíret de terra Ægýpti: * linguam, quam non nóverat, audívit.
}\switchcolumn\portugues{
Estabeleceu isto como lei para José, quando saía da terra do Egipto: * quando ouviu uma língua que não entendia.
}\switchcolumn*\latim{
Divértit ab onéribus dorsum ejus: * manus ejus in cóphino serviérunt.
}\switchcolumn\portugues{
Libertou os seus ombros dos fardos: * as suas mãos escravizadas nos cestos.
}\switchcolumn*\latim{
In tribulatióne invocásti me, et liberávi te: * exaudívi te in abscóndito tempestátis: probávi te apud aquam contradictiónis.
}\switchcolumn\portugues{
Na tribulação me invocaste e eu te livrei: * ouvi-te no recôndito da tempestade, provei-te junto das águas da contradição.
}\switchcolumn*\latim{
Audi, pópulus meus, et contestábor te: * Israël, si audíeris me, non erit in te deus recens, neque adorábis deum aliénum.
}\switchcolumn\portugues{
Ouve, ó povo meu, e eu te instruirei: * Israel, se me ouvires, não haverá em ti deus novo, nem deus estranho adorarás.
}\switchcolumn*\latim{
Ego enim sum Dóminus Deus tuus, qui edúxi te de terra Ægýpti: * diláta os tuum, et implébo illud.
}\switchcolumn\portugues{
Eu sou, de facto, o Senhor teu Deus, que te tirei da terra do Egipto: * abre a tua boca e a rechearei.
}\switchcolumn*\latim{
Et non audívit pópulus meus vocem meam: * et Israël non inténdit mihi.
}\switchcolumn\portugues{
Meu povo não ouviu minha voz: * e Israel me não atendeu.
}\switchcolumn*\latim{
Et dimísi eos secúndum desidéria cordis eórum: * ibunt in adinventiónibus suis.
}\switchcolumn\portugues{
Abandonei-os aos desejos do seu coração: * eles irão caminhando atrás dos seus devaneios.
}\switchcolumn*\latim{
Si pópulus meus audísset me: * Israël si in viis meis ambulásset:
}\switchcolumn\portugues{
Se o meu povo me tivesse ouvido: * se Israel tivesse andado nos meus caminhos:
}\switchcolumn*\latim{
Pro níhilo fórsitan inimícos eórum humiliássem: * et super tribulántes eos misíssem manum meam.
}\switchcolumn\portugues{
Facilmente teria podido humilhar os seus inimigos: * e a minha mão teria caído sobre os seus opressores.
}\switchcolumn*\latim{
Inimíci Dómini mentíti sunt ei: * et erit tempus eórum in sǽcula.
}\switchcolumn\portugues{
Os inimigos do Senhor mentiram-Lhe: * e o tempo deles será eterno.
}\switchcolumn*\latim{
Et cibávit eos ex ádipe fruménti: * et de petra, melle saturávit eos.
}\switchcolumn\portugues{
Apesar disso alimentou-os da flor do trigo: * e saciou-os de mel saído da pedra.
}\end{paracol}


\subsectioninfo{Salmo 81}{Deus stetit in synagoga deorum}\label{salmo81}
\begin{paracol}{2}\latim{
\rlettrine{D}{eus} stetit in synagóga deórum: * in médio autem deos dijúdicat.
}\switchcolumn\portugues{
\rlettrine{D}{eus} está presente no conselho dos deuses: * no meio deles julga os mesmos deuses.
}\switchcolumn*\latim{
Úsquequo judicátis iniquitátem: * et fácies peccatórum súmitis?
}\switchcolumn\portugues{
Até quando julgareis injustamente: * e tereis em favor os pecadores?
}\switchcolumn*\latim{
Judicáte egéno, et pupíllo: * húmilem, et páuperem justificáte.
}\switchcolumn\portugues{
Fazei justiça ao necessitado e ao órfão: * atendei à razão do humilde e do pobre.
}\switchcolumn*\latim{
Erípite páuperem: * et egénum de manu peccatóris liberáte.
}\switchcolumn\portugues{
Resgatai o pobre: * e livrai o desvalido da mão do pecador.
}\switchcolumn*\latim{
Nesciérunt, neque intellexérunt, in ténebris ámbulant: * movebúntur ómnia fundaménta terræ.
}\switchcolumn\portugues{
Não souberam nem entenderam, andam nas trevas: * serão abalados todos os fundamentos da terra.
}\switchcolumn*\latim{
Ego dixi: Dii estis, * et fílii Excélsi omnes.
}\switchcolumn\portugues{
Eu disse: sois deuses, * e todos filhos do Excelso.
}\switchcolumn*\latim{
Vos autem sicut hómines moriémini: * et sicut unus de princípibus cadétis.
}\switchcolumn\portugues{
Contudo, vós como homens morrereis: * e caireis como um qualquer príncipe.
}\switchcolumn*\latim{
Surge, Deus, júdica terram: * quóniam Tu hereditábis in ómnibus géntibus.
}\switchcolumn\portugues{
Levantai-Vos, ó Deus, julgai a terra: * porque todos as gentes são vossa herança.
}\end{paracol}


\subsectioninfo{Salmo 82}{Deus, quis similis}\label{salmo82}
\begin{paracol}{2}\latim{
\rlettrine{D}{eus,} quis símilis erit tibi? * Ne táceas, neque compescáris, Deus.
}\switchcolumn\portugues{
\slettrine{Ó}{} Deus, quem a Vós será semelhante? * Não silenciais, ó Deus, nem Vos detenhais.
}\switchcolumn*\latim{
Quóniam ecce inimíci tui sonuérunt: * et qui odérunt Te extulérunt caput.
}\switchcolumn\portugues{
Porque eis que os vossos inimigos gritam: * e os que Vos odeiam levantaram a cabeça.
}\switchcolumn*\latim{
Super pópulum tuum malignavérunt consílium: * et cogitavérunt advérsus sanctos tuos.
}\switchcolumn\portugues{
Vil planos formaram contra o vosso povo: * e conspiraram contra os vossos santos.
}\switchcolumn*\latim{
Dixérunt: veníte, et disperdámus eos de gente: * et non memorétur nomen Israël ultra.
}\switchcolumn\portugues{
Disseram: vinde e exterminemo-los entre as gentes: * e não haja mais memória do nome de Israel.
}\switchcolumn*\latim{
Quóniam cogitavérunt unanímiter: * simul advérsum Te testaméntum disposuérunt, tabernácula Idumæórum et Ismahelítæ:
}\switchcolumn\portugues{
Pensaram de acordo: * e todos juntos fizeram aliança contra Vós, as tendas dos Idumeus e os Ismælitas:
}\switchcolumn*\latim{
Moab, et Agaréni, Gebal, et Ammon, et Ámalec: * alienígenæ cum habitántibus Tyrum.
}\switchcolumn\portugues{
Moab e os Agarenos, Gebal e Amon e Amalec: * os estrangeiros com os moradores de Tiro.
}\switchcolumn*\latim{
Étenim Assur venit cum illis: * facti sunt in adjutórium fíliis Lot.
}\switchcolumn\portugues{
Assur também se uniu com eles: * juntaram-se para auxiliarem os filhos de Lot.
}\switchcolumn*\latim{
Fac illis sicut Mádian, et Sísaræ: * sicut Jabin in torrénte Cisson.
}\switchcolumn\portugues{
Fazei-lhes como a Median e a Sisara: * como a Jabin no ribeiro de Cisson.
}\switchcolumn*\latim{
Disperiérunt in Endor: * facti sunt ut stercus terræ.
}\switchcolumn\portugues{
Foram destruídos em Endor: * tornaram-se como o esterco da terra.
}\switchcolumn*\latim{
Pone príncipes eórum sicut Oreb, et Zeb, * et Zébee, et Sálmana:
}\switchcolumn\portugues{
Tratai os seus príncipes como Oreb e Zeb, * e Zebee e Salmana:
}\switchcolumn*\latim{
Omnes príncipes eórum: * qui dixérunt: hereditáte possideámus Sanctuárium Dei.
}\switchcolumn\portugues{
Todos seus príncipes: * os quais tinham dito: apoderemos como herança o santuário de Deus.
}\switchcolumn*\latim{
Deus meus, pone illos ut rotam: * et sicut stípulam ante fáciem venti.
}\switchcolumn\portugues{
Ó meu Deus, agitai-os como uma roda: * e como uma palhinha diante do vento.
}\switchcolumn*\latim{
Sicut ignis, qui combúrit silvam: * et sicut flamma combúrens montes:
}\switchcolumn\portugues{
Como fogo que queima uma selva: * e como chama que abrasa os montes:
}\switchcolumn*\latim{
Ita persequéris illos in tempestáte tua: * et in ira tua turbábis eos.
}\switchcolumn\portugues{
Assim os perseguireis com vossa tempestade: * e com vossa ira os aterrareis.
}\switchcolumn*\latim{
Imple fácies eórum ignomínia: * et quǽrent nomen tuum, Dómine.
}\switchcolumn\portugues{
Cobri os seus rostos de ignomínia: * e deste modo buscarão o vosso nome, ó Senhor.
}\switchcolumn*\latim{
Erubéscant, et conturbéntur in sǽculum sǽculi: * et confundántur, et péreant.
}\switchcolumn\portugues{
Sejam envergonhados e conturbados para sempre: * sejam confundidos e pereçam.
}\switchcolumn*\latim{
Et cognóscant quia nomen tibi Dóminus: * Tu solus Altíssimus in omni terra.
}\switchcolumn\portugues{
Conheçam que Vos é próprio o nome de Senhor: * e que só Vós sois o Altíssimo em toda a terra.
}\end{paracol}


\subsectioninfo{Salmo 83}{Quam dilécta tabernacula}\label{salmo83}
\begin{paracol}{2}\latim{
\qlettrine{Q}{uam} dilécta tabernácula tua, Dómine virtútum: * concupíscit, et déficit ánima mea in átria Dómini.
}\switchcolumn\portugues{
\qlettrine{Q}{uão} amáveis são os vossos tabernáculos, Senhor dos exércitos: * a minha alma suspira e desfalece pelos átrios do Senhor.
}\switchcolumn*\latim{
Cor meum, et caro mea * exsultavérunt in Deum vivum.
}\switchcolumn\portugues{
Meu coração e minha carne * regozijam-se no Deus vivo.
}\switchcolumn*\latim{
Étenim passer invénit sibi domum: * et turtur nidum sibi, ubi ponat pullos suos.
}\switchcolumn\portugues{
De facto, o passarinho acha casa para si: * e a rola um ninho para lá pôr os seus filhinhos.
}\switchcolumn*\latim{
Altária tua, Dómine virtútum: * Rex meus, et Deus meus.
}\switchcolumn\portugues{
Vossos altares, Senhor dos exércitos: * meu Rei e meu Deus.
}\switchcolumn*\latim{
Beáti, qui hábitant in domo tua, Dómine: * in sǽcula sæculórum laudábunt Te.
}\switchcolumn\portugues{
Senhor, bem-aventurados os que moram na vossa casa: * pelos séculos dos séculos Vos louvarão.
}\switchcolumn*\latim{
Beátus vir, cujus est auxílium abs Te: * ascensiónes in corde suo dispósuit, in valle lacrimárum in loco, quem pósuit.
}\switchcolumn\portugues{
Bem-aventurado o varão que de Vós espera auxílio: * preparou elevações no seu coração, neste vale de lágrimas, no lugar que destinou.
}\switchcolumn*\latim{
Étenim benedictiónem dabit legislátor, ibunt de virtúte in virtútem: * vidébitur Deus deórum in Sion.
}\switchcolumn\portugues{
De facto, o legislador dar-lhe-á a sua bênção, caminhará de virtude em virtude: * será visto o Deus dos deuses em Sião.
}\switchcolumn*\latim{
Dómine, Deus virtútum, exáudi oratiónem meam: * áuribus pércipe, Deus Jacob.
}\switchcolumn\portugues{
Senhor, ó Deus dos exércitos, ouvi a minha oração: * prestai ouvidos, ó Deus de Jacob.
}\switchcolumn*\latim{
Protéctor noster, áspice, Deus: * et réspice in fáciem Christi tui:
}\switchcolumn\portugues{
Ó Deus nosso protector, olhai para nós: * e ponde os olhos no rosto de vosso Cristo:
}\switchcolumn*\latim{
Quia mélior est dies una in átriis tuis, * super míllia.
}\switchcolumn\portugues{
Pois é melhor um só dia nos vossos átrios, * que milhares.
}\switchcolumn*\latim{
Elégi abjéctus esse in domo Dei mei: * magis quam habitáre in tabernáculis peccatórum.
}\switchcolumn\portugues{
Preferi ser o último na casa do meu Deus: * a morar nas tendas dos pecadores.
}\switchcolumn*\latim{
Quia misericórdiam, et veritátem díligit Deus: * grátiam et glóriam dabit Dóminus.
}\switchcolumn\portugues{
Pois Deus ama a misericórdia e a verdade: * o Senhor dará a graça e a glória.
}\switchcolumn*\latim{
Non privábit bonis eos, qui ámbulant in innocéntia: * Dómine virtútum, beátus homo, qui sperat in Te.
}\switchcolumn\portugues{
Não privará de seus bens aqueles que andam na inocência: * ó Senhor dos exércitos, bem-aventurado o homem que em Vós espera.
}\end{paracol}


\subsectioninfo{Salmo 84}{Benedixisti, Domine}\label{salmo84}
\begin{paracol}{2}\latim{
\rlettrine{B}{enedixísti,} Dómine, terram tuam: * avertísti captivitátem Jacob.
}\switchcolumn\portugues{
\rlettrine{A}{bençoastes,} ó Senhor, a vossa terra: * libertastes Jacob do cativeiro.
}\switchcolumn*\latim{
Remisísti iniquitátem plebis tuæ: * operuísti ómnia peccáta eórum.
}\switchcolumn\portugues{
Perdoastes a iniquidade de vosso povo: * cobristes todos seus pecados.
}\switchcolumn*\latim{
Mitigásti omnem iram tuam: * avertísti ab ira indignatiónis tuæ.
}\switchcolumn\portugues{
Mitigastes toda vossa ira: * suspendestes a raiva de vossa indignação.
}\switchcolumn*\latim{
Convérte nos, Deus, salutáris noster: * et avérte iram tuam a nobis.
}\switchcolumn\portugues{
Convertei-nos, ó Deus, nosso Salvador: * e afastai de nós a vossa ira.
}\switchcolumn*\latim{
Numquid in ætérnum irascéris nobis? * Aut exténdes iram tuam a generatióne in generatiónem?
}\switchcolumn\portugues{
Estareis porventura para sempre irado connosco? * Ou estendereis a vossa ira de geração em geração?
}\switchcolumn*\latim{
Deus, Tu convérsus vivificábis nos: * et plebs tua lætábitur in Te.
}\switchcolumn\portugues{
Ó Deus, voltando-Vos restituir-nos-eis a vida: * e o vosso povo alegrar-se-á em Vós.
}\switchcolumn*\latim{
Osténde nobis, Dómine, misericórdiam tuam: * et salutáre tuum da nobis.
}\switchcolumn\portugues{
Mostrai-nos, ó Senhor, a vossa misericórdia: * e dai-nos a vossa salvação.
}\switchcolumn*\latim{
Audiam quid loquátur in me Dóminus Deus: * quóniam loquétur pacem in plebem suam.
}\switchcolumn\portugues{
Ouvirei o que me disser o Senhor Deus: * porque anunciará Ele a paz ao seu povo.
}\switchcolumn*\latim{
Et super sanctos suos: * et in eos, qui convertúntur ad cor.
}\switchcolumn\portugues{
Aos seus santos: * e àqueles que se convertem de coração.
}\switchcolumn*\latim{
Verúmtamen prope timéntes eum salutáre ipsíus: * ut inhábitet glória in terra nostra.
}\switchcolumn\portugues{
Sim, a sua salvação está perto dos que O temem: * e a glória habitará na nossa terra.
}\switchcolumn*\latim{
Misericórdia, et véritas obviavérunt sibi: * justítia, et pax osculátæ sunt.
}\switchcolumn\portugues{
A misericórdia e a verdade se encontraram: * a justiça e a paz se beijaram.
}\switchcolumn*\latim{
Véritas de terra orta est: * et justítia de cælo prospéxit.
}\switchcolumn\portugues{
A verdade brotou da terra: * e a justiça olhou do céu.
}\switchcolumn*\latim{
Étenim Dóminus dabit benignitátem: * et terra nostra dabit fructum suum.
}\switchcolumn\portugues{
De facto, o Senhor dará a sua bondade: * e a nossa terra produzirá o seu fruto.
}\switchcolumn*\latim{
Justítia ante eum ambulábit: * et ponet in via gressus suos.
}\switchcolumn\portugues{
Adiante d’Ele irá a justiça : * e imprimirá os seus passos no caminho.
}\end{paracol}


\subsectioninfo{Salmo 85}{Inclina, Domine}\label{salmo85}
\begin{paracol}{2}\latim{
\rlettrine{I}{nclína,} Dómine, aurem tuam, et exáudi me: * quóniam inops, et pauper sum ego.
}\switchcolumn\portugues{
\rlettrine{I}{nclinai,} ó Senhor, o vosso ouvido e ouvi-me: * porque estou carente e pobre.
}\switchcolumn*\latim{
Custódi ánimam meam, quóniam sanctus sum: * salvum fac servum tuum, Deus meus, sperántem in Te.
}\switchcolumn\portugues{
Velai a minha alma, porque sou santo: * salvai, ó meu Deus, o vosso servo, que em Vós espera.
}\switchcolumn*\latim{
Miserére mei, Dómine, quóniam ad Te clamávi tota die: * lætífica ánimam servi tui, quóniam ad Te, Dómine, ánimam meam levávi.
}\switchcolumn\portugues{
Senhor, tende misericórdia de mim, porque a Vós clamei todo dia: * alegrai a alma de vosso servo, porque a Vós, ó Senhor, elevei a minha alma.
}\switchcolumn*\latim{
Quóniam Tu, Dómine, suávis, et mitis: * et multæ misericórdiæ ómnibus invocántibus Te.
}\switchcolumn\portugues{
Porque Vós, ó Senhor, sois suave e manso: * e de muita misericórdia para todos os que Vos invocam.
}\switchcolumn*\latim{
Áuribus pércipe, Dómine, oratiónem meam: * et inténde voci deprecatiónis meæ.
}\switchcolumn\portugues{
Prestai ouvidos, ó Senhor, à minha oração: * e atendei à voz da minha súplica.
}\switchcolumn*\latim{
In die tribulatiónis meæ clamávi ad Te: * quia exaudísti me.
}\switchcolumn\portugues{
No dia da minha tribulação clamei a Vós: * pois me tendes ouvido.
}\switchcolumn*\latim{
Non est símilis tui in diis, Dómine: * et non est secúndum ópera tua.
}\switchcolumn\portugues{
Não há semelhante a Vós nos deuses, ó Senhor: * e conforme vossas obras não há.
}\switchcolumn*\latim{
Omnes gentes quascúmque fecísti, vénient, et adorábunt coram Te, Dómine: * et glorificábunt nomen tuum.
}\switchcolumn\portugues{
Senhor, todas as gentes que criastes virão e prostradas Vos adorarão: * e glorificarão o vosso nome.
}\switchcolumn*\latim{
Quóniam magnus es Tu, et fáciens mirabília: * Tu es Deus solus.
}\switchcolumn\portugues{
Porque Vós sois grande e fazeis maravilhas: * só Vós sois Deus.
}\switchcolumn*\latim{
Deduc me, Dómine, in via tua, et ingrédiar in veritáte tua: * lætétur cor meum ut tímeat nomen tuum.
}\switchcolumn\portugues{
Guiai-me, ó Senhor, pelo vosso caminho e andarei na vossa verdade: * alegre-se o meu coração no temor do vosso nome.
}\switchcolumn*\latim{
Confitébor tibi, Dómine, Deus meus, in toto corde meo, * et glorificábo nomen tuum in ætérnum:
}\switchcolumn\portugues{
Louvar-Vos-ei, ó Senhor meu Deus, com todo meu coração, * e glorificarei eternamente o vosso nome:
}\switchcolumn*\latim{
Quia misericórdia tua magna est super me: * et eruísti ánimam meam ex inférno inferióri.
}\switchcolumn\portugues{
Pois vossa misericórdia é grande para comigo: * e livrastes a minha alma do mais profundo inferno.
}\switchcolumn*\latim{
Deus, iníqui insurrexérunt super me, et synagóga poténtium quæsiérunt ánimam meam: * et non proposuérunt Te in conspéctu suo.
}\switchcolumn\portugues{
Ó Deus, levantaram-se os maus contra mim e atentou contra a minha vida uma reunião de poderosos: * sem que Vos tivessem ante seus olhos presente.
}\switchcolumn*\latim{
Et Tu, Dómine, Deus miserátor et miséricors, * pátiens, et multæ misericórdiæ, et verax,
}\switchcolumn\portugues{
Vós sois, ó Senhor Deus, compassivo e clemente, * paciente, de muita misericórdia e veraz,
}\switchcolumn*\latim{
Réspice in me, et miserére mei, * da impérium tuum púero tuo: et salvum fac fílium ancíllæ tuæ.
}\switchcolumn\portugues{
Olhai para mim e tende piedade de mim, * dai o vosso império ao vosso servo e salvai o filho de vossa serva.
}\switchcolumn*\latim{
Fac mecum signum in bonum, ut vídeant qui odérunt me, et confundántur: * quóniam Tu, Dómine, adjuvísti me, et consolátus es me.
}\switchcolumn\portugues{
Operai em mim sinais de bondade, para que vejam os que me odeiam e sejam confundidos: * porque Vós, Senhor, me tendes socorrido e consolado.
}\end{paracol}


\subsectioninfo{Salmo 86}{Fundamenta ejus}\label{salmo86}
\begin{paracol}{2}\latim{
\rlettrine{F}{undaménta} ejus in móntibus sanctis: * díligit Dóminus portas Sion super ómnia tabernácula Jacob.
}\switchcolumn\portugues{
\rlettrine{O}{s} seus fundamentos estão sobre os montes santos: * o Senhor ama as portas de Sião mais que todos os tabernáculos de Jacob.
}\switchcolumn*\latim{
Gloriósa dicta sunt de te, * cívitas Dei.
}\switchcolumn\portugues{
Cousas gloriosas se têm dito de ti, * ó cidade de Deus.
}\switchcolumn*\latim{
Memor ero Rahab, et Babylónis * sciéntium me.
}\switchcolumn\portugues{
Lembrar-me-ei de Raab e de Babilónia, * que me conhecem.
}\switchcolumn*\latim{
Ecce, alienígenæ, et Tyrus, et pópulus Æthíopum, * hi fuérunt illic.
}\switchcolumn\portugues{
Eis os estrangeiros, Tiro e o povo dos Etíopes, * todos estes estarão lá.
}\switchcolumn*\latim{
Numquid Sion dicet: homo, et homo natus est in ea: * et ipse fundávit eam Altíssimus?
}\switchcolumn\portugues{
Porventura se não dirá a Sião: um grande número de homens nasceu nela: * e a fundou o mesmo Altíssimo?
}\switchcolumn*\latim{
Dóminus narrábit in scriptúris populórum, et príncipum: * horum, qui fuérunt in ea.
}\switchcolumn\portugues{
O Senhor poderá contar, no registo dos povos e dos príncipes: * o número daqueles que nela estiveram.
}\switchcolumn*\latim{
Sicut lætántium ómnium * habitátio est in te.
}\switchcolumn\portugues{
Estão todos  cheios de alegria* os que habitam dentro de ti.
}\end{paracol}


\subsectioninfo{Salmo 87}{Domine, Deus salutis meæ}\label{salmo87}
\begin{paracol}{2}\latim{
\rlettrine{D}{ómine,} Deus salútis meæ: * in die clamávi, et nocte coram Te.
}\switchcolumn\portugues{
\rlettrine{S}{enhor} Deus da minha salvação: * de dia e de noite clamei ante Vós.
}\switchcolumn*\latim{
Intret in conspéctu tuo orátio mea: * inclína aurem tuam ad precem meam:
}\switchcolumn\portugues{
Chegue à vossa presença a minha oração: * inclinai o vosso ouvido à minha súplica:
}\switchcolumn*\latim{
Quia repléta est malis ánima mea: * et vita mea inférno appropinquávit.
}\switchcolumn\portugues{
Pois a minha alma está repleta de males: * e a minha vida aproxima-se do inferno.
}\switchcolumn*\latim{
Æstimátus sum cum descendéntibus in lacum: * factus sum sicut homo sine adjutório, inter mórtuos liber.
}\switchcolumn\portugues{
Sou contado entre os que descem à cova: * tornei-me como um homem sem socorro, abandonado entre os mortos.
}\switchcolumn*\latim{
Sicut vulneráti dormiéntes in sepúlcris, quorum non es memor ámplius: * et ipsi de manu tua repúlsi sunt.
}\switchcolumn\portugues{
Como os feridos que dormem nos sepulcros, de quem já Vos não lembras: * e que foram repelidos de vossa mão.
}\switchcolumn*\latim{
Posuérunt me in lacu inferióri: * in tenebrósis, et in umbra mortis.
}\switchcolumn\portugues{
Puseram-me num fosso profundo: * em lugares tenebrosos e na sombra da morte.
}\switchcolumn*\latim{
Super me confirmátus est furor tuus: * et omnes fluctus tuos induxísti super me.
}\switchcolumn\portugues{
Sobre mim pesou a vossa fúria: * e fizestes vir sobre mim todas vossas ondas.
}\switchcolumn*\latim{
Longe fecísti notos meos a me: * posuérunt me abominatiónem sibi.
}\switchcolumn\portugues{
Afastastes de mim os meus conhecidos, fizeram de mim o objecto da sua abominação.
}\switchcolumn*\latim{
Tráditus sum, et non egrediébar: * óculi mei languérunt præ inópia.
}\switchcolumn\portugues{
Entregue fui e sem poder sair: * os meus olhos desfaleceram de miséria.
}\switchcolumn*\latim{
Clamávi ad Te, Dómine, tota die: * expándi ad Te manus meas.
}\switchcolumn\portugues{
A Vós, ó Senhor, clamei todo o dia: * para Vós estendi as minhas mãos.
}\switchcolumn*\latim{
Numquid mórtuis fácies mirabília: * aut médici suscitábunt, et confitebúntur tibi?
}\switchcolumn\portugues{
Porventura fareis milagres em mercê dos mortos: * porventura os médicos os ressuscitarão, para que Vos louvem?
}\switchcolumn*\latim{
Numquid narrábit áliquis in sepúlcro misericórdiam tuam, * et veritátem tuam in perditióne?
}\switchcolumn\portugues{
Acaso publicará alguém na sepultura a vossa misericórdia, * e a vossa verdade na perdição?
}\switchcolumn*\latim{
Numquid narrábit áliquis in sepúlcro misericórdiam tuam, * et veritátem tuam in perditióne?
}\switchcolumn\portugues{
Acaso publicará alguém na sepultura a vossa misericórdia, * e a vossa verdade no túmulo?
}\switchcolumn*\latim{
Numquid cognoscéntur in ténebris mirabília tua, * et justítia tua in terra obliviónis?
}\switchcolumn\portugues{
Porventura as vossas maravilhas serão conhecidas nas trevas, * e a vossa justiça na terra do esquecimento?
}\switchcolumn*\latim{
Et ego ad Te, Dómine, clamávi: * et mane orátio mea prævéniet Te.
}\switchcolumn\portugues{
Por isso eu, ó Senhor, a Vós clamo: * e logo de manhã vai ante Vós a minha oração.
}\switchcolumn*\latim{
Ut quid, Dómine, repéllis oratiónem meam: * avértis fáciem tuam a me?
}\switchcolumn\portugues{
Porque rejeitais, ó Senhor, a minha oração: * e apartais de mim a vossa face?
}\switchcolumn*\latim{
Pauper sum ego, et in labóribus a juventúte mea: * exaltátus autem, humiliátus sum et conturbátus.
}\switchcolumn\portugues{
Sou um pobre e vivo em trabalhos desde a minha mocidade: * e, depois de exaltado, fui humilhado e conturbado.
}\switchcolumn*\latim{
In me transiérunt iræ tuæ: * et terróres tui conturbavérunt me.
}\switchcolumn\portugues{
Por cima de mim passaram as vossas iras: * e os vossos terrores me conturbaram.
}\switchcolumn*\latim{
Circumdedérunt me sicut aqua tota die: * circumdedérunt me simul.
}\switchcolumn\portugues{
Cercaram-me com água todo o dia: * juntos me cercaram.
}\switchcolumn*\latim{
Elongásti a me amícum et próximum: * et notos meos a miséria.
}\switchcolumn\portugues{
Afastastes de mim amigos e parentes: * e os meus conhecidos, devido à miséria.
}\end{paracol}


\subsectioninfo{Salmo 88}{Misericordias Domini}\label{salmo88}
\begin{paracol}{2}\latim{
\rlettrine{M}{isericórdias} Dómini * in ætérnum cantábo.
}\switchcolumn\portugues{
\rlettrine{A}{s} misericórdias do Senhor * cantarei eternamente.
}\switchcolumn*\latim{
In generatiónem et generatiónem * annuntiábo veritátem tuam in ore meo.
}\switchcolumn\portugues{
De geração em geração * pela minha boca anunciarei a vossa verdade.
}\switchcolumn*\latim{
Quóniam dixísti: in ætérnum misericórdia ædificábitur in cælis: * præparábitur véritas tua in eis.
}\switchcolumn\portugues{
Porquanto dissestes: a misericórdia edificar-se-á eternamente nos céus: * a vossa verdade será preparada neles.
}\switchcolumn*\latim{
Dispósui testaméntum eléctis meis, jurávi David, servo meo: * Usque in ætérnum præparábo semen tuum.
}\switchcolumn\portugues{
Fiz aliança com meus escolhidos, jurei a David meu servo: * conservarei eternamente a vossa descendência.
}\switchcolumn*\latim{
Et ædificábo in generatiónem et generatiónem * sedem tuam.
}\switchcolumn\portugues{
De geração em geração edificarei * o vosso trono.
}\switchcolumn*\latim{
Confitebúntur cæli mirabília tua, Dómine: * étenim veritátem tuam in ecclésia sanctórum.
}\switchcolumn\portugues{
Os céus declararão as vossas maravilhas, ó Senhor: * e também na igreja dos santos a vossa verdade.
}\switchcolumn*\latim{
Quóniam quis in núbibus æquábitur Dómino: * símilis erit Deo in fíliis Dei?
}\switchcolumn\portugues{
Porque quem, nas nuvens, será igual ao Senhor: * e quem dos filhos de Deus, será semelhante a Deus?
}\switchcolumn*\latim{
Deus, qui glorificátur in consílio sanctórum: * magnus et terríbilis super omnes qui in circúitu ejus sunt.
}\switchcolumn\portugues{
A Deus, que é glorificado no conselho dos santos: * grande e terrível sobre todos os que estão à volta d’Ele.
}\switchcolumn*\latim{
Dómine, Deus virtútum, quis símilis tibi? * Potens es, Dómine, et véritas tua in circúitu tuo.
}\switchcolumn\portugues{
Ó Senhor Deus dos exércitos, quem é semelhante a Vós? * Sois poderoso, ó Senhor e a vossa verdade Vos rodeia.
}\switchcolumn*\latim{
Tu domináris potestáti maris: * motum autem flúctuum ejus Tu mítigas.
}\switchcolumn\portugues{
Vós dominais sobre o poder do mar: * e amansas o movimento das suas ondas.
}\switchcolumn*\latim{
Tu humiliásti sicut vulnerátum, supérbum: * in brácchio virtútis tuæ dispersísti inimícos tuos.
}\switchcolumn\portugues{
Vós humilhastes o soberbo, como a um ferido: * com a força de vosso braço desprezastes os vossos inimigos.
}\switchcolumn*\latim{
Tui sunt cæli, et tua est terra, orbem terræ et plenitúdinem ejus Tu fundásti: * aquilónem, et mare Tu creásti.
}\switchcolumn\portugues{
Vossos são os céus e vossa é a terra, Vós fundastes o mundo e tudo o que ele contém: * Vós criastes o norte e o mar.
}\switchcolumn*\latim{
Thabor et Hermon in nómine tuo exsultábunt: * tuum brácchium cum poténtia.
}\switchcolumn\portugues{
O Tabor e o Hermon exultarão em vosso nome: * o vosso braço está cheio de poder.
}\switchcolumn*\latim{
Firmétur manus tua, et exaltétur déxtera tua: * justítia et judícium præparátio sedis tuæ.
}\switchcolumn\portugues{
Firmada seja a vossa mão e erga-se a vossa dextra: * justiça e julgamento são a base de vosso trono.
}\switchcolumn*\latim{
Misericórdia et véritas præcédent fáciem tuam: * beátus pópulus, qui scit jubilatiónem.
}\switchcolumn\portugues{
Misericórdia e verdade irão adiante de vossa face: * bem-aventurado o povo que se sabe alegrar.
}\switchcolumn*\latim{
Dómine, in lúmine vultus tui ambulábunt, et in nómine tuo exsultábunt tota die: * et in justítia tua exaltabúntur.
}\switchcolumn\portugues{
Ó Senhor, eles caminharão à luz de vosso rosto e em vosso nome se regozijarão todo o dia: * e pela vossa justiça serão exaltados.
}\switchcolumn*\latim{
Quóniam glória virtútis eórum Tu es: * et in beneplácito tuo exaltábitur cornu nostrum.
}\switchcolumn\portugues{
Porque Vós sois a glória da sua força: * e por vossa boa-vontade será exaltado o nosso poder.
}\switchcolumn*\latim{
Quia Dómini est assúmptio nostra, * et Sancti Israël, regis nostri.
}\switchcolumn\portugues{
Pois o Senhor tomou-nos por seus, * e o Santo de Israel é nosso rei.
}\switchcolumn*\latim{
Tunc locútus es in visióne sanctis tuis, et dixísti: * Pósui adjutórium in poténte: et exaltávi eléctum de plebe mea.
}\switchcolumn\portugues{
Então falastes numa visão aos vossos santos e dissestes: * prestei o meu socorro ao poderoso e exaltei aquele que escolhi do meu povo.
}\switchcolumn*\latim{
Invéni David, servum meum: * óleo sancto meo unxi eum.
}\switchcolumn\portugues{
Encontrei David, meu servo: * e com meu santo óleo o ungi.
}\switchcolumn*\latim{
Manus enim mea auxiliábitur ei: * et brácchium meum confortábit eum.
}\switchcolumn\portugues{
Minha mão assisti-lo-á efectivamente: * e o meu braço fortificá-lo-á.
}\switchcolumn*\latim{
Nihil profíciet inimícus in eo, * et fílius iniquitátis non appónet nocére ei.
}\switchcolumn\portugues{
O inimigo em nada prevalecerá contra ele, * e o filho da iniquidade não poderá ofendê-lo.
}\switchcolumn*\latim{
Et concídam a fácie ipsíus inimícos ejus: * et odiéntes eum in fugam convértam.
}\switchcolumn\portugues{
Exterminarei de diante dele os seus inimigos: * e porei em fuga os que o odeiam.
}\switchcolumn*\latim{
Et véritas mea, et misericórdia mea cum ipso: * et in nómine meo exaltábitur cornu ejus.
}\switchcolumn\portugues{
Minha verdade e a minha misericórdia serão com ele: * e no meu nome será exaltado o seu poder.
}\switchcolumn*\latim{
Et ponam in mari manum ejus: * et in flumínibus déxteram ejus.
}\switchcolumn\portugues{
Estenderei a sua mão sobre o mar: * e a sua dextra sobre os rios.
}\switchcolumn*\latim{
Ipse invocábit me: Pater meus es Tu: * Deus meus, et suscéptor salútis meæ.
}\switchcolumn\portugues{
Ele invocar-me-á, dizendo: Vós sois meu Pai: * meu Deus, e o suporte da minha salvação.
}\switchcolumn*\latim{
Et ego primogénitum ponam illum * excélsum præ régibus terræ.
}\switchcolumn\portugues{
Eu o estabelecerei por primogénito, * o mais elevado entre os reis da terra.
}\switchcolumn*\latim{
In ætérnum servábo illi misericórdiam meam: * et testaméntum meum fidéle ipsi.
}\switchcolumn\portugues{
Eternamente guardá-lo-á a minha misericórdia: * e a minha aliança com ele será estável.
}\switchcolumn*\latim{
Et ponam in sǽculum sǽculi semen ejus: * et thronum ejus sicut dies cæli.
}\switchcolumn\portugues{
Farei que sua descendência subsista por todos os séculos: * e que seu trono dure tanto como os dias do céu.
}\switchcolumn*\latim{
Si autem derelíquerint fílii ejus legem meam: * et in judíciis meis non ambuláverint:
}\switchcolumn\portugues{
Mas, se seus filhos abandonarem a minha lei: * e não andarem nos meus preceitos:
}\switchcolumn*\latim{
Si justítias meas profanáverint: * et mandáta mea non custodíerint:
}\switchcolumn\portugues{
Se violarem as minhas justiças: * e não guardarem os meus mandamentos:
}\switchcolumn*\latim{
Visitábo in virga iniquitátes eórum: * et in verbéribus peccáta eórum.
}\switchcolumn\portugues{
Visitarei com vara as suas maldades: * e com açoites os seus pecados.
}\switchcolumn*\latim{
Misericórdiam autem meam non dispérgam ab eo: * neque nocébo in veritáte mea:
}\switchcolumn\portugues{
Porém, não retirarei dele a minha misericórdia: * nem lhe faltarei à verdade:
}\switchcolumn*\latim{
Neque profanábo testaméntum meum: * et quæ procédunt de lábiis meis, non fáciam írrita.
}\switchcolumn\portugues{
Nem violarei a minha aliança: * nem farei vãs as promessas saídas dos meus lábios.
}\switchcolumn*\latim{
Semel jurávi in sancto meo: si David méntiar: * semen ejus in ætérnum manébit.
}\switchcolumn\portugues{
Jurei uma vez pela minha santidade, me nãontirei a David: * a sua descendência permanecerá eternamente.
}\switchcolumn*\latim{
Et thronus ejus sicut sol in conspéctu meo, * et sicut luna perfécta in ætérnum: et testis in cælo fidélis.
}\switchcolumn\portugues{
Seu trono será como o sol ante mim, * como a lua cheia para sempre e como testemunho fiel do céu.
}\switchcolumn*\latim{
Tu vero repulísti et despexísti: * distulísti Christum tuum.
}\switchcolumn\portugues{
Apesar disso Senhor, Vós rejeitastes e desprezastes: * repelistes a vosso Cristo.
}\switchcolumn*\latim{
Evertísti testaméntum servi tui: * profanásti in terra Sanctuárium ejus.
}\switchcolumn\portugues{
Anulastes a aliança feita com vosso servo: * lançastes por terra o seu santuário.
}\switchcolumn*\latim{
Destruxísti omnes sepes ejus: * posuísti firmaméntum ejus formídinem.
}\switchcolumn\portugues{
Destruístes todas suas sebes: * pusestes o medo nas suas fortalezas.
}\switchcolumn*\latim{
Diripuérunt eum omnes transeúntes viam: * factus est oppróbrium vicínis suis.
}\switchcolumn\portugues{
Saquearam-no todos os que passavam pelo caminho: * chegou a ser a desonra dos seus vizinhos.
}\switchcolumn*\latim{
Exaltásti déxteram depriméntium eum: * lætificásti omnes inimícos ejus.
}\switchcolumn\portugues{
Exaltastes a dextra dos que o humilhavam: * alegrastes todos seus inimigos.
}\switchcolumn*\latim{
Avertísti adjutórium gládii ejus: * et non es auxiliátus ei in bello.
}\switchcolumn\portugues{
Tirastes toda a força à sua espada: * e o não auxiliastes na guerra.
}\switchcolumn*\latim{
Destruxísti eum ab emundatióne: * et sedem ejus in terram collisísti.
}\switchcolumn\portugues{
Aniquilastes o seu esplendor: * e derrubastes por terra o seu trono.
}\switchcolumn*\latim{
Minorásti dies témporis ejus: * perfudísti eum confusióne.
}\switchcolumn\portugues{
Abreviastes os dias do seu tempo: * cobriste-lo de confusão.
}\switchcolumn*\latim{
Úsquequo, Dómine, avértis in finem: * exardéscet sicut ignis ira tua?
}\switchcolumn\portugues{
Até quando, Senhor, continuareis adverso até ao fim: * arderá como fogo a vossa ira?
}\switchcolumn*\latim{
Memoráre quæ mea substántia: * numquid enim vane constituísti omnes fílios hóminum?
}\switchcolumn\portugues{
Lembrai-Vos do que é a minha natureza: * porventura criastes em vão todos os filhos dos homens?
}\switchcolumn*\latim{
Quis est homo, qui vivet, et non vidébit mortem: * éruet ánimam suam de manu ínferi?
}\switchcolumn\portugues{
Que homem há, que viva sem jamais ver a morte: * que possa arrancar a sua alma do poder do inferno?
}\switchcolumn*\latim{
Ubi sunt misericórdiæ tuæ antíquæ, Dómine, * sicut jurásti David in veritáte tua?
}\switchcolumn\portugues{
Onde estão as vossas antigas misericórdias, ó Senhor, * as quais na vossa verdade jurastes a David?
}\switchcolumn*\latim{
Memor esto, Dómine, oppróbrii servórum tuórum * quod contínui in sinu meo multárum géntium.
}\switchcolumn\portugues{
Lembrai-Vos, ó Senhor, a desonra de vossos servos * que guardo no meu peito de gentes numerosas.
}\switchcolumn*\latim{
Quod exprobravérunt inimíci tui, Dómine, * quod exprobravérunt commutatiónem Christi tui.
}\switchcolumn\portugues{
Com que têm insultado os vossos inimigos, ó Senhor, * com que têm insultado a mudança de vosso Cristo.
}\switchcolumn*\latim{
Benedíctus Dóminus in ætérnum: * fiat, fiat.
}\switchcolumn\portugues{
Bendito seja o Senhor para sempre: * assim seja, assim seja.
}\end{paracol}


\subsectioninfo{Salmo 89}{Domine, refugium factus}\label{salmo89}
\begin{paracol}{2}\latim{
\rlettrine{D}{ómine,} refúgium factus es nobis: * a generatióne in generatiónem.
}\switchcolumn\portugues{
\rlettrine{S}{enhor} tendes sido o nosso refúgio: * de geração em geração.
}\switchcolumn*\latim{
Priúsquam montes fíerent, aut formarétur terra et orbis: * a sǽculo et usque in sǽculum Tu es, Deus.
}\switchcolumn\portugues{
Antes que os montes fossem feitos, ou que a terra e o mundo fossem formados: * Deus sois desde toda a eternidade e pelos séculos.
}\switchcolumn*\latim{
Ne avértas hóminem in humilitátem: * et dixísti: convertímini, fílii hóminum.
}\switchcolumn\portugues{
Não reduzais o homem ao abatimento: * e dissestes: convertei-vos, filhos dos homens.
}\switchcolumn*\latim{
Quóniam mille anni ante óculos tuos, * tamquam dies hestérna, quæ prætériit,
}\switchcolumn\portugues{
Porque mil anos, aos vossos olhos, * são como o dia de ontem, que passou,
}\switchcolumn*\latim{
Et custódia in nocte, * quæ pro níhilo habéntur, eórum anni erunt.
}\switchcolumn\portugues{
Como uma vigília da noite, * coisas que em nada se estimam, assim serão os seus anos.
}\switchcolumn*\latim{
Mane sicut herba tránseat, mane flóreat, et tránseat: * véspere décidat, indúret et aréscat.
}\switchcolumn\portugues{
De manhã levanta-se como a erva, pela manhã floresce e passa: * à tarde cai, endurece e seca.
}\switchcolumn*\latim{
Quia defécimus in ira tua, * et in furóre tuo turbáti sumus.
}\switchcolumn\portugues{
Pois desfalecemos na vossa ira, * e na vossa fúria somos turvados.
}\switchcolumn*\latim{
Posuísti iniquitátes nostras in conspéctu tuo: * sǽculum nostrum in illuminatióne vultus tui.
}\switchcolumn\portugues{
Pusestes as nossas maldades à vossa vista: * o nosso proceder à luz de vosso rosto.
}\switchcolumn*\latim{
Quóniam omnes dies nostri defecérunt: * et in ira tua defécimus.
}\switchcolumn\portugues{
Por isso todos nossos dias se desvaneceram: * e fomos consumidos pela vossa ira.
}\switchcolumn*\latim{
Anni nostri sicut aránea meditabúntur: * dies annórum nostrórum in ipsis, septuagínta anni.
}\switchcolumn\portugues{
Os nossos anos serão considerados como uma aranha: * os anos da nossa vida são em si setenta.
}\switchcolumn*\latim{
Si autem in potentátibus, octogínta anni: * et ámplius eórum, labor et dolor.
}\switchcolumn\portugues{
Nos mais robustos oitenta anos: * e o que passa destes mais não é que trabalho e dor.
}\switchcolumn*\latim{
Quóniam supervénit mansuetúdo: * et corripiémur.
}\switchcolumn\portugues{
Porque então sucede a fraqueza: * e nós somos arrebatados.
}\switchcolumn*\latim{
Quis novit potestátem iræ tuæ: * et præ timóre tuo iram tuam dinumeráre?
}\switchcolumn\portugues{
Quem poderá conhecer o poder de vossa ira: * e compreender quão terrível é a vossa indignação?
}\switchcolumn*\latim{
Déxteram tuam sic notam fac: * et erudítos corde in sapiéntia.
}\switchcolumn\portugues{
Ensinai-nos a conhecer a vossa dextra: * e instrui o nosso coração na sabedoria.
}\switchcolumn*\latim{
Convértere, Dómine, úsquequo? * Et deprecábilis esto super servos tuos.
}\switchcolumn\portugues{
Voltai-Vos, ó Senhor, até quando? * Sede compassivo para com vossos servos.
}\switchcolumn*\latim{
Repléti sumus mane misericórdia tua: * et exsultávimus, et delectáti sumus ómnibus diébus nostris.
}\switchcolumn\portugues{
Fomos cumulados de vossa misericórdia desde a manhã: * e exultamos de alegria e felicidade todos nossos dias.
}\switchcolumn*\latim{
Lætáti sumus pro diébus, quibus nos humiliásti: * annis, quibus vídimus mala.
}\switchcolumn\portugues{
Alegramo-nos pelos dias em que nos humilhastes: * pelos anos em que males vimos.
}\switchcolumn*\latim{
Réspice in servos tuos, et in ópera tua: * et dírige fílios eórum.
}\switchcolumn\portugues{
Ponde os olhos nos vossos servos e nas vossas obras: * e guiai os seus filhos.
}\switchcolumn*\latim{
Et sit splendor Dómini, Dei nostri, super nos, et ópera mánuum nostrárum dírige super nos: * et opus mánuum nostrárum dírige.
}\switchcolumn\portugues{
Brilhe sobre nós a luz do Senhor nosso Deus, dirigi em nós as obras de nossas mãos: * sim, dirigi a obra de nossas mãos.
}\end{paracol}


\subsectioninfo{Salmo 90}{Qui habitat in adjutorio Altissimi}\label{salmo90}
\begin{paracol}{2}\latim{
\qlettrine{Q}{ui} hábitat in adjutório Altíssimi, * in protectióne Dei cæli commorábitur.
}\switchcolumn\portugues{
\rlettrine{O}{} que habita à sombra do Altíssimo, * descansará na protecção do Deus do céu.
}\switchcolumn*\latim{
Dicet Dómino: suscéptor meus es Tu, et refúgium meum: * Deus meus sperábo in eum.
}\switchcolumn\portugues{
Dirá ao Senhor: Vós sois o meu defensor e o meu refúgio: * o meu Deus, em quem esperarei.
}\switchcolumn*\latim{
Quóniam ipse liberávit me de láqueo venántium, * et a verbo áspero.
}\switchcolumn\portugues{
Porque Ele me livrou do laço dos caçadores, * e da palavra áspera.
}\switchcolumn*\latim{
Scápulis suis obumbrábit tibi: * et sub pennis ejus sperábis.
}\switchcolumn\portugues{
Com seus ombros fazer-te-á sombra: * e debaixo das suas asas esperarás.
}\switchcolumn*\latim{
Scuto circúmdabit te véritas ejus: * non timébis a timóre noctúrno,
}\switchcolumn\portugues{
Cercar-te-á como um escudo a sua verdade: * assombros nocturnos não temerás,
}\switchcolumn*\latim{
A sagítta volánte in die, a negótio perambulánte in ténebris: * ab incúrsu, et dæmónio meridiáno.
}\switchcolumn\portugues{
Da seta que voa de dia, nem da trama que ambula nas trevas: * de assaltos, nem do demónio do meio-dia.
}\switchcolumn*\latim{
Cadent a látere tuo mille, et decem míllia a dextris tuis: * ad te autem non appropinquábit.
}\switchcolumn\portugues{
Cairão mil a teu lado e dez mil à tua direita: * mas se não aproximará de ti.
}\switchcolumn*\latim{
Verúmtamen óculis tuis considerábis: * et retributiónem peccatórum vidébis.
}\switchcolumn\portugues{
Com teus olhos então contemplarás: * e verás o castigo dos pecadores.
}\switchcolumn*\latim{
Quóniam Tu es, Dómine, spes mea: * Altíssimum posuísti refúgium tuum.
}\switchcolumn\portugues{
Porque Vós sois, ó Senhor, a minha esperança: * o Altíssimo tomaste por teu refúgio.
}\switchcolumn*\latim{
Non accédet ad te malum: * et flagéllum non appropinquábit tabernáculo tuo.
}\switchcolumn\portugues{
O mal não virá sobre ti: * e o flagelo se não aproximará de tua tenda.
}\switchcolumn*\latim{
Quóniam Ángelis suis mandávit de te: * ut custódiant te in ómnibus viis tuis.
}\switchcolumn\portugues{
Porque mandou os seus anjos a ti: * para que te velem em todos teus caminhos.
}\switchcolumn*\latim{
In mánibus portábunt te: * ne forte offéndas ad lápidem pedem tuum.
}\switchcolumn\portugues{
Eles levar-te-ão nas suas mãos: * para que não tropece o teu pé em pedra alguma.
}\switchcolumn*\latim{
Super áspidem, et basilíscum ambulábis: * et conculcábis leónem et dracónem.
}\switchcolumn\portugues{
Sobre a víbora e o basilisco andarás: * e calcarás o leão e o dragão.
}\switchcolumn*\latim{
Quóniam in me sperávit, liberábo eum: * prótegam eum, quóniam cognóvit nomen meum.
}\switchcolumn\portugues{
Porque esperou em mim, livrá-lo-ei: * protegê-lo-ei, porque conheceu o meu nome.
}\switchcolumn*\latim{
Clamábit ad me, et ego exáudiam eum: * cum ipso sum in tribulatióne: erípiam eum et glorificábo eum.
}\switchcolumn\portugues{
A mim clamará e eu o ouvirei: * com ele estou na tribulação, livrá-lo-ei e glorificá-lo-ei.
}\switchcolumn*\latim{
Longitúdine diérum replébo eum: * et osténdam illi salutáre meum.
}\switchcolumn\portugues{
Enchê-lo-ei de longos dias: * e mostrar-lhe-ei a minha salvação.
}\end{paracol}


\subsectioninfo{Salmo 91}{Bonum est confiteri Domino}\label{salmo91}
\begin{paracol}{2}\latim{
\rlettrine{B}{onum} est confitéri Dómino: * et psállere nómini tuo, Altíssime.
}\switchcolumn\portugues{
\rlettrine{B}{om} é louvar ao Senhor: * e cantar ao vosso nome, ó Altíssimo.
}\switchcolumn*\latim{
Ad annuntiándum mane misericórdiam tuam: * et veritátem tuam per noctem.
}\switchcolumn\portugues{
Para publicar pela manhã a vossa misericórdia: * e durante a noite a vossa verdade.
}\switchcolumn*\latim{
In decachórdo, psaltério: * cum cántico, in cíthara.
}\switchcolumn\portugues{
Com o saltério de dez cordas: * com cântico ao som da cítara.
}\switchcolumn*\latim{
Quia delectásti me, Dómine, in factúra tua: * et in opéribus mánuum tuárum exsultábo.
}\switchcolumn\portugues{
Pois me alegrastes, ó Senhor, com vossas obras: * e exulto com as obras de vossas mãos.
}\switchcolumn*\latim{
Quam magnificáta sunt ópera tua, Dómine! * nimis profúndæ factæ sunt cogitatiónes tuæ.
}\switchcolumn\portugues{
Quão magníficas são, ó Senhor, as vossas obras! * Profundíssimos são os vossos pensamentos.
}\switchcolumn*\latim{
Vir insípiens non cognóscet: * et stultus non intélleget hæc.
}\switchcolumn\portugues{
O varão parvo não conhecerá: * e o ignorante não compreenderá estas cousas.
}\switchcolumn*\latim{
Cum exórti fúerint peccatóres sicut fænum: * et apparúerint omnes, qui operántur iniquitátem:
}\switchcolumn\portugues{
Quando os pecadores crescerem como a erva: * e aparecerem todos os que cometem a iniquidade:
}\switchcolumn*\latim{
Ut intéreant in sǽculum sǽculi: * Tu autem Altíssimus in ætérnum, Dómine.
}\switchcolumn\portugues{
Imediatamente perecerão para sempre: * mas Vós, ó Senhor, sois eternamente o Altíssimo.
}\switchcolumn*\latim{
Quóniam ecce inimíci tui, Dómine, quóniam ecce inimíci tui períbunt: * et dispergéntur omnes, qui operántur iniquitátem.
}\switchcolumn\portugues{
Porque eis que os vossos inimigos, Senhor, eis que os vossos inimigos perecerão: * e serão dissipados todos os que praticam a iniquidade.
}\switchcolumn*\latim{
Et exaltábitur sicut unicórnis cornu meum: * et senéctus mea in misericórdia úberi.
}\switchcolumn\portugues{
Será exaltada a minha força como a do unicórnio: * e a minha velhice com a abundância de vossa misericórdia.
}\switchcolumn*\latim{
Et despéxit óculus meus inimícos meos: * et in insurgéntibus in me malignántibus áudiet auris mea.
}\switchcolumn\portugues{
Meus olhos olharão com desprezo para os meus inimigos: * e os meus ouvidos ouvirão falar dos revoltosos que se levantam contra mim.
}\switchcolumn*\latim{
Justus, ut palma florébit: * sicut cedrus Líbani multiplicábitur.
}\switchcolumn\portugues{
O justo florescerá como a palmeira: * e como o cedro do Líbano multiplicar-se-á.
}\switchcolumn*\latim{
Plantáti in domo Dómini, * in átriis domus Dei nostri florébunt.
}\switchcolumn\portugues{
Plantados na casa do Senhor, * florescerão nos átrios da casa do nosso Deus.
}\switchcolumn*\latim{
Adhuc multiplicabúntur in senécta úberi: * et bene patiéntes erunt, ut annúntient:
}\switchcolumn\portugues{
Eles se multiplicarão em uma velhice fecunda: * e estarão cheios de vigor, para anunciar:
}\switchcolumn*\latim{
Quóniam rectus Dóminus, Deus noster: * et non est iníquitas in eo.
}\switchcolumn\portugues{
Que o Senhor nosso Deus é recto: * e que não há injustiça n’Ele.
}\end{paracol}


\subsectioninfo{Salmo 92}{Dominus regnavit}\label{salmo92}
\begin{paracol}{2}\latim{
\rlettrine{D}{óminus} regnávit, decórem indútus est: * indútus est Dóminus fortitúdinem, et præcínxit se.
}\switchcolumn\portugues{
\rlettrine{O}{} Senhor reinou e vestiu-se de magnificência: * vestiu-se o Senhor de fortaleza e cingiu-se dela.
}\switchcolumn*\latim{
Étenim firmávit orbem terræ, * qui non commovébitur.
}\switchcolumn\portugues{
Pois firmou a órbita da terra, * que não será abalada.
}\switchcolumn*\latim{
Paráta sedes tua ex tunc: * a sǽculo Tu es.
}\switchcolumn\portugues{
De então ficou vosso trono preparado: * Vós sois desde a eternidade.
}\switchcolumn*\latim{
Elevavérunt flúmina, Dómine: * elevavérunt flúmina vocem suam.
}\switchcolumn\portugues{
Os rios levantaram, ó Senhor: * os rios levantaram a sua voz.
}\switchcolumn*\latim{
Elevavérunt flúmina fluctus suos, * a vócibus aquárum multárum.
}\switchcolumn\portugues{
Levantaram os rios o som das suas ondas, * com estrondo das muitas águas.
}\switchcolumn*\latim{
Mirábiles elatiónes maris: * mirábilis in altis Dóminus.
}\switchcolumn\portugues{
Maravilhosas as elevações do mar: * admirável o Senhor nas alturas.
}\switchcolumn*\latim{
Testimónia tua credibília facta sunt nimis: * domum tuam decet sanctitúdo, Dómine, in longitúdinem diérum.
}\switchcolumn\portugues{
Vossos testemunhos são digníssimos de fé: * a santidade convém à vossa casa, ó Senhor, na longitude dos dias.
}\end{paracol}


\subsectioninfo{Salmo 93}{Deus ultionum Dominus}\label{salmo93}
\begin{paracol}{2}\latim{
\rlettrine{D}{eus} ultiónum Dóminus: * Deus ultiónum líbere egit.
}\switchcolumn\portugues{
\rlettrine{D}{eus} da vingança é o Senhor: * agiu o Deus da vingança livremente.
}\switchcolumn*\latim{
Exaltáre, qui júdicas terram: * redde retributiónem supérbis.
}\switchcolumn\portugues{
Exaltai-Vos Vós que julgais a terra: * dai aos soberbos o que merecem.
}\switchcolumn*\latim{
Úsquequo peccatóres, Dómine, * úsquequo peccatóres gloriabúntur:
}\switchcolumn\portugues{
Até quando é que os pecadores, ó Senhor, * até quando é que os pecadores triunfarão:
}\switchcolumn*\latim{
Effabúntur, et loquéntur iniquitátem: * loquéntur omnes, qui operántur injustítiam?
}\switchcolumn\portugues{
Pronunciarão e falarão iniquidade: * e levantarão a voz todos os que praticam a injustiça?
}\switchcolumn*\latim{
Pópulum tuum, Dómine, humiliavérunt: * et hereditátem tuam vexavérunt.
}\switchcolumn\portugues{
Humilharam, ó Senhor, o vosso povo: * e oprimiram a vossa herança.
}\switchcolumn*\latim{
Víduam, et ádvenam interfecérunt: * et pupíllos occidérunt.
}\switchcolumn\portugues{
Mataram a viúva e o estrangeiro: * e tiraram a vida aos órfãos.
}\switchcolumn*\latim{
Et dixérunt: non vidébit Dóminus, * nec intélleget Deus Jacob.
}\switchcolumn\portugues{
Disseram: não verá o Senhor, * nem saberá o Deus de Jacob.
}\switchcolumn*\latim{
Intellégite, insipiéntes in pópulo: * et stulti, aliquándo sápite.
}\switchcolumn\portugues{
Reflecti, ó insensatos do povo: * e, ó ignorantes, sede finalmente prudentes.
}\switchcolumn*\latim{
Qui plantávit aurem, non áudiet? * Aut qui finxit óculum, non consíderat?
}\switchcolumn\portugues{
Porventura Aquele que criou o ouvido, não ouvirá? * Ou O que formou os olhos, não verá?
}\switchcolumn*\latim{
Qui córripit gentes, non árguet: * qui docet hóminem sciéntiam?
}\switchcolumn\portugues{
O que castiga as gentes, não repreenderá: * Ele que ensina ao homem a ciência?
}\switchcolumn*\latim{
Dóminus scit cogitatiónes hóminum, * quóniam vanæ sunt.
}\switchcolumn\portugues{
O Senhor conhece os pensamentos dos homens, * que são vãos.
}\switchcolumn*\latim{
Beátus homo, quem Tu erudíeris, Dómine: * et de lege tua docúeris eum,
}\switchcolumn\portugues{
Bem-aventurado o homem a quem Vós instruirdes, ó Senhor: * e amestrardes na vossa lei,
}\switchcolumn*\latim{
Ut mítiges ei a diébus malis: * donec fodiátur peccatóri fóvea.
}\switchcolumn\portugues{
A fim de lhe suavizar os dias maus: * até que se abra a cova para o pecador.
}\switchcolumn*\latim{
Quia non repéllet Dóminus plebem suam: * et hereditátem suam non derelínquet.
}\switchcolumn\portugues{
Pois o Senhor não repelirá o seu povo: * nem abandonará a sua herança.
}\switchcolumn*\latim{
Quoadúsque justítia convertátur in judícium: * et qui juxta illam omnes qui recto sunt corde.
}\switchcolumn\portugues{
Até que a justiça faça brilhar o seu julgamento: * e estejam perto dela todos os que são rectos de coração.
}\switchcolumn*\latim{
Quis consúrget mihi advérsus malignántes? * Aut quis stabit mecum advérsus operántes iniquitátem?
}\switchcolumn\portugues{
Quem contra os maus se levantará por mim? * Ou quem contra os que praticam a iniquidade estará comigo?
}\switchcolumn*\latim{
Nisi quia Dóminus adjúvit me: * paulo minus habitásset in inférno ánima mea.
}\switchcolumn\portugues{
Se o Senhor me não tivesse socorrido: * por pouco que seria o inferno a minha morada.
}\switchcolumn*\latim{
Si dicébam: motus est pes meus: * misericórdia tua, Dómine, adjuvábat me.
}\switchcolumn\portugues{
Se dizia: meu pé está vacilante: * a vossa misericórdia, ó Senhor, me sustentava.
}\switchcolumn*\latim{
Secúndum multitúdinem dolórum meórum in corde meo: * consolatiónes tuæ lætificavérunt ánimam meam.
}\switchcolumn\portugues{
Segundo as muitas dores que atormentaram o meu coração: * as vossas consolações alegraram a minha alma.
}\switchcolumn*\latim{
Numquid adhǽret tibi sedes iniquitátis: * qui fingis labórem in præcépto?
}\switchcolumn\portugues{
É porventura a cadeira da iniquidade vossa aliada: * que inventa penosos mandamentos?
}\switchcolumn*\latim{
Captábunt in ánimam justi: * et sánguinem innocéntem condemnábunt.
}\switchcolumn\portugues{
A alma do justo perseguirão: * e condenarão o sangue inocente.
}\switchcolumn*\latim{
Et factus est mihi Dóminus in refúgium: * et Deus meus in adjutórium spei meæ.
}\switchcolumn\portugues{
O Senhor é o meu refúgio: * e o meu Deus, o apoio da minha esperança.
}\switchcolumn*\latim{
Et reddet illis iniquitátem ipsórum: et in malítia eórum dispérdet eos: * dispérdet illos Dóminus, Deus noster.
}\switchcolumn\portugues{
Fará cair sobre eles a sua iniquidade e na sua malícia os destruirá: * destruí-los-á o Senhor nosso Deus.
}\end{paracol}


\subsectioninfo{Salmo 94}{Venite, exsultemus Domino}\label{salmo94}
\begin{paracol}{2}\latim{
\rlettrine{V}{eníte,} exsultémus Dómino: * jubilémus Deo salutári nostro:
}\switchcolumn\portugues{
\rlettrine{V}{inde,} exultemos o Senhor: * cantemos alegres a de Deus nosso salvador:
}\switchcolumn*\latim{
Præoccupémus fáciem ejus in confessióne: * et in psalmis jubilémus ei.
}\switchcolumn\portugues{
Apresentemo-nos diante d’Ele em acção de graças: * e celebremo-l’O com salmos.
}\switchcolumn*\latim{
Quóniam Deus magnus Dóminus: * et Rex magnus super omnes deos.
}\switchcolumn\portugues{
Porque o Senhor é o grande Deus: * e o Rei grande sobre todos os deuses.
}\switchcolumn*\latim{
Quia in manu ejus sunt omnes fines terræ: * et altitúdines móntium ipsíus sunt.
}\switchcolumn\portugues{
Pois na sua mão estão todos os confins da terra: * e são suas as alturas dos montes.
}\switchcolumn*\latim{
Quóniam ipsíus est mare, et ipse fecit illud: * et siccam manus ejus formavérunt.
}\switchcolumn\portugues{
Seu é o mar e Ele o fez: * e as suas mãos formaram a terra árida.
}\switchcolumn*\latim{
Veníte, adorémus, et procidámus, * et plorémus ante Dóminum qui fecit nos.
}\switchcolumn\portugues{
Vinde, adoremos e prostremo-nos, * e choremos diante do Senhor que nos criou.
}\switchcolumn*\latim{
Quia ipse est Dóminus Deus noster, * et nos pópulus páscuæ ejus, et oves manus ejus.
}\switchcolumn\portugues{
Pois Ele é o Senhor nosso Deus, * e nós somos o povo do seu pasto e as ovelhas da sua manada.
}\switchcolumn*\latim{
Hódie si vocem ejus audiéritis, * nolíte obduráre corda vestra:
}\switchcolumn\portugues{
Se hoje ouvirdes a sua voz, * não endureceis os vossos corações:
}\switchcolumn*\latim{
Sicut in irritatióne secúndum diem tentatiónis in desérto: * ubi tentavérunt me patres vestri, probavérunt me, et vidérunt ópera mea.
}\switchcolumn\portugues{
Como quando me provocaram à ira, no dia da tentação no deserto: * onde vossos pais me tentaram, me testaram e viram as minhas obras.
}\switchcolumn*\latim{
Quadragínta annis offénsus fui generatióni illi, * et dixi: semper hi errant corde.
}\switchcolumn\portugues{
Quarenta anos estive irritado contra esta geração, * e disse: é um povo de coração errante.
}\switchcolumn*\latim{
Et isti non cognovérunt vias meas, ut jurávi in ira mea: * Si introíbunt in réquiem meam.
}\switchcolumn\portugues{
Eles não conheceram os meus caminhos, pelo que jurei na minha ira: * no meu repouso não entrarão.
}\end{paracol}


\subsectioninfo{Salmo 95}{Cantate Domino}\label{salmo95}
\begin{paracol}{2}\latim{
\rlettrine{C}{antáte} Dómino cánticum novum: * cantáte Dómino, omnis terra.
}\switchcolumn\portugues{
\rlettrine{C}{antai} ao Senhor um cântico novo: * cantai ao Senhor, toda a terra.
}\switchcolumn*\latim{
Cantáte Dómino, et benedícite nómini ejus: * annuntiáte de die in diem salutáre ejus.
}\switchcolumn\portugues{
Cantai ao Senhor e bendizei o seu nome: * anunciai dia a dia a sua salvação.
}\switchcolumn*\latim{
Annuntiáte inter gentes glóriam ejus, * in ómnibus pópulis mirabília ejus.
}\switchcolumn\portugues{
Anunciai entre as gentes a sua glória, * entre todos os povos suas maravilhas.
}\switchcolumn*\latim{
Quóniam magnus Dóminus, et laudábilis nimis: * terríbilis est super omnes deos.
}\switchcolumn\portugues{
Porque o Senhor é grande e digníssimo de ser louvado: * mais terrível que todos os deuses.
}\switchcolumn*\latim{
Quóniam omnes dii géntium dæmónia: * Dóminus autem cælos fecit.
}\switchcolumn\portugues{
Porque todos os deuses das gentes são demónios: * porém, o Senhor fez os céus.
}\switchcolumn*\latim{
Conféssio, et pulchritúdo in conspéctu ejus: * sanctimónia et magnificéntia in sanctificatióne ejus.
}\switchcolumn\portugues{
O louvor e o esplendor estão diante d’Ele: * a santidade e a grandeza no seu santuário.
}\switchcolumn*\latim{
Afférte Dómino, pátriæ géntium, afférte Dómino glóriam et honórem: * afférte Dómino glóriam nómini ejus.
}\switchcolumn\portugues{
Dai ao Senhor, ó famílias das gentes, dai ao Senhor glória e honra: * dai ao Senhor a glória devida a seu nome.
}\switchcolumn*\latim{
Tóllite hóstias, et introíte in átria ejus: * adoráte Dóminum in átrio sancto ejus.
}\switchcolumn\portugues{
Elevai-Lhe sacrifícios e entrai nos seus átrios: * adorai o Senhor no átrio do seu santuário.
}\switchcolumn*\latim{
Commoveátur a fácie ejus univérsa terra: * dícite in géntibus quia Dóminus regnávit.
}\switchcolumn\portugues{
Trema toda a terra na sua presença: * dizei entre as gentes que é o Senhor quem reina.
}\switchcolumn*\latim{
Étenim corréxit orbem terræ qui non commovébitur: * judicábit pópulos in æquitáte.
}\switchcolumn\portugues{
Pois estabeleceu toda a terra, que não será abalada: * Ele julgará os povos com equidade.
}\switchcolumn*\latim{
Læténtur cæli, et exsúltet terra: commoveátur mare, et plenitúdo ejus: * gaudébunt campi, et ómnia quæ in eis sunt.
}\switchcolumn\portugues{
Alegrem-se os céus e exulte-se a terra, comova-se o mar e o que ele contém: * alegrar-se-ão os campos e tudo que neles há.
}\switchcolumn*\latim{
Tunc exsultábunt ómnia ligna silvárum a fácie Dómini, quia venit: * quóniam venit judicáre terram.
}\switchcolumn\portugues{
Então exultar-se-ão todas as árvores dos bosques ante o Senhor, porque vem: * porque vem julgar a terra.
}\switchcolumn*\latim{
Judicábit orbem terræ in æquitáte, * et pópulos in veritáte sua.
}\switchcolumn\portugues{
Ele julgará toda a terra com equidade, * e os povos segundo a sua verdade.
}\end{paracol}


\subsectioninfo{Salmo 96}{Dominus regnavit: exsultet terra}\label{salmo96}
\begin{paracol}{2}\latim{
\rlettrine{D}{óminus} regnávit, exsúltet terra: * læténtur ínsulæ multæ.
}\switchcolumn\portugues{
\rlettrine{O}{} Senhor é rei, exulte-se a terra: * alegrem-se as muitas ilhas.
}\switchcolumn*\latim{
Nubes, et calígo in circúitu ejus: * justítia, et judícium corréctio sedis ejus.
}\switchcolumn\portugues{
As nuvens e a escuridão estão em redor d’Ele: * a justiça e a equidade são a base do seu trono.
}\switchcolumn*\latim{
Ignis ante ípsum præcédet, * et inflammábit in circúitu inimícos ejus.
}\switchcolumn\portugues{
O fogo irá adiante d’Ele, * e abrasará em redor dos seus inimigos.
}\switchcolumn*\latim{
Illuxérunt fúlgura ejus orbi terræ: * vidit, et commóta est terra.
}\switchcolumn\portugues{
Seus relâmpagos iluminaram todo o mundo: * viu-os a terra e tremeu.
}\switchcolumn*\latim{
Montes, sicut cera fluxérunt a fácie Dómini: * a fácie Dómini omnis terra.
}\switchcolumn\portugues{
Os montes fundiram-se como cera ante o Senhor: * ante o Senhor de toda a terra.
}\switchcolumn*\latim{
Annuntiavérunt cæli justítiam ejus: * et vidérunt omnes pópuli glóriam ejus.
}\switchcolumn\portugues{
Os céus anunciaram a sua justiça: * e todos os povos viram a sua glória.
}\switchcolumn*\latim{
Confundántur omnes, qui adórant sculptília: * et qui gloriántur in simulácris suis.
}\switchcolumn\portugues{
Confundidos sejam todos os que adoram ídolos: * e os que se vangloriam nos seus simulacros.
}\switchcolumn*\latim{
Adoráte eum, omnes Ángeli ejus: * audívit, et lætáta est Sion.
}\switchcolumn\portugues{
Adorai o Senhor vós todos, ó seus anjos: * Sião ouviu-O e se alegrou.
}\switchcolumn*\latim{
Et exsultavérunt fíliæ Judæ, * propter judícia tua, Dómine:
}\switchcolumn\portugues{
As filhas de Judá exultaram-se, * por causa de vossos juízos, ó Senhor:
}\switchcolumn*\latim{
Quóniam Tu Dóminus Altíssimus super omnem terram: * nimis exaltátus es super omnes deos.
}\switchcolumn\portugues{
Porque Vós sois o Senhor altíssimo sobre toda a terra: * exaltadíssimo sois sobre todos os deuses.
}\switchcolumn*\latim{
Qui dilígitis Dóminum, odíte malum: * custódit Dóminus ánimas sanctórum suórum, de manu peccatóris liberábit eos.
}\switchcolumn\portugues{
Vós que amais o Senhor, odiai o mal: * o Senhor guarda as almas dos seus santos, livrá-los-á da mão do pecador.
}\switchcolumn*\latim{
Lux orta est justo, * et rectis corde lætítia.
}\switchcolumn\portugues{
Nasceu a luz para os justos, * e a alegria para os rectos de coração.
}\switchcolumn*\latim{
Lætámini, justi, in Dómino: * et confitémini memóriæ sanctificatiónis ejus.
}\switchcolumn\portugues{
Alegrai-vos, ó justos, no Senhor: * e celebrai a memória da sua santidade.
}\end{paracol}


\subsectioninfo{Salmo 97}{Cantate Domino canticum novum}\label{salmo97}
\begin{paracol}{2}\latim{
\rlettrine{C}{antáte} Dómino cánticum novum: * quia mirabília fecit.
}\switchcolumn\portugues{
\rlettrine{C}{antai} ao Senhor um cântico novo: * pois operou maravilhas.
}\switchcolumn*\latim{
Salvávit sibi déxtera ejus: * et brácchium sanctum ejus.
}\switchcolumn\portugues{
Sua dextra fê-l’O triunfar: * e o seu santo braço.
}\switchcolumn*\latim{
Notum fecit Dóminus salutáre suum: * in conspéctu géntium revelávit justítiam suam.
}\switchcolumn\portugues{
O Senhor manifestou a sua salvação: * revelou a sua justiça aos olhos das gentes.
}\switchcolumn*\latim{
Recordátus est misericórdiæ suæ, * et veritátis suæ dómui Israël.
}\switchcolumn\portugues{
Lembrou-se da sua misericórdia, * e da sua verdade para a casa de Israel.
}\switchcolumn*\latim{
Vidérunt omnes términi terræ * salutáre Dei nostri.
}\switchcolumn\portugues{
Todos os confins da terra viram * a salvação do nosso Deus.
}\switchcolumn*\latim{
Jubiláte Deo, omnis terra: * cantáte, et exsultáte, et psállite.
}\switchcolumn\portugues{
Aclamai a Deus, toda a terra: * cantai, exultai e salmodiai.
}\switchcolumn*\latim{
Psállite Dómino in cíthara, in cíthara et voce psalmi: * in tubis ductílibus, et voce tubæ córneæ.
}\switchcolumn\portugues{
Cantai ao Senhor com a cítara, com a cítara e com voz de salmo: * com trombetas de metal e som de corneta.
}\switchcolumn*\latim{
Jubiláte in conspéctu regis Dómini: * moveátur mare, et plenitúdo ejus: orbis terrárum, et qui hábitant in eo.
}\switchcolumn\portugues{
Jubilai-vos na presença do rei Senhor: * mova-se o mar e quanto nele há, toda a terra e os que a habitam.
}\switchcolumn*\latim{
Flúmina plaudent manu, simul montes exsultábunt a conspéctu Dómini: * quóniam venit judicáre terram.
}\switchcolumn\portugues{
Os rios baterão palmas, ao mesmo tempo os montes alegrar-se-ão à vista do Senhor: * porque vem julgar a terra.
}\switchcolumn*\latim{
Judicábit orbem terrárum in justítia, * et pópulos in æquitáte.
}\switchcolumn\portugues{
Julgará toda a terra com justiça, * e os povos com equidade.
}\end{paracol}


\subsectioninfo{Salmo 98}{Dominus regnavit: irascantur populi}\label{salmo98}
\begin{paracol}{2}\latim{
\rlettrine{D}{óminus} regnávit, irascántur pópuli: * qui sedet super Chérubim, moveátur terra.
}\switchcolumn\portugues{
\rlettrine{O}{} Senhor reinou, irritem-se os povos: * reina O que está sentado sobre Querubins, agite-se a terra.
}\switchcolumn*\latim{
Dóminus in Sion magnus: * et excélsus super omnes pópulos.
}\switchcolumn\portugues{
O Senhor é grande em Sião: * e está elevado sobre todos os povos.
}\switchcolumn*\latim{
Confiteántur nómini tuo magno: quóniam terríbile, et sanctum est: * et honor regis judícium díligit.
}\switchcolumn\portugues{
Dêem glória ao vosso grande nome, porque é terrível e santo: * e a honra do rei está em amar a justiça.
}\switchcolumn*\latim{
Tu parásti directiónes: * judícium et justítiam in Jacob Tu fecísti.
}\switchcolumn\portugues{
Vós preparastes direcções: * Vós exercestes o julgamento e a justiça em Jacob.
}\switchcolumn*\latim{
Exaltáte Dóminum, Deum nostrum, et adoráte scabéllum pedum ejus: * quóniam sanctum est.
}\switchcolumn\portugues{
Exaltai o Senhor nosso Deus e adorai o escabelo de seus pés: * pois santo é.
}\switchcolumn*\latim{
Móyses et Aaron in sacerdótibus ejus: * et Sámuel inter eos, qui ínvocant nomen ejus:
}\switchcolumn\portugues{
Moisés e Arão estavam entre os seus sacerdotes: * e Samuel entre aqueles que invocam o seu nome:
}\switchcolumn*\latim{
Invocábant Dóminum, et ipse exaudiébat eos: * in colúmna nubis loquebátur ad eos.
}\switchcolumn\portugues{
Invocavam o Senhor e Ele os atendia: * falava-lhes na coluna de nuvem.
}\switchcolumn*\latim{
Custodiébant testimónia ejus, * et præcéptum quod dedit illis.
}\switchcolumn\portugues{
Guardavam os seus mandamentos, * e o preceito que lhes tinha dado.
}\switchcolumn*\latim{
Dómine, Deus noster, Tu exaudiébas eos: * Deus, Tu propítius fuísti eis, et ulcíscens in omnes adinventiónes eórum.
}\switchcolumn\portugues{
Senhor nosso Deus, Vós os ouvíeis: * ó Deus, Vós lhes fostes propício, até em punir todas suas maquinações.
}\switchcolumn*\latim{
Exaltáte Dóminum, Deum nostrum, et adoráte in monte sancto ejus: * quóniam sanctus Dóminus, Deus noster.
}\switchcolumn\portugues{
Exaltai o Senhor nosso Deus e adorai-O sobre o seu santo monte: * pois santo é o Senhor nosso Deus.
}\end{paracol}


\subsectioninfo{Salmo 99}{Jubilate Deo, omnis terra}\label{salmo99}
\begin{paracol}{2}\latim{
\qlettrine{J}{ubiláte} Deo, omnis terra: * servíte Dómino in lætítia.
}\switchcolumn\portugues{
\rlettrine{A}{clamai} a Deus, todos os povos da terra: * servi o Senhor com alegria.
}\switchcolumn*\latim{
Introíte in conspéctu ejus, * in exsultatióne.
}\switchcolumn\portugues{
Vinde à sua presença * em grande exaltação.
}\switchcolumn*\latim{
Scitóte quóniam Dóminus ipse est Deus: * ipse fecit nos, et non ipsi nos.
}\switchcolumn\portugues{
Sabei que o Senhor é Deus: * nos fez Ele e não nós a nós mesmos.
}\switchcolumn*\latim{
Pópulus ejus, et oves páscuæ ejus: * introíte portas ejus in confessióne, átria ejus in hymnis: confitémini illi.
}\switchcolumn\portugues{
Nós somos o seu povo e as ovelhas do seu pasto: * entrai nos seus portões com louvor, nos seus átrios com hinos: glorificai-O.
}\switchcolumn*\latim{
Laudáte nomen ejus: quóniam suávis est Dóminus, in ætérnum misericórdia ejus, * et usque in generatiónem et generatiónem véritas ejus.
}\switchcolumn\portugues{
Louvai o seu nome: porque o Senhor é suave, a sua misericórdia é eterna: * e a sua verdade permanece de geração em geração.
}\end{paracol}


\subsectioninfo{Salmo 100}{Misericordiam et judicium}\label{salmo100}
\begin{paracol}{2}\latim{
\rlettrine{M}{isericórdiam} et judícium * cantábo tibi, Dómine:
}\switchcolumn\portugues{
\rlettrine{M}{isericórdia} e justiça * Vos cantarei, ó Senhor:
}\switchcolumn*\latim{
Psallam, et intéllegam in via immaculáta, * quando vénies ad me.
}\switchcolumn\portugues{
Cantarei e procurarei conhecer o caminho da perfeição, * quando vierdes a mim.
}\switchcolumn*\latim{
Perambulábam in innocéntia cordis mei, * in médio domus meæ.
}\switchcolumn\portugues{
Caminhava na inocência do meu coração, * no meio da minha casa.
}\switchcolumn*\latim{
Non proponébam ante óculos meos rem injústam: * faciéntes prævaricatiónes odívi.
}\switchcolumn\portugues{
Não punha ante meus olhos cousa injusta: * aborrecia os que cometiam transgressões.
}\switchcolumn*\latim{
Non adhǽsit mihi cor pravum: * declinántem a me malígnum non cognoscébam.
}\switchcolumn\portugues{
Não se unia a mim coração depravado: * o mau afastava-se de mim e eu o não conhecia.
}\switchcolumn*\latim{
Detrahéntem secréto próximo suo, * hunc persequébar.
}\switchcolumn\portugues{
Ao que secretamente detraia o seu próximo, * eu o perseguia.
}\switchcolumn*\latim{
Supérbo óculo, et insatiábili corde, * cum hoc non edébam.
}\switchcolumn\portugues{
Com homem de olhos soberbos e de coração insaciável, * com esse não comia.
}\switchcolumn*\latim{
Óculi mei ad fidéles terræ ut sédeant mecum: * ámbulans in via immaculáta, hic mihi ministrábat.
}\switchcolumn\portugues{
Meus olhos buscavam os fiéis da terra para que se sentassem comigo: * andava por caminho inocente, esse me servia.
}\switchcolumn*\latim{
Non habitábit in médio domus meæ qui facit supérbiam: * qui lóquitur iníqua, non diréxit in conspéctu oculórum meórum.
}\switchcolumn\portugues{
Não habitará na minha casa o que com soberba procede: * o que diz iníquidade não pôde tornar-se agradável aos meus olhos.
}\switchcolumn*\latim{
In matutíno interficiébam omnes peccatóres terræ: * ut dispérderem de civitáte Dómini omnes operántes iniquitátem.
}\switchcolumn\portugues{
Pela manhã exterminava todos os pecadores da terra: * a fim de suprimir da cidade do Senhor todos os que cometem a iniquidade.
}\end{paracol}


\subsectioninfo{Salmo 101}{Domine, exaudi orationem}\label{salmo101}
\begin{paracol}{2}\latim{
\rlettrine{D}{ómine,} exáudi oratiónem meam: * et clamor meus ad Te véniat.
}\switchcolumn\portugues{
\rlettrine{S}{enhor,} ouvi a minha oração: * e chegue até Vós o meu clamor.
}\switchcolumn*\latim{
Non avértas fáciem tuam a me: * in quacúmque die tríbulor, inclína ad me aurem tuam.
}\switchcolumn\portugues{
Não aparteis de mim o vosso rosto: * no dia do tormento, inclinai para mim o vosso ouvido.
}\switchcolumn*\latim{
In quacúmque die invocávero Te, * velóciter exáudi me.
}\switchcolumn\portugues{
Em qualquer dia que Vos invocar, * prontamente me ouvi.
}\switchcolumn*\latim{
Quia defecérunt sicut fumus dies mei: * et ossa mea sicut crémium aruérunt.
}\switchcolumn\portugues{
Pois os meus dias dissiparam-se como fumo: * e os meus ossos secaram como acendalhas.
}\switchcolumn*\latim{
Percússus sum ut fænum, et áruit cor meum: * quia oblítus sum comédere panem meum.
}\switchcolumn\portugues{
Fui ferido como feno e o meu coração secou-se: * pois me esqueci de comer o meu pão.
}\switchcolumn*\latim{
A voce gémitus mei * adhǽsit os meum carni meæ.
}\switchcolumn\portugues{
À voz dos meus gemidos, * pegaram-se os meus ossos à minha pele.
}\switchcolumn*\latim{
Símilis factus sum pellicáno solitúdinis: * factus sum sicut nyctícorax in domicílio.
}\switchcolumn\portugues{
Tornei-me semelhante ao pelicano do deserto: * tornei-me como a coruja no seu albergue.
}\switchcolumn*\latim{
Vigilávi, * et factus sum sicut passer solitárius in tecto.
}\switchcolumn\portugues{
Velei * e tornei-me como o pássaro solitário no telhado.
}\switchcolumn*\latim{
Tota die exprobrábant mihi inimíci mei: * et qui laudábant me, advérsum me jurábant.
}\switchcolumn\portugues{
Todo o dia me injuriavam os meus inimigos: * e os que me louvavam conspiravam contra mim.
}\switchcolumn*\latim{
Quia cínerem tamquam panem manducábam, * et potum meum cum fletu miscébam.
}\switchcolumn\portugues{
Pois comia cinza como pão, * e misturava a minha bebida com minhas lágrimas.
}\switchcolumn*\latim{
A fácie iræ et indignatiónis tuæ: * quia élevans allisísti me.
}\switchcolumn\portugues{
À vista de vossa ira e indignação: * pois depois de me elevares, me arrojastes.
}\switchcolumn*\latim{
Dies mei sicut umbra declinavérunt: * et ego sicut fænum árui.
}\switchcolumn\portugues{
Meus dias declinaram como a sombra: * e eu sequei-me como feno.
}\switchcolumn*\latim{
Tu autem, Dómine, in ætérnum pérmanes: * et memoriále tuum in generatiónem et generatiónem.
}\switchcolumn\portugues{
Contudo, ó Senhor, Vós permaneceis para sempre: * e o vosso nome de geração em geração.
}\switchcolumn*\latim{
Tu exsúrgens miseréberis Sion: * quia tempus miseréndi ejus, quia venit tempus.
}\switchcolumn\portugues{
Vós, levantando-Vos, tereis piedade de Sião: * pois é tempo de terdes piedade dela e o tempo já chegou.
}\switchcolumn*\latim{
Quóniam placuérunt servis tuis lápides ejus: * et terræ ejus miserebúntur.
}\switchcolumn\portugues{
Porque as suas próprias ruínas são amadas pelos vossos servos: * e se compadecerão da sua terra.
}\switchcolumn*\latim{
Et timébunt gentes nomen tuum, Dómine, * et omnes reges terræ glóriam tuam.
}\switchcolumn\portugues{
As gentes temerão o vosso nome, ó Senhor, * e todos os reis da terra respeitarão a vossa glória.
}\switchcolumn*\latim{
Quia ædificávit Dóminus Sion: * et vidébitur in glória sua.
}\switchcolumn\portugues{
Pois o Senhor edificou Sião: * e será visto na sua glória.
}\switchcolumn*\latim{
Respéxit in oratiónem humílium: * et non sprevit precem eórum.
}\switchcolumn\portugues{
Atendeu à oração dos humildes: * e não desprezou a sua prece.
}\switchcolumn*\latim{
Scribántur hæc in generatióne áltera: * et pópulus, qui creábitur, laudábit Dóminum:
}\switchcolumn\portugues{
Escrevam estas cousas para a geração futura: * e o povo, que há-de ser criado, louvará o Senhor:
}\switchcolumn*\latim{
Quia prospéxit de excélso sancto suo: * Dóminus de cælo in terram aspéxit:
}\switchcolumn\portugues{
Pois olhou do alto do seu santuário: * o Senhor olhou do céu sobre a terra:
}\switchcolumn*\latim{
Ut audíret gémitus compeditórum: * ut sólveret fílios interemptórum:
}\switchcolumn\portugues{
Para ouvir os gemidos dos encarcerados: * para libertar os filhos dos condenados à morte:
}\switchcolumn*\latim{
Ut annúntient in Sion nomen Dómini: * et laudem ejus in Jerúsalem.
}\switchcolumn\portugues{
A fim de que anunciem em Sião o nome do Senhor: * e o seu louvor em Jerusalém.
}\switchcolumn*\latim{
In conveniéndo pópulos in unum, * et reges ut sérviant Dómino.
}\switchcolumn\portugues{
Quando se juntarem os povos * e os reis para servirem ao Senhor.
}\switchcolumn*\latim{
Respóndit ei in via virtútis suæ: * Paucitátem diérum meórum núntia mihi.
}\switchcolumn\portugues{
Disse-lhe na expansão da sua força: * manifestai-me o curto número de meus dias.
}\switchcolumn*\latim{
Ne révoces me in dimídio diérum meórum: * in generatiónem et generatiónem anni tui.
}\switchcolumn\portugues{
Não me chameis na metade de meus dias: * os vossos anos estendem-se de geração em geração.
}\switchcolumn*\latim{
Inítio Tu, Dómine, terram fundásti: * et ópera mánuum tuárum sunt cæli.
}\switchcolumn\portugues{
No princípio, ó Senhor, fundastes a terra: * e os céus são obra de vossas mãos.
}\switchcolumn*\latim{
Ipsi períbunt, Tu autem pérmanes: * et omnes sicut vestiméntum veteráscent.
}\switchcolumn\portugues{
Eles perecerão, mas Vós permanecereis: * todos eles envelhecerão como um vestido.
}\switchcolumn*\latim{
Et sicut opertórium mutábis eos, et mutabúntur: * Tu autem idem ipse es, et anni tui non defícient.
}\switchcolumn\portugues{
Como roupa os mudareis e serão mudados: * Vós, porém, sois sempre o mesmo e os vossos anos não terão fim.
}\switchcolumn*\latim{
Fílii servórum tuórum habitábunt: * et semen eórum in sǽculum dirigétur.
}\switchcolumn\portugues{
Os filhos de vossos servos habitarão: * e a sua posteridade será orientada eternamente.
}\end{paracol}


\subsectioninfo{Salmo 102}{Benedic, anima mea}\label{salmo102}
\begin{paracol}{2}\latim{
\rlettrine{B}{énedic,} ánima mea, Dómino: * et ómnia, quæ intra me sunt, nómini sancto ejus.
}\switchcolumn\portugues{
\rlettrine{M}{inha} alma, bendiz o Senhor: * e tudo o que em mim há, o seu santo nome.
}\switchcolumn*\latim{
Bénedic, ánima mea, Dómino: * et noli oblivísci omnes retributiónes ejus.
}\switchcolumn\portugues{
Bendiz o Senhor, ó minha alma: * e não esqueças nem um dos seus benefícios.
}\switchcolumn*\latim{
Qui propitiátur ómnibus iniquitátibus tuis: * qui sanat omnes infirmitátes tuas.
}\switchcolumn\portugues{
É Ele que perdoa todas tuas iniquidades: * e que sara todas tuas enfermidades.
}\switchcolumn*\latim{
Qui rédimit de intéritu vitam tuam: * qui corónat te in misericórdia et miseratiónibus.
}\switchcolumn\portugues{
É Ele que resgata da morte a tua vida: * e que te coroa da sua misericórdia e das suas graças.
}\switchcolumn*\latim{
Qui replet in bonis desidérium tuum: * renovábitur ut áquilæ juvéntus tua:
}\switchcolumn\portugues{
É Ele que com bens sacia o teu desejo: * a tua mocidade renovar-se-á como a da águia:
}\switchcolumn*\latim{
Fáciens misericórdias Dóminus: * et judícium ómnibus injúriam patiéntibus.
}\switchcolumn\portugues{
O Senhor faz misericórdias: * e justiça a todos os que sofrem agravos.
}\switchcolumn*\latim{
Notas fecit vias suas Móysi, * fíliis Israël voluntátes suas.
}\switchcolumn\portugues{
Fez conhecer a Moisés os seus caminhos, * e aos filhos de Israel as suas vontades.
}\switchcolumn*\latim{
Miserátor, et miséricors Dóminus: * longánimis, et multum miséricors.
}\switchcolumn\portugues{
O Senhor é compassivo e misericordioso: * paciente e de muita misericórdia.
}\switchcolumn*\latim{
Non in perpétuum irascétur: * neque in ætérnum comminábitur.
}\switchcolumn\portugues{
Não ficará irado para sempre: * nem ameaçará perpetuamente.
}\switchcolumn*\latim{
Non secúndum peccáta nostra fecit nobis: * neque secúndum iniquitátes nostras retríbuit nobis.
}\switchcolumn\portugues{
Segundo os nossos pecados nos não tratou: * nem nos puniu segundo as nossas iniquidades.
}\switchcolumn*\latim{
Quóniam secúndum altitúdinem cæli a terra: * corroborávit misericórdiam suam super timéntes se.
}\switchcolumn\portugues{
Porque segundo a altura do céu acima da terra: * estabeleceu Ele a sua misericórdia sobre os que O temem.
}\switchcolumn*\latim{
Quantum distat ortus ab occidénte: * longe fecit a nobis iniquitátes nostras.
}\switchcolumn\portugues{
Quanto o oriente dista do ocidente: * tanto Ele afastou de nós as nossas iniquidades.
}\switchcolumn*\latim{
Quómodo miserétur pater filiórum, misértus est Dóminus timéntibus se: * quóniam ipse cognóvit figméntum nostrum.
}\switchcolumn\portugues{
Como um pai se compadece dos seus filhos, assim se compadeceu o Senhor dos que O temem: * porque conhece a nossa forma.
}\switchcolumn*\latim{
Recordátus est quóniam pulvis sumus: * homo, sicut fænum dies ejus, tamquam flos agri sic efflorébit.
}\switchcolumn\portugues{
Lembrou-se que somos pó: * os dias do homem passam como o feno, como a flor do campo, assim floresce.
}\switchcolumn*\latim{
Quóniam spíritus pertransíbit in illo, et non subsístet: * et non cognóscet ámplius locum suum.
}\switchcolumn\portugues{
Porque um sopro de vento passará sobre ele e não subsistirá: * e seu lugar o não mais conhecerá.
}\switchcolumn*\latim{
Misericórdia autem Dómini ab ætérno, * et usque in ætérnum super timéntes eum.
}\switchcolumn\portugues{
Porém, da eternidade vem a misericórdia do Senhor, * e até à eternidade sobre os que O temem.
}\switchcolumn*\latim{
Et justítia illíus in fílios filiórum, * his qui servant testaméntum ejus:
}\switchcolumn\portugues{
Sua justiça nos filhos dos filhos, * para aqueles que guardam a sua aliança:
}\switchcolumn*\latim{
Et mémores sunt mandatórum ipsíus, * ad faciéndum ea.
}\switchcolumn\portugues{
Se lembram dos seus mandamentos, * para os observar.
}\switchcolumn*\latim{
Dóminus in cælo parávit sedem suam: * et regnum ipsíus ómnibus dominábitur.
}\switchcolumn\portugues{
O Senhor preparou o seu trono no céu: * e sobre todos dominará o seu reino.
}\switchcolumn*\latim{
Benedícite Dómino, omnes Ángeli ejus: * poténtes virtúte, faciéntes verbum illíus, ad audiéndam vocem sermónum ejus.
}\switchcolumn\portugues{
Bendizei o Senhor, todos seus anjos: * poderosos em força, que executais a sua palavra, ouvindo a voz das suas ordens.
}\switchcolumn*\latim{
Benedícite Dómino, omnes virtútes ejus: * minístri ejus, qui fácitis voluntátem ejus.
}\switchcolumn\portugues{
Bendizei o Senhor, todos seus exércitos: * seus ministros, que fazeis a sua vontade.
}\switchcolumn*\latim{
Benedícite Dómino, ómnia ópera ejus: * in omni loco dominatiónis ejus, bénedic, ánima mea, Dómino.
}\switchcolumn\portugues{
Bendizei o Senhor, todas suas obras: * ó minha alma, bendiz o Senhor em todo o lugar do seu domínio.
}\end{paracol}


\subsectioninfo{Salmo 103}{Benedic, anima mea, Domino}\label{salmo103}
\begin{paracol}{2}\latim{
\rlettrine{B}{énedic,} ánima mea, Dómino: * Dómine, Deus meus, magnificátus es veheménter.
}\switchcolumn\portugues{
\rlettrine{B}{endiz} o Senhor, ó minha alma: * ó Senhor meu Deus, Vos engrandecestes sumamente.
}\switchcolumn*\latim{
Confessiónem, et decórem induísti: * amíctus lúmine sicut vestiménto:
}\switchcolumn\portugues{
Com glória e majestade Vos revestistes: * como um traje coberto de luz.
}\switchcolumn*\latim{
Exténdens cælum sicut pellem: * qui tegis aquis superióra ejus.
}\switchcolumn\portugues{
Como a tenda, estendeis o céu: * que cobris de água a sua cobertura.
}\switchcolumn*\latim{
Qui ponis nubem ascénsum tuum: * qui ámbulas super pennas ventórum.
}\switchcolumn\portugues{
Que subis sobre as nuvens: * e sobre as asas dos ventos andeis.
}\switchcolumn*\latim{
Qui facis ángelos tuos, spíritus: * et minístros tuos ignem uréntem.
}\switchcolumn\portugues{
Que fazeis os vossos anjos espíritos: * e que os vossos ministros sejam fogo ardente.
}\switchcolumn*\latim{
Qui fundásti terram super stabilitátem suam: * non inclinábitur in sǽculum sǽculi.
}\switchcolumn\portugues{
Que fundastes a terra sobre as suas bases: * ela se não desnivelará pelos séculos dos séculos.
}\switchcolumn*\latim{
Abýssus, sicut vestiméntum, amíctus ejus: * super montes stabunt aquæ.
}\switchcolumn\portugues{
O abysmo cinge-a como um traje: * as águas elevam-se acima das montanhas.
}\switchcolumn*\latim{
Ab increpatióne tua fúgient: * a voce tonítrui tui formidábunt.
}\switchcolumn\portugues{
À vossa ameaça fugirão: * temerão à voz de vosso trovão.
}\switchcolumn*\latim{
Ascéndunt montes: et descéndunt campi * in locum, quem fundásti eis.
}\switchcolumn\portugues{
Elevam-se as montanhas e os vales descem, * ao lugar que lhes estabelecestes.
}\switchcolumn*\latim{
Términum posuísti, quem non transgrediéntur: * neque converténtur operíre terram.
}\switchcolumn\portugues{
Instituístes-lhes limites, que não ultrapassarão: * e não volverão a cobrir a terra.
}\switchcolumn*\latim{
Qui emíttis fontes in convállibus: * inter médium móntium pertransíbunt aquæ.
}\switchcolumn\portugues{
Vós fazeis sair as fontes nos vales: * as águas passam por meio dos montes.
}\switchcolumn*\latim{
Potábunt omnes béstiæ agri: * exspectábunt ónagri in siti sua.
}\switchcolumn\portugues{
Todos os animais do campo beberão: * suspiram os asnos selvagens na sua sede.
}\switchcolumn*\latim{
Super ea vólucres cæli habitábunt: * de médio petrárum dabunt voces.
}\switchcolumn\portugues{
Sobre elas habitam as aves do céu: * do meio dos rochedos, farão ouvir as suas vozes.
}\switchcolumn*\latim{
Rigans montes de superióribus suis: * de fructu óperum tuórum satiábitur terra:
}\switchcolumn\portugues{
Regais os montes dos altos: * com o fruto de vossas obras a terra será saciada:
}\switchcolumn*\latim{
Prodúcens fænum juméntis, * et herbam servitúti hóminum:
}\switchcolumn\portugues{
Feno produzis para os animais, * e plantas para uso dos homens:
}\switchcolumn*\latim{
Ut edúcas panem de terra: * et vinum lætíficet cor hóminis:
}\switchcolumn\portugues{
Pão fazeis sair do seio da terra: * e vinho que alegra o coração do homem:
}\switchcolumn*\latim{
Ut exhílaret fáciem in óleo: * et panis cor hóminis confírmet.
}\switchcolumn\portugues{
Azeite para espalhar a alegria sobre o rosto: * e pão para fortificar o coração.
}\switchcolumn*\latim{
Saturabúntur ligna campi, et cedri Líbani, quas plantávit: * illic pásseres nidificábunt.
}\switchcolumn\portugues{
Encher-se-ão de seiva as árvores do campo e os cedros do Líbano que plantou: * ali farão ninhos as aves.
}\switchcolumn*\latim{
Heródii domus dux est eórum: * montes excélsi cervis: petra refúgium herináciis.
}\switchcolumn\portugues{
A casa da cegonha lhes serve de guia: * os montes altos são refúgio dos veados e os penhascos dos ouriços.
}\switchcolumn*\latim{
Fecit lunam in témpora: * sol cognóvit occásum suum.
}\switchcolumn\portugues{
Fez a lua para marcar os tempos: * o sol conhece o seu ocaso.
}\switchcolumn*\latim{
Posuísti ténebras, et facta est nox: * in ipsa pertransíbunt omnes béstiæ silvæ.
}\switchcolumn\portugues{
Espalhastes as trevas e a noite se fez: * vagueiam então todos os animais da selva.
}\switchcolumn*\latim{
Cátuli leónum rugiéntes, ut rápiant, * et quǽrant a Deo escam sibi.
}\switchcolumn\portugues{
Os leõezinhos rugem em busca da presa, * e pedem a Deus o seu sustento.
}\switchcolumn*\latim{
Ortus est sol, et congregáti sunt: * et in cubílibus suis collocabúntur.
}\switchcolumn\portugues{
Desponta o sol e reúnem-se: * e vão esconder-se nos seus covis.
}\switchcolumn*\latim{
Exíbit homo ad opus suum: * et ad operatiónem suam usque ad vésperum.
}\switchcolumn\portugues{
Sairá o homem para a sua obra: * e para os seus trabalhos até entardecer.
}\switchcolumn*\latim{
Quam magnificáta sunt ópera tua, Dómine! * ómnia in sapiéntia fecísti: impléta est terra possessióne tua.
}\switchcolumn\portugues{
Quão magníficas são as vossas obras, ó Senhor! * Fizestes com sabedoria todas as cousas: a terra está cheia das vossas riquezas.
}\switchcolumn*\latim{
Hoc mare magnum, et spatiósum mánibus: * illic reptília, quorum non est númerus.
}\switchcolumn\portugues{
Este mar grande e de longos braços: * nele existem peixes sem número.
}\switchcolumn*\latim{
Animália pusílla cum magnis: * illic naves pertransíbunt.
}\switchcolumn\portugues{
Animais pequenos e grandes: * por ele transitam os navios.
}\switchcolumn*\latim{
Draco iste, quem formásti ad illudéndum ei: * ómnia a Te exspéctant ut des illis escam in témpore.
}\switchcolumn\portugues{
Lá brinca esse dragão que formastes: * todos esperam de Vós que lhes deis de comer a seu tempo.
}\switchcolumn*\latim{
Dante Te illis, cólligent: * aperiénte Te manum tuam, ómnia implebúntur bonitáte.
}\switchcolumn\portugues{
Dando-lho Vós, eles o recolhem: * abrindo Vós vossa mão, todos se encherão de bens.
}\switchcolumn*\latim{
Averténte autem Te fáciem, turbabúntur: * áuferes spíritum eórum, et defícient, et in púlverem suum reverténtur.
}\switchcolumn\portugues{
Mas, se apartardes o vosso rosto, turvar-se-ão: * tirar-lhes-eis o espírito, deixarão de ser e ao pó retornarão.
}\switchcolumn*\latim{
Emíttes spíritum tuum, et creabúntur: * et renovábis fáciem terræ.
}\switchcolumn\portugues{
Enviareis o vosso espírito e serão criados: * e renovareis a face da terra.
}\switchcolumn*\latim{
Sit glória Dómini in sǽculum: * lætábitur Dóminus in opéribus suis:
}\switchcolumn\portugues{
Seja celebrada a glória do Senhor para sempre: * alegrar-se-á o Senhor nas suas obras:
}\switchcolumn*\latim{
Qui réspicit terram, et facit eam trémere: * qui tangit montes, et fúmigant.
}\switchcolumn\portugues{
Olha para a terra e tremer a faz: * toca os montes e eles fumegam.
}\switchcolumn*\latim{
Cantábo Dómino in vita mea: * psallam Deo meo, quámdiu sum.
}\switchcolumn\portugues{
Cantarei ao Senhor durante a minha vida: * cantarei hinos a meu Deus enquanto existir.
}\switchcolumn*\latim{
Jucúndum sit ei elóquium meum: * ego vero delectábor in Dómino.
}\switchcolumn\portugues{
Sejam-Lhe agradáveis as minhas palavras: * quanto a mim, deleitar-me-ei no Senhor.
}\switchcolumn*\latim{
Defíciant peccatóres a terra, et iníqui ita ut non sint: * bénedic, ánima mea, Dómino.
}\switchcolumn\portugues{
Desapareçam da terra os pecadores e os iníquos não mais existam: * bendiz o Senhor, ó minha alma.
}\end{paracol}


\subsectioninfo{Salmo 104}{Confitemini Domino}\label{salmo104}
\begin{paracol}{2}\latim{
\rlettrine{C}{onfitémini} Dómino, et invocáte nomen ejus: * annuntiáte inter gentes ópera ejus.
}\switchcolumn\portugues{
\rlettrine{L}{ouvai} o Senhor e invocai o seu nome: * anunciai as suas obras entre as gentes.
}\switchcolumn*\latim{
Cantáte ei, et psállite ei: * narráte ómnia mirabília ejus.
}\switchcolumn\portugues{
Cantai-Lhe e salmodiai-Lhe: * narrai todas suas maravilhas.
}\switchcolumn*\latim{
Laudámini in nómine sancto ejus: * lætétur cor quæréntium Dóminum.
}\switchcolumn\portugues{
Gloriai-vos em seu santo nome: * alegre-se o coração dos que buscam o Senhor.
}\switchcolumn*\latim{
Quǽrite Dóminum, et confirmámini: * quǽrite fáciem ejus semper.
}\switchcolumn\portugues{
Buscai o Senhor e fortalecei-vos n’Ele: * buscai sempre a sua face.
}\switchcolumn*\latim{
Mementóte mirabílium ejus, quæ fecit: * prodígia ejus, et judícia oris ejus.
}\switchcolumn\portugues{
Lembrai-vos das maravilhas que fez: * dos seus prodígios e as sentenças da sua boca:
}\switchcolumn*\latim{
Semen Ábraham, servi ejus: * fílii Jacob, elécti ejus.
}\switchcolumn\portugues{
Descendentes de Abraão, seus servos: * filhos de Jacob, seus escolhidos.
}\switchcolumn*\latim{
Ipse Dóminus Deus noster: * in univérsa terra judícia ejus.
}\switchcolumn\portugues{
Ele é o Senhor nosso Deus: * os seus juízos exercem-se em toda a terra.
}\switchcolumn*\latim{
Memor fuit in sǽculum testaménti sui: * verbi, quod mandávit in mille generatiónes:
}\switchcolumn\portugues{
Ele lembrou-se para sempre da sua aliança: * e da palavra que comandou a mil gerações:
}\switchcolumn*\latim{
Quod dispósuit ad Ábraham: * et juraménti sui ad Isaac:
}\switchcolumn\portugues{
Que a Abraão fez: * e do seu juramento a Isaac:
}\switchcolumn*\latim{
Et státuit illud Jacob in præcéptum: * et Israël in testaméntum ætérnum:
}\switchcolumn\portugues{
O mesmo confirmou a Jacob como lei: * e a Israel para uma eterna aliança:
}\switchcolumn*\latim{
Dicens: tibi dabo terram Chánaan, * funículum hereditátis vestræ.
}\switchcolumn\portugues{
Dizendo: dar-te-ei a terra de Canaan, * porção de tua herança.
}\switchcolumn*\latim{
Cum essent número brevi, * paucíssimi et íncolæ ejus:
}\switchcolumn\portugues{
Quando em pequeno número, * sendo pouquíssimos e estrangeiros:
}\switchcolumn*\latim{
Et pertransiérunt de gente in gentem, * et de regno ad pópulum álterum.
}\switchcolumn\portugues{
Passavam de gente para gente, * e dum reino para outro povo.
}\switchcolumn*\latim{
Non relíquit hóminem nocére eis: * et corrípuit pro eis reges.
}\switchcolumn\portugues{
Homem que os ofendesse não permitiu: * e reis castigou por causa deles.
}\switchcolumn*\latim{
Nolíte tángere christos meos: * et in prophétis meis nolíte malignári.
}\switchcolumn\portugues{
Não toqueis os meus ungidos: * e meus profetas os não maltrateis.
}\switchcolumn*\latim{
Et vocávit famem super terram: * et omne firmaméntum panis contrívit.
}\switchcolumn\portugues{
Chamou a fome sobre a terra: * e destruiu todo o sustento do pão.
}\switchcolumn*\latim{
Misit ante eos virum: * in servum venúmdatus est Joseph.
}\switchcolumn\portugues{
Enviou adiante deles um homem: * a José que foi vendido como escravo.
}\switchcolumn*\latim{
Humiliavérunt in compédibus pedes ejus, ferrum pertránsiit ánimam ejus * donec veníret verbum ejus.
}\switchcolumn\portugues{
Humilharam-no com grilhões nos pés, o ferro traspassou a sua alma * até que se cumpriu o seu vaticínio.
}\switchcolumn*\latim{
Elóquium Dómini inflammávit eum: * misit rex, et solvit eum; princeps populórum, et dimísit eum.
}\switchcolumn\portugues{
A palavra do Senhor o tinha inflamado: * o rei mandou que o soltassem, o príncipe dos povos deu-lhe a liberdade.
}\switchcolumn*\latim{
Constítuit eum dóminum domus suæ: * et príncipem omnis possessiónis suæ:
}\switchcolumn\portugues{
Constituiu-o senhor da sua casa: * e príncipe de tudo quanto possuía:
}\switchcolumn*\latim{
Ut erudíret príncipes ejus sicut semetípsum: * et senes ejus prudéntiam docéret.
}\switchcolumn\portugues{
Para que instruísse os seus príncipes como a si: * e ensinasse a prudência aos seus anciãos.
}\switchcolumn*\latim{
Et intrávit Israël in Ægýptum: * et Jacob áccola fuit in terra Cham.
}\switchcolumn\portugues{
Israel entrou no Egipto: * e Jacob foi hóspede na terra de Cam.
}\switchcolumn*\latim{
Et auxit pópulum suum veheménter: * et firmávit eum super inimícos ejus.
}\switchcolumn\portugues{
Aumentou veementemente o seu povo: * e tornou-o forte sobre os seus inimigos.
}\switchcolumn*\latim{
Convértit cor eórum ut odírent pópulum ejus: * et dolum fácerent in servos ejus.
}\switchcolumn\portugues{
Converteu-lhes o coração para que odiassem o seu povo: * e usassem de dolos com seus servos.
}\switchcolumn*\latim{
Misit Móysen, servum suum: * Aaron, quem elégit ipsum.
}\switchcolumn\portugues{
Enviou Moisés, seu servo: * e Arão, a quem tinha escolhido.
}\switchcolumn*\latim{
Pósuit in eis verba signórum suórum: * et prodigiórum in terra Cham.
}\switchcolumn\portugues{
Deu-lhes poder para fazer milagres: * e prodígios na terra de Cam.
}\switchcolumn*\latim{
Misit ténebras, et obscurávit: * et non exacerbávit sermónes suos.
}\switchcolumn\portugues{
Enviou trevas e escureceu: * e com suas palavras se não exacerbaram.
}\switchcolumn*\latim{
Convértit aquas eórum in sánguinem: * et occídit pisces eórum.
}\switchcolumn\portugues{
Converteu-lhes as águas em sangue: * e matou os seus peixes.
}\switchcolumn*\latim{
Édidit terra eórum ranas: * in penetrálibus regum ipsórum.
}\switchcolumn\portugues{
Sua terra produziu rãs: * até nos aposentos dos próprios reis.
}\switchcolumn*\latim{
Dixit, et venit cœnomyía: * et cínifes in ómnibus fínibus eórum.
}\switchcolumn\portugues{
Falou e vieram moscas: * e mosquitos por todo o território.
}\switchcolumn*\latim{
Pósuit plúvias eórum grándinem: * ignem comburéntem in terra ipsórum.
}\switchcolumn\portugues{
Em vez de água lhes fez chover granizo: * lançou um fogo abrasador na terra deles.
}\switchcolumn*\latim{
Et percússit víneas eórum, et ficúlneas eórum: * et contrívit lignum fínium eórum.
}\switchcolumn\portugues{
Feriu as suas vinhas e os seus figueirais: * e quebrou as árvores que havia nos seus limites.
}\switchcolumn*\latim{
Dixit, et venit locústa, et bruchus, * cujus non erat númerus:
}\switchcolumn\portugues{
Falou e vieram gafanhotos e lagartos * cujos não tinham número:
}\switchcolumn*\latim{
Et comédit omne fænum in terra eórum: * et comédit omnem fructum terræ eórum.
}\switchcolumn\portugues{
Devoraram toda a erva dos prados: * e comeram todos os frutos dos seus campos.
}\switchcolumn*\latim{
Et percússit omne primogénitum in terra eórum: * primítias omnis labóris eórum.
}\switchcolumn\portugues{
Feriu todos os primogénitos da sua terra: * as primícias de todo seu trabalho.
}\switchcolumn*\latim{
Et edúxit eos cum argénto et auro: * et non erat in tríbubus eórum infírmus.
}\switchcolumn\portugues{
Os conduziu com prata e com ouro: * e não havia enfermo nas suas tribos.
}\switchcolumn*\latim{
Lætáta est Ægýptus in profectióne eórum: * quia incúbuit timor eórum super eos.
}\switchcolumn\portugues{
Alegrou-se o Egipto com a partida deles: * pois era sobre eles o seu temor.
}\switchcolumn*\latim{
Expándit nubem in protectiónem eórum: * et ignem ut lucéret eis per noctem.
}\switchcolumn\portugues{
Estendeu uma nuvem que os cobrisse: * e um fogo que os alumiasse de noite.
}\switchcolumn*\latim{
Petiérunt, et venit cotúrnix: * et pane cæli saturávit eos.
}\switchcolumn\portugues{
Pediram e vieram codornizes: * e de pão do céu os saciou.
}\switchcolumn*\latim{
Dirúpit petram et fluxérunt aquæ: * abiérunt in sicco flúmina;
}\switchcolumn\portugues{
Fendeu a pedra e brotaram águas: * correram rios no deserto.
}\switchcolumn*\latim{
Quóniam memor fuit verbi sancti sui: * quod hábuit ad Ábraham, púerum suum.
}\switchcolumn\portugues{
Porque se lembrou da sua santa palavra: * que tinha dado a Abraão, seu servo.
}\switchcolumn*\latim{
Et edúxit pópulum suum in exsultatióne, * et eléctos suos in lætítia.
}\switchcolumn\portugues{
Fez sair o seu povo com exaltação, * e os seus escolhidos com alegria.
}\switchcolumn*\latim{
Et dedit illis regiónes géntium: * et labóres populórum possedérunt:
}\switchcolumn\portugues{
Deu-lhes as terras das gentes: * e herdaram o trabalho dos povos:
}\switchcolumn*\latim{
Ut custódiant justificatiónes ejus, * et legem ejus requírant.
}\switchcolumn\portugues{
Para que guardassem os seus mandamentos, * e buscassem a sua lei.
}\end{paracol}


\subsectioninfo{Salmo 105}{Confitemini Domino, quoniam bonus}\label{salmo105}
\begin{paracol}{2}\latim{
\rlettrine{C}{onfitémini} Dómino, quóniam bonus: * quóniam in sǽculum misericórdia ejus.
}\switchcolumn\portugues{
\rlettrine{L}{ouvai} o Senhor, porque é bom: * e porque a sua misericórdia é eterna.
}\switchcolumn*\latim{
Quis loquétur poténtias Dómini, * audítas fáciet omnes laudes ejus?
}\switchcolumn\portugues{
Quem contará os poderes do Senhor, * quem fará que sejam ouvidos todos seus louvores?
}\switchcolumn*\latim{
Beáti, qui custódiunt judícium, * et fáciunt justítiam in omni témpore.
}\switchcolumn\portugues{
Bem-aventurados os que observam a lei, * e praticam a justiça em todo o tempo.
}\switchcolumn*\latim{
Meménto nostri, Dómine, in beneplácito pópuli tui: * vísita nos in salutári tuo:
}\switchcolumn\portugues{
Lembrai-Vos de nós, ó Senhor, em mercê de vosso povo: * visitai-nos com vossa salvação:
}\switchcolumn*\latim{
Ad vidéndum in bonitáte electórum tuórum, ad lætándum in lætítia gentis tuæ: * ut laudéris cum hereditáte tua.
}\switchcolumn\portugues{
Para vermos a felicidade de vossos escolhidos, gozemos a alegria de vosso povo: * para serdes glorificado na vossa herança.
}\switchcolumn*\latim{
Peccávimus cum pátribus nostris: * injúste égimus, iniquitátem fécimus.
}\switchcolumn\portugues{
Pecámos com os nossos pais: * procedemos injustamente, cometemos a iniquidade.
}\switchcolumn*\latim{
Patres nostri in Ægýpto non intellexérunt mirabília tua: * non fuérunt mémores multitúdinis misericórdiæ tuæ.
}\switchcolumn\portugues{
Nossos pais no Egipto não entenderam as vossas maravilhas: * se não lembraram da multidão de vossas misericórdias.
}\switchcolumn*\latim{
Et irritavérunt ascendéntes in mare, * Mare Rubrum.
}\switchcolumn\portugues{
Irritaram-Vos indo para o mar, * mar Vermelho.
}\switchcolumn*\latim{
Et salvávit eos propter nomen suum: * ut notam fáceret poténtiam suam.
}\switchcolumn\portugues{
Os salvou, por amor do seu nome: * para mostrar o seu poder.
}\switchcolumn*\latim{
Et incrépuit Mare Rubrum, et exsiccátum est, * et dedúxit eos in abýssis sicut in desérto.
}\switchcolumn\portugues{
Ameaçou o mar Vermelho e ele secou-se, * e levou-os pelos abysmos, como por um deserto.
}\switchcolumn*\latim{
Et salvávit eos de manu odiéntium: * et redémit eos de manu inimíci.
}\switchcolumn\portugues{
Salvou-os da mão dos que os odiavam: * e livrou-os da mão do inimigo.
}\switchcolumn*\latim{
Et opéruit aqua tribulántes eos: * unus ex eis non remánsit.
}\switchcolumn\portugues{
A água cobriu os perseguidores: * deles não escapou um só.
}\switchcolumn*\latim{
Et credidérunt verbis ejus: * et laudavérunt laudem ejus.
}\switchcolumn\portugues{
Deram crédito às suas palavras: * e cantaram o seu louvor.
}\switchcolumn*\latim{
Cito fecérunt, oblíti sunt óperum ejus: * et non sustinuérunt consílium ejus.
}\switchcolumn\portugues{
Porém, depressa esqueceram as suas obras: * e não esperaram o seu conselho.
}\switchcolumn*\latim{
Et concupiérunt concupiscéntiam in desérto: * et tentavérunt Deum in inaquóso.
}\switchcolumn\portugues{
Cobiçaram delícias no deserto: * e tentaram a Deus no lugar sem água.
}\switchcolumn*\latim{
Et dedit eis petitiónem ipsórum: * et misit saturitátem in ánimas eórum.
}\switchcolumn\portugues{
Concedeu-lhes o que pediam: * e enviou fartura às suas almas.
}\switchcolumn*\latim{
Et irritavérunt Moysen in castris: * Aaron, sanctum Dómini.
}\switchcolumn\portugues{
Irritaram Moisés no acampamento: * e Arão, o santo do Senhor.
}\switchcolumn*\latim{
Apérta est terra, et deglutívit Dathan: * et opéruit super congregatiónem Abíron.
}\switchcolumn\portugues{
Abriu-se a terra e engoliu Datan: * e sepultou Abiron com seus compinchas.
}\switchcolumn*\latim{
Et exársit ignis in synagóga eorum: * flamma combússit peccatóres.
}\switchcolumn\portugues{
Ateou-se fogo no meio da congregação: * a chama incendiou os pecadores.
}\switchcolumn*\latim{
Et fecérunt vítulum in Horeb: * et adoravérunt scúlptile.
}\switchcolumn\portugues{
Fizeram um bezerro em Horeb: * e adoraram a estátua.
}\switchcolumn*\latim{
Et mutavérunt glóriam suam * in similitúdinem vítuli comedéntis fænum.
}\switchcolumn\portugues{
Trocaram a sua glória * pelo simulacro dum bezerro que come feno.
}\switchcolumn*\latim{
Oblíti sunt Deum, qui salvávit eos, * qui fecit magnália in Ægýpto, mirabília in terra Cham: terribília in Mari Rubro.
}\switchcolumn\portugues{
Esqueceram-se de Deus, que os tinha salvado, * que tinha feito maravilhas no Egipto, milagres na terra de Cam, cousas terríveis no mar Vermelho.
}\switchcolumn*\latim{
Et dixit ut dispérderet eos: * si non Móyses, eléctus ejus, stetísset in confractióne in conspéctu ejus:
}\switchcolumn\portugues{
Disse que os destruiria: * se Moisés, seu escolhido, se não tivesse posto no meio ante ele sobre a brecha:
}\switchcolumn*\latim{
Ut avérteret iram ejus ne dispérderet eos: * et pro níhilo habuérunt terram desiderábilem:
}\switchcolumn\portugues{
A fim de afastar a sua ira, para que os não destruísse: * desprezaram aquela terra desejável:
}\switchcolumn*\latim{
Non credidérunt verbo ejus, et murmuravérunt in tabernáculis suis: * non exaudiérunt vocem Dómini.
}\switchcolumn\portugues{
Não acreditaram na sua palavra e murmuraram nas suas tendas: * e não atenderam à voz do Senhor.
}\switchcolumn*\latim{
Et elevávit manum suam super eos: * ut prostérneret eos in desérto:
}\switchcolumn\portugues{
Ele levantou a sua mão contra eles: * para os exterminar no deserto:
}\switchcolumn*\latim{
Et ut deíceret semen eórum in natiónibus: * et dispérgeret eos in regiónibus.
}\switchcolumn\portugues{
Para envilecer a sua estirpe entre as nações: * e dispersá-los pelas regiões.
}\switchcolumn*\latim{
Et initiáti sunt Beélphegor: * et comedérunt sacrifícia mortuórum.
}\switchcolumn\portugues{
Consagraram-se a Beelfegor: * e comeram os sacrifícios dos mortos.
}\switchcolumn*\latim{
Et irritavérunt eum in adinventiónibus suis: * et multiplicáta est in eis ruína.
}\switchcolumn\portugues{
Irritaram o Senhor com suas inovações: * e multiplicou-se neles a ruína.
}\switchcolumn*\latim{
Et stetit Phínees, et placávit: * et cessávit quassátio.
}\switchcolumn\portugues{
Apresentou-se Finéas, e acalmou-O: * e cessou o flagelo.
}\switchcolumn*\latim{
Et reputátum est ei in justítiam: * in generatiónem et generatiónem usque in sempitérnum.
}\switchcolumn\portugues{
Foi-lhe imputado a justiça: * de geração em geração para sempre.
}\switchcolumn*\latim{
Et irritavérunt eum ad aquas contradictiónis: * et vexátus est Móyses propter eos: quia exacerbavérunt spíritum ejus.
}\switchcolumn\portugues{
Irritaram-n’O nas Águas da contradição: * e Moisés foi castigado por causa deles: pois exacerbaram o seu espírito.
}\switchcolumn*\latim{
Et distínxit in lábiis suis: * non disperdidérunt gentes, quas dixit Dóminus illis.
}\switchcolumn\portugues{
Foi duvidoso nas suas palavras: * não exterminaram as gentes que o Senhor lhes tinha indicado.
}\switchcolumn*\latim{
Et commísti sunt inter gentes, et didicérunt ópera eórum: et serviérunt sculptílibus eórum: * et factum est illis in scándalum.
}\switchcolumn\portugues{
Mesclaram-se com as gentes e imitaram os seus costumes: e servirão os seus ídolos: * e isto foi-lhes causa de ruína.
}\switchcolumn*\latim{
Et immolavérunt fílios suos, * et filias suas dæmóniis.
}\switchcolumn\portugues{
Imolaram os seus filhos, * e as suas filhas aos demónios.
}\switchcolumn*\latim{
Et effudérunt sánguinem innocéntem: * sánguinem filiórum suórum et filiárum suárum, quas sacrificavérunt sculptílibus Chánaan.
}\switchcolumn\portugues{
Derramaram o sangue inocente: * o sangue de seus filhos e de suas filhas, que tinham sacrificado aos ídolos de Canaan.
}\switchcolumn*\latim{
Et infécta est terra in sanguínibus, et contamináta est in opéribus eórum: * et fornicáti sunt in adinventiónibus suis.
}\switchcolumn\portugues{
A terra ficou infectada com tanto sangue e contaminou-se com suas obras: * e prostituíram-se suas invenções.
}\switchcolumn*\latim{
Et irátus est furóre Dóminus in pópulum suum: * et abominátus est hereditátem suam.
}\switchcolumn\portugues{
O Senhor incendiou-se de fúria contra o seu povo: * e abominou a sua herança.
}\switchcolumn*\latim{
Et trádidit eos in manus géntium: * et domináti sunt eórum qui odérunt eos.
}\switchcolumn\portugues{
Entregou-os ao poder das gentes: * e dominaram-nos aqueles que os odiavam.
}\switchcolumn*\latim{
Et tribulavérunt eos inimíci eórum, et humiliáti sunt sub mánibus eórum: * sæpe liberávit eos.
}\switchcolumn\portugues{
Seus inimigos angustiaram-nos e foram humilhados sob o seu poder: * muitas vezes Ele os livrou.
}\switchcolumn*\latim{
Ipsi autem exacerbavérunt eum in consílio suo: * et humiliáti sunt in iniquitátibus suis.
}\switchcolumn\portugues{
Eles, porém, exacerbaram-n’O com seu conselho: * e foram humilhados pelas suas próprias iniquidades.
}\switchcolumn*\latim{
Et vidit, cum tribularéntur: * et audívit oratiónem eórum.
}\switchcolumn\portugues{
Ele olhou-os quando estavam atribulados: * e ouviu a sua oração.
}\switchcolumn*\latim{
Et memor fuit testaménti sui: * et pœnítuit eum secúndum multitúdinem misericórdiæ suæ.
}\switchcolumn\portugues{
Lembrou-se da sua aliança: * e teve piedade deles segundo a sua grande misericórdia.
}\switchcolumn*\latim{
Et dedit eos in misericórdias * in conspéctu ómnium qui céperant eos.
}\switchcolumn\portugues{
Empregou neles as suas misericórdias, * à vista de todos aqueles que os tinham cativos.
}\switchcolumn*\latim{
Salvos nos fac, Dómine, Deus noster: * et cóngrega nos de natiónibus:
}\switchcolumn\portugues{
Salvai-nos, ó Senhor nosso Deus: * e reuni-nos de entre as nações:
}\switchcolumn*\latim{
Ut confiteámur nómini sancto tuo: * et gloriémur in laude tua.
}\switchcolumn\portugues{
Para que celebremos o vosso santo nome: * e nos gloriemos em louvar-Vos.
}\switchcolumn*\latim{
Benedíctus Dóminus, Deus Israël, a sǽculo et usque in sǽculum: * et dicet omnis pópulus: fiat, fiat.
}\switchcolumn\portugues{
Bendito seja o Senhor, Deus de Israel, pelos séculos dos séculos: * e todo o povo responderá: assim seja, assim seja.
}\end{paracol}


\subsectioninfo{Salmo 106}{Confitemini Domino, quoniam bonus, quoniam}\label{salmo106}
\begin{paracol}{2}\latim{
\rlettrine{C}{onfitémini} Dómino quóniam bonus: * quóniam in sǽculum misericórdia ejus.
}\switchcolumn\portugues{
\rlettrine{L}{ouvai} o Senhor, porque Ele é bom: * porque a sua misericórdia é eterna.
}\switchcolumn*\latim{
Dicant qui redémpti sunt a Dómino, quos redémit de manu inimíci: * et de regiónibus congregávit eos:
}\switchcolumn\portugues{
Digam-no os que foram resgatados pelo Senhor, os que Ele resgatou da mão do inimigo: * e os que congregou de entre as regiões:
}\switchcolumn*\latim{
A solis ortu, et occásu: * ab aquilóne, et mari.
}\switchcolumn\portugues{
Do oriente e do poente: * do aquilão e do mar.
}\switchcolumn*\latim{
Erravérunt in solitúdine in inaquóso: * viam civitátis habitáculi non invenérunt.
}\switchcolumn\portugues{
Erravam por lugares áridos: * não encontraram caminho para uma cidade habitável.
}\switchcolumn*\latim{
Esuriéntes, et sitiéntes: * ánima eórum in ipsis defécit.
}\switchcolumn\portugues{
Padecendo fome e sede: * desfaleceu a sua alma.
}\switchcolumn*\latim{
Et clamavérunt ad Dóminum cum tribularéntur: * et de necessitátibus eórum erípuit eos.
}\switchcolumn\portugues{
Clamaram ao Senhor no meio das suas tribulações: * e Ele os livrou das suas necessidades.
}\switchcolumn*\latim{
Et dedúxit eos in viam rectam: * ut irent in civitátem habitatiónis.
}\switchcolumn\portugues{
Conduziu-os por caminho recto: * para que fossem à cidade de habitação.
}\switchcolumn*\latim{
Confiteántur Dómino misericórdiæ ejus: * et mirabília ejus fíliis hóminum.
}\switchcolumn\portugues{
Glorifiquem o Senhor as suas misericórdias: * e suas maravilhas aos filhos dos homens.
}\switchcolumn*\latim{
Quia satiávit ánimam inánem: * et ánimam esuriéntem satiávit bonis.
}\switchcolumn\portugues{
Pois saciou a alma que estava exausta: * e encheu de bens a alma faminta.
}\switchcolumn*\latim{
Sedéntes in ténebris, et umbra mortis: * vinctos in mendicitáte et ferro.
}\switchcolumn\portugues{
Estavam sentados nas trevas e na sombra da morte: * aprisionados, na mendiguez e em ferros.
}\switchcolumn*\latim{
Quia exacerbavérunt elóquia Dei: * et consílium Altíssimi irritavérunt.
}\switchcolumn\portugues{
Pois exacerbaram as palavras de Deus: * e tinham desprezado o conselho do Altíssimo.
}\switchcolumn*\latim{
Et humiliátum est in labóribus cor eórum: * infirmáti sunt, nec fuit qui adjuváret.
}\switchcolumn\portugues{
Seu coração foi humilhado em trabalhos: * ficaram sem forças, não houve quem os ajudasse.
}\switchcolumn*\latim{
Et clamavérunt ad Dóminum cum tribularéntur: * et de necessitátibus eórum liberávit eos.
}\switchcolumn\portugues{
Clamaram ao Senhor no meio das suas tribulações: * e Ele os livrou de suas necessidades.
}\switchcolumn*\latim{
Et edúxit eos de ténebris, et umbra mortis: * et víncula eórum disrúpit.
}\switchcolumn\portugues{
Tirou-os das trevas e da sombra da morte: * e quebrou os seus vínculos.
}\switchcolumn*\latim{
Confiteántur Dómino misericórdiæ ejus: * et mirabília ejus fíliis hóminum.
}\switchcolumn\portugues{
Glorifiquem o Senhor as suas misericórdias: * e suas maravilhas aos filhos dos homens.
}\switchcolumn*\latim{
Quia contrívit portas ǽreas: * et vectes férreos confrégit.
}\switchcolumn\portugues{
Pois arrombou as portas de bronze: * e quebrou os ferrolhos de ferro.
}\switchcolumn*\latim{
Suscépit eos de via iniquitátis eórum: * propter injustítias enim suas humiliáti sunt.
}\switchcolumn\portugues{
Retirou-os do caminho da sua iniquidade: * pois tinham sido humilhados devido às suas injustiças.
}\switchcolumn*\latim{
Omnem escam abomináta est ánima eórum: * et appropinquavérunt usque ad portas mortis.
}\switchcolumn\portugues{
 Sua alma abominava toda a carne: * e chegaram até às portas da morte.
}\switchcolumn*\latim{
Et clamavérunt ad Dóminum cum tribularéntur: * et de necessitátibus eórum liberávit eos.
}\switchcolumn\portugues{
Clamaram ao Senhor no meio das suas tribulações: * e Ele livrou-os das suas necessidades.
}\switchcolumn*\latim{
Misit verbum suum, et sanávit eos: * et erípuit eos de interitiónibus eórum.
}\switchcolumn\portugues{
Enviou a sua palavra e sarou-os: * e livrou-os da destruição.
}\switchcolumn*\latim{
Confiteántur Dómino misericórdiæ ejus: * et mirabília ejus fíliis hóminum.
}\switchcolumn\portugues{
Glorifiquem o Senhor as suas misericórdias: * e suas maravilhas aos filhos dos homens.
}\switchcolumn*\latim{
Et sacríficent sacrifícium laudis: * et annúntient ópera ejus in exsultatióne.
}\switchcolumn\portugues{
Ofereçam-Lhe um sacrifício de louvor: * e anunciem as suas obras com alegria.
}\switchcolumn*\latim{
Qui descéndunt mare in návibus, * faciéntes operatiónem in aquis multis.
}\switchcolumn\portugues{
Os que descem ao mar em naus, * e fazem as suas manobras nas muitas águas.
}\switchcolumn*\latim{
Ipsi vidérunt ópera Dómini, * et mirabília ejus in profúndo.
}\switchcolumn\portugues{
Viram as obras do Senhor, * e as suas maravilhas no profundo.
}\switchcolumn*\latim{
Dixit, et stetit spíritus procéllæ: * et exaltáti sunt fluctus ejus.
}\switchcolumn\portugues{
Disse e levantou-se um vento de tempestade: * e empolaram-se as ondas.
}\switchcolumn*\latim{
Ascéndunt usque ad cælos, et descéndunt usque ad abýssos: * ánima eórum in malis tabescébat.
}\switchcolumn\portugues{
Sobem até aos céus e descem até aos abysmos: * desfalecia com males a alma deles.
}\switchcolumn*\latim{
Turbáti sunt, et moti sunt sicut ébrius: * et omnis sapiéntia eórum devoráta est.
}\switchcolumn\portugues{
Foram turvados e cambalearam como um embriagado: * e toda sua sabedoria se desvaneceu.
}\switchcolumn*\latim{
Et clamavérunt ad Dóminum cum tribularéntur: * et de necessitátibus eórum edúxit eos.
}\switchcolumn\portugues{
Clamaram ao Senhor no meio das suas tribulações: * e livrou-os das suas necessidades.
}\switchcolumn*\latim{
Et státuit procéllam ejus in auram: * et siluérunt fluctus ejus.
}\switchcolumn\portugues{
Transformou a tempestade em brisa: * e as ondas do mar acalmaram.
}\switchcolumn*\latim{
Et lætáti sunt quia siluérunt: * et dedúxit eos in portum voluntátis eórum.
}\switchcolumn\portugues{
Eles alegraram-se, pois ficou calmo: * e Ele conduziu-os ao porto que desejavam.
}\switchcolumn*\latim{
Confiteántur Dómino misericórdiæ ejus: * et mirabília ejus fíliis hóminum.
}\switchcolumn\portugues{
As suas misericórdias glorifiquem o Senhor : * e suas maravilhas os filhos dos homens.
}\switchcolumn*\latim{
Et exáltent eum in ecclésia plebis: * et in cáthedra seniórum laudent eum.
}\switchcolumn\portugues{
Exaltem-n’O na igreja do povo: * e louvem-n’O na cadeira dos anciãos.
}\switchcolumn*\latim{
Pósuit flúmina in desértum: * et éxitus aquárum in sitim.
}\switchcolumn\portugues{
Converteu os rios em desertos: * e os mananciais das águas em terra sedenta.
}\switchcolumn*\latim{
Terram fructíferam in salsúginem: * a malítia inhabitántium in ea.
}\switchcolumn\portugues{
A terra frutífera em deserto de sal: * por causa da malícia dos seus habitantes.
}\switchcolumn*\latim{
Pósuit desértum in stagna aquárum: * et terram sine aqua in éxitus aquárum.
}\switchcolumn\portugues{
Virou o deserto em tanques de água: * e a terra árida em mananciais de águas.
}\switchcolumn*\latim{
Et collocávit illic esuriéntes: * et constituérunt civitátem habitatiónis.
}\switchcolumn\portugues{
Estabeleceu ali os famintos: * e eles fundaram cidades para habitação.
}\switchcolumn*\latim{
Et seminavérunt agros, et plantavérunt víneas: * et fecérunt fructum nativitátis.
}\switchcolumn\portugues{
Semearam os campos e plantaram vinhas: * e colheram nativos frutos.
}\switchcolumn*\latim{
Et benedíxit eis, et multiplicáti sunt nimis: * et juménta eórum non minorávit.
}\switchcolumn\portugues{
Abençoou-os e multiplicaram-se muitíssimo: * e não diminuiu os seus animais.
}\switchcolumn*\latim{
Et pauci facti sunt: * et vexáti sunt a tribulatióne malórum, et dolóre.
}\switchcolumn\portugues{
Foram depois reduzidos a um pequeno número: * e foram oprimidos com males e dores.
}\switchcolumn*\latim{
Effúsa est contémptio super príncipes: * et erráre fecit eos in ínvio, et non in via.
}\switchcolumn\portugues{
Caiu o desprezo sobre os príncipes: * e Ele fê-los andar em erro por onde caminho não existia.
}\switchcolumn*\latim{
Et adjúvit páuperem de inópia: * et pósuit sicut oves famílias.
}\switchcolumn\portugues{
Aliviou o pobre da sua miséria: * e multiplicou as famílias como ovelhas.
}\switchcolumn*\latim{
Vidébunt recti, et lætabúntur: * et omnis iníquitas oppilábit os suum.
}\switchcolumn\portugues{
Os justos verão e alegrar-se-ão: * e toda a iniquidade fechará a boca.
}\switchcolumn*\latim{
Quis sápiens et custódiet hæc? * Et intélleget misericórdias Dómini.
}\switchcolumn\portugues{
Quem é sábio para conservar estas cousas * e compreender as misericórdias do Senhor?
}\end{paracol}


\subsectioninfo{Salmo 107}{Paratum cor meum}\label{salmo107}
\begin{paracol}{2}\latim{
\rlettrine{P}{arátum} cor meum, Deus, parátum cor meum: * cantábo, et psallam in glória mea.
}\switchcolumn\portugues{
\rlettrine{P}{ronto} está o meu coração, ó Deus, pronto está o meu coração: * cantarei e salmodiarei na minha glória.
}\switchcolumn*\latim{
Exsúrge, glória mea, exsúrge, psaltérium et cíthara: * exsúrgam dilúculo.
}\switchcolumn\portugues{
Desperta, ó glória minha, desperta, saltério e cítara: * levantar-me-ei ao romper da alva.
}\switchcolumn*\latim{
Confitébor tibi in pópulis, Dómine: * et psallam tibi in natiónibus.
}\switchcolumn\portugues{
Louvar-Vos-ei no meio dos povos, ó Senhor: * e entoar-Vos-ei salmos entre as nações.
}\switchcolumn*\latim{
Quia magna est super cælos misericórdia tua: * et usque ad nubes véritas tua:
}\switchcolumn\portugues{
Pois a vossa misericórdia elevou-se acima dos céus: * e a vossa verdade até às nuvens:
}\switchcolumn*\latim{
Exaltáre super cælos, Deus, et super omnem terram glória tua: * ut liberéntur dilécti tui.
}\switchcolumn\portugues{
Exaltai-Vos, ó Deus, sobre os céus, sobre toda a terra a vossa glória: * para que sejam livres os vossos eleitos.
}\switchcolumn*\latim{
Salvum fac déxtera tua, et exáudi me: * Deus locútus est in sancto suo:
}\switchcolumn\portugues{
Salvai-me com vossa direita e ouvi-me: * Deus falou no seu santuário:
}\switchcolumn*\latim{
Exsultábo, et dívidam Síchimam, * et convállem tabernaculórum dimétiar.
}\switchcolumn\portugues{
Alegrar-me-ei e repartirei Siquém, * e medirei o vale dos Tabernáculos.
}\switchcolumn*\latim{
Meus est Gálaad, et meus est Manásses: * et Éphraim suscéptio cápitis mei.
}\switchcolumn\portugues{
Meu é Galaad e meu é Manassés: * e Efraim é a segurança da minha cabeça.
}\switchcolumn*\latim{
Juda rex meus: * Moab lebes spei meæ.
}\switchcolumn\portugues{
Judá é o meu rei: * o Moab a bacia da minha esperança.
}\switchcolumn*\latim{
In Idumǽam exténdam calceaméntum meum: * mihi alienígenæ amíci facti sunt.
}\switchcolumn\portugues{
Estenderei o meu calçado sobre a Idumeia: * os estrangeiros tornaram-se meus amigos.
}\switchcolumn*\latim{
Quis dedúcet me in civitátem munítam? * Quis dedúcet me usque in Idumǽam?
}\switchcolumn\portugues{
Quem me conduzirá à cidade fortificada? * Quem me conduzirá até à Idumeia?
}\switchcolumn*\latim{
Nonne Tu, Deus, qui repulísti nos, * et non exíbis, Deus, in virtútibus nostris?
}\switchcolumn\portugues{
Porventura não sois Vós, Deus, que nos desamparastes, * não vireis Vós, Deus, com os nossos exércitos?
}\switchcolumn*\latim{
Da nobis auxílium de tribulatióne: * quia vana salus hóminis.
}\switchcolumn\portugues{
Dai-nos socorro na tribulação: * pois vã é a ajuda do homem.
}\switchcolumn*\latim{
In Deo faciémus virtútem: * et ipse ad níhilum dedúcet inimícos nostros.
}\switchcolumn\portugues{
Em Deus faremos proezas: * e Ele reduzirá os nossos inimigos a nada.
}\end{paracol}


\subsectioninfo{Salmo 108}{Deus, laudem meam}\label{salmo108}
\begin{paracol}{2}\latim{
\rlettrine{D}{eus,} laudem meam ne tacúeris: * quia os peccatóris, et os dolósi super me apértum est.
}\switchcolumn\portugues{
\rlettrine{D}{eus,} Vos não caleis ao meu louvor: * porque abriram-se contra mim a boca do pecador e do traidor.
}\switchcolumn*\latim{
Locúti sunt advérsum me lingua dolósa, et sermónibus ódii circumdedérunt me: * et expugnavérunt me gratis.
}\switchcolumn\portugues{
Falaram contra mim com língua dolosa, me cercaram com palavras de ódio: * e gratuitamente me expugnaram.
}\switchcolumn*\latim{
Pro eo ut me dilígerent, detrahébant mihi: * ego autem orábam.
}\switchcolumn\portugues{
Em vez de me amar, me caluniavam: * eu, porém, orava.
}\switchcolumn*\latim{
Et posuérunt advérsum me mala pro bonis: * et ódium pro dilectióne mea.
}\switchcolumn\portugues{
Me deram males por bens: * e ódio em troca do amor que lhes tinha.
}\switchcolumn*\latim{
Constítue super eum peccatórem: * et diábolus stet a dextris ejus.
}\switchcolumn\portugues{
Sujeitai-o ao domínio do pecador: * e o demónio esteja à sua direita.
}\switchcolumn*\latim{
Cum judicátur, éxeat condemnátus: * et orátio ejus fiat in peccátum.
}\switchcolumn\portugues{
Quando for julgado, saia condenado: * e a sua oração se converta em pecado.
}\switchcolumn*\latim{
Fiant dies ejus pauci: * et episcopátum ejus accípiat alter.
}\switchcolumn\portugues{
Sejam abreviados os seus dias: * e receba outro seu bispado.
}\switchcolumn*\latim{
Fiant fílii ejus órphani: * et uxor ejus vídua.
}\switchcolumn\portugues{
Fiquem seus filhos órfãos: * e sua mulher viúva.
}\switchcolumn*\latim{
Nutántes transferántur fílii ejus, et mendícent: * et eiciántur de habitatiónibus suis.
}\switchcolumn\portugues{
Andem vagabundos dum lugar para outro os seus filhos e mendiguem: * e sejam lançados fora das suas habitações.
}\switchcolumn*\latim{
Scrutétur fænerátor omnem substántiam ejus: * et dirípiant aliéni labóres ejus.
}\switchcolumn\portugues{
O usurário dê caça a todos seus bens: * e os estranhos roubem os seus trabalhos.
}\switchcolumn*\latim{
Non sit illi adjútor: * nec sit qui misereátur pupíllis ejus.
}\switchcolumn\portugues{
Não tenha quem o ajude: * nem haja quem se compadeça dos seus órfãos.
}\switchcolumn*\latim{
Fiant nati ejus in intéritum: * in generatióne una deleátur nomen ejus.
}\switchcolumn\portugues{
Sejam exterminados todos seus filhos: * em uma só geração fique apagado o seu nome.
}\switchcolumn*\latim{
In memóriam rédeat iníquitas patrum ejus in conspéctu Dómini: * et peccátum matris ejus non deleátur.
}\switchcolumn\portugues{
Reviva a lembrança da iniquidade de seus pais na presença do Senhor: * e o pecado de sua mãe não seja apagado.
}\switchcolumn*\latim{
Fiant contra Dóminum semper, et dispéreat de terra memória eórum: * pro eo quod non est recordátus fácere misericórdiam.
}\switchcolumn\portugues{
Estejam sempre diante do Senhor e desapareça da terra a sua memória: * porque se não lembrou de usar de misericórdia.
}\switchcolumn*\latim{
Et persecútus est hóminem ínopem, et mendícum, * et compúnctum corde mortificáre.
}\switchcolumn\portugues{
Perseguiu o homem desamparado e mendigo, * o homem aflito do coração, para lhe dar a morte.
}\switchcolumn*\latim{
Et diléxit maledictiónem, et véniet ei: * et nóluit benedictiónem, et elongábitur ab eo.
}\switchcolumn\portugues{
E, como amou a maldição, ela lhe virá: * e, como não quis a bênção, ela afastar-se-á dele.
}\switchcolumn*\latim{
Et índuit maledictiónem sicut vestiméntum, * et intrávit sicut aqua in interióra ejus, et sicut óleum in óssibus ejus.
}\switchcolumn\portugues{
Vestiu-se de maldição como um vestido, * e ela penetrou como água nas suas entranhas e como azeite nos seus ossos.
}\switchcolumn*\latim{
Fiat ei sicut vestiméntum, quo operítur: * et sicut zona, qua semper præcíngitur.
}\switchcolumn\portugues{
Que ela seja para ele como o vestido com que se cobre: * e como a cinta com que sempre se cinge.
}\switchcolumn*\latim{
Hoc opus eórum, qui détrahunt mihi apud Dóminum: * et qui loquúntur mala advérsus ánimam meam.
}\switchcolumn\portugues{
Tal é diante do Senhor a obra daqueles que me caluniam: * e que dizem males contra a minha alma.
}\switchcolumn*\latim{
Et tu, Dómine, Dómine, fac mecum propter nomen tuum: * quia suávis est misericórdia tua.
}\switchcolumn\portugues{
Vós, ó Senhor, fazei comigo de acordo com vosso nome: * pois é suave a vossa misericórdia.
}\switchcolumn*\latim{
Líbera me quia egénus, et pauper ego sum: * et cor meum conturbátum est intra me.
}\switchcolumn\portugues{
Livrai-me, pois sou necessitado e pobre: * e o meu coração abalado está dentro de mim.
}\switchcolumn*\latim{
Sicut umbra cum declínat, ablátus sum: * et excússus sum sicut locústæ.
}\switchcolumn\portugues{
Desapareço como a sombra que vai caindo: * e sou escorraçado como os gafanhotos.
}\switchcolumn*\latim{
Génua mea infirmáta sunt a jejúnio: * et caro mea immutáta est propter óleum.
}\switchcolumn\portugues{
Meus joelhos enfraqueceram com o jejum: * e a minha carne mudou por falta de azeite.
}\switchcolumn*\latim{
Et ego factus sum oppróbrium illis: * vidérunt me, et movérunt cápita sua.
}\switchcolumn\portugues{
Tornei-me para eles um objecto de escárnio: * me viram e abanaram as suas cabeças.
}\switchcolumn*\latim{
Ádjuva me, Dómine, Deus meus: * salvum me fac secúndum misericórdiam tuam.
}\switchcolumn\portugues{
Assisti-me, ó Senhor meu Deus: * salvai-me segundo a vossa misericórdia.
}\switchcolumn*\latim{
Et sciant quia manus tua hæc: * et tu, Dómine, fecísti eam.
}\switchcolumn\portugues{
Saibam que isto é de vossa mão: * e que Vós, ó Senhor, tendes feito estas cousas.
}\switchcolumn*\latim{
Maledícent illi, et Tu benedíces: * qui insúrgunt in me, confundántur: servus autem tuus lætábitur.
}\switchcolumn\portugues{
Eles me amaldiçoaram e Vós me abençoareis: * confundidos sejam os que se levantam contra mim, entretanto o vosso servo alegrar-se-á.
}\switchcolumn*\latim{
Induántur qui détrahunt mihi, pudóre: * et operiántur sicut diplóide confusióne sua.
}\switchcolumn\portugues{
Sejam cobertos de afronta os que me caluniam: * e fiquem envolvidos na sua confusão como numa capa dupla.
}\switchcolumn*\latim{
Confitébor Dómino nimis in ore meo: * et in médio multórum laudábo eum.
}\switchcolumn\portugues{
Muito glorificarei o Senhor com minha boca: * e no meio de muitos o louvarei.
}\switchcolumn*\latim{
Quia ástitit a dextris páuperis, * ut salvam fáceret a persequéntibus ánimam meam.
}\switchcolumn\portugues{
Pois se pôs à direita deste pobre, * para salvar a sua vida daqueles que a perseguem.
}\end{paracol}


\subsectioninfo{Salmo 109}{Dixit Dominus Domino meo}\label{salmo109}
\begin{paracol}{2}\latim{
\rlettrine{D}{ixit} Dóminus Dómino meo: * Sede a dextris meis:
}\switchcolumn\portugues{
\rlettrine{D}{isse} o Senhor ao meu senhor: * senta-te à minha direita:
}\switchcolumn*\latim{
Donec ponam inimícos tuos, * scabéllum pedum tuórum.
}\switchcolumn\portugues{
Até que ponha os teus inimigos, * por escabelo de teus pés.
}\switchcolumn*\latim{
Virgam virtútis tuæ emíttet Dóminus ex Sion: * domináre in médio inimicórum tuórum.
}\switchcolumn\portugues{
O Senhor fará sair de Sião o ceptro de teu poder: * domina tu no meio de teus inimigos.
}\switchcolumn*\latim{
Tecum princípium in die virtútis tuæ in splendóribus sanctórum: * ex útero ante lucíferum génui te.
}\switchcolumn\portugues{
Contigo está o principado no dia de tua força, entre os resplendores dos santos: * das minhas entranhas te gerei antes da aurora.
}\switchcolumn*\latim{
Jurávit Dóminus, et non pœnitébit eum: * Tu es sacérdos in ætérnum secúndum órdinem Melchísedech.
}\switchcolumn\portugues{
Jurou o Senhor e se não arrependerá: * tu és sacerdote eternamente, segundo a ordem de Melquisedech.
}\switchcolumn*\latim{
Dóminus a dextris tuis, * confrégit in die iræ suæ reges.
}\switchcolumn\portugues{
O Senhor está à tua direita, * Ele despedaçou os reis no dia da sua ira.
}\switchcolumn*\latim{
Judicábit in natiónibus, implébit ruínas: * conquassábit cápita in terra multórum.
}\switchcolumn\portugues{
Ajuizará no meio das nações, encherá tudo de ruínas: * esmagará as cabeças de muitos na terra.
}\switchcolumn*\latim{
De torrénte in via bibet: * proptérea exaltábit caput.
}\switchcolumn\portugues{
Beberá da torrente no caminho: * por isso erguerá a sua cabeça.
}\end{paracol}


\subsectioninfo{Salmo 110}{Confitebor tibi, Domine}\label{salmo110}
\begin{paracol}{2}\latim{
\rlettrine{C}{onfitébor} tibi, Dómine, in toto corde meo: * in consílio justórum, et congregatióne.
}\switchcolumn\portugues{
\rlettrine{L}{ouvar-Vos-ei,} ó Senhor, com todo meu coração: * no conselho e na congregação dos justos.
}\switchcolumn*\latim{
Magna ópera Dómini: * exquisíta in omnes voluntátes ejus.
}\switchcolumn\portugues{
Grandes são as obras do Senhor: * apropriadas a todas suas vontades.
}\switchcolumn*\latim{
Conféssio et magnificéntia opus ejus: * et justítia ejus manet in sǽculum sǽculi.
}\switchcolumn\portugues{
Sua obra é glória e magnificência: * e a sua justiça permanece pelos séculos dos séculos.
}\switchcolumn*\latim{
Memóriam fecit mirabílium suórum, miséricors et miserátor Dóminus: * escam dedit timéntibus se.
}\switchcolumn\portugues{
Instituiu um memorial das suas maravilhas, o Senhor que é misericordioso e compassivo: * deu alimento aos que O temem.
}\switchcolumn*\latim{
Memor erit in sǽculum testaménti sui: * virtútem óperum suórum annuntiábit pópulo suo:
}\switchcolumn\portugues{
Lembrar-se-á eternamente da sua aliança: * anunciará ao seu povo o poder das suas obras:
}\switchcolumn*\latim{
Ut det illis hereditátem géntium: * ópera mánuum ejus véritas, et judícium.
}\switchcolumn\portugues{
Dando-lhe a herança das gentes: * as obras das suas mãos são verdade e justiça.
}\switchcolumn*\latim{
Fidélia ómnia mandáta ejus: confirmáta in sǽculum sǽculi, * facta in veritáte et æquitáte.
}\switchcolumn\portugues{
Fiéis são todos seus mandamentos, confirmados em todos os séculos, * feitos em verdade e equidade.
}\switchcolumn*\latim{
Redemptiónem misit pópulo suo: * mandávit in ætérnum testaméntum suum.
}\switchcolumn\portugues{
Enviou a redenção ao seu povo: * estabeleceu para sempre a sua aliança.
}\switchcolumn*\latim{
\emph{(fit reverentia)} Sanctum, et terríbile nomen ejus: * inítium sapiéntiæ timor Dómini.
}\switchcolumn\portugues{
\emph{(inclinar a cabeça)} Santo e terrível é o seu nome: * o temor do Senhor é o princípio da sabedoria.
}\switchcolumn*\latim{
Intelléctus bonus ómnibus faciéntibus eum: * laudátio ejus manet in sǽculum sǽculi.
}\switchcolumn\portugues{
São sábios todos os que o praticam: * seu louvor permanece para sempre.
}\end{paracol}


\subsectioninfo{Salmo 111}{Beatus vir qui timet Dominum}\label{salmo111}
\begin{paracol}{2}\latim{
\rlettrine{B}{eátus} vir, qui timet Dóminum: * in mandátis ejus volet nimis.
}\switchcolumn\portugues{
\rlettrine{B}{em-aventurado} o varão que teme o Senhor: * muito se deliciará nos seus mandamentos.
}\switchcolumn*\latim{
Potens in terra erit semen ejus: * generátio rectórum benedicétur.
}\switchcolumn\portugues{
Poderosa será a sua semente sobre a terra: * bendita será a geração dos justos.
}\switchcolumn*\latim{
Glória, et divítiæ in domo ejus: * et justítia ejus manet in sǽculum sǽculi.
}\switchcolumn\portugues{
Haverá glória e riqueza na sua casa: * e a sua justiça permanece por todos os séculos.
}\switchcolumn*\latim{
Exórtum est in ténebris lumen rectis: * miséricors, et miserátor, et justus.
}\switchcolumn\portugues{
Nas trevas surgiu uma luz para os rectos: * ele é misericordioso, compassivo e justo.
}\switchcolumn*\latim{
Jucúndus homo qui miserétur et cómmodat, dispónet sermónes suos in judício: * quia in ætérnum non commovébitur.
}\switchcolumn\portugues{
Ditoso o homem que se compadece e empresta, ele disporá os seus discursos com juízo: * pois nunca será abalado.
}\switchcolumn*\latim{
In memória ætérna erit justus: * ab auditióne mala non timébit.
}\switchcolumn\portugues{
A memória do justo será eterna: * não temerá ouvir notícias funestas.
}\switchcolumn*\latim{
Parátum cor ejus speráre in Dómino, confirmátum est cor ejus: * non commovébitur donec despíciat inimícos suos.
}\switchcolumn\portugues{
Seu coração está disposto a esperar no Senhor, fortalecido está o seu coração: * não será abalado até que observe os seus inimigos.
}\switchcolumn*\latim{
Dispérsit, dedit paupéribus: justítia ejus manet in sǽculum sǽculi, * cornu ejus exaltábitur in glória.
}\switchcolumn\portugues{
Distribuiu, deu aos pobres: a sua justiça permanece por todos os séculos, * o seu poder será exaltado em glória.
}\switchcolumn*\latim{
Peccátor vidébit, et irascétur, déntibus suis fremet et tabéscet: * desidérium peccatórum períbit.
}\switchcolumn\portugues{
Vê-lo-á o pecador e indignar-se-á, rangerá os dentes e dissipar-se-á: * o desejo dos pecadores perecerá.
}\end{paracol}


\subsectioninfo{Salmo 112}{Laudate, pueri}\label{salmo112}
\begin{paracol}{2}\latim{
\rlettrine{L}{audáte,} púeri, Dóminum: * laudáte nomen Dómini.
}\switchcolumn\portugues{
\rlettrine{L}{ouvai} o Senhor, ó meninos: * louvai o nome do Senhor.
}\switchcolumn*\latim{
\emph{(fit reverentia)} Sit nomen Dómini benedíctum, * ex hoc nunc, et usque in sǽculum.
}\switchcolumn\portugues{
\emph{(inclinar a cabeça)} Seja bendito o nome do Senhor, * desde agora e para sempre.
}\switchcolumn*\latim{
A solis ortu usque ad occásum, * laudábile nomen Dómini.
}\switchcolumn\portugues{
Desde o nascer ao pôr do sol, * é digno de louvor o nome do Senhor.
}\switchcolumn*\latim{
Excélsus super omnes gentes Dóminus, * et super cælos glória ejus.
}\switchcolumn\portugues{
Excelso é o Senhor sobre todas as gentes, * e a sua glória sobre os céus.
}\switchcolumn*\latim{
Quis sicut Dóminus, Deus noster, qui in altis hábitat, * et humília réspicit in cælo et in terra?
}\switchcolumn\portugues{
Quem há como o Senhor nosso Deus, que habita nas alturas: * e atende os humildes no céu e na terra?
}\switchcolumn*\latim{
Súscitans a terra ínopem, * et de stércore érigens páuperem:
}\switchcolumn\portugues{
Levantando da terra o desvalido, * e tirando da imundície o pobre:
}\switchcolumn*\latim{
Ut cóllocet eum cum princípibus, * cum princípibus pópuli sui.
}\switchcolumn\portugues{
Para o colocar com os príncipes, * com os príncipes do seu povo.
}\switchcolumn*\latim{
Qui habitáre facit stérilem in domo, * matrem filiórum lætántem.
}\switchcolumn\portugues{
Que faz a mulher estéril viver em sua casa, * alegre mãe de filhos.
}\end{paracol}


\subsubsectioninfo{Salmo 113}{In exitu Israël}\label{salmo113}
\begin{paracol}{2}\latim{
\rlettrine{I}{n} éxitu Israël de Ægýpto, * domus Jacob de pópulo bárbaro:
}\switchcolumn\portugues{
\qlettrine{Q}{uando} Israel saiu do Egipto, * e a casa de Jacob de um povo bárbaro:
}\switchcolumn*\latim{
Facta est Judǽa sanctificátio ejus, * Israël potéstas ejus.
}\switchcolumn\portugues{
Judá foi feito seu santuário, * e Israel o seu domínio.
}\switchcolumn*\latim{
Mare vidit, et fugit: * Jordánis convérsus est retrórsum.
}\switchcolumn\portugues{
O mar viu e fugiu: * o Jordão voltou atrás.
}\switchcolumn*\latim{
Montes exsultavérunt ut aríetes, * et colles sicut agni óvium.
}\switchcolumn\portugues{
Os montes saltaram como carneiros, * e as colinas como cordeiros do rebanho.
}\switchcolumn*\latim{
Quid est tibi, mare, quod fugísti: * et tu, Jordánis, quia convérsus es retrórsum?
}\switchcolumn\portugues{
Que tiveste tu, ó mar, para fugir: * e tu, Jordão, para retroceder?
}\switchcolumn*\latim{
Montes, exsultástis sicut aríetes, * et colles, sicut agni óvium.
}\switchcolumn\portugues{
Ó montes, porque saltastes como carneiros, * e vós, colinas, como cordeiros?
}\switchcolumn*\latim{
A fácie Dómini mota est terra, * a fácie Dei Jacob.
}\switchcolumn\portugues{
Comoveu-se a terra na presença do Senhor, * perante o Deus de Jacob.
}\switchcolumn*\latim{
Qui convértit petram in stagna aquárum, * et rupem in fontes aquárum.
}\switchcolumn\portugues{
Que converteu as pedras em tanques de águas, * e a rocha em fontes de águas.
}\switchcolumn*\latim{
Non nobis, Dómine, non nobis: * sed nómini tuo da glóriam.
}\switchcolumn\portugues{
Não a nós, ó Senhor, não a nós: * mas ao vosso nome dai glória.
}\switchcolumn*\latim{
Super misericórdia tua, et veritáte tua: * nequándo dicant gentes: Ubi est Deus eórum?
}\switchcolumn\portugues{
Pela vossa misericórdia e a vossa verdade: * para que nunca digam as gentes: o seu Deus onde está?
}\switchcolumn*\latim{
Deus autem noster in cælo: * ómnia quæcúmque vóluit, fecit.
}\switchcolumn\portugues{
Nosso Deus está no céu: * tudo quanto quis, Ele o fez.
}\switchcolumn*\latim{
Simulácra géntium argéntum, et aurum, * ópera mánuum hóminum.
}\switchcolumn\portugues{
Os ídolos das gentes são prata e oiro, * obras das mãos dos homens.
}\switchcolumn*\latim{
Os habent, et non loquéntur: * óculos habent, et non vidébunt.
}\switchcolumn\portugues{
Têm boca e não falam: * têm olhos e não vêem.
}\switchcolumn*\latim{
Aures habent, et non áudient: * nares habent, et non odorábunt.
}\switchcolumn\portugues{
Têm ouvidos e não ouvem: * têm narizes e não cheiram.
}\switchcolumn*\latim{
Manus habent, et non palpábunt: pedes habent, et non ambulábunt: * non clamábunt in gútture suo.
}\switchcolumn\portugues{
Têm mãos e não apalpam: têm pés e não andam: * não clamam com sua garganta.
}\switchcolumn*\latim{
Símiles illis fiant qui fáciunt ea: * et omnes qui confídunt in eis.
}\switchcolumn\portugues{
Sejam semelhantes a eles os que os fazem: * e todos os que confiam neles.
}\switchcolumn*\latim{
Domus Israël sperávit in Dómino: * adjútor eórum et protéctor eórum est,
}\switchcolumn\portugues{
A casa de Israel esperou no Senhor: * Ele é o seu amparo e o seu protector.
}\switchcolumn*\latim{
Domus Aaron sperávit in Dómino: * adjútor eórum et protéctor eórum est,
}\switchcolumn\portugues{
A casa de Arão esperou no Senhor: * Ele é o seu amparo e o seu protector.
}\switchcolumn*\latim{
Qui timent Dóminum, speravérunt in Dómino: * adjútor eórum et protéctor eórum est.
}\switchcolumn\portugues{
Os que temem o Senhor, esperarão no Senhor: * Ele é o seu amparo e o seu protector.
}\switchcolumn*\latim{
Dóminus memor fuit nostri: * et benedíxit nobis:
}\switchcolumn\portugues{
O Senhor lembrou-se de nós: * e abençoou-nos:
}\switchcolumn*\latim{
Benedíxit dómui Israël: * benedíxit dómui Aaron.
}\switchcolumn\portugues{
Abençoou a casa de Israel: * abençoou a casa de Arão.
}\switchcolumn*\latim{
Benedíxit ómnibus, qui timent Dóminum, * pusíllis cum majóribus.
}\switchcolumn\portugues{
Abençoou todos os que temem o Senhor, * os pequenos e os grandes.
}\switchcolumn*\latim{
Adíciat Dóminus super vos: * super vos, et super fílios vestros.
}\switchcolumn\portugues{
Aumente o Senhor sobre vós: * sobre vós e sobre vossos filhos.
}\switchcolumn*\latim{
Benedícti vos a Dómino, * qui fecit cælum, et terram.
}\switchcolumn\portugues{
Sede benditos do Senhor, * que fez o céu e a terra.
}\switchcolumn*\latim{
Cælum cæli Dómino: * terram autem dedit fíliis hóminum.
}\switchcolumn\portugues{
O mais alto dos céus é para o Senhor: * mas a terra deu-a aos filhos dos homens.
}\switchcolumn*\latim{
Non mórtui laudábunt te, Dómine: * neque omnes, qui descéndunt in inférnum.
}\switchcolumn\portugues{
Os mortos, ó Senhor, Vos não louvarão: * nem nenhum dos que descem ao inferno.
}\switchcolumn*\latim{
Sed nos qui vívimus, benedícimus Dómino, * ex hoc nunc et usque in sǽculum.
}\switchcolumn\portugues{
Mas nós, que vivemos, nós bendizemos o Senhor, * desde agora e por todos os séculos.
}\end{paracol}


\subsectioninfo{Salmo 114}{Dilexi, quoniam exaudiet}\label{salmo114}
\begin{paracol}{2}\latim{
\rlettrine{D}{iléxi,} quóniam exáudiet Dóminus * vocem oratiónis meæ.
}\switchcolumn\portugues{
\rlettrine{A}{mei,} porque o Senhor ouvirá * a voz da minha oração.
}\switchcolumn*\latim{
Quia inclinávit aurem suam mihi: * et in diébus meis invocábo.
}\switchcolumn\portugues{
Pois inclinou para mim o seu ouvido: * e O invocarei todos meus dias.
}\switchcolumn*\latim{
Circumdedérunt me dolóres mortis: * et perícula inférni invenérunt me.
}\switchcolumn\portugues{
Dores de morte me cercaram: * e perigos do inferno vieram sobre mim.
}\switchcolumn*\latim{
Tribulatiónem et dolórem invéni: * et nomen Dómini invocávi.
}\switchcolumn\portugues{
Encontrei-me na tribulação e na dor: * e invoquei o nome do Senhor.
}\switchcolumn*\latim{
O Dómine, líbera ánimam meam: * miséricors Dóminus, et justus, et Deus noster miserétur.
}\switchcolumn\portugues{
Ó Senhor, livrai a minha alma: * o Senhor é misericordioso e justo e o nosso Deus é compassivo.
}\switchcolumn*\latim{
Custódiens párvulos Dóminus: * humiliátus sum, et liberávit me.
}\switchcolumn\portugues{
O Senhor é que guarda os pequeninos: * fui humilhado e Ele me livrou.
}\switchcolumn*\latim{
Convértere, ánima mea, in réquiem tuam: * quia Dóminus benefécit tibi.
}\switchcolumn\portugues{
Volta, ó minha alma, ao teu repouso: * pois o Senhor te cumulou de bens.
}\switchcolumn*\latim{
Quia erípuit ánimam meam de morte: * óculos meos a lácrimis, pedes meos a lapsu.
}\switchcolumn\portugues{
Porque livrou da morte a minha alma: * os meus olhos das lágrimas, os meus pés da queda.
}\switchcolumn*\latim{
Placébo Dómino * in regióne vivórum.
}\switchcolumn\portugues{
Agradarei ao Senhor * na região dos vivos.
}\end{paracol}


\subsectioninfo{Salmo 115}{Credidi, propter}\label{salmo115}
\begin{paracol}{2}\latim{
\rlettrine{C}{rédidi,} propter quod locútus sum: * ego autem humiliátus sum nimis.
}\switchcolumn\portugues{
\rlettrine{A}{creditei,} por isso falei: * contudo, fui grandemente humilhado.
}\switchcolumn*\latim{
Ego dixi in excéssu meo: * Omnis homo mendax.
}\switchcolumn\portugues{
Disse eu no meu êxtase: * todo o homem é mentiroso.
}\switchcolumn*\latim{
Quid retríbuam Dómino, * pro ómnibus, quæ retríbuit mihi?
}\switchcolumn\portugues{
Que darei em retribuição ao Senhor, * por tudo que me deu?
}\switchcolumn*\latim{
Cálicem salutáris accípiam: * et nomen Dómini invocábo.
}\switchcolumn\portugues{
Tomarei o cálice da salvação: * e invocarei o nome do Senhor.
}\switchcolumn*\latim{
Vota mea Dómino reddam coram omni pópulo ejus: * pretiósa in conspéctu Dómini mors sanctórum ejus:
}\switchcolumn\portugues{
Cumprirei os meus votos ao Senhor, ante todo seu povo: * é preciosa aos olhos do Senhor a morte dos seus santos:
}\switchcolumn*\latim{
O Dómine, quia ego servus tuus: * ego servus tuus, et fílius ancíllæ tuæ.
}\switchcolumn\portugues{
Ó Senhor, eu sou vosso servo: * eu sou vosso servo e filho de vossa serva.
}\switchcolumn*\latim{
Dirupísti víncula mea: * tibi sacrificábo hóstiam laudis, et nomen Dómini invocábo.
}\switchcolumn\portugues{
Quebrastes as minhas cadeias: * Vos oferecerei uma hóstia de louvor e invocarei o nome do Senhor.
}\switchcolumn*\latim{
Vota mea Dómino reddam in conspéctu omnis pópuli ejus: * in átriis domus Dómini, in médio tui, Jerúsalem.
}\switchcolumn\portugues{
Cumprirei os meus votos ao Senhor ante todo seu povo: * nos átrios da casa do Senhor, no meio de Vós, ó Jerusalém.
}\end{paracol}


\subsectioninfo{Salmo 116}{Laudate Dominum}\label{salmo116}
\begin{paracol}{2}\latim{
\rlettrine{L}{audáte} Dóminum, omnes gentes: * laudáte eum, omnes pópuli:
}\switchcolumn\portugues{
\slettrine{Ó}{} gentes, louvai todas o Senhor: * louvai-O todos, ó povos:
}\switchcolumn*\latim{
Quóniam confirmáta est super nos misericórdia ejus: * et véritas Dómini manet in ætérnum.
}\switchcolumn\portugues{
Porque sobre nós foi confirmada a sua misericórdia: * e a verdade do Senhor permanece eternamente.
}\end{paracol}


\subsectioninfo{Salmo 117}{Confitemini Domino, quoniam bonus, quoniam in sæculum}\label{salmo117}
\begin{paracol}{2}\latim{
\rlettrine{C}{onfitémini} Dómino quóniam bonus: * quóniam in sǽculum misericórdia ejus.
}\switchcolumn\portugues{
\rlettrine{L}{ouvai} o Senhor, porque Ele é bom: * porque a sua misericórdia é eterna.
}\switchcolumn*\latim{
Dicat nunc Israël quóniam bonus: * quóniam in sǽculum misericórdia ejus.
}\switchcolumn\portugues{
Diga agora Israel que o Senhor é bom: * e que sua misericórdia é eterna.
}\switchcolumn*\latim{
Dicat nunc domus Aaron: * quóniam in sǽculum misericórdia ejus.
}\switchcolumn\portugues{
Diga agora a casa de Arão: * que sua misericórdia é eterna.
}\switchcolumn*\latim{
Dicant nunc qui timent Dóminum: * quóniam in sǽculum misericórdia ejus.
}\switchcolumn\portugues{
Digam agora os que temem o Senhor: * que sua misericórdia é eterna.
}\switchcolumn*\latim{
De tribulatióne invocávi Dóminum: * et exaudívit me in latitúdine Dóminus.
}\switchcolumn\portugues{
No meio da tribulação invoquei o Senhor: * e o Senhor me ouviu e me pôs ao largo.
}\switchcolumn*\latim{
Dóminus mihi adjútor: * non timébo quid fáciat mihi homo.
}\switchcolumn\portugues{
O Senhor é o meu amparo: * não temerei o que o homem me possa fazer.
}\switchcolumn*\latim{
Dóminus mihi adjútor: * et ego despíciam inimícos meos.
}\switchcolumn\portugues{
O Senhor é o meu amparo: * e eu desprezarei os meus inimigos.
}\switchcolumn*\latim{
Bonum est confídere in Dómino, * quam confídere in hómine:
}\switchcolumn\portugues{
É melhor confiar no Senhor, * que esperar no homem.
}\switchcolumn*\latim{
Bonum est speráre in Dómino, * quam speráre in princípibus.
}\switchcolumn\portugues{
É melhor confiar no Senhor, * que confiar nos príncipes.
}\switchcolumn*\latim{
Omnes gentes circuiérunt me: * et in nómine Dómini quia ultus sum in eos.
}\switchcolumn\portugues{
Todas as gentes me cercaram: * e no nome do Senhor vinguei-me delas.
}\switchcolumn*\latim{
Circumdántes circumdedérunt me: * et in nómine Dómini quia ultus sum in eos.
}\switchcolumn\portugues{
Pondo-se à minha volta me cercaram: * e no nome do Senhor vinguei-me delas.
}\switchcolumn*\latim{
Circumdedérunt me sicut apes, et exarsérunt sicut ignis in spinis: * et in nómine Dómini quia ultus sum in eos.
}\switchcolumn\portugues{
Cercaram-me como abelhas, incendiaram-se como fogo em espinhos: * e no nome do Senhor vinguei-me delas.
}\switchcolumn*\latim{
Impúlsus evérsus sum ut cáderem: * et Dóminus suscépit me.
}\switchcolumn\portugues{
Empurraram-me para cair: * mas o Senhor me susteve.
}\switchcolumn*\latim{
Fortitúdo mea, et laus mea Dóminus: * et factus est mihi in salútem.
}\switchcolumn\portugues{
O Senhor é a minha fortaleza e o meu louvor: * e tornou-se a minha salvação.
}\switchcolumn*\latim{
Vox exsultatiónis, et salútis * in tabernáculis justórum.
}\switchcolumn\portugues{
Voz de júbilo e de salvação * nas tendas dos justos.
}\switchcolumn*\latim{
Déxtera Dómini fecit virtútem: déxtera Dómini exaltávit me, * déxtera Dómini fecit virtútem.
}\switchcolumn\portugues{
A dextra do Senhor mostrou o seu poder: a dextra do Senhor me ergueu, * a dextra do Senhor mostrou o seu poder.
}\switchcolumn*\latim{
Non móriar, sed vivam: * et narrábo ópera Dómini.
}\switchcolumn\portugues{
Não morrerei, mas viverei: * e narrarei as obras do Senhor.
}\switchcolumn*\latim{
Castígans castigávit me Dóminus: * et morti non trádidit me.
}\switchcolumn\portugues{
O Senhor castigou-me severamente: * mas me não entregou à morte.
}\switchcolumn*\latim{
Aperíte mihi portas justítiæ, ingréssus in eas confitébor Dómino: * hæc porta Dómini, justi intrábunt in eam.
}\switchcolumn\portugues{
Abri-me as portas da justiça, entrarei por elas e ao Senhor louvarei: * esta é a porta do Senhor, os justos entrarão por ela.
}\switchcolumn*\latim{
Confitébor tibi quóniam exaudísti me: * et factus es mihi in salútem.
}\switchcolumn\portugues{
Vos louvarei porque me ouvistes: * e Vos tornastes a minha salvação.
}\switchcolumn*\latim{
Lápidem, quem reprobavérunt ædificántes: * hic factus est in caput ánguli.
}\switchcolumn\portugues{
A pedra que os construtores rejeitaram: * tornou-se a pedra angular.
}\switchcolumn*\latim{
A Dómino factum est istud: * et est mirábile in óculis nostris.
}\switchcolumn\portugues{
Foi o Senhor que fez isto: * e é uma cousa admirável aos nossos olhos.
}\switchcolumn*\latim{
Hæc est dies, quam fecit Dóminus: * exsultémus, et lætémur in ea.
}\switchcolumn\portugues{
Este é o dia que o Senhor fez: * exultemos e alegremo-nos n’Ele.
}\switchcolumn*\latim{
O Dómine, salvum me fac, o Dómine, bene prosperáre: * benedíctus qui venit in nómine Dómini.
}\switchcolumn\portugues{
Ó Senhor, salvai-me, ó Senhor, fazei que tenha prosperidade: * bendito o que vem em nome do Senhor.
}\switchcolumn*\latim{
Benedíximus vobis de domo Dómini: * Deus Dóminus, et illúxit nobis.
}\switchcolumn\portugues{
A vós bendizemos que sois da casa do Senhor: * o Senhor é Deus e nos manifestou a sua luz.
}\switchcolumn*\latim{
Constitúite diem solémnem in condénsis, * usque ad cornu altáris.
}\switchcolumn\portugues{
Tornai esse dia solene cobrindo de folhagem, * até ao ângulo do altar.
}\switchcolumn*\latim{
Deus meus es Tu, et confitébor tibi: * Deus meus es Tu, et exaltábo Te.
}\switchcolumn\portugues{
Vós sois o meu Deus e Vos louvarei: * Vós sois o meu Deus e Vos exaltarei.
}\switchcolumn*\latim{
Confitébor tibi quóniam exaudísti me * et factus es mihi in salútem.
}\switchcolumn\portugues{
Vos louvarei porque me atendestes, * e Vos tornastes a minha salvação.
}\switchcolumn*\latim{
Confitémini Dómino quóniam bonus: * quóniam in sǽculum misericórdia ejus.
}\switchcolumn\portugues{
Louvai o Senhor, porque é bom: * porque a sua misericórdia é eterna.
}\end{paracol}


\subsectioninfo{Salmo 118}{Beati immaculati in via}\label{salmo118}
\paragraph{ALEPH}
\begin{paracol}{2}\latim{
\rlettrine{B}{eáti} immaculáti in via: * qui ámbulant in lege Dómini.
}\switchcolumn\portugues{
\rlettrine{B}{em-aventurados} os imaculados no caminho: * que andam na lei do Senhor.
}\switchcolumn*\latim{
Beáti, qui scrutántur testimónia ejus: * in toto corde exquírunt eum.
}\switchcolumn\portugues{
Bem-aventurados os que escrutinam seus testemunhos: * em todo o coração O buscam.
}\switchcolumn*\latim{
Non enim qui operántur iniquitátem, * in viis ejus ambulavérunt.
}\switchcolumn\portugues{
Porque os que praticam a iniquidade, * nos seus caminhos não andam.
}\switchcolumn*\latim{
Tu mandásti * mandáta tua custodíri nimis.
}\switchcolumn\portugues{
Vós ordenastes * que os vossos mandamentos fossem guardados à risca.
}\switchcolumn*\latim{
Útinam dirigántur viæ meæ, * ad custodiéndas justificatiónes tuas!
}\switchcolumn\portugues{
Oxalá que meus passos sejam dirigidos * ao guardar de vossas justificações!
}\switchcolumn*\latim{
Tunc non confúndar, * cum perspéxero in ómnibus mandátis tuis.
}\switchcolumn\portugues{
Então não serei confundido, * quando observar todos vossos mandamentos.
}\switchcolumn*\latim{
Confitébor tibi in directióne cordis: * in eo quod dídici judícia justítiæ tuæ.
}\switchcolumn\portugues{
Vos louvarei com rectidão de coração: * porque aprendi os julgamentos da vossa justiça.
}\switchcolumn*\latim{
Justificatiónes tuas custódiam: * non me derelínquas usquequáque.
}\switchcolumn\portugues{
Guardarei as vossas justificações: * me não desampareis jamais.
}\end{paracol}

\paragraph{BETH}
\begin{paracol}{2}\latim{
\rlettrine{I}{n} quo córrigit adolescéntior viam suam? * In custodiéndo sermónes tuos.
}\switchcolumn\portugues{
\rlettrine{D}{e} que modo corrigirá o jovem o seu proceder? * Guardando as vossas palavras.
}\switchcolumn*\latim{
In toto corde meo exquisívi te: * ne repéllas me a mandátis tuis.
}\switchcolumn\portugues{
De todo meu coração Vos busquei: * me não deixeis transviar dos vossos mandamentos.
}\switchcolumn*\latim{
In corde meo abscóndi elóquia tua: * ut non peccem tibi.
}\switchcolumn\portugues{
Escondi no meu coração as vossas palavras: * para contra Vós não pecar.
}\switchcolumn*\latim{
Benedíctus es, Dómine: * doce me justificatiónes tuas.
}\switchcolumn\portugues{
Bendito sois, ó Senhor: * ensinai-me as vossas justificações.
}\switchcolumn*\latim{
In lábiis meis, * pronuntiávi ómnia judícia oris tui.
}\switchcolumn\portugues{
Nos meus lábios, * pronunciei todos os juízos da vossa boca.
}\switchcolumn*\latim{
In via testimoniórum tuórum delectátus sum, * sicut in ómnibus divítiis.
}\switchcolumn\portugues{
Deleitei-me no caminho dos vossos testemunhos, * como em todas as riquezas.
}\switchcolumn*\latim{
In mandátis tuis exercébor: * et considerábo vias tuas.
}\switchcolumn\portugues{
Nos vossos mandamentos me exercitarei: * e considerarei os vossos caminhos.
}\switchcolumn*\latim{
In justificatiónibus tuis meditábor: * non oblivíscar sermónes tuos.
}\switchcolumn\portugues{
Nas vossas justificações meditarei: * das vossas palavras me não esquecerei.
}\end{paracol}

\paragraph{GHIMEL}
\begin{paracol}{2}\latim{
\rlettrine{R}{etríbue} servo tuo, vivífica me: * et custódiam sermónes tuos:
}\switchcolumn\portugues{
\rlettrine{R}{etribuí} ao vosso servo, dai-me vida: * e guardarei as vossas palavras:
}\switchcolumn*\latim{
Revéla óculos meos: * et considerábo mirabília de lege tua.
}\switchcolumn\portugues{
Revelai meus olhos: * e considerarei as maravilhas da vossa lei.
}\switchcolumn*\latim{
Íncola ego sum in terra: * non abscóndas a me mandáta tua.
}\switchcolumn\portugues{
Sou peregrino na terra: * os vossos mandamentos não escondeis de mim.
}\switchcolumn*\latim{
Concupívit ánima mea desideráre justificatiónes tuas, * in omni témpore.
}\switchcolumn\portugues{
Ansiosa minha alma desejou as vossas justificações, * em todo o tempo.
}\switchcolumn*\latim{
Increpásti supérbos: * maledícti qui declínant a mandátis tuis.
}\switchcolumn\portugues{
Ameaçastes os soberbos: * malditos os que se desviam dos vossos mandamentos.
}\switchcolumn*\latim{
Aufer a me oppróbrium, et contémptum: * quia testimónia tua exquisívi.
}\switchcolumn\portugues{
Livrai-me do escárnio e do desprezo: * pois procurei os vossos testemunhos.
}\switchcolumn*\latim{
Étenim sedérunt príncipes, et advérsum me loquebántur: * servus autem tuus exercebátur in justificatiónibus tuis.
}\switchcolumn\portugues{
Porque os príncipes se sentaram e falavam contra mim: * o vosso servo, todavia, meditava nas vossas justificações.
}\switchcolumn*\latim{
Nam et testimónia tua meditátio mea est: * et consílium meum justificatiónes tuæ.
}\switchcolumn\portugues{
Pois os vossos testemunhos são a minha meditação: * e as vossas justificações o meu conselho.
}\end{paracol}

\paragraph{DALETH}
\begin{paracol}{2}\latim{
\rlettrine{A}{dhǽsit} paviménto ánima mea: * vivífica me secúndum verbum tuum.
}\switchcolumn\portugues{
\rlettrine{A}{} minha alma prostrou-se por terra: * vivificai-me segundo a vossa palavra.
}\switchcolumn*\latim{
Vias meas enuntiávi, et exaudísti me: * doce me justificatiónes tuas.
}\switchcolumn\portugues{
Expus os meus caminhos e me atendestes: * ensinai-me as vossas justificações.
}\switchcolumn*\latim{
Viam justificatiónum tuárum ínstrue me: * et exercébor in mirabílibus tuis.
}\switchcolumn\portugues{
Instruí-me no caminho das vossas leis: * e meditarei nas vossas maravilhas.
}\switchcolumn*\latim{
Dormitávit ánima mea præ tædio: * confírma me in verbis tuis.
}\switchcolumn\portugues{
Minha alma adormeceu de tédio: * fortalecei-me com vossas palavras.
}\switchcolumn*\latim{
Viam iniquitátis ámove a me: * et de lege tua miserére mei.
}\switchcolumn\portugues{
Afastai de mim o caminho da iniquidade: * e na vossa lei, tende misericórdia de mim.
}\switchcolumn*\latim{
Viam veritátis elégi: * judícia tua non sum oblítus.
}\switchcolumn\portugues{
Escolhi o caminho da verdade: * dos vossos juízos me não esqueci.
}\switchcolumn*\latim{
Adhǽsi testimóniis tuis, Dómine: * noli me confúndere.
}\switchcolumn\portugues{
Aderi aos vossos testemunhos, ó Senhor: * me não queirais confundir.
}\switchcolumn*\latim{
Viam mandatórum tuórum cucúrri, * cum dilatásti cor meum.
}\switchcolumn\portugues{
Corri pelo caminho dos vossos mandamentos, * quando dilatastes o meu coração.
}\end{paracol}

\paragraph{HE}
\begin{paracol}{2}\latim{
\rlettrine{L}{egem} pone mihi, Dómine, viam justificatiónum tuárum: * et exquíram eam semper.
}\switchcolumn\portugues{
\rlettrine{I}{mpõe-me} por lei, ó Senhor, o caminho dos vossos decretos: * e buscá-lo-ei sempre.
}\switchcolumn*\latim{
Da mihi intelléctum, et scrutábor legem tuam: * et custódiam illam in toto corde meo.
}\switchcolumn\portugues{
Dai-me inteligência e estudarei a vossa lei: * e a guardarei de todo meu coração.
}\switchcolumn*\latim{
Deduc me in sémitam mandatórum tuórum: * quia ipsam vólui.
}\switchcolumn\portugues{
Guiai-me pela senda dos vossos mandamentos: * pois essa mesma desejei.
}\switchcolumn*\latim{
Inclína cor meum in testimónia tua: * et non in avarítiam.
}\switchcolumn\portugues{
Inclinai o meu coração para os vossos testemunhos: * e não para a avareza.
}\switchcolumn*\latim{
Avérte óculos meos ne vídeant vanitátem: * in via tua vivífica me.
}\switchcolumn\portugues{
Desviai os meus olhos, para que não vejam a vaidade: * fazei-me viver no vosso caminho.
}\switchcolumn*\latim{
Státue servo tuo elóquium tuum, * in timóre tuo.
}\switchcolumn\portugues{
Estabelecei para o vosso servo a vossa palavra, * no vosso temor.
}\switchcolumn*\latim{
Ámputa oppróbrium meum quod suspicátus sum: * quia judícia tua jucúnda.
}\switchcolumn\portugues{
Afastai de mim a desonra que receio: * pois os vossos juízos são agradáveis.
}\switchcolumn*\latim{
Ecce, concupívi mandáta tua: * in æquitáte tua vivífica me.
}\switchcolumn\portugues{
Eis como suspirei pelos vossos mandamentos: * vivificai-me segundo a vossa justiça.
}\end{paracol}

\paragraph{VAU}
\begin{paracol}{2}\latim{
\rlettrine{E}{t} véniat super me misericórdia tua, Dómine: * salutáre tuum secúndum elóquium tuum.
}\switchcolumn\portugues{
\rlettrine{E}{} venha sobre mim a vossa misericórdia, ó Senhor: * e a vossa salvação, segundo a vossa palavra.
}\switchcolumn*\latim{
Et respondébo exprobrántibus mihi verbum: * quia sperávi in sermónibus tuis.
}\switchcolumn\portugues{
Poderei responder aos que me insultam: * que pus a minha esperança nas vossas palavras.
}\switchcolumn*\latim{
Et ne áuferas de ore meo verbum veritátis usquequáque: * quia in judíciis tuis supersperávi.
}\switchcolumn\portugues{
Não tireis jamais da minha boca a palavra da verdade: * pois muito confiei nas vossas juízos.
}\switchcolumn*\latim{
Et custódiam legem tuam semper: * in sǽculum et in sǽculum sǽculi.
}\switchcolumn\portugues{
Guardarei sempre a vossa lei: * pelos séculos e pelos séculos dos séculos.
}\switchcolumn*\latim{
Et ambulábam in latitúdine: * quia mandáta tua exquisívi.
}\switchcolumn\portugues{
Caminharei ao largo: * pois procurei os vossos mandamentos.
}\switchcolumn*\latim{
Et loquébar in testimóniis tuis in conspéctu regum: * et non confundébar.
}\switchcolumn\portugues{
Diante dos reis falarei dos vossos preceitos: * e me não envergonharei.
}\switchcolumn*\latim{
Et meditábar in mandátis tuis, * quæ diléxi.
}\switchcolumn\portugues{
Meditarei nos vossos mandamentos, * que amo.
}\switchcolumn*\latim{
Et levávi manus meas ad mandáta tua, quæ diléxi: * et exercébar in justificatiónibus tuis.
}\switchcolumn\portugues{
Levantarei as minhas mãos para os vossos mandamentos, que amo: * e exercitar-me-ei nas vossas justificações.
}\end{paracol}

\paragraph{ZAIN}
\begin{paracol}{2}\latim{
\rlettrine{M}{emor} esto verbi tui servo tuo, * in quo mihi spem dedísti.
}\switchcolumn\portugues{
\rlettrine{L}{embrai-Vos} da promessa que fizestes ao vosso servo, * com a qual me destes esperança.
}\switchcolumn*\latim{
Hæc me consoláta est in humilitáte mea: * quia elóquium tuum vivificávit me.
}\switchcolumn\portugues{
Consolou-me isto no meu abatimento: * pois a vossa palavra me vivificou.
}\switchcolumn*\latim{
Supérbi iníque agébant usquequáque: * a lege autem tua non declinávi.
}\switchcolumn\portugues{
Os soberbos procediam iniquamente sem cessar: * mas eu me não afastei da vossa lei.
}\switchcolumn*\latim{
Memor fui judiciórum tuórum a sǽculo, Dómine: * et consolátus sum.
}\switchcolumn\portugues{
Lembrei-me dos juízos que exercestes em todos os séculos, ó Senhor: * e consolei-me.
}\switchcolumn*\latim{
Deféctio ténuit me, * pro peccatóribus derelinquéntibus legem tuam.
}\switchcolumn\portugues{
Desfaleci, * vendo os pecadores que abandonavam a vossa lei.
}\switchcolumn*\latim{
Cantábiles mihi erant justificatiónes tuæ, * in loco peregrinatiónis meæ.
}\switchcolumn\portugues{
Vossas justificações eram cantadas por mim, * no lugar da minha peregrinação.
}\switchcolumn*\latim{
Memor fui nocte nóminis tui, Dómine: * et custodívi legem tuam.
}\switchcolumn\portugues{
Lembrei-me do vosso nome, ó Senhor, durante a noite: * e guardei a vossa lei.
}\switchcolumn*\latim{
Hæc facta est mihi: * quia justificatiónes tuas exquisívi.
}\switchcolumn\portugues{
Isto me aconteceu: * pois busquei cuidadoso as vossas justificações.
}\end{paracol}

\paragraph{HETH}
\begin{paracol}{2}\latim{
\rlettrine{P}{órtio} mea, Dómine, * dixi custodíre legem tuam.
}\switchcolumn\portugues{
\rlettrine{A}{} minha porção, ó Senhor, * é guardar a vossa lei, disse eu.
}\switchcolumn*\latim{
Deprecátus sum fáciem tuam in toto corde meo: * miserére mei secúndum elóquium tuum.
}\switchcolumn\portugues{
Supliquei na vossa presença de todo meu coração: * compadecei-Vos de mim, segundo a vossa palavra.
}\switchcolumn*\latim{
Cogitávi vias meas: * et convérti pedes meos in testimónia tua.
}\switchcolumn\portugues{
Considerei os meus caminhos: * e voltei os meus pés para os vossos testemunhos.
}\switchcolumn*\latim{
Parátus sum, et non sum turbátus: * ut custódiam mandáta tua.
}\switchcolumn\portugues{
Estou resolvido e me não tenho perturbado: * a guardar os vossos mandamentos.
}\switchcolumn*\latim{
Funes peccatórum circumpléxi sunt me: * et legem tuam non sum oblítus.
}\switchcolumn\portugues{
As redes dos pecadores me cingiram: * mas da vossa lei me não esqueci.
}\switchcolumn*\latim{
Média nocte surgébam ad confiténdum tibi, * super judícia justificatiónis tuæ.
}\switchcolumn\portugues{
À meia noite levantava-me para Vos louvar, * pelos juízos de vossas justificações.
}\switchcolumn*\latim{
Párticeps ego sum ómnium timéntium te: * et custodiéntium mandáta tua.
}\switchcolumn\portugues{
Sou associado de todos os que Vos temem: * e dos que guardam os vossos mandamentos.
}\switchcolumn*\latim{
Misericórdia tua, Dómine, plena est terra: * justificatiónes tuas doce me.
}\switchcolumn\portugues{
A terra está cheia, ó Senhor, da vossa misericórdia: * ensinai-me as vossas justificações.
}\end{paracol}

\paragraph{TETH}
\begin{paracol}{2}\latim{
\rlettrine{B}{onitátem} fecísti cum servo tuo, Dómine, * secúndum verbum tuum.
}\switchcolumn\portugues{
\rlettrine{T}{endes} usado de bondade com vosso servo, ó Senhor, * segundo a vossa palavra.
}\switchcolumn*\latim{
Bonitátem, et disciplínam, et sciéntiam doce me: * quia mandátis tuis crédidi.
}\switchcolumn\portugues{
Ensinai-me bondade, disciplina e ciência: * pois acreditei nos vossos mandamentos.
}\switchcolumn*\latim{
Priúsquam humiliárer ego delíqui: * proptérea elóquium tuum custodívi.
}\switchcolumn\portugues{
Antes de ser humilhado eu pequei: * mas agora obedeço à vossa palavra.
}\switchcolumn*\latim{
Bonus es tu: * et in bonitáte tua doce me justificatiónes tuas.
}\switchcolumn\portugues{
Bom sois Vós: * e por vossa bondade, ensinai-me as vossas justificações.
}\switchcolumn*\latim{
Multiplicáta est super me iníquitas superbórum: * ego autem in toto corde meo scrutábor mandáta tua.
}\switchcolumn\portugues{
A iniquidade dos soberbos multiplicou-se sobre mim: * porém eu, guardarei de todo meu coração os vossos mandamentos.
}\switchcolumn*\latim{
Coagulátum est sicut lac cor eórum: * ego vero legem tuam meditátus sum.
}\switchcolumn\portugues{
O coração deles coalhou-se como leite: * porém, ocupei-me em meditar na vossa lei.
}\switchcolumn*\latim{
Bonum mihi quia humiliásti me: * ut discam justificatiónes tuas.
}\switchcolumn\portugues{
Para mim foi bom que me humilhásseis: * para aprender as vossas justificações.
}\switchcolumn*\latim{
Bonum mihi lex oris tui, * super míllia auri et argénti.
}\switchcolumn\portugues{
A lei da vossa boca é boa para mim, * melhor que milhares de ouro e prata.
}\end{paracol}

\paragraph{JOD}
\begin{paracol}{2}\latim{
\rlettrine{M}{anus} tuæ fecérunt me, et plasmavérunt me: * da mihi intelléctum, et discam mandáta tua.
}\switchcolumn\portugues{
\rlettrine{V}{ossas} mãos fizeram-me e formaram-me: * dai-me inteligência e aprenderei os vossos mandamentos.
}\switchcolumn*\latim{
Qui timent te vidébunt me, et lætabúntur: * quia in verba tua supersperávi.
}\switchcolumn\portugues{
Os que Vos temem ver-me-ão e alegrar-se-ão: * pois nas vossas palavras pus grande esperança.
}\switchcolumn*\latim{
Cognóvi, Dómine, quia ǽquitas judícia tua: * et in veritáte tua humiliásti me.
}\switchcolumn\portugues{
Conheci, ó Senhor, que os vossos juízos são de equidade: * e que me humilhastes segundo a vossa justiça.
}\switchcolumn*\latim{
Fiat misericórdia tua ut consolétur me, * secúndum elóquium tuum servo tuo.
}\switchcolumn\portugues{
Venha a vossa misericórdia consolar-me, * segundo a palavra que destes ao vosso servo.
}\switchcolumn*\latim{
Véniant mihi miseratiónes tuæ, et vivam: * quia lex tua meditátio mea est.
}\switchcolumn\portugues{
Venham a mim as vossas misericórdias e viverei: * pois a vossa lei é a minha meditação.
}\switchcolumn*\latim{
Confundántur supérbi, quia injúste iniquitátem fecérunt in me: * ego autem exercébor in mandátis tuis.
}\switchcolumn\portugues{
Sejam confundidos os soberbos, pois injustamente maquinaram males contra mim: * eu, porém, reflectirei os vossos mandamentos.
}\switchcolumn*\latim{
Convertántur mihi timéntes te: * et qui novérunt testimónia tua.
}\switchcolumn\portugues{
Voltem-se para mim os que Vos temem: * e os que conhecem os vossos testemunhos.
}\switchcolumn*\latim{
Fiat cor meum immaculátum in justificatiónibus tuis, * ut non confúndar.
}\switchcolumn\portugues{
Seja imaculado o meu coração nas vossas justificações, * para que não seja confundido.
}\end{paracol}

\paragraph{CAPH}
\begin{paracol}{2}\latim{
\rlettrine{D}{efécit} in salutáre tuum ánima mea: * et in verbum tuum supersperávi.
}\switchcolumn\portugues{
\rlettrine{M}{inha} alma desfaleceu à espera da vossa salvação: * e na vossa palavra tenho confiado.
}\switchcolumn*\latim{
Defecérunt óculi mei in elóquium tuum, * dicéntes: Quando consoláberis me?
}\switchcolumn\portugues{
Meus olhos cansaram-se de esperar a vossa palavra, * dizendo: quando me consolareis?
}\switchcolumn*\latim{
Quia factus sum sicut uter in pruína: * justificatiónes tuas non sum oblítus.
}\switchcolumn\portugues{
Pois tornei-me como um odre exposto à geada: * mas me não esqueci das vossas justificações.
}\switchcolumn*\latim{
Quot sunt dies servi tui? * quando fácies de persequéntibus me judícium?
}\switchcolumn\portugues{
Quantos são os dias do vosso servo? * Quando fareis justiça aos que me perseguem?
}\switchcolumn*\latim{
Narravérunt mihi iníqui fabulatiónes: * sed non ut lex tua.
}\switchcolumn\portugues{
Narraram-me ímpias fábulas: * mas não são como a vossa lei.
}\switchcolumn*\latim{
Omnia mandáta tua véritas: * iníque persecúti sunt me, ádjuva me.
}\switchcolumn\portugues{
Todos vossos mandamentos são verdade: * injustamente me têm perseguido, socorrei-me.
}\switchcolumn*\latim{
Paulo minus consummavérunt me in terra: * ego autem non derelíqui mandáta tua.
}\switchcolumn\portugues{
Por pouco não deram cabo de mim na terra: * eu, porém, não abandonei os vossos mandamentos.
}\switchcolumn*\latim{
Secúndum misericórdiam tuam vivífica me: * et custódiam testimónia oris tui.
}\switchcolumn\portugues{
Vivificai-me segundo a vossa misericórdia: * e guardarei os testemunhos saídos da vossa boca.
}\end{paracol}

\paragraph{LAMED}
\begin{paracol}{2}\latim{
\rlettrine{I}{n} ætérnum, Dómine, * verbum tuum pérmanet in cælo.
}\switchcolumn\portugues{
\rlettrine{P}{ara} sempre, ó Senhor, * permanece no céu a vossa palavra.
}\switchcolumn*\latim{
In generatiónem et generatiónem véritas tua: * fundásti terram, et pérmanet.
}\switchcolumn\portugues{
Vossa verdade de geração em geração: * fundastes a terra e ela permanece.
}\switchcolumn*\latim{
Ordinatióne tua persevérat dies: * quóniam ómnia sérviunt tibi.
}\switchcolumn\portugues{
Por vossa ordem perseveram os dias: * pois tudo Vos serve.
}\switchcolumn*\latim{
Nisi quod lex tua meditátio mea est: * tunc forte periíssem in humilitáte mea.
}\switchcolumn\portugues{
Se a vossa lei não tivesse sido a minha meditação: * por certo teria perecido na minha humilhação.
}\switchcolumn*\latim{
In ætérnum non oblivíscar justificatiónes tuas: * quia in ipsis vivificásti me.
}\switchcolumn\portugues{
Jamais me esquecerei das vossas justificações: * pois nelas me destes a vida.
}\switchcolumn*\latim{
Tuus sum ego, salvum me fac: * quóniam justificatiónes tuas exquisívi.
}\switchcolumn\portugues{
Eu sou vosso, salvai-me: * porque procurei as vossas justificações.
}\switchcolumn*\latim{
Me exspectavérunt peccatóres ut pérderent me: * testimónia tua intelléxi.
}\switchcolumn\portugues{
Os pecadores esperaram-me para me perder: * porém, estive atento aos vossos testemunhos.
}\switchcolumn*\latim{
Omnis consummatiónis vidi finem: * latum mandátum tuum nimis.
}\switchcolumn\portugues{
Vi o fim de tudo o que é perfeito: * somente a vossa lei não tem limites.
}\end{paracol}

\paragraph{MEM}
\begin{paracol}{2}\latim{
\qlettrine{Q}{uómodo} diléxi legem tuam, Dómine? * tota die meditátio mea est.
}\switchcolumn\portugues{
\rlettrine{O}{} quanto amo a vossa lei, ó Senhor! * É a minha meditação todo o dia.
}\switchcolumn*\latim{
Super inimícos meos prudéntem me fecísti mandáto tuo: * quia in ætérnum mihi est.
}\switchcolumn\portugues{
Com vossos mandamentos me tornastes mais prudente que meus inimigos: * pois os tenho sempre comigo.
}\switchcolumn*\latim{
Super omnes docéntes me intelléxi: * quia testimónia tua meditátio mea est.
}\switchcolumn\portugues{
Compreendi mais que todos meus mestres: * pois os vossos testemunhos são a minha meditação.
}\switchcolumn*\latim{
Super senes intelléxi: * quia mandáta tua quæsívi.
}\switchcolumn\portugues{
Entendi mais que os anciãos: * pois procurei os vossos mandamentos.
}\switchcolumn*\latim{
Ab omni via mala prohíbui pedes meos: * ut custódiam verba tua.
}\switchcolumn\portugues{
De todo o mau caminho retirei os meus pés: * para guardar as vossas palavras.
}\switchcolumn*\latim{
A judíciis tuis non declinávi: * quia tu legem posuísti mihi.
}\switchcolumn\portugues{
Não me desviei dos vossos juízos: * pois Vós me prescrevestes uma lei.
}\switchcolumn*\latim{
Quam dúlcia fáucibus meis elóquia tua, * super mel ori meo!
}\switchcolumn\portugues{
Quão doces são as vossas palavras ao meu paladar, * à minha boca são-no mais que o mel.
}\switchcolumn*\latim{
A mandátis tuis intelléxi: * proptérea odívi omnem viam iniquitátis.
}\switchcolumn\portugues{
Com vossos mandamentos aprendi: * por isso odeio todo o caminho da iniquidade.
}\end{paracol}

\paragraph{NUN}
\begin{paracol}{2}\latim{
\rlettrine{L}{ucérna} pédibus meis verbum tuum, * et lumen sémitis meis.
}\switchcolumn\portugues{
\rlettrine{L}{anterna} para os meus passos é a vossa palavra * e luz para os meus caminhos.
}\switchcolumn*\latim{
Jurávi, et státui * custodíre judícia justítiæ tuæ.
}\switchcolumn\portugues{
Jurei e determinei * guardar os juízos da vossa justiça.
}\switchcolumn*\latim{
Humiliátus sum usquequáque, Dómine: * vivífica me secúndum verbum tuum.
}\switchcolumn\portugues{
Tenho sido imensamente humilhado, ó Senhor: * vivificai-me segundo a vossa palavra.
}\switchcolumn*\latim{
Voluntária oris mei beneplácita fac, Dómine: * et judícia tua doce me.
}\switchcolumn\portugues{
Fazei, ó Senhor, que Vos seja agradável as ofertas da minha boca: * e ensinai-me os vossos juízos.
}\switchcolumn*\latim{
Ánima mea in mánibus meis semper: * et legem tuam non sum oblítus.
}\switchcolumn\portugues{
Minha alma está sempre nas minhas mãos: * e me não esqueci da vossa lei.
}\switchcolumn*\latim{
Posuérunt peccatóres láqueum mihi: * et de mandátis tuis non errávi.
}\switchcolumn\portugues{
Os pecadores armaram-me laços: * e me não apartei dos vossos mandamentos.
}\switchcolumn*\latim{
Hereditáte acquisívi testimónia tua in ætérnum: * quia exsultátio cordis mei sunt.
}\switchcolumn\portugues{
Adquiri os vossos testemunhos para que sejam eternamente o meu património: * pois são a alegria do meu coração.
}\switchcolumn*\latim{
Inclinávi cor meum ad faciéndas justificatiónes tuas in ætérnum, * propter retributiónem.
}\switchcolumn\portugues{
Inclinei o meu coração a praticar sempre as vossas justificações, * pela retribuição.
}\end{paracol}

\paragraph{SAMECH}
\begin{paracol}{2}\latim{
\rlettrine{I}{níquos} ódio hábui: * et legem tuam diléxi.
}\switchcolumn\portugues{
\rlettrine{O}{diei} os iníquos: * e amei a vossa lei.
}\switchcolumn*\latim{
Adjútor et suscéptor meus es tu: * et in verbum tuum supersperávi.
}\switchcolumn\portugues{
Vós sois o meu defensor e o meu amparo: * e pus toda minha esperança na vossa palavra.
}\switchcolumn*\latim{
Declináte a me, malígni: * et scrutábor mandáta Dei mei.
}\switchcolumn\portugues{
Retirai-vos de mim, malignos: * e estudarei os mandamentos do meu Deus.
}\switchcolumn*\latim{
Súscipe me secúndum elóquium tuum, et vivam: * et non confúndas me ab exspectatióne mea.
}\switchcolumn\portugues{
Amparai-me segundo a vossa promessa e viverei: * e não permitais que seja confundido no que espero.
}\switchcolumn*\latim{
Ádjuva me, et salvus ero: * et meditábor in justificatiónibus tuis semper.
}\switchcolumn\portugues{
Ajudai-me e serei salvo: * e meditarei sempre nas vossas justificações.
}\switchcolumn*\latim{
Sprevísti omnes discedéntes a judíciis tuis: * quia injústa cogitátio eórum.
}\switchcolumn\portugues{
Desprezastes todos os que se desviam dos vossos juízos: * pois é injusto o seu pensamento.
}\switchcolumn*\latim{
Prævaricántes reputávi omnes peccatóres terræ: * ídeo diléxi testimónia tua.
}\switchcolumn\portugues{
Avaliei como prevaricadores todos os pecadores da terra: * por isso amei os vossos testemunhos.
}\switchcolumn*\latim{
Confíge timóre tuo carnes meas: * a judíciis enim tuis tímui.
}\switchcolumn\portugues{
Traspassai as minhas carnes com vosso temor: * de facto, temi os vossos juízos.
}\end{paracol}

\paragraph{AIN}
\begin{paracol}{2}\latim{
\rlettrine{F}{eci} judícium et justítiam: * non tradas me calumniántibus me.
}\switchcolumn\portugues{
\rlettrine{T}{enho} feito juízo e a justiça: * aos que me caluniam me não entregues.
}\switchcolumn*\latim{
Súscipe servum tuum in bonum: * non calumniéntur me supérbi.
}\switchcolumn\portugues{
Amparai o vosso servo para o bem: * me não caluniem os soberbos.
}\switchcolumn*\latim{
Óculi mei defecérunt in salutáre tuum: * et in elóquium justítiæ tuæ.
}\switchcolumn\portugues{
Meus olhos desfaleceram à espera da vossa salvação: * e das promessas da vossa justiça.
}\switchcolumn*\latim{
Fac cum servo tuo secúndum misericórdiam tuam: * et justificatiónes tuas doce me.
}\switchcolumn\portugues{
Tratai o vosso servo segundo a vossa misericórdia: * e ensinai-me as vossas justificações.
}\switchcolumn*\latim{
Servus tuus sum ego: * da mihi intelléctum, ut sciam testimónia tua.
}\switchcolumn\portugues{
Eu sou vosso servo: * dai-me inteligência, para que conheça os vossos testemunhos.
}\switchcolumn*\latim{
Tempus faciéndi, Dómine: * dissipavérunt legem tuam.
}\switchcolumn\portugues{
É tempo, ó Senhor, de procederdes: * dissiparam a vossa lei.
}\switchcolumn*\latim{
Ídeo diléxi mandáta tua, * super aurum et topázion.
}\switchcolumn\portugues{
Por isso amei os vossos mandamentos, * mais do que o ouro e o topázio.
}\switchcolumn*\latim{
Proptérea ad ómnia mandáta tua dirigébar: * omnem viam iníquam ódio hábui.
}\switchcolumn\portugues{
Por isso me tenho dirigido a todos vossos mandamentos: * e odiei todo o caminho injusto.
}\end{paracol}

\paragraph{PHE}
\begin{paracol}{2}\latim{
\rlettrine{M}{irabília} testimónia tua: * ídeo scrutáta est ea ánima mea.
}\switchcolumn\portugues{
\rlettrine{O}{s} vossos testemunhos são admiráveis: * por isso os tem investigado a minha alma.
}\switchcolumn*\latim{
Declarátio sermónum tuórum illúminat: * et intelléctum dat párvulis.
}\switchcolumn\portugues{
A decleração das vossas palavras alumia: * e dá inteligência aos pequenos.
}\switchcolumn*\latim{
Os meum apérui, et attráxi spíritum: * quia mandáta tua desiderábam.
}\switchcolumn\portugues{
Abri a minha boca e respirei: * pois desejava os vossos mandamentos.
}\switchcolumn*\latim{
Áspice in me, et miserére mei, * secúndum judícium diligéntium nomen tuum.
}\switchcolumn\portugues{
Olhai para mim e compadecei-Vos de mim, * segundo o juízo que usais com os que amam o vosso nome.
}\switchcolumn*\latim{
Gressus meos dírige secúndum elóquium tuum: * et non dominétur mei omnis injustítia.
}\switchcolumn\portugues{
Encaminhai os meus passos segundo a vossa palavra: * e me não domine iniquidade alguma.
}\switchcolumn*\latim{
Rédime me a calúmniis hóminum: * ut custódiam mandáta tua.
}\switchcolumn\portugues{
Livrai-me das injúrias dos homens: * para que guarde os vossos mandamentos.
}\switchcolumn*\latim{
Fáciem tuam illúmina super servum tuum: * et doce me justificatiónes tuas.
}\switchcolumn\portugues{
Fazei que a luz do vosso rosto reluza sobre o vosso servo: * e ensinai-me as vossas justificações.
}\switchcolumn*\latim{
Éxitus aquárum deduxérunt óculi mei: * quia non custodiérunt legem tuam.
}\switchcolumn\portugues{
Meus olhos derramaram rios de lágrimas: * por a vossa lei não terem guardado.
}\end{paracol}

\paragraph{SADE}
\begin{paracol}{2}\latim{
\qlettrine{J}{ustus} es, Dómine: * et rectum judícium tuum.
}\switchcolumn\portugues{
\rlettrine{V}{ós} sois justo, ó Senhor: * e o vosso juízo é recto.
}\switchcolumn*\latim{
Mandásti justítiam testimónia tua: * et veritátem tuam nimis.
}\switchcolumn\portugues{
Ordenastes os vossos testemunhos com justiça: * como a vossa suma verdade.
}\switchcolumn*\latim{
Tabéscere me fecit zelus meus: * quia oblíti sunt verba tua inimíci mei.
}\switchcolumn\portugues{
Meu zelo consumiu-me: * pois os meus inimigos se esqueceram das vossas palavras.
}\switchcolumn*\latim{
Ignítum elóquium tuum veheménter: * et servus tuus diléxit illud.
}\switchcolumn\portugues{
Refinadíssima é vossa palavra: * e o vosso servo a tem amado.
}\switchcolumn*\latim{
Adolescéntulus sum ego et contémptus: * justificatiónes tuas non sum oblítus.
}\switchcolumn\portugues{
Eu sou pequeno e desprezível: * mas me não esqueci das vossas justificações.
}\switchcolumn*\latim{
Justítia tua, justítia in ætérnum: * et lex tua véritas.
}\switchcolumn\portugues{
Vossa justiça é justiça eterna: * e a vossa lei é a verdade.
}\switchcolumn*\latim{
Tribulátio, et angústia invenérunt me: * mandáta tua meditátio mea est.
}\switchcolumn\portugues{
A tribulação e a angústia surpreenderam-me: * os vossos mandamentos são minha meditação.
}\switchcolumn*\latim{
Æquitas testimónia tua in ætérnum: * intelléctum da mihi, et vivam.
}\switchcolumn\portugues{
Vossos testemunhos são equidade eterna: * dai-me a inteligência deles e viverei.
}\end{paracol}

\paragraph{COPH}
\begin{paracol}{2}\latim{
\rlettrine{C}{lamávi} in toto corde meo, exáudi me, Dómine: * justificatiónes tuas requíram.
}\switchcolumn\portugues{
\rlettrine{C}{lamei} de todo meu coração, ouvi-me, ó Senhor: * buscarei as vossas justificações.
}\switchcolumn*\latim{
Clamávi ad te, salvum me fac: * ut custódiam mandáta tua.
}\switchcolumn\portugues{
A Vós Clamei, salvai-me: * para que guarde os vossos mandamentos.
}\switchcolumn*\latim{
Prævéni in maturitáte, et clamávi: * quia in verba tua supersperávi.
}\switchcolumn\portugues{
Antecipei a aurora e clamei: * pois muito esperei nas vossas palavras.
}\switchcolumn*\latim{
Prævenérunt óculi mei ad te dilúculo: * ut meditárer elóquia tua.
}\switchcolumn\portugues{
Meus olhos anteciparam-se para Vós desde a aurora: * para meditar as vossas palavras.
}\switchcolumn*\latim{
Vocem meam audi secúndum misericórdiam tuam, Dómine: * et secúndum judícium tuum vivífica me.
}\switchcolumn\portugues{
Ouvi a minha voz, ó Senhor, segundo a vossa misericórdia: * e dai-me vida segundo o vosso juízo.
}\switchcolumn*\latim{
Appropinquavérunt persequéntes me iniquitáti: * a lege autem tua longe facti sunt.
}\switchcolumn\portugues{
Meus perseguidores aproximaram-se da iniquidade: * e desviaram-se da vossa lei.
}\switchcolumn*\latim{
Prope es tu, Dómine: * et omnes viæ tuæ véritas.
}\switchcolumn\portugues{
Perto estais, ó Senhor: * e todos vossos caminhos são verdade.
}\switchcolumn*\latim{
Inítio cognóvi de testimóniis tuis: * quia in ætérnum fundásti ea.
}\switchcolumn\portugues{
Desde o princípio soube dos vossos testemunhos: * que estabelecestes para sempre.
}\end{paracol}

\paragraph{RES}
\begin{paracol}{2}\latim{
\rlettrine{V}{ide} humilitátem meam, et éripe me: * quia legem tuam non sum oblítus.
}\switchcolumn\portugues{
\rlettrine{O}{lhai} para o meu abatimento e livrai-me: * pois me não tenho esquecido da vossa lei.
}\switchcolumn*\latim{
Júdica judícium meum, et rédime me: * propter elóquium tuum vivífica me.
}\switchcolumn\portugues{
Julgai a minha causa e libertai-me: * dai-me a vida segundo a vossa palavra.
}\switchcolumn*\latim{
Longe a peccatóribus salus: * quia justificatiónes tuas non exquisiérunt.
}\switchcolumn\portugues{
A salvação está longe dos pecadores: * pois não buscam as vossas justificações.
}\switchcolumn*\latim{
Misericórdiæ tuæ multæ, Dómine: * secúndum judícium tuum vivífica me.
}\switchcolumn\portugues{
Muitas são as vossas misericórdias, ó Senhor: * dai-me a vida segundo o vosso juízo.
}\switchcolumn*\latim{
Multi qui persequúntur me, et tríbulant me: * a testimóniis tuis non declinávi.
}\switchcolumn\portugues{
Muitos são os que me perseguem e me atribulam: * porém, me não desviei dos vossos mandamentos.
}\switchcolumn*\latim{
Vidi prævaricántes, et tabescébam: * quia elóquia tua non custodiérunt.
}\switchcolumn\portugues{
Vi os prevaricadores e consumia-me: * pois não guardaram eles as vossas palavras.
}\switchcolumn*\latim{
Vide quóniam mandáta tua diléxi, Dómine: * in misericórdia tua vivífica me.
}\switchcolumn\portugues{
Vede, ó Senhor, quanto tenho amado os vossos mandamentos: * vivificai-me na vossa misericórdia.
}\switchcolumn*\latim{
Princípium verbórum tuórum, véritas: * in ætérnum ómnia judícia justítiæ tuæ.
}\switchcolumn\portugues{
O princípio das vossas palavras é a verdade: * todos os juízos da vossa justiça são eternos.
}\end{paracol}

\paragraph{SIN}
\begin{paracol}{2}\latim{
\rlettrine{P}{ríncipes} persecúti sunt me gratis: * et a verbis tuis formidávit cor meum.
}\switchcolumn\portugues{
\rlettrine{O}{s} príncipes perseguiram-me sem causa: * porém, o meu coração temeu as vossas palavras.
}\switchcolumn*\latim{
Lætábor ego super elóquia tua: * sicut qui invénit spólia multa.
}\switchcolumn\portugues{
Eu alegro-me nas vossas promessas: * como quem encontra muitos despojos.
}\switchcolumn*\latim{
Iniquitátem ódio hábui, et abominátus sum: * legem autem tuam diléxi.
}\switchcolumn\portugues{
Odiei e detestei a iniquidade: * mas amei a vossa lei.
}\switchcolumn*\latim{
Sépties in die laudem dixi tibi, * super judícia justítiæ tuæ.
}\switchcolumn\portugues{
Sete vezes ao dia Vos dirigi louvores, * pelos juízos da vossa justiça.
}\switchcolumn*\latim{
Pax multa diligéntibus legem tuam: * et non est illis scándalum.
}\switchcolumn\portugues{
Possuem muita paz os que amam a vossa lei: * e não há para eles escândalo.
}\switchcolumn*\latim{
Exspectábam salutáre tuum, Dómine: * et mandáta tua diléxi.
}\switchcolumn\portugues{
Esperava a vossa salvação, ó Senhor: * e amei os vossos mandamentos.
}\switchcolumn*\latim{
Custodívit ánima mea testimónia tua: * et diléxit ea veheménter.
}\switchcolumn\portugues{
Minha alma guardou os vossos testemunhos: * e ardentemente os amou.
}\switchcolumn*\latim{
Servávi mandáta tua, et testimónia tua: * quia omnes viæ meæ in conspéctu tuo.
}\switchcolumn\portugues{
Guardei os vossos mandamentos e os vossos testemunhos: * pois todos meus caminhos estão diante de Vós.
}\end{paracol}

\paragraph{TAU}
\begin{paracol}{2}\latim{
\rlettrine{A}{ppropínquet} deprecátio mea in conspéctu tuo, Dómine: * juxta elóquium tuum da mihi intelléctum.
}\switchcolumn\portugues{
\rlettrine{C}{hegue} a minha súplica à vossa presença, ó Senhor: * dai-me entendimento segundo a vossa palavra.
}\switchcolumn*\latim{
Intret postulátio mea in conspéctu tuo: * secúndum elóquium tuum éripe me.
}\switchcolumn\portugues{
Entre a minha petição até à vossa presença: * livrai-me segundo a vossa palavra.
}\switchcolumn*\latim{
Eructábunt lábia mea hymnum, * cum docúeris me justificatiónes tuas.
}\switchcolumn\portugues{
Dos meus lábios sairá um hino, * quando me ensinardes as vossas justificações.
}\switchcolumn*\latim{
Pronuntiábit lingua mea elóquium tuum: * quia ómnia mandáta tua ǽquitas.
}\switchcolumn\portugues{
Minha língua anunciará a vossa palavra: * pois todos vossos mandamentos são equidade.
}\switchcolumn*\latim{
Fiat manus tua ut salvet me: * quóniam mandáta tua elégi.
}\switchcolumn\portugues{
Estendei a vossa mão para me salvar: * porque escolhi os vossos mandamentos.
}\switchcolumn*\latim{
Concupívi salutáre tuum, Dómine: * et lex tua meditátio mea est.
}\switchcolumn\portugues{
Tenho desejado a vossa salvação, ó Senhor : * e a vossa lei é a minha meditação.
}\switchcolumn*\latim{
Vivet ánima mea, et laudábit te: * et judícia tua adjuvábunt me.
}\switchcolumn\portugues{
Minha alma viverá e Vos louvará: * e os vossos juízos serão o meu apoio.
}\switchcolumn*\latim{
Errávi, sicut ovis, quæ périit: * quǽre servum tuum, quia mandáta tua non sum oblítus.
}\switchcolumn\portugues{
Errante, como ovelha que se extraviou: * buscai o vosso servo, pois me não esqueci dos vossos mandamentos.
}\end{paracol}


\subsectioninfo{Salmo 119}{Ad Dominum cum tribularer clamavi}\label{salmo119}
\begin{paracol}{2}\latim{
\rlettrine{A}{d} Dóminum cum tribulárer clamávi: * et exaudívit me.
}\switchcolumn\portugues{
\rlettrine{N}{a} minha tribulação, clamei ao Senhor: * e ouviu-me.
}\switchcolumn*\latim{
Dómine, líbera ánimam meam a lábiis iníquis, * et a lingua dolósa.
}\switchcolumn\portugues{
Ó Senhor, livrai a minha alma dos lábios iníquos, * e da língua dolosa.
}\switchcolumn*\latim{
Quid detur tibi, aut quid apponátur tibi * ad linguam dolósam?
}\switchcolumn\portugues{
Que te será dado, ou que te será acrescentado, * ó língua dolosa?
}\switchcolumn*\latim{
Sagíttæ poténtis acútæ, * cum carbónibus desolatóriis.
}\switchcolumn\portugues{
Setas agudas do poderoso, * com brasas devoradoras.
}\switchcolumn*\latim{
Heu mihi, quia incolátus meus prolongátus est: habitávi cum habitántibus Cedar: * multum íncola fuit ánima mea.
}\switchcolumn\portugues{
Ai de mim, o meu desterro prolongou-se, habitei com os moradores de Cedar: * muito andou peregrinando a minha alma.
}\switchcolumn*\latim{
Cum his, qui odérunt pacem, eram pacíficus: * cum loquébar illis, impugnábant me gratis.
}\switchcolumn\portugues{
Com os que odiavam a paz eu era pacífico: * quando lhes falava, me contradiziam sem motivo.
}\end{paracol}

\subsectioninfo{Salmo 120}{Levavi oculos meos}\label{salmo120}
\begin{paracol}{2}\latim{
\rlettrine{L}{evávi} óculos meos in montes, * unde véniet auxílium mihi.
}\switchcolumn\portugues{
\rlettrine{L}{evantei} os meus olhos para os montes, * donde me virá o socorro.
}\switchcolumn*\latim{
Auxílium meum a Dómino, * qui fecit cælum et terram.
}\switchcolumn\portugues{
Meu socorro vem do Senhor, * que fez o céu e a terra.
}\switchcolumn*\latim{
Non det in commotiónem pedem tuum: * neque dormítet qui custódit te.
}\switchcolumn\portugues{
Não permita Ele que vacile o teu pé: * nem adormeça Aquele que te guarda.
}\switchcolumn*\latim{
Ecce, non dormitábit neque dórmiet, * qui custódit Israël.
}\switchcolumn\portugues{
Eis que não adormecerá, nem dormirá, * O que guarda Israel.
}\switchcolumn*\latim{
Dóminus custódit te, Dóminus protéctio tua, * super manum déxteram tuam.
}\switchcolumn\portugues{
O Senhor te guarda, o Senhor é a tua protecção, * Ele está à tua direita.
}\switchcolumn*\latim{
Per diem sol non uret te: * neque luna per noctem.
}\switchcolumn\portugues{
Durante o dia o sol te não queimará: * nem de noite a lua.
}\switchcolumn*\latim{
Dóminus custódit te ab omni malo: * custódiat ánimam tuam Dóminus.
}\switchcolumn\portugues{
O Senhor te guarde de todo o mal: * o Senhor guarde a tua alma.
}\switchcolumn*\latim{
Dóminus custódiat intróitum tuum, et éxitum tuum: * ex hoc nunc, et usque in sǽculum.
}\switchcolumn\portugues{
O Senhor guarde a tua entrada e a tua saída: * desde agora e para sempre.
}\end{paracol}


\subsectioninfo{Salmo 121}{Lætatus sum}\label{salmo121}
\begin{paracol}{2}\latim{
\rlettrine{L}{ætátus} sum in his, quæ dicta sunt mihi: * In domum Dómini íbimus.
}\switchcolumn\portugues{
\rlettrine{A}{legrei-me} com o que me foi dito: * iremos à casa do Senhor.
}\switchcolumn*\latim{
Stantes erant pedes nostri, * in átriis tuis, Jerúsalem.
}\switchcolumn\portugues{
Estavam os nossos pés parados, * às tuas portas, ó Jerusalém.
}\switchcolumn*\latim{
Jerúsalem, quæ ædificátur ut cívitas: * cujus participátio ejus in idípsum.
}\switchcolumn\portugues{
Jerusalém, que está edificada como uma cidade: * cujas partes estão em união.
}\switchcolumn*\latim{
Illuc enim ascendérunt tribus, tribus Dómini: * testimónium Israël ad confiténdum nómini Dómini.
}\switchcolumn\portugues{
De facto, lá subiram as tribos, as tribos do Senhor: * como testemunho a Israel, para louvar o nome do Senhor.
}\switchcolumn*\latim{
Quia illic sedérunt sedes in judício, * sedes super domum David.
}\switchcolumn\portugues{
Pois ali se estabeleceram as sedes em julgamento, * sedes sobre a casa de David.
}\switchcolumn*\latim{
Rogáte quæ ad pacem sunt Jerúsalem: * et abundántia diligéntibus te:
}\switchcolumn\portugues{
Roguei graças de paz para Jerusalém: * e abundância para os que a amam.
}\switchcolumn*\latim{
Fiat pax in virtúte tua: * et abundántia in túrribus tuis.
}\switchcolumn\portugues{
Reine a paz na tua força, * e abundância nas tuas torres.
}\switchcolumn*\latim{
Propter fratres meos, et próximos meos, * loquébar pacem de te:
}\switchcolumn\portugues{
Por causa dos meus irmãos e dos meus vizinhos, * pedi a paz para ti.
}\switchcolumn*\latim{
Propter domum Dómini, Dei nostri, * quæsívi bona tibi.
}\switchcolumn\portugues{
Por causa da casa do Senhor nosso Deus, * procurei o bem para ti.
}\end{paracol}


\subsectioninfo{Salmo 122}{Ad Te levavi oculos meos}\label{salmo122}
\begin{paracol}{2}\latim{
\rlettrine{A}{d} Te levávi óculos meos, * qui hábitas in cælis.
}\switchcolumn\portugues{
\rlettrine{L}{evantei} os meus olhos para Vós, * que habitais nos céus.
}\switchcolumn*\latim{
Ecce, sicut óculi servórum * in mánibus dominórum suórum,
}\switchcolumn\portugues{
Eis que, assim como os olhos dos servos * estão nas mãos dos seus senhores,
}\switchcolumn*\latim{
Sicut óculi ancíllæ in mánibus dóminæ suæ: * ita óculi nostri ad Dóminum, Deum nostrum, donec misereátur nostri.
}\switchcolumn\portugues{
Como os olhos da serva nas mãos de sua senhora: * assim os nossos olhos estão no Senhor nosso Deus, até que tenha misericórdia de nós.
}\switchcolumn*\latim{
Miserére nostri, Dómine, miserére nostri: * quia multum repléti sumus despectióne:
}\switchcolumn\portugues{
Tende misericórdia de nós, ó Senhor, tende misericórdia de nós: * pois estamos cheios de desprezo.
}\switchcolumn*\latim{
Quia multum repléta est ánima nostra: * oppróbrium abundántibus, et despéctio supérbis.
}\switchcolumn\portugues{
Pois a nossa alma está cheiíssima: * de ser o objecto de escárnio para os ricos e de desprezo para os soberbos.
}\end{paracol}


\subsectioninfo{Salmo 123}{Nisi quia Dominus}\label{salmo123}
\begin{paracol}{2}\latim{
\rlettrine{N}{isi} quia Dóminus erat in nobis, dicat nunc Israël: * nisi quia Dóminus erat in nobis,
}\switchcolumn\portugues{
\rlettrine{S}{e} o Senhor não tivesse estado connosco, diga-o agora Israel: * se o Senhor não tivesse estado connosco,
}\switchcolumn*\latim{
Cum exsúrgerent hómines in nos, * forte vivos deglutíssent nos:
}\switchcolumn\portugues{
Quando os homens se levantavam contra nós, * de certo nos teriam devorado vivos:
}\switchcolumn*\latim{
Cum irascerétur furor eórum in nos, * fórsitan aqua absorbuísset nos.
}\switchcolumn\portugues{
Quando se inflamou a ira deles contra nós, * sem dúvida a água nos teria afogado.
}\switchcolumn*\latim{
Torréntem pertransívit ánima nostra: * fórsitan pertransísset ánima nostra aquam intolerábilem.
}\switchcolumn\portugues{
A nossa alma passou a torrente: * talvez a nossa alma poderia ter passado a água intolerável.
}\switchcolumn*\latim{
Benedíctus Dóminus * qui non dedit nos in captiónem déntibus eórum.
}\switchcolumn\portugues{
Bendito o Senhor, * que nos não deu por presa aos seus dentes.
}\switchcolumn*\latim{
Ánima nostra sicut passer erépta est * de láqueo venántium:
}\switchcolumn\portugues{
A nossa alma escapou como o pássaro * do laço dos caçadores:
}\switchcolumn*\latim{
Láqueus contrítus est, * et nos liberáti sumus.
}\switchcolumn\portugues{
O laço foi quebrado, * e nós ficámos livres.
}\switchcolumn*\latim{
Adjutórium nostrum in nómine Dómini, * qui fecit cælum et terram.
}\switchcolumn\portugues{
Nosso auxílio está no nome do Senhor, * que fez o céu e a terra.
}\end{paracol}


\subsectioninfo{Salmo 124}{Qui confidunt in Domino}\label{salmo124}
\begin{paracol}{2}\latim{
\qlettrine{Q}{ui} confídunt in Dómino, sicut mons Sion: * non commovébitur in ætérnum, qui hábitat in Jerúsalem.
}\switchcolumn\portugues{
\rlettrine{O}{s} que confiam no Senhor serão como o monte Sião: * nunca será abalado o que habita em Jerusalém.
}\switchcolumn*\latim{
Montes in circúitu ejus: * et Dóminus in circúitu pópuli sui, ex hoc nunc et usque in sǽculum.
}\switchcolumn\portugues{
Ela está cercada de montes: * e o Senhor está ao redor do seu povo, desde agora e para sempre.
}\switchcolumn*\latim{
Quia non relínquet Dóminus virgam peccatórum super sortem justórum: * ut non exténdant justi ad iniquitátem manus suas.
}\switchcolumn\portugues{
Pois o Senhor não deixará a vara dos pecadores sobre a herança dos justos: * para que os justos não estendam as suas mãos para a iniquidade.
}\switchcolumn*\latim{
Bénefac, Dómine, bonis, * et rectis corde.
}\switchcolumn\portugues{
Senhor, fazei bem aos bons * e aos rectos de coração.
}\switchcolumn*\latim{
Declinántes autem in obligatiónes addúcet Dóminus cum operántibus iniquitátem: * pax super Israël.
}\switchcolumn\portugues{
Aos que se desviam para caminhos tortuosos, levá-los-á o Senhor com os que praticam a iniquidade: * a paz seja sobre Israel.
}\end{paracol}


\subsectioninfo{Salmo 125}{In convertendo Dominus}\label{salmo125}
\begin{paracol}{2}\latim{
\rlettrine{I}{n} converténdo Dóminus captivitátem Sion: * facti sumus sicut consoláti:
}\switchcolumn\portugues{
\qlettrine{Q}{uando} o Senhor fez volver os cativos de Sião: * nós ficámos cheios de consolação:
}\switchcolumn*\latim{
Tunc replétum est gáudio os nostrum: * et lingua nostra exsultatióne.
}\switchcolumn\portugues{
Então a nossa boca encheu-se de alegria: * e a nossa língua exultou.
}\switchcolumn*\latim{
Tunc dicent inter gentes: * Magnificávit Dóminus fácere cum eis.
}\switchcolumn\portugues{
Então dir-se-á entre as gentes: * grandes coisas fez o Senhor para eles.
}\switchcolumn*\latim{
Magnificávit Dóminus fácere nobíscum: * facti sumus lætántes.
}\switchcolumn\portugues{
Grandes coisas fez o Senhor por nós: * estamos cheios de júbilo.
}\switchcolumn*\latim{
Convérte, Dómine, captivitátem nostram, * sicut torrens in Austro.
}\switchcolumn\portugues{
Fazei, ó Senhor, volver os nossos cativos, * como as torrentes do sul.
}\switchcolumn*\latim{
Qui séminant in lácrimis, * in exsultatióne metent.
}\switchcolumn\portugues{
Os que semeiam em lágrimas, * com alegria ceifarão.
}\switchcolumn*\latim{
Eúntes ibant et flebant, * mitténtes sémina sua.
}\switchcolumn\portugues{
Andando iam e choravam, * lançando as suas sementes.
}\switchcolumn*\latim{
Veniéntes autem vénient cum exsultatióne, * portántes manípulos suos.
}\switchcolumn\portugues{
Vindo, todavia, virão contentes, * trazendo os seus feixes.
}\end{paracol}


\subsectioninfo{Salmo 126}{Nsi Dominus ædificaverit domum}\label{salmo126}
\begin{paracol}{2}\latim{
\rlettrine{N}{isi} Dóminus ædificáverit domum, * in vanum laboravérunt qui ædíficant eam.
}\switchcolumn\portugues{
\rlettrine{S}{e} o Senhor não edificar a casa, * é em vão que trabalham os que a edificam.
}\switchcolumn*\latim{
Nisi Dóminus custodíerit civitátem, * frustra vígilat qui custódit eam.
}\switchcolumn\portugues{
Se o Senhor não guardar a cidade, * inutilmente vigia o que a guarda.
}\switchcolumn*\latim{
Vanum est vobis ante lucem súrgere: * súrgite postquam sedéritis, qui manducátis panem dolóris.
}\switchcolumn\portugues{
Em vão vos levantais antes de amanhecer: * levantai-vos, depois que tiverdes repousado, vós que comeis o pão da dor.
}\switchcolumn*\latim{
Cum déderit diléctis suis somnum: * ecce heréditas Dómini fílii: merces, fructus ventris.
}\switchcolumn\portugues{
Quando Ele der o sono aos seus amados: * eis que a herança do Senhor são filhos, o fruto do ventre.
}\switchcolumn*\latim{
Sicut sagíttæ in manu poténtis: * ita fílii excussórum.
}\switchcolumn\portugues{
Como setas na mão do valente: * assim são os filhos dos atribulados.
}\switchcolumn*\latim{
Beátus vir, qui implévit desidérium suum ex ipsis: * non confundétur cum loquétur inimícis suis in porta.
}\switchcolumn\portugues{
Ditoso o varão que viu cumprido o seu desejo com eles: * não será confundido quando falar com seus inimigos no portão.
}\end{paracol}


\subsectioninfo{Salmo 127}{Beati omnes qui timent Dominum}\label{salmo127}
\begin{paracol}{2}\latim{
\rlettrine{B}{eáti} omnes, qui timent Dóminum, * qui ámbulant in viis ejus.
}\switchcolumn\portugues{
\rlettrine{B}{em-aventurados} todos os que temem o Senhor, * e que andam nos seus caminhos.
}\switchcolumn*\latim{
Labóres mánuum tuárum quia manducábis: * beátus es, et bene tibi erit.
}\switchcolumn\portugues{
Pois comerás dos labores de tuas mãos: * bem-aventurado és e ficarás bem.
}\switchcolumn*\latim{
Uxor tua sicut vitis abúndans, * in latéribus domus tuæ.
}\switchcolumn\portugues{
Tua esposa será como uma videira fecunda, * no interior de tua casa.
}\switchcolumn*\latim{
Fílii tui sicut novéllæ olivárum, * in circúitu mensæ tuæ.
}\switchcolumn\portugues{
Teus filhos, como pimpolhos de oliveiras, * ao redor de tua mesa.
}\switchcolumn*\latim{
Ecce, sic benedicétur homo, * qui timet Dóminum.
}\switchcolumn\portugues{
Eis como será abençoado o homem, * que teme o Senhor.
}\switchcolumn*\latim{
Benedícat tibi Dóminus ex Sion: * et vídeas bona Jerúsalem ómnibus diébus vitæ tuæ.
}\switchcolumn\portugues{
Te abençoe o Senhor desde Sião: * e vejas os bens de Jerusalém todos os dias de tua vida.
}\switchcolumn*\latim{
Et vídeas fílios filiórum tuórum, * pacem super Israël.
}\switchcolumn\portugues{
Vejas os filhos de teus filhos, * e a paz sobre Israel.
}\end{paracol}


\subsectioninfo{Salmo 128}{Sæpe expugnaverunt me}\label{salmo128}
\begin{paracol}{2}\latim{
\rlettrine{S}{æpe} expugnavérunt me a juventúte mea, * dicat nunc Israël.
}\switchcolumn\portugues{
\rlettrine{A}{miúde} me combateram desde a minha mocidade, * diga-o agora Israel.
}\switchcolumn*\latim{
Sæpe expugnavérunt me a juventúte mea: * étenim non potuérunt mihi.
}\switchcolumn\portugues{
Muitas vezes me combateram desde a minha mocidade: * todavia, não prevaleceram contra mim.
}\switchcolumn*\latim{
Supra dorsum meum fabricavérunt peccatóres: * prolongavérunt iniquitátem suam.
}\switchcolumn\portugues{
Sobre o meu dorso fabricaram os pecadores: * prolongaram a sua iniquidade.
}\switchcolumn*\latim{
Dóminus justus concídit cervíces peccatórum: * confundántur et convertántur retrórsum omnes, qui odérunt Sion.
}\switchcolumn\portugues{
O Senhor que é justo cortou os pescoços dos pecadores: * fiquem confundidos e retrocedam todos os que odeiam Sião.
}\switchcolumn*\latim{
Fiant sicut fænum tectórum: * quod priúsquam evellátur, exáruit:
}\switchcolumn\portugues{
Sejam como a erva dos telhados: * a qual seca antes de ser arrancada:
}\switchcolumn*\latim{
De quo non implévit manum suam qui metit, * et sinum suum qui manípulos cólligit.
}\switchcolumn\portugues{
Da qual o ceifeiro não encheu a mão, * nem seus braços o que apanha seus feixes.
}\switchcolumn*\latim{
Et non dixérunt qui præteríbant: benedíctio Dómini super vos: * benedíximus vobis in nómine Dómini.
}\switchcolumn\portugues{
Não disseram os que passavam: a bênção do Senhor seja sobre vós: * nós vos abençoamos no nome do Senhor.
}\end{paracol}


\subsectioninfo{Salmo 129}{De profundis clamavi ad Te}\label{salmo129}
\begin{paracol}{2}\latim{
\rlettrine{D}{e} profúndis clamávi ad Te, Dómine: * Dómine, exáudi vocem meam:
}\switchcolumn\portugues{
\rlettrine{D}{o} profundo clamei a Vós, Senhor: * ó Senhor, escutai a minha voz:
}\switchcolumn*\latim{
Fiant aures tuæ intendéntes, * in vocem deprecatiónis meæ.
}\switchcolumn\portugues{
Estejam atentos os vossos ouvidos, * à voz da minha súplica.
}\switchcolumn*\latim{
Si iniquitátes observáveris, Dómine: * Dómine, quis sustinébit?
}\switchcolumn\portugues{
Se observardes as nossas iniquidades, Senhor: * ó Senhor, quem poderá subsistir?
}\switchcolumn*\latim{
Quia apud Te propitiátio est: * et propter legem tuam sustínui Te, Dómine.
}\switchcolumn\portugues{
Pois em Vós está a clemência: * e devido à vossa lei, Senhor, sustive em Vós.
}\switchcolumn*\latim{
Sustínuit ánima mea in verbo ejus: * sperávit ánima mea in Dómino.
}\switchcolumn\portugues{
Minha alma confia na sua palavra: * esperou a minha alma no Senhor.
}\switchcolumn*\latim{
A custódia matutína usque ad noctem: * speret Israël in Dómino.
}\switchcolumn\portugues{
Desde a vigília matutina até à noite: * espere Israel no Senhor.
}\switchcolumn*\latim{
Quia apud Dóminum misericórdia: * et copiósa apud eum redémptio.
}\switchcolumn\portugues{
Pois no Senhor está a misericórdia: * e há n’Ele abundante redenção.
}\switchcolumn*\latim{
Et ipse rédimet Israël, * ex ómnibus iniquitátibus ejus.
}\switchcolumn\portugues{
Ele mesmo redimirá Israel, * de todas suas iniquidades.
}\end{paracol}


\subsectioninfo{Salmo 130}{Domine, non est exaltatum cor meum}\label{salmo130}
\begin{paracol}{2}\latim{
\rlettrine{D}{ómine,} non est exaltátum cor meum: * neque eláti sunt óculi mei.
}\switchcolumn\portugues{
\rlettrine{S}{enhor,} o meu coração se não exaltou: * nem os meus olhos se mostraram altivos.
}\switchcolumn*\latim{
Neque ambulávi in magnis: * neque in mirabílibus super me.
}\switchcolumn\portugues{
Não andei em grandezas: * nem em pompas superiores a mim.
}\switchcolumn*\latim{
Si non humíliter sentiébam: * sed exaltávi ánimam meam:
}\switchcolumn\portugues{
Se não tinha sentimentos humildes: * mas exaltava a minha alma:
}\switchcolumn*\latim{
Sicut ablactátus est super matre sua, * ita retribútio in ánima mea.
}\switchcolumn\portugues{
Como o ablactado é para sua mãe, * assim seja retribuída a minha alma.
}\switchcolumn*\latim{
Speret Israël in Dómino, * ex hoc nunc et usque in sǽculum.
}\switchcolumn\portugues{
Espere Israel no Senhor, * desde agora e para sempre.
}\end{paracol}


\subsectioninfo{Salmo 131}{Memento, Domine}\label{salmo131}
\begin{paracol}{2}\latim{
\rlettrine{M}{eménto,} Dómine, David, * et omnis mansuetúdinis ejus:
}\switchcolumn\portugues{
\rlettrine{L}{embrai-Vos,} ó Senhor, de David, * e de toda sua mansidão:
}\switchcolumn*\latim{
Sicut jurávit Dómino, * votum vovit Deo Jacob:
}\switchcolumn\portugues{
Como fez um juramento ao Senhor, * um voto ao Deus de Jacob:
}\switchcolumn*\latim{
Si introíero in tabernáculum domus meæ, * si ascéndero in lectum strati mei:
}\switchcolumn\portugues{
Se entrar na tenda de minha casa, * se subir ao leito do meu estrado:
}\switchcolumn*\latim{
Si dédero somnum óculis meis, * et pálpebris meis dormitatiónem:
}\switchcolumn\portugues{
Se der sono aos meus olhos, * e às minhas pestanas adormecimento:
}\switchcolumn*\latim{
Et réquiem tempóribus meis: donec invéniam locum Dómino, * tabernáculum Deo Jacob.
}\switchcolumn\portugues{
Repouso aos meus templos, até que encontre um lugar para o Senhor, * um tabernáculo para o Deus de Jacob.
}\switchcolumn*\latim{
Ecce, audívimus eam in Éphrata: * invénimus eam in campis silvæ.
}\switchcolumn\portugues{
Eis que ouvimos dizer que estava em Efrata: * e a encontrámos nos campos da selva.
}\switchcolumn*\latim{
Introíbimus in tabernáculum ejus: * adorábimus in loco, ubi stetérunt pedes ejus.
}\switchcolumn\portugues{
Entraremos no seu tabernáculo: * nós o adoraremos no lugar onde estiveram os seus pés.
}\switchcolumn*\latim{
Surge, Dómine, in réquiem tuam, * Tu et arca sanctificatiónis tuæ.
}\switchcolumn\portugues{
Levantai-Vos, ó Senhor, entrai no vosso repouso, * Vós e a arca de vossa santificação.
}\switchcolumn*\latim{
Sacerdótes tui induántur justítiam: * et sancti tui exsúltent.
}\switchcolumn\portugues{
Vistam-se os vossos sacerdotes de justiça: * e exultem-se os vossos santos.
}\switchcolumn*\latim{
Propter David, servum tuum, * non avértas fáciem Christi tui.
}\switchcolumn\portugues{
Por amor de David vosso servo, * não desprezeis o rosto de vosso Cristo.
}\switchcolumn*\latim{
Jurávit Dóminus David veritátem, et non frustrábitur eam: * De fructu ventris tui ponam super sedem tuam.
}\switchcolumn\portugues{
Jurou o Senhor verdade a David e não deixará de cumpri-la: * sobre o teu trono porei do fruto de teu ventre.
}\switchcolumn*\latim{
Si custodíerint fílii tui testaméntum meum, * et testimónia mea hæc, quæ docébo eos:
}\switchcolumn\portugues{
Se os teus filhos guardarem a minha aliança, * e os testemunhos que lhes ensinarei:
}\switchcolumn*\latim{
Et fílii eórum usque in sǽculum, * sedébunt super sedem tuam.
}\switchcolumn\portugues{
Também os seus filhos para sempre, * se sentarão sobre o teu trono.
}\switchcolumn*\latim{
Quóniam elégit Dóminus Sion: * elégit eam in habitatiónem sibi.
}\switchcolumn\portugues{
Porque o Senhor escolheu Sião: * escolheu-a para sua habitação.
}\switchcolumn*\latim{
Hæc réquies mea in sǽculum sǽculi: * hic habitábo quóniam elégi eam.
}\switchcolumn\portugues{
Este é o meu repouso para sempre: * aqui habitarei porque o escolhi.
}\switchcolumn*\latim{
Víduam ejus benedícens benedícam: * páuperes ejus saturábo pánibus.
}\switchcolumn\portugues{
Abençoarei copiosamente a sua viúva: * saciarei de pães os seus pobres.
}\switchcolumn*\latim{
Sacerdótes ejus índuam salutári: * et sancti ejus exsultatióne exsultábunt.
}\switchcolumn\portugues{
Vestirei os seus sacerdotes de salvação: * e os seus santos exultarão de júbilo.
}\switchcolumn*\latim{
Illuc prodúcam cornu David, * parávi lucérnam Christo meo.
}\switchcolumn\portugues{
Ali dilatarei o poder de David, * preparei uma lâmpada para o meu Cristo.
}\switchcolumn*\latim{
Inimícos ejus índuam confusióne: * super ipsum autem efflorébit sanctificátio mea.
}\switchcolumn\portugues{
Cobrirei de confusão os seus inimigos: * mas sobre eles florescerá a minha santidade.
}\end{paracol}


\subsectioninfo{Salmo 132}{Ecce quam bonum}\label{salmo132}
\begin{paracol}{2}\latim{
\rlettrine{E}{cce} quam bonum et quam jucúndum, * habitáre fratres in unum:
}\switchcolumn\portugues{
\rlettrine{O}{} quão bom e quão suave é, * viverem os irmãos em união:
}\switchcolumn*\latim{
Sicut unguéntum in cápite, * quod descéndit in barbam, barbam Aaron,
}\switchcolumn\portugues{
É como unção na cabeça, * que desce sobre a barba de Arão,
}\switchcolumn*\latim{
Quod descéndit in oram vestiménti ejus: * sicut ros Hermon, qui descéndit in montem Sion.
}\switchcolumn\portugues{
Que desce até à orla do seu manto: * e como o orvalho do Hermon, que desce sobre o monte Sião.
}\switchcolumn*\latim{
Quóniam illic mandávit Dóminus benedictiónem, * et vitam usque in sǽculum.
}\switchcolumn\portugues{
Porque o Senhor derramou ali a sua bênção, * e vida para sempre.
}\end{paracol}


\subsectioninfo{Salmo 133}{Ecce nunc benedicite}\label{salmo133}
\begin{paracol}{2}\latim{
\rlettrine{E}{cce} nunc benedícite Dóminum, * omnes servi Dómini:
}\switchcolumn\portugues{
\rlettrine{A}{gora,} pois, bendizei o Senhor, * todos os servos do Senhor:
}\switchcolumn*\latim{
Qui statis in domo Dómini, * in átriis domus Dei nostri.
}\switchcolumn\portugues{
Vós que estais na casa do Senhor, * nos átrios da casa do nosso Deus.
}\switchcolumn*\latim{
In nóctibus extóllite manus vestras in sancta, * et benedícite Dóminum.
}\switchcolumn\portugues{
De noite levantai as vossas mãos para o santuário, * e bendizei o Senhor.
}\switchcolumn*\latim{
Benedícat te Dóminus ex Sion, * qui fecit cælum et terram.
}\switchcolumn\portugues{
Te abençoe de Sião o Senhor, * que fez o céu e a terra.
}\end{paracol}


\subsectioninfo{Salmo 134}{Laudate nomen Domini}\label{salmo134}
\begin{paracol}{2}\latim{
\rlettrine{L}{audáte} nomen Dómini, * laudáte, servi, Dóminum.
}\switchcolumn\portugues{
\rlettrine{L}{ouvai} o nome do Senhor, * louvai o Senhor, vós seus servos.
}\switchcolumn*\latim{
Qui statis in domo Dómini, * in átriis domus Dei nostri.
}\switchcolumn\portugues{
Vós que estais na casa do Senhor, * nos átrios da casa do nosso Deus.
}\switchcolumn*\latim{
Laudáte Dóminum, quia bonus Dóminus: * psállite nómini ejus, quóniam suáve.
}\switchcolumn\portugues{
Louvai o Senhor, pois o Senhor é bom: * cantai ao seu nome, porque é suave.
}\switchcolumn*\latim{
Quóniam Jacob elégit sibi Dóminus, * Israël in possessiónem sibi.
}\switchcolumn\portugues{
Porque o Senhor escolheu para si Jacob, * e Israel para sua possessão.
}\switchcolumn*\latim{
Quia ego cognóvi quod magnus est Dóminus, * et Deus noster præ ómnibus diis.
}\switchcolumn\portugues{
Pois eu conheci que o Senhor é grande, * e que o nosso Deus é sobre todos os deuses.
}\switchcolumn*\latim{
Omnia quæcúmque vóluit, Dóminus fecit in cælo, et in terra, * in mari, et in ómnibus abýssis.
}\switchcolumn\portugues{
Tudo o que quis, o fez o Senhor no céu, na terra, * no mar e em todos os abismos.
}\switchcolumn*\latim{
Edúcens nubes ab extrémo terræ: * fúlgura in plúviam fecit.
}\switchcolumn\portugues{
Ele faz subir as nuvens das extremidades da terra: * converte os relâmpagos em chuva.
}\switchcolumn*\latim{
Qui prodúcit ventos de thesáuris suis: * qui percússit primogénita Ægýpti ab hómine usque ad pecus.
}\switchcolumn\portugues{
Ele faz sair os ventos dos seus tesouros: * ele feriu os primogénitos do Egipto, desde o homem até ao animal.
}\switchcolumn*\latim{
Et misit signa, et prodígia in médio tui, Ægýpte: * in Pharaónem, et in omnes servos ejus.
}\switchcolumn\portugues{
E enviou sinais e prodígios no meio de ti, ó Egipto: * contra Faraó e contra todos seus servos.
}\switchcolumn*\latim{
Qui percússit gentes multas: * et occídit reges fortes:
}\switchcolumn\portugues{
Ele feriu muitas gentes: * e matou reis poderosos:
}\switchcolumn*\latim{
Sehon, regem Amorrhæórum, et Og, regem Basan, * et ómnia regna Chánaan.
}\switchcolumn\portugues{
Seon, rei dos Amorreus e Ogue, rei de Basã, * e todos os reinos de Canaan.
}\switchcolumn*\latim{
Et dedit terram eórum hereditátem, * hereditátem Israël, pópulo suo.
}\switchcolumn\portugues{
E deu as terras deles em herança, * em herança a Israel, seu povo.
}\switchcolumn*\latim{
Dómine, nomen tuum in ætérnum: * Dómine, memoriále tuum in generatiónem et generatiónem.
}\switchcolumn\portugues{
Vosso nome, ó Senhor, subsistirá eternamente: * vossa memória, ó Senhor, passará de geração em geração.
}\switchcolumn*\latim{
Quia judicábit Dóminus pópulum suum: * et in servis suis deprecábitur.
}\switchcolumn\portugues{
Pois o Senhor julgará o seu povo: * e compadecer-se-á dos seus servos.
}\switchcolumn*\latim{
Simulácra géntium argéntum, et aurum, * ópera mánuum hóminum.
}\switchcolumn\portugues{
Os ídolos das gentes são prata e ouro, * obras das mãos dos homens.
}\switchcolumn*\latim{
Os habent, et non loquéntur: * óculos habent, et non vidébunt.
}\switchcolumn\portugues{
Têm boca e não falam: * têm olhos e não vêem.
}\switchcolumn*\latim{
Aures habent, et non áudient: * neque enim est spíritus in ore ipsórum.
}\switchcolumn\portugues{
Têm ouvidos e não ouvem: * pois na sua boca nem há qualquer respiração.
}\switchcolumn*\latim{
Símiles illis fiant qui fáciunt ea: * et omnes qui confídunt in eis.
}\switchcolumn\portugues{
Sejam semelhantes a eles os que os fazem: * e todos os que confiam neles.
}\switchcolumn*\latim{
Domus Israël, benedícite Dómino: * domus Aaron, benedícite Dómino.
}\switchcolumn\portugues{
Bendizei o Senhor, ó casa de Israel: * bendizei o Senhor, ó casa de Arão.
}\switchcolumn*\latim{
Domus Levi, benedícite Dómino: * qui timétis Dóminum, benedícite Dómino.
}\switchcolumn\portugues{
Bendizei o Senhor, ó casa de Levi: * vós, que temeis o Senhor, bendizei o Senhor.
}\switchcolumn*\latim{
Benedíctus Dóminus ex Sion, * qui hábitat in Jerúsalem.
}\switchcolumn\portugues{
Desde Sião seja bendito o Senhor, * que habita em Jerusalém.
}\end{paracol}


\subsectioninfo{Salmo 135}{Confitemini Domino, quoniam bonus, quoniam in æternum}\label{salmo135}
\begin{paracol}{2}\latim{
\rlettrine{C}{onfitémini} Dómino quóniam bonus: * quóniam in ætérnum misericórdia ejus.
}\switchcolumn\portugues{
\rlettrine{G}{lorificai} o Senhor, porque é bom: * pois eterna é a sua misericórdia.
}\switchcolumn*\latim{
Confitémini Deo deórum: * quóniam in ætérnum misericórdia ejus.
}\switchcolumn\portugues{
Glorificai o Deus dos deuses: * pois eterna é a sua misericórdia.
}\switchcolumn*\latim{
Confitémini Dómino dominórum: * quóniam in ætérnum misericórdia ejus.
}\switchcolumn\portugues{
Glorificai o Senhor dos senhores: * pois eterna é a sua misericórdia.
}\switchcolumn*\latim{
Qui facit mirabília magna solus: * quóniam in ætérnum misericórdia ejus.
}\switchcolumn\portugues{
O único que faz grandes maravilhas: * pois eterna é a sua misericórdia.
}\switchcolumn*\latim{
Qui fecit cælos in intelléctu: * quóniam in ætérnum misericórdia ejus.
}\switchcolumn\portugues{
O que fez os céus com sabedoria: * pois eterna é a sua misericórdia.
}\switchcolumn*\latim{
Qui firmávit terram super aquas: * quóniam in ætérnum misericórdia ejus.
}\switchcolumn\portugues{
O que firmou a terra sobre as águas: * pois eterna é a sua misericórdia.
}\switchcolumn*\latim{
Qui fecit luminária magna: * quóniam in ætérnum misericórdia ejus.
}\switchcolumn\portugues{
O que fez os grandes luminares: * pois eterna é a sua misericórdia.
}\switchcolumn*\latim{
Solem in potestátem diéi: * quóniam in ætérnum misericórdia ejus.
}\switchcolumn\portugues{
O sol para presidir ao dia: * pois eterna é a sua misericórdia.
}\switchcolumn*\latim{
Lunam, et stellas in potestátem noctis: * quóniam in ætérnum misericórdia ejus.
}\switchcolumn\portugues{
A lua e as estrelas para presidirem à noite: * pois eterna é a sua misericórdia.
}\switchcolumn*\latim{
Qui percússit Ægýptum cum primogénitis eórum: * quóniam in ætérnum misericórdia ejus.
}\switchcolumn\portugues{
O que feriu o Egipto com seus primogénitos: * pois eterna é a sua misericórdia.
}\switchcolumn*\latim{
Qui edúxit Israël de médio eórum: * quóniam in ætérnum misericórdia ejus.
}\switchcolumn\portugues{
O que tirou Israel do meio deles: * pois eterna é a sua misericórdia.
}\switchcolumn*\latim{
In manu poténti, et brácchio excélso: * quóniam in ætérnum misericórdia ejus.
}\switchcolumn\portugues{
Com mão poderosa e braço levantado: * pois eterna é a sua misericórdia.
}\switchcolumn*\latim{
Qui divísit Mare Rubrum in divisiónes: * quóniam in ætérnum misericórdia ejus.
}\switchcolumn\portugues{
O que dividiu em duas partes o mar Vermelho: * pois eterna é a sua misericórdia.
}\switchcolumn*\latim{
Et edúxit Israël per médium ejus: * quóniam in ætérnum misericórdia ejus.
}\switchcolumn\portugues{
Fez passar Israel pelo meio dele: * pois eterna é a sua misericórdia.
}\switchcolumn*\latim{
Et excússit Pharaónem, et virtútem ejus in Mari Rubro: * quóniam in ætérnum misericórdia ejus.
}\switchcolumn\portugues{
Precipitou Faraó e o seu exército no mar Vermelho: * pois eterna é a sua misericórdia.
}\switchcolumn*\latim{
Qui tradúxit pópulum suum per desértum: * quóniam in ætérnum misericórdia ejus.
}\switchcolumn\portugues{
O que conduziu o seu povo pelo deserto: * pois eterna é a sua misericórdia.
}\switchcolumn*\latim{
Qui percússit reges magnos: * quóniam in ætérnum misericórdia ejus.
}\switchcolumn\portugues{
O que feriu grandes reis: * pois eterna é a sua misericórdia.
}\switchcolumn*\latim{
Et occídit reges fortes: * quóniam in ætérnum misericórdia ejus.
}\switchcolumn\portugues{
Matou reis fortes: * pois eterna é a sua misericórdia.
}\switchcolumn*\latim{
Sehon, regem Amorrhæórum: * quóniam in ætérnum misericórdia ejus.
}\switchcolumn\portugues{
Seon, rei dos Amorreus: * pois eterna é a sua misericórdia.
}\switchcolumn*\latim{
Et Og, regem Basan: * quóniam in ætérnum misericórdia ejus.
}\switchcolumn\portugues{
A Ogue, rei de Basã: * pois eterna é a sua misericórdia.
}\switchcolumn*\latim{
Et dedit terram eórum hereditátem: * quóniam in ætérnum misericórdia ejus.
}\switchcolumn\portugues{
Deu a terra deles em herança: * pois eterna é a sua misericórdia.
}\switchcolumn*\latim{
Hereditátem Israël, servo suo: * quóniam in ætérnum misericórdia ejus.
}\switchcolumn\portugues{
Em herança a Israel, seu servo: * pois eterna é a sua misericórdia.
}\switchcolumn*\latim{
Quia in humilitáte nostra memor fuit nostri: * quóniam in ætérnum misericórdia ejus.
}\switchcolumn\portugues{
Em nosso abatimento de nós se lembrou: * pois eterna é a sua misericórdia.
}\switchcolumn*\latim{
Et redémit nos ab inimícis nostris: * quóniam in ætérnum misericórdia ejus.
}\switchcolumn\portugues{
Livrou-nos dos nossos inimigos: * pois eterna é a sua misericórdia.
}\switchcolumn*\latim{
Qui dat escam omni carni: * quóniam in ætérnum misericórdia ejus.
}\switchcolumn\portugues{
O que dá alimento a toda a carne: * pois eterna é a sua misericórdia.
}\switchcolumn*\latim{
Confitémini Deo cæli: * quóniam in ætérnum misericórdia ejus.
}\switchcolumn\portugues{
Dai glória a Deus do céu: * pois eterna é a sua misericórdia.
}\switchcolumn*\latim{
Confitémini Dómino dominórum: * quóniam in ætérnum misericórdia ejus.
}\switchcolumn\portugues{
Dai glória ao Senhor dos senhores: * pois eterna é a sua misericórdia.
}\end{paracol}


\subsectioninfo{Salmo 136}{Super flumina Babylonis}\label{salmo136}
\begin{paracol}{2}\latim{
\rlettrine{S}{uper} flúmina Babylónis, illic sédimus et flévimus: * cum recordarémur Sion:
}\switchcolumn\portugues{
\qlettrine{J}{unto} dos rios da Babilónia, ali nos assentámos a chorar: * lembrando-nos de Sião:
}\switchcolumn*\latim{
In salícibus in médio ejus, * suspéndimus órgana nostra.
}\switchcolumn\portugues{
Nos salgueiros que lá havia, * as nossas harpas pendurámos.
}\switchcolumn*\latim{
Quia illic interrogavérunt nos, qui captívos duxérunt nos, * verba cantiónum:
}\switchcolumn\portugues{
Os mesmos que nos tinham levado cativos pediam-nos, * palavras de canções:
}\switchcolumn*\latim{
Et qui abduxérunt nos: * Hymnum cantáte nobis de cánticis Sion.
}\switchcolumn\portugues{
Os que à força nos tinham levado diziam: * cantai-nos um hino dos cânticos de Sião.
}\switchcolumn*\latim{
Quómodo cantábimus cánticum Dómini * in terra aliéna?
}\switchcolumn\portugues{
Como poderíamos nós cantar o cântico do Senhor * em estranha terra?
}\switchcolumn*\latim{
Si oblítus fúero tui, Jerúsalem, * oblivióni detur déxtera mea.
}\switchcolumn\portugues{
Se me esquecer de ti, ó Jerusalém, * ao esquecimento seja entregue a minha direita.
}\switchcolumn*\latim{
Adhǽreat lingua mea fáucibus meis, * si non memínero tui:
}\switchcolumn\portugues{
Apegue-se-me a língua à garganta, * se eu me não lembrar de ti:
}\switchcolumn*\latim{
Si non proposúero Jerúsalem, * in princípio lætítiæ meæ.
}\switchcolumn\portugues{
Não se propuser Jerusalém, * como o início da minha alegria.
}\switchcolumn*\latim{
Memor esto, Dómine, filiórum Edom, * in die Jerúsalem:
}\switchcolumn\portugues{
Lembrai-Vos, ó Senhor, dos filhos de Edom, * no dia de Jerusalém:
}\switchcolumn*\latim{
Qui dicunt: exinaníte, exinaníte * usque ad fundaméntum in ea.
}\switchcolumn\portugues{
Que diziam: arrasai, arrasai * até aos alicerces.
}\switchcolumn*\latim{
Fília Babylónis mísera: * beátus, qui retríbuet tibi retributiónem tuam, quam retribuísti nobis.
}\switchcolumn\portugues{
Ó desgraçada filha da Babilónia: * bem-aventurado o que te der a paga do que nos pagastes.
}\switchcolumn*\latim{
Beátus, qui tenébit, * et allídet párvulos tuos ad petram.
}\switchcolumn\portugues{
Bem-aventurado o que agarrar, * em teus filhinhos e os despedaçar contra um rochedo.
}\end{paracol}


\subsectioninfo{Salmo 137}{Confitebor tibi, Domine}\label{salmo137}
\begin{paracol}{2}\latim{
\rlettrine{C}{onfitébor} tibi, Dómine, in toto corde meo: * quóniam audísti verba oris mei.
}\switchcolumn\portugues{
\rlettrine{E}{u} Vos glorificarei, ó Senhor, de todo o coração: * porque ouvistes as palavras da minha boca.
}\switchcolumn*\latim{
In conspéctu Angelórum psallam tibi: * adorábo ad templum sanctum tuum, et confitébor nómini tuo.
}\switchcolumn\portugues{
Em presença dos anjos Vos cantarei salmos: * Vos adorarei no vosso santo templo e glorificarei o vosso nome.
}\switchcolumn*\latim{
Super misericórdia tua, et veritáte tua: * quóniam magnificásti super omne, nomen sanctum tuum.
}\switchcolumn\portugues{
Por causa de vossa misericórdia e de vossa verdade: * porque engrandecestes o vosso santo nome sobre tudo.
}\switchcolumn*\latim{
In quacúmque die invocávero Te, exáudi me: * multiplicábis in ánima mea virtútem.
}\switchcolumn\portugues{
Em qualquer dia que Vos invocar, ouvi-me: * Vós aumentareis a fortaleza na minha alma.
}\switchcolumn*\latim{
Confiteántur tibi, Dómine, omnes reges terræ: * quia audiérunt ómnia verba oris tui:
}\switchcolumn\portugues{
Louvem-Vos, ó Senhor, todos os reis da terra: * pois ouviram todas as palavras de vossa boca:
}\switchcolumn*\latim{
Et cantent in viis Dómini: * quóniam magna est glória Dómini.
}\switchcolumn\portugues{
Cantem nos caminhos do Senhor: * porque a glória do Senhor é grande.
}\switchcolumn*\latim{
Quóniam excélsus Dóminus, et humília réspicit: * et alta a longe cognóscit.
}\switchcolumn\portugues{
Porque, sendo o Senhor excelso, todavia, olha os humildes: * e conhece de longe os altos.
}\switchcolumn*\latim{
Si ambulávero in médio tribulatiónis, vivificábis me: * et super iram inimicórum meórum extendísti manum tuam, et salvum me fecit déxtera tua.
}\switchcolumn\portugues{
Se andar no meio da tribulação, me dareis a vida: * estendestes a vossa mão contra a ira dos meus inimigos e a vossa direita me salvou.
}\switchcolumn*\latim{
Dóminus retríbuet pro me: * Dómine, misericórdia tua in sǽculum: ópera mánuum tuárum ne despícias.
}\switchcolumn\portugues{
O Senhor retribuirá por mim: * ó Senhor, a vossa misericórdia é eterna: não desprezeis as obras de vossas mãos.
}\end{paracol}


\subsectioninfo{Salmo 138}{Domine, probasti me}\label{salmo138}
\begin{paracol}{2}\latim{
\rlettrine{D}{ómine,} probásti me, et cognovísti me: * Tu cognovísti sessiónem meam, et resurrectiónem meam.
}\switchcolumn\portugues{
\rlettrine{S}{enhor,} provastes-me e conhecestes-me: * Vós sabeis quando me sento e quando me levanto.
}\switchcolumn*\latim{
Intellexísti cogitatiónes meas de longe: * sémitam meam, et funículum meum investigásti.
}\switchcolumn\portugues{
De longe penetrastes os meus pensamentos: * a minha vereda e averiguastes os meus passos.
}\switchcolumn*\latim{
Et omnes vias meas prævidísti: * quia non est sermo in lingua mea.
}\switchcolumn\portugues{
Previstes todos meus caminhos: * pois nenhuma palavra estava na minha língua.
}\switchcolumn*\latim{
Ecce, Dómine, Tu cognovísti ómnia novíssima, et antíqua: * Tu formásti me, et posuísti super me manum tuam.
}\switchcolumn\portugues{
Eis, ó Senhor, que conhecestes todas as cousas, as novíssimas e as antigas: * Vós me formastes e pusestes sobre mim a vossa mão.
}\switchcolumn*\latim{
Mirábilis facta est sciéntia tua ex me: * confortáta est, et non pótero ad eam.
}\switchcolumn\portugues{
Maravilhosa acima de mim se mostrou a vossa ciência: * é sublime e não poderei atingi-la.
}\switchcolumn*\latim{
Quo ibo a spíritu tuo? * Et quo a fácie tua fúgiam?
}\switchcolumn\portugues{
Para onde irei de vosso espírito? * E para onde fugirei de vossa presença?
}\switchcolumn*\latim{
Si ascéndero in cælum, Tu illic es: * si descéndero in inférnum, ades.
}\switchcolumn\portugues{
Se subo ao céu, Vós lá estais: * se desço ao inferno, n’Ele Vos encontrais presente.
}\switchcolumn*\latim{
Si súmpsero pennas meas dilúculo, * et habitávero in extrémis maris:
}\switchcolumn\portugues{
Se levar as minhas asas pela aurora, * e habitar nas extremidades do mar:
}\switchcolumn*\latim{
Étenim illuc manus tua dedúcet me: * et tenébit me déxtera tua.
}\switchcolumn\portugues{
Ainda lá me guiará a vossa mão: * e me susterá a vossa direita.
}\switchcolumn*\latim{
Et dixi: fórsitan ténebræ conculcábunt me: * et nox illuminátio mea in delíciis meis.
}\switchcolumn\portugues{
Disse: talvez me cubrirão as trevas: * e a noite será claridade nos meus deleites.
}\switchcolumn*\latim{
Quia ténebræ non obscurabúntur a Te, et nox sicut dies illuminábitur: * sicut ténebræ ejus, ita et lumen ejus.
}\switchcolumn\portugues{
Pois as trevas não são escuras para Vós, a noite brilha como o dia: * como são as trevas para Vós, assim é a luz.
}\switchcolumn*\latim{
Quia Tu possedísti renes meos: * suscepísti me de útero matris meæ.
}\switchcolumn\portugues{
Pois Vós possuístes os meus afectos: * recebestes-me desde o ventre de minha mãe.
}\switchcolumn*\latim{
Confitébor tibi quia terribíliter magnificátus es: * mirabília ópera tua, et ánima mea cognóscit nimis.
}\switchcolumn\portugues{
Vos glorificarei, pois sois terrivelmente magnífico: * maravilhosas são as vossas obras e a minha alma o bem sabe.
}\switchcolumn*\latim{
Non est occultátum os meum a Te, quod fecísti in occúlto: * et substántia mea in inferióribus terræ.
}\switchcolumn\portugues{
Meus ossos, que formastes em segredo, Vos não são ocultos: * nem a minha substância nas entranhas da terra.
}\switchcolumn*\latim{
Imperféctum meum vidérunt óculi tui, et in libro tuo omnes scribéntur: * dies formabúntur, et nemo in eis.
}\switchcolumn\portugues{
Vossos olhos me viram em bruto e no vosso livro todos estão escritos: * num dia serão criados, mas deles nem um.
}\switchcolumn*\latim{
Mihi autem nimis honorificáti sunt amíci tui, Deus: * nimis confortátus est principátus eórum.
}\switchcolumn\portugues{
Vejo, contudo, ó Deus, que singularmente honrastes os vossos amigos: * muito se fortaleceu o seu principado.
}\switchcolumn*\latim{
Dinumerábo eos, et super arénam multiplicabúntur: * exsurréxi, et adhuc sum tecum.
}\switchcolumn\portugues{
Contá-los-ei e multiplicar-se-ão mais que a areia: * despertei e ainda estou convosco.
}\switchcolumn*\latim{
Si occíderis, Deus, peccatóres: * viri sánguinum, declináte a me:
}\switchcolumn\portugues{
Se matares os pecadores, ó Deus: * ó varões sanguinários, retirai-vos de mim:
}\switchcolumn*\latim{
Quia dícitis in cogitatióne: * Accípient in vanitáte civitátes tuas.
}\switchcolumn\portugues{
Pois dizeis no vosso pensamento: * tomarão em vão as vossas cidades.
}\switchcolumn*\latim{
Nonne qui odérunt Te, Dómine, óderam? * Et super inimícos tuos tabescébam?
}\switchcolumn\portugues{
Não odiei eu, ó Senhor, os que Vos odiavam? * Me não desgastava eu devido aos vossos inimigos?
}\switchcolumn*\latim{
Perfécto ódio óderam illos: * et inimíci facti sunt mihi.
}\switchcolumn\portugues{
Com ódio perfeito os odiei: * e eles tornaram-se meus inimigos.
}\switchcolumn*\latim{
Proba me, Deus, et scito cor meum: * intérroga me, et cognósce sémitas meas.
}\switchcolumn\portugues{
Provai-me, ó Deus, e sondai o meu coração: * interrogai-me e conhecei os meus caminhos.
}\switchcolumn*\latim{
Et vide, si via iniquitátis in me est: * et deduc me in via ætérna.
}\switchcolumn\portugues{
Vede se há em mim caminho de iniquidade: * e conduzi-me pelo caminho eterno.
}\end{paracol}


\subsectioninfo{Salmo 139}{Eripe me, Domine}\label{salmo139}
\begin{paracol}{2}\latim{
\slettrine{É}{ripe} me, Dómine, ab hómine malo: * a viro iníquo éripe me.
}\switchcolumn\portugues{
\rlettrine{L}{ivrai-me,} ó Senhor, do homem malvado: * livrai-me do homem iníquo.
}\switchcolumn*\latim{
Qui cogitavérunt iniquitátes in corde: * tota die constituébant prǽlia.
}\switchcolumn\portugues{
Maquinam iniquidades no coração: * todo o dia armam combates.
}\switchcolumn*\latim{
Acuérunt linguas suas sicut serpéntis: * venénum áspidum sub lábiis eórum.
}\switchcolumn\portugues{
Afiaram as suas línguas como serpentes: * têm veneno de áspides debaixo de seus lábios.
}\switchcolumn*\latim{
Custódi me, Dómine, de manu peccatóris: * et ab homínibus iníquis éripe me.
}\switchcolumn\portugues{
Guardai-me, ó Senhor, da mão do pecador: * e livrai-me dos homens iníquos.
}\switchcolumn*\latim{
Qui cogitavérunt supplantáre gressus meos: * abscondérunt supérbi láqueum mihi:
}\switchcolumn\portugues{
Que planearam derrubar os meus passos: * os soberbos me armaram ocultamente um laço.
}\switchcolumn*\latim{
Et funes extendérunt in láqueum: * juxta iter scándalum posuérunt mihi.
}\switchcolumn\portugues{
Estenderam redes para o embuste: * junto do caminho me colocavam obstáculos.
}\switchcolumn*\latim{
Dixi Dómino: Deus meus es Tu: * exáudi, Dómine, vocem deprecatiónis meæ.
}\switchcolumn\portugues{
Disse ao Senhor: Vós sois o meu Deus: * atendei, ó Senhor, à voz da minha súplica.
}\switchcolumn*\latim{
Dómine, Dómine, virtus salútis meæ: * obumbrásti super caput meum in die belli.
}\switchcolumn\portugues{
Senhor, ó Senhor, fortaleza da minha salvação: * cobristes a minha cabeça no dia da batalha.
}\switchcolumn*\latim{
Ne tradas me, Dómine, a desidério meo peccatóri: * cogitavérunt contra me, ne derelínquas me, ne forte exalténtur.
}\switchcolumn\portugues{
Contra o meu desejo me não entregueis ao pecador, Senhor: * eles maquinaram contra mim, me não desampareis, para que se não exultem.
}\switchcolumn*\latim{
Caput circúitus eórum: * labor labiórum ipsórum opériet eos.
}\switchcolumn\portugues{
A cabeça daqueles que me cercam: *  o trabalho dos seus lábios os cobrirá.
}\switchcolumn*\latim{
Cadent super eos carbónes, in ignem deícies eos: * in misériis non subsístent.
}\switchcolumn\portugues{
Cairão sobre eles brasas, no fogo os lançareis: * não subsistirão nas misérias.
}\switchcolumn*\latim{
Vir linguósus non dirigétur in terra: * virum injústum mala cápient in intéritu.
}\switchcolumn\portugues{
O varão caluniador não prosperará sobre a terra: * o varão injusto caçará o mal até à morte.
}\switchcolumn*\latim{
Cognóvi quia fáciet Dóminus judícium ínopis: * et vindíctam páuperum.
}\switchcolumn\portugues{
Sei que o Senhor fará justiça ao desvalido: * e que vingará os pobres.
}\switchcolumn*\latim{
Verúmtamen justi confitebúntur nómini tuo: * et habitábunt recti cum vultu tuo.
}\switchcolumn\portugues{
Contudo, os justos glorificarão o vosso nome: * e os rectos habitarão na vossa presença.
}\end{paracol}


\subsectioninfo{Salmo 140}{Domine, clamavi ad Te}\label{salmo140}
\begin{paracol}{2}\latim{
\rlettrine{D}{ómine,} clamávi ad Te, exáudi me: * inténde voci meæ, cum clamávero ad Te.
}\switchcolumn\portugues{
\rlettrine{A}{} Vós clamei, ó Senhor, ouvi-me: * atendei à minha voz, quando a Vós clamo.
}\switchcolumn*\latim{
Dirigátur orátio mea sicut incénsum in conspéctu tuo: * elevátio mánuum meárum sacrifícium vespertínum.
}\switchcolumn\portugues{
Suba direita como incenso a minha oração na vossa presença: * seja a elevação das minhas mãos como o sacrifício da tarde.
}\switchcolumn*\latim{
Pone, Dómine, custódiam ori meo: * et óstium circumstántiæ lábiis meis.
}\switchcolumn\portugues{
Ponde uma guarda, ó Senhor, à minha boca: * e aos meus lábios uma porta que os feche.
}\switchcolumn*\latim{
Non declínes cor meum in verba malítiæ, * ad excusándas excusatiónes in peccátis.
}\switchcolumn\portugues{
Não deixais que meu coração se incline para palavras de malícia, * para buscar desculpas nos pecados.
}\switchcolumn*\latim{
Cum homínibus operántibus iniquitátem: * et non communicábo cum eléctis eórum.
}\switchcolumn\portugues{
Como fazem os homens que operam a iniquidade: * não quero ter parte nas suas escolhas.
}\switchcolumn*\latim{
Corrípiet me justus in misericórdia, et increpábit me: * óleum autem peccatóris non impínguet caput meum.
}\switchcolumn\portugues{
Corrija-me o justo e advirta-me com misericórdia: * mas o azeite do pecador não chegue a ungir a minha cabeça.
}\switchcolumn*\latim{
Quóniam adhuc et orátio mea in beneplácitis eórum: * absórpti sunt juncti petræ júdices eórum.
}\switchcolumn\portugues{
Porque até a minha oração é contra o que lhe agrada: * os seus juízes serão precipitados ao longo dos rochedos.
}\switchcolumn*\latim{
Audient verba mea quóniam potuérunt: * sicut crassitúdo terræ erúpta est super terram.
}\switchcolumn\portugues{
Ouvirão as minhas palavras porque elas são poderosas: * como o torrão se desfaz à flor do solo.
}\switchcolumn*\latim{
Dissipáta sunt ossa nostra secus inférnum: * quia ad Te, Dómine, Dómine, óculi mei: in Te sperávi, non áuferas ánimam meam.
}\switchcolumn\portugues{
Foram dispersos os nossos ossos junto do inferno: * mas para Vós, Senhor, ó Senhor, estão os meus olhos: em Vós tenho esperado, me não tireis a vida.
}\switchcolumn*\latim{
Custódi me a láqueo, quem statuérunt mihi: * et a scándalis operántium iniquitátem.
}\switchcolumn\portugues{
Guardai-me do laço que me armaram: * e das emboscadas dos que praticam a iniquidade.
}\switchcolumn*\latim{
Cadent in retiáculo ejus peccatóres: * singuláriter sum ego donec tránseam.
}\switchcolumn\portugues{
Os pecadores cairão na sua rede: * quanto a mim, estou só até conseguir passar.
}\end{paracol}


\subsectioninfo{Salmo 141}{Voce mea ad Dominum clamavi}\label{salmo141}
\begin{paracol}{2}\latim{
\rlettrine{V}{oce} mea ad Dóminum clamávi: * voce mea ad Dóminum deprecátus sum:
}\switchcolumn\portugues{
\rlettrine{C}{om} a minha voz clamei ao Senhor: * com minha voz supliquei ao Senhor:
}\switchcolumn*\latim{
Effúndo in conspéctu ejus oratiónem meam, * et tribulatiónem meam ante ipsum pronúntio.
}\switchcolumn\portugues{
Derramo na sua presença a minha oração, * e exponho diante d’Ele a minha tribulação.
}\switchcolumn*\latim{
In deficiéndo ex me spíritum meum, * et Tu cognovísti sémitas meas.
}\switchcolumn\portugues{
Quando o meu espírito foi desfalecendo, * Vós conhecestes as minhas veredas.
}\switchcolumn*\latim{
In via hac, qua ambulábam, * abscondérunt láqueum mihi.
}\switchcolumn\portugues{
No caminho por onde andava, * me armaram ocultos laços.
}\switchcolumn*\latim{
Considerábam ad déxteram, et vidébam: * et non erat qui cognósceret me.
}\switchcolumn\portugues{
Voltava-me para a minha direita e olhava: * e não havia quem me conhecesse.
}\switchcolumn*\latim{
Périit fuga a me, * et non est qui requírat ánimam meam.
}\switchcolumn\portugues{
Não me ficou possibilidade de fuga, * e não há quem se importe com minha vida.
}\switchcolumn*\latim{
Clamávi ad Te, Dómine, * dixi: Tu es spes mea, pórtio mea in terra vivéntium.
}\switchcolumn\portugues{
A Vós clamei, ó Senhor, * e disse: Vós sois a minha esperança, a minha porção na terra dos viventes.
}\switchcolumn*\latim{
Inténde ad deprecatiónem meam: * quia humiliátus sum nimis.
}\switchcolumn\portugues{
Atendei à minha súplica: * pois fui sumamente humilhado.
}\switchcolumn*\latim{
Líbera me a persequéntibus me: * quia confortáti sunt super me.
}\switchcolumn\portugues{
Livrai-me dos que me perseguem: * pois se tornaram mais fortes do que eu.
}\switchcolumn*\latim{
Educ de custódia ánimam meam ad confiténdum nómini tuo: * me exspéctant justi, donec retríbuas mihi.
}\switchcolumn\portugues{
Tirai a minha alma desta prisão para dar glória ao vosso nome: * estão-me esperando os justos, até que me façais justiça.
}\end{paracol}


\subsectioninfo{Salmo 142}{Domine, exaudi orationem meam}\label{salmo142}
\begin{paracol}{2}\latim{
\rlettrine{D}{ómine,} exáudi oratiónem meam: áuribus pércipe obsecratiónem meam in veritáte tua: * exáudi me in tua justítia.
}\switchcolumn\portugues{
\rlettrine{O}{uvi,} ó Senhor, a minha oração, prestai ouvidos aos meus rogos, segundo a vossa verdade: * atendei-me na vossa justiça.
}\switchcolumn*\latim{
Et non intres in judícium cum servo tuo: * quia non justificábitur in conspéctu tuo omnis vivens.
}\switchcolumn\portugues{
Não entreis em juízo com vosso servo: * pois nem um vivente será justificado na vossa presença.
}\switchcolumn*\latim{
Quia persecútus est inimícus ánimam meam: * humiliávit in terra vitam meam.
}\switchcolumn\portugues{
Pois o inimigo perseguiu a minha alma: * humilhou a minha vida até ao chão.
}\switchcolumn*\latim{
Collocávit me in obscúris sicut mórtuos sǽculi: * et anxiátus est super me spíritus meus, in me turbátum est cor meum.
}\switchcolumn\portugues{
Colocou-me na escuridão como a dos mortos de séculos: * e está angustiado sobre mim o meu espírito, em mim se turvou meu coração.
}\switchcolumn*\latim{
Memor fui diérum antiquórum, meditátus sum in ómnibus opéribus tuis: * in factis mánuum tuárum meditábar.
}\switchcolumn\portugues{
Tenho recordado os dias antigos, meditei em todas vossas obras: * meditei nas obras de vossas mãos.
}\switchcolumn*\latim{
Expándi manus meas ad Te: * ánima mea sicut terra sine aqua tibi.
}\switchcolumn\portugues{
Estendi as minhas mãos para Vós: * a minha alma ante Vós é como terra sedenta.
}\switchcolumn*\latim{
Velóciter exáudi me, Dómine: * defécit spíritus meus.
}\switchcolumn\portugues{
Atendei-me depressa, ó Senhor: * o meu espírito desfaleceu.
}\switchcolumn*\latim{
Non avértas fáciem tuam a me: * et símilis ero descendéntibus in lacum.
}\switchcolumn\portugues{
Não afasteis de mim a vossa face: * para que não seja semelhante aos que descem ao abysmo.
}\switchcolumn*\latim{
Audítam fac mihi mane misericórdiam tuam: * quia in Te sperávi.
}\switchcolumn\portugues{
Desde a manhã fazei-me sentir a vossa misericórdia: * pois em Vós tenho esperado.
}\switchcolumn*\latim{
Notam fac mihi viam, in qua ámbulem: * quia ad Te levávi ánimam meam.
}\switchcolumn\portugues{
Fazei-me conhecer o caminho em que hei-de andar: * pois a Vós elevei a minha alma.
}\switchcolumn*\latim{
Éripe me de inimícis meis, Dómine, ad Te confúgi: * doce me fácere voluntátem tuam, quia Deus meus es Tu.
}\switchcolumn\portugues{
Livrai-me dos meus inimigos, ó Senhor, junto de Vós me refugio: * ensinai-me a fazer a vossa vontade, pois Vós sois o meu Deus.
}\switchcolumn*\latim{
Spíritus tuus bonus dedúcet me in terram rectam: * propter nomen tuum, Dómine, vivificábis me, in æquitáte tua.
}\switchcolumn\portugues{
Vosso bom espírito conduzir-me-á à terra da rectidão: * por causa de vosso nome, ó Senhor, dar-me-eis a vida.
}\switchcolumn*\latim{
Edúces de tribulatióne ánimam meam: * et in misericórdia tua dispérdes inimícos meos.
}\switchcolumn\portugues{
Tirareis a minha alma da tribulação: * e pela vossa misericórdia, dissipareis os meus inimigos.
}\switchcolumn*\latim{
Et perdes omnes, qui tríbulant ánimam meam: * quóniam ego servus tuus sum.
}\switchcolumn\portugues{
Destruireis todos os que atribulam a minha alma: * porque eu sou vosso servo.
}\end{paracol}


\subsectioninfo{Salmo 143}{Benedictus Dominus}\label{salmo143}
\begin{paracol}{2}\latim{
\rlettrine{B}{enedíctus} Dóminus, Deus meus, qui docet manus meas ad prǽlium, * et dígitos meos ad bellum.
}\switchcolumn\portugues{
\rlettrine{B}{endito} seja o Senhor meu Deus, que adestra minhas mãos para a batalha, * e meus dedos para a guerra.
}\switchcolumn*\latim{
Misericórdia mea, et refúgium meum: * suscéptor meus, et liberátor meus:
}\switchcolumn\portugues{
Minha misericórdia e meu refúgio: * meu defensor e meu libertador:
}\switchcolumn*\latim{
Protéctor meus, et in ipso sperávi: * qui subdit pópulum meum sub me.
}\switchcolumn\portugues{
Meu protector e é n’Ele que espero: * quem submete o meu povo sob mim.
}\switchcolumn*\latim{
Dómine, quid est homo, quia innotuísti ei? * aut fílius hóminis, quia réputas eum?
}\switchcolumn\portugues{
Que é o homem, ó Senhor, para que a ele Vos tenhais manifestado? * Ou o filho do homem, para assim o estimardes?
}\switchcolumn*\latim{
Homo vanitáti símilis factus est: * dies ejus sicut umbra prætéreunt.
}\switchcolumn\portugues{
O homem fez-se semelhante à vaidade: * seus dias passam como a sombra.
}\switchcolumn*\latim{
Dómine, inclína cælos tuos, et descénde: * tange montes, et fumigábunt.
}\switchcolumn\portugues{
Senhor, inclinai os vossos céus e descei: * tocai os montes e fumegarão.
}\switchcolumn*\latim{
Fúlgura coruscatiónem, et dissipábis eos: * emítte sagíttas tuas, et conturbábis eos.
}\switchcolumn\portugues{
Desferi raios e os dissipareis: * disparai as vossas setas e conturbá-los-eis.
}\switchcolumn*\latim{
Emítte manum tuam de alto, éripe me, et líbera me de aquis multis: * de manu filiórum alienórum.
}\switchcolumn\portugues{
Enviai a vossa mão lá do alto, tirai-me e livrai-me das muitas águas: * da mão dos estranhos. filhos
}\switchcolumn*\latim{
Quorum os locútum est vanitátem: * et déxtera eórum, déxtera iniquitátis.
}\switchcolumn\portugues{
Cuja boca falou vaidade: * e cuja direita é uma direita de iniquidade.
}\switchcolumn*\latim{
Deus, cánticum novum cantábo tibi: * in psaltério decachórdo psallam tibi.
}\switchcolumn\portugues{
A Vós, ó Deus, cantarei um cântico novo: * com o saltério de dez cordas Vos louvarei.
}\switchcolumn*\latim{
Qui das salútem régibus: * qui redemísti David, servum tuum, de gládio malígno: éripe me.
}\switchcolumn\portugues{
Vós que dais saúde aos reis: * que livrastes vosso servo David da espada maligna: livrai-me.
}\switchcolumn*\latim{
Et érue me de manu filiórum alienórum, quorum os locútum est vanitátem: * et déxtera eórum, déxtera iniquitátis:
}\switchcolumn\portugues{
E tirai-me da mão dos filhos estranhos, cuja boca falou vaidade: * e cuja direita é uma direita de iniquidade.
}\switchcolumn*\latim{
Quorum fílii, sicut novéllæ plantatiónes * in juventúte sua.
}\switchcolumn\portugues{
Cujos filhos são como novas plantas * na sua mocidade.
}\switchcolumn*\latim{
Fíliæ eórum compósitæ: * circumornátæ ut similitúdo templi.
}\switchcolumn\portugues{
Suas filhas decoradas: * adornadas como um templo.
}\switchcolumn*\latim{
Promptuária eórum plena, * eructántia ex hoc in illud.
}\switchcolumn\portugues{
Seus celeiros estão cheios, * a transbordar duns para outros.
}\switchcolumn*\latim{
Oves eórum fœtósæ, abundántes in egréssibus suis: * boves eórum crassæ.
}\switchcolumn\portugues{
Suas ovelhas são fecundas, vão pastar abundantemente: * as suas vacas são gordas.
}\switchcolumn*\latim{
Non est ruína macériæ, neque tránsitus: * neque clamor in platéis eórum.
}\switchcolumn\portugues{
Não há ruína de muro, nem passagem na sua cerca: * nem gritos nas suas praças.
}\switchcolumn*\latim{
Beátum dixérunt pópulum, cui hæc sunt: * beátus pópulus, cujus Dóminus Deus ejus.
}\switchcolumn\portugues{
Bem-aventurado chamarão ao povo que tem estes bens: * bem-aventurado o povo que tem o Senhor por seu Deus.
}\end{paracol}


\subsectioninfo{Salmo 144}{Exaltabo Te, Deus meus rex}\label{salmo144}
\begin{paracol}{2}\latim{
\rlettrine{E}{xaltábo} Te, Deus meus, rex: * et benedícam nómini tuo in sǽculum, et in sǽculum sǽculi.
}\switchcolumn\portugues{
\rlettrine{E}{u} Vos exaltarei, ó Deus meu rei: * e bendirei o vosso nome para sempre e pelos séculos dos séculos.
}\switchcolumn*\latim{
Per síngulos dies benedícam tibi: * et laudábo nomen tuum in sǽculum, et in sǽculum sǽculi.
}\switchcolumn\portugues{
Cada dia Vos bendirei: * e louvarei o vosso nome para sempre e pelos séculos dos séculos.
}\switchcolumn*\latim{
Magnus Dóminus, et laudábilis nimis: * et magnitúdinis ejus non est finis.
}\switchcolumn\portugues{
Grande é o Senhor e digníssimo de louvor: * e a sua grandeza não tem limites.
}\switchcolumn*\latim{
Generátio et generátio laudábit ópera tua: * et poténtiam tuam pronuntiábunt.
}\switchcolumn\portugues{
Todas as gerações louvarão as vossas obras: * e publicarão o vosso poder.
}\switchcolumn*\latim{
Magnificéntiam glóriæ sanctitátis tuæ loquéntur: * et mirabília tua narrábunt.
}\switchcolumn\portugues{
Falarão da magnificência da glória de vossa santidade: * e contarão as vossas maravilhas.
}\switchcolumn*\latim{
Et virtútem terribílium tuórum dicent: * et magnitúdinem tuam narrábunt.
}\switchcolumn\portugues{
Dirão quanto é terrível o vosso poder: * e contarão a vossa grandeza.
}\switchcolumn*\latim{
Memóriam abundántiæ suavitátis tuæ eructábunt: * et justítia tua exsultábunt.
}\switchcolumn\portugues{
Expandir-se-ão na lembrança de vossa imensa bondade: * e exultarão com vossa justiça.
}\switchcolumn*\latim{
Miserátor, et miséricors Dóminus: * pátiens, et multum miséricors.
}\switchcolumn\portugues{
Clemente e misericordioso é o Senhor: * paciente e muito misericordioso.
}\switchcolumn*\latim{
Suávis Dóminus univérsis: * et miseratiónes ejus super ómnia ópera ejus.
}\switchcolumn\portugues{
Suave é o Senhor para com todos: * e as suas misericórdias sobre todas suas obras.
}\switchcolumn*\latim{
Confiteántur tibi, Dómine, ómnia ópera tua: * et sancti tui benedícant tibi.
}\switchcolumn\portugues{
Dêem-Vos glória, ó Senhor, todas vossas obras: * e Vos bendigam os vossos santos.
}\switchcolumn*\latim{
Glóriam regni tui dicent: * et poténtiam tuam loquéntur:
}\switchcolumn\portugues{
Eles publicarão a glória de vosso reino: * e falarão de vosso poder:
}\switchcolumn*\latim{
Ut notam fáciant fíliis hóminum poténtiam tuam: * et glóriam magnificéntiæ regni tui.
}\switchcolumn\portugues{
Para fazerem conhecer aos filhos dos homens o vosso poder: * e a gloriosa magnificência de vosso reino.
}\switchcolumn*\latim{
Regnum tuum regnum ómnium sæculórum: * et dominátio tua in omni generatióne et generatiónem.
}\switchcolumn\portugues{
Vosso reino é um reino que se estende a todos os séculos: * e vosso império a todas as gerações.
}\switchcolumn*\latim{
Fidélis Dóminus in ómnibus verbis suis: * et sanctus in ómnibus opéribus suis.
}\switchcolumn\portugues{
O Senhor é fiel em todas suas palavras: * e santo em todas suas obras.
}\switchcolumn*\latim{
Allevat Dóminus omnes qui córruunt: * et érigit omnes elísos.
}\switchcolumn\portugues{
O Senhor sustém todos os que estão para cair: * e levanta todos os prostrados.
}\switchcolumn*\latim{
Óculi ómnium in Te sperant, Dómine: * et Tu das escam illórum in témpore opportúno.
}\switchcolumn\portugues{
Os olhos de todos esperam em Vós, ó Senhor: * e Vós lhes dais o sustento em tempo oportuno.
}\switchcolumn*\latim{
Aperis Tu manum tuam: * et imples omne ánimal benedictióne.
}\switchcolumn\portugues{
Vós abris a vossa mão: * e encheis de bênção todos os viventes.
}\switchcolumn*\latim{
Justus Dóminus in ómnibus viis suis: * et sanctus in ómnibus opéribus suis.
}\switchcolumn\portugues{
Justo é o Senhor em todos seus caminhos: * e santo em todas suas obras.
}\switchcolumn*\latim{
Prope est Dóminus ómnibus invocántibus eum: * ómnibus invocántibus eum in veritáte.
}\switchcolumn\portugues{
O Senhor está perto de todos os que O invocam: * de todos os que O invocam com verdade.
}\switchcolumn*\latim{
Voluntátem timéntium se fáciet: * et deprecatiónem eórum exáudiet: et salvos fáciet eos.
}\switchcolumn\portugues{
Ele fará a vontade dos que O temem: * atenderá a sua oração e salvá-los-á.
}\switchcolumn*\latim{
Custódit Dóminus omnes diligéntes se: * et omnes peccatóres dispérdet.
}\switchcolumn\portugues{
O Senhor guarda todos os que O amam: * e exterminará todos os pecadores.
}\switchcolumn*\latim{
Laudatiónem Dómini loquétur os meum: * et benedícat omnis caro nómini sancto ejus in sǽculum, et in sǽculum sǽculi.
}\switchcolumn\portugues{
Minha boca publicará o louvor do Senhor: * e bendiga toda a carne o seu santo nome, para sempre e pelos séculos dos séculos.
}\end{paracol}


\subsectioninfo{Salmo 145}{Lauda, anima mea, Dominum}\label{salmo145}
\begin{paracol}{2}\latim{
\rlettrine{L}{auda,} ánima mea, Dóminum, laudábo Dóminum in vita mea: * psallam Deo meo quámdiu fúero.
}\switchcolumn\portugues{
\rlettrine{L}{ouva} o Senhor, ó minha alma, louvarei o Senhor durante a minha vida: * cantarei salmos ao meu Deus até perecer.
}\switchcolumn*\latim{
Nolíte confídere in princípibus: * in fíliis hóminum, in quibus non est salus.
}\switchcolumn\portugues{
Não confies nos príncipes: * nem nos filhos dos homens, em quem não há salvação.
}\switchcolumn*\latim{
Exíbit spíritus ejus, et revertétur in terram suam: * in illa die períbunt omnes cogitatiónes eórum.
}\switchcolumn\portugues{
Sairá o seu espírito e retornará à sua terra: * nesse dia se desvanecerão todos seus desígnios.
}\switchcolumn*\latim{
Beátus, cujus Deus Jacob adjútor ejus, spes ejus in Dómino, Deo ipsíus: * qui fecit cælum et terram, mare, et ómnia, quæ in eis sunt.
}\switchcolumn\portugues{
Bem-aventurado de quem é protector o Deus de Jacob e cuja esperança está no Senhor seu Deus: * que fez o céu e a terra, o mar e todas as cousas que neles há.
}\switchcolumn*\latim{
Qui custódit veritátem in sǽculum, facit judícium injúriam patiéntibus: * dat escam esuriéntibus.
}\switchcolumn\portugues{
O qual conserva eternamente a verdade, faz justiça aos que sofrem injúria: * dá sustento aos famintos.
}\switchcolumn*\latim{
Dóminus solvit compedítos: * Dóminus illúminat cæcos.
}\switchcolumn\portugues{
O Senhor dá liberdade aos cativos: * o Senhor alumia os cegos.
}\switchcolumn*\latim{
Dóminus érigit elísos, * Dóminus díligit justos.
}\switchcolumn\portugues{
O Senhor levanta os caídos, * o Senhor ama os justos.
}\switchcolumn*\latim{
Dóminus custódit ádvenas, pupíllum et víduam suscípiet: * et vias peccatórum dispérdet.
}\switchcolumn\portugues{
O Senhor protege os peregrinos, ampara o órfão e a viúva: * e destruirá os caminhos dos pecadores.
}\switchcolumn*\latim{
Regnábit Dóminus in sǽcula, Deus tuus, Sion, * in generatiónem et generatiónem.
}\switchcolumn\portugues{
O Senhor reinará pelos séculos, o teu Deus, ó Sião, * por todas as gerações.
}\end{paracol}


\subsectioninfo{Salmo 146}{Laudate Dominum, quoniam}\label{salmo146}
\begin{paracol}{2}\latim{
\rlettrine{L}{audáte} Dóminum quóniam bonus est psalmus: * Deo nostro sit jucúnda, decóraque laudátio.
}\switchcolumn\portugues{
\rlettrine{L}{ouvai} o Senhor, porque é bom salmodiar: * sê alegre para o nosso Deus, louvai-O graciosamente.
}\switchcolumn*\latim{
Ædíficans Jerúsalem Dóminus: * dispersiónes Israélis congregábit.
}\switchcolumn\portugues{
O Senhor que edifica Jerusalém: * congregará os dispersos de Israel.
}\switchcolumn*\latim{
Qui sanat contrítos corde: * et álligat contritiónes eórum.
}\switchcolumn\portugues{
É Ele que sara os de coração contrito: * e liga as suas chagas.
}\switchcolumn*\latim{
Qui númerat multitúdinem stellárum: * et ómnibus eis nómina vocat.
}\switchcolumn\portugues{
É Ele que conta a multidão das estrelas: * e as chama todas pelos seus nomes.
}\switchcolumn*\latim{
Magnus Dóminus noster, et magna virtus ejus: * et sapiéntiæ ejus non est númerus.
}\switchcolumn\portugues{
Grande é o nosso Senhor e grande o seu poder: * e a sua sabedoria não tem limites.
}\switchcolumn*\latim{
Suscípiens mansuétos Dóminus: * humílians autem peccatóres usque ad terram.
}\switchcolumn\portugues{
O Senhor é quem ampara os mansos: * e abate os pecadores até à terra.
}\switchcolumn*\latim{
Præcínite Dómino in confessióne: * psállite Deo nostro in cíthara.
}\switchcolumn\portugues{
Entoai cânticos ao Senhor em seu louvor: * cantai ao nosso Deus com a cítara.
}\switchcolumn*\latim{
Qui óperit cælum núbibus: * et parat terræ plúviam.
}\switchcolumn\portugues{
É Ele que cobre o céu de nuvens: * e prepara assim chuva para a terra.
}\switchcolumn*\latim{
Qui prodúcit in móntibus fænum: * et herbam servitúti hóminum.
}\switchcolumn\portugues{
É Ele que produz feno nos montes: * e erva para serviço dos homens.
}\switchcolumn*\latim{
Qui dat juméntis escam ipsórum: * et pullis corvórum invocántibus eum.
}\switchcolumn\portugues{
É Ele que dá aos animais o seu alimento próprio: * e aos filhinhos dos corvos que O chamam.
}\switchcolumn*\latim{
Non in fortitúdine equi voluntátem habébit: * nec in tíbiis viri beneplácitum erit ei.
}\switchcolumn\portugues{
Não se agradará da força do cavalo: * nem se agradará nos pés robustos do varão.
}\switchcolumn*\latim{
Beneplácitum est Dómino super timéntes eum: * et in eis, qui sperant super misericórdia ejus.
}\switchcolumn\portugues{
O Senhor agradou-se sempre dos que O temem: * e daqueles que esperam na sua misericórdia.
}\end{paracol}


\subsectioninfo{Salmo 148}{Laudate Dominum de cælis}\label{salmo148}
\begin{paracol}{2}\latim{
\rlettrine{L}{auda,} Jerúsalem, Dóminum: * lauda Deum tuum, Sion.
}\switchcolumn\portugues{
\rlettrine{L}{ouva,} ó Jerusalém, o Senhor: * louva, ó Sião, o teu Deus.
}\switchcolumn*\latim{
Quóniam confortávit seras portárum tuárum: * benedíxit fíliis tuis in te.
}\switchcolumn\portugues{
Porque reforçou os ferrolhos de tuas portas: * abençoou os teus filhos dentro de ti.
}\switchcolumn*\latim{
Qui pósuit fines tuos pacem: * et ádipe fruménti sátiat te.
}\switchcolumn\portugues{
Foi Ele que estabeleceu a paz nas tuas fronteiras: * e da flor da farinha te sacia.
}\switchcolumn*\latim{
Qui emíttit elóquium suum terræ: * velóciter currit sermo ejus.
}\switchcolumn\portugues{
É Ele que envia as suas ordens à terra: * e a sua palavra corre velozmente.
}\switchcolumn*\latim{
Qui dat nivem sicut lanam: * nébulam sicut cínerem spargit.
}\switchcolumn\portugues{
É Ele que faz cair a neve como lã: * espalha a névoa como cinza.
}\switchcolumn*\latim{
Mittit crystállum suam sicut buccéllas: * ante fáciem frígoris ejus quis sustinébit?
}\switchcolumn\portugues{
Envia o seu gelo aos pedaços: * ao rigor do seu frio quem poderá resistir?
}\switchcolumn*\latim{
Emíttet verbum suum, et liquefáciet ea: * flabit spíritus ejus, et fluent aquæ.
}\switchcolumn\portugues{
Enviará a sua palavra e os derreterá: * soprará o seu vento e correrão as águas.
}\switchcolumn*\latim{
Qui annúntiat verbum suum Jacob: * justítias, et judícia sua Israël.
}\switchcolumn\portugues{
Que anuncia sua palavra a Jacob: * suas justiças e seus preceitos a Israel.
}\switchcolumn*\latim{
Non fecit táliter omni natióni: * et judícia sua non manifestávit eis.
}\switchcolumn\portugues{
Não fez assim a todas as nações: * e lhes não manifestou os seus preceitos.
}\end{paracol}


\subsectioninfo{Salmo 147}{Lauda, Jerusalem}\label{salmo147}
\begin{paracol}{2}\latim{
\rlettrine{L}{audáte} Dóminum de cælis: * laudáte eum in excélsis.
}\switchcolumn\portugues{
\rlettrine{L}{ouvai} o Senhor do alto dos céus: * louvai-O nas alturas.
}\switchcolumn*\latim{
Laudáte eum, omnes Ángeli ejus: * laudáte eum, omnes virtútes ejus.
}\switchcolumn\portugues{
Louvai-O, todos seus anjos: * louvai-O, todas os seus exércitos.
}\switchcolumn*\latim{
Laudáte eum, sol et luna: * laudáte eum, omnes stellæ et lumen.
}\switchcolumn\portugues{
Louvai-O, sol e lua: * louvai-O, todas as estrelas luminosas.
}\switchcolumn*\latim{
Laudáte eum, cæli cælórum: * et aquæ omnes, quæ super cælos sunt, laudent nomen Dómini.
}\switchcolumn\portugues{
Louvai-O, céus dos céus: * e todas as águas que estão sobre os céus, louvem o nome do Senhor.
}\switchcolumn*\latim{
Quia ipse dixit, et facta sunt: * ipse mandávit, et creáta sunt.
}\switchcolumn\portugues{
Pois Ele falou e foram feitas: * mandou e foram criadas.
}\switchcolumn*\latim{
Státuit ea in ætérnum, et in sǽculum sǽculi: * præcéptum pósuit, et non præteríbit.
}\switchcolumn\portugues{
Ele estabeleceu-as para sempre e pelos séculos dos séculos: * fixou-lhes uma doutrina que não passará.
}\switchcolumn*\latim{
Laudáte Dóminum de terra, * dracónes, et omnes abýssi.
}\switchcolumn\portugues{
Louvai o Senhor criaturas da terra, * ó dragões, e todos os abismos.
}\switchcolumn*\latim{
Ignis, grando, nix, glácies, spíritus procellárum: * quæ fáciunt verbum ejus:
}\switchcolumn\portugues{
O fogo, o granizo, a neve, a geada, o espírito das tempestades: * que executam a sua palavra:
}\switchcolumn*\latim{
Montes, et omnes colles: * ligna fructífera, et omnes cedri.
}\switchcolumn\portugues{
Os montes e todos os outeiros: * as árvores frutíferas e todos os cedros.
}\switchcolumn*\latim{
Béstiæ, et univérsa pécora: * serpéntes, et vólucres pennátæ:
}\switchcolumn\portugues{
Os animais e todos os gados: * as serpentes e as aves que voam:
}\switchcolumn*\latim{
Reges terræ, et omnes pópuli: * príncipes, et omnes júdices terræ.
}\switchcolumn\portugues{
Os reis da terra e todos os povos: * os príncipes e todos os juízes da terra.
}\switchcolumn*\latim{
Júvenes, et vírgines: senes cum junióribus laudent nomen Dómini: * quia exaltátum est nomen ejus solíus.
}\switchcolumn\portugues{
Os jovens e as donzelas, os velhos e os meninos louvem o nome do Senhor: * pois só o seu nome é digno de ser exaltado.
}\switchcolumn*\latim{
Conféssio ejus super cælum et terram: * et exaltávit cornu pópuli sui.
}\switchcolumn\portugues{
Seu louvor está acima do céu e da terra: * Ele ergueu o poder do seu povo.
}\switchcolumn*\latim{
Hymnus ómnibus sanctis ejus: * fíliis Israël, pópulo appropinquánti sibi.
}\switchcolumn\portugues{
Cantem-Lhe hinos todos seus santos: * os filhos de Israel, o povo que se aproxima d’Ele.
}\end{paracol}


\subsectioninfo{Salmo 149}{Cantate Domino canticum novum}\label{salmo149}
\begin{paracol}{2}\latim{
\rlettrine{C}{antáte} Dómino cánticum novum: * laus ejus in ecclésia sanctórum.
}\switchcolumn\portugues{
\rlettrine{C}{antai} ao Senhor um cântico novo: * o seu louvor na igreja dos santos.
}\switchcolumn*\latim{
Lætétur Israël in eo, qui fecit eum: * et fílii Sion exsúltent in rege suo.
}\switchcolumn\portugues{
Alegre-se Israel n’Aquele que o criou: * e os filhos de Sião exultem-se em seu rei.
}\switchcolumn*\latim{
Laudent nomen ejus in choro: * in týmpano, et psaltério psallant ei:
}\switchcolumn\portugues{
Louvem em coro o seu nome: * cantem ao som do tambor e do saltério:
}\switchcolumn*\latim{
Quia beneplácitum est Dómino in pópulo suo: * et exaltábit mansuétos in salútem.
}\switchcolumn\portugues{
Pois o Senhor tem-se comprazido no seu povo: * e há-de exaltar os mansos e salvá-los.
}\switchcolumn*\latim{
Exsultábunt sancti in glória: * lætabúntur in cubílibus suis.
}\switchcolumn\portugues{
Exultar-se-ão os santos na glória: * eles alegrar-se-ão nas suas mansões.
}\switchcolumn*\latim{
Exaltatiónes Dei in gútture eórum: * et gládii ancípites in mánibus eórum.
}\switchcolumn\portugues{
As exaltações de Deus estarão na sua boca: * e espadas de dois gumes nas suas mãos.
}\switchcolumn*\latim{
Ad faciéndam vindíctam in natiónibus: * increpatiónes in pópulis.
}\switchcolumn\portugues{
Para exercer a vingança entre as nações: * e o castigo entre os povos.
}\switchcolumn*\latim{
Ad alligándos reges eórum in compédibus: * et nóbiles eórum in mánicis férreis.
}\switchcolumn\portugues{
Para prender os seus reis com grilhões: * e os seus Nobres com algemas de ferro.
}\switchcolumn*\latim{
Ut fáciant in eis judícium conscríptum: * glória hæc est ómnibus sanctis ejus.
}\switchcolumn\portugues{
Para executar contra eles a sentença escrita: * tal é a glória reservada a todos seus santos.
}\end{paracol}


\subsectioninfo{Salmo 150}{Laudate Dominum in sanctis ejus}\label{salmo150}
\begin{paracol}{2}\latim{
\rlettrine{L}{audáte} Dóminum in sanctis ejus: * laudáte eum in firmaménto virtútis ejus.
}\switchcolumn\portugues{
\rlettrine{L}{ouvai} o Senhor no seu santuário: * louvai-O no seu augusto firmamento.
}\switchcolumn*\latim{
Laudáte eum in virtútibus ejus: * laudáte eum secúndum multitúdinem magnitúdinis ejus.
}\switchcolumn\portugues{
Louvai-O nas suas virtudes: * louvai-O segundo a multidão da sua grandeza.
}\switchcolumn*\latim{
Laudáte eum in sono tubæ: * laudáte eum in psaltério, et cíthara.
}\switchcolumn\portugues{
Louvai-O ao som da trombeta: * louvai-O com o saltério e a cítara.
}\switchcolumn*\latim{
Laudáte eum in týmpano, et choro: * laudáte eum in chordis, et órgano.
}\switchcolumn\portugues{
Louvai-O com timbales e em coro: * louvai-O com cordas e órgão.
}\switchcolumn*\latim{
Laudáte eum in cýmbalis benesonántibus: laudáte eum in cýmbalis jubilatiónis: * omnis spíritus laudet Dóminum.
}\switchcolumn\portugues{
Louvai-O com címbalos melodiosos: louvai-O com címbalos de júbilo: * todo o espirito louve o Senhor.
}\end{paracol}


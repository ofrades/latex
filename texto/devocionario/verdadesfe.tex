\section{Verdades da Fé}

\subsubsection{Os dez mandamentos}

1 - Adorar a Deus e amá-Lo sobre todas as cousas.
2 - Não invocar o santo nome de Deus em vão.
3 - Santificar os domingos e festas de guarda.
4 - Honrar pai e mãe (e os outros legítimos superiores).
5 - Não matar (nem causar outro dano, no corpo ou na alma, a si mesmo ou ao próximo).
6 - Guardar castidade nas palavras e nas obras.
7 - Não furtar (nem injustamente reter ou danificar os bens do próximo).
8 - Não levantar falsos testemunhos (nem de qualquer outro modo faltar à verdade ou difamar o próximo).
9 - Guardar castidade nos pensamentos e nos desejos.
10 - Não cobiçar as cousas alheias.

\subsubsection{Os dous mandamentos de caridade}

\rlettrine{A}{marás} o Senhor teu Deus, com todo teu coração, com toda tua alma e com toda tua mente; Amarás ao próximo como a ti mesmo.

\subsubsection{A regra de ouro}

Tudo quanto quiserdes que os homens vos façam, fazei-lho vós também.

\subsubsection{Os cinco preceitos da Igreja}

1 - Participar na Missa, aos domingos e festas de guarda e abster-se de trabalhos e actividades que impeçam a santificação desses dias.
2 - Confessar os pecados ao menos uma vez cada ano.
3 - Comungar o sacramento da Eucaristia ao menos pela Páscoa.
4 - Guardar a abstinência e jejuar nos dias determinados pela Igreja.
5 - Contribuir para as necessidades materiais da Igreja, segundo as possibilidades.

\subsubsection{Sacramentos}

Baptismo; Confirmação; Eucaristia; Penitência ou Confissão; Extrema Unção; Ordem; Matrimónio.

\subsubsection{As Bem-Aventuranças}

\blettrine{B}{em-aventurados} os pobres em espírito, porque deles é o reino dos céus. Bem-aventurados os que choram, porque serão consolados. Bem-aventurados os mansos, porque possuirão a terra. Bem-aventurados os que têm fome e sede de justiça, porque serão saciados. Bem-aventurados os misericordiosos, porque alcançarão misericórdia. Bem-aventurados os puros de coração, porque verão a Deus. Bem-aventurados os pacificadores, porque serão chamados filhos de Deus. Bem-aventurados os que sofrem perseguição por causa da justiça, porque deles é o reino dos céus. Bem-aventurados sereis quando vos insultarem, vos perseguirem e, mentindo, disserem toda a espécie de calúnias contra vós. Alegrai-vos e exultai, porque será grande a vossa recompensa nos céus.

\subsubsection{Dias de Obrigação}

\emph{Para além de todos os Domingos}
1 de Janeiro - Solenidade de Santa Maria, Mãe de Deus;
6 de Janeiro - Epifania;
19 de Março - Solenidade de São José;
Ascensão de Jesus - Quinta-feira da sexta semana da Páscoa;
Corpus Christi - Primeira quinta-feira após o Domingo da Santíssima Trindade;
29 de Junho - Solenidade dos Apóstolos São Pedro e São Paulo; 15 de Agosto - Assunção de Maria;
1 de Novembro - Dia de Todos-os-Santos;
8 de Dezembro - Imaculada Conceição de Maria;
25 de Dezembro - Natal.

\subsubsection{Trabalhos de Misericórdia}

\begin{paracol}{2}
\begin{nscenter}{\redx Corporais}\end{nscenter}
\switchcolumn
\begin{nscenter}{\redx Espirituais}\end{nscenter}
\switchcolumn*
Dar de comer a quem tem fome; Dar de beber a quem tem sede; Vestir os nus; Dar pousada aos peregrinos; Visitar os enfermos; Visitar os presos; Enterrar os mortos.
\switchcolumn
Dar bons conselhos; Ensinar os ignorantes; Corrigir os que erram; Consolar os tristes; Perdoar as injúrias; Suportar com paciência as fraquezas do nosso próximo; Rezar a Deus por vivos e defuntos.
\end{paracol}

\subsubsection{Virtudes}

\begin{paracol}{2}
\begin{nscenter}{\redx Cardeais}\end{nscenter}
\switchcolumn
\begin{nscenter}{\redx Teologais}\end{nscenter}
\switchcolumn*
Prudência; Justiça; Fortaleza; Temperança.
\switchcolumn
Fé; Esperança; Caridade.
\end{paracol}

\subsubsection{Pecados Contra o Espírito Santo}

\emph{Pecados de pura malícia, que são contrários à bondade que se atribui ao Espírito Santo.}

Desesperar da salvação; Presunção de se salvar sem merecimentos; Combater a verdade conhecida; Ter inveja das graças que Deus dá a outrem; Obstinar-se no pecado; Morrer na impenitência final.

\subsubsection{Pecados que Bradam aos Céus}

\emph{Sua malícia é tão grave e manifesta, que provoca Deus a puni-los com os mais severos castigos.}

Homicídio voluntário; Pecado impuro contra a natureza; Opressão dos pobres, principalmente órfãos e viúvas; Não pagar o salário a quem trabalha.

\subsubsection{Do Espírito Santo}

\begin{paracol}{2}
\begin{nscenter}{\redx Dons}\end{nscenter}
\switchcolumn
\begin{nscenter}{\redx Frutos}\end{nscenter}
\switchcolumn*
Sabedoria; Entendimento; Conselho; Fortaleza; Ciência; Piedade; Temor de Deus.
\switchcolumn
Amor; Alegria; Paz; Paciência; Longanimidade; Bondade; Benignidade; Mansidão; Fé; Modéstia; Continência; Castidade.
\end{paracol}

\begin{paracol}{2}
\subsubsection{Pecados Capitais}
\switchcolumn
\subsubsection{Virtudes Opostas}
\switchcolumn*
Soberba; Avareza; Luxúria; Ira; Gula; Inveja.
\switchcolumn
Humildade; Caridade; Castidade; Paciência; Temperança; Bondade.
\end{paracol}

\subsubsection{Novíssimos}

\begin{paracol}{2}
Mors; Iudicium; Infernus; Paradisus.
\switchcolumn
Morte; Juízo; Inferno; Paraíso.
\end{paracol}

\subsubsection{Assuntos para Meditação Diária}

\begin{paracol}{2}
Deum glorificare; Jesum imitari; Beatissimam Virginem et Sanctos venerari; Angelos invocare; Animam salvare; Corpus mortificare; Virtutes a Deo exorare; Peccata expiare; Paradisum comparare; Infernum evitare; Aeternitatem considerare; Tempus bene applicare; Proximum ædificare; Mundum formidare; Dæmones impugnare; Passiones frenare; Mortem semper exspectare; Ad iudicium te præparare.
\switchcolumn
Deus para glorificar; Jesus para imitar; A abençoada Virgem e os Santos para venerar; Os Anjos para invocar; A alma para salvar; O corpo para mortificar; Virtudes para conquistar; Pecados para expiar; O paraíso para ganhar; O inferno para evitar; Eternidade para preparar; Tempo para bem aproveitar; O próximo para edificar; O mundo para desprezar; Demónios para combater; Paixões para refrear; A morte sempre esperar; E o julgamento para se preparar.
\end{paracol}

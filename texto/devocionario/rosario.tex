\section{Rosário}

\subsection{Sinal da Cruz}
\begin{paracol}{2}\latim{
\cruz In nómine Patris, et Fílii, et Spíritus Sancti. Amen.
}\switchcolumn\portugues{
\cruz Em nome do Pai e do Filho e do Espírito Santo. Amen.
}\end{paracol}


\subsectioninfo{Símbolo dos Apóstolos}{Página \pageref{simboloapostolos}}

\subsubsection{Oferecimento do Santo Rosário}

\rlettrine{S}{antíssima} Virgem, Mãe de Deus, eu Vos ofereço este rosário em desagravo do Santíssimo Coração de Nosso Senhor Jesus Cristo, vosso Filho, e em desagravo do vosso Coração Imaculado; e pelas intenções que Vos apresento: \emph{(Referir as intenções.)}

\subsubsection{Intenções do Santo Padre}
\begin{compactitem}
\begin{paracol}{2}\latim{
\item Exaltatio S. Matris Ecclesiæ.
}\switchcolumn\portugues{
\item Exaltação da Santa Igreja.
}\switchcolumn*\latim{
\item Propagatio fidei.
}\switchcolumn\portugues{
\item Propagação da fé.
}\switchcolumn*\latim{
\item Extirpatio hæresum.
}\switchcolumn\portugues{
\item Extirpação das heresias.
}\switchcolumn*\latim{
\item Conversio peccatorum.
}\switchcolumn\portugues{
\item Conversão dos pecadores.
}\switchcolumn*\latim{
\item Pax inter principes christianos.
}\switchcolumn\portugues{
\item Paz entre os Reis e Príncipes católicos.
}\end{paracol}
\end{compactitem}

\subsection{Pedras Maiores}
\subsubsection{Pai Nosso}
\begin{paracol}{2}\latim{
℣. Pater noster, qui es in cælis: sanctificétur nomen tuum: advéniat regnum tuum: fiat volúntas tua, sicut in cælo, et in terra.
}\switchcolumn\portugues{
℣. Pai Nosso, que estais nos céus, santificado seja o vosso Nome, venha a nós o vosso Reino; seja feita a vossa vontade assim na terra como no Céu.
}\switchcolumn*\latim{
℟. Panem nostrum quotidiánum da nobis hódie: et dimítte nobis débita nostra, sicut et nos dimíttimus debitóribus nostris. Et ne nos indúcas in tentatiónem. Sed líbera nos a malo. Amen.
}\switchcolumn\portugues{
℟. O pão nosso de cada dia nos dai hoje; perdoai-nos as nossas ofensas, assim como nós perdoamos a quem nos tem ofendido; e não nos deixeis cair em tentação; mas livrai-nos do mal. Amen.
}\end{paracol}


\subsection{Pedras Menores}
\subsubsection{Ave Maria}
\begin{paracol}{2}\latim{
℣. Ave, María, grátia plena, Dóminus tecum; benedícta tu in muliéribus, et benedíctus fructus ventris tui, Jesus.
}\switchcolumn\portugues{
℣. Ave, Maria, Cheia de graça, o Senhor é convosco; bendita sois Vós entre as mulheres, e bendito é o fruto do vosso ventre, Jesus.
}\switchcolumn*\latim{
℟. Sancta María, Mater Dei, ora pro nobis peccatóribus, nunc, et in hora mortis nostræ. Amen.
}\switchcolumn\portugues{
℟. Santa Maria, Mãe de Deus, rogai por nós, pecadores, agora e na hora da nossa morte. Amen.
}\end{paracol}


\subsection{Fim da Dezena}
\subsubsection{Glória}
\begin{paracol}{2}\latim{
℣. Glória Patri, et Fílio, et Spíritui Sancto.
}\switchcolumn\portugues{
℣. Glória ao Pai, e ao Filho e ao Espírito Santo.
}\switchcolumn*\latim{
℟. Sicut erat in pricípio, et nunc, et semper, et in sǽcula sæculórum. Amen.
}\switchcolumn\portugues{
℟. Assim como era no princípio, agora e sempre, e por todos os séculos dos séculos. Amen.
}\end{paracol}


\subsubsection{Nossa Senhora a Santa Catarina Labouré}\label{concebidasempecado}
\begin{paracol}{2}\latim{
℣. O Maria sine labe concepta.
}\switchcolumn\portugues{
℣. Ó Maria concebida sem pecado.
}\switchcolumn*\latim{
℟. Ora pro nobis, qui confugimus ad te.
}\switchcolumn\portugues{
℟. Rogai por nós que recorremos a vós.
}\end{paracol}

\subsubsection{Nossa Senhora aos Santos Pastorinhos}
\begin{paracol}{2}\latim{
℣. Oh mi Jesu, dimitte nobis débita nostra, líbera nos ab igne inférni,
}\switchcolumn\portugues{
℣. Ó meu Jesus, perdoai-nos e livrai-nos do fogo do inferno,
}\switchcolumn*\latim{
℟. Conduc in cælum omnes animas, præsértim illas quæ máxime indigent misericórdia tua.
}\switchcolumn\portugues{
℟. Levai as alminhas todas para o Céu e socorrei principalmente as que mais precisarem.
}\end{paracol}


\subsection{Meditações do Rosário}

\subsubsection{Mystérios Gozosos}
\begin{nscenter}\emph{Segunda-feira e Quinta-feira}\end{nscenter}

{\redx Primeiro Mystério:} Meditemos na Anunciação do Arcanjo São Gabriel à Santíssima Virgem, e roguemos a virtude da humildade.\par
{\redx Segundo Mystério:} Meditemos na Visitação da Santíssima Virgem a sua Prima, Santa Isabel, e roguemos a caridade para com o próximo.\par
{\redx Terceiro Mystério:} Meditemos no Nascimento do Menino Jesus, e roguemos o desprendimento dos bens do mundo.\par
{\redx Quarto Mystério:} Meditemos na Apresentação do Menino Jesus no Templo e na Purificação de Nossa Senhora, e roguemos a obediência e a pureza do espírito e do coração.\par
{\redx Quinto Mystério:} Meditemos na Perda e no Encontro do Menino Jesus no Templo, e roguemos o conhecimento das cousas divinas e a prontidão no serviço de Deus.

\subsubsection{Mystérios Dolorosos}
\begin{nscenter}\emph{Terça-feira e Sexta-feira}\end{nscenter}

{\redx Primeiro Mystério:} Meditemos na Agonia de N. S. Jesus Cristo, e roguemos a contrição dos nossos pecados.\par
{\redx Segundo Mystério:} Meditemos na flagelação de N. S. Jesus Cristo, e roguemos a mortificação dos sentidos.\par
{\redx Terceiro Mystério:} Meditemos na Coroação de Espinhos de N. S. Jesus Cristo, e roguemos a mortificação do espírito e do coração.\par
{\redx Quarto Mystério:} Meditemos em N. S. Jesus Cristo levando a Cruz para o Calvário, e roguemos a paciência e a resignação.\par
{\redx Quinto Mystério:} Meditemos na Crucifixão e Morte de N. S. Jesus Cristo, e roguemos o amor a Deus e a salvação das almas.

\subsubsection{Mystérios Gloriosos}
\begin{nscenter}\emph{Quarta-feira, Sábado e Domingo}\end{nscenter}

{\redx Primeiro Mystério:} Meditemos na Ressurreição de N. S. Jesus Cristo, e roguemos para recebermos o dom da fé e para a conversão dos pecadores.\par
{\redx Segundo Mystério:} Meditemos na Ascensão de N. S. Jesus Cristo, e roguemos a esperança e o desejo do céu.\par
{\redx Terceiro Mystério:} Meditemos na descida do Divino Espírito Santo, e roguemos o amor a Deus e o zelo da salvação das almas.\par
{\redx Quarto Mystério:} Meditemos na Assunção da Santíssima Virgem, e roguemos a graça de uma boa morte e a devoção a Nossa Senhora.\par
{\redx Quinto Mystério:} Meditemos na Coroação da Santíssima Virgem, e roguemos a perseverança final e a confiança em Nossa Senhora.

\subsection{Fim do Rosário}

\emph{Procurar Antifona de Nossa Senhora própria do Tempo na página \pageref{antifonasnossasenhora} ou rezar a Salve Regina.}

\subsubsectioninfo{Salve Regina}{Página \pageref{salveregina}}

\subsubsectioninfo{Ladainha de Nossa Senhora}{Página \pageref{ladainhaloreto}}
